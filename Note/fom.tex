Wasserstein distance $D_w$ is a metric between two distributions, discrete or continuous. To qualify the timing characteristics of a waveform analysis result, we take the distance between normalized $\hat{\phi}(t)$ and sampled light curve $\tilde{\phi}(t)$ defined in~\eqref{eq:lc-sample}.
\begin{equation}
  D_w\left[\hat{\phi}_*, \tilde{\phi}_*\right] = \inf_{\gamma \in \Gamma} \left[\int \left\vert t_1 - t_2 \right\vert^p \gamma(t_1, t_2)\mathrm{d}t_1\mathrm{d}t_2\right]^{\frac{1}{p}}
\end{equation}
where $\hat{\phi}_*$ denotes the normalized $\hat{\phi}$. $\Gamma$ is the collection of joint distributions with marginals $\hat{\phi}_*(t)$ and $\tilde{\phi}_*(t)$,
\begin{equation*}
  \label{eq:joint}
  \Gamma = \left\{\gamma(t_1, t_2) ~\middle\vert~ \int\gamma(t_1,t_2)\mathrm{d}t_1 = \tilde{\phi}_*(t_2) , \int\gamma(t_1,t_2)\mathrm{d}t_2 = \hat{\phi}_*(t_1)  \right\}.
\end{equation*}
When $p=1$, it is also known as the \textit{earth mover's distance}~\cite{levina_earth_2001} in that it encodes the minimum cost to transport mass from one distribution to the other.
For Fig.~\ref{fig:l2}, $D_w(a, b_1) = 0.5$ while $D_w(a, b_2) = 1.5$, which qualifies that $a$ is closer to $b_1$.  Therefore, Wasserstein distance complements RSS and offers a timing-sensitive metric suitable for the sparse PE space.

Although calculating Wasserstein distance is generally costly, when $p=1$ it is straightforward to implement by cumulative distribution functions (CDF).  Denote the CDF of $\hat{\phi}_*(t)$ as $\hat\Phi(t)$.  $D_w$ is equivalent to the $\ell_1$-distance between the CDFs,
\begin{equation}
    D_w\left[\hat{\phi}_*, \tilde{\phi}_*\right] = \int\left|\hat{\Phi}(t) - \tilde{\Phi}(t)\right| \mathrm{d}t.
    \label{eq:numerical}
\end{equation}
