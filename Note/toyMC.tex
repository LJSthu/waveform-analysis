\section{Towards Timing Resolution} % (fold)
\label{sec:toyMC}

\subsection{PMT waveform}

The incident particle, no matter we welcome or not, reacts with the material and deposit energy in the detector. During the de-excitation process, an indeterminate number of photons are emitted from the vertex of the event and the trajectory of the incident particle. A certain proportion of these photons are collected by the PMTs. The time distribution of collected PEs is time profile. 

Typical PMT response includes several individual processes. First, single photon hits on the surface of PMT, and transmission happens in the glass window and photoelectron (PE) conversion happens on the photocathode. Then PE is collected by the first dynode. Finally, the amplification of electrons between dynodes provides a measurable electrical signal. So a single incoming photon has a certain probability to be observed via PMT voltage (see figure~\ref{fig:spe}), which is a Bernoulli process. But if the hittime of photons is very short, shorter the data acquisition (DAQ) window, the PE's voltage response will pile-up (see figure~\ref{fig:pile}) and discrimination of these PE will be difficult, as well as the waveform analysis. Pile-up will significantly worsen timing resolution. 

\begin{figure}[H]
\begin{minipage}[b]{.5\textwidth}
\begin{figure}[H]
    \centering
    \resizebox{\textwidth}{!}{version https://git-lfs.github.com/spec/v1
oid sha256:5a853c23cbe60b468303354775e78f93380955b674ea7498b6f6c2431aeff0cd
size 41585
}
    \caption{\label{fig:spe} SPE response of PMT}
\end{figure}
\end{minipage}
\begin{minipage}[b]{.5\textwidth}
\begin{figure}[H]
    \centering
    \resizebox{\textwidth}{!}{%% Creator: Matplotlib, PGF backend
%%
%% To include the figure in your LaTeX document, write
%%   \input{<filename>.pgf}
%%
%% Make sure the required packages are loaded in your preamble
%%   \usepackage{pgf}
%%
%% and, on pdftex
%%   \usepackage[utf8]{inputenc}\DeclareUnicodeCharacter{2212}{-}
%%
%% or, on luatex and xetex
%%   \usepackage{unicode-math}
%%
%% Figures using additional raster images can only be included by \input if
%% they are in the same directory as the main LaTeX file. For loading figures
%% from other directories you can use the `import` package
%%   \usepackage{import}
%%
%% and then include the figures with
%%   \import{<path to file>}{<filename>.pgf}
%%
%% Matplotlib used the following preamble
%%   \usepackage[detect-all,locale=DE]{siunitx}
%%
\begingroup%
\makeatletter%
\begin{pgfpicture}%
\pgfpathrectangle{\pgfpointorigin}{\pgfqpoint{8.000000in}{6.000000in}}%
\pgfusepath{use as bounding box, clip}%
\begin{pgfscope}%
\pgfsetbuttcap%
\pgfsetmiterjoin%
\definecolor{currentfill}{rgb}{1.000000,1.000000,1.000000}%
\pgfsetfillcolor{currentfill}%
\pgfsetlinewidth{0.000000pt}%
\definecolor{currentstroke}{rgb}{1.000000,1.000000,1.000000}%
\pgfsetstrokecolor{currentstroke}%
\pgfsetdash{}{0pt}%
\pgfpathmoveto{\pgfqpoint{0.000000in}{0.000000in}}%
\pgfpathlineto{\pgfqpoint{8.000000in}{0.000000in}}%
\pgfpathlineto{\pgfqpoint{8.000000in}{6.000000in}}%
\pgfpathlineto{\pgfqpoint{0.000000in}{6.000000in}}%
\pgfpathclose%
\pgfusepath{fill}%
\end{pgfscope}%
\begin{pgfscope}%
\pgfsetbuttcap%
\pgfsetmiterjoin%
\definecolor{currentfill}{rgb}{1.000000,1.000000,1.000000}%
\pgfsetfillcolor{currentfill}%
\pgfsetlinewidth{0.000000pt}%
\definecolor{currentstroke}{rgb}{0.000000,0.000000,0.000000}%
\pgfsetstrokecolor{currentstroke}%
\pgfsetstrokeopacity{0.000000}%
\pgfsetdash{}{0pt}%
\pgfpathmoveto{\pgfqpoint{1.200000in}{0.900000in}}%
\pgfpathlineto{\pgfqpoint{6.800000in}{0.900000in}}%
\pgfpathlineto{\pgfqpoint{6.800000in}{5.700000in}}%
\pgfpathlineto{\pgfqpoint{1.200000in}{5.700000in}}%
\pgfpathclose%
\pgfusepath{fill}%
\end{pgfscope}%
\begin{pgfscope}%
\pgfpathrectangle{\pgfqpoint{1.200000in}{0.900000in}}{\pgfqpoint{5.600000in}{4.800000in}}%
\pgfusepath{clip}%
\pgfsetrectcap%
\pgfsetroundjoin%
\pgfsetlinewidth{2.007500pt}%
\definecolor{currentstroke}{rgb}{0.000000,0.000000,1.000000}%
\pgfsetstrokecolor{currentstroke}%
\pgfsetdash{}{0pt}%
\pgfpathmoveto{\pgfqpoint{1.200000in}{1.317391in}}%
\pgfpathlineto{\pgfqpoint{1.210884in}{1.317391in}}%
\pgfpathlineto{\pgfqpoint{1.216327in}{1.233913in}}%
\pgfpathlineto{\pgfqpoint{1.221769in}{1.400870in}}%
\pgfpathlineto{\pgfqpoint{1.232653in}{1.400870in}}%
\pgfpathlineto{\pgfqpoint{1.238095in}{1.317391in}}%
\pgfpathlineto{\pgfqpoint{1.243537in}{1.317391in}}%
\pgfpathlineto{\pgfqpoint{1.248980in}{1.150435in}}%
\pgfpathlineto{\pgfqpoint{1.254422in}{1.150435in}}%
\pgfpathlineto{\pgfqpoint{1.265306in}{1.317391in}}%
\pgfpathlineto{\pgfqpoint{1.287075in}{1.317391in}}%
\pgfpathlineto{\pgfqpoint{1.297959in}{1.484348in}}%
\pgfpathlineto{\pgfqpoint{1.303401in}{1.233913in}}%
\pgfpathlineto{\pgfqpoint{1.308844in}{1.400870in}}%
\pgfpathlineto{\pgfqpoint{1.314286in}{1.233913in}}%
\pgfpathlineto{\pgfqpoint{1.319728in}{1.317391in}}%
\pgfpathlineto{\pgfqpoint{1.325170in}{1.317391in}}%
\pgfpathlineto{\pgfqpoint{1.330612in}{1.233913in}}%
\pgfpathlineto{\pgfqpoint{1.336054in}{1.233913in}}%
\pgfpathlineto{\pgfqpoint{1.341497in}{1.317391in}}%
\pgfpathlineto{\pgfqpoint{1.352381in}{1.317391in}}%
\pgfpathlineto{\pgfqpoint{1.357823in}{1.150435in}}%
\pgfpathlineto{\pgfqpoint{1.363265in}{1.150435in}}%
\pgfpathlineto{\pgfqpoint{1.368707in}{1.400870in}}%
\pgfpathlineto{\pgfqpoint{1.379592in}{1.400870in}}%
\pgfpathlineto{\pgfqpoint{1.385034in}{1.233913in}}%
\pgfpathlineto{\pgfqpoint{1.390476in}{1.317391in}}%
\pgfpathlineto{\pgfqpoint{1.395918in}{1.317391in}}%
\pgfpathlineto{\pgfqpoint{1.401361in}{1.400870in}}%
\pgfpathlineto{\pgfqpoint{1.406803in}{1.317391in}}%
\pgfpathlineto{\pgfqpoint{1.412245in}{1.150435in}}%
\pgfpathlineto{\pgfqpoint{1.417687in}{1.400870in}}%
\pgfpathlineto{\pgfqpoint{1.423129in}{1.233913in}}%
\pgfpathlineto{\pgfqpoint{1.428571in}{1.400870in}}%
\pgfpathlineto{\pgfqpoint{1.450340in}{1.400870in}}%
\pgfpathlineto{\pgfqpoint{1.455782in}{1.317391in}}%
\pgfpathlineto{\pgfqpoint{1.461224in}{1.150435in}}%
\pgfpathlineto{\pgfqpoint{1.466667in}{1.484348in}}%
\pgfpathlineto{\pgfqpoint{1.472109in}{1.317391in}}%
\pgfpathlineto{\pgfqpoint{1.499320in}{1.317391in}}%
\pgfpathlineto{\pgfqpoint{1.504762in}{1.400870in}}%
\pgfpathlineto{\pgfqpoint{1.515646in}{1.233913in}}%
\pgfpathlineto{\pgfqpoint{1.521088in}{1.233913in}}%
\pgfpathlineto{\pgfqpoint{1.526531in}{1.400870in}}%
\pgfpathlineto{\pgfqpoint{1.531973in}{1.400870in}}%
\pgfpathlineto{\pgfqpoint{1.537415in}{1.233913in}}%
\pgfpathlineto{\pgfqpoint{1.542857in}{1.233913in}}%
\pgfpathlineto{\pgfqpoint{1.553741in}{1.400870in}}%
\pgfpathlineto{\pgfqpoint{1.564626in}{1.233913in}}%
\pgfpathlineto{\pgfqpoint{1.570068in}{1.233913in}}%
\pgfpathlineto{\pgfqpoint{1.575510in}{1.317391in}}%
\pgfpathlineto{\pgfqpoint{1.580952in}{1.317391in}}%
\pgfpathlineto{\pgfqpoint{1.586395in}{1.233913in}}%
\pgfpathlineto{\pgfqpoint{1.591837in}{1.233913in}}%
\pgfpathlineto{\pgfqpoint{1.597279in}{1.317391in}}%
\pgfpathlineto{\pgfqpoint{1.602721in}{1.317391in}}%
\pgfpathlineto{\pgfqpoint{1.608163in}{1.233913in}}%
\pgfpathlineto{\pgfqpoint{1.613605in}{1.317391in}}%
\pgfpathlineto{\pgfqpoint{1.619048in}{1.317391in}}%
\pgfpathlineto{\pgfqpoint{1.624490in}{1.400870in}}%
\pgfpathlineto{\pgfqpoint{1.629932in}{1.317391in}}%
\pgfpathlineto{\pgfqpoint{1.635374in}{1.317391in}}%
\pgfpathlineto{\pgfqpoint{1.640816in}{1.400870in}}%
\pgfpathlineto{\pgfqpoint{1.646259in}{1.233913in}}%
\pgfpathlineto{\pgfqpoint{1.657143in}{1.233913in}}%
\pgfpathlineto{\pgfqpoint{1.662585in}{1.317391in}}%
\pgfpathlineto{\pgfqpoint{1.668027in}{1.233913in}}%
\pgfpathlineto{\pgfqpoint{1.673469in}{1.233913in}}%
\pgfpathlineto{\pgfqpoint{1.678912in}{1.317391in}}%
\pgfpathlineto{\pgfqpoint{1.684354in}{1.317391in}}%
\pgfpathlineto{\pgfqpoint{1.689796in}{1.400870in}}%
\pgfpathlineto{\pgfqpoint{1.695238in}{1.317391in}}%
\pgfpathlineto{\pgfqpoint{1.700680in}{1.400870in}}%
\pgfpathlineto{\pgfqpoint{1.711565in}{1.400870in}}%
\pgfpathlineto{\pgfqpoint{1.717007in}{1.317391in}}%
\pgfpathlineto{\pgfqpoint{1.722449in}{1.400870in}}%
\pgfpathlineto{\pgfqpoint{1.727891in}{1.233913in}}%
\pgfpathlineto{\pgfqpoint{1.733333in}{1.233913in}}%
\pgfpathlineto{\pgfqpoint{1.738776in}{1.400870in}}%
\pgfpathlineto{\pgfqpoint{1.744218in}{1.317391in}}%
\pgfpathlineto{\pgfqpoint{1.749660in}{1.400870in}}%
\pgfpathlineto{\pgfqpoint{1.755102in}{1.317391in}}%
\pgfpathlineto{\pgfqpoint{1.760544in}{1.317391in}}%
\pgfpathlineto{\pgfqpoint{1.765986in}{1.400870in}}%
\pgfpathlineto{\pgfqpoint{1.771429in}{1.400870in}}%
\pgfpathlineto{\pgfqpoint{1.776871in}{1.233913in}}%
\pgfpathlineto{\pgfqpoint{1.782313in}{1.400870in}}%
\pgfpathlineto{\pgfqpoint{1.787755in}{1.400870in}}%
\pgfpathlineto{\pgfqpoint{1.793197in}{1.317391in}}%
\pgfpathlineto{\pgfqpoint{1.798639in}{1.484348in}}%
\pgfpathlineto{\pgfqpoint{1.804082in}{1.484348in}}%
\pgfpathlineto{\pgfqpoint{1.809524in}{1.317391in}}%
\pgfpathlineto{\pgfqpoint{1.814966in}{1.233913in}}%
\pgfpathlineto{\pgfqpoint{1.820408in}{1.317391in}}%
\pgfpathlineto{\pgfqpoint{1.825850in}{1.484348in}}%
\pgfpathlineto{\pgfqpoint{1.831293in}{1.317391in}}%
\pgfpathlineto{\pgfqpoint{1.836735in}{1.400870in}}%
\pgfpathlineto{\pgfqpoint{1.847619in}{1.233913in}}%
\pgfpathlineto{\pgfqpoint{1.858503in}{1.400870in}}%
\pgfpathlineto{\pgfqpoint{1.863946in}{1.150435in}}%
\pgfpathlineto{\pgfqpoint{1.869388in}{1.317391in}}%
\pgfpathlineto{\pgfqpoint{1.874830in}{1.233913in}}%
\pgfpathlineto{\pgfqpoint{1.891156in}{1.233913in}}%
\pgfpathlineto{\pgfqpoint{1.896599in}{1.317391in}}%
\pgfpathlineto{\pgfqpoint{1.902041in}{1.233913in}}%
\pgfpathlineto{\pgfqpoint{1.907483in}{1.484348in}}%
\pgfpathlineto{\pgfqpoint{1.912925in}{1.317391in}}%
\pgfpathlineto{\pgfqpoint{1.918367in}{1.317391in}}%
\pgfpathlineto{\pgfqpoint{1.923810in}{1.484348in}}%
\pgfpathlineto{\pgfqpoint{1.929252in}{1.317391in}}%
\pgfpathlineto{\pgfqpoint{1.934694in}{1.317391in}}%
\pgfpathlineto{\pgfqpoint{1.945578in}{1.150435in}}%
\pgfpathlineto{\pgfqpoint{1.951020in}{1.400870in}}%
\pgfpathlineto{\pgfqpoint{1.961905in}{1.233913in}}%
\pgfpathlineto{\pgfqpoint{1.967347in}{1.400870in}}%
\pgfpathlineto{\pgfqpoint{1.978231in}{1.233913in}}%
\pgfpathlineto{\pgfqpoint{1.989116in}{1.233913in}}%
\pgfpathlineto{\pgfqpoint{1.994558in}{1.150435in}}%
\pgfpathlineto{\pgfqpoint{2.000000in}{1.233913in}}%
\pgfpathlineto{\pgfqpoint{2.005442in}{1.150435in}}%
\pgfpathlineto{\pgfqpoint{2.010884in}{1.317391in}}%
\pgfpathlineto{\pgfqpoint{2.027211in}{1.317391in}}%
\pgfpathlineto{\pgfqpoint{2.032653in}{1.233913in}}%
\pgfpathlineto{\pgfqpoint{2.038095in}{1.233913in}}%
\pgfpathlineto{\pgfqpoint{2.043537in}{1.400870in}}%
\pgfpathlineto{\pgfqpoint{2.048980in}{1.317391in}}%
\pgfpathlineto{\pgfqpoint{2.054422in}{1.400870in}}%
\pgfpathlineto{\pgfqpoint{2.059864in}{1.233913in}}%
\pgfpathlineto{\pgfqpoint{2.065306in}{1.233913in}}%
\pgfpathlineto{\pgfqpoint{2.070748in}{1.317391in}}%
\pgfpathlineto{\pgfqpoint{2.076190in}{1.317391in}}%
\pgfpathlineto{\pgfqpoint{2.081633in}{1.400870in}}%
\pgfpathlineto{\pgfqpoint{2.087075in}{1.233913in}}%
\pgfpathlineto{\pgfqpoint{2.092517in}{1.484348in}}%
\pgfpathlineto{\pgfqpoint{2.103401in}{1.317391in}}%
\pgfpathlineto{\pgfqpoint{2.108844in}{1.400870in}}%
\pgfpathlineto{\pgfqpoint{2.119728in}{1.233913in}}%
\pgfpathlineto{\pgfqpoint{2.125170in}{1.400870in}}%
\pgfpathlineto{\pgfqpoint{2.130612in}{1.400870in}}%
\pgfpathlineto{\pgfqpoint{2.141497in}{1.233913in}}%
\pgfpathlineto{\pgfqpoint{2.146939in}{1.317391in}}%
\pgfpathlineto{\pgfqpoint{2.152381in}{1.233913in}}%
\pgfpathlineto{\pgfqpoint{2.157823in}{1.317391in}}%
\pgfpathlineto{\pgfqpoint{2.163265in}{1.317391in}}%
\pgfpathlineto{\pgfqpoint{2.168707in}{1.233913in}}%
\pgfpathlineto{\pgfqpoint{2.174150in}{1.317391in}}%
\pgfpathlineto{\pgfqpoint{2.179592in}{1.233913in}}%
\pgfpathlineto{\pgfqpoint{2.185034in}{1.400870in}}%
\pgfpathlineto{\pgfqpoint{2.190476in}{1.400870in}}%
\pgfpathlineto{\pgfqpoint{2.195918in}{1.317391in}}%
\pgfpathlineto{\pgfqpoint{2.206803in}{1.317391in}}%
\pgfpathlineto{\pgfqpoint{2.217687in}{1.484348in}}%
\pgfpathlineto{\pgfqpoint{2.223129in}{1.317391in}}%
\pgfpathlineto{\pgfqpoint{2.228571in}{1.233913in}}%
\pgfpathlineto{\pgfqpoint{2.234014in}{1.400870in}}%
\pgfpathlineto{\pgfqpoint{2.239456in}{1.233913in}}%
\pgfpathlineto{\pgfqpoint{2.244898in}{1.317391in}}%
\pgfpathlineto{\pgfqpoint{2.250340in}{1.317391in}}%
\pgfpathlineto{\pgfqpoint{2.255782in}{1.233913in}}%
\pgfpathlineto{\pgfqpoint{2.261224in}{1.317391in}}%
\pgfpathlineto{\pgfqpoint{2.266667in}{1.233913in}}%
\pgfpathlineto{\pgfqpoint{2.272109in}{1.233913in}}%
\pgfpathlineto{\pgfqpoint{2.282993in}{1.400870in}}%
\pgfpathlineto{\pgfqpoint{2.288435in}{1.317391in}}%
\pgfpathlineto{\pgfqpoint{2.293878in}{1.317391in}}%
\pgfpathlineto{\pgfqpoint{2.299320in}{1.400870in}}%
\pgfpathlineto{\pgfqpoint{2.304762in}{1.233913in}}%
\pgfpathlineto{\pgfqpoint{2.310204in}{1.233913in}}%
\pgfpathlineto{\pgfqpoint{2.315646in}{1.317391in}}%
\pgfpathlineto{\pgfqpoint{2.321088in}{1.317391in}}%
\pgfpathlineto{\pgfqpoint{2.326531in}{1.400870in}}%
\pgfpathlineto{\pgfqpoint{2.337415in}{1.233913in}}%
\pgfpathlineto{\pgfqpoint{2.342857in}{1.400870in}}%
\pgfpathlineto{\pgfqpoint{2.348299in}{1.150435in}}%
\pgfpathlineto{\pgfqpoint{2.353741in}{1.317391in}}%
\pgfpathlineto{\pgfqpoint{2.359184in}{1.400870in}}%
\pgfpathlineto{\pgfqpoint{2.364626in}{1.317391in}}%
\pgfpathlineto{\pgfqpoint{2.370068in}{1.484348in}}%
\pgfpathlineto{\pgfqpoint{2.375510in}{1.317391in}}%
\pgfpathlineto{\pgfqpoint{2.380952in}{1.233913in}}%
\pgfpathlineto{\pgfqpoint{2.386395in}{1.317391in}}%
\pgfpathlineto{\pgfqpoint{2.391837in}{1.317391in}}%
\pgfpathlineto{\pgfqpoint{2.397279in}{1.400870in}}%
\pgfpathlineto{\pgfqpoint{2.402721in}{1.317391in}}%
\pgfpathlineto{\pgfqpoint{2.408163in}{1.317391in}}%
\pgfpathlineto{\pgfqpoint{2.413605in}{1.400870in}}%
\pgfpathlineto{\pgfqpoint{2.419048in}{1.400870in}}%
\pgfpathlineto{\pgfqpoint{2.429932in}{1.233913in}}%
\pgfpathlineto{\pgfqpoint{2.435374in}{1.317391in}}%
\pgfpathlineto{\pgfqpoint{2.440816in}{1.317391in}}%
\pgfpathlineto{\pgfqpoint{2.446259in}{1.400870in}}%
\pgfpathlineto{\pgfqpoint{2.451701in}{1.317391in}}%
\pgfpathlineto{\pgfqpoint{2.457143in}{1.066957in}}%
\pgfpathlineto{\pgfqpoint{2.462585in}{1.317391in}}%
\pgfpathlineto{\pgfqpoint{2.468027in}{1.400870in}}%
\pgfpathlineto{\pgfqpoint{2.473469in}{1.317391in}}%
\pgfpathlineto{\pgfqpoint{2.478912in}{1.400870in}}%
\pgfpathlineto{\pgfqpoint{2.484354in}{1.400870in}}%
\pgfpathlineto{\pgfqpoint{2.489796in}{1.233913in}}%
\pgfpathlineto{\pgfqpoint{2.495238in}{1.150435in}}%
\pgfpathlineto{\pgfqpoint{2.511565in}{1.400870in}}%
\pgfpathlineto{\pgfqpoint{2.522449in}{1.400870in}}%
\pgfpathlineto{\pgfqpoint{2.527891in}{1.233913in}}%
\pgfpathlineto{\pgfqpoint{2.533333in}{1.484348in}}%
\pgfpathlineto{\pgfqpoint{2.538776in}{1.484348in}}%
\pgfpathlineto{\pgfqpoint{2.549660in}{1.317391in}}%
\pgfpathlineto{\pgfqpoint{2.555102in}{1.400870in}}%
\pgfpathlineto{\pgfqpoint{2.560544in}{1.233913in}}%
\pgfpathlineto{\pgfqpoint{2.565986in}{1.317391in}}%
\pgfpathlineto{\pgfqpoint{2.571429in}{1.233913in}}%
\pgfpathlineto{\pgfqpoint{2.576871in}{1.233913in}}%
\pgfpathlineto{\pgfqpoint{2.582313in}{1.150435in}}%
\pgfpathlineto{\pgfqpoint{2.593197in}{1.317391in}}%
\pgfpathlineto{\pgfqpoint{2.604082in}{1.317391in}}%
\pgfpathlineto{\pgfqpoint{2.609524in}{1.400870in}}%
\pgfpathlineto{\pgfqpoint{2.614966in}{1.400870in}}%
\pgfpathlineto{\pgfqpoint{2.620408in}{1.317391in}}%
\pgfpathlineto{\pgfqpoint{2.625850in}{1.150435in}}%
\pgfpathlineto{\pgfqpoint{2.631293in}{1.066957in}}%
\pgfpathlineto{\pgfqpoint{2.636735in}{1.317391in}}%
\pgfpathlineto{\pgfqpoint{2.647619in}{1.317391in}}%
\pgfpathlineto{\pgfqpoint{2.653061in}{1.400870in}}%
\pgfpathlineto{\pgfqpoint{2.658503in}{1.317391in}}%
\pgfpathlineto{\pgfqpoint{2.669388in}{1.484348in}}%
\pgfpathlineto{\pgfqpoint{2.674830in}{1.400870in}}%
\pgfpathlineto{\pgfqpoint{2.680272in}{1.233913in}}%
\pgfpathlineto{\pgfqpoint{2.685714in}{1.317391in}}%
\pgfpathlineto{\pgfqpoint{2.691156in}{1.233913in}}%
\pgfpathlineto{\pgfqpoint{2.696599in}{1.400870in}}%
\pgfpathlineto{\pgfqpoint{2.702041in}{1.484348in}}%
\pgfpathlineto{\pgfqpoint{2.707483in}{1.317391in}}%
\pgfpathlineto{\pgfqpoint{2.712925in}{1.400870in}}%
\pgfpathlineto{\pgfqpoint{2.718367in}{1.233913in}}%
\pgfpathlineto{\pgfqpoint{2.729252in}{1.400870in}}%
\pgfpathlineto{\pgfqpoint{2.734694in}{1.317391in}}%
\pgfpathlineto{\pgfqpoint{2.740136in}{1.317391in}}%
\pgfpathlineto{\pgfqpoint{2.745578in}{1.400870in}}%
\pgfpathlineto{\pgfqpoint{2.751020in}{1.317391in}}%
\pgfpathlineto{\pgfqpoint{2.756463in}{1.317391in}}%
\pgfpathlineto{\pgfqpoint{2.761905in}{1.233913in}}%
\pgfpathlineto{\pgfqpoint{2.767347in}{1.233913in}}%
\pgfpathlineto{\pgfqpoint{2.772789in}{1.317391in}}%
\pgfpathlineto{\pgfqpoint{2.778231in}{1.233913in}}%
\pgfpathlineto{\pgfqpoint{2.783673in}{1.233913in}}%
\pgfpathlineto{\pgfqpoint{2.794558in}{1.400870in}}%
\pgfpathlineto{\pgfqpoint{2.805442in}{1.400870in}}%
\pgfpathlineto{\pgfqpoint{2.810884in}{1.150435in}}%
\pgfpathlineto{\pgfqpoint{2.821769in}{1.484348in}}%
\pgfpathlineto{\pgfqpoint{2.827211in}{2.068696in}}%
\pgfpathlineto{\pgfqpoint{2.832653in}{2.820000in}}%
\pgfpathlineto{\pgfqpoint{2.838095in}{3.821739in}}%
\pgfpathlineto{\pgfqpoint{2.843537in}{4.573043in}}%
\pgfpathlineto{\pgfqpoint{2.848980in}{5.073913in}}%
\pgfpathlineto{\pgfqpoint{2.854422in}{5.240870in}}%
\pgfpathlineto{\pgfqpoint{2.859864in}{5.157391in}}%
\pgfpathlineto{\pgfqpoint{2.876190in}{3.988696in}}%
\pgfpathlineto{\pgfqpoint{2.881633in}{3.654783in}}%
\pgfpathlineto{\pgfqpoint{2.887075in}{3.487826in}}%
\pgfpathlineto{\pgfqpoint{2.892517in}{3.070435in}}%
\pgfpathlineto{\pgfqpoint{2.897959in}{2.820000in}}%
\pgfpathlineto{\pgfqpoint{2.903401in}{2.986957in}}%
\pgfpathlineto{\pgfqpoint{2.914286in}{3.153913in}}%
\pgfpathlineto{\pgfqpoint{2.919728in}{3.153913in}}%
\pgfpathlineto{\pgfqpoint{2.925170in}{3.070435in}}%
\pgfpathlineto{\pgfqpoint{2.930612in}{2.903478in}}%
\pgfpathlineto{\pgfqpoint{2.936054in}{2.820000in}}%
\pgfpathlineto{\pgfqpoint{2.941497in}{2.569565in}}%
\pgfpathlineto{\pgfqpoint{2.946939in}{2.486087in}}%
\pgfpathlineto{\pgfqpoint{2.952381in}{2.235652in}}%
\pgfpathlineto{\pgfqpoint{2.963265in}{2.235652in}}%
\pgfpathlineto{\pgfqpoint{2.968707in}{2.736522in}}%
\pgfpathlineto{\pgfqpoint{2.974150in}{2.736522in}}%
\pgfpathlineto{\pgfqpoint{2.979592in}{2.903478in}}%
\pgfpathlineto{\pgfqpoint{2.985034in}{2.986957in}}%
\pgfpathlineto{\pgfqpoint{2.990476in}{2.820000in}}%
\pgfpathlineto{\pgfqpoint{2.995918in}{2.736522in}}%
\pgfpathlineto{\pgfqpoint{3.001361in}{2.569565in}}%
\pgfpathlineto{\pgfqpoint{3.006803in}{2.569565in}}%
\pgfpathlineto{\pgfqpoint{3.012245in}{2.319130in}}%
\pgfpathlineto{\pgfqpoint{3.017687in}{2.319130in}}%
\pgfpathlineto{\pgfqpoint{3.023129in}{1.985217in}}%
\pgfpathlineto{\pgfqpoint{3.028571in}{2.235652in}}%
\pgfpathlineto{\pgfqpoint{3.034014in}{2.319130in}}%
\pgfpathlineto{\pgfqpoint{3.039456in}{2.486087in}}%
\pgfpathlineto{\pgfqpoint{3.044898in}{2.569565in}}%
\pgfpathlineto{\pgfqpoint{3.050340in}{2.820000in}}%
\pgfpathlineto{\pgfqpoint{3.055782in}{2.736522in}}%
\pgfpathlineto{\pgfqpoint{3.066667in}{2.736522in}}%
\pgfpathlineto{\pgfqpoint{3.077551in}{2.402609in}}%
\pgfpathlineto{\pgfqpoint{3.082993in}{2.152174in}}%
\pgfpathlineto{\pgfqpoint{3.088435in}{2.235652in}}%
\pgfpathlineto{\pgfqpoint{3.093878in}{2.569565in}}%
\pgfpathlineto{\pgfqpoint{3.104762in}{2.903478in}}%
\pgfpathlineto{\pgfqpoint{3.110204in}{2.903478in}}%
\pgfpathlineto{\pgfqpoint{3.115646in}{2.986957in}}%
\pgfpathlineto{\pgfqpoint{3.121088in}{2.820000in}}%
\pgfpathlineto{\pgfqpoint{3.126531in}{2.736522in}}%
\pgfpathlineto{\pgfqpoint{3.131973in}{2.486087in}}%
\pgfpathlineto{\pgfqpoint{3.137415in}{2.569565in}}%
\pgfpathlineto{\pgfqpoint{3.142857in}{2.319130in}}%
\pgfpathlineto{\pgfqpoint{3.148299in}{2.152174in}}%
\pgfpathlineto{\pgfqpoint{3.159184in}{1.985217in}}%
\pgfpathlineto{\pgfqpoint{3.164626in}{1.818261in}}%
\pgfpathlineto{\pgfqpoint{3.170068in}{1.734783in}}%
\pgfpathlineto{\pgfqpoint{3.175510in}{2.152174in}}%
\pgfpathlineto{\pgfqpoint{3.180952in}{2.402609in}}%
\pgfpathlineto{\pgfqpoint{3.186395in}{2.319130in}}%
\pgfpathlineto{\pgfqpoint{3.191837in}{2.569565in}}%
\pgfpathlineto{\pgfqpoint{3.197279in}{2.653043in}}%
\pgfpathlineto{\pgfqpoint{3.202721in}{2.569565in}}%
\pgfpathlineto{\pgfqpoint{3.208163in}{2.235652in}}%
\pgfpathlineto{\pgfqpoint{3.213605in}{2.235652in}}%
\pgfpathlineto{\pgfqpoint{3.219048in}{2.068696in}}%
\pgfpathlineto{\pgfqpoint{3.224490in}{2.068696in}}%
\pgfpathlineto{\pgfqpoint{3.229932in}{1.901739in}}%
\pgfpathlineto{\pgfqpoint{3.235374in}{1.901739in}}%
\pgfpathlineto{\pgfqpoint{3.246259in}{1.567826in}}%
\pgfpathlineto{\pgfqpoint{3.251701in}{1.567826in}}%
\pgfpathlineto{\pgfqpoint{3.257143in}{1.734783in}}%
\pgfpathlineto{\pgfqpoint{3.262585in}{1.651304in}}%
\pgfpathlineto{\pgfqpoint{3.268027in}{1.484348in}}%
\pgfpathlineto{\pgfqpoint{3.273469in}{1.567826in}}%
\pgfpathlineto{\pgfqpoint{3.278912in}{1.400870in}}%
\pgfpathlineto{\pgfqpoint{3.284354in}{1.484348in}}%
\pgfpathlineto{\pgfqpoint{3.289796in}{1.400870in}}%
\pgfpathlineto{\pgfqpoint{3.295238in}{1.484348in}}%
\pgfpathlineto{\pgfqpoint{3.300680in}{1.317391in}}%
\pgfpathlineto{\pgfqpoint{3.311565in}{1.317391in}}%
\pgfpathlineto{\pgfqpoint{3.317007in}{1.150435in}}%
\pgfpathlineto{\pgfqpoint{3.322449in}{1.484348in}}%
\pgfpathlineto{\pgfqpoint{3.327891in}{1.317391in}}%
\pgfpathlineto{\pgfqpoint{3.333333in}{1.400870in}}%
\pgfpathlineto{\pgfqpoint{3.338776in}{1.317391in}}%
\pgfpathlineto{\pgfqpoint{3.344218in}{1.400870in}}%
\pgfpathlineto{\pgfqpoint{3.355102in}{1.400870in}}%
\pgfpathlineto{\pgfqpoint{3.360544in}{1.317391in}}%
\pgfpathlineto{\pgfqpoint{3.365986in}{1.484348in}}%
\pgfpathlineto{\pgfqpoint{3.371429in}{1.233913in}}%
\pgfpathlineto{\pgfqpoint{3.376871in}{1.317391in}}%
\pgfpathlineto{\pgfqpoint{3.382313in}{1.150435in}}%
\pgfpathlineto{\pgfqpoint{3.387755in}{1.484348in}}%
\pgfpathlineto{\pgfqpoint{3.393197in}{1.400870in}}%
\pgfpathlineto{\pgfqpoint{3.404082in}{1.400870in}}%
\pgfpathlineto{\pgfqpoint{3.409524in}{1.317391in}}%
\pgfpathlineto{\pgfqpoint{3.420408in}{1.317391in}}%
\pgfpathlineto{\pgfqpoint{3.425850in}{1.484348in}}%
\pgfpathlineto{\pgfqpoint{3.436735in}{1.484348in}}%
\pgfpathlineto{\pgfqpoint{3.442177in}{1.317391in}}%
\pgfpathlineto{\pgfqpoint{3.447619in}{1.400870in}}%
\pgfpathlineto{\pgfqpoint{3.458503in}{1.400870in}}%
\pgfpathlineto{\pgfqpoint{3.463946in}{1.317391in}}%
\pgfpathlineto{\pgfqpoint{3.469388in}{1.400870in}}%
\pgfpathlineto{\pgfqpoint{3.480272in}{1.400870in}}%
\pgfpathlineto{\pgfqpoint{3.485714in}{1.567826in}}%
\pgfpathlineto{\pgfqpoint{3.491156in}{1.317391in}}%
\pgfpathlineto{\pgfqpoint{3.496599in}{1.150435in}}%
\pgfpathlineto{\pgfqpoint{3.502041in}{1.400870in}}%
\pgfpathlineto{\pgfqpoint{3.507483in}{1.233913in}}%
\pgfpathlineto{\pgfqpoint{3.512925in}{1.233913in}}%
\pgfpathlineto{\pgfqpoint{3.518367in}{1.150435in}}%
\pgfpathlineto{\pgfqpoint{3.523810in}{1.150435in}}%
\pgfpathlineto{\pgfqpoint{3.529252in}{1.233913in}}%
\pgfpathlineto{\pgfqpoint{3.534694in}{1.400870in}}%
\pgfpathlineto{\pgfqpoint{3.540136in}{1.317391in}}%
\pgfpathlineto{\pgfqpoint{3.545578in}{1.484348in}}%
\pgfpathlineto{\pgfqpoint{3.551020in}{1.233913in}}%
\pgfpathlineto{\pgfqpoint{3.556463in}{1.233913in}}%
\pgfpathlineto{\pgfqpoint{3.567347in}{1.400870in}}%
\pgfpathlineto{\pgfqpoint{3.572789in}{1.400870in}}%
\pgfpathlineto{\pgfqpoint{3.578231in}{1.150435in}}%
\pgfpathlineto{\pgfqpoint{3.583673in}{1.233913in}}%
\pgfpathlineto{\pgfqpoint{3.589116in}{1.233913in}}%
\pgfpathlineto{\pgfqpoint{3.594558in}{1.400870in}}%
\pgfpathlineto{\pgfqpoint{3.600000in}{1.233913in}}%
\pgfpathlineto{\pgfqpoint{3.605442in}{1.233913in}}%
\pgfpathlineto{\pgfqpoint{3.610884in}{1.317391in}}%
\pgfpathlineto{\pgfqpoint{3.616327in}{1.484348in}}%
\pgfpathlineto{\pgfqpoint{3.621769in}{1.400870in}}%
\pgfpathlineto{\pgfqpoint{3.627211in}{1.400870in}}%
\pgfpathlineto{\pgfqpoint{3.638095in}{1.233913in}}%
\pgfpathlineto{\pgfqpoint{3.643537in}{1.317391in}}%
\pgfpathlineto{\pgfqpoint{3.654422in}{1.317391in}}%
\pgfpathlineto{\pgfqpoint{3.665306in}{1.484348in}}%
\pgfpathlineto{\pgfqpoint{3.670748in}{1.317391in}}%
\pgfpathlineto{\pgfqpoint{3.676190in}{1.233913in}}%
\pgfpathlineto{\pgfqpoint{3.681633in}{1.317391in}}%
\pgfpathlineto{\pgfqpoint{3.687075in}{1.233913in}}%
\pgfpathlineto{\pgfqpoint{3.692517in}{1.317391in}}%
\pgfpathlineto{\pgfqpoint{3.697959in}{1.484348in}}%
\pgfpathlineto{\pgfqpoint{3.703401in}{1.400870in}}%
\pgfpathlineto{\pgfqpoint{3.708844in}{1.400870in}}%
\pgfpathlineto{\pgfqpoint{3.714286in}{1.233913in}}%
\pgfpathlineto{\pgfqpoint{3.719728in}{1.484348in}}%
\pgfpathlineto{\pgfqpoint{3.730612in}{1.317391in}}%
\pgfpathlineto{\pgfqpoint{3.741497in}{1.317391in}}%
\pgfpathlineto{\pgfqpoint{3.746939in}{1.484348in}}%
\pgfpathlineto{\pgfqpoint{3.752381in}{1.317391in}}%
\pgfpathlineto{\pgfqpoint{3.757823in}{1.400870in}}%
\pgfpathlineto{\pgfqpoint{3.768707in}{1.400870in}}%
\pgfpathlineto{\pgfqpoint{3.774150in}{1.233913in}}%
\pgfpathlineto{\pgfqpoint{3.779592in}{1.317391in}}%
\pgfpathlineto{\pgfqpoint{3.790476in}{1.317391in}}%
\pgfpathlineto{\pgfqpoint{3.795918in}{1.400870in}}%
\pgfpathlineto{\pgfqpoint{3.801361in}{1.233913in}}%
\pgfpathlineto{\pgfqpoint{3.806803in}{1.400870in}}%
\pgfpathlineto{\pgfqpoint{3.812245in}{1.317391in}}%
\pgfpathlineto{\pgfqpoint{3.823129in}{1.317391in}}%
\pgfpathlineto{\pgfqpoint{3.828571in}{1.233913in}}%
\pgfpathlineto{\pgfqpoint{3.834014in}{1.400870in}}%
\pgfpathlineto{\pgfqpoint{3.839456in}{1.233913in}}%
\pgfpathlineto{\pgfqpoint{3.844898in}{1.150435in}}%
\pgfpathlineto{\pgfqpoint{3.850340in}{1.317391in}}%
\pgfpathlineto{\pgfqpoint{3.861224in}{1.317391in}}%
\pgfpathlineto{\pgfqpoint{3.866667in}{1.150435in}}%
\pgfpathlineto{\pgfqpoint{3.872109in}{1.400870in}}%
\pgfpathlineto{\pgfqpoint{3.877551in}{1.317391in}}%
\pgfpathlineto{\pgfqpoint{3.882993in}{1.150435in}}%
\pgfpathlineto{\pgfqpoint{3.888435in}{1.317391in}}%
\pgfpathlineto{\pgfqpoint{3.893878in}{1.400870in}}%
\pgfpathlineto{\pgfqpoint{3.904762in}{1.400870in}}%
\pgfpathlineto{\pgfqpoint{3.910204in}{1.317391in}}%
\pgfpathlineto{\pgfqpoint{3.915646in}{1.317391in}}%
\pgfpathlineto{\pgfqpoint{3.921088in}{1.400870in}}%
\pgfpathlineto{\pgfqpoint{3.931973in}{1.400870in}}%
\pgfpathlineto{\pgfqpoint{3.937415in}{1.233913in}}%
\pgfpathlineto{\pgfqpoint{3.942857in}{1.484348in}}%
\pgfpathlineto{\pgfqpoint{3.948299in}{1.150435in}}%
\pgfpathlineto{\pgfqpoint{3.953741in}{1.317391in}}%
\pgfpathlineto{\pgfqpoint{3.959184in}{1.400870in}}%
\pgfpathlineto{\pgfqpoint{3.964626in}{1.150435in}}%
\pgfpathlineto{\pgfqpoint{3.970068in}{1.233913in}}%
\pgfpathlineto{\pgfqpoint{3.975510in}{1.233913in}}%
\pgfpathlineto{\pgfqpoint{3.980952in}{1.400870in}}%
\pgfpathlineto{\pgfqpoint{3.986395in}{1.317391in}}%
\pgfpathlineto{\pgfqpoint{3.991837in}{1.400870in}}%
\pgfpathlineto{\pgfqpoint{3.997279in}{1.400870in}}%
\pgfpathlineto{\pgfqpoint{4.008163in}{1.233913in}}%
\pgfpathlineto{\pgfqpoint{4.013605in}{1.400870in}}%
\pgfpathlineto{\pgfqpoint{4.019048in}{1.317391in}}%
\pgfpathlineto{\pgfqpoint{4.024490in}{1.317391in}}%
\pgfpathlineto{\pgfqpoint{4.029932in}{1.484348in}}%
\pgfpathlineto{\pgfqpoint{4.035374in}{1.233913in}}%
\pgfpathlineto{\pgfqpoint{4.046259in}{1.400870in}}%
\pgfpathlineto{\pgfqpoint{4.051701in}{1.233913in}}%
\pgfpathlineto{\pgfqpoint{4.057143in}{1.317391in}}%
\pgfpathlineto{\pgfqpoint{4.062585in}{1.233913in}}%
\pgfpathlineto{\pgfqpoint{4.068027in}{1.484348in}}%
\pgfpathlineto{\pgfqpoint{4.073469in}{1.317391in}}%
\pgfpathlineto{\pgfqpoint{4.078912in}{1.233913in}}%
\pgfpathlineto{\pgfqpoint{4.089796in}{1.233913in}}%
\pgfpathlineto{\pgfqpoint{4.095238in}{1.400870in}}%
\pgfpathlineto{\pgfqpoint{4.100680in}{1.400870in}}%
\pgfpathlineto{\pgfqpoint{4.106122in}{1.233913in}}%
\pgfpathlineto{\pgfqpoint{4.111565in}{1.400870in}}%
\pgfpathlineto{\pgfqpoint{4.117007in}{1.233913in}}%
\pgfpathlineto{\pgfqpoint{4.122449in}{1.233913in}}%
\pgfpathlineto{\pgfqpoint{4.127891in}{1.400870in}}%
\pgfpathlineto{\pgfqpoint{4.138776in}{1.233913in}}%
\pgfpathlineto{\pgfqpoint{4.144218in}{1.233913in}}%
\pgfpathlineto{\pgfqpoint{4.149660in}{1.317391in}}%
\pgfpathlineto{\pgfqpoint{4.160544in}{1.317391in}}%
\pgfpathlineto{\pgfqpoint{4.165986in}{1.400870in}}%
\pgfpathlineto{\pgfqpoint{4.171429in}{1.066957in}}%
\pgfpathlineto{\pgfqpoint{4.176871in}{1.400870in}}%
\pgfpathlineto{\pgfqpoint{4.182313in}{1.233913in}}%
\pgfpathlineto{\pgfqpoint{4.187755in}{1.484348in}}%
\pgfpathlineto{\pgfqpoint{4.193197in}{1.317391in}}%
\pgfpathlineto{\pgfqpoint{4.198639in}{1.317391in}}%
\pgfpathlineto{\pgfqpoint{4.204082in}{1.150435in}}%
\pgfpathlineto{\pgfqpoint{4.209524in}{1.400870in}}%
\pgfpathlineto{\pgfqpoint{4.214966in}{1.317391in}}%
\pgfpathlineto{\pgfqpoint{4.220408in}{1.400870in}}%
\pgfpathlineto{\pgfqpoint{4.231293in}{1.400870in}}%
\pgfpathlineto{\pgfqpoint{4.236735in}{1.317391in}}%
\pgfpathlineto{\pgfqpoint{4.242177in}{1.317391in}}%
\pgfpathlineto{\pgfqpoint{4.253061in}{1.484348in}}%
\pgfpathlineto{\pgfqpoint{4.258503in}{1.400870in}}%
\pgfpathlineto{\pgfqpoint{4.263946in}{1.400870in}}%
\pgfpathlineto{\pgfqpoint{4.269388in}{1.150435in}}%
\pgfpathlineto{\pgfqpoint{4.280272in}{1.317391in}}%
\pgfpathlineto{\pgfqpoint{4.285714in}{1.233913in}}%
\pgfpathlineto{\pgfqpoint{4.291156in}{1.233913in}}%
\pgfpathlineto{\pgfqpoint{4.296599in}{1.400870in}}%
\pgfpathlineto{\pgfqpoint{4.302041in}{1.233913in}}%
\pgfpathlineto{\pgfqpoint{4.307483in}{1.233913in}}%
\pgfpathlineto{\pgfqpoint{4.312925in}{1.400870in}}%
\pgfpathlineto{\pgfqpoint{4.318367in}{1.233913in}}%
\pgfpathlineto{\pgfqpoint{4.323810in}{1.317391in}}%
\pgfpathlineto{\pgfqpoint{4.329252in}{1.317391in}}%
\pgfpathlineto{\pgfqpoint{4.334694in}{1.233913in}}%
\pgfpathlineto{\pgfqpoint{4.340136in}{1.233913in}}%
\pgfpathlineto{\pgfqpoint{4.351020in}{1.400870in}}%
\pgfpathlineto{\pgfqpoint{4.356463in}{1.400870in}}%
\pgfpathlineto{\pgfqpoint{4.361905in}{1.150435in}}%
\pgfpathlineto{\pgfqpoint{4.372789in}{1.150435in}}%
\pgfpathlineto{\pgfqpoint{4.378231in}{1.317391in}}%
\pgfpathlineto{\pgfqpoint{4.383673in}{1.317391in}}%
\pgfpathlineto{\pgfqpoint{4.394558in}{1.484348in}}%
\pgfpathlineto{\pgfqpoint{4.400000in}{1.484348in}}%
\pgfpathlineto{\pgfqpoint{4.405442in}{1.317391in}}%
\pgfpathlineto{\pgfqpoint{4.410884in}{1.317391in}}%
\pgfpathlineto{\pgfqpoint{4.421769in}{1.150435in}}%
\pgfpathlineto{\pgfqpoint{4.427211in}{1.233913in}}%
\pgfpathlineto{\pgfqpoint{4.432653in}{1.484348in}}%
\pgfpathlineto{\pgfqpoint{4.438095in}{1.233913in}}%
\pgfpathlineto{\pgfqpoint{4.443537in}{1.233913in}}%
\pgfpathlineto{\pgfqpoint{4.448980in}{1.317391in}}%
\pgfpathlineto{\pgfqpoint{4.459864in}{1.317391in}}%
\pgfpathlineto{\pgfqpoint{4.465306in}{1.233913in}}%
\pgfpathlineto{\pgfqpoint{4.470748in}{1.317391in}}%
\pgfpathlineto{\pgfqpoint{4.476190in}{1.317391in}}%
\pgfpathlineto{\pgfqpoint{4.481633in}{1.233913in}}%
\pgfpathlineto{\pgfqpoint{4.487075in}{1.317391in}}%
\pgfpathlineto{\pgfqpoint{4.492517in}{1.233913in}}%
\pgfpathlineto{\pgfqpoint{4.497959in}{1.317391in}}%
\pgfpathlineto{\pgfqpoint{4.503401in}{1.317391in}}%
\pgfpathlineto{\pgfqpoint{4.508844in}{1.400870in}}%
\pgfpathlineto{\pgfqpoint{4.514286in}{1.317391in}}%
\pgfpathlineto{\pgfqpoint{4.519728in}{1.317391in}}%
\pgfpathlineto{\pgfqpoint{4.525170in}{1.150435in}}%
\pgfpathlineto{\pgfqpoint{4.530612in}{1.400870in}}%
\pgfpathlineto{\pgfqpoint{4.536054in}{1.400870in}}%
\pgfpathlineto{\pgfqpoint{4.541497in}{1.317391in}}%
\pgfpathlineto{\pgfqpoint{4.546939in}{1.317391in}}%
\pgfpathlineto{\pgfqpoint{4.552381in}{1.233913in}}%
\pgfpathlineto{\pgfqpoint{4.557823in}{1.233913in}}%
\pgfpathlineto{\pgfqpoint{4.563265in}{1.317391in}}%
\pgfpathlineto{\pgfqpoint{4.568707in}{1.484348in}}%
\pgfpathlineto{\pgfqpoint{4.574150in}{1.400870in}}%
\pgfpathlineto{\pgfqpoint{4.579592in}{1.400870in}}%
\pgfpathlineto{\pgfqpoint{4.585034in}{1.233913in}}%
\pgfpathlineto{\pgfqpoint{4.590476in}{1.317391in}}%
\pgfpathlineto{\pgfqpoint{4.595918in}{1.317391in}}%
\pgfpathlineto{\pgfqpoint{4.601361in}{1.150435in}}%
\pgfpathlineto{\pgfqpoint{4.606803in}{1.400870in}}%
\pgfpathlineto{\pgfqpoint{4.612245in}{1.400870in}}%
\pgfpathlineto{\pgfqpoint{4.617687in}{1.150435in}}%
\pgfpathlineto{\pgfqpoint{4.623129in}{1.317391in}}%
\pgfpathlineto{\pgfqpoint{4.628571in}{1.233913in}}%
\pgfpathlineto{\pgfqpoint{4.634014in}{1.066957in}}%
\pgfpathlineto{\pgfqpoint{4.639456in}{1.567826in}}%
\pgfpathlineto{\pgfqpoint{4.650340in}{1.233913in}}%
\pgfpathlineto{\pgfqpoint{4.655782in}{1.317391in}}%
\pgfpathlineto{\pgfqpoint{4.661224in}{1.317391in}}%
\pgfpathlineto{\pgfqpoint{4.666667in}{1.400870in}}%
\pgfpathlineto{\pgfqpoint{4.672109in}{1.400870in}}%
\pgfpathlineto{\pgfqpoint{4.677551in}{1.484348in}}%
\pgfpathlineto{\pgfqpoint{4.682993in}{1.317391in}}%
\pgfpathlineto{\pgfqpoint{4.688435in}{1.484348in}}%
\pgfpathlineto{\pgfqpoint{4.693878in}{1.400870in}}%
\pgfpathlineto{\pgfqpoint{4.699320in}{1.233913in}}%
\pgfpathlineto{\pgfqpoint{4.704762in}{1.317391in}}%
\pgfpathlineto{\pgfqpoint{4.715646in}{1.317391in}}%
\pgfpathlineto{\pgfqpoint{4.721088in}{1.233913in}}%
\pgfpathlineto{\pgfqpoint{4.726531in}{1.400870in}}%
\pgfpathlineto{\pgfqpoint{4.731973in}{1.317391in}}%
\pgfpathlineto{\pgfqpoint{4.737415in}{1.317391in}}%
\pgfpathlineto{\pgfqpoint{4.742857in}{1.233913in}}%
\pgfpathlineto{\pgfqpoint{4.748299in}{1.233913in}}%
\pgfpathlineto{\pgfqpoint{4.753741in}{1.400870in}}%
\pgfpathlineto{\pgfqpoint{4.759184in}{1.317391in}}%
\pgfpathlineto{\pgfqpoint{4.775510in}{1.317391in}}%
\pgfpathlineto{\pgfqpoint{4.780952in}{1.400870in}}%
\pgfpathlineto{\pgfqpoint{4.786395in}{1.233913in}}%
\pgfpathlineto{\pgfqpoint{4.791837in}{1.400870in}}%
\pgfpathlineto{\pgfqpoint{4.797279in}{1.317391in}}%
\pgfpathlineto{\pgfqpoint{4.808163in}{1.317391in}}%
\pgfpathlineto{\pgfqpoint{4.813605in}{1.484348in}}%
\pgfpathlineto{\pgfqpoint{4.819048in}{1.317391in}}%
\pgfpathlineto{\pgfqpoint{4.824490in}{1.400870in}}%
\pgfpathlineto{\pgfqpoint{4.835374in}{1.400870in}}%
\pgfpathlineto{\pgfqpoint{4.840816in}{1.317391in}}%
\pgfpathlineto{\pgfqpoint{4.851701in}{1.317391in}}%
\pgfpathlineto{\pgfqpoint{4.857143in}{1.233913in}}%
\pgfpathlineto{\pgfqpoint{4.862585in}{1.317391in}}%
\pgfpathlineto{\pgfqpoint{4.868027in}{1.233913in}}%
\pgfpathlineto{\pgfqpoint{4.873469in}{1.400870in}}%
\pgfpathlineto{\pgfqpoint{4.878912in}{1.400870in}}%
\pgfpathlineto{\pgfqpoint{4.884354in}{1.484348in}}%
\pgfpathlineto{\pgfqpoint{4.889796in}{1.233913in}}%
\pgfpathlineto{\pgfqpoint{4.895238in}{1.400870in}}%
\pgfpathlineto{\pgfqpoint{4.906122in}{1.233913in}}%
\pgfpathlineto{\pgfqpoint{4.911565in}{1.400870in}}%
\pgfpathlineto{\pgfqpoint{4.917007in}{1.233913in}}%
\pgfpathlineto{\pgfqpoint{4.922449in}{1.233913in}}%
\pgfpathlineto{\pgfqpoint{4.927891in}{1.150435in}}%
\pgfpathlineto{\pgfqpoint{4.933333in}{1.400870in}}%
\pgfpathlineto{\pgfqpoint{4.938776in}{1.317391in}}%
\pgfpathlineto{\pgfqpoint{4.949660in}{1.317391in}}%
\pgfpathlineto{\pgfqpoint{4.955102in}{1.567826in}}%
\pgfpathlineto{\pgfqpoint{4.960544in}{1.317391in}}%
\pgfpathlineto{\pgfqpoint{4.965986in}{1.233913in}}%
\pgfpathlineto{\pgfqpoint{4.971429in}{1.484348in}}%
\pgfpathlineto{\pgfqpoint{4.976871in}{1.317391in}}%
\pgfpathlineto{\pgfqpoint{4.982313in}{1.233913in}}%
\pgfpathlineto{\pgfqpoint{4.987755in}{1.233913in}}%
\pgfpathlineto{\pgfqpoint{4.993197in}{1.150435in}}%
\pgfpathlineto{\pgfqpoint{4.998639in}{1.317391in}}%
\pgfpathlineto{\pgfqpoint{5.004082in}{1.233913in}}%
\pgfpathlineto{\pgfqpoint{5.014966in}{1.400870in}}%
\pgfpathlineto{\pgfqpoint{5.020408in}{1.317391in}}%
\pgfpathlineto{\pgfqpoint{5.025850in}{1.317391in}}%
\pgfpathlineto{\pgfqpoint{5.031293in}{1.484348in}}%
\pgfpathlineto{\pgfqpoint{5.036735in}{1.233913in}}%
\pgfpathlineto{\pgfqpoint{5.042177in}{1.233913in}}%
\pgfpathlineto{\pgfqpoint{5.047619in}{1.400870in}}%
\pgfpathlineto{\pgfqpoint{5.053061in}{1.233913in}}%
\pgfpathlineto{\pgfqpoint{5.058503in}{1.400870in}}%
\pgfpathlineto{\pgfqpoint{5.063946in}{1.317391in}}%
\pgfpathlineto{\pgfqpoint{5.069388in}{1.150435in}}%
\pgfpathlineto{\pgfqpoint{5.074830in}{1.317391in}}%
\pgfpathlineto{\pgfqpoint{5.080272in}{1.233913in}}%
\pgfpathlineto{\pgfqpoint{5.085714in}{1.233913in}}%
\pgfpathlineto{\pgfqpoint{5.091156in}{1.317391in}}%
\pgfpathlineto{\pgfqpoint{5.096599in}{1.233913in}}%
\pgfpathlineto{\pgfqpoint{5.102041in}{1.233913in}}%
\pgfpathlineto{\pgfqpoint{5.107483in}{1.651304in}}%
\pgfpathlineto{\pgfqpoint{5.112925in}{1.400870in}}%
\pgfpathlineto{\pgfqpoint{5.118367in}{1.400870in}}%
\pgfpathlineto{\pgfqpoint{5.123810in}{1.317391in}}%
\pgfpathlineto{\pgfqpoint{5.129252in}{1.317391in}}%
\pgfpathlineto{\pgfqpoint{5.134694in}{1.400870in}}%
\pgfpathlineto{\pgfqpoint{5.140136in}{1.400870in}}%
\pgfpathlineto{\pgfqpoint{5.145578in}{1.233913in}}%
\pgfpathlineto{\pgfqpoint{5.151020in}{1.150435in}}%
\pgfpathlineto{\pgfqpoint{5.156463in}{1.400870in}}%
\pgfpathlineto{\pgfqpoint{5.161905in}{1.484348in}}%
\pgfpathlineto{\pgfqpoint{5.167347in}{1.233913in}}%
\pgfpathlineto{\pgfqpoint{5.189116in}{1.233913in}}%
\pgfpathlineto{\pgfqpoint{5.194558in}{1.317391in}}%
\pgfpathlineto{\pgfqpoint{5.205442in}{1.150435in}}%
\pgfpathlineto{\pgfqpoint{5.210884in}{1.317391in}}%
\pgfpathlineto{\pgfqpoint{5.216327in}{1.317391in}}%
\pgfpathlineto{\pgfqpoint{5.227211in}{1.150435in}}%
\pgfpathlineto{\pgfqpoint{5.232653in}{1.317391in}}%
\pgfpathlineto{\pgfqpoint{5.238095in}{1.400870in}}%
\pgfpathlineto{\pgfqpoint{5.243537in}{1.400870in}}%
\pgfpathlineto{\pgfqpoint{5.248980in}{1.317391in}}%
\pgfpathlineto{\pgfqpoint{5.254422in}{1.317391in}}%
\pgfpathlineto{\pgfqpoint{5.259864in}{1.233913in}}%
\pgfpathlineto{\pgfqpoint{5.265306in}{1.484348in}}%
\pgfpathlineto{\pgfqpoint{5.270748in}{1.317391in}}%
\pgfpathlineto{\pgfqpoint{5.276190in}{1.484348in}}%
\pgfpathlineto{\pgfqpoint{5.281633in}{1.233913in}}%
\pgfpathlineto{\pgfqpoint{5.287075in}{1.317391in}}%
\pgfpathlineto{\pgfqpoint{5.292517in}{1.150435in}}%
\pgfpathlineto{\pgfqpoint{5.297959in}{1.317391in}}%
\pgfpathlineto{\pgfqpoint{5.303401in}{1.233913in}}%
\pgfpathlineto{\pgfqpoint{5.308844in}{1.317391in}}%
\pgfpathlineto{\pgfqpoint{5.314286in}{1.317391in}}%
\pgfpathlineto{\pgfqpoint{5.319728in}{1.400870in}}%
\pgfpathlineto{\pgfqpoint{5.325170in}{1.233913in}}%
\pgfpathlineto{\pgfqpoint{5.336054in}{1.400870in}}%
\pgfpathlineto{\pgfqpoint{5.341497in}{1.317391in}}%
\pgfpathlineto{\pgfqpoint{5.346939in}{1.150435in}}%
\pgfpathlineto{\pgfqpoint{5.352381in}{1.484348in}}%
\pgfpathlineto{\pgfqpoint{5.357823in}{1.233913in}}%
\pgfpathlineto{\pgfqpoint{5.363265in}{1.317391in}}%
\pgfpathlineto{\pgfqpoint{5.368707in}{1.233913in}}%
\pgfpathlineto{\pgfqpoint{5.374150in}{1.233913in}}%
\pgfpathlineto{\pgfqpoint{5.379592in}{1.317391in}}%
\pgfpathlineto{\pgfqpoint{5.385034in}{1.150435in}}%
\pgfpathlineto{\pgfqpoint{5.390476in}{1.317391in}}%
\pgfpathlineto{\pgfqpoint{5.395918in}{1.317391in}}%
\pgfpathlineto{\pgfqpoint{5.401361in}{1.484348in}}%
\pgfpathlineto{\pgfqpoint{5.412245in}{1.317391in}}%
\pgfpathlineto{\pgfqpoint{5.417687in}{1.400870in}}%
\pgfpathlineto{\pgfqpoint{5.428571in}{1.400870in}}%
\pgfpathlineto{\pgfqpoint{5.439456in}{1.233913in}}%
\pgfpathlineto{\pgfqpoint{5.444898in}{1.317391in}}%
\pgfpathlineto{\pgfqpoint{5.450340in}{1.317391in}}%
\pgfpathlineto{\pgfqpoint{5.455782in}{1.400870in}}%
\pgfpathlineto{\pgfqpoint{5.466667in}{1.400870in}}%
\pgfpathlineto{\pgfqpoint{5.472109in}{1.150435in}}%
\pgfpathlineto{\pgfqpoint{5.477551in}{1.317391in}}%
\pgfpathlineto{\pgfqpoint{5.482993in}{1.400870in}}%
\pgfpathlineto{\pgfqpoint{5.488435in}{1.233913in}}%
\pgfpathlineto{\pgfqpoint{5.493878in}{1.317391in}}%
\pgfpathlineto{\pgfqpoint{5.499320in}{1.233913in}}%
\pgfpathlineto{\pgfqpoint{5.504762in}{1.400870in}}%
\pgfpathlineto{\pgfqpoint{5.510204in}{1.317391in}}%
\pgfpathlineto{\pgfqpoint{5.515646in}{1.400870in}}%
\pgfpathlineto{\pgfqpoint{5.521088in}{1.317391in}}%
\pgfpathlineto{\pgfqpoint{5.526531in}{1.317391in}}%
\pgfpathlineto{\pgfqpoint{5.531973in}{1.400870in}}%
\pgfpathlineto{\pgfqpoint{5.537415in}{1.317391in}}%
\pgfpathlineto{\pgfqpoint{5.548299in}{1.317391in}}%
\pgfpathlineto{\pgfqpoint{5.553741in}{1.233913in}}%
\pgfpathlineto{\pgfqpoint{5.559184in}{1.400870in}}%
\pgfpathlineto{\pgfqpoint{5.564626in}{1.317391in}}%
\pgfpathlineto{\pgfqpoint{5.570068in}{1.317391in}}%
\pgfpathlineto{\pgfqpoint{5.575510in}{1.567826in}}%
\pgfpathlineto{\pgfqpoint{5.580952in}{1.400870in}}%
\pgfpathlineto{\pgfqpoint{5.586395in}{1.317391in}}%
\pgfpathlineto{\pgfqpoint{5.591837in}{1.400870in}}%
\pgfpathlineto{\pgfqpoint{5.597279in}{1.233913in}}%
\pgfpathlineto{\pgfqpoint{5.602721in}{1.317391in}}%
\pgfpathlineto{\pgfqpoint{5.613605in}{1.150435in}}%
\pgfpathlineto{\pgfqpoint{5.619048in}{1.400870in}}%
\pgfpathlineto{\pgfqpoint{5.624490in}{1.400870in}}%
\pgfpathlineto{\pgfqpoint{5.629932in}{1.233913in}}%
\pgfpathlineto{\pgfqpoint{5.635374in}{1.317391in}}%
\pgfpathlineto{\pgfqpoint{5.640816in}{1.317391in}}%
\pgfpathlineto{\pgfqpoint{5.651701in}{1.150435in}}%
\pgfpathlineto{\pgfqpoint{5.657143in}{1.233913in}}%
\pgfpathlineto{\pgfqpoint{5.662585in}{1.567826in}}%
\pgfpathlineto{\pgfqpoint{5.668027in}{1.317391in}}%
\pgfpathlineto{\pgfqpoint{5.673469in}{1.317391in}}%
\pgfpathlineto{\pgfqpoint{5.678912in}{1.150435in}}%
\pgfpathlineto{\pgfqpoint{5.684354in}{1.400870in}}%
\pgfpathlineto{\pgfqpoint{5.689796in}{1.400870in}}%
\pgfpathlineto{\pgfqpoint{5.695238in}{1.233913in}}%
\pgfpathlineto{\pgfqpoint{5.700680in}{1.317391in}}%
\pgfpathlineto{\pgfqpoint{5.706122in}{1.317391in}}%
\pgfpathlineto{\pgfqpoint{5.711565in}{1.400870in}}%
\pgfpathlineto{\pgfqpoint{5.722449in}{1.400870in}}%
\pgfpathlineto{\pgfqpoint{5.727891in}{1.317391in}}%
\pgfpathlineto{\pgfqpoint{5.733333in}{1.484348in}}%
\pgfpathlineto{\pgfqpoint{5.738776in}{1.233913in}}%
\pgfpathlineto{\pgfqpoint{5.744218in}{1.400870in}}%
\pgfpathlineto{\pgfqpoint{5.755102in}{1.233913in}}%
\pgfpathlineto{\pgfqpoint{5.765986in}{1.233913in}}%
\pgfpathlineto{\pgfqpoint{5.776871in}{1.400870in}}%
\pgfpathlineto{\pgfqpoint{5.782313in}{1.317391in}}%
\pgfpathlineto{\pgfqpoint{5.787755in}{1.484348in}}%
\pgfpathlineto{\pgfqpoint{5.793197in}{1.317391in}}%
\pgfpathlineto{\pgfqpoint{5.798639in}{1.400870in}}%
\pgfpathlineto{\pgfqpoint{5.804082in}{1.317391in}}%
\pgfpathlineto{\pgfqpoint{5.809524in}{1.400870in}}%
\pgfpathlineto{\pgfqpoint{5.814966in}{1.317391in}}%
\pgfpathlineto{\pgfqpoint{5.820408in}{1.400870in}}%
\pgfpathlineto{\pgfqpoint{5.825850in}{1.400870in}}%
\pgfpathlineto{\pgfqpoint{5.831293in}{1.233913in}}%
\pgfpathlineto{\pgfqpoint{5.836735in}{1.233913in}}%
\pgfpathlineto{\pgfqpoint{5.842177in}{1.400870in}}%
\pgfpathlineto{\pgfqpoint{5.847619in}{1.400870in}}%
\pgfpathlineto{\pgfqpoint{5.858503in}{1.233913in}}%
\pgfpathlineto{\pgfqpoint{5.863946in}{1.317391in}}%
\pgfpathlineto{\pgfqpoint{5.869388in}{1.150435in}}%
\pgfpathlineto{\pgfqpoint{5.874830in}{1.400870in}}%
\pgfpathlineto{\pgfqpoint{5.880272in}{1.317391in}}%
\pgfpathlineto{\pgfqpoint{5.885714in}{1.400870in}}%
\pgfpathlineto{\pgfqpoint{5.891156in}{1.150435in}}%
\pgfpathlineto{\pgfqpoint{5.896599in}{1.317391in}}%
\pgfpathlineto{\pgfqpoint{5.902041in}{1.400870in}}%
\pgfpathlineto{\pgfqpoint{5.907483in}{1.317391in}}%
\pgfpathlineto{\pgfqpoint{5.918367in}{1.484348in}}%
\pgfpathlineto{\pgfqpoint{5.923810in}{1.400870in}}%
\pgfpathlineto{\pgfqpoint{5.929252in}{1.484348in}}%
\pgfpathlineto{\pgfqpoint{5.934694in}{1.233913in}}%
\pgfpathlineto{\pgfqpoint{5.940136in}{1.400870in}}%
\pgfpathlineto{\pgfqpoint{5.945578in}{1.317391in}}%
\pgfpathlineto{\pgfqpoint{5.951020in}{1.484348in}}%
\pgfpathlineto{\pgfqpoint{5.956463in}{1.317391in}}%
\pgfpathlineto{\pgfqpoint{5.972789in}{1.317391in}}%
\pgfpathlineto{\pgfqpoint{5.978231in}{1.233913in}}%
\pgfpathlineto{\pgfqpoint{5.983673in}{1.400870in}}%
\pgfpathlineto{\pgfqpoint{5.989116in}{1.484348in}}%
\pgfpathlineto{\pgfqpoint{5.994558in}{1.317391in}}%
\pgfpathlineto{\pgfqpoint{6.005442in}{1.317391in}}%
\pgfpathlineto{\pgfqpoint{6.010884in}{1.150435in}}%
\pgfpathlineto{\pgfqpoint{6.016327in}{1.233913in}}%
\pgfpathlineto{\pgfqpoint{6.021769in}{1.150435in}}%
\pgfpathlineto{\pgfqpoint{6.027211in}{1.317391in}}%
\pgfpathlineto{\pgfqpoint{6.032653in}{1.400870in}}%
\pgfpathlineto{\pgfqpoint{6.038095in}{1.317391in}}%
\pgfpathlineto{\pgfqpoint{6.043537in}{1.400870in}}%
\pgfpathlineto{\pgfqpoint{6.048980in}{1.400870in}}%
\pgfpathlineto{\pgfqpoint{6.059864in}{1.233913in}}%
\pgfpathlineto{\pgfqpoint{6.065306in}{1.233913in}}%
\pgfpathlineto{\pgfqpoint{6.070748in}{1.400870in}}%
\pgfpathlineto{\pgfqpoint{6.081633in}{1.233913in}}%
\pgfpathlineto{\pgfqpoint{6.087075in}{1.317391in}}%
\pgfpathlineto{\pgfqpoint{6.092517in}{1.233913in}}%
\pgfpathlineto{\pgfqpoint{6.103401in}{1.400870in}}%
\pgfpathlineto{\pgfqpoint{6.108844in}{1.317391in}}%
\pgfpathlineto{\pgfqpoint{6.114286in}{1.150435in}}%
\pgfpathlineto{\pgfqpoint{6.119728in}{1.317391in}}%
\pgfpathlineto{\pgfqpoint{6.125170in}{1.150435in}}%
\pgfpathlineto{\pgfqpoint{6.130612in}{1.484348in}}%
\pgfpathlineto{\pgfqpoint{6.136054in}{1.317391in}}%
\pgfpathlineto{\pgfqpoint{6.141497in}{1.317391in}}%
\pgfpathlineto{\pgfqpoint{6.146939in}{1.233913in}}%
\pgfpathlineto{\pgfqpoint{6.152381in}{1.400870in}}%
\pgfpathlineto{\pgfqpoint{6.157823in}{1.317391in}}%
\pgfpathlineto{\pgfqpoint{6.163265in}{1.484348in}}%
\pgfpathlineto{\pgfqpoint{6.179592in}{1.233913in}}%
\pgfpathlineto{\pgfqpoint{6.185034in}{1.317391in}}%
\pgfpathlineto{\pgfqpoint{6.190476in}{1.317391in}}%
\pgfpathlineto{\pgfqpoint{6.195918in}{1.484348in}}%
\pgfpathlineto{\pgfqpoint{6.201361in}{1.233913in}}%
\pgfpathlineto{\pgfqpoint{6.206803in}{1.400870in}}%
\pgfpathlineto{\pgfqpoint{6.212245in}{1.484348in}}%
\pgfpathlineto{\pgfqpoint{6.217687in}{1.400870in}}%
\pgfpathlineto{\pgfqpoint{6.223129in}{1.484348in}}%
\pgfpathlineto{\pgfqpoint{6.228571in}{1.400870in}}%
\pgfpathlineto{\pgfqpoint{6.234014in}{1.233913in}}%
\pgfpathlineto{\pgfqpoint{6.239456in}{1.317391in}}%
\pgfpathlineto{\pgfqpoint{6.244898in}{1.233913in}}%
\pgfpathlineto{\pgfqpoint{6.250340in}{1.400870in}}%
\pgfpathlineto{\pgfqpoint{6.255782in}{1.317391in}}%
\pgfpathlineto{\pgfqpoint{6.261224in}{1.317391in}}%
\pgfpathlineto{\pgfqpoint{6.266667in}{1.233913in}}%
\pgfpathlineto{\pgfqpoint{6.272109in}{1.317391in}}%
\pgfpathlineto{\pgfqpoint{6.277551in}{1.233913in}}%
\pgfpathlineto{\pgfqpoint{6.282993in}{1.317391in}}%
\pgfpathlineto{\pgfqpoint{6.288435in}{1.317391in}}%
\pgfpathlineto{\pgfqpoint{6.293878in}{1.400870in}}%
\pgfpathlineto{\pgfqpoint{6.299320in}{1.233913in}}%
\pgfpathlineto{\pgfqpoint{6.304762in}{1.484348in}}%
\pgfpathlineto{\pgfqpoint{6.310204in}{1.317391in}}%
\pgfpathlineto{\pgfqpoint{6.315646in}{1.484348in}}%
\pgfpathlineto{\pgfqpoint{6.321088in}{1.317391in}}%
\pgfpathlineto{\pgfqpoint{6.326531in}{1.233913in}}%
\pgfpathlineto{\pgfqpoint{6.331973in}{1.233913in}}%
\pgfpathlineto{\pgfqpoint{6.342857in}{1.400870in}}%
\pgfpathlineto{\pgfqpoint{6.348299in}{1.400870in}}%
\pgfpathlineto{\pgfqpoint{6.353741in}{1.233913in}}%
\pgfpathlineto{\pgfqpoint{6.359184in}{1.400870in}}%
\pgfpathlineto{\pgfqpoint{6.364626in}{1.233913in}}%
\pgfpathlineto{\pgfqpoint{6.370068in}{1.317391in}}%
\pgfpathlineto{\pgfqpoint{6.375510in}{1.233913in}}%
\pgfpathlineto{\pgfqpoint{6.380952in}{1.400870in}}%
\pgfpathlineto{\pgfqpoint{6.391837in}{1.233913in}}%
\pgfpathlineto{\pgfqpoint{6.397279in}{1.317391in}}%
\pgfpathlineto{\pgfqpoint{6.402721in}{1.233913in}}%
\pgfpathlineto{\pgfqpoint{6.413605in}{1.400870in}}%
\pgfpathlineto{\pgfqpoint{6.419048in}{1.317391in}}%
\pgfpathlineto{\pgfqpoint{6.424490in}{1.400870in}}%
\pgfpathlineto{\pgfqpoint{6.435374in}{1.233913in}}%
\pgfpathlineto{\pgfqpoint{6.440816in}{1.233913in}}%
\pgfpathlineto{\pgfqpoint{6.446259in}{1.317391in}}%
\pgfpathlineto{\pgfqpoint{6.451701in}{1.233913in}}%
\pgfpathlineto{\pgfqpoint{6.457143in}{1.317391in}}%
\pgfpathlineto{\pgfqpoint{6.462585in}{1.233913in}}%
\pgfpathlineto{\pgfqpoint{6.468027in}{1.233913in}}%
\pgfpathlineto{\pgfqpoint{6.473469in}{1.150435in}}%
\pgfpathlineto{\pgfqpoint{6.478912in}{1.150435in}}%
\pgfpathlineto{\pgfqpoint{6.484354in}{1.317391in}}%
\pgfpathlineto{\pgfqpoint{6.495238in}{1.484348in}}%
\pgfpathlineto{\pgfqpoint{6.506122in}{1.317391in}}%
\pgfpathlineto{\pgfqpoint{6.511565in}{1.400870in}}%
\pgfpathlineto{\pgfqpoint{6.517007in}{1.317391in}}%
\pgfpathlineto{\pgfqpoint{6.522449in}{1.317391in}}%
\pgfpathlineto{\pgfqpoint{6.527891in}{1.233913in}}%
\pgfpathlineto{\pgfqpoint{6.538776in}{1.400870in}}%
\pgfpathlineto{\pgfqpoint{6.549660in}{1.233913in}}%
\pgfpathlineto{\pgfqpoint{6.555102in}{1.233913in}}%
\pgfpathlineto{\pgfqpoint{6.560544in}{1.400870in}}%
\pgfpathlineto{\pgfqpoint{6.565986in}{1.233913in}}%
\pgfpathlineto{\pgfqpoint{6.571429in}{1.400870in}}%
\pgfpathlineto{\pgfqpoint{6.576871in}{1.233913in}}%
\pgfpathlineto{\pgfqpoint{6.582313in}{1.567826in}}%
\pgfpathlineto{\pgfqpoint{6.587755in}{1.233913in}}%
\pgfpathlineto{\pgfqpoint{6.593197in}{1.400870in}}%
\pgfpathlineto{\pgfqpoint{6.598639in}{1.317391in}}%
\pgfpathlineto{\pgfqpoint{6.604082in}{1.400870in}}%
\pgfpathlineto{\pgfqpoint{6.609524in}{1.233913in}}%
\pgfpathlineto{\pgfqpoint{6.620408in}{1.400870in}}%
\pgfpathlineto{\pgfqpoint{6.625850in}{1.317391in}}%
\pgfpathlineto{\pgfqpoint{6.642177in}{1.317391in}}%
\pgfpathlineto{\pgfqpoint{6.647619in}{1.400870in}}%
\pgfpathlineto{\pgfqpoint{6.653061in}{1.233913in}}%
\pgfpathlineto{\pgfqpoint{6.658503in}{1.317391in}}%
\pgfpathlineto{\pgfqpoint{6.663946in}{1.317391in}}%
\pgfpathlineto{\pgfqpoint{6.669388in}{1.233913in}}%
\pgfpathlineto{\pgfqpoint{6.674830in}{1.317391in}}%
\pgfpathlineto{\pgfqpoint{6.680272in}{1.150435in}}%
\pgfpathlineto{\pgfqpoint{6.691156in}{1.317391in}}%
\pgfpathlineto{\pgfqpoint{6.696599in}{1.233913in}}%
\pgfpathlineto{\pgfqpoint{6.702041in}{1.317391in}}%
\pgfpathlineto{\pgfqpoint{6.707483in}{1.317391in}}%
\pgfpathlineto{\pgfqpoint{6.718367in}{1.150435in}}%
\pgfpathlineto{\pgfqpoint{6.729252in}{1.484348in}}%
\pgfpathlineto{\pgfqpoint{6.745578in}{1.233913in}}%
\pgfpathlineto{\pgfqpoint{6.751020in}{1.317391in}}%
\pgfpathlineto{\pgfqpoint{6.756463in}{1.233913in}}%
\pgfpathlineto{\pgfqpoint{6.761905in}{1.317391in}}%
\pgfpathlineto{\pgfqpoint{6.767347in}{1.233913in}}%
\pgfpathlineto{\pgfqpoint{6.772789in}{1.317391in}}%
\pgfpathlineto{\pgfqpoint{6.778231in}{1.484348in}}%
\pgfpathlineto{\pgfqpoint{6.783673in}{1.150435in}}%
\pgfpathlineto{\pgfqpoint{6.789116in}{1.317391in}}%
\pgfpathlineto{\pgfqpoint{6.794558in}{1.400870in}}%
\pgfpathlineto{\pgfqpoint{6.794558in}{1.400870in}}%
\pgfusepath{stroke}%
\end{pgfscope}%
\begin{pgfscope}%
\pgfsetrectcap%
\pgfsetmiterjoin%
\pgfsetlinewidth{1.003750pt}%
\definecolor{currentstroke}{rgb}{0.000000,0.000000,0.000000}%
\pgfsetstrokecolor{currentstroke}%
\pgfsetdash{}{0pt}%
\pgfpathmoveto{\pgfqpoint{1.200000in}{0.900000in}}%
\pgfpathlineto{\pgfqpoint{1.200000in}{5.700000in}}%
\pgfusepath{stroke}%
\end{pgfscope}%
\begin{pgfscope}%
\pgfsetrectcap%
\pgfsetmiterjoin%
\pgfsetlinewidth{1.003750pt}%
\definecolor{currentstroke}{rgb}{0.000000,0.000000,0.000000}%
\pgfsetstrokecolor{currentstroke}%
\pgfsetdash{}{0pt}%
\pgfpathmoveto{\pgfqpoint{6.800000in}{0.900000in}}%
\pgfpathlineto{\pgfqpoint{6.800000in}{5.700000in}}%
\pgfusepath{stroke}%
\end{pgfscope}%
\begin{pgfscope}%
\pgfsetrectcap%
\pgfsetmiterjoin%
\pgfsetlinewidth{1.003750pt}%
\definecolor{currentstroke}{rgb}{0.000000,0.000000,0.000000}%
\pgfsetstrokecolor{currentstroke}%
\pgfsetdash{}{0pt}%
\pgfpathmoveto{\pgfqpoint{1.200000in}{0.900000in}}%
\pgfpathlineto{\pgfqpoint{6.800000in}{0.900000in}}%
\pgfusepath{stroke}%
\end{pgfscope}%
\begin{pgfscope}%
\pgfsetrectcap%
\pgfsetmiterjoin%
\pgfsetlinewidth{1.003750pt}%
\definecolor{currentstroke}{rgb}{0.000000,0.000000,0.000000}%
\pgfsetstrokecolor{currentstroke}%
\pgfsetdash{}{0pt}%
\pgfpathmoveto{\pgfqpoint{1.200000in}{5.700000in}}%
\pgfpathlineto{\pgfqpoint{6.800000in}{5.700000in}}%
\pgfusepath{stroke}%
\end{pgfscope}%
\begin{pgfscope}%
\pgfsetbuttcap%
\pgfsetroundjoin%
\definecolor{currentfill}{rgb}{0.000000,0.000000,0.000000}%
\pgfsetfillcolor{currentfill}%
\pgfsetlinewidth{0.501875pt}%
\definecolor{currentstroke}{rgb}{0.000000,0.000000,0.000000}%
\pgfsetstrokecolor{currentstroke}%
\pgfsetdash{}{0pt}%
\pgfsys@defobject{currentmarker}{\pgfqpoint{0.000000in}{0.000000in}}{\pgfqpoint{0.000000in}{0.055556in}}{%
\pgfpathmoveto{\pgfqpoint{0.000000in}{0.000000in}}%
\pgfpathlineto{\pgfqpoint{0.000000in}{0.055556in}}%
\pgfusepath{stroke,fill}%
}%
\begin{pgfscope}%
\pgfsys@transformshift{1.200000in}{0.900000in}%
\pgfsys@useobject{currentmarker}{}%
\end{pgfscope}%
\end{pgfscope}%
\begin{pgfscope}%
\pgfsetbuttcap%
\pgfsetroundjoin%
\definecolor{currentfill}{rgb}{0.000000,0.000000,0.000000}%
\pgfsetfillcolor{currentfill}%
\pgfsetlinewidth{0.501875pt}%
\definecolor{currentstroke}{rgb}{0.000000,0.000000,0.000000}%
\pgfsetstrokecolor{currentstroke}%
\pgfsetdash{}{0pt}%
\pgfsys@defobject{currentmarker}{\pgfqpoint{0.000000in}{-0.055556in}}{\pgfqpoint{0.000000in}{0.000000in}}{%
\pgfpathmoveto{\pgfqpoint{0.000000in}{0.000000in}}%
\pgfpathlineto{\pgfqpoint{0.000000in}{-0.055556in}}%
\pgfusepath{stroke,fill}%
}%
\begin{pgfscope}%
\pgfsys@transformshift{1.200000in}{5.700000in}%
\pgfsys@useobject{currentmarker}{}%
\end{pgfscope}%
\end{pgfscope}%
\begin{pgfscope}%
\definecolor{textcolor}{rgb}{0.000000,0.000000,0.000000}%
\pgfsetstrokecolor{textcolor}%
\pgfsetfillcolor{textcolor}%
\pgftext[x=1.200000in,y=0.844444in,,top]{\color{textcolor}\sffamily\fontsize{20.000000}{24.000000}\selectfont \(\displaystyle {0}\)}%
\end{pgfscope}%
\begin{pgfscope}%
\pgfsetbuttcap%
\pgfsetroundjoin%
\definecolor{currentfill}{rgb}{0.000000,0.000000,0.000000}%
\pgfsetfillcolor{currentfill}%
\pgfsetlinewidth{0.501875pt}%
\definecolor{currentstroke}{rgb}{0.000000,0.000000,0.000000}%
\pgfsetstrokecolor{currentstroke}%
\pgfsetdash{}{0pt}%
\pgfsys@defobject{currentmarker}{\pgfqpoint{0.000000in}{0.000000in}}{\pgfqpoint{0.000000in}{0.055556in}}{%
\pgfpathmoveto{\pgfqpoint{0.000000in}{0.000000in}}%
\pgfpathlineto{\pgfqpoint{0.000000in}{0.055556in}}%
\pgfusepath{stroke,fill}%
}%
\begin{pgfscope}%
\pgfsys@transformshift{2.288435in}{0.900000in}%
\pgfsys@useobject{currentmarker}{}%
\end{pgfscope}%
\end{pgfscope}%
\begin{pgfscope}%
\pgfsetbuttcap%
\pgfsetroundjoin%
\definecolor{currentfill}{rgb}{0.000000,0.000000,0.000000}%
\pgfsetfillcolor{currentfill}%
\pgfsetlinewidth{0.501875pt}%
\definecolor{currentstroke}{rgb}{0.000000,0.000000,0.000000}%
\pgfsetstrokecolor{currentstroke}%
\pgfsetdash{}{0pt}%
\pgfsys@defobject{currentmarker}{\pgfqpoint{0.000000in}{-0.055556in}}{\pgfqpoint{0.000000in}{0.000000in}}{%
\pgfpathmoveto{\pgfqpoint{0.000000in}{0.000000in}}%
\pgfpathlineto{\pgfqpoint{0.000000in}{-0.055556in}}%
\pgfusepath{stroke,fill}%
}%
\begin{pgfscope}%
\pgfsys@transformshift{2.288435in}{5.700000in}%
\pgfsys@useobject{currentmarker}{}%
\end{pgfscope}%
\end{pgfscope}%
\begin{pgfscope}%
\definecolor{textcolor}{rgb}{0.000000,0.000000,0.000000}%
\pgfsetstrokecolor{textcolor}%
\pgfsetfillcolor{textcolor}%
\pgftext[x=2.288435in,y=0.844444in,,top]{\color{textcolor}\sffamily\fontsize{20.000000}{24.000000}\selectfont \(\displaystyle {200}\)}%
\end{pgfscope}%
\begin{pgfscope}%
\pgfsetbuttcap%
\pgfsetroundjoin%
\definecolor{currentfill}{rgb}{0.000000,0.000000,0.000000}%
\pgfsetfillcolor{currentfill}%
\pgfsetlinewidth{0.501875pt}%
\definecolor{currentstroke}{rgb}{0.000000,0.000000,0.000000}%
\pgfsetstrokecolor{currentstroke}%
\pgfsetdash{}{0pt}%
\pgfsys@defobject{currentmarker}{\pgfqpoint{0.000000in}{0.000000in}}{\pgfqpoint{0.000000in}{0.055556in}}{%
\pgfpathmoveto{\pgfqpoint{0.000000in}{0.000000in}}%
\pgfpathlineto{\pgfqpoint{0.000000in}{0.055556in}}%
\pgfusepath{stroke,fill}%
}%
\begin{pgfscope}%
\pgfsys@transformshift{3.376871in}{0.900000in}%
\pgfsys@useobject{currentmarker}{}%
\end{pgfscope}%
\end{pgfscope}%
\begin{pgfscope}%
\pgfsetbuttcap%
\pgfsetroundjoin%
\definecolor{currentfill}{rgb}{0.000000,0.000000,0.000000}%
\pgfsetfillcolor{currentfill}%
\pgfsetlinewidth{0.501875pt}%
\definecolor{currentstroke}{rgb}{0.000000,0.000000,0.000000}%
\pgfsetstrokecolor{currentstroke}%
\pgfsetdash{}{0pt}%
\pgfsys@defobject{currentmarker}{\pgfqpoint{0.000000in}{-0.055556in}}{\pgfqpoint{0.000000in}{0.000000in}}{%
\pgfpathmoveto{\pgfqpoint{0.000000in}{0.000000in}}%
\pgfpathlineto{\pgfqpoint{0.000000in}{-0.055556in}}%
\pgfusepath{stroke,fill}%
}%
\begin{pgfscope}%
\pgfsys@transformshift{3.376871in}{5.700000in}%
\pgfsys@useobject{currentmarker}{}%
\end{pgfscope}%
\end{pgfscope}%
\begin{pgfscope}%
\definecolor{textcolor}{rgb}{0.000000,0.000000,0.000000}%
\pgfsetstrokecolor{textcolor}%
\pgfsetfillcolor{textcolor}%
\pgftext[x=3.376871in,y=0.844444in,,top]{\color{textcolor}\sffamily\fontsize{20.000000}{24.000000}\selectfont \(\displaystyle {400}\)}%
\end{pgfscope}%
\begin{pgfscope}%
\pgfsetbuttcap%
\pgfsetroundjoin%
\definecolor{currentfill}{rgb}{0.000000,0.000000,0.000000}%
\pgfsetfillcolor{currentfill}%
\pgfsetlinewidth{0.501875pt}%
\definecolor{currentstroke}{rgb}{0.000000,0.000000,0.000000}%
\pgfsetstrokecolor{currentstroke}%
\pgfsetdash{}{0pt}%
\pgfsys@defobject{currentmarker}{\pgfqpoint{0.000000in}{0.000000in}}{\pgfqpoint{0.000000in}{0.055556in}}{%
\pgfpathmoveto{\pgfqpoint{0.000000in}{0.000000in}}%
\pgfpathlineto{\pgfqpoint{0.000000in}{0.055556in}}%
\pgfusepath{stroke,fill}%
}%
\begin{pgfscope}%
\pgfsys@transformshift{4.465306in}{0.900000in}%
\pgfsys@useobject{currentmarker}{}%
\end{pgfscope}%
\end{pgfscope}%
\begin{pgfscope}%
\pgfsetbuttcap%
\pgfsetroundjoin%
\definecolor{currentfill}{rgb}{0.000000,0.000000,0.000000}%
\pgfsetfillcolor{currentfill}%
\pgfsetlinewidth{0.501875pt}%
\definecolor{currentstroke}{rgb}{0.000000,0.000000,0.000000}%
\pgfsetstrokecolor{currentstroke}%
\pgfsetdash{}{0pt}%
\pgfsys@defobject{currentmarker}{\pgfqpoint{0.000000in}{-0.055556in}}{\pgfqpoint{0.000000in}{0.000000in}}{%
\pgfpathmoveto{\pgfqpoint{0.000000in}{0.000000in}}%
\pgfpathlineto{\pgfqpoint{0.000000in}{-0.055556in}}%
\pgfusepath{stroke,fill}%
}%
\begin{pgfscope}%
\pgfsys@transformshift{4.465306in}{5.700000in}%
\pgfsys@useobject{currentmarker}{}%
\end{pgfscope}%
\end{pgfscope}%
\begin{pgfscope}%
\definecolor{textcolor}{rgb}{0.000000,0.000000,0.000000}%
\pgfsetstrokecolor{textcolor}%
\pgfsetfillcolor{textcolor}%
\pgftext[x=4.465306in,y=0.844444in,,top]{\color{textcolor}\sffamily\fontsize{20.000000}{24.000000}\selectfont \(\displaystyle {600}\)}%
\end{pgfscope}%
\begin{pgfscope}%
\pgfsetbuttcap%
\pgfsetroundjoin%
\definecolor{currentfill}{rgb}{0.000000,0.000000,0.000000}%
\pgfsetfillcolor{currentfill}%
\pgfsetlinewidth{0.501875pt}%
\definecolor{currentstroke}{rgb}{0.000000,0.000000,0.000000}%
\pgfsetstrokecolor{currentstroke}%
\pgfsetdash{}{0pt}%
\pgfsys@defobject{currentmarker}{\pgfqpoint{0.000000in}{0.000000in}}{\pgfqpoint{0.000000in}{0.055556in}}{%
\pgfpathmoveto{\pgfqpoint{0.000000in}{0.000000in}}%
\pgfpathlineto{\pgfqpoint{0.000000in}{0.055556in}}%
\pgfusepath{stroke,fill}%
}%
\begin{pgfscope}%
\pgfsys@transformshift{5.553741in}{0.900000in}%
\pgfsys@useobject{currentmarker}{}%
\end{pgfscope}%
\end{pgfscope}%
\begin{pgfscope}%
\pgfsetbuttcap%
\pgfsetroundjoin%
\definecolor{currentfill}{rgb}{0.000000,0.000000,0.000000}%
\pgfsetfillcolor{currentfill}%
\pgfsetlinewidth{0.501875pt}%
\definecolor{currentstroke}{rgb}{0.000000,0.000000,0.000000}%
\pgfsetstrokecolor{currentstroke}%
\pgfsetdash{}{0pt}%
\pgfsys@defobject{currentmarker}{\pgfqpoint{0.000000in}{-0.055556in}}{\pgfqpoint{0.000000in}{0.000000in}}{%
\pgfpathmoveto{\pgfqpoint{0.000000in}{0.000000in}}%
\pgfpathlineto{\pgfqpoint{0.000000in}{-0.055556in}}%
\pgfusepath{stroke,fill}%
}%
\begin{pgfscope}%
\pgfsys@transformshift{5.553741in}{5.700000in}%
\pgfsys@useobject{currentmarker}{}%
\end{pgfscope}%
\end{pgfscope}%
\begin{pgfscope}%
\definecolor{textcolor}{rgb}{0.000000,0.000000,0.000000}%
\pgfsetstrokecolor{textcolor}%
\pgfsetfillcolor{textcolor}%
\pgftext[x=5.553741in,y=0.844444in,,top]{\color{textcolor}\sffamily\fontsize{20.000000}{24.000000}\selectfont \(\displaystyle {800}\)}%
\end{pgfscope}%
\begin{pgfscope}%
\pgfsetbuttcap%
\pgfsetroundjoin%
\definecolor{currentfill}{rgb}{0.000000,0.000000,0.000000}%
\pgfsetfillcolor{currentfill}%
\pgfsetlinewidth{0.501875pt}%
\definecolor{currentstroke}{rgb}{0.000000,0.000000,0.000000}%
\pgfsetstrokecolor{currentstroke}%
\pgfsetdash{}{0pt}%
\pgfsys@defobject{currentmarker}{\pgfqpoint{0.000000in}{0.000000in}}{\pgfqpoint{0.000000in}{0.055556in}}{%
\pgfpathmoveto{\pgfqpoint{0.000000in}{0.000000in}}%
\pgfpathlineto{\pgfqpoint{0.000000in}{0.055556in}}%
\pgfusepath{stroke,fill}%
}%
\begin{pgfscope}%
\pgfsys@transformshift{6.642177in}{0.900000in}%
\pgfsys@useobject{currentmarker}{}%
\end{pgfscope}%
\end{pgfscope}%
\begin{pgfscope}%
\pgfsetbuttcap%
\pgfsetroundjoin%
\definecolor{currentfill}{rgb}{0.000000,0.000000,0.000000}%
\pgfsetfillcolor{currentfill}%
\pgfsetlinewidth{0.501875pt}%
\definecolor{currentstroke}{rgb}{0.000000,0.000000,0.000000}%
\pgfsetstrokecolor{currentstroke}%
\pgfsetdash{}{0pt}%
\pgfsys@defobject{currentmarker}{\pgfqpoint{0.000000in}{-0.055556in}}{\pgfqpoint{0.000000in}{0.000000in}}{%
\pgfpathmoveto{\pgfqpoint{0.000000in}{0.000000in}}%
\pgfpathlineto{\pgfqpoint{0.000000in}{-0.055556in}}%
\pgfusepath{stroke,fill}%
}%
\begin{pgfscope}%
\pgfsys@transformshift{6.642177in}{5.700000in}%
\pgfsys@useobject{currentmarker}{}%
\end{pgfscope}%
\end{pgfscope}%
\begin{pgfscope}%
\definecolor{textcolor}{rgb}{0.000000,0.000000,0.000000}%
\pgfsetstrokecolor{textcolor}%
\pgfsetfillcolor{textcolor}%
\pgftext[x=6.642177in,y=0.844444in,,top]{\color{textcolor}\sffamily\fontsize{20.000000}{24.000000}\selectfont \(\displaystyle {1000}\)}%
\end{pgfscope}%
\begin{pgfscope}%
\definecolor{textcolor}{rgb}{0.000000,0.000000,0.000000}%
\pgfsetstrokecolor{textcolor}%
\pgfsetfillcolor{textcolor}%
\pgftext[x=4.000000in,y=0.518932in,,top]{\color{textcolor}\sffamily\fontsize{20.000000}{24.000000}\selectfont \(\displaystyle \mathrm{t}/\si{ns}\)}%
\end{pgfscope}%
\begin{pgfscope}%
\pgfsetbuttcap%
\pgfsetroundjoin%
\definecolor{currentfill}{rgb}{0.000000,0.000000,0.000000}%
\pgfsetfillcolor{currentfill}%
\pgfsetlinewidth{0.501875pt}%
\definecolor{currentstroke}{rgb}{0.000000,0.000000,0.000000}%
\pgfsetstrokecolor{currentstroke}%
\pgfsetdash{}{0pt}%
\pgfsys@defobject{currentmarker}{\pgfqpoint{0.000000in}{0.000000in}}{\pgfqpoint{0.055556in}{0.000000in}}{%
\pgfpathmoveto{\pgfqpoint{0.000000in}{0.000000in}}%
\pgfpathlineto{\pgfqpoint{0.055556in}{0.000000in}}%
\pgfusepath{stroke,fill}%
}%
\begin{pgfscope}%
\pgfsys@transformshift{1.200000in}{1.317391in}%
\pgfsys@useobject{currentmarker}{}%
\end{pgfscope}%
\end{pgfscope}%
\begin{pgfscope}%
\pgfsetbuttcap%
\pgfsetroundjoin%
\definecolor{currentfill}{rgb}{0.000000,0.000000,0.000000}%
\pgfsetfillcolor{currentfill}%
\pgfsetlinewidth{0.501875pt}%
\definecolor{currentstroke}{rgb}{0.000000,0.000000,0.000000}%
\pgfsetstrokecolor{currentstroke}%
\pgfsetdash{}{0pt}%
\pgfsys@defobject{currentmarker}{\pgfqpoint{-0.055556in}{0.000000in}}{\pgfqpoint{-0.000000in}{0.000000in}}{%
\pgfpathmoveto{\pgfqpoint{-0.000000in}{0.000000in}}%
\pgfpathlineto{\pgfqpoint{-0.055556in}{0.000000in}}%
\pgfusepath{stroke,fill}%
}%
\begin{pgfscope}%
\pgfsys@transformshift{6.800000in}{1.317391in}%
\pgfsys@useobject{currentmarker}{}%
\end{pgfscope}%
\end{pgfscope}%
\begin{pgfscope}%
\definecolor{textcolor}{rgb}{0.000000,0.000000,0.000000}%
\pgfsetstrokecolor{textcolor}%
\pgfsetfillcolor{textcolor}%
\pgftext[x=1.144444in,y=1.317391in,right,]{\color{textcolor}\sffamily\fontsize{20.000000}{24.000000}\selectfont \(\displaystyle {0}\)}%
\end{pgfscope}%
\begin{pgfscope}%
\pgfsetbuttcap%
\pgfsetroundjoin%
\definecolor{currentfill}{rgb}{0.000000,0.000000,0.000000}%
\pgfsetfillcolor{currentfill}%
\pgfsetlinewidth{0.501875pt}%
\definecolor{currentstroke}{rgb}{0.000000,0.000000,0.000000}%
\pgfsetstrokecolor{currentstroke}%
\pgfsetdash{}{0pt}%
\pgfsys@defobject{currentmarker}{\pgfqpoint{0.000000in}{0.000000in}}{\pgfqpoint{0.055556in}{0.000000in}}{%
\pgfpathmoveto{\pgfqpoint{0.000000in}{0.000000in}}%
\pgfpathlineto{\pgfqpoint{0.055556in}{0.000000in}}%
\pgfusepath{stroke,fill}%
}%
\begin{pgfscope}%
\pgfsys@transformshift{1.200000in}{2.152174in}%
\pgfsys@useobject{currentmarker}{}%
\end{pgfscope}%
\end{pgfscope}%
\begin{pgfscope}%
\pgfsetbuttcap%
\pgfsetroundjoin%
\definecolor{currentfill}{rgb}{0.000000,0.000000,0.000000}%
\pgfsetfillcolor{currentfill}%
\pgfsetlinewidth{0.501875pt}%
\definecolor{currentstroke}{rgb}{0.000000,0.000000,0.000000}%
\pgfsetstrokecolor{currentstroke}%
\pgfsetdash{}{0pt}%
\pgfsys@defobject{currentmarker}{\pgfqpoint{-0.055556in}{0.000000in}}{\pgfqpoint{-0.000000in}{0.000000in}}{%
\pgfpathmoveto{\pgfqpoint{-0.000000in}{0.000000in}}%
\pgfpathlineto{\pgfqpoint{-0.055556in}{0.000000in}}%
\pgfusepath{stroke,fill}%
}%
\begin{pgfscope}%
\pgfsys@transformshift{6.800000in}{2.152174in}%
\pgfsys@useobject{currentmarker}{}%
\end{pgfscope}%
\end{pgfscope}%
\begin{pgfscope}%
\definecolor{textcolor}{rgb}{0.000000,0.000000,0.000000}%
\pgfsetstrokecolor{textcolor}%
\pgfsetfillcolor{textcolor}%
\pgftext[x=1.144444in,y=2.152174in,right,]{\color{textcolor}\sffamily\fontsize{20.000000}{24.000000}\selectfont \(\displaystyle {10}\)}%
\end{pgfscope}%
\begin{pgfscope}%
\pgfsetbuttcap%
\pgfsetroundjoin%
\definecolor{currentfill}{rgb}{0.000000,0.000000,0.000000}%
\pgfsetfillcolor{currentfill}%
\pgfsetlinewidth{0.501875pt}%
\definecolor{currentstroke}{rgb}{0.000000,0.000000,0.000000}%
\pgfsetstrokecolor{currentstroke}%
\pgfsetdash{}{0pt}%
\pgfsys@defobject{currentmarker}{\pgfqpoint{0.000000in}{0.000000in}}{\pgfqpoint{0.055556in}{0.000000in}}{%
\pgfpathmoveto{\pgfqpoint{0.000000in}{0.000000in}}%
\pgfpathlineto{\pgfqpoint{0.055556in}{0.000000in}}%
\pgfusepath{stroke,fill}%
}%
\begin{pgfscope}%
\pgfsys@transformshift{1.200000in}{2.986957in}%
\pgfsys@useobject{currentmarker}{}%
\end{pgfscope}%
\end{pgfscope}%
\begin{pgfscope}%
\pgfsetbuttcap%
\pgfsetroundjoin%
\definecolor{currentfill}{rgb}{0.000000,0.000000,0.000000}%
\pgfsetfillcolor{currentfill}%
\pgfsetlinewidth{0.501875pt}%
\definecolor{currentstroke}{rgb}{0.000000,0.000000,0.000000}%
\pgfsetstrokecolor{currentstroke}%
\pgfsetdash{}{0pt}%
\pgfsys@defobject{currentmarker}{\pgfqpoint{-0.055556in}{0.000000in}}{\pgfqpoint{-0.000000in}{0.000000in}}{%
\pgfpathmoveto{\pgfqpoint{-0.000000in}{0.000000in}}%
\pgfpathlineto{\pgfqpoint{-0.055556in}{0.000000in}}%
\pgfusepath{stroke,fill}%
}%
\begin{pgfscope}%
\pgfsys@transformshift{6.800000in}{2.986957in}%
\pgfsys@useobject{currentmarker}{}%
\end{pgfscope}%
\end{pgfscope}%
\begin{pgfscope}%
\definecolor{textcolor}{rgb}{0.000000,0.000000,0.000000}%
\pgfsetstrokecolor{textcolor}%
\pgfsetfillcolor{textcolor}%
\pgftext[x=1.144444in,y=2.986957in,right,]{\color{textcolor}\sffamily\fontsize{20.000000}{24.000000}\selectfont \(\displaystyle {20}\)}%
\end{pgfscope}%
\begin{pgfscope}%
\pgfsetbuttcap%
\pgfsetroundjoin%
\definecolor{currentfill}{rgb}{0.000000,0.000000,0.000000}%
\pgfsetfillcolor{currentfill}%
\pgfsetlinewidth{0.501875pt}%
\definecolor{currentstroke}{rgb}{0.000000,0.000000,0.000000}%
\pgfsetstrokecolor{currentstroke}%
\pgfsetdash{}{0pt}%
\pgfsys@defobject{currentmarker}{\pgfqpoint{0.000000in}{0.000000in}}{\pgfqpoint{0.055556in}{0.000000in}}{%
\pgfpathmoveto{\pgfqpoint{0.000000in}{0.000000in}}%
\pgfpathlineto{\pgfqpoint{0.055556in}{0.000000in}}%
\pgfusepath{stroke,fill}%
}%
\begin{pgfscope}%
\pgfsys@transformshift{1.200000in}{3.821739in}%
\pgfsys@useobject{currentmarker}{}%
\end{pgfscope}%
\end{pgfscope}%
\begin{pgfscope}%
\pgfsetbuttcap%
\pgfsetroundjoin%
\definecolor{currentfill}{rgb}{0.000000,0.000000,0.000000}%
\pgfsetfillcolor{currentfill}%
\pgfsetlinewidth{0.501875pt}%
\definecolor{currentstroke}{rgb}{0.000000,0.000000,0.000000}%
\pgfsetstrokecolor{currentstroke}%
\pgfsetdash{}{0pt}%
\pgfsys@defobject{currentmarker}{\pgfqpoint{-0.055556in}{0.000000in}}{\pgfqpoint{-0.000000in}{0.000000in}}{%
\pgfpathmoveto{\pgfqpoint{-0.000000in}{0.000000in}}%
\pgfpathlineto{\pgfqpoint{-0.055556in}{0.000000in}}%
\pgfusepath{stroke,fill}%
}%
\begin{pgfscope}%
\pgfsys@transformshift{6.800000in}{3.821739in}%
\pgfsys@useobject{currentmarker}{}%
\end{pgfscope}%
\end{pgfscope}%
\begin{pgfscope}%
\definecolor{textcolor}{rgb}{0.000000,0.000000,0.000000}%
\pgfsetstrokecolor{textcolor}%
\pgfsetfillcolor{textcolor}%
\pgftext[x=1.144444in,y=3.821739in,right,]{\color{textcolor}\sffamily\fontsize{20.000000}{24.000000}\selectfont \(\displaystyle {30}\)}%
\end{pgfscope}%
\begin{pgfscope}%
\pgfsetbuttcap%
\pgfsetroundjoin%
\definecolor{currentfill}{rgb}{0.000000,0.000000,0.000000}%
\pgfsetfillcolor{currentfill}%
\pgfsetlinewidth{0.501875pt}%
\definecolor{currentstroke}{rgb}{0.000000,0.000000,0.000000}%
\pgfsetstrokecolor{currentstroke}%
\pgfsetdash{}{0pt}%
\pgfsys@defobject{currentmarker}{\pgfqpoint{0.000000in}{0.000000in}}{\pgfqpoint{0.055556in}{0.000000in}}{%
\pgfpathmoveto{\pgfqpoint{0.000000in}{0.000000in}}%
\pgfpathlineto{\pgfqpoint{0.055556in}{0.000000in}}%
\pgfusepath{stroke,fill}%
}%
\begin{pgfscope}%
\pgfsys@transformshift{1.200000in}{4.656522in}%
\pgfsys@useobject{currentmarker}{}%
\end{pgfscope}%
\end{pgfscope}%
\begin{pgfscope}%
\pgfsetbuttcap%
\pgfsetroundjoin%
\definecolor{currentfill}{rgb}{0.000000,0.000000,0.000000}%
\pgfsetfillcolor{currentfill}%
\pgfsetlinewidth{0.501875pt}%
\definecolor{currentstroke}{rgb}{0.000000,0.000000,0.000000}%
\pgfsetstrokecolor{currentstroke}%
\pgfsetdash{}{0pt}%
\pgfsys@defobject{currentmarker}{\pgfqpoint{-0.055556in}{0.000000in}}{\pgfqpoint{-0.000000in}{0.000000in}}{%
\pgfpathmoveto{\pgfqpoint{-0.000000in}{0.000000in}}%
\pgfpathlineto{\pgfqpoint{-0.055556in}{0.000000in}}%
\pgfusepath{stroke,fill}%
}%
\begin{pgfscope}%
\pgfsys@transformshift{6.800000in}{4.656522in}%
\pgfsys@useobject{currentmarker}{}%
\end{pgfscope}%
\end{pgfscope}%
\begin{pgfscope}%
\definecolor{textcolor}{rgb}{0.000000,0.000000,0.000000}%
\pgfsetstrokecolor{textcolor}%
\pgfsetfillcolor{textcolor}%
\pgftext[x=1.144444in,y=4.656522in,right,]{\color{textcolor}\sffamily\fontsize{20.000000}{24.000000}\selectfont \(\displaystyle {40}\)}%
\end{pgfscope}%
\begin{pgfscope}%
\pgfsetbuttcap%
\pgfsetroundjoin%
\definecolor{currentfill}{rgb}{0.000000,0.000000,0.000000}%
\pgfsetfillcolor{currentfill}%
\pgfsetlinewidth{0.501875pt}%
\definecolor{currentstroke}{rgb}{0.000000,0.000000,0.000000}%
\pgfsetstrokecolor{currentstroke}%
\pgfsetdash{}{0pt}%
\pgfsys@defobject{currentmarker}{\pgfqpoint{0.000000in}{0.000000in}}{\pgfqpoint{0.055556in}{0.000000in}}{%
\pgfpathmoveto{\pgfqpoint{0.000000in}{0.000000in}}%
\pgfpathlineto{\pgfqpoint{0.055556in}{0.000000in}}%
\pgfusepath{stroke,fill}%
}%
\begin{pgfscope}%
\pgfsys@transformshift{1.200000in}{5.491304in}%
\pgfsys@useobject{currentmarker}{}%
\end{pgfscope}%
\end{pgfscope}%
\begin{pgfscope}%
\pgfsetbuttcap%
\pgfsetroundjoin%
\definecolor{currentfill}{rgb}{0.000000,0.000000,0.000000}%
\pgfsetfillcolor{currentfill}%
\pgfsetlinewidth{0.501875pt}%
\definecolor{currentstroke}{rgb}{0.000000,0.000000,0.000000}%
\pgfsetstrokecolor{currentstroke}%
\pgfsetdash{}{0pt}%
\pgfsys@defobject{currentmarker}{\pgfqpoint{-0.055556in}{0.000000in}}{\pgfqpoint{-0.000000in}{0.000000in}}{%
\pgfpathmoveto{\pgfqpoint{-0.000000in}{0.000000in}}%
\pgfpathlineto{\pgfqpoint{-0.055556in}{0.000000in}}%
\pgfusepath{stroke,fill}%
}%
\begin{pgfscope}%
\pgfsys@transformshift{6.800000in}{5.491304in}%
\pgfsys@useobject{currentmarker}{}%
\end{pgfscope}%
\end{pgfscope}%
\begin{pgfscope}%
\definecolor{textcolor}{rgb}{0.000000,0.000000,0.000000}%
\pgfsetstrokecolor{textcolor}%
\pgfsetfillcolor{textcolor}%
\pgftext[x=1.144444in,y=5.491304in,right,]{\color{textcolor}\sffamily\fontsize{20.000000}{24.000000}\selectfont \(\displaystyle {50}\)}%
\end{pgfscope}%
\begin{pgfscope}%
\definecolor{textcolor}{rgb}{0.000000,0.000000,0.000000}%
\pgfsetstrokecolor{textcolor}%
\pgfsetfillcolor{textcolor}%
\pgftext[x=0.810785in,y=3.300000in,,bottom,rotate=90.000000]{\color{textcolor}\sffamily\fontsize{20.000000}{24.000000}\selectfont \(\displaystyle \mathrm{Voltage}/\si{mV}\)}%
\end{pgfscope}%
\begin{pgfscope}%
\pgfsetbuttcap%
\pgfsetmiterjoin%
\definecolor{currentfill}{rgb}{1.000000,1.000000,1.000000}%
\pgfsetfillcolor{currentfill}%
\pgfsetlinewidth{1.003750pt}%
\definecolor{currentstroke}{rgb}{0.000000,0.000000,0.000000}%
\pgfsetstrokecolor{currentstroke}%
\pgfsetdash{}{0pt}%
\pgfpathmoveto{\pgfqpoint{4.066020in}{4.959484in}}%
\pgfpathlineto{\pgfqpoint{6.633333in}{4.959484in}}%
\pgfpathlineto{\pgfqpoint{6.633333in}{5.533333in}}%
\pgfpathlineto{\pgfqpoint{4.066020in}{5.533333in}}%
\pgfpathclose%
\pgfusepath{stroke,fill}%
\end{pgfscope}%
\begin{pgfscope}%
\pgfsetrectcap%
\pgfsetroundjoin%
\pgfsetlinewidth{2.007500pt}%
\definecolor{currentstroke}{rgb}{0.000000,0.000000,1.000000}%
\pgfsetstrokecolor{currentstroke}%
\pgfsetdash{}{0pt}%
\pgfpathmoveto{\pgfqpoint{4.299353in}{5.276697in}}%
\pgfpathlineto{\pgfqpoint{4.766020in}{5.276697in}}%
\pgfusepath{stroke}%
\end{pgfscope}%
\begin{pgfscope}%
\definecolor{textcolor}{rgb}{0.000000,0.000000,0.000000}%
\pgfsetstrokecolor{textcolor}%
\pgfsetfillcolor{textcolor}%
\pgftext[x=5.132687in,y=5.160031in,left,base]{\color{textcolor}\sffamily\fontsize{24.000000}{28.800000}\selectfont Waveform}%
\end{pgfscope}%
\end{pgfpicture}%
\makeatother%
\endgroup%
}
    \caption{\label{fig:pile} Pile-up in PMT waveform}
\end{figure}
\end{minipage}
\end{figure}

The data acquisition system will digitalize the time and voltage of recorded PMT output. Most of the time, when handling PMT waveform, we record the time of the first PE according to the threshold of voltage, $v_{th}$, and total charge in a DAQ window. The voltage which is higher than a certain threshold is assumed to be the consequence of photoelectron. The total charge is the integration of the waveform. Charge induced single photoelectron (SPE) can be a wide distribution (usually truncated normal distribution, mean is $\mu_{g}$, the standard deviation is $\sigma_{g}$) rather than a single value (see the black curve in figure~\ref{fig:recchargehist}). In these practical scenarios, one waveform is converted to a pair of numbers. More detailed information on the waveform was lost. 

\subsection{toy Monte Carlo}

Currently, there are two successful detector materials: water (heavy water) and liquid scintillator. The time profile of the scintillation light is always described as several components with different decay times and Cherenkov light is always described as $\delta$ function \cite{ludhova_particle_2020}. 

In toy MC, we describe the light transmit process using a hypothetical parameterized time profile model. We assume the time profile of the liquid scintillator is an exponential function convoluted by a gaussian function \cite{li_separation_2016} caused by the transit time spread (TTS) of PMT, with the form in the formula \eqref{eq:time-pro} where $\tau$ is the decay time and $\sigma$ is the TTS while $t_{0}$ is the starting time of time profile which account for the unknown beginning time of light emission. Figure~\ref{fig:time-pro} where $t_{0}=0$ shows the shape of time profile. 

% \begin{figure}[H]
\begin{minipage}{.4\textwidth}
\begin{align}
    P_{t}(t) = \frac{1}{\tau}e^{-\frac{t - t_{0}}{\tau}} \otimes \mathcal{N}(\sigma^{2})
    \label{eq:time-pro}
\end{align}
\end{minipage}
\begin{minipage}{.6\textwidth}
\begin{figure}[H]
    \centering
    \resizebox{\textwidth}{!}{%% Creator: Matplotlib, PGF backend
%%
%% To include the figure in your LaTeX document, write
%%   \input{<filename>.pgf}
%%
%% Make sure the required packages are loaded in your preamble
%%   \usepackage{pgf}
%%
%% and, on pdftex
%%   \usepackage[utf8]{inputenc}\DeclareUnicodeCharacter{2212}{-}
%%
%% or, on luatex and xetex
%%   \usepackage{unicode-math}
%%
%% Figures using additional raster images can only be included by \input if
%% they are in the same directory as the main LaTeX file. For loading figures
%% from other directories you can use the `import` package
%%   \usepackage{import}
%%
%% and then include the figures with
%%   \import{<path to file>}{<filename>.pgf}
%%
%% Matplotlib used the following preamble
%%
\begingroup%
\makeatletter%
\begin{pgfpicture}%
\pgfpathrectangle{\pgfpointorigin}{\pgfqpoint{8.000000in}{6.000000in}}%
\pgfusepath{use as bounding box, clip}%
\begin{pgfscope}%
\pgfsetbuttcap%
\pgfsetmiterjoin%
\definecolor{currentfill}{rgb}{1.000000,1.000000,1.000000}%
\pgfsetfillcolor{currentfill}%
\pgfsetlinewidth{0.000000pt}%
\definecolor{currentstroke}{rgb}{1.000000,1.000000,1.000000}%
\pgfsetstrokecolor{currentstroke}%
\pgfsetdash{}{0pt}%
\pgfpathmoveto{\pgfqpoint{0.000000in}{0.000000in}}%
\pgfpathlineto{\pgfqpoint{8.000000in}{0.000000in}}%
\pgfpathlineto{\pgfqpoint{8.000000in}{6.000000in}}%
\pgfpathlineto{\pgfqpoint{0.000000in}{6.000000in}}%
\pgfpathclose%
\pgfusepath{fill}%
\end{pgfscope}%
\begin{pgfscope}%
\pgfsetbuttcap%
\pgfsetmiterjoin%
\definecolor{currentfill}{rgb}{1.000000,1.000000,1.000000}%
\pgfsetfillcolor{currentfill}%
\pgfsetlinewidth{0.000000pt}%
\definecolor{currentstroke}{rgb}{0.000000,0.000000,0.000000}%
\pgfsetstrokecolor{currentstroke}%
\pgfsetstrokeopacity{0.000000}%
\pgfsetdash{}{0pt}%
\pgfpathmoveto{\pgfqpoint{1.200000in}{0.900000in}}%
\pgfpathlineto{\pgfqpoint{6.800000in}{0.900000in}}%
\pgfpathlineto{\pgfqpoint{6.800000in}{5.700000in}}%
\pgfpathlineto{\pgfqpoint{1.200000in}{5.700000in}}%
\pgfpathclose%
\pgfusepath{fill}%
\end{pgfscope}%
\begin{pgfscope}%
\pgfpathrectangle{\pgfqpoint{1.200000in}{0.900000in}}{\pgfqpoint{5.600000in}{4.800000in}}%
\pgfusepath{clip}%
\pgfsetrectcap%
\pgfsetroundjoin%
\pgfsetlinewidth{2.007500pt}%
\definecolor{currentstroke}{rgb}{0.000000,0.000000,1.000000}%
\pgfsetstrokecolor{currentstroke}%
\pgfsetdash{}{0pt}%
\pgfpathmoveto{\pgfqpoint{1.305660in}{1.883216in}}%
\pgfpathlineto{\pgfqpoint{1.326792in}{2.036248in}}%
\pgfpathlineto{\pgfqpoint{1.347925in}{2.202479in}}%
\pgfpathlineto{\pgfqpoint{1.379623in}{2.474750in}}%
\pgfpathlineto{\pgfqpoint{1.411321in}{2.770719in}}%
\pgfpathlineto{\pgfqpoint{1.453585in}{3.191895in}}%
\pgfpathlineto{\pgfqpoint{1.548679in}{4.159094in}}%
\pgfpathlineto{\pgfqpoint{1.580377in}{4.456532in}}%
\pgfpathlineto{\pgfqpoint{1.612075in}{4.727863in}}%
\pgfpathlineto{\pgfqpoint{1.633208in}{4.890990in}}%
\pgfpathlineto{\pgfqpoint{1.654340in}{5.038107in}}%
\pgfpathlineto{\pgfqpoint{1.675472in}{5.168088in}}%
\pgfpathlineto{\pgfqpoint{1.696604in}{5.280190in}}%
\pgfpathlineto{\pgfqpoint{1.717736in}{5.374061in}}%
\pgfpathlineto{\pgfqpoint{1.728302in}{5.414151in}}%
\pgfpathlineto{\pgfqpoint{1.738868in}{5.449716in}}%
\pgfpathlineto{\pgfqpoint{1.749434in}{5.480812in}}%
\pgfpathlineto{\pgfqpoint{1.760000in}{5.507512in}}%
\pgfpathlineto{\pgfqpoint{1.770566in}{5.529908in}}%
\pgfpathlineto{\pgfqpoint{1.781132in}{5.548108in}}%
\pgfpathlineto{\pgfqpoint{1.791698in}{5.562234in}}%
\pgfpathlineto{\pgfqpoint{1.802264in}{5.572421in}}%
\pgfpathlineto{\pgfqpoint{1.812830in}{5.578813in}}%
\pgfpathlineto{\pgfqpoint{1.823396in}{5.581568in}}%
\pgfpathlineto{\pgfqpoint{1.833962in}{5.580847in}}%
\pgfpathlineto{\pgfqpoint{1.844528in}{5.576820in}}%
\pgfpathlineto{\pgfqpoint{1.855094in}{5.569661in}}%
\pgfpathlineto{\pgfqpoint{1.865660in}{5.559546in}}%
\pgfpathlineto{\pgfqpoint{1.876226in}{5.546654in}}%
\pgfpathlineto{\pgfqpoint{1.886792in}{5.531163in}}%
\pgfpathlineto{\pgfqpoint{1.897358in}{5.513251in}}%
\pgfpathlineto{\pgfqpoint{1.918491in}{5.470862in}}%
\pgfpathlineto{\pgfqpoint{1.939623in}{5.420846in}}%
\pgfpathlineto{\pgfqpoint{1.960755in}{5.364485in}}%
\pgfpathlineto{\pgfqpoint{1.992453in}{5.270608in}}%
\pgfpathlineto{\pgfqpoint{2.024151in}{5.168617in}}%
\pgfpathlineto{\pgfqpoint{2.066415in}{5.024973in}}%
\pgfpathlineto{\pgfqpoint{2.235472in}{4.441688in}}%
\pgfpathlineto{\pgfqpoint{2.288302in}{4.270330in}}%
\pgfpathlineto{\pgfqpoint{2.341132in}{4.106510in}}%
\pgfpathlineto{\pgfqpoint{2.393962in}{3.950336in}}%
\pgfpathlineto{\pgfqpoint{2.446792in}{3.801644in}}%
\pgfpathlineto{\pgfqpoint{2.499623in}{3.660154in}}%
\pgfpathlineto{\pgfqpoint{2.552453in}{3.525548in}}%
\pgfpathlineto{\pgfqpoint{2.605283in}{3.397501in}}%
\pgfpathlineto{\pgfqpoint{2.658113in}{3.275697in}}%
\pgfpathlineto{\pgfqpoint{2.710943in}{3.159833in}}%
\pgfpathlineto{\pgfqpoint{2.763774in}{3.049620in}}%
\pgfpathlineto{\pgfqpoint{2.816604in}{2.944781in}}%
\pgfpathlineto{\pgfqpoint{2.869434in}{2.845056in}}%
\pgfpathlineto{\pgfqpoint{2.922264in}{2.750195in}}%
\pgfpathlineto{\pgfqpoint{2.975094in}{2.659960in}}%
\pgfpathlineto{\pgfqpoint{3.027925in}{2.574125in}}%
\pgfpathlineto{\pgfqpoint{3.080755in}{2.492477in}}%
\pgfpathlineto{\pgfqpoint{3.133585in}{2.414811in}}%
\pgfpathlineto{\pgfqpoint{3.186415in}{2.340933in}}%
\pgfpathlineto{\pgfqpoint{3.239245in}{2.270658in}}%
\pgfpathlineto{\pgfqpoint{3.292075in}{2.203810in}}%
\pgfpathlineto{\pgfqpoint{3.344906in}{2.140223in}}%
\pgfpathlineto{\pgfqpoint{3.397736in}{2.079736in}}%
\pgfpathlineto{\pgfqpoint{3.450566in}{2.022200in}}%
\pgfpathlineto{\pgfqpoint{3.503396in}{1.967469in}}%
\pgfpathlineto{\pgfqpoint{3.566792in}{1.905305in}}%
\pgfpathlineto{\pgfqpoint{3.630189in}{1.846760in}}%
\pgfpathlineto{\pgfqpoint{3.693585in}{1.791625in}}%
\pgfpathlineto{\pgfqpoint{3.756981in}{1.739701in}}%
\pgfpathlineto{\pgfqpoint{3.820377in}{1.690801in}}%
\pgfpathlineto{\pgfqpoint{3.883774in}{1.644748in}}%
\pgfpathlineto{\pgfqpoint{3.947170in}{1.601377in}}%
\pgfpathlineto{\pgfqpoint{4.010566in}{1.560532in}}%
\pgfpathlineto{\pgfqpoint{4.073962in}{1.522066in}}%
\pgfpathlineto{\pgfqpoint{4.147925in}{1.480010in}}%
\pgfpathlineto{\pgfqpoint{4.221887in}{1.440798in}}%
\pgfpathlineto{\pgfqpoint{4.295849in}{1.404237in}}%
\pgfpathlineto{\pgfqpoint{4.369811in}{1.370147in}}%
\pgfpathlineto{\pgfqpoint{4.443774in}{1.338362in}}%
\pgfpathlineto{\pgfqpoint{4.528302in}{1.304660in}}%
\pgfpathlineto{\pgfqpoint{4.612830in}{1.273548in}}%
\pgfpathlineto{\pgfqpoint{4.697358in}{1.244828in}}%
\pgfpathlineto{\pgfqpoint{4.781887in}{1.218316in}}%
\pgfpathlineto{\pgfqpoint{4.876981in}{1.190919in}}%
\pgfpathlineto{\pgfqpoint{4.972075in}{1.165880in}}%
\pgfpathlineto{\pgfqpoint{5.067170in}{1.142996in}}%
\pgfpathlineto{\pgfqpoint{5.172830in}{1.119872in}}%
\pgfpathlineto{\pgfqpoint{5.278491in}{1.098949in}}%
\pgfpathlineto{\pgfqpoint{5.394717in}{1.078225in}}%
\pgfpathlineto{\pgfqpoint{5.510943in}{1.059660in}}%
\pgfpathlineto{\pgfqpoint{5.637736in}{1.041606in}}%
\pgfpathlineto{\pgfqpoint{5.775094in}{1.024343in}}%
\pgfpathlineto{\pgfqpoint{5.923019in}{1.008099in}}%
\pgfpathlineto{\pgfqpoint{6.081509in}{0.993042in}}%
\pgfpathlineto{\pgfqpoint{6.250566in}{0.979285in}}%
\pgfpathlineto{\pgfqpoint{6.430189in}{0.966890in}}%
\pgfpathlineto{\pgfqpoint{6.630943in}{0.955315in}}%
\pgfpathlineto{\pgfqpoint{6.789434in}{0.947610in}}%
\pgfpathlineto{\pgfqpoint{6.789434in}{0.947610in}}%
\pgfusepath{stroke}%
\end{pgfscope}%
\begin{pgfscope}%
\pgfsetrectcap%
\pgfsetmiterjoin%
\pgfsetlinewidth{1.003750pt}%
\definecolor{currentstroke}{rgb}{0.000000,0.000000,0.000000}%
\pgfsetstrokecolor{currentstroke}%
\pgfsetdash{}{0pt}%
\pgfpathmoveto{\pgfqpoint{1.200000in}{0.900000in}}%
\pgfpathlineto{\pgfqpoint{1.200000in}{5.700000in}}%
\pgfusepath{stroke}%
\end{pgfscope}%
\begin{pgfscope}%
\pgfsetrectcap%
\pgfsetmiterjoin%
\pgfsetlinewidth{1.003750pt}%
\definecolor{currentstroke}{rgb}{0.000000,0.000000,0.000000}%
\pgfsetstrokecolor{currentstroke}%
\pgfsetdash{}{0pt}%
\pgfpathmoveto{\pgfqpoint{6.800000in}{0.900000in}}%
\pgfpathlineto{\pgfqpoint{6.800000in}{5.700000in}}%
\pgfusepath{stroke}%
\end{pgfscope}%
\begin{pgfscope}%
\pgfsetrectcap%
\pgfsetmiterjoin%
\pgfsetlinewidth{1.003750pt}%
\definecolor{currentstroke}{rgb}{0.000000,0.000000,0.000000}%
\pgfsetstrokecolor{currentstroke}%
\pgfsetdash{}{0pt}%
\pgfpathmoveto{\pgfqpoint{1.200000in}{0.900000in}}%
\pgfpathlineto{\pgfqpoint{6.800000in}{0.900000in}}%
\pgfusepath{stroke}%
\end{pgfscope}%
\begin{pgfscope}%
\pgfsetrectcap%
\pgfsetmiterjoin%
\pgfsetlinewidth{1.003750pt}%
\definecolor{currentstroke}{rgb}{0.000000,0.000000,0.000000}%
\pgfsetstrokecolor{currentstroke}%
\pgfsetdash{}{0pt}%
\pgfpathmoveto{\pgfqpoint{1.200000in}{5.700000in}}%
\pgfpathlineto{\pgfqpoint{6.800000in}{5.700000in}}%
\pgfusepath{stroke}%
\end{pgfscope}%
\begin{pgfscope}%
\pgfpathrectangle{\pgfqpoint{1.200000in}{0.900000in}}{\pgfqpoint{5.600000in}{4.800000in}}%
\pgfusepath{clip}%
\pgfsetbuttcap%
\pgfsetroundjoin%
\pgfsetlinewidth{0.501875pt}%
\definecolor{currentstroke}{rgb}{0.000000,0.000000,0.000000}%
\pgfsetstrokecolor{currentstroke}%
\pgfsetdash{{1.000000pt}{3.000000pt}}{0.000000pt}%
\pgfpathmoveto{\pgfqpoint{1.516981in}{0.900000in}}%
\pgfpathlineto{\pgfqpoint{1.516981in}{5.700000in}}%
\pgfusepath{stroke}%
\end{pgfscope}%
\begin{pgfscope}%
\pgfsetbuttcap%
\pgfsetroundjoin%
\definecolor{currentfill}{rgb}{0.000000,0.000000,0.000000}%
\pgfsetfillcolor{currentfill}%
\pgfsetlinewidth{0.501875pt}%
\definecolor{currentstroke}{rgb}{0.000000,0.000000,0.000000}%
\pgfsetstrokecolor{currentstroke}%
\pgfsetdash{}{0pt}%
\pgfsys@defobject{currentmarker}{\pgfqpoint{0.000000in}{0.000000in}}{\pgfqpoint{0.000000in}{0.055556in}}{%
\pgfpathmoveto{\pgfqpoint{0.000000in}{0.000000in}}%
\pgfpathlineto{\pgfqpoint{0.000000in}{0.055556in}}%
\pgfusepath{stroke,fill}%
}%
\begin{pgfscope}%
\pgfsys@transformshift{1.516981in}{0.900000in}%
\pgfsys@useobject{currentmarker}{}%
\end{pgfscope}%
\end{pgfscope}%
\begin{pgfscope}%
\pgfsetbuttcap%
\pgfsetroundjoin%
\definecolor{currentfill}{rgb}{0.000000,0.000000,0.000000}%
\pgfsetfillcolor{currentfill}%
\pgfsetlinewidth{0.501875pt}%
\definecolor{currentstroke}{rgb}{0.000000,0.000000,0.000000}%
\pgfsetstrokecolor{currentstroke}%
\pgfsetdash{}{0pt}%
\pgfsys@defobject{currentmarker}{\pgfqpoint{0.000000in}{-0.055556in}}{\pgfqpoint{0.000000in}{0.000000in}}{%
\pgfpathmoveto{\pgfqpoint{0.000000in}{0.000000in}}%
\pgfpathlineto{\pgfqpoint{0.000000in}{-0.055556in}}%
\pgfusepath{stroke,fill}%
}%
\begin{pgfscope}%
\pgfsys@transformshift{1.516981in}{5.700000in}%
\pgfsys@useobject{currentmarker}{}%
\end{pgfscope}%
\end{pgfscope}%
\begin{pgfscope}%
\definecolor{textcolor}{rgb}{0.000000,0.000000,0.000000}%
\pgfsetstrokecolor{textcolor}%
\pgfsetfillcolor{textcolor}%
\pgftext[x=1.516981in,y=0.844444in,,top]{\color{textcolor}\sffamily\fontsize{20.000000}{24.000000}\selectfont \(\displaystyle {0}\)}%
\end{pgfscope}%
\begin{pgfscope}%
\pgfpathrectangle{\pgfqpoint{1.200000in}{0.900000in}}{\pgfqpoint{5.600000in}{4.800000in}}%
\pgfusepath{clip}%
\pgfsetbuttcap%
\pgfsetroundjoin%
\pgfsetlinewidth{0.501875pt}%
\definecolor{currentstroke}{rgb}{0.000000,0.000000,0.000000}%
\pgfsetstrokecolor{currentstroke}%
\pgfsetdash{{1.000000pt}{3.000000pt}}{0.000000pt}%
\pgfpathmoveto{\pgfqpoint{2.573585in}{0.900000in}}%
\pgfpathlineto{\pgfqpoint{2.573585in}{5.700000in}}%
\pgfusepath{stroke}%
\end{pgfscope}%
\begin{pgfscope}%
\pgfsetbuttcap%
\pgfsetroundjoin%
\definecolor{currentfill}{rgb}{0.000000,0.000000,0.000000}%
\pgfsetfillcolor{currentfill}%
\pgfsetlinewidth{0.501875pt}%
\definecolor{currentstroke}{rgb}{0.000000,0.000000,0.000000}%
\pgfsetstrokecolor{currentstroke}%
\pgfsetdash{}{0pt}%
\pgfsys@defobject{currentmarker}{\pgfqpoint{0.000000in}{0.000000in}}{\pgfqpoint{0.000000in}{0.055556in}}{%
\pgfpathmoveto{\pgfqpoint{0.000000in}{0.000000in}}%
\pgfpathlineto{\pgfqpoint{0.000000in}{0.055556in}}%
\pgfusepath{stroke,fill}%
}%
\begin{pgfscope}%
\pgfsys@transformshift{2.573585in}{0.900000in}%
\pgfsys@useobject{currentmarker}{}%
\end{pgfscope}%
\end{pgfscope}%
\begin{pgfscope}%
\pgfsetbuttcap%
\pgfsetroundjoin%
\definecolor{currentfill}{rgb}{0.000000,0.000000,0.000000}%
\pgfsetfillcolor{currentfill}%
\pgfsetlinewidth{0.501875pt}%
\definecolor{currentstroke}{rgb}{0.000000,0.000000,0.000000}%
\pgfsetstrokecolor{currentstroke}%
\pgfsetdash{}{0pt}%
\pgfsys@defobject{currentmarker}{\pgfqpoint{0.000000in}{-0.055556in}}{\pgfqpoint{0.000000in}{0.000000in}}{%
\pgfpathmoveto{\pgfqpoint{0.000000in}{0.000000in}}%
\pgfpathlineto{\pgfqpoint{0.000000in}{-0.055556in}}%
\pgfusepath{stroke,fill}%
}%
\begin{pgfscope}%
\pgfsys@transformshift{2.573585in}{5.700000in}%
\pgfsys@useobject{currentmarker}{}%
\end{pgfscope}%
\end{pgfscope}%
\begin{pgfscope}%
\definecolor{textcolor}{rgb}{0.000000,0.000000,0.000000}%
\pgfsetstrokecolor{textcolor}%
\pgfsetfillcolor{textcolor}%
\pgftext[x=2.573585in,y=0.844444in,,top]{\color{textcolor}\sffamily\fontsize{20.000000}{24.000000}\selectfont \(\displaystyle {10}\)}%
\end{pgfscope}%
\begin{pgfscope}%
\pgfpathrectangle{\pgfqpoint{1.200000in}{0.900000in}}{\pgfqpoint{5.600000in}{4.800000in}}%
\pgfusepath{clip}%
\pgfsetbuttcap%
\pgfsetroundjoin%
\pgfsetlinewidth{0.501875pt}%
\definecolor{currentstroke}{rgb}{0.000000,0.000000,0.000000}%
\pgfsetstrokecolor{currentstroke}%
\pgfsetdash{{1.000000pt}{3.000000pt}}{0.000000pt}%
\pgfpathmoveto{\pgfqpoint{3.630189in}{0.900000in}}%
\pgfpathlineto{\pgfqpoint{3.630189in}{5.700000in}}%
\pgfusepath{stroke}%
\end{pgfscope}%
\begin{pgfscope}%
\pgfsetbuttcap%
\pgfsetroundjoin%
\definecolor{currentfill}{rgb}{0.000000,0.000000,0.000000}%
\pgfsetfillcolor{currentfill}%
\pgfsetlinewidth{0.501875pt}%
\definecolor{currentstroke}{rgb}{0.000000,0.000000,0.000000}%
\pgfsetstrokecolor{currentstroke}%
\pgfsetdash{}{0pt}%
\pgfsys@defobject{currentmarker}{\pgfqpoint{0.000000in}{0.000000in}}{\pgfqpoint{0.000000in}{0.055556in}}{%
\pgfpathmoveto{\pgfqpoint{0.000000in}{0.000000in}}%
\pgfpathlineto{\pgfqpoint{0.000000in}{0.055556in}}%
\pgfusepath{stroke,fill}%
}%
\begin{pgfscope}%
\pgfsys@transformshift{3.630189in}{0.900000in}%
\pgfsys@useobject{currentmarker}{}%
\end{pgfscope}%
\end{pgfscope}%
\begin{pgfscope}%
\pgfsetbuttcap%
\pgfsetroundjoin%
\definecolor{currentfill}{rgb}{0.000000,0.000000,0.000000}%
\pgfsetfillcolor{currentfill}%
\pgfsetlinewidth{0.501875pt}%
\definecolor{currentstroke}{rgb}{0.000000,0.000000,0.000000}%
\pgfsetstrokecolor{currentstroke}%
\pgfsetdash{}{0pt}%
\pgfsys@defobject{currentmarker}{\pgfqpoint{0.000000in}{-0.055556in}}{\pgfqpoint{0.000000in}{0.000000in}}{%
\pgfpathmoveto{\pgfqpoint{0.000000in}{0.000000in}}%
\pgfpathlineto{\pgfqpoint{0.000000in}{-0.055556in}}%
\pgfusepath{stroke,fill}%
}%
\begin{pgfscope}%
\pgfsys@transformshift{3.630189in}{5.700000in}%
\pgfsys@useobject{currentmarker}{}%
\end{pgfscope}%
\end{pgfscope}%
\begin{pgfscope}%
\definecolor{textcolor}{rgb}{0.000000,0.000000,0.000000}%
\pgfsetstrokecolor{textcolor}%
\pgfsetfillcolor{textcolor}%
\pgftext[x=3.630189in,y=0.844444in,,top]{\color{textcolor}\sffamily\fontsize{20.000000}{24.000000}\selectfont \(\displaystyle {20}\)}%
\end{pgfscope}%
\begin{pgfscope}%
\pgfpathrectangle{\pgfqpoint{1.200000in}{0.900000in}}{\pgfqpoint{5.600000in}{4.800000in}}%
\pgfusepath{clip}%
\pgfsetbuttcap%
\pgfsetroundjoin%
\pgfsetlinewidth{0.501875pt}%
\definecolor{currentstroke}{rgb}{0.000000,0.000000,0.000000}%
\pgfsetstrokecolor{currentstroke}%
\pgfsetdash{{1.000000pt}{3.000000pt}}{0.000000pt}%
\pgfpathmoveto{\pgfqpoint{4.686792in}{0.900000in}}%
\pgfpathlineto{\pgfqpoint{4.686792in}{5.700000in}}%
\pgfusepath{stroke}%
\end{pgfscope}%
\begin{pgfscope}%
\pgfsetbuttcap%
\pgfsetroundjoin%
\definecolor{currentfill}{rgb}{0.000000,0.000000,0.000000}%
\pgfsetfillcolor{currentfill}%
\pgfsetlinewidth{0.501875pt}%
\definecolor{currentstroke}{rgb}{0.000000,0.000000,0.000000}%
\pgfsetstrokecolor{currentstroke}%
\pgfsetdash{}{0pt}%
\pgfsys@defobject{currentmarker}{\pgfqpoint{0.000000in}{0.000000in}}{\pgfqpoint{0.000000in}{0.055556in}}{%
\pgfpathmoveto{\pgfqpoint{0.000000in}{0.000000in}}%
\pgfpathlineto{\pgfqpoint{0.000000in}{0.055556in}}%
\pgfusepath{stroke,fill}%
}%
\begin{pgfscope}%
\pgfsys@transformshift{4.686792in}{0.900000in}%
\pgfsys@useobject{currentmarker}{}%
\end{pgfscope}%
\end{pgfscope}%
\begin{pgfscope}%
\pgfsetbuttcap%
\pgfsetroundjoin%
\definecolor{currentfill}{rgb}{0.000000,0.000000,0.000000}%
\pgfsetfillcolor{currentfill}%
\pgfsetlinewidth{0.501875pt}%
\definecolor{currentstroke}{rgb}{0.000000,0.000000,0.000000}%
\pgfsetstrokecolor{currentstroke}%
\pgfsetdash{}{0pt}%
\pgfsys@defobject{currentmarker}{\pgfqpoint{0.000000in}{-0.055556in}}{\pgfqpoint{0.000000in}{0.000000in}}{%
\pgfpathmoveto{\pgfqpoint{0.000000in}{0.000000in}}%
\pgfpathlineto{\pgfqpoint{0.000000in}{-0.055556in}}%
\pgfusepath{stroke,fill}%
}%
\begin{pgfscope}%
\pgfsys@transformshift{4.686792in}{5.700000in}%
\pgfsys@useobject{currentmarker}{}%
\end{pgfscope}%
\end{pgfscope}%
\begin{pgfscope}%
\definecolor{textcolor}{rgb}{0.000000,0.000000,0.000000}%
\pgfsetstrokecolor{textcolor}%
\pgfsetfillcolor{textcolor}%
\pgftext[x=4.686792in,y=0.844444in,,top]{\color{textcolor}\sffamily\fontsize{20.000000}{24.000000}\selectfont \(\displaystyle {30}\)}%
\end{pgfscope}%
\begin{pgfscope}%
\pgfpathrectangle{\pgfqpoint{1.200000in}{0.900000in}}{\pgfqpoint{5.600000in}{4.800000in}}%
\pgfusepath{clip}%
\pgfsetbuttcap%
\pgfsetroundjoin%
\pgfsetlinewidth{0.501875pt}%
\definecolor{currentstroke}{rgb}{0.000000,0.000000,0.000000}%
\pgfsetstrokecolor{currentstroke}%
\pgfsetdash{{1.000000pt}{3.000000pt}}{0.000000pt}%
\pgfpathmoveto{\pgfqpoint{5.743396in}{0.900000in}}%
\pgfpathlineto{\pgfqpoint{5.743396in}{5.700000in}}%
\pgfusepath{stroke}%
\end{pgfscope}%
\begin{pgfscope}%
\pgfsetbuttcap%
\pgfsetroundjoin%
\definecolor{currentfill}{rgb}{0.000000,0.000000,0.000000}%
\pgfsetfillcolor{currentfill}%
\pgfsetlinewidth{0.501875pt}%
\definecolor{currentstroke}{rgb}{0.000000,0.000000,0.000000}%
\pgfsetstrokecolor{currentstroke}%
\pgfsetdash{}{0pt}%
\pgfsys@defobject{currentmarker}{\pgfqpoint{0.000000in}{0.000000in}}{\pgfqpoint{0.000000in}{0.055556in}}{%
\pgfpathmoveto{\pgfqpoint{0.000000in}{0.000000in}}%
\pgfpathlineto{\pgfqpoint{0.000000in}{0.055556in}}%
\pgfusepath{stroke,fill}%
}%
\begin{pgfscope}%
\pgfsys@transformshift{5.743396in}{0.900000in}%
\pgfsys@useobject{currentmarker}{}%
\end{pgfscope}%
\end{pgfscope}%
\begin{pgfscope}%
\pgfsetbuttcap%
\pgfsetroundjoin%
\definecolor{currentfill}{rgb}{0.000000,0.000000,0.000000}%
\pgfsetfillcolor{currentfill}%
\pgfsetlinewidth{0.501875pt}%
\definecolor{currentstroke}{rgb}{0.000000,0.000000,0.000000}%
\pgfsetstrokecolor{currentstroke}%
\pgfsetdash{}{0pt}%
\pgfsys@defobject{currentmarker}{\pgfqpoint{0.000000in}{-0.055556in}}{\pgfqpoint{0.000000in}{0.000000in}}{%
\pgfpathmoveto{\pgfqpoint{0.000000in}{0.000000in}}%
\pgfpathlineto{\pgfqpoint{0.000000in}{-0.055556in}}%
\pgfusepath{stroke,fill}%
}%
\begin{pgfscope}%
\pgfsys@transformshift{5.743396in}{5.700000in}%
\pgfsys@useobject{currentmarker}{}%
\end{pgfscope}%
\end{pgfscope}%
\begin{pgfscope}%
\definecolor{textcolor}{rgb}{0.000000,0.000000,0.000000}%
\pgfsetstrokecolor{textcolor}%
\pgfsetfillcolor{textcolor}%
\pgftext[x=5.743396in,y=0.844444in,,top]{\color{textcolor}\sffamily\fontsize{20.000000}{24.000000}\selectfont \(\displaystyle {40}\)}%
\end{pgfscope}%
\begin{pgfscope}%
\pgfpathrectangle{\pgfqpoint{1.200000in}{0.900000in}}{\pgfqpoint{5.600000in}{4.800000in}}%
\pgfusepath{clip}%
\pgfsetbuttcap%
\pgfsetroundjoin%
\pgfsetlinewidth{0.501875pt}%
\definecolor{currentstroke}{rgb}{0.000000,0.000000,0.000000}%
\pgfsetstrokecolor{currentstroke}%
\pgfsetdash{{1.000000pt}{3.000000pt}}{0.000000pt}%
\pgfpathmoveto{\pgfqpoint{6.800000in}{0.900000in}}%
\pgfpathlineto{\pgfqpoint{6.800000in}{5.700000in}}%
\pgfusepath{stroke}%
\end{pgfscope}%
\begin{pgfscope}%
\pgfsetbuttcap%
\pgfsetroundjoin%
\definecolor{currentfill}{rgb}{0.000000,0.000000,0.000000}%
\pgfsetfillcolor{currentfill}%
\pgfsetlinewidth{0.501875pt}%
\definecolor{currentstroke}{rgb}{0.000000,0.000000,0.000000}%
\pgfsetstrokecolor{currentstroke}%
\pgfsetdash{}{0pt}%
\pgfsys@defobject{currentmarker}{\pgfqpoint{0.000000in}{0.000000in}}{\pgfqpoint{0.000000in}{0.055556in}}{%
\pgfpathmoveto{\pgfqpoint{0.000000in}{0.000000in}}%
\pgfpathlineto{\pgfqpoint{0.000000in}{0.055556in}}%
\pgfusepath{stroke,fill}%
}%
\begin{pgfscope}%
\pgfsys@transformshift{6.800000in}{0.900000in}%
\pgfsys@useobject{currentmarker}{}%
\end{pgfscope}%
\end{pgfscope}%
\begin{pgfscope}%
\pgfsetbuttcap%
\pgfsetroundjoin%
\definecolor{currentfill}{rgb}{0.000000,0.000000,0.000000}%
\pgfsetfillcolor{currentfill}%
\pgfsetlinewidth{0.501875pt}%
\definecolor{currentstroke}{rgb}{0.000000,0.000000,0.000000}%
\pgfsetstrokecolor{currentstroke}%
\pgfsetdash{}{0pt}%
\pgfsys@defobject{currentmarker}{\pgfqpoint{0.000000in}{-0.055556in}}{\pgfqpoint{0.000000in}{0.000000in}}{%
\pgfpathmoveto{\pgfqpoint{0.000000in}{0.000000in}}%
\pgfpathlineto{\pgfqpoint{0.000000in}{-0.055556in}}%
\pgfusepath{stroke,fill}%
}%
\begin{pgfscope}%
\pgfsys@transformshift{6.800000in}{5.700000in}%
\pgfsys@useobject{currentmarker}{}%
\end{pgfscope}%
\end{pgfscope}%
\begin{pgfscope}%
\definecolor{textcolor}{rgb}{0.000000,0.000000,0.000000}%
\pgfsetstrokecolor{textcolor}%
\pgfsetfillcolor{textcolor}%
\pgftext[x=6.800000in,y=0.844444in,,top]{\color{textcolor}\sffamily\fontsize{20.000000}{24.000000}\selectfont \(\displaystyle {50}\)}%
\end{pgfscope}%
\begin{pgfscope}%
\definecolor{textcolor}{rgb}{0.000000,0.000000,0.000000}%
\pgfsetstrokecolor{textcolor}%
\pgfsetfillcolor{textcolor}%
\pgftext[x=4.000000in,y=0.518932in,,top]{\color{textcolor}\sffamily\fontsize{20.000000}{24.000000}\selectfont \(\displaystyle t/\mathrm{ns}\)}%
\end{pgfscope}%
\begin{pgfscope}%
\pgfpathrectangle{\pgfqpoint{1.200000in}{0.900000in}}{\pgfqpoint{5.600000in}{4.800000in}}%
\pgfusepath{clip}%
\pgfsetbuttcap%
\pgfsetroundjoin%
\pgfsetlinewidth{0.501875pt}%
\definecolor{currentstroke}{rgb}{0.000000,0.000000,0.000000}%
\pgfsetstrokecolor{currentstroke}%
\pgfsetdash{{1.000000pt}{3.000000pt}}{0.000000pt}%
\pgfpathmoveto{\pgfqpoint{1.200000in}{0.900000in}}%
\pgfpathlineto{\pgfqpoint{6.800000in}{0.900000in}}%
\pgfusepath{stroke}%
\end{pgfscope}%
\begin{pgfscope}%
\pgfsetbuttcap%
\pgfsetroundjoin%
\definecolor{currentfill}{rgb}{0.000000,0.000000,0.000000}%
\pgfsetfillcolor{currentfill}%
\pgfsetlinewidth{0.501875pt}%
\definecolor{currentstroke}{rgb}{0.000000,0.000000,0.000000}%
\pgfsetstrokecolor{currentstroke}%
\pgfsetdash{}{0pt}%
\pgfsys@defobject{currentmarker}{\pgfqpoint{0.000000in}{0.000000in}}{\pgfqpoint{0.055556in}{0.000000in}}{%
\pgfpathmoveto{\pgfqpoint{0.000000in}{0.000000in}}%
\pgfpathlineto{\pgfqpoint{0.055556in}{0.000000in}}%
\pgfusepath{stroke,fill}%
}%
\begin{pgfscope}%
\pgfsys@transformshift{1.200000in}{0.900000in}%
\pgfsys@useobject{currentmarker}{}%
\end{pgfscope}%
\end{pgfscope}%
\begin{pgfscope}%
\pgfsetbuttcap%
\pgfsetroundjoin%
\definecolor{currentfill}{rgb}{0.000000,0.000000,0.000000}%
\pgfsetfillcolor{currentfill}%
\pgfsetlinewidth{0.501875pt}%
\definecolor{currentstroke}{rgb}{0.000000,0.000000,0.000000}%
\pgfsetstrokecolor{currentstroke}%
\pgfsetdash{}{0pt}%
\pgfsys@defobject{currentmarker}{\pgfqpoint{-0.055556in}{0.000000in}}{\pgfqpoint{0.000000in}{0.000000in}}{%
\pgfpathmoveto{\pgfqpoint{0.000000in}{0.000000in}}%
\pgfpathlineto{\pgfqpoint{-0.055556in}{0.000000in}}%
\pgfusepath{stroke,fill}%
}%
\begin{pgfscope}%
\pgfsys@transformshift{6.800000in}{0.900000in}%
\pgfsys@useobject{currentmarker}{}%
\end{pgfscope}%
\end{pgfscope}%
\begin{pgfscope}%
\definecolor{textcolor}{rgb}{0.000000,0.000000,0.000000}%
\pgfsetstrokecolor{textcolor}%
\pgfsetfillcolor{textcolor}%
\pgftext[x=1.144444in,y=0.900000in,right,]{\color{textcolor}\sffamily\fontsize{20.000000}{24.000000}\selectfont \(\displaystyle {0.00}\)}%
\end{pgfscope}%
\begin{pgfscope}%
\pgfpathrectangle{\pgfqpoint{1.200000in}{0.900000in}}{\pgfqpoint{5.600000in}{4.800000in}}%
\pgfusepath{clip}%
\pgfsetbuttcap%
\pgfsetroundjoin%
\pgfsetlinewidth{0.501875pt}%
\definecolor{currentstroke}{rgb}{0.000000,0.000000,0.000000}%
\pgfsetstrokecolor{currentstroke}%
\pgfsetdash{{1.000000pt}{3.000000pt}}{0.000000pt}%
\pgfpathmoveto{\pgfqpoint{1.200000in}{1.585714in}}%
\pgfpathlineto{\pgfqpoint{6.800000in}{1.585714in}}%
\pgfusepath{stroke}%
\end{pgfscope}%
\begin{pgfscope}%
\pgfsetbuttcap%
\pgfsetroundjoin%
\definecolor{currentfill}{rgb}{0.000000,0.000000,0.000000}%
\pgfsetfillcolor{currentfill}%
\pgfsetlinewidth{0.501875pt}%
\definecolor{currentstroke}{rgb}{0.000000,0.000000,0.000000}%
\pgfsetstrokecolor{currentstroke}%
\pgfsetdash{}{0pt}%
\pgfsys@defobject{currentmarker}{\pgfqpoint{0.000000in}{0.000000in}}{\pgfqpoint{0.055556in}{0.000000in}}{%
\pgfpathmoveto{\pgfqpoint{0.000000in}{0.000000in}}%
\pgfpathlineto{\pgfqpoint{0.055556in}{0.000000in}}%
\pgfusepath{stroke,fill}%
}%
\begin{pgfscope}%
\pgfsys@transformshift{1.200000in}{1.585714in}%
\pgfsys@useobject{currentmarker}{}%
\end{pgfscope}%
\end{pgfscope}%
\begin{pgfscope}%
\pgfsetbuttcap%
\pgfsetroundjoin%
\definecolor{currentfill}{rgb}{0.000000,0.000000,0.000000}%
\pgfsetfillcolor{currentfill}%
\pgfsetlinewidth{0.501875pt}%
\definecolor{currentstroke}{rgb}{0.000000,0.000000,0.000000}%
\pgfsetstrokecolor{currentstroke}%
\pgfsetdash{}{0pt}%
\pgfsys@defobject{currentmarker}{\pgfqpoint{-0.055556in}{0.000000in}}{\pgfqpoint{0.000000in}{0.000000in}}{%
\pgfpathmoveto{\pgfqpoint{0.000000in}{0.000000in}}%
\pgfpathlineto{\pgfqpoint{-0.055556in}{0.000000in}}%
\pgfusepath{stroke,fill}%
}%
\begin{pgfscope}%
\pgfsys@transformshift{6.800000in}{1.585714in}%
\pgfsys@useobject{currentmarker}{}%
\end{pgfscope}%
\end{pgfscope}%
\begin{pgfscope}%
\definecolor{textcolor}{rgb}{0.000000,0.000000,0.000000}%
\pgfsetstrokecolor{textcolor}%
\pgfsetfillcolor{textcolor}%
\pgftext[x=1.144444in,y=1.585714in,right,]{\color{textcolor}\sffamily\fontsize{20.000000}{24.000000}\selectfont \(\displaystyle {0.01}\)}%
\end{pgfscope}%
\begin{pgfscope}%
\pgfpathrectangle{\pgfqpoint{1.200000in}{0.900000in}}{\pgfqpoint{5.600000in}{4.800000in}}%
\pgfusepath{clip}%
\pgfsetbuttcap%
\pgfsetroundjoin%
\pgfsetlinewidth{0.501875pt}%
\definecolor{currentstroke}{rgb}{0.000000,0.000000,0.000000}%
\pgfsetstrokecolor{currentstroke}%
\pgfsetdash{{1.000000pt}{3.000000pt}}{0.000000pt}%
\pgfpathmoveto{\pgfqpoint{1.200000in}{2.271429in}}%
\pgfpathlineto{\pgfqpoint{6.800000in}{2.271429in}}%
\pgfusepath{stroke}%
\end{pgfscope}%
\begin{pgfscope}%
\pgfsetbuttcap%
\pgfsetroundjoin%
\definecolor{currentfill}{rgb}{0.000000,0.000000,0.000000}%
\pgfsetfillcolor{currentfill}%
\pgfsetlinewidth{0.501875pt}%
\definecolor{currentstroke}{rgb}{0.000000,0.000000,0.000000}%
\pgfsetstrokecolor{currentstroke}%
\pgfsetdash{}{0pt}%
\pgfsys@defobject{currentmarker}{\pgfqpoint{0.000000in}{0.000000in}}{\pgfqpoint{0.055556in}{0.000000in}}{%
\pgfpathmoveto{\pgfqpoint{0.000000in}{0.000000in}}%
\pgfpathlineto{\pgfqpoint{0.055556in}{0.000000in}}%
\pgfusepath{stroke,fill}%
}%
\begin{pgfscope}%
\pgfsys@transformshift{1.200000in}{2.271429in}%
\pgfsys@useobject{currentmarker}{}%
\end{pgfscope}%
\end{pgfscope}%
\begin{pgfscope}%
\pgfsetbuttcap%
\pgfsetroundjoin%
\definecolor{currentfill}{rgb}{0.000000,0.000000,0.000000}%
\pgfsetfillcolor{currentfill}%
\pgfsetlinewidth{0.501875pt}%
\definecolor{currentstroke}{rgb}{0.000000,0.000000,0.000000}%
\pgfsetstrokecolor{currentstroke}%
\pgfsetdash{}{0pt}%
\pgfsys@defobject{currentmarker}{\pgfqpoint{-0.055556in}{0.000000in}}{\pgfqpoint{0.000000in}{0.000000in}}{%
\pgfpathmoveto{\pgfqpoint{0.000000in}{0.000000in}}%
\pgfpathlineto{\pgfqpoint{-0.055556in}{0.000000in}}%
\pgfusepath{stroke,fill}%
}%
\begin{pgfscope}%
\pgfsys@transformshift{6.800000in}{2.271429in}%
\pgfsys@useobject{currentmarker}{}%
\end{pgfscope}%
\end{pgfscope}%
\begin{pgfscope}%
\definecolor{textcolor}{rgb}{0.000000,0.000000,0.000000}%
\pgfsetstrokecolor{textcolor}%
\pgfsetfillcolor{textcolor}%
\pgftext[x=1.144444in,y=2.271429in,right,]{\color{textcolor}\sffamily\fontsize{20.000000}{24.000000}\selectfont \(\displaystyle {0.02}\)}%
\end{pgfscope}%
\begin{pgfscope}%
\pgfpathrectangle{\pgfqpoint{1.200000in}{0.900000in}}{\pgfqpoint{5.600000in}{4.800000in}}%
\pgfusepath{clip}%
\pgfsetbuttcap%
\pgfsetroundjoin%
\pgfsetlinewidth{0.501875pt}%
\definecolor{currentstroke}{rgb}{0.000000,0.000000,0.000000}%
\pgfsetstrokecolor{currentstroke}%
\pgfsetdash{{1.000000pt}{3.000000pt}}{0.000000pt}%
\pgfpathmoveto{\pgfqpoint{1.200000in}{2.957143in}}%
\pgfpathlineto{\pgfqpoint{6.800000in}{2.957143in}}%
\pgfusepath{stroke}%
\end{pgfscope}%
\begin{pgfscope}%
\pgfsetbuttcap%
\pgfsetroundjoin%
\definecolor{currentfill}{rgb}{0.000000,0.000000,0.000000}%
\pgfsetfillcolor{currentfill}%
\pgfsetlinewidth{0.501875pt}%
\definecolor{currentstroke}{rgb}{0.000000,0.000000,0.000000}%
\pgfsetstrokecolor{currentstroke}%
\pgfsetdash{}{0pt}%
\pgfsys@defobject{currentmarker}{\pgfqpoint{0.000000in}{0.000000in}}{\pgfqpoint{0.055556in}{0.000000in}}{%
\pgfpathmoveto{\pgfqpoint{0.000000in}{0.000000in}}%
\pgfpathlineto{\pgfqpoint{0.055556in}{0.000000in}}%
\pgfusepath{stroke,fill}%
}%
\begin{pgfscope}%
\pgfsys@transformshift{1.200000in}{2.957143in}%
\pgfsys@useobject{currentmarker}{}%
\end{pgfscope}%
\end{pgfscope}%
\begin{pgfscope}%
\pgfsetbuttcap%
\pgfsetroundjoin%
\definecolor{currentfill}{rgb}{0.000000,0.000000,0.000000}%
\pgfsetfillcolor{currentfill}%
\pgfsetlinewidth{0.501875pt}%
\definecolor{currentstroke}{rgb}{0.000000,0.000000,0.000000}%
\pgfsetstrokecolor{currentstroke}%
\pgfsetdash{}{0pt}%
\pgfsys@defobject{currentmarker}{\pgfqpoint{-0.055556in}{0.000000in}}{\pgfqpoint{0.000000in}{0.000000in}}{%
\pgfpathmoveto{\pgfqpoint{0.000000in}{0.000000in}}%
\pgfpathlineto{\pgfqpoint{-0.055556in}{0.000000in}}%
\pgfusepath{stroke,fill}%
}%
\begin{pgfscope}%
\pgfsys@transformshift{6.800000in}{2.957143in}%
\pgfsys@useobject{currentmarker}{}%
\end{pgfscope}%
\end{pgfscope}%
\begin{pgfscope}%
\definecolor{textcolor}{rgb}{0.000000,0.000000,0.000000}%
\pgfsetstrokecolor{textcolor}%
\pgfsetfillcolor{textcolor}%
\pgftext[x=1.144444in,y=2.957143in,right,]{\color{textcolor}\sffamily\fontsize{20.000000}{24.000000}\selectfont \(\displaystyle {0.03}\)}%
\end{pgfscope}%
\begin{pgfscope}%
\pgfpathrectangle{\pgfqpoint{1.200000in}{0.900000in}}{\pgfqpoint{5.600000in}{4.800000in}}%
\pgfusepath{clip}%
\pgfsetbuttcap%
\pgfsetroundjoin%
\pgfsetlinewidth{0.501875pt}%
\definecolor{currentstroke}{rgb}{0.000000,0.000000,0.000000}%
\pgfsetstrokecolor{currentstroke}%
\pgfsetdash{{1.000000pt}{3.000000pt}}{0.000000pt}%
\pgfpathmoveto{\pgfqpoint{1.200000in}{3.642857in}}%
\pgfpathlineto{\pgfqpoint{6.800000in}{3.642857in}}%
\pgfusepath{stroke}%
\end{pgfscope}%
\begin{pgfscope}%
\pgfsetbuttcap%
\pgfsetroundjoin%
\definecolor{currentfill}{rgb}{0.000000,0.000000,0.000000}%
\pgfsetfillcolor{currentfill}%
\pgfsetlinewidth{0.501875pt}%
\definecolor{currentstroke}{rgb}{0.000000,0.000000,0.000000}%
\pgfsetstrokecolor{currentstroke}%
\pgfsetdash{}{0pt}%
\pgfsys@defobject{currentmarker}{\pgfqpoint{0.000000in}{0.000000in}}{\pgfqpoint{0.055556in}{0.000000in}}{%
\pgfpathmoveto{\pgfqpoint{0.000000in}{0.000000in}}%
\pgfpathlineto{\pgfqpoint{0.055556in}{0.000000in}}%
\pgfusepath{stroke,fill}%
}%
\begin{pgfscope}%
\pgfsys@transformshift{1.200000in}{3.642857in}%
\pgfsys@useobject{currentmarker}{}%
\end{pgfscope}%
\end{pgfscope}%
\begin{pgfscope}%
\pgfsetbuttcap%
\pgfsetroundjoin%
\definecolor{currentfill}{rgb}{0.000000,0.000000,0.000000}%
\pgfsetfillcolor{currentfill}%
\pgfsetlinewidth{0.501875pt}%
\definecolor{currentstroke}{rgb}{0.000000,0.000000,0.000000}%
\pgfsetstrokecolor{currentstroke}%
\pgfsetdash{}{0pt}%
\pgfsys@defobject{currentmarker}{\pgfqpoint{-0.055556in}{0.000000in}}{\pgfqpoint{0.000000in}{0.000000in}}{%
\pgfpathmoveto{\pgfqpoint{0.000000in}{0.000000in}}%
\pgfpathlineto{\pgfqpoint{-0.055556in}{0.000000in}}%
\pgfusepath{stroke,fill}%
}%
\begin{pgfscope}%
\pgfsys@transformshift{6.800000in}{3.642857in}%
\pgfsys@useobject{currentmarker}{}%
\end{pgfscope}%
\end{pgfscope}%
\begin{pgfscope}%
\definecolor{textcolor}{rgb}{0.000000,0.000000,0.000000}%
\pgfsetstrokecolor{textcolor}%
\pgfsetfillcolor{textcolor}%
\pgftext[x=1.144444in,y=3.642857in,right,]{\color{textcolor}\sffamily\fontsize{20.000000}{24.000000}\selectfont \(\displaystyle {0.04}\)}%
\end{pgfscope}%
\begin{pgfscope}%
\pgfpathrectangle{\pgfqpoint{1.200000in}{0.900000in}}{\pgfqpoint{5.600000in}{4.800000in}}%
\pgfusepath{clip}%
\pgfsetbuttcap%
\pgfsetroundjoin%
\pgfsetlinewidth{0.501875pt}%
\definecolor{currentstroke}{rgb}{0.000000,0.000000,0.000000}%
\pgfsetstrokecolor{currentstroke}%
\pgfsetdash{{1.000000pt}{3.000000pt}}{0.000000pt}%
\pgfpathmoveto{\pgfqpoint{1.200000in}{4.328571in}}%
\pgfpathlineto{\pgfqpoint{6.800000in}{4.328571in}}%
\pgfusepath{stroke}%
\end{pgfscope}%
\begin{pgfscope}%
\pgfsetbuttcap%
\pgfsetroundjoin%
\definecolor{currentfill}{rgb}{0.000000,0.000000,0.000000}%
\pgfsetfillcolor{currentfill}%
\pgfsetlinewidth{0.501875pt}%
\definecolor{currentstroke}{rgb}{0.000000,0.000000,0.000000}%
\pgfsetstrokecolor{currentstroke}%
\pgfsetdash{}{0pt}%
\pgfsys@defobject{currentmarker}{\pgfqpoint{0.000000in}{0.000000in}}{\pgfqpoint{0.055556in}{0.000000in}}{%
\pgfpathmoveto{\pgfqpoint{0.000000in}{0.000000in}}%
\pgfpathlineto{\pgfqpoint{0.055556in}{0.000000in}}%
\pgfusepath{stroke,fill}%
}%
\begin{pgfscope}%
\pgfsys@transformshift{1.200000in}{4.328571in}%
\pgfsys@useobject{currentmarker}{}%
\end{pgfscope}%
\end{pgfscope}%
\begin{pgfscope}%
\pgfsetbuttcap%
\pgfsetroundjoin%
\definecolor{currentfill}{rgb}{0.000000,0.000000,0.000000}%
\pgfsetfillcolor{currentfill}%
\pgfsetlinewidth{0.501875pt}%
\definecolor{currentstroke}{rgb}{0.000000,0.000000,0.000000}%
\pgfsetstrokecolor{currentstroke}%
\pgfsetdash{}{0pt}%
\pgfsys@defobject{currentmarker}{\pgfqpoint{-0.055556in}{0.000000in}}{\pgfqpoint{0.000000in}{0.000000in}}{%
\pgfpathmoveto{\pgfqpoint{0.000000in}{0.000000in}}%
\pgfpathlineto{\pgfqpoint{-0.055556in}{0.000000in}}%
\pgfusepath{stroke,fill}%
}%
\begin{pgfscope}%
\pgfsys@transformshift{6.800000in}{4.328571in}%
\pgfsys@useobject{currentmarker}{}%
\end{pgfscope}%
\end{pgfscope}%
\begin{pgfscope}%
\definecolor{textcolor}{rgb}{0.000000,0.000000,0.000000}%
\pgfsetstrokecolor{textcolor}%
\pgfsetfillcolor{textcolor}%
\pgftext[x=1.144444in,y=4.328571in,right,]{\color{textcolor}\sffamily\fontsize{20.000000}{24.000000}\selectfont \(\displaystyle {0.05}\)}%
\end{pgfscope}%
\begin{pgfscope}%
\pgfpathrectangle{\pgfqpoint{1.200000in}{0.900000in}}{\pgfqpoint{5.600000in}{4.800000in}}%
\pgfusepath{clip}%
\pgfsetbuttcap%
\pgfsetroundjoin%
\pgfsetlinewidth{0.501875pt}%
\definecolor{currentstroke}{rgb}{0.000000,0.000000,0.000000}%
\pgfsetstrokecolor{currentstroke}%
\pgfsetdash{{1.000000pt}{3.000000pt}}{0.000000pt}%
\pgfpathmoveto{\pgfqpoint{1.200000in}{5.014286in}}%
\pgfpathlineto{\pgfqpoint{6.800000in}{5.014286in}}%
\pgfusepath{stroke}%
\end{pgfscope}%
\begin{pgfscope}%
\pgfsetbuttcap%
\pgfsetroundjoin%
\definecolor{currentfill}{rgb}{0.000000,0.000000,0.000000}%
\pgfsetfillcolor{currentfill}%
\pgfsetlinewidth{0.501875pt}%
\definecolor{currentstroke}{rgb}{0.000000,0.000000,0.000000}%
\pgfsetstrokecolor{currentstroke}%
\pgfsetdash{}{0pt}%
\pgfsys@defobject{currentmarker}{\pgfqpoint{0.000000in}{0.000000in}}{\pgfqpoint{0.055556in}{0.000000in}}{%
\pgfpathmoveto{\pgfqpoint{0.000000in}{0.000000in}}%
\pgfpathlineto{\pgfqpoint{0.055556in}{0.000000in}}%
\pgfusepath{stroke,fill}%
}%
\begin{pgfscope}%
\pgfsys@transformshift{1.200000in}{5.014286in}%
\pgfsys@useobject{currentmarker}{}%
\end{pgfscope}%
\end{pgfscope}%
\begin{pgfscope}%
\pgfsetbuttcap%
\pgfsetroundjoin%
\definecolor{currentfill}{rgb}{0.000000,0.000000,0.000000}%
\pgfsetfillcolor{currentfill}%
\pgfsetlinewidth{0.501875pt}%
\definecolor{currentstroke}{rgb}{0.000000,0.000000,0.000000}%
\pgfsetstrokecolor{currentstroke}%
\pgfsetdash{}{0pt}%
\pgfsys@defobject{currentmarker}{\pgfqpoint{-0.055556in}{0.000000in}}{\pgfqpoint{0.000000in}{0.000000in}}{%
\pgfpathmoveto{\pgfqpoint{0.000000in}{0.000000in}}%
\pgfpathlineto{\pgfqpoint{-0.055556in}{0.000000in}}%
\pgfusepath{stroke,fill}%
}%
\begin{pgfscope}%
\pgfsys@transformshift{6.800000in}{5.014286in}%
\pgfsys@useobject{currentmarker}{}%
\end{pgfscope}%
\end{pgfscope}%
\begin{pgfscope}%
\definecolor{textcolor}{rgb}{0.000000,0.000000,0.000000}%
\pgfsetstrokecolor{textcolor}%
\pgfsetfillcolor{textcolor}%
\pgftext[x=1.144444in,y=5.014286in,right,]{\color{textcolor}\sffamily\fontsize{20.000000}{24.000000}\selectfont \(\displaystyle {0.06}\)}%
\end{pgfscope}%
\begin{pgfscope}%
\pgfpathrectangle{\pgfqpoint{1.200000in}{0.900000in}}{\pgfqpoint{5.600000in}{4.800000in}}%
\pgfusepath{clip}%
\pgfsetbuttcap%
\pgfsetroundjoin%
\pgfsetlinewidth{0.501875pt}%
\definecolor{currentstroke}{rgb}{0.000000,0.000000,0.000000}%
\pgfsetstrokecolor{currentstroke}%
\pgfsetdash{{1.000000pt}{3.000000pt}}{0.000000pt}%
\pgfpathmoveto{\pgfqpoint{1.200000in}{5.700000in}}%
\pgfpathlineto{\pgfqpoint{6.800000in}{5.700000in}}%
\pgfusepath{stroke}%
\end{pgfscope}%
\begin{pgfscope}%
\pgfsetbuttcap%
\pgfsetroundjoin%
\definecolor{currentfill}{rgb}{0.000000,0.000000,0.000000}%
\pgfsetfillcolor{currentfill}%
\pgfsetlinewidth{0.501875pt}%
\definecolor{currentstroke}{rgb}{0.000000,0.000000,0.000000}%
\pgfsetstrokecolor{currentstroke}%
\pgfsetdash{}{0pt}%
\pgfsys@defobject{currentmarker}{\pgfqpoint{0.000000in}{0.000000in}}{\pgfqpoint{0.055556in}{0.000000in}}{%
\pgfpathmoveto{\pgfqpoint{0.000000in}{0.000000in}}%
\pgfpathlineto{\pgfqpoint{0.055556in}{0.000000in}}%
\pgfusepath{stroke,fill}%
}%
\begin{pgfscope}%
\pgfsys@transformshift{1.200000in}{5.700000in}%
\pgfsys@useobject{currentmarker}{}%
\end{pgfscope}%
\end{pgfscope}%
\begin{pgfscope}%
\pgfsetbuttcap%
\pgfsetroundjoin%
\definecolor{currentfill}{rgb}{0.000000,0.000000,0.000000}%
\pgfsetfillcolor{currentfill}%
\pgfsetlinewidth{0.501875pt}%
\definecolor{currentstroke}{rgb}{0.000000,0.000000,0.000000}%
\pgfsetstrokecolor{currentstroke}%
\pgfsetdash{}{0pt}%
\pgfsys@defobject{currentmarker}{\pgfqpoint{-0.055556in}{0.000000in}}{\pgfqpoint{0.000000in}{0.000000in}}{%
\pgfpathmoveto{\pgfqpoint{0.000000in}{0.000000in}}%
\pgfpathlineto{\pgfqpoint{-0.055556in}{0.000000in}}%
\pgfusepath{stroke,fill}%
}%
\begin{pgfscope}%
\pgfsys@transformshift{6.800000in}{5.700000in}%
\pgfsys@useobject{currentmarker}{}%
\end{pgfscope}%
\end{pgfscope}%
\begin{pgfscope}%
\definecolor{textcolor}{rgb}{0.000000,0.000000,0.000000}%
\pgfsetstrokecolor{textcolor}%
\pgfsetfillcolor{textcolor}%
\pgftext[x=1.144444in,y=5.700000in,right,]{\color{textcolor}\sffamily\fontsize{20.000000}{24.000000}\selectfont \(\displaystyle {0.07}\)}%
\end{pgfscope}%
\begin{pgfscope}%
\definecolor{textcolor}{rgb}{0.000000,0.000000,0.000000}%
\pgfsetstrokecolor{textcolor}%
\pgfsetfillcolor{textcolor}%
\pgftext[x=0.600330in,y=3.300000in,,bottom,rotate=90.000000]{\color{textcolor}\sffamily\fontsize{20.000000}{24.000000}\selectfont \(\displaystyle PDF\)}%
\end{pgfscope}%
\begin{pgfscope}%
\pgfsetbuttcap%
\pgfsetmiterjoin%
\definecolor{currentfill}{rgb}{1.000000,1.000000,1.000000}%
\pgfsetfillcolor{currentfill}%
\pgfsetlinewidth{1.003750pt}%
\definecolor{currentstroke}{rgb}{0.000000,0.000000,0.000000}%
\pgfsetstrokecolor{currentstroke}%
\pgfsetdash{}{0pt}%
\pgfpathmoveto{\pgfqpoint{3.749258in}{4.959484in}}%
\pgfpathlineto{\pgfqpoint{6.633333in}{4.959484in}}%
\pgfpathlineto{\pgfqpoint{6.633333in}{5.533333in}}%
\pgfpathlineto{\pgfqpoint{3.749258in}{5.533333in}}%
\pgfpathclose%
\pgfusepath{stroke,fill}%
\end{pgfscope}%
\begin{pgfscope}%
\pgfsetrectcap%
\pgfsetroundjoin%
\pgfsetlinewidth{2.007500pt}%
\definecolor{currentstroke}{rgb}{0.000000,0.000000,1.000000}%
\pgfsetstrokecolor{currentstroke}%
\pgfsetdash{}{0pt}%
\pgfpathmoveto{\pgfqpoint{3.982591in}{5.276697in}}%
\pgfpathlineto{\pgfqpoint{4.449258in}{5.276697in}}%
\pgfusepath{stroke}%
\end{pgfscope}%
\begin{pgfscope}%
\definecolor{textcolor}{rgb}{0.000000,0.000000,0.000000}%
\pgfsetstrokecolor{textcolor}%
\pgfsetfillcolor{textcolor}%
\pgftext[x=4.815924in,y=5.160031in,left,base]{\color{textcolor}\sffamily\fontsize{24.000000}{28.800000}\selectfont Time Profile}%
\end{pgfscope}%
\end{pgfpicture}%
\makeatother%
\endgroup%
}
    \caption{\label{fig:time-pro} Time profile, $\tau$=20ns $\sigma$=10ns}
\end{figure}
\end{minipage}
% \end{figure}

Here we discuss the configuration and process of simulation in toy MC. The total number of PE in a DAQ window (in a single waveform), $N_{pe}$, is sampled from a Truncated Poisson distribution ($N_{pe}=0$ samples are abandoned) with expectation $\mu$. The hittime $t_{tru}$ of each PE is sampled from the time profile $P_{t}(t)$ of the liquid scintillator. The charge of each PE $q_{tru}$ is sampled from parameterized calibration result, which can be described as a truncated normal distribution with probability distribution function \eqref{eq:truncated}. The simulation datasets are combination of hittime $t_{tru}$ and charge of PE, $q_{tru}$. 

\begin{equation}
    P_{q}(q) = A\cdot\mathcal{N}(\mu_{q},\sigma_{q}^{2}),\,(q>0)
    \label{eq:truncated}
\end{equation}

The single photonelectron response is described as the formula \eqref{eq:dayaspe} \cite{jetter_pmt_2012}. An gaussian noise is added to waveform after sampling $t_{tru}$ and $q_{tru}$. 

The typical combination of decay time ($\tau$) and transit time spread ($\sigma$) are listed in figure (see figure~\ref{fig:deltamethods}). We take the value of $\tau$ according to typical liquid scintillator used in JUNO \cite{ludhova_particle_2020}. For every combination of PE number expectation $\mu$, decay time $\tau$ and TTS $\sigma$, a sample set (size $N=10,000$) is generated to estimate the timing resolution $\delta$. 

\begin{equation}
    V(t) = V_{0}\exp\left(-\frac{1}{2}\left(\frac{\ln(t/\tau)}{\sigma}\right)^{2}\right)
    \label{eq:dayaspe}
\end{equation}

We can use Maximum Likelihood Estimation (MLE) to estimate $t_{0}$ in formula \eqref{eq:time-pro}, which is the starting time of time profile, using charge $q$ and corresponding hittime. The difference between real starting time $t_{0tru}$ and reconstructed starting time $t_{0rec}$ are collected to produce the starting time residual ($\Delta t_{0}=t_{0rec}-t_{0tru}$) distribution. The time resolution $\delta$ is the standard deviation (STD) of the starting time residual distribution. The likelihood $\mathcal{L}$ defined in MLE is formula \eqref{eq:likelihood}. 

\begin{align}
    \mathcal{L}(t_{0}) &= \Pi_{i=0}^{N_{pe}}\left(\sum_{j=a}^{b}[P_{i}(q_{i},j)(P_{t}(t_{i}))^{j}]\right)
    \label{eq:likelihood} \\
    P_{i}(q_{i},j) &= \frac{P_{q}(q_{i}|j)}{\sum_{j=a}^{b}P_{q}(q_{i}|j)}
\end{align}

For reconstructed hittime and charge produced by waveform analysis algorithms, each charge $q$ is registered to a single hittime $t$, but the number of PE at a single hittime is uncertain. $P_{i}(q_{i},j)$ is the posterior probability if the number of PE is $j$ with charge $q_{i}$. $a$ and $b$ are the upper and lower bound estimation of the number of PE $j$. 

The estimation result of $t_{0}$ using only simulated data is shown in figure~\ref{fig:reso-diff}, where $\delta_{1st}$ and $\delta_{all}$ are timing resolution using the first PE and all PE when reconstruction, respectively. The vertical axis is ratio, $\delta_{all}/\delta_{1st}$. Because time resolution estimation here is only based on simulated data, the result will reveal the advantage of using detailed information of all PE. It is distinct that the timing resolution is smaller when reconstructing rise time using all PE. 

\begin{figure}[H]
    \centering
    \scalebox{0.5}{%% Creator: Matplotlib, PGF backend
%%
%% To include the figure in your LaTeX document, write
%%   \input{<filename>.pgf}
%%
%% Make sure the required packages are loaded in your preamble
%%   \usepackage{pgf}
%%
%% and, on pdftex
%%   \usepackage[utf8]{inputenc}\DeclareUnicodeCharacter{2212}{-}
%%
%% or, on luatex and xetex
%%   \usepackage{unicode-math}
%%
%% Figures using additional raster images can only be included by \input if
%% they are in the same directory as the main LaTeX file. For loading figures
%% from other directories you can use the `import` package
%%   \usepackage{import}
%%
%% and then include the figures with
%%   \import{<path to file>}{<filename>.pgf}
%%
%% Matplotlib used the following preamble
%%   \usepackage{fontspec}
%%   \setmainfont{DejaVuSerif.ttf}[Path=/home/xdc/.local/lib/python3.8/site-packages/matplotlib/mpl-data/fonts/ttf/]
%%   \setsansfont{DejaVuSans.ttf}[Path=/home/xdc/.local/lib/python3.8/site-packages/matplotlib/mpl-data/fonts/ttf/]
%%   \setmonofont{DejaVuSansMono.ttf}[Path=/home/xdc/.local/lib/python3.8/site-packages/matplotlib/mpl-data/fonts/ttf/]
%%
\begingroup%
\makeatletter%
\begin{pgfpicture}%
\pgfpathrectangle{\pgfpointorigin}{\pgfqpoint{6.000000in}{6.000000in}}%
\pgfusepath{use as bounding box, clip}%
\begin{pgfscope}%
\pgfsetbuttcap%
\pgfsetmiterjoin%
\definecolor{currentfill}{rgb}{1.000000,1.000000,1.000000}%
\pgfsetfillcolor{currentfill}%
\pgfsetlinewidth{0.000000pt}%
\definecolor{currentstroke}{rgb}{1.000000,1.000000,1.000000}%
\pgfsetstrokecolor{currentstroke}%
\pgfsetdash{}{0pt}%
\pgfpathmoveto{\pgfqpoint{0.000000in}{0.000000in}}%
\pgfpathlineto{\pgfqpoint{6.000000in}{0.000000in}}%
\pgfpathlineto{\pgfqpoint{6.000000in}{6.000000in}}%
\pgfpathlineto{\pgfqpoint{0.000000in}{6.000000in}}%
\pgfpathclose%
\pgfusepath{fill}%
\end{pgfscope}%
\begin{pgfscope}%
\pgfsetbuttcap%
\pgfsetmiterjoin%
\definecolor{currentfill}{rgb}{1.000000,1.000000,1.000000}%
\pgfsetfillcolor{currentfill}%
\pgfsetlinewidth{0.000000pt}%
\definecolor{currentstroke}{rgb}{0.000000,0.000000,0.000000}%
\pgfsetstrokecolor{currentstroke}%
\pgfsetstrokeopacity{0.000000}%
\pgfsetdash{}{0pt}%
\pgfpathmoveto{\pgfqpoint{0.750000in}{0.660000in}}%
\pgfpathlineto{\pgfqpoint{5.400000in}{0.660000in}}%
\pgfpathlineto{\pgfqpoint{5.400000in}{5.280000in}}%
\pgfpathlineto{\pgfqpoint{0.750000in}{5.280000in}}%
\pgfpathclose%
\pgfusepath{fill}%
\end{pgfscope}%
\begin{pgfscope}%
\pgfpathrectangle{\pgfqpoint{0.750000in}{0.660000in}}{\pgfqpoint{4.650000in}{4.620000in}}%
\pgfusepath{clip}%
\pgfsetrectcap%
\pgfsetroundjoin%
\pgfsetlinewidth{0.803000pt}%
\definecolor{currentstroke}{rgb}{0.690196,0.690196,0.690196}%
\pgfsetstrokecolor{currentstroke}%
\pgfsetdash{}{0pt}%
\pgfpathmoveto{\pgfqpoint{0.815596in}{0.660000in}}%
\pgfpathlineto{\pgfqpoint{0.815596in}{5.280000in}}%
\pgfusepath{stroke}%
\end{pgfscope}%
\begin{pgfscope}%
\pgfsetbuttcap%
\pgfsetroundjoin%
\definecolor{currentfill}{rgb}{0.000000,0.000000,0.000000}%
\pgfsetfillcolor{currentfill}%
\pgfsetlinewidth{0.803000pt}%
\definecolor{currentstroke}{rgb}{0.000000,0.000000,0.000000}%
\pgfsetstrokecolor{currentstroke}%
\pgfsetdash{}{0pt}%
\pgfsys@defobject{currentmarker}{\pgfqpoint{0.000000in}{-0.048611in}}{\pgfqpoint{0.000000in}{0.000000in}}{%
\pgfpathmoveto{\pgfqpoint{0.000000in}{0.000000in}}%
\pgfpathlineto{\pgfqpoint{0.000000in}{-0.048611in}}%
\pgfusepath{stroke,fill}%
}%
\begin{pgfscope}%
\pgfsys@transformshift{0.815596in}{0.660000in}%
\pgfsys@useobject{currentmarker}{}%
\end{pgfscope}%
\end{pgfscope}%
\begin{pgfscope}%
\definecolor{textcolor}{rgb}{0.000000,0.000000,0.000000}%
\pgfsetstrokecolor{textcolor}%
\pgfsetfillcolor{textcolor}%
\pgftext[x=0.815596in,y=0.562778in,,top]{\color{textcolor}\sffamily\fontsize{15.000000}{18.000000}\selectfont 0}%
\end{pgfscope}%
\begin{pgfscope}%
\pgfpathrectangle{\pgfqpoint{0.750000in}{0.660000in}}{\pgfqpoint{4.650000in}{4.620000in}}%
\pgfusepath{clip}%
\pgfsetrectcap%
\pgfsetroundjoin%
\pgfsetlinewidth{0.803000pt}%
\definecolor{currentstroke}{rgb}{0.690196,0.690196,0.690196}%
\pgfsetstrokecolor{currentstroke}%
\pgfsetdash{}{0pt}%
\pgfpathmoveto{\pgfqpoint{1.544436in}{0.660000in}}%
\pgfpathlineto{\pgfqpoint{1.544436in}{5.280000in}}%
\pgfusepath{stroke}%
\end{pgfscope}%
\begin{pgfscope}%
\pgfsetbuttcap%
\pgfsetroundjoin%
\definecolor{currentfill}{rgb}{0.000000,0.000000,0.000000}%
\pgfsetfillcolor{currentfill}%
\pgfsetlinewidth{0.803000pt}%
\definecolor{currentstroke}{rgb}{0.000000,0.000000,0.000000}%
\pgfsetstrokecolor{currentstroke}%
\pgfsetdash{}{0pt}%
\pgfsys@defobject{currentmarker}{\pgfqpoint{0.000000in}{-0.048611in}}{\pgfqpoint{0.000000in}{0.000000in}}{%
\pgfpathmoveto{\pgfqpoint{0.000000in}{0.000000in}}%
\pgfpathlineto{\pgfqpoint{0.000000in}{-0.048611in}}%
\pgfusepath{stroke,fill}%
}%
\begin{pgfscope}%
\pgfsys@transformshift{1.544436in}{0.660000in}%
\pgfsys@useobject{currentmarker}{}%
\end{pgfscope}%
\end{pgfscope}%
\begin{pgfscope}%
\definecolor{textcolor}{rgb}{0.000000,0.000000,0.000000}%
\pgfsetstrokecolor{textcolor}%
\pgfsetfillcolor{textcolor}%
\pgftext[x=1.544436in,y=0.562778in,,top]{\color{textcolor}\sffamily\fontsize{15.000000}{18.000000}\selectfont 5}%
\end{pgfscope}%
\begin{pgfscope}%
\pgfpathrectangle{\pgfqpoint{0.750000in}{0.660000in}}{\pgfqpoint{4.650000in}{4.620000in}}%
\pgfusepath{clip}%
\pgfsetrectcap%
\pgfsetroundjoin%
\pgfsetlinewidth{0.803000pt}%
\definecolor{currentstroke}{rgb}{0.690196,0.690196,0.690196}%
\pgfsetstrokecolor{currentstroke}%
\pgfsetdash{}{0pt}%
\pgfpathmoveto{\pgfqpoint{2.273276in}{0.660000in}}%
\pgfpathlineto{\pgfqpoint{2.273276in}{5.280000in}}%
\pgfusepath{stroke}%
\end{pgfscope}%
\begin{pgfscope}%
\pgfsetbuttcap%
\pgfsetroundjoin%
\definecolor{currentfill}{rgb}{0.000000,0.000000,0.000000}%
\pgfsetfillcolor{currentfill}%
\pgfsetlinewidth{0.803000pt}%
\definecolor{currentstroke}{rgb}{0.000000,0.000000,0.000000}%
\pgfsetstrokecolor{currentstroke}%
\pgfsetdash{}{0pt}%
\pgfsys@defobject{currentmarker}{\pgfqpoint{0.000000in}{-0.048611in}}{\pgfqpoint{0.000000in}{0.000000in}}{%
\pgfpathmoveto{\pgfqpoint{0.000000in}{0.000000in}}%
\pgfpathlineto{\pgfqpoint{0.000000in}{-0.048611in}}%
\pgfusepath{stroke,fill}%
}%
\begin{pgfscope}%
\pgfsys@transformshift{2.273276in}{0.660000in}%
\pgfsys@useobject{currentmarker}{}%
\end{pgfscope}%
\end{pgfscope}%
\begin{pgfscope}%
\definecolor{textcolor}{rgb}{0.000000,0.000000,0.000000}%
\pgfsetstrokecolor{textcolor}%
\pgfsetfillcolor{textcolor}%
\pgftext[x=2.273276in,y=0.562778in,,top]{\color{textcolor}\sffamily\fontsize{15.000000}{18.000000}\selectfont 10}%
\end{pgfscope}%
\begin{pgfscope}%
\pgfpathrectangle{\pgfqpoint{0.750000in}{0.660000in}}{\pgfqpoint{4.650000in}{4.620000in}}%
\pgfusepath{clip}%
\pgfsetrectcap%
\pgfsetroundjoin%
\pgfsetlinewidth{0.803000pt}%
\definecolor{currentstroke}{rgb}{0.690196,0.690196,0.690196}%
\pgfsetstrokecolor{currentstroke}%
\pgfsetdash{}{0pt}%
\pgfpathmoveto{\pgfqpoint{3.002116in}{0.660000in}}%
\pgfpathlineto{\pgfqpoint{3.002116in}{5.280000in}}%
\pgfusepath{stroke}%
\end{pgfscope}%
\begin{pgfscope}%
\pgfsetbuttcap%
\pgfsetroundjoin%
\definecolor{currentfill}{rgb}{0.000000,0.000000,0.000000}%
\pgfsetfillcolor{currentfill}%
\pgfsetlinewidth{0.803000pt}%
\definecolor{currentstroke}{rgb}{0.000000,0.000000,0.000000}%
\pgfsetstrokecolor{currentstroke}%
\pgfsetdash{}{0pt}%
\pgfsys@defobject{currentmarker}{\pgfqpoint{0.000000in}{-0.048611in}}{\pgfqpoint{0.000000in}{0.000000in}}{%
\pgfpathmoveto{\pgfqpoint{0.000000in}{0.000000in}}%
\pgfpathlineto{\pgfqpoint{0.000000in}{-0.048611in}}%
\pgfusepath{stroke,fill}%
}%
\begin{pgfscope}%
\pgfsys@transformshift{3.002116in}{0.660000in}%
\pgfsys@useobject{currentmarker}{}%
\end{pgfscope}%
\end{pgfscope}%
\begin{pgfscope}%
\definecolor{textcolor}{rgb}{0.000000,0.000000,0.000000}%
\pgfsetstrokecolor{textcolor}%
\pgfsetfillcolor{textcolor}%
\pgftext[x=3.002116in,y=0.562778in,,top]{\color{textcolor}\sffamily\fontsize{15.000000}{18.000000}\selectfont 15}%
\end{pgfscope}%
\begin{pgfscope}%
\pgfpathrectangle{\pgfqpoint{0.750000in}{0.660000in}}{\pgfqpoint{4.650000in}{4.620000in}}%
\pgfusepath{clip}%
\pgfsetrectcap%
\pgfsetroundjoin%
\pgfsetlinewidth{0.803000pt}%
\definecolor{currentstroke}{rgb}{0.690196,0.690196,0.690196}%
\pgfsetstrokecolor{currentstroke}%
\pgfsetdash{}{0pt}%
\pgfpathmoveto{\pgfqpoint{3.730956in}{0.660000in}}%
\pgfpathlineto{\pgfqpoint{3.730956in}{5.280000in}}%
\pgfusepath{stroke}%
\end{pgfscope}%
\begin{pgfscope}%
\pgfsetbuttcap%
\pgfsetroundjoin%
\definecolor{currentfill}{rgb}{0.000000,0.000000,0.000000}%
\pgfsetfillcolor{currentfill}%
\pgfsetlinewidth{0.803000pt}%
\definecolor{currentstroke}{rgb}{0.000000,0.000000,0.000000}%
\pgfsetstrokecolor{currentstroke}%
\pgfsetdash{}{0pt}%
\pgfsys@defobject{currentmarker}{\pgfqpoint{0.000000in}{-0.048611in}}{\pgfqpoint{0.000000in}{0.000000in}}{%
\pgfpathmoveto{\pgfqpoint{0.000000in}{0.000000in}}%
\pgfpathlineto{\pgfqpoint{0.000000in}{-0.048611in}}%
\pgfusepath{stroke,fill}%
}%
\begin{pgfscope}%
\pgfsys@transformshift{3.730956in}{0.660000in}%
\pgfsys@useobject{currentmarker}{}%
\end{pgfscope}%
\end{pgfscope}%
\begin{pgfscope}%
\definecolor{textcolor}{rgb}{0.000000,0.000000,0.000000}%
\pgfsetstrokecolor{textcolor}%
\pgfsetfillcolor{textcolor}%
\pgftext[x=3.730956in,y=0.562778in,,top]{\color{textcolor}\sffamily\fontsize{15.000000}{18.000000}\selectfont 20}%
\end{pgfscope}%
\begin{pgfscope}%
\pgfpathrectangle{\pgfqpoint{0.750000in}{0.660000in}}{\pgfqpoint{4.650000in}{4.620000in}}%
\pgfusepath{clip}%
\pgfsetrectcap%
\pgfsetroundjoin%
\pgfsetlinewidth{0.803000pt}%
\definecolor{currentstroke}{rgb}{0.690196,0.690196,0.690196}%
\pgfsetstrokecolor{currentstroke}%
\pgfsetdash{}{0pt}%
\pgfpathmoveto{\pgfqpoint{4.459796in}{0.660000in}}%
\pgfpathlineto{\pgfqpoint{4.459796in}{5.280000in}}%
\pgfusepath{stroke}%
\end{pgfscope}%
\begin{pgfscope}%
\pgfsetbuttcap%
\pgfsetroundjoin%
\definecolor{currentfill}{rgb}{0.000000,0.000000,0.000000}%
\pgfsetfillcolor{currentfill}%
\pgfsetlinewidth{0.803000pt}%
\definecolor{currentstroke}{rgb}{0.000000,0.000000,0.000000}%
\pgfsetstrokecolor{currentstroke}%
\pgfsetdash{}{0pt}%
\pgfsys@defobject{currentmarker}{\pgfqpoint{0.000000in}{-0.048611in}}{\pgfqpoint{0.000000in}{0.000000in}}{%
\pgfpathmoveto{\pgfqpoint{0.000000in}{0.000000in}}%
\pgfpathlineto{\pgfqpoint{0.000000in}{-0.048611in}}%
\pgfusepath{stroke,fill}%
}%
\begin{pgfscope}%
\pgfsys@transformshift{4.459796in}{0.660000in}%
\pgfsys@useobject{currentmarker}{}%
\end{pgfscope}%
\end{pgfscope}%
\begin{pgfscope}%
\definecolor{textcolor}{rgb}{0.000000,0.000000,0.000000}%
\pgfsetstrokecolor{textcolor}%
\pgfsetfillcolor{textcolor}%
\pgftext[x=4.459796in,y=0.562778in,,top]{\color{textcolor}\sffamily\fontsize{15.000000}{18.000000}\selectfont 25}%
\end{pgfscope}%
\begin{pgfscope}%
\pgfpathrectangle{\pgfqpoint{0.750000in}{0.660000in}}{\pgfqpoint{4.650000in}{4.620000in}}%
\pgfusepath{clip}%
\pgfsetrectcap%
\pgfsetroundjoin%
\pgfsetlinewidth{0.803000pt}%
\definecolor{currentstroke}{rgb}{0.690196,0.690196,0.690196}%
\pgfsetstrokecolor{currentstroke}%
\pgfsetdash{}{0pt}%
\pgfpathmoveto{\pgfqpoint{5.188636in}{0.660000in}}%
\pgfpathlineto{\pgfqpoint{5.188636in}{5.280000in}}%
\pgfusepath{stroke}%
\end{pgfscope}%
\begin{pgfscope}%
\pgfsetbuttcap%
\pgfsetroundjoin%
\definecolor{currentfill}{rgb}{0.000000,0.000000,0.000000}%
\pgfsetfillcolor{currentfill}%
\pgfsetlinewidth{0.803000pt}%
\definecolor{currentstroke}{rgb}{0.000000,0.000000,0.000000}%
\pgfsetstrokecolor{currentstroke}%
\pgfsetdash{}{0pt}%
\pgfsys@defobject{currentmarker}{\pgfqpoint{0.000000in}{-0.048611in}}{\pgfqpoint{0.000000in}{0.000000in}}{%
\pgfpathmoveto{\pgfqpoint{0.000000in}{0.000000in}}%
\pgfpathlineto{\pgfqpoint{0.000000in}{-0.048611in}}%
\pgfusepath{stroke,fill}%
}%
\begin{pgfscope}%
\pgfsys@transformshift{5.188636in}{0.660000in}%
\pgfsys@useobject{currentmarker}{}%
\end{pgfscope}%
\end{pgfscope}%
\begin{pgfscope}%
\definecolor{textcolor}{rgb}{0.000000,0.000000,0.000000}%
\pgfsetstrokecolor{textcolor}%
\pgfsetfillcolor{textcolor}%
\pgftext[x=5.188636in,y=0.562778in,,top]{\color{textcolor}\sffamily\fontsize{15.000000}{18.000000}\selectfont 30}%
\end{pgfscope}%
\begin{pgfscope}%
\definecolor{textcolor}{rgb}{0.000000,0.000000,0.000000}%
\pgfsetstrokecolor{textcolor}%
\pgfsetfillcolor{textcolor}%
\pgftext[x=3.075000in,y=0.305603in,,top]{\color{textcolor}\sffamily\fontsize{15.000000}{18.000000}\selectfont \(\displaystyle \mu\)}%
\end{pgfscope}%
\begin{pgfscope}%
\pgfpathrectangle{\pgfqpoint{0.750000in}{0.660000in}}{\pgfqpoint{4.650000in}{4.620000in}}%
\pgfusepath{clip}%
\pgfsetrectcap%
\pgfsetroundjoin%
\pgfsetlinewidth{0.803000pt}%
\definecolor{currentstroke}{rgb}{0.690196,0.690196,0.690196}%
\pgfsetstrokecolor{currentstroke}%
\pgfsetdash{}{0pt}%
\pgfpathmoveto{\pgfqpoint{0.750000in}{0.660000in}}%
\pgfpathlineto{\pgfqpoint{5.400000in}{0.660000in}}%
\pgfusepath{stroke}%
\end{pgfscope}%
\begin{pgfscope}%
\pgfsetbuttcap%
\pgfsetroundjoin%
\definecolor{currentfill}{rgb}{0.000000,0.000000,0.000000}%
\pgfsetfillcolor{currentfill}%
\pgfsetlinewidth{0.803000pt}%
\definecolor{currentstroke}{rgb}{0.000000,0.000000,0.000000}%
\pgfsetstrokecolor{currentstroke}%
\pgfsetdash{}{0pt}%
\pgfsys@defobject{currentmarker}{\pgfqpoint{-0.048611in}{0.000000in}}{\pgfqpoint{-0.000000in}{0.000000in}}{%
\pgfpathmoveto{\pgfqpoint{-0.000000in}{0.000000in}}%
\pgfpathlineto{\pgfqpoint{-0.048611in}{0.000000in}}%
\pgfusepath{stroke,fill}%
}%
\begin{pgfscope}%
\pgfsys@transformshift{0.750000in}{0.660000in}%
\pgfsys@useobject{currentmarker}{}%
\end{pgfscope}%
\end{pgfscope}%
\begin{pgfscope}%
\definecolor{textcolor}{rgb}{0.000000,0.000000,0.000000}%
\pgfsetstrokecolor{textcolor}%
\pgfsetfillcolor{textcolor}%
\pgftext[x=0.321459in, y=0.580858in, left, base]{\color{textcolor}\sffamily\fontsize{15.000000}{18.000000}\selectfont 0.3}%
\end{pgfscope}%
\begin{pgfscope}%
\pgfpathrectangle{\pgfqpoint{0.750000in}{0.660000in}}{\pgfqpoint{4.650000in}{4.620000in}}%
\pgfusepath{clip}%
\pgfsetrectcap%
\pgfsetroundjoin%
\pgfsetlinewidth{0.803000pt}%
\definecolor{currentstroke}{rgb}{0.690196,0.690196,0.690196}%
\pgfsetstrokecolor{currentstroke}%
\pgfsetdash{}{0pt}%
\pgfpathmoveto{\pgfqpoint{0.750000in}{1.237500in}}%
\pgfpathlineto{\pgfqpoint{5.400000in}{1.237500in}}%
\pgfusepath{stroke}%
\end{pgfscope}%
\begin{pgfscope}%
\pgfsetbuttcap%
\pgfsetroundjoin%
\definecolor{currentfill}{rgb}{0.000000,0.000000,0.000000}%
\pgfsetfillcolor{currentfill}%
\pgfsetlinewidth{0.803000pt}%
\definecolor{currentstroke}{rgb}{0.000000,0.000000,0.000000}%
\pgfsetstrokecolor{currentstroke}%
\pgfsetdash{}{0pt}%
\pgfsys@defobject{currentmarker}{\pgfqpoint{-0.048611in}{0.000000in}}{\pgfqpoint{-0.000000in}{0.000000in}}{%
\pgfpathmoveto{\pgfqpoint{-0.000000in}{0.000000in}}%
\pgfpathlineto{\pgfqpoint{-0.048611in}{0.000000in}}%
\pgfusepath{stroke,fill}%
}%
\begin{pgfscope}%
\pgfsys@transformshift{0.750000in}{1.237500in}%
\pgfsys@useobject{currentmarker}{}%
\end{pgfscope}%
\end{pgfscope}%
\begin{pgfscope}%
\definecolor{textcolor}{rgb}{0.000000,0.000000,0.000000}%
\pgfsetstrokecolor{textcolor}%
\pgfsetfillcolor{textcolor}%
\pgftext[x=0.321459in, y=1.158358in, left, base]{\color{textcolor}\sffamily\fontsize{15.000000}{18.000000}\selectfont 0.4}%
\end{pgfscope}%
\begin{pgfscope}%
\pgfpathrectangle{\pgfqpoint{0.750000in}{0.660000in}}{\pgfqpoint{4.650000in}{4.620000in}}%
\pgfusepath{clip}%
\pgfsetrectcap%
\pgfsetroundjoin%
\pgfsetlinewidth{0.803000pt}%
\definecolor{currentstroke}{rgb}{0.690196,0.690196,0.690196}%
\pgfsetstrokecolor{currentstroke}%
\pgfsetdash{}{0pt}%
\pgfpathmoveto{\pgfqpoint{0.750000in}{1.815000in}}%
\pgfpathlineto{\pgfqpoint{5.400000in}{1.815000in}}%
\pgfusepath{stroke}%
\end{pgfscope}%
\begin{pgfscope}%
\pgfsetbuttcap%
\pgfsetroundjoin%
\definecolor{currentfill}{rgb}{0.000000,0.000000,0.000000}%
\pgfsetfillcolor{currentfill}%
\pgfsetlinewidth{0.803000pt}%
\definecolor{currentstroke}{rgb}{0.000000,0.000000,0.000000}%
\pgfsetstrokecolor{currentstroke}%
\pgfsetdash{}{0pt}%
\pgfsys@defobject{currentmarker}{\pgfqpoint{-0.048611in}{0.000000in}}{\pgfqpoint{-0.000000in}{0.000000in}}{%
\pgfpathmoveto{\pgfqpoint{-0.000000in}{0.000000in}}%
\pgfpathlineto{\pgfqpoint{-0.048611in}{0.000000in}}%
\pgfusepath{stroke,fill}%
}%
\begin{pgfscope}%
\pgfsys@transformshift{0.750000in}{1.815000in}%
\pgfsys@useobject{currentmarker}{}%
\end{pgfscope}%
\end{pgfscope}%
\begin{pgfscope}%
\definecolor{textcolor}{rgb}{0.000000,0.000000,0.000000}%
\pgfsetstrokecolor{textcolor}%
\pgfsetfillcolor{textcolor}%
\pgftext[x=0.321459in, y=1.735858in, left, base]{\color{textcolor}\sffamily\fontsize{15.000000}{18.000000}\selectfont 0.5}%
\end{pgfscope}%
\begin{pgfscope}%
\pgfpathrectangle{\pgfqpoint{0.750000in}{0.660000in}}{\pgfqpoint{4.650000in}{4.620000in}}%
\pgfusepath{clip}%
\pgfsetrectcap%
\pgfsetroundjoin%
\pgfsetlinewidth{0.803000pt}%
\definecolor{currentstroke}{rgb}{0.690196,0.690196,0.690196}%
\pgfsetstrokecolor{currentstroke}%
\pgfsetdash{}{0pt}%
\pgfpathmoveto{\pgfqpoint{0.750000in}{2.392500in}}%
\pgfpathlineto{\pgfqpoint{5.400000in}{2.392500in}}%
\pgfusepath{stroke}%
\end{pgfscope}%
\begin{pgfscope}%
\pgfsetbuttcap%
\pgfsetroundjoin%
\definecolor{currentfill}{rgb}{0.000000,0.000000,0.000000}%
\pgfsetfillcolor{currentfill}%
\pgfsetlinewidth{0.803000pt}%
\definecolor{currentstroke}{rgb}{0.000000,0.000000,0.000000}%
\pgfsetstrokecolor{currentstroke}%
\pgfsetdash{}{0pt}%
\pgfsys@defobject{currentmarker}{\pgfqpoint{-0.048611in}{0.000000in}}{\pgfqpoint{-0.000000in}{0.000000in}}{%
\pgfpathmoveto{\pgfqpoint{-0.000000in}{0.000000in}}%
\pgfpathlineto{\pgfqpoint{-0.048611in}{0.000000in}}%
\pgfusepath{stroke,fill}%
}%
\begin{pgfscope}%
\pgfsys@transformshift{0.750000in}{2.392500in}%
\pgfsys@useobject{currentmarker}{}%
\end{pgfscope}%
\end{pgfscope}%
\begin{pgfscope}%
\definecolor{textcolor}{rgb}{0.000000,0.000000,0.000000}%
\pgfsetstrokecolor{textcolor}%
\pgfsetfillcolor{textcolor}%
\pgftext[x=0.321459in, y=2.313358in, left, base]{\color{textcolor}\sffamily\fontsize{15.000000}{18.000000}\selectfont 0.6}%
\end{pgfscope}%
\begin{pgfscope}%
\pgfpathrectangle{\pgfqpoint{0.750000in}{0.660000in}}{\pgfqpoint{4.650000in}{4.620000in}}%
\pgfusepath{clip}%
\pgfsetrectcap%
\pgfsetroundjoin%
\pgfsetlinewidth{0.803000pt}%
\definecolor{currentstroke}{rgb}{0.690196,0.690196,0.690196}%
\pgfsetstrokecolor{currentstroke}%
\pgfsetdash{}{0pt}%
\pgfpathmoveto{\pgfqpoint{0.750000in}{2.970000in}}%
\pgfpathlineto{\pgfqpoint{5.400000in}{2.970000in}}%
\pgfusepath{stroke}%
\end{pgfscope}%
\begin{pgfscope}%
\pgfsetbuttcap%
\pgfsetroundjoin%
\definecolor{currentfill}{rgb}{0.000000,0.000000,0.000000}%
\pgfsetfillcolor{currentfill}%
\pgfsetlinewidth{0.803000pt}%
\definecolor{currentstroke}{rgb}{0.000000,0.000000,0.000000}%
\pgfsetstrokecolor{currentstroke}%
\pgfsetdash{}{0pt}%
\pgfsys@defobject{currentmarker}{\pgfqpoint{-0.048611in}{0.000000in}}{\pgfqpoint{-0.000000in}{0.000000in}}{%
\pgfpathmoveto{\pgfqpoint{-0.000000in}{0.000000in}}%
\pgfpathlineto{\pgfqpoint{-0.048611in}{0.000000in}}%
\pgfusepath{stroke,fill}%
}%
\begin{pgfscope}%
\pgfsys@transformshift{0.750000in}{2.970000in}%
\pgfsys@useobject{currentmarker}{}%
\end{pgfscope}%
\end{pgfscope}%
\begin{pgfscope}%
\definecolor{textcolor}{rgb}{0.000000,0.000000,0.000000}%
\pgfsetstrokecolor{textcolor}%
\pgfsetfillcolor{textcolor}%
\pgftext[x=0.321459in, y=2.890858in, left, base]{\color{textcolor}\sffamily\fontsize{15.000000}{18.000000}\selectfont 0.7}%
\end{pgfscope}%
\begin{pgfscope}%
\pgfpathrectangle{\pgfqpoint{0.750000in}{0.660000in}}{\pgfqpoint{4.650000in}{4.620000in}}%
\pgfusepath{clip}%
\pgfsetrectcap%
\pgfsetroundjoin%
\pgfsetlinewidth{0.803000pt}%
\definecolor{currentstroke}{rgb}{0.690196,0.690196,0.690196}%
\pgfsetstrokecolor{currentstroke}%
\pgfsetdash{}{0pt}%
\pgfpathmoveto{\pgfqpoint{0.750000in}{3.547500in}}%
\pgfpathlineto{\pgfqpoint{5.400000in}{3.547500in}}%
\pgfusepath{stroke}%
\end{pgfscope}%
\begin{pgfscope}%
\pgfsetbuttcap%
\pgfsetroundjoin%
\definecolor{currentfill}{rgb}{0.000000,0.000000,0.000000}%
\pgfsetfillcolor{currentfill}%
\pgfsetlinewidth{0.803000pt}%
\definecolor{currentstroke}{rgb}{0.000000,0.000000,0.000000}%
\pgfsetstrokecolor{currentstroke}%
\pgfsetdash{}{0pt}%
\pgfsys@defobject{currentmarker}{\pgfqpoint{-0.048611in}{0.000000in}}{\pgfqpoint{-0.000000in}{0.000000in}}{%
\pgfpathmoveto{\pgfqpoint{-0.000000in}{0.000000in}}%
\pgfpathlineto{\pgfqpoint{-0.048611in}{0.000000in}}%
\pgfusepath{stroke,fill}%
}%
\begin{pgfscope}%
\pgfsys@transformshift{0.750000in}{3.547500in}%
\pgfsys@useobject{currentmarker}{}%
\end{pgfscope}%
\end{pgfscope}%
\begin{pgfscope}%
\definecolor{textcolor}{rgb}{0.000000,0.000000,0.000000}%
\pgfsetstrokecolor{textcolor}%
\pgfsetfillcolor{textcolor}%
\pgftext[x=0.321459in, y=3.468358in, left, base]{\color{textcolor}\sffamily\fontsize{15.000000}{18.000000}\selectfont 0.8}%
\end{pgfscope}%
\begin{pgfscope}%
\pgfpathrectangle{\pgfqpoint{0.750000in}{0.660000in}}{\pgfqpoint{4.650000in}{4.620000in}}%
\pgfusepath{clip}%
\pgfsetrectcap%
\pgfsetroundjoin%
\pgfsetlinewidth{0.803000pt}%
\definecolor{currentstroke}{rgb}{0.690196,0.690196,0.690196}%
\pgfsetstrokecolor{currentstroke}%
\pgfsetdash{}{0pt}%
\pgfpathmoveto{\pgfqpoint{0.750000in}{4.125000in}}%
\pgfpathlineto{\pgfqpoint{5.400000in}{4.125000in}}%
\pgfusepath{stroke}%
\end{pgfscope}%
\begin{pgfscope}%
\pgfsetbuttcap%
\pgfsetroundjoin%
\definecolor{currentfill}{rgb}{0.000000,0.000000,0.000000}%
\pgfsetfillcolor{currentfill}%
\pgfsetlinewidth{0.803000pt}%
\definecolor{currentstroke}{rgb}{0.000000,0.000000,0.000000}%
\pgfsetstrokecolor{currentstroke}%
\pgfsetdash{}{0pt}%
\pgfsys@defobject{currentmarker}{\pgfqpoint{-0.048611in}{0.000000in}}{\pgfqpoint{-0.000000in}{0.000000in}}{%
\pgfpathmoveto{\pgfqpoint{-0.000000in}{0.000000in}}%
\pgfpathlineto{\pgfqpoint{-0.048611in}{0.000000in}}%
\pgfusepath{stroke,fill}%
}%
\begin{pgfscope}%
\pgfsys@transformshift{0.750000in}{4.125000in}%
\pgfsys@useobject{currentmarker}{}%
\end{pgfscope}%
\end{pgfscope}%
\begin{pgfscope}%
\definecolor{textcolor}{rgb}{0.000000,0.000000,0.000000}%
\pgfsetstrokecolor{textcolor}%
\pgfsetfillcolor{textcolor}%
\pgftext[x=0.321459in, y=4.045858in, left, base]{\color{textcolor}\sffamily\fontsize{15.000000}{18.000000}\selectfont 0.9}%
\end{pgfscope}%
\begin{pgfscope}%
\pgfpathrectangle{\pgfqpoint{0.750000in}{0.660000in}}{\pgfqpoint{4.650000in}{4.620000in}}%
\pgfusepath{clip}%
\pgfsetrectcap%
\pgfsetroundjoin%
\pgfsetlinewidth{0.803000pt}%
\definecolor{currentstroke}{rgb}{0.690196,0.690196,0.690196}%
\pgfsetstrokecolor{currentstroke}%
\pgfsetdash{}{0pt}%
\pgfpathmoveto{\pgfqpoint{0.750000in}{4.702500in}}%
\pgfpathlineto{\pgfqpoint{5.400000in}{4.702500in}}%
\pgfusepath{stroke}%
\end{pgfscope}%
\begin{pgfscope}%
\pgfsetbuttcap%
\pgfsetroundjoin%
\definecolor{currentfill}{rgb}{0.000000,0.000000,0.000000}%
\pgfsetfillcolor{currentfill}%
\pgfsetlinewidth{0.803000pt}%
\definecolor{currentstroke}{rgb}{0.000000,0.000000,0.000000}%
\pgfsetstrokecolor{currentstroke}%
\pgfsetdash{}{0pt}%
\pgfsys@defobject{currentmarker}{\pgfqpoint{-0.048611in}{0.000000in}}{\pgfqpoint{-0.000000in}{0.000000in}}{%
\pgfpathmoveto{\pgfqpoint{-0.000000in}{0.000000in}}%
\pgfpathlineto{\pgfqpoint{-0.048611in}{0.000000in}}%
\pgfusepath{stroke,fill}%
}%
\begin{pgfscope}%
\pgfsys@transformshift{0.750000in}{4.702500in}%
\pgfsys@useobject{currentmarker}{}%
\end{pgfscope}%
\end{pgfscope}%
\begin{pgfscope}%
\definecolor{textcolor}{rgb}{0.000000,0.000000,0.000000}%
\pgfsetstrokecolor{textcolor}%
\pgfsetfillcolor{textcolor}%
\pgftext[x=0.321459in, y=4.623358in, left, base]{\color{textcolor}\sffamily\fontsize{15.000000}{18.000000}\selectfont 1.0}%
\end{pgfscope}%
\begin{pgfscope}%
\pgfpathrectangle{\pgfqpoint{0.750000in}{0.660000in}}{\pgfqpoint{4.650000in}{4.620000in}}%
\pgfusepath{clip}%
\pgfsetrectcap%
\pgfsetroundjoin%
\pgfsetlinewidth{0.803000pt}%
\definecolor{currentstroke}{rgb}{0.690196,0.690196,0.690196}%
\pgfsetstrokecolor{currentstroke}%
\pgfsetdash{}{0pt}%
\pgfpathmoveto{\pgfqpoint{0.750000in}{5.280000in}}%
\pgfpathlineto{\pgfqpoint{5.400000in}{5.280000in}}%
\pgfusepath{stroke}%
\end{pgfscope}%
\begin{pgfscope}%
\pgfsetbuttcap%
\pgfsetroundjoin%
\definecolor{currentfill}{rgb}{0.000000,0.000000,0.000000}%
\pgfsetfillcolor{currentfill}%
\pgfsetlinewidth{0.803000pt}%
\definecolor{currentstroke}{rgb}{0.000000,0.000000,0.000000}%
\pgfsetstrokecolor{currentstroke}%
\pgfsetdash{}{0pt}%
\pgfsys@defobject{currentmarker}{\pgfqpoint{-0.048611in}{0.000000in}}{\pgfqpoint{-0.000000in}{0.000000in}}{%
\pgfpathmoveto{\pgfqpoint{-0.000000in}{0.000000in}}%
\pgfpathlineto{\pgfqpoint{-0.048611in}{0.000000in}}%
\pgfusepath{stroke,fill}%
}%
\begin{pgfscope}%
\pgfsys@transformshift{0.750000in}{5.280000in}%
\pgfsys@useobject{currentmarker}{}%
\end{pgfscope}%
\end{pgfscope}%
\begin{pgfscope}%
\definecolor{textcolor}{rgb}{0.000000,0.000000,0.000000}%
\pgfsetstrokecolor{textcolor}%
\pgfsetfillcolor{textcolor}%
\pgftext[x=0.321459in, y=5.200858in, left, base]{\color{textcolor}\sffamily\fontsize{15.000000}{18.000000}\selectfont 1.1}%
\end{pgfscope}%
\begin{pgfscope}%
\definecolor{textcolor}{rgb}{0.000000,0.000000,0.000000}%
\pgfsetstrokecolor{textcolor}%
\pgfsetfillcolor{textcolor}%
\pgftext[x=0.265903in,y=2.970000in,,bottom,rotate=90.000000]{\color{textcolor}\sffamily\fontsize{15.000000}{18.000000}\selectfont \(\displaystyle \mathrm{ratio}\)}%
\end{pgfscope}%
\begin{pgfscope}%
\pgfpathrectangle{\pgfqpoint{0.750000in}{0.660000in}}{\pgfqpoint{4.650000in}{4.620000in}}%
\pgfusepath{clip}%
\pgfsetbuttcap%
\pgfsetroundjoin%
\pgfsetlinewidth{1.505625pt}%
\definecolor{currentstroke}{rgb}{0.121569,0.466667,0.705882}%
\pgfsetstrokecolor{currentstroke}%
\pgfsetdash{}{0pt}%
\pgfpathmoveto{\pgfqpoint{0.961364in}{4.356658in}}%
\pgfpathlineto{\pgfqpoint{0.961364in}{4.466377in}}%
\pgfusepath{stroke}%
\end{pgfscope}%
\begin{pgfscope}%
\pgfpathrectangle{\pgfqpoint{0.750000in}{0.660000in}}{\pgfqpoint{4.650000in}{4.620000in}}%
\pgfusepath{clip}%
\pgfsetbuttcap%
\pgfsetroundjoin%
\pgfsetlinewidth{1.505625pt}%
\definecolor{currentstroke}{rgb}{0.121569,0.466667,0.705882}%
\pgfsetstrokecolor{currentstroke}%
\pgfsetdash{}{0pt}%
\pgfpathmoveto{\pgfqpoint{1.107132in}{4.110661in}}%
\pgfpathlineto{\pgfqpoint{1.107132in}{4.215413in}}%
\pgfusepath{stroke}%
\end{pgfscope}%
\begin{pgfscope}%
\pgfpathrectangle{\pgfqpoint{0.750000in}{0.660000in}}{\pgfqpoint{4.650000in}{4.620000in}}%
\pgfusepath{clip}%
\pgfsetbuttcap%
\pgfsetroundjoin%
\pgfsetlinewidth{1.505625pt}%
\definecolor{currentstroke}{rgb}{0.121569,0.466667,0.705882}%
\pgfsetstrokecolor{currentstroke}%
\pgfsetdash{}{0pt}%
\pgfpathmoveto{\pgfqpoint{1.252900in}{3.873468in}}%
\pgfpathlineto{\pgfqpoint{1.252900in}{3.973437in}}%
\pgfusepath{stroke}%
\end{pgfscope}%
\begin{pgfscope}%
\pgfpathrectangle{\pgfqpoint{0.750000in}{0.660000in}}{\pgfqpoint{4.650000in}{4.620000in}}%
\pgfusepath{clip}%
\pgfsetbuttcap%
\pgfsetroundjoin%
\pgfsetlinewidth{1.505625pt}%
\definecolor{currentstroke}{rgb}{0.121569,0.466667,0.705882}%
\pgfsetstrokecolor{currentstroke}%
\pgfsetdash{}{0pt}%
\pgfpathmoveto{\pgfqpoint{1.398668in}{3.761246in}}%
\pgfpathlineto{\pgfqpoint{1.398668in}{3.858936in}}%
\pgfusepath{stroke}%
\end{pgfscope}%
\begin{pgfscope}%
\pgfpathrectangle{\pgfqpoint{0.750000in}{0.660000in}}{\pgfqpoint{4.650000in}{4.620000in}}%
\pgfusepath{clip}%
\pgfsetbuttcap%
\pgfsetroundjoin%
\pgfsetlinewidth{1.505625pt}%
\definecolor{currentstroke}{rgb}{0.121569,0.466667,0.705882}%
\pgfsetstrokecolor{currentstroke}%
\pgfsetdash{}{0pt}%
\pgfpathmoveto{\pgfqpoint{1.544436in}{3.621804in}}%
\pgfpathlineto{\pgfqpoint{1.544436in}{3.716696in}}%
\pgfusepath{stroke}%
\end{pgfscope}%
\begin{pgfscope}%
\pgfpathrectangle{\pgfqpoint{0.750000in}{0.660000in}}{\pgfqpoint{4.650000in}{4.620000in}}%
\pgfusepath{clip}%
\pgfsetbuttcap%
\pgfsetroundjoin%
\pgfsetlinewidth{1.505625pt}%
\definecolor{currentstroke}{rgb}{0.121569,0.466667,0.705882}%
\pgfsetstrokecolor{currentstroke}%
\pgfsetdash{}{0pt}%
\pgfpathmoveto{\pgfqpoint{1.690204in}{3.510473in}}%
\pgfpathlineto{\pgfqpoint{1.690204in}{3.603091in}}%
\pgfusepath{stroke}%
\end{pgfscope}%
\begin{pgfscope}%
\pgfpathrectangle{\pgfqpoint{0.750000in}{0.660000in}}{\pgfqpoint{4.650000in}{4.620000in}}%
\pgfusepath{clip}%
\pgfsetbuttcap%
\pgfsetroundjoin%
\pgfsetlinewidth{1.505625pt}%
\definecolor{currentstroke}{rgb}{0.121569,0.466667,0.705882}%
\pgfsetstrokecolor{currentstroke}%
\pgfsetdash{}{0pt}%
\pgfpathmoveto{\pgfqpoint{1.981740in}{3.290116in}}%
\pgfpathlineto{\pgfqpoint{1.981740in}{3.378263in}}%
\pgfusepath{stroke}%
\end{pgfscope}%
\begin{pgfscope}%
\pgfpathrectangle{\pgfqpoint{0.750000in}{0.660000in}}{\pgfqpoint{4.650000in}{4.620000in}}%
\pgfusepath{clip}%
\pgfsetbuttcap%
\pgfsetroundjoin%
\pgfsetlinewidth{1.505625pt}%
\definecolor{currentstroke}{rgb}{0.121569,0.466667,0.705882}%
\pgfsetstrokecolor{currentstroke}%
\pgfsetdash{}{0pt}%
\pgfpathmoveto{\pgfqpoint{2.273276in}{3.049869in}}%
\pgfpathlineto{\pgfqpoint{2.273276in}{3.133149in}}%
\pgfusepath{stroke}%
\end{pgfscope}%
\begin{pgfscope}%
\pgfpathrectangle{\pgfqpoint{0.750000in}{0.660000in}}{\pgfqpoint{4.650000in}{4.620000in}}%
\pgfusepath{clip}%
\pgfsetbuttcap%
\pgfsetroundjoin%
\pgfsetlinewidth{1.505625pt}%
\definecolor{currentstroke}{rgb}{0.121569,0.466667,0.705882}%
\pgfsetstrokecolor{currentstroke}%
\pgfsetdash{}{0pt}%
\pgfpathmoveto{\pgfqpoint{3.002116in}{2.670890in}}%
\pgfpathlineto{\pgfqpoint{3.002116in}{2.746514in}}%
\pgfusepath{stroke}%
\end{pgfscope}%
\begin{pgfscope}%
\pgfpathrectangle{\pgfqpoint{0.750000in}{0.660000in}}{\pgfqpoint{4.650000in}{4.620000in}}%
\pgfusepath{clip}%
\pgfsetbuttcap%
\pgfsetroundjoin%
\pgfsetlinewidth{1.505625pt}%
\definecolor{currentstroke}{rgb}{0.121569,0.466667,0.705882}%
\pgfsetstrokecolor{currentstroke}%
\pgfsetdash{}{0pt}%
\pgfpathmoveto{\pgfqpoint{3.730956in}{2.328265in}}%
\pgfpathlineto{\pgfqpoint{3.730956in}{2.396967in}}%
\pgfusepath{stroke}%
\end{pgfscope}%
\begin{pgfscope}%
\pgfpathrectangle{\pgfqpoint{0.750000in}{0.660000in}}{\pgfqpoint{4.650000in}{4.620000in}}%
\pgfusepath{clip}%
\pgfsetbuttcap%
\pgfsetroundjoin%
\pgfsetlinewidth{1.505625pt}%
\definecolor{currentstroke}{rgb}{0.121569,0.466667,0.705882}%
\pgfsetstrokecolor{currentstroke}%
\pgfsetdash{}{0pt}%
\pgfpathmoveto{\pgfqpoint{4.459796in}{2.122302in}}%
\pgfpathlineto{\pgfqpoint{4.459796in}{2.186843in}}%
\pgfusepath{stroke}%
\end{pgfscope}%
\begin{pgfscope}%
\pgfpathrectangle{\pgfqpoint{0.750000in}{0.660000in}}{\pgfqpoint{4.650000in}{4.620000in}}%
\pgfusepath{clip}%
\pgfsetbuttcap%
\pgfsetroundjoin%
\pgfsetlinewidth{1.505625pt}%
\definecolor{currentstroke}{rgb}{0.121569,0.466667,0.705882}%
\pgfsetstrokecolor{currentstroke}%
\pgfsetdash{}{0pt}%
\pgfpathmoveto{\pgfqpoint{5.188636in}{1.928159in}}%
\pgfpathlineto{\pgfqpoint{5.188636in}{1.988779in}}%
\pgfusepath{stroke}%
\end{pgfscope}%
\begin{pgfscope}%
\pgfpathrectangle{\pgfqpoint{0.750000in}{0.660000in}}{\pgfqpoint{4.650000in}{4.620000in}}%
\pgfusepath{clip}%
\pgfsetrectcap%
\pgfsetroundjoin%
\pgfsetlinewidth{1.505625pt}%
\definecolor{currentstroke}{rgb}{0.121569,0.466667,0.705882}%
\pgfsetstrokecolor{currentstroke}%
\pgfsetdash{}{0pt}%
\pgfpathmoveto{\pgfqpoint{0.961364in}{4.411517in}}%
\pgfpathlineto{\pgfqpoint{1.107132in}{4.163037in}}%
\pgfpathlineto{\pgfqpoint{1.252900in}{3.923453in}}%
\pgfpathlineto{\pgfqpoint{1.398668in}{3.810091in}}%
\pgfpathlineto{\pgfqpoint{1.544436in}{3.669250in}}%
\pgfpathlineto{\pgfqpoint{1.690204in}{3.556782in}}%
\pgfpathlineto{\pgfqpoint{1.981740in}{3.334189in}}%
\pgfpathlineto{\pgfqpoint{2.273276in}{3.091509in}}%
\pgfpathlineto{\pgfqpoint{3.002116in}{2.708702in}}%
\pgfpathlineto{\pgfqpoint{3.730956in}{2.362616in}}%
\pgfpathlineto{\pgfqpoint{4.459796in}{2.154572in}}%
\pgfpathlineto{\pgfqpoint{5.188636in}{1.958469in}}%
\pgfusepath{stroke}%
\end{pgfscope}%
\begin{pgfscope}%
\pgfpathrectangle{\pgfqpoint{0.750000in}{0.660000in}}{\pgfqpoint{4.650000in}{4.620000in}}%
\pgfusepath{clip}%
\pgfsetbuttcap%
\pgfsetmiterjoin%
\definecolor{currentfill}{rgb}{0.121569,0.466667,0.705882}%
\pgfsetfillcolor{currentfill}%
\pgfsetlinewidth{1.003750pt}%
\definecolor{currentstroke}{rgb}{0.121569,0.466667,0.705882}%
\pgfsetstrokecolor{currentstroke}%
\pgfsetdash{}{0pt}%
\pgfsys@defobject{currentmarker}{\pgfqpoint{-0.041667in}{-0.041667in}}{\pgfqpoint{0.041667in}{0.041667in}}{%
\pgfpathmoveto{\pgfqpoint{0.000000in}{0.041667in}}%
\pgfpathlineto{\pgfqpoint{-0.041667in}{-0.041667in}}%
\pgfpathlineto{\pgfqpoint{0.041667in}{-0.041667in}}%
\pgfpathclose%
\pgfusepath{stroke,fill}%
}%
\begin{pgfscope}%
\pgfsys@transformshift{0.961364in}{4.411517in}%
\pgfsys@useobject{currentmarker}{}%
\end{pgfscope}%
\begin{pgfscope}%
\pgfsys@transformshift{1.107132in}{4.163037in}%
\pgfsys@useobject{currentmarker}{}%
\end{pgfscope}%
\begin{pgfscope}%
\pgfsys@transformshift{1.252900in}{3.923453in}%
\pgfsys@useobject{currentmarker}{}%
\end{pgfscope}%
\begin{pgfscope}%
\pgfsys@transformshift{1.398668in}{3.810091in}%
\pgfsys@useobject{currentmarker}{}%
\end{pgfscope}%
\begin{pgfscope}%
\pgfsys@transformshift{1.544436in}{3.669250in}%
\pgfsys@useobject{currentmarker}{}%
\end{pgfscope}%
\begin{pgfscope}%
\pgfsys@transformshift{1.690204in}{3.556782in}%
\pgfsys@useobject{currentmarker}{}%
\end{pgfscope}%
\begin{pgfscope}%
\pgfsys@transformshift{1.981740in}{3.334189in}%
\pgfsys@useobject{currentmarker}{}%
\end{pgfscope}%
\begin{pgfscope}%
\pgfsys@transformshift{2.273276in}{3.091509in}%
\pgfsys@useobject{currentmarker}{}%
\end{pgfscope}%
\begin{pgfscope}%
\pgfsys@transformshift{3.002116in}{2.708702in}%
\pgfsys@useobject{currentmarker}{}%
\end{pgfscope}%
\begin{pgfscope}%
\pgfsys@transformshift{3.730956in}{2.362616in}%
\pgfsys@useobject{currentmarker}{}%
\end{pgfscope}%
\begin{pgfscope}%
\pgfsys@transformshift{4.459796in}{2.154572in}%
\pgfsys@useobject{currentmarker}{}%
\end{pgfscope}%
\begin{pgfscope}%
\pgfsys@transformshift{5.188636in}{1.958469in}%
\pgfsys@useobject{currentmarker}{}%
\end{pgfscope}%
\end{pgfscope}%
\begin{pgfscope}%
\pgfsetrectcap%
\pgfsetmiterjoin%
\pgfsetlinewidth{0.803000pt}%
\definecolor{currentstroke}{rgb}{0.000000,0.000000,0.000000}%
\pgfsetstrokecolor{currentstroke}%
\pgfsetdash{}{0pt}%
\pgfpathmoveto{\pgfqpoint{0.750000in}{0.660000in}}%
\pgfpathlineto{\pgfqpoint{0.750000in}{5.280000in}}%
\pgfusepath{stroke}%
\end{pgfscope}%
\begin{pgfscope}%
\pgfsetrectcap%
\pgfsetmiterjoin%
\pgfsetlinewidth{0.803000pt}%
\definecolor{currentstroke}{rgb}{0.000000,0.000000,0.000000}%
\pgfsetstrokecolor{currentstroke}%
\pgfsetdash{}{0pt}%
\pgfpathmoveto{\pgfqpoint{5.400000in}{0.660000in}}%
\pgfpathlineto{\pgfqpoint{5.400000in}{5.280000in}}%
\pgfusepath{stroke}%
\end{pgfscope}%
\begin{pgfscope}%
\pgfsetrectcap%
\pgfsetmiterjoin%
\pgfsetlinewidth{0.803000pt}%
\definecolor{currentstroke}{rgb}{0.000000,0.000000,0.000000}%
\pgfsetstrokecolor{currentstroke}%
\pgfsetdash{}{0pt}%
\pgfpathmoveto{\pgfqpoint{0.750000in}{0.660000in}}%
\pgfpathlineto{\pgfqpoint{5.400000in}{0.660000in}}%
\pgfusepath{stroke}%
\end{pgfscope}%
\begin{pgfscope}%
\pgfsetrectcap%
\pgfsetmiterjoin%
\pgfsetlinewidth{0.803000pt}%
\definecolor{currentstroke}{rgb}{0.000000,0.000000,0.000000}%
\pgfsetstrokecolor{currentstroke}%
\pgfsetdash{}{0pt}%
\pgfpathmoveto{\pgfqpoint{0.750000in}{5.280000in}}%
\pgfpathlineto{\pgfqpoint{5.400000in}{5.280000in}}%
\pgfusepath{stroke}%
\end{pgfscope}%
\begin{pgfscope}%
\definecolor{textcolor}{rgb}{0.000000,0.000000,0.000000}%
\pgfsetstrokecolor{textcolor}%
\pgfsetfillcolor{textcolor}%
\pgftext[x=3.075000in,y=5.363333in,,base]{\color{textcolor}\sffamily\fontsize{18.000000}{21.600000}\selectfont \(\displaystyle \tau=20\mathrm{ns},\,\sigma=10\mathrm{ns}\)}%
\end{pgfscope}%
\begin{pgfscope}%
\pgfsetbuttcap%
\pgfsetmiterjoin%
\definecolor{currentfill}{rgb}{1.000000,1.000000,1.000000}%
\pgfsetfillcolor{currentfill}%
\pgfsetfillopacity{0.800000}%
\pgfsetlinewidth{1.003750pt}%
\definecolor{currentstroke}{rgb}{0.800000,0.800000,0.800000}%
\pgfsetstrokecolor{currentstroke}%
\pgfsetstrokeopacity{0.800000}%
\pgfsetdash{}{0pt}%
\pgfpathmoveto{\pgfqpoint{0.895833in}{0.764167in}}%
\pgfpathlineto{\pgfqpoint{2.487430in}{0.764167in}}%
\pgfpathquadraticcurveto{\pgfqpoint{2.529096in}{0.764167in}}{\pgfqpoint{2.529096in}{0.805833in}}%
\pgfpathlineto{\pgfqpoint{2.529096in}{1.097451in}}%
\pgfpathquadraticcurveto{\pgfqpoint{2.529096in}{1.139118in}}{\pgfqpoint{2.487430in}{1.139118in}}%
\pgfpathlineto{\pgfqpoint{0.895833in}{1.139118in}}%
\pgfpathquadraticcurveto{\pgfqpoint{0.854167in}{1.139118in}}{\pgfqpoint{0.854167in}{1.097451in}}%
\pgfpathlineto{\pgfqpoint{0.854167in}{0.805833in}}%
\pgfpathquadraticcurveto{\pgfqpoint{0.854167in}{0.764167in}}{\pgfqpoint{0.895833in}{0.764167in}}%
\pgfpathclose%
\pgfusepath{stroke,fill}%
\end{pgfscope}%
\begin{pgfscope}%
\pgfsetbuttcap%
\pgfsetroundjoin%
\pgfsetlinewidth{1.505625pt}%
\definecolor{currentstroke}{rgb}{0.121569,0.466667,0.705882}%
\pgfsetstrokecolor{currentstroke}%
\pgfsetdash{}{0pt}%
\pgfpathmoveto{\pgfqpoint{1.145833in}{0.866250in}}%
\pgfpathlineto{\pgfqpoint{1.145833in}{1.074583in}}%
\pgfusepath{stroke}%
\end{pgfscope}%
\begin{pgfscope}%
\pgfsetrectcap%
\pgfsetroundjoin%
\pgfsetlinewidth{1.505625pt}%
\definecolor{currentstroke}{rgb}{0.121569,0.466667,0.705882}%
\pgfsetstrokecolor{currentstroke}%
\pgfsetdash{}{0pt}%
\pgfpathmoveto{\pgfqpoint{0.937500in}{0.970417in}}%
\pgfpathlineto{\pgfqpoint{1.354167in}{0.970417in}}%
\pgfusepath{stroke}%
\end{pgfscope}%
\begin{pgfscope}%
\pgfsetbuttcap%
\pgfsetmiterjoin%
\definecolor{currentfill}{rgb}{0.121569,0.466667,0.705882}%
\pgfsetfillcolor{currentfill}%
\pgfsetlinewidth{1.003750pt}%
\definecolor{currentstroke}{rgb}{0.121569,0.466667,0.705882}%
\pgfsetstrokecolor{currentstroke}%
\pgfsetdash{}{0pt}%
\pgfsys@defobject{currentmarker}{\pgfqpoint{-0.041667in}{-0.041667in}}{\pgfqpoint{0.041667in}{0.041667in}}{%
\pgfpathmoveto{\pgfqpoint{0.000000in}{0.041667in}}%
\pgfpathlineto{\pgfqpoint{-0.041667in}{-0.041667in}}%
\pgfpathlineto{\pgfqpoint{0.041667in}{-0.041667in}}%
\pgfpathclose%
\pgfusepath{stroke,fill}%
}%
\begin{pgfscope}%
\pgfsys@transformshift{1.145833in}{0.970417in}%
\pgfsys@useobject{currentmarker}{}%
\end{pgfscope}%
\end{pgfscope}%
\begin{pgfscope}%
\definecolor{textcolor}{rgb}{0.000000,0.000000,0.000000}%
\pgfsetstrokecolor{textcolor}%
\pgfsetfillcolor{textcolor}%
\pgftext[x=1.520833in,y=0.897500in,left,base]{\color{textcolor}\sffamily\fontsize{15.000000}{18.000000}\selectfont \(\displaystyle \delta_\mathrm{Truth}/\delta_\mathrm{1st}\)}%
\end{pgfscope}%
\end{pgfpicture}%
\makeatother%
\endgroup%
}
    \caption{\label{fig:reso-diff} Timing resolution improved by waveform analysis, $\tau$=20ns $\sigma$=10ns}
\end{figure}

Before taking a deep look into the methods, input and output need to be defined. The input for waveform analysis is pedestal subtracted PMT waveform (see figure~\ref{fig:input}), and the output is the $q_{rec}$ sequence (see figure~\ref{fig:output}), which is interpreted as an approximation of time profile. 

\begin{figure}[H]
\begin{minipage}[b]{.5\textwidth}
\begin{figure}[H]
    \centering
    \resizebox{\textwidth}{!}{%% Creator: Matplotlib, PGF backend
%%
%% To include the figure in your LaTeX document, write
%%   \input{<filename>.pgf}
%%
%% Make sure the required packages are loaded in your preamble
%%   \usepackage{pgf}
%%
%% and, on pdftex
%%   \usepackage[utf8]{inputenc}\DeclareUnicodeCharacter{2212}{-}
%%
%% or, on luatex and xetex
%%   \usepackage{unicode-math}
%%
%% Figures using additional raster images can only be included by \input if
%% they are in the same directory as the main LaTeX file. For loading figures
%% from other directories you can use the `import` package
%%   \usepackage{import}
%%
%% and then include the figures with
%%   \import{<path to file>}{<filename>.pgf}
%%
%% Matplotlib used the following preamble
%%   \usepackage[detect-all,locale=DE]{siunitx}
%%
\begingroup%
\makeatletter%
\begin{pgfpicture}%
\pgfpathrectangle{\pgfpointorigin}{\pgfqpoint{8.000000in}{6.000000in}}%
\pgfusepath{use as bounding box, clip}%
\begin{pgfscope}%
\pgfsetbuttcap%
\pgfsetmiterjoin%
\definecolor{currentfill}{rgb}{1.000000,1.000000,1.000000}%
\pgfsetfillcolor{currentfill}%
\pgfsetlinewidth{0.000000pt}%
\definecolor{currentstroke}{rgb}{1.000000,1.000000,1.000000}%
\pgfsetstrokecolor{currentstroke}%
\pgfsetdash{}{0pt}%
\pgfpathmoveto{\pgfqpoint{0.000000in}{0.000000in}}%
\pgfpathlineto{\pgfqpoint{8.000000in}{0.000000in}}%
\pgfpathlineto{\pgfqpoint{8.000000in}{6.000000in}}%
\pgfpathlineto{\pgfqpoint{0.000000in}{6.000000in}}%
\pgfpathclose%
\pgfusepath{fill}%
\end{pgfscope}%
\begin{pgfscope}%
\pgfsetbuttcap%
\pgfsetmiterjoin%
\definecolor{currentfill}{rgb}{1.000000,1.000000,1.000000}%
\pgfsetfillcolor{currentfill}%
\pgfsetlinewidth{0.000000pt}%
\definecolor{currentstroke}{rgb}{0.000000,0.000000,0.000000}%
\pgfsetstrokecolor{currentstroke}%
\pgfsetstrokeopacity{0.000000}%
\pgfsetdash{}{0pt}%
\pgfpathmoveto{\pgfqpoint{1.200000in}{0.900000in}}%
\pgfpathlineto{\pgfqpoint{6.800000in}{0.900000in}}%
\pgfpathlineto{\pgfqpoint{6.800000in}{5.700000in}}%
\pgfpathlineto{\pgfqpoint{1.200000in}{5.700000in}}%
\pgfpathclose%
\pgfusepath{fill}%
\end{pgfscope}%
\begin{pgfscope}%
\pgfpathrectangle{\pgfqpoint{1.200000in}{0.900000in}}{\pgfqpoint{5.600000in}{4.800000in}}%
\pgfusepath{clip}%
\pgfsetrectcap%
\pgfsetroundjoin%
\pgfsetlinewidth{2.007500pt}%
\definecolor{currentstroke}{rgb}{0.000000,0.000000,1.000000}%
\pgfsetstrokecolor{currentstroke}%
\pgfsetdash{}{0pt}%
\pgfpathmoveto{\pgfqpoint{1.200000in}{1.317391in}}%
\pgfpathlineto{\pgfqpoint{1.210884in}{1.317391in}}%
\pgfpathlineto{\pgfqpoint{1.216327in}{1.233913in}}%
\pgfpathlineto{\pgfqpoint{1.221769in}{1.400870in}}%
\pgfpathlineto{\pgfqpoint{1.232653in}{1.400870in}}%
\pgfpathlineto{\pgfqpoint{1.238095in}{1.317391in}}%
\pgfpathlineto{\pgfqpoint{1.243537in}{1.317391in}}%
\pgfpathlineto{\pgfqpoint{1.248980in}{1.150435in}}%
\pgfpathlineto{\pgfqpoint{1.254422in}{1.150435in}}%
\pgfpathlineto{\pgfqpoint{1.265306in}{1.317391in}}%
\pgfpathlineto{\pgfqpoint{1.287075in}{1.317391in}}%
\pgfpathlineto{\pgfqpoint{1.297959in}{1.484348in}}%
\pgfpathlineto{\pgfqpoint{1.303401in}{1.233913in}}%
\pgfpathlineto{\pgfqpoint{1.308844in}{1.400870in}}%
\pgfpathlineto{\pgfqpoint{1.314286in}{1.233913in}}%
\pgfpathlineto{\pgfqpoint{1.319728in}{1.317391in}}%
\pgfpathlineto{\pgfqpoint{1.325170in}{1.317391in}}%
\pgfpathlineto{\pgfqpoint{1.330612in}{1.233913in}}%
\pgfpathlineto{\pgfqpoint{1.336054in}{1.233913in}}%
\pgfpathlineto{\pgfqpoint{1.341497in}{1.317391in}}%
\pgfpathlineto{\pgfqpoint{1.352381in}{1.317391in}}%
\pgfpathlineto{\pgfqpoint{1.357823in}{1.150435in}}%
\pgfpathlineto{\pgfqpoint{1.363265in}{1.150435in}}%
\pgfpathlineto{\pgfqpoint{1.368707in}{1.400870in}}%
\pgfpathlineto{\pgfqpoint{1.379592in}{1.400870in}}%
\pgfpathlineto{\pgfqpoint{1.385034in}{1.233913in}}%
\pgfpathlineto{\pgfqpoint{1.390476in}{1.317391in}}%
\pgfpathlineto{\pgfqpoint{1.395918in}{1.317391in}}%
\pgfpathlineto{\pgfqpoint{1.401361in}{1.400870in}}%
\pgfpathlineto{\pgfqpoint{1.406803in}{1.317391in}}%
\pgfpathlineto{\pgfqpoint{1.412245in}{1.150435in}}%
\pgfpathlineto{\pgfqpoint{1.417687in}{1.400870in}}%
\pgfpathlineto{\pgfqpoint{1.423129in}{1.233913in}}%
\pgfpathlineto{\pgfqpoint{1.428571in}{1.400870in}}%
\pgfpathlineto{\pgfqpoint{1.450340in}{1.400870in}}%
\pgfpathlineto{\pgfqpoint{1.455782in}{1.317391in}}%
\pgfpathlineto{\pgfqpoint{1.461224in}{1.150435in}}%
\pgfpathlineto{\pgfqpoint{1.466667in}{1.484348in}}%
\pgfpathlineto{\pgfqpoint{1.472109in}{1.317391in}}%
\pgfpathlineto{\pgfqpoint{1.499320in}{1.317391in}}%
\pgfpathlineto{\pgfqpoint{1.504762in}{1.400870in}}%
\pgfpathlineto{\pgfqpoint{1.515646in}{1.233913in}}%
\pgfpathlineto{\pgfqpoint{1.521088in}{1.233913in}}%
\pgfpathlineto{\pgfqpoint{1.526531in}{1.400870in}}%
\pgfpathlineto{\pgfqpoint{1.531973in}{1.400870in}}%
\pgfpathlineto{\pgfqpoint{1.537415in}{1.233913in}}%
\pgfpathlineto{\pgfqpoint{1.542857in}{1.233913in}}%
\pgfpathlineto{\pgfqpoint{1.553741in}{1.400870in}}%
\pgfpathlineto{\pgfqpoint{1.564626in}{1.233913in}}%
\pgfpathlineto{\pgfqpoint{1.570068in}{1.233913in}}%
\pgfpathlineto{\pgfqpoint{1.575510in}{1.317391in}}%
\pgfpathlineto{\pgfqpoint{1.580952in}{1.317391in}}%
\pgfpathlineto{\pgfqpoint{1.586395in}{1.233913in}}%
\pgfpathlineto{\pgfqpoint{1.591837in}{1.233913in}}%
\pgfpathlineto{\pgfqpoint{1.597279in}{1.317391in}}%
\pgfpathlineto{\pgfqpoint{1.602721in}{1.317391in}}%
\pgfpathlineto{\pgfqpoint{1.608163in}{1.233913in}}%
\pgfpathlineto{\pgfqpoint{1.613605in}{1.317391in}}%
\pgfpathlineto{\pgfqpoint{1.619048in}{1.317391in}}%
\pgfpathlineto{\pgfqpoint{1.624490in}{1.400870in}}%
\pgfpathlineto{\pgfqpoint{1.629932in}{1.317391in}}%
\pgfpathlineto{\pgfqpoint{1.635374in}{1.317391in}}%
\pgfpathlineto{\pgfqpoint{1.640816in}{1.400870in}}%
\pgfpathlineto{\pgfqpoint{1.646259in}{1.233913in}}%
\pgfpathlineto{\pgfqpoint{1.657143in}{1.233913in}}%
\pgfpathlineto{\pgfqpoint{1.662585in}{1.317391in}}%
\pgfpathlineto{\pgfqpoint{1.668027in}{1.233913in}}%
\pgfpathlineto{\pgfqpoint{1.673469in}{1.233913in}}%
\pgfpathlineto{\pgfqpoint{1.678912in}{1.317391in}}%
\pgfpathlineto{\pgfqpoint{1.684354in}{1.317391in}}%
\pgfpathlineto{\pgfqpoint{1.689796in}{1.400870in}}%
\pgfpathlineto{\pgfqpoint{1.695238in}{1.317391in}}%
\pgfpathlineto{\pgfqpoint{1.700680in}{1.400870in}}%
\pgfpathlineto{\pgfqpoint{1.711565in}{1.400870in}}%
\pgfpathlineto{\pgfqpoint{1.717007in}{1.317391in}}%
\pgfpathlineto{\pgfqpoint{1.722449in}{1.400870in}}%
\pgfpathlineto{\pgfqpoint{1.727891in}{1.233913in}}%
\pgfpathlineto{\pgfqpoint{1.733333in}{1.233913in}}%
\pgfpathlineto{\pgfqpoint{1.738776in}{1.400870in}}%
\pgfpathlineto{\pgfqpoint{1.744218in}{1.317391in}}%
\pgfpathlineto{\pgfqpoint{1.749660in}{1.400870in}}%
\pgfpathlineto{\pgfqpoint{1.755102in}{1.317391in}}%
\pgfpathlineto{\pgfqpoint{1.760544in}{1.317391in}}%
\pgfpathlineto{\pgfqpoint{1.765986in}{1.400870in}}%
\pgfpathlineto{\pgfqpoint{1.771429in}{1.400870in}}%
\pgfpathlineto{\pgfqpoint{1.776871in}{1.233913in}}%
\pgfpathlineto{\pgfqpoint{1.782313in}{1.400870in}}%
\pgfpathlineto{\pgfqpoint{1.787755in}{1.400870in}}%
\pgfpathlineto{\pgfqpoint{1.793197in}{1.317391in}}%
\pgfpathlineto{\pgfqpoint{1.798639in}{1.484348in}}%
\pgfpathlineto{\pgfqpoint{1.804082in}{1.484348in}}%
\pgfpathlineto{\pgfqpoint{1.809524in}{1.317391in}}%
\pgfpathlineto{\pgfqpoint{1.814966in}{1.233913in}}%
\pgfpathlineto{\pgfqpoint{1.820408in}{1.317391in}}%
\pgfpathlineto{\pgfqpoint{1.825850in}{1.484348in}}%
\pgfpathlineto{\pgfqpoint{1.831293in}{1.317391in}}%
\pgfpathlineto{\pgfqpoint{1.836735in}{1.400870in}}%
\pgfpathlineto{\pgfqpoint{1.847619in}{1.233913in}}%
\pgfpathlineto{\pgfqpoint{1.858503in}{1.400870in}}%
\pgfpathlineto{\pgfqpoint{1.863946in}{1.150435in}}%
\pgfpathlineto{\pgfqpoint{1.869388in}{1.317391in}}%
\pgfpathlineto{\pgfqpoint{1.874830in}{1.233913in}}%
\pgfpathlineto{\pgfqpoint{1.891156in}{1.233913in}}%
\pgfpathlineto{\pgfqpoint{1.896599in}{1.317391in}}%
\pgfpathlineto{\pgfqpoint{1.902041in}{1.233913in}}%
\pgfpathlineto{\pgfqpoint{1.907483in}{1.484348in}}%
\pgfpathlineto{\pgfqpoint{1.912925in}{1.317391in}}%
\pgfpathlineto{\pgfqpoint{1.918367in}{1.317391in}}%
\pgfpathlineto{\pgfqpoint{1.923810in}{1.484348in}}%
\pgfpathlineto{\pgfqpoint{1.929252in}{1.317391in}}%
\pgfpathlineto{\pgfqpoint{1.934694in}{1.317391in}}%
\pgfpathlineto{\pgfqpoint{1.945578in}{1.150435in}}%
\pgfpathlineto{\pgfqpoint{1.951020in}{1.400870in}}%
\pgfpathlineto{\pgfqpoint{1.961905in}{1.233913in}}%
\pgfpathlineto{\pgfqpoint{1.967347in}{1.400870in}}%
\pgfpathlineto{\pgfqpoint{1.978231in}{1.233913in}}%
\pgfpathlineto{\pgfqpoint{1.989116in}{1.233913in}}%
\pgfpathlineto{\pgfqpoint{1.994558in}{1.150435in}}%
\pgfpathlineto{\pgfqpoint{2.000000in}{1.233913in}}%
\pgfpathlineto{\pgfqpoint{2.005442in}{1.150435in}}%
\pgfpathlineto{\pgfqpoint{2.010884in}{1.317391in}}%
\pgfpathlineto{\pgfqpoint{2.027211in}{1.317391in}}%
\pgfpathlineto{\pgfqpoint{2.032653in}{1.233913in}}%
\pgfpathlineto{\pgfqpoint{2.038095in}{1.233913in}}%
\pgfpathlineto{\pgfqpoint{2.043537in}{1.400870in}}%
\pgfpathlineto{\pgfqpoint{2.048980in}{1.317391in}}%
\pgfpathlineto{\pgfqpoint{2.054422in}{1.400870in}}%
\pgfpathlineto{\pgfqpoint{2.059864in}{1.233913in}}%
\pgfpathlineto{\pgfqpoint{2.065306in}{1.233913in}}%
\pgfpathlineto{\pgfqpoint{2.070748in}{1.317391in}}%
\pgfpathlineto{\pgfqpoint{2.076190in}{1.317391in}}%
\pgfpathlineto{\pgfqpoint{2.081633in}{1.400870in}}%
\pgfpathlineto{\pgfqpoint{2.087075in}{1.233913in}}%
\pgfpathlineto{\pgfqpoint{2.092517in}{1.484348in}}%
\pgfpathlineto{\pgfqpoint{2.103401in}{1.317391in}}%
\pgfpathlineto{\pgfqpoint{2.108844in}{1.400870in}}%
\pgfpathlineto{\pgfqpoint{2.119728in}{1.233913in}}%
\pgfpathlineto{\pgfqpoint{2.125170in}{1.400870in}}%
\pgfpathlineto{\pgfqpoint{2.130612in}{1.400870in}}%
\pgfpathlineto{\pgfqpoint{2.141497in}{1.233913in}}%
\pgfpathlineto{\pgfqpoint{2.146939in}{1.317391in}}%
\pgfpathlineto{\pgfqpoint{2.152381in}{1.233913in}}%
\pgfpathlineto{\pgfqpoint{2.157823in}{1.317391in}}%
\pgfpathlineto{\pgfqpoint{2.163265in}{1.317391in}}%
\pgfpathlineto{\pgfqpoint{2.168707in}{1.233913in}}%
\pgfpathlineto{\pgfqpoint{2.174150in}{1.317391in}}%
\pgfpathlineto{\pgfqpoint{2.179592in}{1.233913in}}%
\pgfpathlineto{\pgfqpoint{2.185034in}{1.400870in}}%
\pgfpathlineto{\pgfqpoint{2.190476in}{1.400870in}}%
\pgfpathlineto{\pgfqpoint{2.195918in}{1.317391in}}%
\pgfpathlineto{\pgfqpoint{2.206803in}{1.317391in}}%
\pgfpathlineto{\pgfqpoint{2.217687in}{1.484348in}}%
\pgfpathlineto{\pgfqpoint{2.223129in}{1.317391in}}%
\pgfpathlineto{\pgfqpoint{2.228571in}{1.233913in}}%
\pgfpathlineto{\pgfqpoint{2.234014in}{1.400870in}}%
\pgfpathlineto{\pgfqpoint{2.239456in}{1.233913in}}%
\pgfpathlineto{\pgfqpoint{2.244898in}{1.317391in}}%
\pgfpathlineto{\pgfqpoint{2.250340in}{1.317391in}}%
\pgfpathlineto{\pgfqpoint{2.255782in}{1.233913in}}%
\pgfpathlineto{\pgfqpoint{2.261224in}{1.317391in}}%
\pgfpathlineto{\pgfqpoint{2.266667in}{1.233913in}}%
\pgfpathlineto{\pgfqpoint{2.272109in}{1.233913in}}%
\pgfpathlineto{\pgfqpoint{2.282993in}{1.400870in}}%
\pgfpathlineto{\pgfqpoint{2.288435in}{1.317391in}}%
\pgfpathlineto{\pgfqpoint{2.293878in}{1.317391in}}%
\pgfpathlineto{\pgfqpoint{2.299320in}{1.400870in}}%
\pgfpathlineto{\pgfqpoint{2.304762in}{1.233913in}}%
\pgfpathlineto{\pgfqpoint{2.310204in}{1.233913in}}%
\pgfpathlineto{\pgfqpoint{2.315646in}{1.317391in}}%
\pgfpathlineto{\pgfqpoint{2.321088in}{1.317391in}}%
\pgfpathlineto{\pgfqpoint{2.326531in}{1.400870in}}%
\pgfpathlineto{\pgfqpoint{2.337415in}{1.233913in}}%
\pgfpathlineto{\pgfqpoint{2.342857in}{1.400870in}}%
\pgfpathlineto{\pgfqpoint{2.348299in}{1.150435in}}%
\pgfpathlineto{\pgfqpoint{2.353741in}{1.317391in}}%
\pgfpathlineto{\pgfqpoint{2.359184in}{1.400870in}}%
\pgfpathlineto{\pgfqpoint{2.364626in}{1.317391in}}%
\pgfpathlineto{\pgfqpoint{2.370068in}{1.484348in}}%
\pgfpathlineto{\pgfqpoint{2.375510in}{1.317391in}}%
\pgfpathlineto{\pgfqpoint{2.380952in}{1.233913in}}%
\pgfpathlineto{\pgfqpoint{2.386395in}{1.317391in}}%
\pgfpathlineto{\pgfqpoint{2.391837in}{1.317391in}}%
\pgfpathlineto{\pgfqpoint{2.397279in}{1.400870in}}%
\pgfpathlineto{\pgfqpoint{2.402721in}{1.317391in}}%
\pgfpathlineto{\pgfqpoint{2.408163in}{1.317391in}}%
\pgfpathlineto{\pgfqpoint{2.413605in}{1.400870in}}%
\pgfpathlineto{\pgfqpoint{2.419048in}{1.400870in}}%
\pgfpathlineto{\pgfqpoint{2.429932in}{1.233913in}}%
\pgfpathlineto{\pgfqpoint{2.435374in}{1.317391in}}%
\pgfpathlineto{\pgfqpoint{2.440816in}{1.317391in}}%
\pgfpathlineto{\pgfqpoint{2.446259in}{1.400870in}}%
\pgfpathlineto{\pgfqpoint{2.451701in}{1.317391in}}%
\pgfpathlineto{\pgfqpoint{2.457143in}{1.066957in}}%
\pgfpathlineto{\pgfqpoint{2.462585in}{1.317391in}}%
\pgfpathlineto{\pgfqpoint{2.468027in}{1.400870in}}%
\pgfpathlineto{\pgfqpoint{2.473469in}{1.317391in}}%
\pgfpathlineto{\pgfqpoint{2.478912in}{1.400870in}}%
\pgfpathlineto{\pgfqpoint{2.484354in}{1.400870in}}%
\pgfpathlineto{\pgfqpoint{2.489796in}{1.233913in}}%
\pgfpathlineto{\pgfqpoint{2.495238in}{1.150435in}}%
\pgfpathlineto{\pgfqpoint{2.511565in}{1.400870in}}%
\pgfpathlineto{\pgfqpoint{2.522449in}{1.400870in}}%
\pgfpathlineto{\pgfqpoint{2.527891in}{1.233913in}}%
\pgfpathlineto{\pgfqpoint{2.533333in}{1.484348in}}%
\pgfpathlineto{\pgfqpoint{2.538776in}{1.484348in}}%
\pgfpathlineto{\pgfqpoint{2.549660in}{1.317391in}}%
\pgfpathlineto{\pgfqpoint{2.555102in}{1.400870in}}%
\pgfpathlineto{\pgfqpoint{2.560544in}{1.233913in}}%
\pgfpathlineto{\pgfqpoint{2.565986in}{1.317391in}}%
\pgfpathlineto{\pgfqpoint{2.571429in}{1.233913in}}%
\pgfpathlineto{\pgfqpoint{2.576871in}{1.233913in}}%
\pgfpathlineto{\pgfqpoint{2.582313in}{1.150435in}}%
\pgfpathlineto{\pgfqpoint{2.593197in}{1.317391in}}%
\pgfpathlineto{\pgfqpoint{2.604082in}{1.317391in}}%
\pgfpathlineto{\pgfqpoint{2.609524in}{1.400870in}}%
\pgfpathlineto{\pgfqpoint{2.614966in}{1.400870in}}%
\pgfpathlineto{\pgfqpoint{2.620408in}{1.317391in}}%
\pgfpathlineto{\pgfqpoint{2.625850in}{1.150435in}}%
\pgfpathlineto{\pgfqpoint{2.631293in}{1.066957in}}%
\pgfpathlineto{\pgfqpoint{2.636735in}{1.317391in}}%
\pgfpathlineto{\pgfqpoint{2.647619in}{1.317391in}}%
\pgfpathlineto{\pgfqpoint{2.653061in}{1.400870in}}%
\pgfpathlineto{\pgfqpoint{2.658503in}{1.317391in}}%
\pgfpathlineto{\pgfqpoint{2.669388in}{1.484348in}}%
\pgfpathlineto{\pgfqpoint{2.674830in}{1.400870in}}%
\pgfpathlineto{\pgfqpoint{2.680272in}{1.233913in}}%
\pgfpathlineto{\pgfqpoint{2.685714in}{1.317391in}}%
\pgfpathlineto{\pgfqpoint{2.691156in}{1.233913in}}%
\pgfpathlineto{\pgfqpoint{2.696599in}{1.400870in}}%
\pgfpathlineto{\pgfqpoint{2.702041in}{1.484348in}}%
\pgfpathlineto{\pgfqpoint{2.707483in}{1.317391in}}%
\pgfpathlineto{\pgfqpoint{2.712925in}{1.400870in}}%
\pgfpathlineto{\pgfqpoint{2.718367in}{1.233913in}}%
\pgfpathlineto{\pgfqpoint{2.729252in}{1.400870in}}%
\pgfpathlineto{\pgfqpoint{2.734694in}{1.317391in}}%
\pgfpathlineto{\pgfqpoint{2.740136in}{1.317391in}}%
\pgfpathlineto{\pgfqpoint{2.745578in}{1.400870in}}%
\pgfpathlineto{\pgfqpoint{2.751020in}{1.317391in}}%
\pgfpathlineto{\pgfqpoint{2.756463in}{1.317391in}}%
\pgfpathlineto{\pgfqpoint{2.761905in}{1.233913in}}%
\pgfpathlineto{\pgfqpoint{2.767347in}{1.233913in}}%
\pgfpathlineto{\pgfqpoint{2.772789in}{1.317391in}}%
\pgfpathlineto{\pgfqpoint{2.778231in}{1.233913in}}%
\pgfpathlineto{\pgfqpoint{2.783673in}{1.233913in}}%
\pgfpathlineto{\pgfqpoint{2.794558in}{1.400870in}}%
\pgfpathlineto{\pgfqpoint{2.805442in}{1.400870in}}%
\pgfpathlineto{\pgfqpoint{2.810884in}{1.150435in}}%
\pgfpathlineto{\pgfqpoint{2.821769in}{1.484348in}}%
\pgfpathlineto{\pgfqpoint{2.827211in}{2.068696in}}%
\pgfpathlineto{\pgfqpoint{2.832653in}{2.820000in}}%
\pgfpathlineto{\pgfqpoint{2.838095in}{3.821739in}}%
\pgfpathlineto{\pgfqpoint{2.843537in}{4.573043in}}%
\pgfpathlineto{\pgfqpoint{2.848980in}{5.073913in}}%
\pgfpathlineto{\pgfqpoint{2.854422in}{5.240870in}}%
\pgfpathlineto{\pgfqpoint{2.859864in}{5.157391in}}%
\pgfpathlineto{\pgfqpoint{2.876190in}{3.988696in}}%
\pgfpathlineto{\pgfqpoint{2.881633in}{3.654783in}}%
\pgfpathlineto{\pgfqpoint{2.887075in}{3.487826in}}%
\pgfpathlineto{\pgfqpoint{2.892517in}{3.070435in}}%
\pgfpathlineto{\pgfqpoint{2.897959in}{2.820000in}}%
\pgfpathlineto{\pgfqpoint{2.903401in}{2.986957in}}%
\pgfpathlineto{\pgfqpoint{2.914286in}{3.153913in}}%
\pgfpathlineto{\pgfqpoint{2.919728in}{3.153913in}}%
\pgfpathlineto{\pgfqpoint{2.925170in}{3.070435in}}%
\pgfpathlineto{\pgfqpoint{2.930612in}{2.903478in}}%
\pgfpathlineto{\pgfqpoint{2.936054in}{2.820000in}}%
\pgfpathlineto{\pgfqpoint{2.941497in}{2.569565in}}%
\pgfpathlineto{\pgfqpoint{2.946939in}{2.486087in}}%
\pgfpathlineto{\pgfqpoint{2.952381in}{2.235652in}}%
\pgfpathlineto{\pgfqpoint{2.963265in}{2.235652in}}%
\pgfpathlineto{\pgfqpoint{2.968707in}{2.736522in}}%
\pgfpathlineto{\pgfqpoint{2.974150in}{2.736522in}}%
\pgfpathlineto{\pgfqpoint{2.979592in}{2.903478in}}%
\pgfpathlineto{\pgfqpoint{2.985034in}{2.986957in}}%
\pgfpathlineto{\pgfqpoint{2.990476in}{2.820000in}}%
\pgfpathlineto{\pgfqpoint{2.995918in}{2.736522in}}%
\pgfpathlineto{\pgfqpoint{3.001361in}{2.569565in}}%
\pgfpathlineto{\pgfqpoint{3.006803in}{2.569565in}}%
\pgfpathlineto{\pgfqpoint{3.012245in}{2.319130in}}%
\pgfpathlineto{\pgfqpoint{3.017687in}{2.319130in}}%
\pgfpathlineto{\pgfqpoint{3.023129in}{1.985217in}}%
\pgfpathlineto{\pgfqpoint{3.028571in}{2.235652in}}%
\pgfpathlineto{\pgfqpoint{3.034014in}{2.319130in}}%
\pgfpathlineto{\pgfqpoint{3.039456in}{2.486087in}}%
\pgfpathlineto{\pgfqpoint{3.044898in}{2.569565in}}%
\pgfpathlineto{\pgfqpoint{3.050340in}{2.820000in}}%
\pgfpathlineto{\pgfqpoint{3.055782in}{2.736522in}}%
\pgfpathlineto{\pgfqpoint{3.066667in}{2.736522in}}%
\pgfpathlineto{\pgfqpoint{3.077551in}{2.402609in}}%
\pgfpathlineto{\pgfqpoint{3.082993in}{2.152174in}}%
\pgfpathlineto{\pgfqpoint{3.088435in}{2.235652in}}%
\pgfpathlineto{\pgfqpoint{3.093878in}{2.569565in}}%
\pgfpathlineto{\pgfqpoint{3.104762in}{2.903478in}}%
\pgfpathlineto{\pgfqpoint{3.110204in}{2.903478in}}%
\pgfpathlineto{\pgfqpoint{3.115646in}{2.986957in}}%
\pgfpathlineto{\pgfqpoint{3.121088in}{2.820000in}}%
\pgfpathlineto{\pgfqpoint{3.126531in}{2.736522in}}%
\pgfpathlineto{\pgfqpoint{3.131973in}{2.486087in}}%
\pgfpathlineto{\pgfqpoint{3.137415in}{2.569565in}}%
\pgfpathlineto{\pgfqpoint{3.142857in}{2.319130in}}%
\pgfpathlineto{\pgfqpoint{3.148299in}{2.152174in}}%
\pgfpathlineto{\pgfqpoint{3.159184in}{1.985217in}}%
\pgfpathlineto{\pgfqpoint{3.164626in}{1.818261in}}%
\pgfpathlineto{\pgfqpoint{3.170068in}{1.734783in}}%
\pgfpathlineto{\pgfqpoint{3.175510in}{2.152174in}}%
\pgfpathlineto{\pgfqpoint{3.180952in}{2.402609in}}%
\pgfpathlineto{\pgfqpoint{3.186395in}{2.319130in}}%
\pgfpathlineto{\pgfqpoint{3.191837in}{2.569565in}}%
\pgfpathlineto{\pgfqpoint{3.197279in}{2.653043in}}%
\pgfpathlineto{\pgfqpoint{3.202721in}{2.569565in}}%
\pgfpathlineto{\pgfqpoint{3.208163in}{2.235652in}}%
\pgfpathlineto{\pgfqpoint{3.213605in}{2.235652in}}%
\pgfpathlineto{\pgfqpoint{3.219048in}{2.068696in}}%
\pgfpathlineto{\pgfqpoint{3.224490in}{2.068696in}}%
\pgfpathlineto{\pgfqpoint{3.229932in}{1.901739in}}%
\pgfpathlineto{\pgfqpoint{3.235374in}{1.901739in}}%
\pgfpathlineto{\pgfqpoint{3.246259in}{1.567826in}}%
\pgfpathlineto{\pgfqpoint{3.251701in}{1.567826in}}%
\pgfpathlineto{\pgfqpoint{3.257143in}{1.734783in}}%
\pgfpathlineto{\pgfqpoint{3.262585in}{1.651304in}}%
\pgfpathlineto{\pgfqpoint{3.268027in}{1.484348in}}%
\pgfpathlineto{\pgfqpoint{3.273469in}{1.567826in}}%
\pgfpathlineto{\pgfqpoint{3.278912in}{1.400870in}}%
\pgfpathlineto{\pgfqpoint{3.284354in}{1.484348in}}%
\pgfpathlineto{\pgfqpoint{3.289796in}{1.400870in}}%
\pgfpathlineto{\pgfqpoint{3.295238in}{1.484348in}}%
\pgfpathlineto{\pgfqpoint{3.300680in}{1.317391in}}%
\pgfpathlineto{\pgfqpoint{3.311565in}{1.317391in}}%
\pgfpathlineto{\pgfqpoint{3.317007in}{1.150435in}}%
\pgfpathlineto{\pgfqpoint{3.322449in}{1.484348in}}%
\pgfpathlineto{\pgfqpoint{3.327891in}{1.317391in}}%
\pgfpathlineto{\pgfqpoint{3.333333in}{1.400870in}}%
\pgfpathlineto{\pgfqpoint{3.338776in}{1.317391in}}%
\pgfpathlineto{\pgfqpoint{3.344218in}{1.400870in}}%
\pgfpathlineto{\pgfqpoint{3.355102in}{1.400870in}}%
\pgfpathlineto{\pgfqpoint{3.360544in}{1.317391in}}%
\pgfpathlineto{\pgfqpoint{3.365986in}{1.484348in}}%
\pgfpathlineto{\pgfqpoint{3.371429in}{1.233913in}}%
\pgfpathlineto{\pgfqpoint{3.376871in}{1.317391in}}%
\pgfpathlineto{\pgfqpoint{3.382313in}{1.150435in}}%
\pgfpathlineto{\pgfqpoint{3.387755in}{1.484348in}}%
\pgfpathlineto{\pgfqpoint{3.393197in}{1.400870in}}%
\pgfpathlineto{\pgfqpoint{3.404082in}{1.400870in}}%
\pgfpathlineto{\pgfqpoint{3.409524in}{1.317391in}}%
\pgfpathlineto{\pgfqpoint{3.420408in}{1.317391in}}%
\pgfpathlineto{\pgfqpoint{3.425850in}{1.484348in}}%
\pgfpathlineto{\pgfqpoint{3.436735in}{1.484348in}}%
\pgfpathlineto{\pgfqpoint{3.442177in}{1.317391in}}%
\pgfpathlineto{\pgfqpoint{3.447619in}{1.400870in}}%
\pgfpathlineto{\pgfqpoint{3.458503in}{1.400870in}}%
\pgfpathlineto{\pgfqpoint{3.463946in}{1.317391in}}%
\pgfpathlineto{\pgfqpoint{3.469388in}{1.400870in}}%
\pgfpathlineto{\pgfqpoint{3.480272in}{1.400870in}}%
\pgfpathlineto{\pgfqpoint{3.485714in}{1.567826in}}%
\pgfpathlineto{\pgfqpoint{3.491156in}{1.317391in}}%
\pgfpathlineto{\pgfqpoint{3.496599in}{1.150435in}}%
\pgfpathlineto{\pgfqpoint{3.502041in}{1.400870in}}%
\pgfpathlineto{\pgfqpoint{3.507483in}{1.233913in}}%
\pgfpathlineto{\pgfqpoint{3.512925in}{1.233913in}}%
\pgfpathlineto{\pgfqpoint{3.518367in}{1.150435in}}%
\pgfpathlineto{\pgfqpoint{3.523810in}{1.150435in}}%
\pgfpathlineto{\pgfqpoint{3.529252in}{1.233913in}}%
\pgfpathlineto{\pgfqpoint{3.534694in}{1.400870in}}%
\pgfpathlineto{\pgfqpoint{3.540136in}{1.317391in}}%
\pgfpathlineto{\pgfqpoint{3.545578in}{1.484348in}}%
\pgfpathlineto{\pgfqpoint{3.551020in}{1.233913in}}%
\pgfpathlineto{\pgfqpoint{3.556463in}{1.233913in}}%
\pgfpathlineto{\pgfqpoint{3.567347in}{1.400870in}}%
\pgfpathlineto{\pgfqpoint{3.572789in}{1.400870in}}%
\pgfpathlineto{\pgfqpoint{3.578231in}{1.150435in}}%
\pgfpathlineto{\pgfqpoint{3.583673in}{1.233913in}}%
\pgfpathlineto{\pgfqpoint{3.589116in}{1.233913in}}%
\pgfpathlineto{\pgfqpoint{3.594558in}{1.400870in}}%
\pgfpathlineto{\pgfqpoint{3.600000in}{1.233913in}}%
\pgfpathlineto{\pgfqpoint{3.605442in}{1.233913in}}%
\pgfpathlineto{\pgfqpoint{3.610884in}{1.317391in}}%
\pgfpathlineto{\pgfqpoint{3.616327in}{1.484348in}}%
\pgfpathlineto{\pgfqpoint{3.621769in}{1.400870in}}%
\pgfpathlineto{\pgfqpoint{3.627211in}{1.400870in}}%
\pgfpathlineto{\pgfqpoint{3.638095in}{1.233913in}}%
\pgfpathlineto{\pgfqpoint{3.643537in}{1.317391in}}%
\pgfpathlineto{\pgfqpoint{3.654422in}{1.317391in}}%
\pgfpathlineto{\pgfqpoint{3.665306in}{1.484348in}}%
\pgfpathlineto{\pgfqpoint{3.670748in}{1.317391in}}%
\pgfpathlineto{\pgfqpoint{3.676190in}{1.233913in}}%
\pgfpathlineto{\pgfqpoint{3.681633in}{1.317391in}}%
\pgfpathlineto{\pgfqpoint{3.687075in}{1.233913in}}%
\pgfpathlineto{\pgfqpoint{3.692517in}{1.317391in}}%
\pgfpathlineto{\pgfqpoint{3.697959in}{1.484348in}}%
\pgfpathlineto{\pgfqpoint{3.703401in}{1.400870in}}%
\pgfpathlineto{\pgfqpoint{3.708844in}{1.400870in}}%
\pgfpathlineto{\pgfqpoint{3.714286in}{1.233913in}}%
\pgfpathlineto{\pgfqpoint{3.719728in}{1.484348in}}%
\pgfpathlineto{\pgfqpoint{3.730612in}{1.317391in}}%
\pgfpathlineto{\pgfqpoint{3.741497in}{1.317391in}}%
\pgfpathlineto{\pgfqpoint{3.746939in}{1.484348in}}%
\pgfpathlineto{\pgfqpoint{3.752381in}{1.317391in}}%
\pgfpathlineto{\pgfqpoint{3.757823in}{1.400870in}}%
\pgfpathlineto{\pgfqpoint{3.768707in}{1.400870in}}%
\pgfpathlineto{\pgfqpoint{3.774150in}{1.233913in}}%
\pgfpathlineto{\pgfqpoint{3.779592in}{1.317391in}}%
\pgfpathlineto{\pgfqpoint{3.790476in}{1.317391in}}%
\pgfpathlineto{\pgfqpoint{3.795918in}{1.400870in}}%
\pgfpathlineto{\pgfqpoint{3.801361in}{1.233913in}}%
\pgfpathlineto{\pgfqpoint{3.806803in}{1.400870in}}%
\pgfpathlineto{\pgfqpoint{3.812245in}{1.317391in}}%
\pgfpathlineto{\pgfqpoint{3.823129in}{1.317391in}}%
\pgfpathlineto{\pgfqpoint{3.828571in}{1.233913in}}%
\pgfpathlineto{\pgfqpoint{3.834014in}{1.400870in}}%
\pgfpathlineto{\pgfqpoint{3.839456in}{1.233913in}}%
\pgfpathlineto{\pgfqpoint{3.844898in}{1.150435in}}%
\pgfpathlineto{\pgfqpoint{3.850340in}{1.317391in}}%
\pgfpathlineto{\pgfqpoint{3.861224in}{1.317391in}}%
\pgfpathlineto{\pgfqpoint{3.866667in}{1.150435in}}%
\pgfpathlineto{\pgfqpoint{3.872109in}{1.400870in}}%
\pgfpathlineto{\pgfqpoint{3.877551in}{1.317391in}}%
\pgfpathlineto{\pgfqpoint{3.882993in}{1.150435in}}%
\pgfpathlineto{\pgfqpoint{3.888435in}{1.317391in}}%
\pgfpathlineto{\pgfqpoint{3.893878in}{1.400870in}}%
\pgfpathlineto{\pgfqpoint{3.904762in}{1.400870in}}%
\pgfpathlineto{\pgfqpoint{3.910204in}{1.317391in}}%
\pgfpathlineto{\pgfqpoint{3.915646in}{1.317391in}}%
\pgfpathlineto{\pgfqpoint{3.921088in}{1.400870in}}%
\pgfpathlineto{\pgfqpoint{3.931973in}{1.400870in}}%
\pgfpathlineto{\pgfqpoint{3.937415in}{1.233913in}}%
\pgfpathlineto{\pgfqpoint{3.942857in}{1.484348in}}%
\pgfpathlineto{\pgfqpoint{3.948299in}{1.150435in}}%
\pgfpathlineto{\pgfqpoint{3.953741in}{1.317391in}}%
\pgfpathlineto{\pgfqpoint{3.959184in}{1.400870in}}%
\pgfpathlineto{\pgfqpoint{3.964626in}{1.150435in}}%
\pgfpathlineto{\pgfqpoint{3.970068in}{1.233913in}}%
\pgfpathlineto{\pgfqpoint{3.975510in}{1.233913in}}%
\pgfpathlineto{\pgfqpoint{3.980952in}{1.400870in}}%
\pgfpathlineto{\pgfqpoint{3.986395in}{1.317391in}}%
\pgfpathlineto{\pgfqpoint{3.991837in}{1.400870in}}%
\pgfpathlineto{\pgfqpoint{3.997279in}{1.400870in}}%
\pgfpathlineto{\pgfqpoint{4.008163in}{1.233913in}}%
\pgfpathlineto{\pgfqpoint{4.013605in}{1.400870in}}%
\pgfpathlineto{\pgfqpoint{4.019048in}{1.317391in}}%
\pgfpathlineto{\pgfqpoint{4.024490in}{1.317391in}}%
\pgfpathlineto{\pgfqpoint{4.029932in}{1.484348in}}%
\pgfpathlineto{\pgfqpoint{4.035374in}{1.233913in}}%
\pgfpathlineto{\pgfqpoint{4.046259in}{1.400870in}}%
\pgfpathlineto{\pgfqpoint{4.051701in}{1.233913in}}%
\pgfpathlineto{\pgfqpoint{4.057143in}{1.317391in}}%
\pgfpathlineto{\pgfqpoint{4.062585in}{1.233913in}}%
\pgfpathlineto{\pgfqpoint{4.068027in}{1.484348in}}%
\pgfpathlineto{\pgfqpoint{4.073469in}{1.317391in}}%
\pgfpathlineto{\pgfqpoint{4.078912in}{1.233913in}}%
\pgfpathlineto{\pgfqpoint{4.089796in}{1.233913in}}%
\pgfpathlineto{\pgfqpoint{4.095238in}{1.400870in}}%
\pgfpathlineto{\pgfqpoint{4.100680in}{1.400870in}}%
\pgfpathlineto{\pgfqpoint{4.106122in}{1.233913in}}%
\pgfpathlineto{\pgfqpoint{4.111565in}{1.400870in}}%
\pgfpathlineto{\pgfqpoint{4.117007in}{1.233913in}}%
\pgfpathlineto{\pgfqpoint{4.122449in}{1.233913in}}%
\pgfpathlineto{\pgfqpoint{4.127891in}{1.400870in}}%
\pgfpathlineto{\pgfqpoint{4.138776in}{1.233913in}}%
\pgfpathlineto{\pgfqpoint{4.144218in}{1.233913in}}%
\pgfpathlineto{\pgfqpoint{4.149660in}{1.317391in}}%
\pgfpathlineto{\pgfqpoint{4.160544in}{1.317391in}}%
\pgfpathlineto{\pgfqpoint{4.165986in}{1.400870in}}%
\pgfpathlineto{\pgfqpoint{4.171429in}{1.066957in}}%
\pgfpathlineto{\pgfqpoint{4.176871in}{1.400870in}}%
\pgfpathlineto{\pgfqpoint{4.182313in}{1.233913in}}%
\pgfpathlineto{\pgfqpoint{4.187755in}{1.484348in}}%
\pgfpathlineto{\pgfqpoint{4.193197in}{1.317391in}}%
\pgfpathlineto{\pgfqpoint{4.198639in}{1.317391in}}%
\pgfpathlineto{\pgfqpoint{4.204082in}{1.150435in}}%
\pgfpathlineto{\pgfqpoint{4.209524in}{1.400870in}}%
\pgfpathlineto{\pgfqpoint{4.214966in}{1.317391in}}%
\pgfpathlineto{\pgfqpoint{4.220408in}{1.400870in}}%
\pgfpathlineto{\pgfqpoint{4.231293in}{1.400870in}}%
\pgfpathlineto{\pgfqpoint{4.236735in}{1.317391in}}%
\pgfpathlineto{\pgfqpoint{4.242177in}{1.317391in}}%
\pgfpathlineto{\pgfqpoint{4.253061in}{1.484348in}}%
\pgfpathlineto{\pgfqpoint{4.258503in}{1.400870in}}%
\pgfpathlineto{\pgfqpoint{4.263946in}{1.400870in}}%
\pgfpathlineto{\pgfqpoint{4.269388in}{1.150435in}}%
\pgfpathlineto{\pgfqpoint{4.280272in}{1.317391in}}%
\pgfpathlineto{\pgfqpoint{4.285714in}{1.233913in}}%
\pgfpathlineto{\pgfqpoint{4.291156in}{1.233913in}}%
\pgfpathlineto{\pgfqpoint{4.296599in}{1.400870in}}%
\pgfpathlineto{\pgfqpoint{4.302041in}{1.233913in}}%
\pgfpathlineto{\pgfqpoint{4.307483in}{1.233913in}}%
\pgfpathlineto{\pgfqpoint{4.312925in}{1.400870in}}%
\pgfpathlineto{\pgfqpoint{4.318367in}{1.233913in}}%
\pgfpathlineto{\pgfqpoint{4.323810in}{1.317391in}}%
\pgfpathlineto{\pgfqpoint{4.329252in}{1.317391in}}%
\pgfpathlineto{\pgfqpoint{4.334694in}{1.233913in}}%
\pgfpathlineto{\pgfqpoint{4.340136in}{1.233913in}}%
\pgfpathlineto{\pgfqpoint{4.351020in}{1.400870in}}%
\pgfpathlineto{\pgfqpoint{4.356463in}{1.400870in}}%
\pgfpathlineto{\pgfqpoint{4.361905in}{1.150435in}}%
\pgfpathlineto{\pgfqpoint{4.372789in}{1.150435in}}%
\pgfpathlineto{\pgfqpoint{4.378231in}{1.317391in}}%
\pgfpathlineto{\pgfqpoint{4.383673in}{1.317391in}}%
\pgfpathlineto{\pgfqpoint{4.394558in}{1.484348in}}%
\pgfpathlineto{\pgfqpoint{4.400000in}{1.484348in}}%
\pgfpathlineto{\pgfqpoint{4.405442in}{1.317391in}}%
\pgfpathlineto{\pgfqpoint{4.410884in}{1.317391in}}%
\pgfpathlineto{\pgfqpoint{4.421769in}{1.150435in}}%
\pgfpathlineto{\pgfqpoint{4.427211in}{1.233913in}}%
\pgfpathlineto{\pgfqpoint{4.432653in}{1.484348in}}%
\pgfpathlineto{\pgfqpoint{4.438095in}{1.233913in}}%
\pgfpathlineto{\pgfqpoint{4.443537in}{1.233913in}}%
\pgfpathlineto{\pgfqpoint{4.448980in}{1.317391in}}%
\pgfpathlineto{\pgfqpoint{4.459864in}{1.317391in}}%
\pgfpathlineto{\pgfqpoint{4.465306in}{1.233913in}}%
\pgfpathlineto{\pgfqpoint{4.470748in}{1.317391in}}%
\pgfpathlineto{\pgfqpoint{4.476190in}{1.317391in}}%
\pgfpathlineto{\pgfqpoint{4.481633in}{1.233913in}}%
\pgfpathlineto{\pgfqpoint{4.487075in}{1.317391in}}%
\pgfpathlineto{\pgfqpoint{4.492517in}{1.233913in}}%
\pgfpathlineto{\pgfqpoint{4.497959in}{1.317391in}}%
\pgfpathlineto{\pgfqpoint{4.503401in}{1.317391in}}%
\pgfpathlineto{\pgfqpoint{4.508844in}{1.400870in}}%
\pgfpathlineto{\pgfqpoint{4.514286in}{1.317391in}}%
\pgfpathlineto{\pgfqpoint{4.519728in}{1.317391in}}%
\pgfpathlineto{\pgfqpoint{4.525170in}{1.150435in}}%
\pgfpathlineto{\pgfqpoint{4.530612in}{1.400870in}}%
\pgfpathlineto{\pgfqpoint{4.536054in}{1.400870in}}%
\pgfpathlineto{\pgfqpoint{4.541497in}{1.317391in}}%
\pgfpathlineto{\pgfqpoint{4.546939in}{1.317391in}}%
\pgfpathlineto{\pgfqpoint{4.552381in}{1.233913in}}%
\pgfpathlineto{\pgfqpoint{4.557823in}{1.233913in}}%
\pgfpathlineto{\pgfqpoint{4.563265in}{1.317391in}}%
\pgfpathlineto{\pgfqpoint{4.568707in}{1.484348in}}%
\pgfpathlineto{\pgfqpoint{4.574150in}{1.400870in}}%
\pgfpathlineto{\pgfqpoint{4.579592in}{1.400870in}}%
\pgfpathlineto{\pgfqpoint{4.585034in}{1.233913in}}%
\pgfpathlineto{\pgfqpoint{4.590476in}{1.317391in}}%
\pgfpathlineto{\pgfqpoint{4.595918in}{1.317391in}}%
\pgfpathlineto{\pgfqpoint{4.601361in}{1.150435in}}%
\pgfpathlineto{\pgfqpoint{4.606803in}{1.400870in}}%
\pgfpathlineto{\pgfqpoint{4.612245in}{1.400870in}}%
\pgfpathlineto{\pgfqpoint{4.617687in}{1.150435in}}%
\pgfpathlineto{\pgfqpoint{4.623129in}{1.317391in}}%
\pgfpathlineto{\pgfqpoint{4.628571in}{1.233913in}}%
\pgfpathlineto{\pgfqpoint{4.634014in}{1.066957in}}%
\pgfpathlineto{\pgfqpoint{4.639456in}{1.567826in}}%
\pgfpathlineto{\pgfqpoint{4.650340in}{1.233913in}}%
\pgfpathlineto{\pgfqpoint{4.655782in}{1.317391in}}%
\pgfpathlineto{\pgfqpoint{4.661224in}{1.317391in}}%
\pgfpathlineto{\pgfqpoint{4.666667in}{1.400870in}}%
\pgfpathlineto{\pgfqpoint{4.672109in}{1.400870in}}%
\pgfpathlineto{\pgfqpoint{4.677551in}{1.484348in}}%
\pgfpathlineto{\pgfqpoint{4.682993in}{1.317391in}}%
\pgfpathlineto{\pgfqpoint{4.688435in}{1.484348in}}%
\pgfpathlineto{\pgfqpoint{4.693878in}{1.400870in}}%
\pgfpathlineto{\pgfqpoint{4.699320in}{1.233913in}}%
\pgfpathlineto{\pgfqpoint{4.704762in}{1.317391in}}%
\pgfpathlineto{\pgfqpoint{4.715646in}{1.317391in}}%
\pgfpathlineto{\pgfqpoint{4.721088in}{1.233913in}}%
\pgfpathlineto{\pgfqpoint{4.726531in}{1.400870in}}%
\pgfpathlineto{\pgfqpoint{4.731973in}{1.317391in}}%
\pgfpathlineto{\pgfqpoint{4.737415in}{1.317391in}}%
\pgfpathlineto{\pgfqpoint{4.742857in}{1.233913in}}%
\pgfpathlineto{\pgfqpoint{4.748299in}{1.233913in}}%
\pgfpathlineto{\pgfqpoint{4.753741in}{1.400870in}}%
\pgfpathlineto{\pgfqpoint{4.759184in}{1.317391in}}%
\pgfpathlineto{\pgfqpoint{4.775510in}{1.317391in}}%
\pgfpathlineto{\pgfqpoint{4.780952in}{1.400870in}}%
\pgfpathlineto{\pgfqpoint{4.786395in}{1.233913in}}%
\pgfpathlineto{\pgfqpoint{4.791837in}{1.400870in}}%
\pgfpathlineto{\pgfqpoint{4.797279in}{1.317391in}}%
\pgfpathlineto{\pgfqpoint{4.808163in}{1.317391in}}%
\pgfpathlineto{\pgfqpoint{4.813605in}{1.484348in}}%
\pgfpathlineto{\pgfqpoint{4.819048in}{1.317391in}}%
\pgfpathlineto{\pgfqpoint{4.824490in}{1.400870in}}%
\pgfpathlineto{\pgfqpoint{4.835374in}{1.400870in}}%
\pgfpathlineto{\pgfqpoint{4.840816in}{1.317391in}}%
\pgfpathlineto{\pgfqpoint{4.851701in}{1.317391in}}%
\pgfpathlineto{\pgfqpoint{4.857143in}{1.233913in}}%
\pgfpathlineto{\pgfqpoint{4.862585in}{1.317391in}}%
\pgfpathlineto{\pgfqpoint{4.868027in}{1.233913in}}%
\pgfpathlineto{\pgfqpoint{4.873469in}{1.400870in}}%
\pgfpathlineto{\pgfqpoint{4.878912in}{1.400870in}}%
\pgfpathlineto{\pgfqpoint{4.884354in}{1.484348in}}%
\pgfpathlineto{\pgfqpoint{4.889796in}{1.233913in}}%
\pgfpathlineto{\pgfqpoint{4.895238in}{1.400870in}}%
\pgfpathlineto{\pgfqpoint{4.906122in}{1.233913in}}%
\pgfpathlineto{\pgfqpoint{4.911565in}{1.400870in}}%
\pgfpathlineto{\pgfqpoint{4.917007in}{1.233913in}}%
\pgfpathlineto{\pgfqpoint{4.922449in}{1.233913in}}%
\pgfpathlineto{\pgfqpoint{4.927891in}{1.150435in}}%
\pgfpathlineto{\pgfqpoint{4.933333in}{1.400870in}}%
\pgfpathlineto{\pgfqpoint{4.938776in}{1.317391in}}%
\pgfpathlineto{\pgfqpoint{4.949660in}{1.317391in}}%
\pgfpathlineto{\pgfqpoint{4.955102in}{1.567826in}}%
\pgfpathlineto{\pgfqpoint{4.960544in}{1.317391in}}%
\pgfpathlineto{\pgfqpoint{4.965986in}{1.233913in}}%
\pgfpathlineto{\pgfqpoint{4.971429in}{1.484348in}}%
\pgfpathlineto{\pgfqpoint{4.976871in}{1.317391in}}%
\pgfpathlineto{\pgfqpoint{4.982313in}{1.233913in}}%
\pgfpathlineto{\pgfqpoint{4.987755in}{1.233913in}}%
\pgfpathlineto{\pgfqpoint{4.993197in}{1.150435in}}%
\pgfpathlineto{\pgfqpoint{4.998639in}{1.317391in}}%
\pgfpathlineto{\pgfqpoint{5.004082in}{1.233913in}}%
\pgfpathlineto{\pgfqpoint{5.014966in}{1.400870in}}%
\pgfpathlineto{\pgfqpoint{5.020408in}{1.317391in}}%
\pgfpathlineto{\pgfqpoint{5.025850in}{1.317391in}}%
\pgfpathlineto{\pgfqpoint{5.031293in}{1.484348in}}%
\pgfpathlineto{\pgfqpoint{5.036735in}{1.233913in}}%
\pgfpathlineto{\pgfqpoint{5.042177in}{1.233913in}}%
\pgfpathlineto{\pgfqpoint{5.047619in}{1.400870in}}%
\pgfpathlineto{\pgfqpoint{5.053061in}{1.233913in}}%
\pgfpathlineto{\pgfqpoint{5.058503in}{1.400870in}}%
\pgfpathlineto{\pgfqpoint{5.063946in}{1.317391in}}%
\pgfpathlineto{\pgfqpoint{5.069388in}{1.150435in}}%
\pgfpathlineto{\pgfqpoint{5.074830in}{1.317391in}}%
\pgfpathlineto{\pgfqpoint{5.080272in}{1.233913in}}%
\pgfpathlineto{\pgfqpoint{5.085714in}{1.233913in}}%
\pgfpathlineto{\pgfqpoint{5.091156in}{1.317391in}}%
\pgfpathlineto{\pgfqpoint{5.096599in}{1.233913in}}%
\pgfpathlineto{\pgfqpoint{5.102041in}{1.233913in}}%
\pgfpathlineto{\pgfqpoint{5.107483in}{1.651304in}}%
\pgfpathlineto{\pgfqpoint{5.112925in}{1.400870in}}%
\pgfpathlineto{\pgfqpoint{5.118367in}{1.400870in}}%
\pgfpathlineto{\pgfqpoint{5.123810in}{1.317391in}}%
\pgfpathlineto{\pgfqpoint{5.129252in}{1.317391in}}%
\pgfpathlineto{\pgfqpoint{5.134694in}{1.400870in}}%
\pgfpathlineto{\pgfqpoint{5.140136in}{1.400870in}}%
\pgfpathlineto{\pgfqpoint{5.145578in}{1.233913in}}%
\pgfpathlineto{\pgfqpoint{5.151020in}{1.150435in}}%
\pgfpathlineto{\pgfqpoint{5.156463in}{1.400870in}}%
\pgfpathlineto{\pgfqpoint{5.161905in}{1.484348in}}%
\pgfpathlineto{\pgfqpoint{5.167347in}{1.233913in}}%
\pgfpathlineto{\pgfqpoint{5.189116in}{1.233913in}}%
\pgfpathlineto{\pgfqpoint{5.194558in}{1.317391in}}%
\pgfpathlineto{\pgfqpoint{5.205442in}{1.150435in}}%
\pgfpathlineto{\pgfqpoint{5.210884in}{1.317391in}}%
\pgfpathlineto{\pgfqpoint{5.216327in}{1.317391in}}%
\pgfpathlineto{\pgfqpoint{5.227211in}{1.150435in}}%
\pgfpathlineto{\pgfqpoint{5.232653in}{1.317391in}}%
\pgfpathlineto{\pgfqpoint{5.238095in}{1.400870in}}%
\pgfpathlineto{\pgfqpoint{5.243537in}{1.400870in}}%
\pgfpathlineto{\pgfqpoint{5.248980in}{1.317391in}}%
\pgfpathlineto{\pgfqpoint{5.254422in}{1.317391in}}%
\pgfpathlineto{\pgfqpoint{5.259864in}{1.233913in}}%
\pgfpathlineto{\pgfqpoint{5.265306in}{1.484348in}}%
\pgfpathlineto{\pgfqpoint{5.270748in}{1.317391in}}%
\pgfpathlineto{\pgfqpoint{5.276190in}{1.484348in}}%
\pgfpathlineto{\pgfqpoint{5.281633in}{1.233913in}}%
\pgfpathlineto{\pgfqpoint{5.287075in}{1.317391in}}%
\pgfpathlineto{\pgfqpoint{5.292517in}{1.150435in}}%
\pgfpathlineto{\pgfqpoint{5.297959in}{1.317391in}}%
\pgfpathlineto{\pgfqpoint{5.303401in}{1.233913in}}%
\pgfpathlineto{\pgfqpoint{5.308844in}{1.317391in}}%
\pgfpathlineto{\pgfqpoint{5.314286in}{1.317391in}}%
\pgfpathlineto{\pgfqpoint{5.319728in}{1.400870in}}%
\pgfpathlineto{\pgfqpoint{5.325170in}{1.233913in}}%
\pgfpathlineto{\pgfqpoint{5.336054in}{1.400870in}}%
\pgfpathlineto{\pgfqpoint{5.341497in}{1.317391in}}%
\pgfpathlineto{\pgfqpoint{5.346939in}{1.150435in}}%
\pgfpathlineto{\pgfqpoint{5.352381in}{1.484348in}}%
\pgfpathlineto{\pgfqpoint{5.357823in}{1.233913in}}%
\pgfpathlineto{\pgfqpoint{5.363265in}{1.317391in}}%
\pgfpathlineto{\pgfqpoint{5.368707in}{1.233913in}}%
\pgfpathlineto{\pgfqpoint{5.374150in}{1.233913in}}%
\pgfpathlineto{\pgfqpoint{5.379592in}{1.317391in}}%
\pgfpathlineto{\pgfqpoint{5.385034in}{1.150435in}}%
\pgfpathlineto{\pgfqpoint{5.390476in}{1.317391in}}%
\pgfpathlineto{\pgfqpoint{5.395918in}{1.317391in}}%
\pgfpathlineto{\pgfqpoint{5.401361in}{1.484348in}}%
\pgfpathlineto{\pgfqpoint{5.412245in}{1.317391in}}%
\pgfpathlineto{\pgfqpoint{5.417687in}{1.400870in}}%
\pgfpathlineto{\pgfqpoint{5.428571in}{1.400870in}}%
\pgfpathlineto{\pgfqpoint{5.439456in}{1.233913in}}%
\pgfpathlineto{\pgfqpoint{5.444898in}{1.317391in}}%
\pgfpathlineto{\pgfqpoint{5.450340in}{1.317391in}}%
\pgfpathlineto{\pgfqpoint{5.455782in}{1.400870in}}%
\pgfpathlineto{\pgfqpoint{5.466667in}{1.400870in}}%
\pgfpathlineto{\pgfqpoint{5.472109in}{1.150435in}}%
\pgfpathlineto{\pgfqpoint{5.477551in}{1.317391in}}%
\pgfpathlineto{\pgfqpoint{5.482993in}{1.400870in}}%
\pgfpathlineto{\pgfqpoint{5.488435in}{1.233913in}}%
\pgfpathlineto{\pgfqpoint{5.493878in}{1.317391in}}%
\pgfpathlineto{\pgfqpoint{5.499320in}{1.233913in}}%
\pgfpathlineto{\pgfqpoint{5.504762in}{1.400870in}}%
\pgfpathlineto{\pgfqpoint{5.510204in}{1.317391in}}%
\pgfpathlineto{\pgfqpoint{5.515646in}{1.400870in}}%
\pgfpathlineto{\pgfqpoint{5.521088in}{1.317391in}}%
\pgfpathlineto{\pgfqpoint{5.526531in}{1.317391in}}%
\pgfpathlineto{\pgfqpoint{5.531973in}{1.400870in}}%
\pgfpathlineto{\pgfqpoint{5.537415in}{1.317391in}}%
\pgfpathlineto{\pgfqpoint{5.548299in}{1.317391in}}%
\pgfpathlineto{\pgfqpoint{5.553741in}{1.233913in}}%
\pgfpathlineto{\pgfqpoint{5.559184in}{1.400870in}}%
\pgfpathlineto{\pgfqpoint{5.564626in}{1.317391in}}%
\pgfpathlineto{\pgfqpoint{5.570068in}{1.317391in}}%
\pgfpathlineto{\pgfqpoint{5.575510in}{1.567826in}}%
\pgfpathlineto{\pgfqpoint{5.580952in}{1.400870in}}%
\pgfpathlineto{\pgfqpoint{5.586395in}{1.317391in}}%
\pgfpathlineto{\pgfqpoint{5.591837in}{1.400870in}}%
\pgfpathlineto{\pgfqpoint{5.597279in}{1.233913in}}%
\pgfpathlineto{\pgfqpoint{5.602721in}{1.317391in}}%
\pgfpathlineto{\pgfqpoint{5.613605in}{1.150435in}}%
\pgfpathlineto{\pgfqpoint{5.619048in}{1.400870in}}%
\pgfpathlineto{\pgfqpoint{5.624490in}{1.400870in}}%
\pgfpathlineto{\pgfqpoint{5.629932in}{1.233913in}}%
\pgfpathlineto{\pgfqpoint{5.635374in}{1.317391in}}%
\pgfpathlineto{\pgfqpoint{5.640816in}{1.317391in}}%
\pgfpathlineto{\pgfqpoint{5.651701in}{1.150435in}}%
\pgfpathlineto{\pgfqpoint{5.657143in}{1.233913in}}%
\pgfpathlineto{\pgfqpoint{5.662585in}{1.567826in}}%
\pgfpathlineto{\pgfqpoint{5.668027in}{1.317391in}}%
\pgfpathlineto{\pgfqpoint{5.673469in}{1.317391in}}%
\pgfpathlineto{\pgfqpoint{5.678912in}{1.150435in}}%
\pgfpathlineto{\pgfqpoint{5.684354in}{1.400870in}}%
\pgfpathlineto{\pgfqpoint{5.689796in}{1.400870in}}%
\pgfpathlineto{\pgfqpoint{5.695238in}{1.233913in}}%
\pgfpathlineto{\pgfqpoint{5.700680in}{1.317391in}}%
\pgfpathlineto{\pgfqpoint{5.706122in}{1.317391in}}%
\pgfpathlineto{\pgfqpoint{5.711565in}{1.400870in}}%
\pgfpathlineto{\pgfqpoint{5.722449in}{1.400870in}}%
\pgfpathlineto{\pgfqpoint{5.727891in}{1.317391in}}%
\pgfpathlineto{\pgfqpoint{5.733333in}{1.484348in}}%
\pgfpathlineto{\pgfqpoint{5.738776in}{1.233913in}}%
\pgfpathlineto{\pgfqpoint{5.744218in}{1.400870in}}%
\pgfpathlineto{\pgfqpoint{5.755102in}{1.233913in}}%
\pgfpathlineto{\pgfqpoint{5.765986in}{1.233913in}}%
\pgfpathlineto{\pgfqpoint{5.776871in}{1.400870in}}%
\pgfpathlineto{\pgfqpoint{5.782313in}{1.317391in}}%
\pgfpathlineto{\pgfqpoint{5.787755in}{1.484348in}}%
\pgfpathlineto{\pgfqpoint{5.793197in}{1.317391in}}%
\pgfpathlineto{\pgfqpoint{5.798639in}{1.400870in}}%
\pgfpathlineto{\pgfqpoint{5.804082in}{1.317391in}}%
\pgfpathlineto{\pgfqpoint{5.809524in}{1.400870in}}%
\pgfpathlineto{\pgfqpoint{5.814966in}{1.317391in}}%
\pgfpathlineto{\pgfqpoint{5.820408in}{1.400870in}}%
\pgfpathlineto{\pgfqpoint{5.825850in}{1.400870in}}%
\pgfpathlineto{\pgfqpoint{5.831293in}{1.233913in}}%
\pgfpathlineto{\pgfqpoint{5.836735in}{1.233913in}}%
\pgfpathlineto{\pgfqpoint{5.842177in}{1.400870in}}%
\pgfpathlineto{\pgfqpoint{5.847619in}{1.400870in}}%
\pgfpathlineto{\pgfqpoint{5.858503in}{1.233913in}}%
\pgfpathlineto{\pgfqpoint{5.863946in}{1.317391in}}%
\pgfpathlineto{\pgfqpoint{5.869388in}{1.150435in}}%
\pgfpathlineto{\pgfqpoint{5.874830in}{1.400870in}}%
\pgfpathlineto{\pgfqpoint{5.880272in}{1.317391in}}%
\pgfpathlineto{\pgfqpoint{5.885714in}{1.400870in}}%
\pgfpathlineto{\pgfqpoint{5.891156in}{1.150435in}}%
\pgfpathlineto{\pgfqpoint{5.896599in}{1.317391in}}%
\pgfpathlineto{\pgfqpoint{5.902041in}{1.400870in}}%
\pgfpathlineto{\pgfqpoint{5.907483in}{1.317391in}}%
\pgfpathlineto{\pgfqpoint{5.918367in}{1.484348in}}%
\pgfpathlineto{\pgfqpoint{5.923810in}{1.400870in}}%
\pgfpathlineto{\pgfqpoint{5.929252in}{1.484348in}}%
\pgfpathlineto{\pgfqpoint{5.934694in}{1.233913in}}%
\pgfpathlineto{\pgfqpoint{5.940136in}{1.400870in}}%
\pgfpathlineto{\pgfqpoint{5.945578in}{1.317391in}}%
\pgfpathlineto{\pgfqpoint{5.951020in}{1.484348in}}%
\pgfpathlineto{\pgfqpoint{5.956463in}{1.317391in}}%
\pgfpathlineto{\pgfqpoint{5.972789in}{1.317391in}}%
\pgfpathlineto{\pgfqpoint{5.978231in}{1.233913in}}%
\pgfpathlineto{\pgfqpoint{5.983673in}{1.400870in}}%
\pgfpathlineto{\pgfqpoint{5.989116in}{1.484348in}}%
\pgfpathlineto{\pgfqpoint{5.994558in}{1.317391in}}%
\pgfpathlineto{\pgfqpoint{6.005442in}{1.317391in}}%
\pgfpathlineto{\pgfqpoint{6.010884in}{1.150435in}}%
\pgfpathlineto{\pgfqpoint{6.016327in}{1.233913in}}%
\pgfpathlineto{\pgfqpoint{6.021769in}{1.150435in}}%
\pgfpathlineto{\pgfqpoint{6.027211in}{1.317391in}}%
\pgfpathlineto{\pgfqpoint{6.032653in}{1.400870in}}%
\pgfpathlineto{\pgfqpoint{6.038095in}{1.317391in}}%
\pgfpathlineto{\pgfqpoint{6.043537in}{1.400870in}}%
\pgfpathlineto{\pgfqpoint{6.048980in}{1.400870in}}%
\pgfpathlineto{\pgfqpoint{6.059864in}{1.233913in}}%
\pgfpathlineto{\pgfqpoint{6.065306in}{1.233913in}}%
\pgfpathlineto{\pgfqpoint{6.070748in}{1.400870in}}%
\pgfpathlineto{\pgfqpoint{6.081633in}{1.233913in}}%
\pgfpathlineto{\pgfqpoint{6.087075in}{1.317391in}}%
\pgfpathlineto{\pgfqpoint{6.092517in}{1.233913in}}%
\pgfpathlineto{\pgfqpoint{6.103401in}{1.400870in}}%
\pgfpathlineto{\pgfqpoint{6.108844in}{1.317391in}}%
\pgfpathlineto{\pgfqpoint{6.114286in}{1.150435in}}%
\pgfpathlineto{\pgfqpoint{6.119728in}{1.317391in}}%
\pgfpathlineto{\pgfqpoint{6.125170in}{1.150435in}}%
\pgfpathlineto{\pgfqpoint{6.130612in}{1.484348in}}%
\pgfpathlineto{\pgfqpoint{6.136054in}{1.317391in}}%
\pgfpathlineto{\pgfqpoint{6.141497in}{1.317391in}}%
\pgfpathlineto{\pgfqpoint{6.146939in}{1.233913in}}%
\pgfpathlineto{\pgfqpoint{6.152381in}{1.400870in}}%
\pgfpathlineto{\pgfqpoint{6.157823in}{1.317391in}}%
\pgfpathlineto{\pgfqpoint{6.163265in}{1.484348in}}%
\pgfpathlineto{\pgfqpoint{6.179592in}{1.233913in}}%
\pgfpathlineto{\pgfqpoint{6.185034in}{1.317391in}}%
\pgfpathlineto{\pgfqpoint{6.190476in}{1.317391in}}%
\pgfpathlineto{\pgfqpoint{6.195918in}{1.484348in}}%
\pgfpathlineto{\pgfqpoint{6.201361in}{1.233913in}}%
\pgfpathlineto{\pgfqpoint{6.206803in}{1.400870in}}%
\pgfpathlineto{\pgfqpoint{6.212245in}{1.484348in}}%
\pgfpathlineto{\pgfqpoint{6.217687in}{1.400870in}}%
\pgfpathlineto{\pgfqpoint{6.223129in}{1.484348in}}%
\pgfpathlineto{\pgfqpoint{6.228571in}{1.400870in}}%
\pgfpathlineto{\pgfqpoint{6.234014in}{1.233913in}}%
\pgfpathlineto{\pgfqpoint{6.239456in}{1.317391in}}%
\pgfpathlineto{\pgfqpoint{6.244898in}{1.233913in}}%
\pgfpathlineto{\pgfqpoint{6.250340in}{1.400870in}}%
\pgfpathlineto{\pgfqpoint{6.255782in}{1.317391in}}%
\pgfpathlineto{\pgfqpoint{6.261224in}{1.317391in}}%
\pgfpathlineto{\pgfqpoint{6.266667in}{1.233913in}}%
\pgfpathlineto{\pgfqpoint{6.272109in}{1.317391in}}%
\pgfpathlineto{\pgfqpoint{6.277551in}{1.233913in}}%
\pgfpathlineto{\pgfqpoint{6.282993in}{1.317391in}}%
\pgfpathlineto{\pgfqpoint{6.288435in}{1.317391in}}%
\pgfpathlineto{\pgfqpoint{6.293878in}{1.400870in}}%
\pgfpathlineto{\pgfqpoint{6.299320in}{1.233913in}}%
\pgfpathlineto{\pgfqpoint{6.304762in}{1.484348in}}%
\pgfpathlineto{\pgfqpoint{6.310204in}{1.317391in}}%
\pgfpathlineto{\pgfqpoint{6.315646in}{1.484348in}}%
\pgfpathlineto{\pgfqpoint{6.321088in}{1.317391in}}%
\pgfpathlineto{\pgfqpoint{6.326531in}{1.233913in}}%
\pgfpathlineto{\pgfqpoint{6.331973in}{1.233913in}}%
\pgfpathlineto{\pgfqpoint{6.342857in}{1.400870in}}%
\pgfpathlineto{\pgfqpoint{6.348299in}{1.400870in}}%
\pgfpathlineto{\pgfqpoint{6.353741in}{1.233913in}}%
\pgfpathlineto{\pgfqpoint{6.359184in}{1.400870in}}%
\pgfpathlineto{\pgfqpoint{6.364626in}{1.233913in}}%
\pgfpathlineto{\pgfqpoint{6.370068in}{1.317391in}}%
\pgfpathlineto{\pgfqpoint{6.375510in}{1.233913in}}%
\pgfpathlineto{\pgfqpoint{6.380952in}{1.400870in}}%
\pgfpathlineto{\pgfqpoint{6.391837in}{1.233913in}}%
\pgfpathlineto{\pgfqpoint{6.397279in}{1.317391in}}%
\pgfpathlineto{\pgfqpoint{6.402721in}{1.233913in}}%
\pgfpathlineto{\pgfqpoint{6.413605in}{1.400870in}}%
\pgfpathlineto{\pgfqpoint{6.419048in}{1.317391in}}%
\pgfpathlineto{\pgfqpoint{6.424490in}{1.400870in}}%
\pgfpathlineto{\pgfqpoint{6.435374in}{1.233913in}}%
\pgfpathlineto{\pgfqpoint{6.440816in}{1.233913in}}%
\pgfpathlineto{\pgfqpoint{6.446259in}{1.317391in}}%
\pgfpathlineto{\pgfqpoint{6.451701in}{1.233913in}}%
\pgfpathlineto{\pgfqpoint{6.457143in}{1.317391in}}%
\pgfpathlineto{\pgfqpoint{6.462585in}{1.233913in}}%
\pgfpathlineto{\pgfqpoint{6.468027in}{1.233913in}}%
\pgfpathlineto{\pgfqpoint{6.473469in}{1.150435in}}%
\pgfpathlineto{\pgfqpoint{6.478912in}{1.150435in}}%
\pgfpathlineto{\pgfqpoint{6.484354in}{1.317391in}}%
\pgfpathlineto{\pgfqpoint{6.495238in}{1.484348in}}%
\pgfpathlineto{\pgfqpoint{6.506122in}{1.317391in}}%
\pgfpathlineto{\pgfqpoint{6.511565in}{1.400870in}}%
\pgfpathlineto{\pgfqpoint{6.517007in}{1.317391in}}%
\pgfpathlineto{\pgfqpoint{6.522449in}{1.317391in}}%
\pgfpathlineto{\pgfqpoint{6.527891in}{1.233913in}}%
\pgfpathlineto{\pgfqpoint{6.538776in}{1.400870in}}%
\pgfpathlineto{\pgfqpoint{6.549660in}{1.233913in}}%
\pgfpathlineto{\pgfqpoint{6.555102in}{1.233913in}}%
\pgfpathlineto{\pgfqpoint{6.560544in}{1.400870in}}%
\pgfpathlineto{\pgfqpoint{6.565986in}{1.233913in}}%
\pgfpathlineto{\pgfqpoint{6.571429in}{1.400870in}}%
\pgfpathlineto{\pgfqpoint{6.576871in}{1.233913in}}%
\pgfpathlineto{\pgfqpoint{6.582313in}{1.567826in}}%
\pgfpathlineto{\pgfqpoint{6.587755in}{1.233913in}}%
\pgfpathlineto{\pgfqpoint{6.593197in}{1.400870in}}%
\pgfpathlineto{\pgfqpoint{6.598639in}{1.317391in}}%
\pgfpathlineto{\pgfqpoint{6.604082in}{1.400870in}}%
\pgfpathlineto{\pgfqpoint{6.609524in}{1.233913in}}%
\pgfpathlineto{\pgfqpoint{6.620408in}{1.400870in}}%
\pgfpathlineto{\pgfqpoint{6.625850in}{1.317391in}}%
\pgfpathlineto{\pgfqpoint{6.642177in}{1.317391in}}%
\pgfpathlineto{\pgfqpoint{6.647619in}{1.400870in}}%
\pgfpathlineto{\pgfqpoint{6.653061in}{1.233913in}}%
\pgfpathlineto{\pgfqpoint{6.658503in}{1.317391in}}%
\pgfpathlineto{\pgfqpoint{6.663946in}{1.317391in}}%
\pgfpathlineto{\pgfqpoint{6.669388in}{1.233913in}}%
\pgfpathlineto{\pgfqpoint{6.674830in}{1.317391in}}%
\pgfpathlineto{\pgfqpoint{6.680272in}{1.150435in}}%
\pgfpathlineto{\pgfqpoint{6.691156in}{1.317391in}}%
\pgfpathlineto{\pgfqpoint{6.696599in}{1.233913in}}%
\pgfpathlineto{\pgfqpoint{6.702041in}{1.317391in}}%
\pgfpathlineto{\pgfqpoint{6.707483in}{1.317391in}}%
\pgfpathlineto{\pgfqpoint{6.718367in}{1.150435in}}%
\pgfpathlineto{\pgfqpoint{6.729252in}{1.484348in}}%
\pgfpathlineto{\pgfqpoint{6.745578in}{1.233913in}}%
\pgfpathlineto{\pgfqpoint{6.751020in}{1.317391in}}%
\pgfpathlineto{\pgfqpoint{6.756463in}{1.233913in}}%
\pgfpathlineto{\pgfqpoint{6.761905in}{1.317391in}}%
\pgfpathlineto{\pgfqpoint{6.767347in}{1.233913in}}%
\pgfpathlineto{\pgfqpoint{6.772789in}{1.317391in}}%
\pgfpathlineto{\pgfqpoint{6.778231in}{1.484348in}}%
\pgfpathlineto{\pgfqpoint{6.783673in}{1.150435in}}%
\pgfpathlineto{\pgfqpoint{6.789116in}{1.317391in}}%
\pgfpathlineto{\pgfqpoint{6.794558in}{1.400870in}}%
\pgfpathlineto{\pgfqpoint{6.794558in}{1.400870in}}%
\pgfusepath{stroke}%
\end{pgfscope}%
\begin{pgfscope}%
\pgfsetrectcap%
\pgfsetmiterjoin%
\pgfsetlinewidth{1.003750pt}%
\definecolor{currentstroke}{rgb}{0.000000,0.000000,0.000000}%
\pgfsetstrokecolor{currentstroke}%
\pgfsetdash{}{0pt}%
\pgfpathmoveto{\pgfqpoint{1.200000in}{0.900000in}}%
\pgfpathlineto{\pgfqpoint{1.200000in}{5.700000in}}%
\pgfusepath{stroke}%
\end{pgfscope}%
\begin{pgfscope}%
\pgfsetrectcap%
\pgfsetmiterjoin%
\pgfsetlinewidth{1.003750pt}%
\definecolor{currentstroke}{rgb}{0.000000,0.000000,0.000000}%
\pgfsetstrokecolor{currentstroke}%
\pgfsetdash{}{0pt}%
\pgfpathmoveto{\pgfqpoint{6.800000in}{0.900000in}}%
\pgfpathlineto{\pgfqpoint{6.800000in}{5.700000in}}%
\pgfusepath{stroke}%
\end{pgfscope}%
\begin{pgfscope}%
\pgfsetrectcap%
\pgfsetmiterjoin%
\pgfsetlinewidth{1.003750pt}%
\definecolor{currentstroke}{rgb}{0.000000,0.000000,0.000000}%
\pgfsetstrokecolor{currentstroke}%
\pgfsetdash{}{0pt}%
\pgfpathmoveto{\pgfqpoint{1.200000in}{0.900000in}}%
\pgfpathlineto{\pgfqpoint{6.800000in}{0.900000in}}%
\pgfusepath{stroke}%
\end{pgfscope}%
\begin{pgfscope}%
\pgfsetrectcap%
\pgfsetmiterjoin%
\pgfsetlinewidth{1.003750pt}%
\definecolor{currentstroke}{rgb}{0.000000,0.000000,0.000000}%
\pgfsetstrokecolor{currentstroke}%
\pgfsetdash{}{0pt}%
\pgfpathmoveto{\pgfqpoint{1.200000in}{5.700000in}}%
\pgfpathlineto{\pgfqpoint{6.800000in}{5.700000in}}%
\pgfusepath{stroke}%
\end{pgfscope}%
\begin{pgfscope}%
\pgfsetbuttcap%
\pgfsetroundjoin%
\definecolor{currentfill}{rgb}{0.000000,0.000000,0.000000}%
\pgfsetfillcolor{currentfill}%
\pgfsetlinewidth{0.501875pt}%
\definecolor{currentstroke}{rgb}{0.000000,0.000000,0.000000}%
\pgfsetstrokecolor{currentstroke}%
\pgfsetdash{}{0pt}%
\pgfsys@defobject{currentmarker}{\pgfqpoint{0.000000in}{0.000000in}}{\pgfqpoint{0.000000in}{0.055556in}}{%
\pgfpathmoveto{\pgfqpoint{0.000000in}{0.000000in}}%
\pgfpathlineto{\pgfqpoint{0.000000in}{0.055556in}}%
\pgfusepath{stroke,fill}%
}%
\begin{pgfscope}%
\pgfsys@transformshift{1.200000in}{0.900000in}%
\pgfsys@useobject{currentmarker}{}%
\end{pgfscope}%
\end{pgfscope}%
\begin{pgfscope}%
\pgfsetbuttcap%
\pgfsetroundjoin%
\definecolor{currentfill}{rgb}{0.000000,0.000000,0.000000}%
\pgfsetfillcolor{currentfill}%
\pgfsetlinewidth{0.501875pt}%
\definecolor{currentstroke}{rgb}{0.000000,0.000000,0.000000}%
\pgfsetstrokecolor{currentstroke}%
\pgfsetdash{}{0pt}%
\pgfsys@defobject{currentmarker}{\pgfqpoint{0.000000in}{-0.055556in}}{\pgfqpoint{0.000000in}{0.000000in}}{%
\pgfpathmoveto{\pgfqpoint{0.000000in}{0.000000in}}%
\pgfpathlineto{\pgfqpoint{0.000000in}{-0.055556in}}%
\pgfusepath{stroke,fill}%
}%
\begin{pgfscope}%
\pgfsys@transformshift{1.200000in}{5.700000in}%
\pgfsys@useobject{currentmarker}{}%
\end{pgfscope}%
\end{pgfscope}%
\begin{pgfscope}%
\definecolor{textcolor}{rgb}{0.000000,0.000000,0.000000}%
\pgfsetstrokecolor{textcolor}%
\pgfsetfillcolor{textcolor}%
\pgftext[x=1.200000in,y=0.844444in,,top]{\color{textcolor}\sffamily\fontsize{20.000000}{24.000000}\selectfont \(\displaystyle {0}\)}%
\end{pgfscope}%
\begin{pgfscope}%
\pgfsetbuttcap%
\pgfsetroundjoin%
\definecolor{currentfill}{rgb}{0.000000,0.000000,0.000000}%
\pgfsetfillcolor{currentfill}%
\pgfsetlinewidth{0.501875pt}%
\definecolor{currentstroke}{rgb}{0.000000,0.000000,0.000000}%
\pgfsetstrokecolor{currentstroke}%
\pgfsetdash{}{0pt}%
\pgfsys@defobject{currentmarker}{\pgfqpoint{0.000000in}{0.000000in}}{\pgfqpoint{0.000000in}{0.055556in}}{%
\pgfpathmoveto{\pgfqpoint{0.000000in}{0.000000in}}%
\pgfpathlineto{\pgfqpoint{0.000000in}{0.055556in}}%
\pgfusepath{stroke,fill}%
}%
\begin{pgfscope}%
\pgfsys@transformshift{2.288435in}{0.900000in}%
\pgfsys@useobject{currentmarker}{}%
\end{pgfscope}%
\end{pgfscope}%
\begin{pgfscope}%
\pgfsetbuttcap%
\pgfsetroundjoin%
\definecolor{currentfill}{rgb}{0.000000,0.000000,0.000000}%
\pgfsetfillcolor{currentfill}%
\pgfsetlinewidth{0.501875pt}%
\definecolor{currentstroke}{rgb}{0.000000,0.000000,0.000000}%
\pgfsetstrokecolor{currentstroke}%
\pgfsetdash{}{0pt}%
\pgfsys@defobject{currentmarker}{\pgfqpoint{0.000000in}{-0.055556in}}{\pgfqpoint{0.000000in}{0.000000in}}{%
\pgfpathmoveto{\pgfqpoint{0.000000in}{0.000000in}}%
\pgfpathlineto{\pgfqpoint{0.000000in}{-0.055556in}}%
\pgfusepath{stroke,fill}%
}%
\begin{pgfscope}%
\pgfsys@transformshift{2.288435in}{5.700000in}%
\pgfsys@useobject{currentmarker}{}%
\end{pgfscope}%
\end{pgfscope}%
\begin{pgfscope}%
\definecolor{textcolor}{rgb}{0.000000,0.000000,0.000000}%
\pgfsetstrokecolor{textcolor}%
\pgfsetfillcolor{textcolor}%
\pgftext[x=2.288435in,y=0.844444in,,top]{\color{textcolor}\sffamily\fontsize{20.000000}{24.000000}\selectfont \(\displaystyle {200}\)}%
\end{pgfscope}%
\begin{pgfscope}%
\pgfsetbuttcap%
\pgfsetroundjoin%
\definecolor{currentfill}{rgb}{0.000000,0.000000,0.000000}%
\pgfsetfillcolor{currentfill}%
\pgfsetlinewidth{0.501875pt}%
\definecolor{currentstroke}{rgb}{0.000000,0.000000,0.000000}%
\pgfsetstrokecolor{currentstroke}%
\pgfsetdash{}{0pt}%
\pgfsys@defobject{currentmarker}{\pgfqpoint{0.000000in}{0.000000in}}{\pgfqpoint{0.000000in}{0.055556in}}{%
\pgfpathmoveto{\pgfqpoint{0.000000in}{0.000000in}}%
\pgfpathlineto{\pgfqpoint{0.000000in}{0.055556in}}%
\pgfusepath{stroke,fill}%
}%
\begin{pgfscope}%
\pgfsys@transformshift{3.376871in}{0.900000in}%
\pgfsys@useobject{currentmarker}{}%
\end{pgfscope}%
\end{pgfscope}%
\begin{pgfscope}%
\pgfsetbuttcap%
\pgfsetroundjoin%
\definecolor{currentfill}{rgb}{0.000000,0.000000,0.000000}%
\pgfsetfillcolor{currentfill}%
\pgfsetlinewidth{0.501875pt}%
\definecolor{currentstroke}{rgb}{0.000000,0.000000,0.000000}%
\pgfsetstrokecolor{currentstroke}%
\pgfsetdash{}{0pt}%
\pgfsys@defobject{currentmarker}{\pgfqpoint{0.000000in}{-0.055556in}}{\pgfqpoint{0.000000in}{0.000000in}}{%
\pgfpathmoveto{\pgfqpoint{0.000000in}{0.000000in}}%
\pgfpathlineto{\pgfqpoint{0.000000in}{-0.055556in}}%
\pgfusepath{stroke,fill}%
}%
\begin{pgfscope}%
\pgfsys@transformshift{3.376871in}{5.700000in}%
\pgfsys@useobject{currentmarker}{}%
\end{pgfscope}%
\end{pgfscope}%
\begin{pgfscope}%
\definecolor{textcolor}{rgb}{0.000000,0.000000,0.000000}%
\pgfsetstrokecolor{textcolor}%
\pgfsetfillcolor{textcolor}%
\pgftext[x=3.376871in,y=0.844444in,,top]{\color{textcolor}\sffamily\fontsize{20.000000}{24.000000}\selectfont \(\displaystyle {400}\)}%
\end{pgfscope}%
\begin{pgfscope}%
\pgfsetbuttcap%
\pgfsetroundjoin%
\definecolor{currentfill}{rgb}{0.000000,0.000000,0.000000}%
\pgfsetfillcolor{currentfill}%
\pgfsetlinewidth{0.501875pt}%
\definecolor{currentstroke}{rgb}{0.000000,0.000000,0.000000}%
\pgfsetstrokecolor{currentstroke}%
\pgfsetdash{}{0pt}%
\pgfsys@defobject{currentmarker}{\pgfqpoint{0.000000in}{0.000000in}}{\pgfqpoint{0.000000in}{0.055556in}}{%
\pgfpathmoveto{\pgfqpoint{0.000000in}{0.000000in}}%
\pgfpathlineto{\pgfqpoint{0.000000in}{0.055556in}}%
\pgfusepath{stroke,fill}%
}%
\begin{pgfscope}%
\pgfsys@transformshift{4.465306in}{0.900000in}%
\pgfsys@useobject{currentmarker}{}%
\end{pgfscope}%
\end{pgfscope}%
\begin{pgfscope}%
\pgfsetbuttcap%
\pgfsetroundjoin%
\definecolor{currentfill}{rgb}{0.000000,0.000000,0.000000}%
\pgfsetfillcolor{currentfill}%
\pgfsetlinewidth{0.501875pt}%
\definecolor{currentstroke}{rgb}{0.000000,0.000000,0.000000}%
\pgfsetstrokecolor{currentstroke}%
\pgfsetdash{}{0pt}%
\pgfsys@defobject{currentmarker}{\pgfqpoint{0.000000in}{-0.055556in}}{\pgfqpoint{0.000000in}{0.000000in}}{%
\pgfpathmoveto{\pgfqpoint{0.000000in}{0.000000in}}%
\pgfpathlineto{\pgfqpoint{0.000000in}{-0.055556in}}%
\pgfusepath{stroke,fill}%
}%
\begin{pgfscope}%
\pgfsys@transformshift{4.465306in}{5.700000in}%
\pgfsys@useobject{currentmarker}{}%
\end{pgfscope}%
\end{pgfscope}%
\begin{pgfscope}%
\definecolor{textcolor}{rgb}{0.000000,0.000000,0.000000}%
\pgfsetstrokecolor{textcolor}%
\pgfsetfillcolor{textcolor}%
\pgftext[x=4.465306in,y=0.844444in,,top]{\color{textcolor}\sffamily\fontsize{20.000000}{24.000000}\selectfont \(\displaystyle {600}\)}%
\end{pgfscope}%
\begin{pgfscope}%
\pgfsetbuttcap%
\pgfsetroundjoin%
\definecolor{currentfill}{rgb}{0.000000,0.000000,0.000000}%
\pgfsetfillcolor{currentfill}%
\pgfsetlinewidth{0.501875pt}%
\definecolor{currentstroke}{rgb}{0.000000,0.000000,0.000000}%
\pgfsetstrokecolor{currentstroke}%
\pgfsetdash{}{0pt}%
\pgfsys@defobject{currentmarker}{\pgfqpoint{0.000000in}{0.000000in}}{\pgfqpoint{0.000000in}{0.055556in}}{%
\pgfpathmoveto{\pgfqpoint{0.000000in}{0.000000in}}%
\pgfpathlineto{\pgfqpoint{0.000000in}{0.055556in}}%
\pgfusepath{stroke,fill}%
}%
\begin{pgfscope}%
\pgfsys@transformshift{5.553741in}{0.900000in}%
\pgfsys@useobject{currentmarker}{}%
\end{pgfscope}%
\end{pgfscope}%
\begin{pgfscope}%
\pgfsetbuttcap%
\pgfsetroundjoin%
\definecolor{currentfill}{rgb}{0.000000,0.000000,0.000000}%
\pgfsetfillcolor{currentfill}%
\pgfsetlinewidth{0.501875pt}%
\definecolor{currentstroke}{rgb}{0.000000,0.000000,0.000000}%
\pgfsetstrokecolor{currentstroke}%
\pgfsetdash{}{0pt}%
\pgfsys@defobject{currentmarker}{\pgfqpoint{0.000000in}{-0.055556in}}{\pgfqpoint{0.000000in}{0.000000in}}{%
\pgfpathmoveto{\pgfqpoint{0.000000in}{0.000000in}}%
\pgfpathlineto{\pgfqpoint{0.000000in}{-0.055556in}}%
\pgfusepath{stroke,fill}%
}%
\begin{pgfscope}%
\pgfsys@transformshift{5.553741in}{5.700000in}%
\pgfsys@useobject{currentmarker}{}%
\end{pgfscope}%
\end{pgfscope}%
\begin{pgfscope}%
\definecolor{textcolor}{rgb}{0.000000,0.000000,0.000000}%
\pgfsetstrokecolor{textcolor}%
\pgfsetfillcolor{textcolor}%
\pgftext[x=5.553741in,y=0.844444in,,top]{\color{textcolor}\sffamily\fontsize{20.000000}{24.000000}\selectfont \(\displaystyle {800}\)}%
\end{pgfscope}%
\begin{pgfscope}%
\pgfsetbuttcap%
\pgfsetroundjoin%
\definecolor{currentfill}{rgb}{0.000000,0.000000,0.000000}%
\pgfsetfillcolor{currentfill}%
\pgfsetlinewidth{0.501875pt}%
\definecolor{currentstroke}{rgb}{0.000000,0.000000,0.000000}%
\pgfsetstrokecolor{currentstroke}%
\pgfsetdash{}{0pt}%
\pgfsys@defobject{currentmarker}{\pgfqpoint{0.000000in}{0.000000in}}{\pgfqpoint{0.000000in}{0.055556in}}{%
\pgfpathmoveto{\pgfqpoint{0.000000in}{0.000000in}}%
\pgfpathlineto{\pgfqpoint{0.000000in}{0.055556in}}%
\pgfusepath{stroke,fill}%
}%
\begin{pgfscope}%
\pgfsys@transformshift{6.642177in}{0.900000in}%
\pgfsys@useobject{currentmarker}{}%
\end{pgfscope}%
\end{pgfscope}%
\begin{pgfscope}%
\pgfsetbuttcap%
\pgfsetroundjoin%
\definecolor{currentfill}{rgb}{0.000000,0.000000,0.000000}%
\pgfsetfillcolor{currentfill}%
\pgfsetlinewidth{0.501875pt}%
\definecolor{currentstroke}{rgb}{0.000000,0.000000,0.000000}%
\pgfsetstrokecolor{currentstroke}%
\pgfsetdash{}{0pt}%
\pgfsys@defobject{currentmarker}{\pgfqpoint{0.000000in}{-0.055556in}}{\pgfqpoint{0.000000in}{0.000000in}}{%
\pgfpathmoveto{\pgfqpoint{0.000000in}{0.000000in}}%
\pgfpathlineto{\pgfqpoint{0.000000in}{-0.055556in}}%
\pgfusepath{stroke,fill}%
}%
\begin{pgfscope}%
\pgfsys@transformshift{6.642177in}{5.700000in}%
\pgfsys@useobject{currentmarker}{}%
\end{pgfscope}%
\end{pgfscope}%
\begin{pgfscope}%
\definecolor{textcolor}{rgb}{0.000000,0.000000,0.000000}%
\pgfsetstrokecolor{textcolor}%
\pgfsetfillcolor{textcolor}%
\pgftext[x=6.642177in,y=0.844444in,,top]{\color{textcolor}\sffamily\fontsize{20.000000}{24.000000}\selectfont \(\displaystyle {1000}\)}%
\end{pgfscope}%
\begin{pgfscope}%
\definecolor{textcolor}{rgb}{0.000000,0.000000,0.000000}%
\pgfsetstrokecolor{textcolor}%
\pgfsetfillcolor{textcolor}%
\pgftext[x=4.000000in,y=0.518932in,,top]{\color{textcolor}\sffamily\fontsize{20.000000}{24.000000}\selectfont \(\displaystyle \mathrm{t}/\si{ns}\)}%
\end{pgfscope}%
\begin{pgfscope}%
\pgfsetbuttcap%
\pgfsetroundjoin%
\definecolor{currentfill}{rgb}{0.000000,0.000000,0.000000}%
\pgfsetfillcolor{currentfill}%
\pgfsetlinewidth{0.501875pt}%
\definecolor{currentstroke}{rgb}{0.000000,0.000000,0.000000}%
\pgfsetstrokecolor{currentstroke}%
\pgfsetdash{}{0pt}%
\pgfsys@defobject{currentmarker}{\pgfqpoint{0.000000in}{0.000000in}}{\pgfqpoint{0.055556in}{0.000000in}}{%
\pgfpathmoveto{\pgfqpoint{0.000000in}{0.000000in}}%
\pgfpathlineto{\pgfqpoint{0.055556in}{0.000000in}}%
\pgfusepath{stroke,fill}%
}%
\begin{pgfscope}%
\pgfsys@transformshift{1.200000in}{1.317391in}%
\pgfsys@useobject{currentmarker}{}%
\end{pgfscope}%
\end{pgfscope}%
\begin{pgfscope}%
\pgfsetbuttcap%
\pgfsetroundjoin%
\definecolor{currentfill}{rgb}{0.000000,0.000000,0.000000}%
\pgfsetfillcolor{currentfill}%
\pgfsetlinewidth{0.501875pt}%
\definecolor{currentstroke}{rgb}{0.000000,0.000000,0.000000}%
\pgfsetstrokecolor{currentstroke}%
\pgfsetdash{}{0pt}%
\pgfsys@defobject{currentmarker}{\pgfqpoint{-0.055556in}{0.000000in}}{\pgfqpoint{-0.000000in}{0.000000in}}{%
\pgfpathmoveto{\pgfqpoint{-0.000000in}{0.000000in}}%
\pgfpathlineto{\pgfqpoint{-0.055556in}{0.000000in}}%
\pgfusepath{stroke,fill}%
}%
\begin{pgfscope}%
\pgfsys@transformshift{6.800000in}{1.317391in}%
\pgfsys@useobject{currentmarker}{}%
\end{pgfscope}%
\end{pgfscope}%
\begin{pgfscope}%
\definecolor{textcolor}{rgb}{0.000000,0.000000,0.000000}%
\pgfsetstrokecolor{textcolor}%
\pgfsetfillcolor{textcolor}%
\pgftext[x=1.144444in,y=1.317391in,right,]{\color{textcolor}\sffamily\fontsize{20.000000}{24.000000}\selectfont \(\displaystyle {0}\)}%
\end{pgfscope}%
\begin{pgfscope}%
\pgfsetbuttcap%
\pgfsetroundjoin%
\definecolor{currentfill}{rgb}{0.000000,0.000000,0.000000}%
\pgfsetfillcolor{currentfill}%
\pgfsetlinewidth{0.501875pt}%
\definecolor{currentstroke}{rgb}{0.000000,0.000000,0.000000}%
\pgfsetstrokecolor{currentstroke}%
\pgfsetdash{}{0pt}%
\pgfsys@defobject{currentmarker}{\pgfqpoint{0.000000in}{0.000000in}}{\pgfqpoint{0.055556in}{0.000000in}}{%
\pgfpathmoveto{\pgfqpoint{0.000000in}{0.000000in}}%
\pgfpathlineto{\pgfqpoint{0.055556in}{0.000000in}}%
\pgfusepath{stroke,fill}%
}%
\begin{pgfscope}%
\pgfsys@transformshift{1.200000in}{2.152174in}%
\pgfsys@useobject{currentmarker}{}%
\end{pgfscope}%
\end{pgfscope}%
\begin{pgfscope}%
\pgfsetbuttcap%
\pgfsetroundjoin%
\definecolor{currentfill}{rgb}{0.000000,0.000000,0.000000}%
\pgfsetfillcolor{currentfill}%
\pgfsetlinewidth{0.501875pt}%
\definecolor{currentstroke}{rgb}{0.000000,0.000000,0.000000}%
\pgfsetstrokecolor{currentstroke}%
\pgfsetdash{}{0pt}%
\pgfsys@defobject{currentmarker}{\pgfqpoint{-0.055556in}{0.000000in}}{\pgfqpoint{-0.000000in}{0.000000in}}{%
\pgfpathmoveto{\pgfqpoint{-0.000000in}{0.000000in}}%
\pgfpathlineto{\pgfqpoint{-0.055556in}{0.000000in}}%
\pgfusepath{stroke,fill}%
}%
\begin{pgfscope}%
\pgfsys@transformshift{6.800000in}{2.152174in}%
\pgfsys@useobject{currentmarker}{}%
\end{pgfscope}%
\end{pgfscope}%
\begin{pgfscope}%
\definecolor{textcolor}{rgb}{0.000000,0.000000,0.000000}%
\pgfsetstrokecolor{textcolor}%
\pgfsetfillcolor{textcolor}%
\pgftext[x=1.144444in,y=2.152174in,right,]{\color{textcolor}\sffamily\fontsize{20.000000}{24.000000}\selectfont \(\displaystyle {10}\)}%
\end{pgfscope}%
\begin{pgfscope}%
\pgfsetbuttcap%
\pgfsetroundjoin%
\definecolor{currentfill}{rgb}{0.000000,0.000000,0.000000}%
\pgfsetfillcolor{currentfill}%
\pgfsetlinewidth{0.501875pt}%
\definecolor{currentstroke}{rgb}{0.000000,0.000000,0.000000}%
\pgfsetstrokecolor{currentstroke}%
\pgfsetdash{}{0pt}%
\pgfsys@defobject{currentmarker}{\pgfqpoint{0.000000in}{0.000000in}}{\pgfqpoint{0.055556in}{0.000000in}}{%
\pgfpathmoveto{\pgfqpoint{0.000000in}{0.000000in}}%
\pgfpathlineto{\pgfqpoint{0.055556in}{0.000000in}}%
\pgfusepath{stroke,fill}%
}%
\begin{pgfscope}%
\pgfsys@transformshift{1.200000in}{2.986957in}%
\pgfsys@useobject{currentmarker}{}%
\end{pgfscope}%
\end{pgfscope}%
\begin{pgfscope}%
\pgfsetbuttcap%
\pgfsetroundjoin%
\definecolor{currentfill}{rgb}{0.000000,0.000000,0.000000}%
\pgfsetfillcolor{currentfill}%
\pgfsetlinewidth{0.501875pt}%
\definecolor{currentstroke}{rgb}{0.000000,0.000000,0.000000}%
\pgfsetstrokecolor{currentstroke}%
\pgfsetdash{}{0pt}%
\pgfsys@defobject{currentmarker}{\pgfqpoint{-0.055556in}{0.000000in}}{\pgfqpoint{-0.000000in}{0.000000in}}{%
\pgfpathmoveto{\pgfqpoint{-0.000000in}{0.000000in}}%
\pgfpathlineto{\pgfqpoint{-0.055556in}{0.000000in}}%
\pgfusepath{stroke,fill}%
}%
\begin{pgfscope}%
\pgfsys@transformshift{6.800000in}{2.986957in}%
\pgfsys@useobject{currentmarker}{}%
\end{pgfscope}%
\end{pgfscope}%
\begin{pgfscope}%
\definecolor{textcolor}{rgb}{0.000000,0.000000,0.000000}%
\pgfsetstrokecolor{textcolor}%
\pgfsetfillcolor{textcolor}%
\pgftext[x=1.144444in,y=2.986957in,right,]{\color{textcolor}\sffamily\fontsize{20.000000}{24.000000}\selectfont \(\displaystyle {20}\)}%
\end{pgfscope}%
\begin{pgfscope}%
\pgfsetbuttcap%
\pgfsetroundjoin%
\definecolor{currentfill}{rgb}{0.000000,0.000000,0.000000}%
\pgfsetfillcolor{currentfill}%
\pgfsetlinewidth{0.501875pt}%
\definecolor{currentstroke}{rgb}{0.000000,0.000000,0.000000}%
\pgfsetstrokecolor{currentstroke}%
\pgfsetdash{}{0pt}%
\pgfsys@defobject{currentmarker}{\pgfqpoint{0.000000in}{0.000000in}}{\pgfqpoint{0.055556in}{0.000000in}}{%
\pgfpathmoveto{\pgfqpoint{0.000000in}{0.000000in}}%
\pgfpathlineto{\pgfqpoint{0.055556in}{0.000000in}}%
\pgfusepath{stroke,fill}%
}%
\begin{pgfscope}%
\pgfsys@transformshift{1.200000in}{3.821739in}%
\pgfsys@useobject{currentmarker}{}%
\end{pgfscope}%
\end{pgfscope}%
\begin{pgfscope}%
\pgfsetbuttcap%
\pgfsetroundjoin%
\definecolor{currentfill}{rgb}{0.000000,0.000000,0.000000}%
\pgfsetfillcolor{currentfill}%
\pgfsetlinewidth{0.501875pt}%
\definecolor{currentstroke}{rgb}{0.000000,0.000000,0.000000}%
\pgfsetstrokecolor{currentstroke}%
\pgfsetdash{}{0pt}%
\pgfsys@defobject{currentmarker}{\pgfqpoint{-0.055556in}{0.000000in}}{\pgfqpoint{-0.000000in}{0.000000in}}{%
\pgfpathmoveto{\pgfqpoint{-0.000000in}{0.000000in}}%
\pgfpathlineto{\pgfqpoint{-0.055556in}{0.000000in}}%
\pgfusepath{stroke,fill}%
}%
\begin{pgfscope}%
\pgfsys@transformshift{6.800000in}{3.821739in}%
\pgfsys@useobject{currentmarker}{}%
\end{pgfscope}%
\end{pgfscope}%
\begin{pgfscope}%
\definecolor{textcolor}{rgb}{0.000000,0.000000,0.000000}%
\pgfsetstrokecolor{textcolor}%
\pgfsetfillcolor{textcolor}%
\pgftext[x=1.144444in,y=3.821739in,right,]{\color{textcolor}\sffamily\fontsize{20.000000}{24.000000}\selectfont \(\displaystyle {30}\)}%
\end{pgfscope}%
\begin{pgfscope}%
\pgfsetbuttcap%
\pgfsetroundjoin%
\definecolor{currentfill}{rgb}{0.000000,0.000000,0.000000}%
\pgfsetfillcolor{currentfill}%
\pgfsetlinewidth{0.501875pt}%
\definecolor{currentstroke}{rgb}{0.000000,0.000000,0.000000}%
\pgfsetstrokecolor{currentstroke}%
\pgfsetdash{}{0pt}%
\pgfsys@defobject{currentmarker}{\pgfqpoint{0.000000in}{0.000000in}}{\pgfqpoint{0.055556in}{0.000000in}}{%
\pgfpathmoveto{\pgfqpoint{0.000000in}{0.000000in}}%
\pgfpathlineto{\pgfqpoint{0.055556in}{0.000000in}}%
\pgfusepath{stroke,fill}%
}%
\begin{pgfscope}%
\pgfsys@transformshift{1.200000in}{4.656522in}%
\pgfsys@useobject{currentmarker}{}%
\end{pgfscope}%
\end{pgfscope}%
\begin{pgfscope}%
\pgfsetbuttcap%
\pgfsetroundjoin%
\definecolor{currentfill}{rgb}{0.000000,0.000000,0.000000}%
\pgfsetfillcolor{currentfill}%
\pgfsetlinewidth{0.501875pt}%
\definecolor{currentstroke}{rgb}{0.000000,0.000000,0.000000}%
\pgfsetstrokecolor{currentstroke}%
\pgfsetdash{}{0pt}%
\pgfsys@defobject{currentmarker}{\pgfqpoint{-0.055556in}{0.000000in}}{\pgfqpoint{-0.000000in}{0.000000in}}{%
\pgfpathmoveto{\pgfqpoint{-0.000000in}{0.000000in}}%
\pgfpathlineto{\pgfqpoint{-0.055556in}{0.000000in}}%
\pgfusepath{stroke,fill}%
}%
\begin{pgfscope}%
\pgfsys@transformshift{6.800000in}{4.656522in}%
\pgfsys@useobject{currentmarker}{}%
\end{pgfscope}%
\end{pgfscope}%
\begin{pgfscope}%
\definecolor{textcolor}{rgb}{0.000000,0.000000,0.000000}%
\pgfsetstrokecolor{textcolor}%
\pgfsetfillcolor{textcolor}%
\pgftext[x=1.144444in,y=4.656522in,right,]{\color{textcolor}\sffamily\fontsize{20.000000}{24.000000}\selectfont \(\displaystyle {40}\)}%
\end{pgfscope}%
\begin{pgfscope}%
\pgfsetbuttcap%
\pgfsetroundjoin%
\definecolor{currentfill}{rgb}{0.000000,0.000000,0.000000}%
\pgfsetfillcolor{currentfill}%
\pgfsetlinewidth{0.501875pt}%
\definecolor{currentstroke}{rgb}{0.000000,0.000000,0.000000}%
\pgfsetstrokecolor{currentstroke}%
\pgfsetdash{}{0pt}%
\pgfsys@defobject{currentmarker}{\pgfqpoint{0.000000in}{0.000000in}}{\pgfqpoint{0.055556in}{0.000000in}}{%
\pgfpathmoveto{\pgfqpoint{0.000000in}{0.000000in}}%
\pgfpathlineto{\pgfqpoint{0.055556in}{0.000000in}}%
\pgfusepath{stroke,fill}%
}%
\begin{pgfscope}%
\pgfsys@transformshift{1.200000in}{5.491304in}%
\pgfsys@useobject{currentmarker}{}%
\end{pgfscope}%
\end{pgfscope}%
\begin{pgfscope}%
\pgfsetbuttcap%
\pgfsetroundjoin%
\definecolor{currentfill}{rgb}{0.000000,0.000000,0.000000}%
\pgfsetfillcolor{currentfill}%
\pgfsetlinewidth{0.501875pt}%
\definecolor{currentstroke}{rgb}{0.000000,0.000000,0.000000}%
\pgfsetstrokecolor{currentstroke}%
\pgfsetdash{}{0pt}%
\pgfsys@defobject{currentmarker}{\pgfqpoint{-0.055556in}{0.000000in}}{\pgfqpoint{-0.000000in}{0.000000in}}{%
\pgfpathmoveto{\pgfqpoint{-0.000000in}{0.000000in}}%
\pgfpathlineto{\pgfqpoint{-0.055556in}{0.000000in}}%
\pgfusepath{stroke,fill}%
}%
\begin{pgfscope}%
\pgfsys@transformshift{6.800000in}{5.491304in}%
\pgfsys@useobject{currentmarker}{}%
\end{pgfscope}%
\end{pgfscope}%
\begin{pgfscope}%
\definecolor{textcolor}{rgb}{0.000000,0.000000,0.000000}%
\pgfsetstrokecolor{textcolor}%
\pgfsetfillcolor{textcolor}%
\pgftext[x=1.144444in,y=5.491304in,right,]{\color{textcolor}\sffamily\fontsize{20.000000}{24.000000}\selectfont \(\displaystyle {50}\)}%
\end{pgfscope}%
\begin{pgfscope}%
\definecolor{textcolor}{rgb}{0.000000,0.000000,0.000000}%
\pgfsetstrokecolor{textcolor}%
\pgfsetfillcolor{textcolor}%
\pgftext[x=0.810785in,y=3.300000in,,bottom,rotate=90.000000]{\color{textcolor}\sffamily\fontsize{20.000000}{24.000000}\selectfont \(\displaystyle \mathrm{Voltage}/\si{mV}\)}%
\end{pgfscope}%
\begin{pgfscope}%
\pgfsetbuttcap%
\pgfsetmiterjoin%
\definecolor{currentfill}{rgb}{1.000000,1.000000,1.000000}%
\pgfsetfillcolor{currentfill}%
\pgfsetlinewidth{1.003750pt}%
\definecolor{currentstroke}{rgb}{0.000000,0.000000,0.000000}%
\pgfsetstrokecolor{currentstroke}%
\pgfsetdash{}{0pt}%
\pgfpathmoveto{\pgfqpoint{4.066020in}{4.959484in}}%
\pgfpathlineto{\pgfqpoint{6.633333in}{4.959484in}}%
\pgfpathlineto{\pgfqpoint{6.633333in}{5.533333in}}%
\pgfpathlineto{\pgfqpoint{4.066020in}{5.533333in}}%
\pgfpathclose%
\pgfusepath{stroke,fill}%
\end{pgfscope}%
\begin{pgfscope}%
\pgfsetrectcap%
\pgfsetroundjoin%
\pgfsetlinewidth{2.007500pt}%
\definecolor{currentstroke}{rgb}{0.000000,0.000000,1.000000}%
\pgfsetstrokecolor{currentstroke}%
\pgfsetdash{}{0pt}%
\pgfpathmoveto{\pgfqpoint{4.299353in}{5.276697in}}%
\pgfpathlineto{\pgfqpoint{4.766020in}{5.276697in}}%
\pgfusepath{stroke}%
\end{pgfscope}%
\begin{pgfscope}%
\definecolor{textcolor}{rgb}{0.000000,0.000000,0.000000}%
\pgfsetstrokecolor{textcolor}%
\pgfsetfillcolor{textcolor}%
\pgftext[x=5.132687in,y=5.160031in,left,base]{\color{textcolor}\sffamily\fontsize{24.000000}{28.800000}\selectfont Waveform}%
\end{pgfscope}%
\end{pgfpicture}%
\makeatother%
\endgroup%
}
    \caption{\label{fig:input} Data input}
\end{figure}
\end{minipage}
\begin{minipage}[b]{.5\textwidth}
\begin{figure}[H]
    \centering
    \resizebox{\textwidth}{!}{%% Creator: Matplotlib, PGF backend
%%
%% To include the figure in your LaTeX document, write
%%   \input{<filename>.pgf}
%%
%% Make sure the required packages are loaded in your preamble
%%   \usepackage{pgf}
%%
%% and, on pdftex
%%   \usepackage[utf8]{inputenc}\DeclareUnicodeCharacter{2212}{-}
%%
%% or, on luatex and xetex
%%   \usepackage{unicode-math}
%%
%% Figures using additional raster images can only be included by \input if
%% they are in the same directory as the main LaTeX file. For loading figures
%% from other directories you can use the `import` package
%%   \usepackage{import}
%%
%% and then include the figures with
%%   \import{<path to file>}{<filename>.pgf}
%%
%% Matplotlib used the following preamble
%%   \usepackage[detect-all,locale=DE]{siunitx}
%%
\begingroup%
\makeatletter%
\begin{pgfpicture}%
\pgfpathrectangle{\pgfpointorigin}{\pgfqpoint{12.000000in}{6.000000in}}%
\pgfusepath{use as bounding box, clip}%
\begin{pgfscope}%
\pgfsetbuttcap%
\pgfsetmiterjoin%
\definecolor{currentfill}{rgb}{1.000000,1.000000,1.000000}%
\pgfsetfillcolor{currentfill}%
\pgfsetlinewidth{0.000000pt}%
\definecolor{currentstroke}{rgb}{1.000000,1.000000,1.000000}%
\pgfsetstrokecolor{currentstroke}%
\pgfsetdash{}{0pt}%
\pgfpathmoveto{\pgfqpoint{0.000000in}{0.000000in}}%
\pgfpathlineto{\pgfqpoint{12.000000in}{0.000000in}}%
\pgfpathlineto{\pgfqpoint{12.000000in}{6.000000in}}%
\pgfpathlineto{\pgfqpoint{0.000000in}{6.000000in}}%
\pgfpathclose%
\pgfusepath{fill}%
\end{pgfscope}%
\begin{pgfscope}%
\pgfsetbuttcap%
\pgfsetmiterjoin%
\definecolor{currentfill}{rgb}{1.000000,1.000000,1.000000}%
\pgfsetfillcolor{currentfill}%
\pgfsetlinewidth{0.000000pt}%
\definecolor{currentstroke}{rgb}{0.000000,0.000000,0.000000}%
\pgfsetstrokecolor{currentstroke}%
\pgfsetstrokeopacity{0.000000}%
\pgfsetdash{}{0pt}%
\pgfpathmoveto{\pgfqpoint{1.800000in}{0.900000in}}%
\pgfpathlineto{\pgfqpoint{10.200000in}{0.900000in}}%
\pgfpathlineto{\pgfqpoint{10.200000in}{5.700000in}}%
\pgfpathlineto{\pgfqpoint{1.800000in}{5.700000in}}%
\pgfpathclose%
\pgfusepath{fill}%
\end{pgfscope}%
\begin{pgfscope}%
\pgfsetbuttcap%
\pgfsetroundjoin%
\definecolor{currentfill}{rgb}{0.000000,0.000000,0.000000}%
\pgfsetfillcolor{currentfill}%
\pgfsetlinewidth{0.803000pt}%
\definecolor{currentstroke}{rgb}{0.000000,0.000000,0.000000}%
\pgfsetstrokecolor{currentstroke}%
\pgfsetdash{}{0pt}%
\pgfsys@defobject{currentmarker}{\pgfqpoint{0.000000in}{-0.048611in}}{\pgfqpoint{0.000000in}{0.000000in}}{%
\pgfpathmoveto{\pgfqpoint{0.000000in}{0.000000in}}%
\pgfpathlineto{\pgfqpoint{0.000000in}{-0.048611in}}%
\pgfusepath{stroke,fill}%
}%
\begin{pgfscope}%
\pgfsys@transformshift{1.800000in}{0.900000in}%
\pgfsys@useobject{currentmarker}{}%
\end{pgfscope}%
\end{pgfscope}%
\begin{pgfscope}%
\definecolor{textcolor}{rgb}{0.000000,0.000000,0.000000}%
\pgfsetstrokecolor{textcolor}%
\pgfsetfillcolor{textcolor}%
\pgftext[x=1.800000in,y=0.802778in,,top]{\color{textcolor}\sffamily\fontsize{20.000000}{24.000000}\selectfont \(\displaystyle {0}\)}%
\end{pgfscope}%
\begin{pgfscope}%
\pgfsetbuttcap%
\pgfsetroundjoin%
\definecolor{currentfill}{rgb}{0.000000,0.000000,0.000000}%
\pgfsetfillcolor{currentfill}%
\pgfsetlinewidth{0.803000pt}%
\definecolor{currentstroke}{rgb}{0.000000,0.000000,0.000000}%
\pgfsetstrokecolor{currentstroke}%
\pgfsetdash{}{0pt}%
\pgfsys@defobject{currentmarker}{\pgfqpoint{0.000000in}{-0.048611in}}{\pgfqpoint{0.000000in}{0.000000in}}{%
\pgfpathmoveto{\pgfqpoint{0.000000in}{0.000000in}}%
\pgfpathlineto{\pgfqpoint{0.000000in}{-0.048611in}}%
\pgfusepath{stroke,fill}%
}%
\begin{pgfscope}%
\pgfsys@transformshift{3.432653in}{0.900000in}%
\pgfsys@useobject{currentmarker}{}%
\end{pgfscope}%
\end{pgfscope}%
\begin{pgfscope}%
\definecolor{textcolor}{rgb}{0.000000,0.000000,0.000000}%
\pgfsetstrokecolor{textcolor}%
\pgfsetfillcolor{textcolor}%
\pgftext[x=3.432653in,y=0.802778in,,top]{\color{textcolor}\sffamily\fontsize{20.000000}{24.000000}\selectfont \(\displaystyle {200}\)}%
\end{pgfscope}%
\begin{pgfscope}%
\pgfsetbuttcap%
\pgfsetroundjoin%
\definecolor{currentfill}{rgb}{0.000000,0.000000,0.000000}%
\pgfsetfillcolor{currentfill}%
\pgfsetlinewidth{0.803000pt}%
\definecolor{currentstroke}{rgb}{0.000000,0.000000,0.000000}%
\pgfsetstrokecolor{currentstroke}%
\pgfsetdash{}{0pt}%
\pgfsys@defobject{currentmarker}{\pgfqpoint{0.000000in}{-0.048611in}}{\pgfqpoint{0.000000in}{0.000000in}}{%
\pgfpathmoveto{\pgfqpoint{0.000000in}{0.000000in}}%
\pgfpathlineto{\pgfqpoint{0.000000in}{-0.048611in}}%
\pgfusepath{stroke,fill}%
}%
\begin{pgfscope}%
\pgfsys@transformshift{5.065306in}{0.900000in}%
\pgfsys@useobject{currentmarker}{}%
\end{pgfscope}%
\end{pgfscope}%
\begin{pgfscope}%
\definecolor{textcolor}{rgb}{0.000000,0.000000,0.000000}%
\pgfsetstrokecolor{textcolor}%
\pgfsetfillcolor{textcolor}%
\pgftext[x=5.065306in,y=0.802778in,,top]{\color{textcolor}\sffamily\fontsize{20.000000}{24.000000}\selectfont \(\displaystyle {400}\)}%
\end{pgfscope}%
\begin{pgfscope}%
\pgfsetbuttcap%
\pgfsetroundjoin%
\definecolor{currentfill}{rgb}{0.000000,0.000000,0.000000}%
\pgfsetfillcolor{currentfill}%
\pgfsetlinewidth{0.803000pt}%
\definecolor{currentstroke}{rgb}{0.000000,0.000000,0.000000}%
\pgfsetstrokecolor{currentstroke}%
\pgfsetdash{}{0pt}%
\pgfsys@defobject{currentmarker}{\pgfqpoint{0.000000in}{-0.048611in}}{\pgfqpoint{0.000000in}{0.000000in}}{%
\pgfpathmoveto{\pgfqpoint{0.000000in}{0.000000in}}%
\pgfpathlineto{\pgfqpoint{0.000000in}{-0.048611in}}%
\pgfusepath{stroke,fill}%
}%
\begin{pgfscope}%
\pgfsys@transformshift{6.697959in}{0.900000in}%
\pgfsys@useobject{currentmarker}{}%
\end{pgfscope}%
\end{pgfscope}%
\begin{pgfscope}%
\definecolor{textcolor}{rgb}{0.000000,0.000000,0.000000}%
\pgfsetstrokecolor{textcolor}%
\pgfsetfillcolor{textcolor}%
\pgftext[x=6.697959in,y=0.802778in,,top]{\color{textcolor}\sffamily\fontsize{20.000000}{24.000000}\selectfont \(\displaystyle {600}\)}%
\end{pgfscope}%
\begin{pgfscope}%
\pgfsetbuttcap%
\pgfsetroundjoin%
\definecolor{currentfill}{rgb}{0.000000,0.000000,0.000000}%
\pgfsetfillcolor{currentfill}%
\pgfsetlinewidth{0.803000pt}%
\definecolor{currentstroke}{rgb}{0.000000,0.000000,0.000000}%
\pgfsetstrokecolor{currentstroke}%
\pgfsetdash{}{0pt}%
\pgfsys@defobject{currentmarker}{\pgfqpoint{0.000000in}{-0.048611in}}{\pgfqpoint{0.000000in}{0.000000in}}{%
\pgfpathmoveto{\pgfqpoint{0.000000in}{0.000000in}}%
\pgfpathlineto{\pgfqpoint{0.000000in}{-0.048611in}}%
\pgfusepath{stroke,fill}%
}%
\begin{pgfscope}%
\pgfsys@transformshift{8.330612in}{0.900000in}%
\pgfsys@useobject{currentmarker}{}%
\end{pgfscope}%
\end{pgfscope}%
\begin{pgfscope}%
\definecolor{textcolor}{rgb}{0.000000,0.000000,0.000000}%
\pgfsetstrokecolor{textcolor}%
\pgfsetfillcolor{textcolor}%
\pgftext[x=8.330612in,y=0.802778in,,top]{\color{textcolor}\sffamily\fontsize{20.000000}{24.000000}\selectfont \(\displaystyle {800}\)}%
\end{pgfscope}%
\begin{pgfscope}%
\pgfsetbuttcap%
\pgfsetroundjoin%
\definecolor{currentfill}{rgb}{0.000000,0.000000,0.000000}%
\pgfsetfillcolor{currentfill}%
\pgfsetlinewidth{0.803000pt}%
\definecolor{currentstroke}{rgb}{0.000000,0.000000,0.000000}%
\pgfsetstrokecolor{currentstroke}%
\pgfsetdash{}{0pt}%
\pgfsys@defobject{currentmarker}{\pgfqpoint{0.000000in}{-0.048611in}}{\pgfqpoint{0.000000in}{0.000000in}}{%
\pgfpathmoveto{\pgfqpoint{0.000000in}{0.000000in}}%
\pgfpathlineto{\pgfqpoint{0.000000in}{-0.048611in}}%
\pgfusepath{stroke,fill}%
}%
\begin{pgfscope}%
\pgfsys@transformshift{9.963265in}{0.900000in}%
\pgfsys@useobject{currentmarker}{}%
\end{pgfscope}%
\end{pgfscope}%
\begin{pgfscope}%
\definecolor{textcolor}{rgb}{0.000000,0.000000,0.000000}%
\pgfsetstrokecolor{textcolor}%
\pgfsetfillcolor{textcolor}%
\pgftext[x=9.963265in,y=0.802778in,,top]{\color{textcolor}\sffamily\fontsize{20.000000}{24.000000}\selectfont \(\displaystyle {1000}\)}%
\end{pgfscope}%
\begin{pgfscope}%
\definecolor{textcolor}{rgb}{0.000000,0.000000,0.000000}%
\pgfsetstrokecolor{textcolor}%
\pgfsetfillcolor{textcolor}%
\pgftext[x=6.000000in,y=0.491155in,,top]{\color{textcolor}\sffamily\fontsize{20.000000}{24.000000}\selectfont \(\displaystyle \mathrm{t}/\si{ns}\)}%
\end{pgfscope}%
\begin{pgfscope}%
\pgfsetbuttcap%
\pgfsetroundjoin%
\definecolor{currentfill}{rgb}{0.000000,0.000000,0.000000}%
\pgfsetfillcolor{currentfill}%
\pgfsetlinewidth{0.803000pt}%
\definecolor{currentstroke}{rgb}{0.000000,0.000000,0.000000}%
\pgfsetstrokecolor{currentstroke}%
\pgfsetdash{}{0pt}%
\pgfsys@defobject{currentmarker}{\pgfqpoint{-0.048611in}{0.000000in}}{\pgfqpoint{-0.000000in}{0.000000in}}{%
\pgfpathmoveto{\pgfqpoint{-0.000000in}{0.000000in}}%
\pgfpathlineto{\pgfqpoint{-0.048611in}{0.000000in}}%
\pgfusepath{stroke,fill}%
}%
\begin{pgfscope}%
\pgfsys@transformshift{1.800000in}{0.900000in}%
\pgfsys@useobject{currentmarker}{}%
\end{pgfscope}%
\end{pgfscope}%
\begin{pgfscope}%
\definecolor{textcolor}{rgb}{0.000000,0.000000,0.000000}%
\pgfsetstrokecolor{textcolor}%
\pgfsetfillcolor{textcolor}%
\pgftext[x=1.360215in, y=0.799981in, left, base]{\color{textcolor}\sffamily\fontsize{20.000000}{24.000000}\selectfont \(\displaystyle {0.0}\)}%
\end{pgfscope}%
\begin{pgfscope}%
\pgfsetbuttcap%
\pgfsetroundjoin%
\definecolor{currentfill}{rgb}{0.000000,0.000000,0.000000}%
\pgfsetfillcolor{currentfill}%
\pgfsetlinewidth{0.803000pt}%
\definecolor{currentstroke}{rgb}{0.000000,0.000000,0.000000}%
\pgfsetstrokecolor{currentstroke}%
\pgfsetdash{}{0pt}%
\pgfsys@defobject{currentmarker}{\pgfqpoint{-0.048611in}{0.000000in}}{\pgfqpoint{-0.000000in}{0.000000in}}{%
\pgfpathmoveto{\pgfqpoint{-0.000000in}{0.000000in}}%
\pgfpathlineto{\pgfqpoint{-0.048611in}{0.000000in}}%
\pgfusepath{stroke,fill}%
}%
\begin{pgfscope}%
\pgfsys@transformshift{1.800000in}{1.796496in}%
\pgfsys@useobject{currentmarker}{}%
\end{pgfscope}%
\end{pgfscope}%
\begin{pgfscope}%
\definecolor{textcolor}{rgb}{0.000000,0.000000,0.000000}%
\pgfsetstrokecolor{textcolor}%
\pgfsetfillcolor{textcolor}%
\pgftext[x=1.360215in, y=1.696476in, left, base]{\color{textcolor}\sffamily\fontsize{20.000000}{24.000000}\selectfont \(\displaystyle {0.2}\)}%
\end{pgfscope}%
\begin{pgfscope}%
\pgfsetbuttcap%
\pgfsetroundjoin%
\definecolor{currentfill}{rgb}{0.000000,0.000000,0.000000}%
\pgfsetfillcolor{currentfill}%
\pgfsetlinewidth{0.803000pt}%
\definecolor{currentstroke}{rgb}{0.000000,0.000000,0.000000}%
\pgfsetstrokecolor{currentstroke}%
\pgfsetdash{}{0pt}%
\pgfsys@defobject{currentmarker}{\pgfqpoint{-0.048611in}{0.000000in}}{\pgfqpoint{-0.000000in}{0.000000in}}{%
\pgfpathmoveto{\pgfqpoint{-0.000000in}{0.000000in}}%
\pgfpathlineto{\pgfqpoint{-0.048611in}{0.000000in}}%
\pgfusepath{stroke,fill}%
}%
\begin{pgfscope}%
\pgfsys@transformshift{1.800000in}{2.692991in}%
\pgfsys@useobject{currentmarker}{}%
\end{pgfscope}%
\end{pgfscope}%
\begin{pgfscope}%
\definecolor{textcolor}{rgb}{0.000000,0.000000,0.000000}%
\pgfsetstrokecolor{textcolor}%
\pgfsetfillcolor{textcolor}%
\pgftext[x=1.360215in, y=2.592972in, left, base]{\color{textcolor}\sffamily\fontsize{20.000000}{24.000000}\selectfont \(\displaystyle {0.4}\)}%
\end{pgfscope}%
\begin{pgfscope}%
\pgfsetbuttcap%
\pgfsetroundjoin%
\definecolor{currentfill}{rgb}{0.000000,0.000000,0.000000}%
\pgfsetfillcolor{currentfill}%
\pgfsetlinewidth{0.803000pt}%
\definecolor{currentstroke}{rgb}{0.000000,0.000000,0.000000}%
\pgfsetstrokecolor{currentstroke}%
\pgfsetdash{}{0pt}%
\pgfsys@defobject{currentmarker}{\pgfqpoint{-0.048611in}{0.000000in}}{\pgfqpoint{-0.000000in}{0.000000in}}{%
\pgfpathmoveto{\pgfqpoint{-0.000000in}{0.000000in}}%
\pgfpathlineto{\pgfqpoint{-0.048611in}{0.000000in}}%
\pgfusepath{stroke,fill}%
}%
\begin{pgfscope}%
\pgfsys@transformshift{1.800000in}{3.589487in}%
\pgfsys@useobject{currentmarker}{}%
\end{pgfscope}%
\end{pgfscope}%
\begin{pgfscope}%
\definecolor{textcolor}{rgb}{0.000000,0.000000,0.000000}%
\pgfsetstrokecolor{textcolor}%
\pgfsetfillcolor{textcolor}%
\pgftext[x=1.360215in, y=3.489468in, left, base]{\color{textcolor}\sffamily\fontsize{20.000000}{24.000000}\selectfont \(\displaystyle {0.6}\)}%
\end{pgfscope}%
\begin{pgfscope}%
\pgfsetbuttcap%
\pgfsetroundjoin%
\definecolor{currentfill}{rgb}{0.000000,0.000000,0.000000}%
\pgfsetfillcolor{currentfill}%
\pgfsetlinewidth{0.803000pt}%
\definecolor{currentstroke}{rgb}{0.000000,0.000000,0.000000}%
\pgfsetstrokecolor{currentstroke}%
\pgfsetdash{}{0pt}%
\pgfsys@defobject{currentmarker}{\pgfqpoint{-0.048611in}{0.000000in}}{\pgfqpoint{-0.000000in}{0.000000in}}{%
\pgfpathmoveto{\pgfqpoint{-0.000000in}{0.000000in}}%
\pgfpathlineto{\pgfqpoint{-0.048611in}{0.000000in}}%
\pgfusepath{stroke,fill}%
}%
\begin{pgfscope}%
\pgfsys@transformshift{1.800000in}{4.485982in}%
\pgfsys@useobject{currentmarker}{}%
\end{pgfscope}%
\end{pgfscope}%
\begin{pgfscope}%
\definecolor{textcolor}{rgb}{0.000000,0.000000,0.000000}%
\pgfsetstrokecolor{textcolor}%
\pgfsetfillcolor{textcolor}%
\pgftext[x=1.360215in, y=4.385963in, left, base]{\color{textcolor}\sffamily\fontsize{20.000000}{24.000000}\selectfont \(\displaystyle {0.8}\)}%
\end{pgfscope}%
\begin{pgfscope}%
\pgfsetbuttcap%
\pgfsetroundjoin%
\definecolor{currentfill}{rgb}{0.000000,0.000000,0.000000}%
\pgfsetfillcolor{currentfill}%
\pgfsetlinewidth{0.803000pt}%
\definecolor{currentstroke}{rgb}{0.000000,0.000000,0.000000}%
\pgfsetstrokecolor{currentstroke}%
\pgfsetdash{}{0pt}%
\pgfsys@defobject{currentmarker}{\pgfqpoint{-0.048611in}{0.000000in}}{\pgfqpoint{-0.000000in}{0.000000in}}{%
\pgfpathmoveto{\pgfqpoint{-0.000000in}{0.000000in}}%
\pgfpathlineto{\pgfqpoint{-0.048611in}{0.000000in}}%
\pgfusepath{stroke,fill}%
}%
\begin{pgfscope}%
\pgfsys@transformshift{1.800000in}{5.382478in}%
\pgfsys@useobject{currentmarker}{}%
\end{pgfscope}%
\end{pgfscope}%
\begin{pgfscope}%
\definecolor{textcolor}{rgb}{0.000000,0.000000,0.000000}%
\pgfsetstrokecolor{textcolor}%
\pgfsetfillcolor{textcolor}%
\pgftext[x=1.360215in, y=5.282459in, left, base]{\color{textcolor}\sffamily\fontsize{20.000000}{24.000000}\selectfont \(\displaystyle {1.0}\)}%
\end{pgfscope}%
\begin{pgfscope}%
\definecolor{textcolor}{rgb}{0.000000,0.000000,0.000000}%
\pgfsetstrokecolor{textcolor}%
\pgfsetfillcolor{textcolor}%
\pgftext[x=1.304660in,y=3.300000in,,bottom,rotate=90.000000]{\color{textcolor}\sffamily\fontsize{20.000000}{24.000000}\selectfont \(\displaystyle \mathrm{Charge}\)}%
\end{pgfscope}%
\begin{pgfscope}%
\pgfpathrectangle{\pgfqpoint{1.800000in}{0.900000in}}{\pgfqpoint{8.400000in}{4.800000in}}%
\pgfusepath{clip}%
\pgfsetbuttcap%
\pgfsetroundjoin%
\pgfsetlinewidth{2.007500pt}%
\definecolor{currentstroke}{rgb}{1.000000,0.000000,0.000000}%
\pgfsetstrokecolor{currentstroke}%
\pgfsetdash{}{0pt}%
\pgfpathmoveto{\pgfqpoint{3.418777in}{0.900000in}}%
\pgfpathlineto{\pgfqpoint{3.418777in}{3.676722in}}%
\pgfusepath{stroke}%
\end{pgfscope}%
\begin{pgfscope}%
\pgfpathrectangle{\pgfqpoint{1.800000in}{0.900000in}}{\pgfqpoint{8.400000in}{4.800000in}}%
\pgfusepath{clip}%
\pgfsetbuttcap%
\pgfsetroundjoin%
\pgfsetlinewidth{2.007500pt}%
\definecolor{currentstroke}{rgb}{1.000000,0.000000,0.000000}%
\pgfsetstrokecolor{currentstroke}%
\pgfsetdash{}{0pt}%
\pgfpathmoveto{\pgfqpoint{3.535525in}{0.900000in}}%
\pgfpathlineto{\pgfqpoint{3.535525in}{5.188470in}}%
\pgfusepath{stroke}%
\end{pgfscope}%
\begin{pgfscope}%
\pgfpathrectangle{\pgfqpoint{1.800000in}{0.900000in}}{\pgfqpoint{8.400000in}{4.800000in}}%
\pgfusepath{clip}%
\pgfsetbuttcap%
\pgfsetroundjoin%
\pgfsetlinewidth{2.007500pt}%
\definecolor{currentstroke}{rgb}{1.000000,0.000000,0.000000}%
\pgfsetstrokecolor{currentstroke}%
\pgfsetdash{}{0pt}%
\pgfpathmoveto{\pgfqpoint{3.613624in}{0.900000in}}%
\pgfpathlineto{\pgfqpoint{3.613624in}{5.471429in}}%
\pgfusepath{stroke}%
\end{pgfscope}%
\begin{pgfscope}%
\pgfpathrectangle{\pgfqpoint{1.800000in}{0.900000in}}{\pgfqpoint{8.400000in}{4.800000in}}%
\pgfusepath{clip}%
\pgfsetbuttcap%
\pgfsetroundjoin%
\pgfsetlinewidth{2.007500pt}%
\definecolor{currentstroke}{rgb}{1.000000,0.000000,0.000000}%
\pgfsetstrokecolor{currentstroke}%
\pgfsetdash{}{0pt}%
\pgfpathmoveto{\pgfqpoint{3.660706in}{0.900000in}}%
\pgfpathlineto{\pgfqpoint{3.660706in}{3.261364in}}%
\pgfusepath{stroke}%
\end{pgfscope}%
\begin{pgfscope}%
\pgfpathrectangle{\pgfqpoint{1.800000in}{0.900000in}}{\pgfqpoint{8.400000in}{4.800000in}}%
\pgfusepath{clip}%
\pgfsetbuttcap%
\pgfsetroundjoin%
\pgfsetlinewidth{2.007500pt}%
\definecolor{currentstroke}{rgb}{1.000000,0.000000,0.000000}%
\pgfsetstrokecolor{currentstroke}%
\pgfsetdash{}{0pt}%
\pgfpathmoveto{\pgfqpoint{3.814614in}{0.900000in}}%
\pgfpathlineto{\pgfqpoint{3.814614in}{3.158274in}}%
\pgfusepath{stroke}%
\end{pgfscope}%
\begin{pgfscope}%
\pgfsetrectcap%
\pgfsetmiterjoin%
\pgfsetlinewidth{0.803000pt}%
\definecolor{currentstroke}{rgb}{0.000000,0.000000,0.000000}%
\pgfsetstrokecolor{currentstroke}%
\pgfsetdash{}{0pt}%
\pgfpathmoveto{\pgfqpoint{1.800000in}{0.900000in}}%
\pgfpathlineto{\pgfqpoint{1.800000in}{5.700000in}}%
\pgfusepath{stroke}%
\end{pgfscope}%
\begin{pgfscope}%
\pgfsetrectcap%
\pgfsetmiterjoin%
\pgfsetlinewidth{0.803000pt}%
\definecolor{currentstroke}{rgb}{0.000000,0.000000,0.000000}%
\pgfsetstrokecolor{currentstroke}%
\pgfsetdash{}{0pt}%
\pgfpathmoveto{\pgfqpoint{10.200000in}{0.900000in}}%
\pgfpathlineto{\pgfqpoint{10.200000in}{5.700000in}}%
\pgfusepath{stroke}%
\end{pgfscope}%
\begin{pgfscope}%
\pgfsetrectcap%
\pgfsetmiterjoin%
\pgfsetlinewidth{0.803000pt}%
\definecolor{currentstroke}{rgb}{0.000000,0.000000,0.000000}%
\pgfsetstrokecolor{currentstroke}%
\pgfsetdash{}{0pt}%
\pgfpathmoveto{\pgfqpoint{1.800000in}{0.900000in}}%
\pgfpathlineto{\pgfqpoint{10.200000in}{0.900000in}}%
\pgfusepath{stroke}%
\end{pgfscope}%
\begin{pgfscope}%
\pgfsetrectcap%
\pgfsetmiterjoin%
\pgfsetlinewidth{0.803000pt}%
\definecolor{currentstroke}{rgb}{0.000000,0.000000,0.000000}%
\pgfsetstrokecolor{currentstroke}%
\pgfsetdash{}{0pt}%
\pgfpathmoveto{\pgfqpoint{1.800000in}{5.700000in}}%
\pgfpathlineto{\pgfqpoint{10.200000in}{5.700000in}}%
\pgfusepath{stroke}%
\end{pgfscope}%
\begin{pgfscope}%
\pgfsetbuttcap%
\pgfsetmiterjoin%
\definecolor{currentfill}{rgb}{1.000000,1.000000,1.000000}%
\pgfsetfillcolor{currentfill}%
\pgfsetfillopacity{0.800000}%
\pgfsetlinewidth{1.003750pt}%
\definecolor{currentstroke}{rgb}{0.800000,0.800000,0.800000}%
\pgfsetstrokecolor{currentstroke}%
\pgfsetstrokeopacity{0.800000}%
\pgfsetdash{}{0pt}%
\pgfpathmoveto{\pgfqpoint{8.333767in}{5.082821in}}%
\pgfpathlineto{\pgfqpoint{10.005556in}{5.082821in}}%
\pgfpathquadraticcurveto{\pgfqpoint{10.061111in}{5.082821in}}{\pgfqpoint{10.061111in}{5.138377in}}%
\pgfpathlineto{\pgfqpoint{10.061111in}{5.505556in}}%
\pgfpathquadraticcurveto{\pgfqpoint{10.061111in}{5.561111in}}{\pgfqpoint{10.005556in}{5.561111in}}%
\pgfpathlineto{\pgfqpoint{8.333767in}{5.561111in}}%
\pgfpathquadraticcurveto{\pgfqpoint{8.278212in}{5.561111in}}{\pgfqpoint{8.278212in}{5.505556in}}%
\pgfpathlineto{\pgfqpoint{8.278212in}{5.138377in}}%
\pgfpathquadraticcurveto{\pgfqpoint{8.278212in}{5.082821in}}{\pgfqpoint{8.333767in}{5.082821in}}%
\pgfpathclose%
\pgfusepath{stroke,fill}%
\end{pgfscope}%
\begin{pgfscope}%
\pgfsetbuttcap%
\pgfsetroundjoin%
\pgfsetlinewidth{2.007500pt}%
\definecolor{currentstroke}{rgb}{1.000000,0.000000,0.000000}%
\pgfsetstrokecolor{currentstroke}%
\pgfsetdash{}{0pt}%
\pgfpathmoveto{\pgfqpoint{8.389323in}{5.347184in}}%
\pgfpathlineto{\pgfqpoint{8.944878in}{5.347184in}}%
\pgfusepath{stroke}%
\end{pgfscope}%
\begin{pgfscope}%
\definecolor{textcolor}{rgb}{0.000000,0.000000,0.000000}%
\pgfsetstrokecolor{textcolor}%
\pgfsetfillcolor{textcolor}%
\pgftext[x=9.167101in,y=5.249962in,left,base]{\color{textcolor}\sffamily\fontsize{20.000000}{24.000000}\selectfont Charge}%
\end{pgfscope}%
\end{pgfpicture}%
\makeatother%
\endgroup%
}
    \caption{\label{fig:output} Data output}
\end{figure}
\end{minipage}
\end{figure}

The histogram of $N_{pe}$ when $\mu=5$ shows in figure (see figure~\ref{fig:penum}). 

\begin{figure}[H]
    \centering
    \scalebox{0.4}{%% Creator: Matplotlib, PGF backend
%%
%% To include the figure in your LaTeX document, write
%%   \input{<filename>.pgf}
%%
%% Make sure the required packages are loaded in your preamble
%%   \usepackage{pgf}
%%
%% and, on pdftex
%%   \usepackage[utf8]{inputenc}\DeclareUnicodeCharacter{2212}{-}
%%
%% or, on luatex and xetex
%%   \usepackage{unicode-math}
%%
%% Figures using additional raster images can only be included by \input if
%% they are in the same directory as the main LaTeX file. For loading figures
%% from other directories you can use the `import` package
%%   \usepackage{import}
%%
%% and then include the figures with
%%   \import{<path to file>}{<filename>.pgf}
%%
%% Matplotlib used the following preamble
%%
\begingroup%
\makeatletter%
\begin{pgfpicture}%
\pgfpathrectangle{\pgfpointorigin}{\pgfqpoint{8.000000in}{6.000000in}}%
\pgfusepath{use as bounding box, clip}%
\begin{pgfscope}%
\pgfsetbuttcap%
\pgfsetmiterjoin%
\definecolor{currentfill}{rgb}{1.000000,1.000000,1.000000}%
\pgfsetfillcolor{currentfill}%
\pgfsetlinewidth{0.000000pt}%
\definecolor{currentstroke}{rgb}{1.000000,1.000000,1.000000}%
\pgfsetstrokecolor{currentstroke}%
\pgfsetdash{}{0pt}%
\pgfpathmoveto{\pgfqpoint{0.000000in}{0.000000in}}%
\pgfpathlineto{\pgfqpoint{8.000000in}{0.000000in}}%
\pgfpathlineto{\pgfqpoint{8.000000in}{6.000000in}}%
\pgfpathlineto{\pgfqpoint{0.000000in}{6.000000in}}%
\pgfpathclose%
\pgfusepath{fill}%
\end{pgfscope}%
\begin{pgfscope}%
\pgfsetbuttcap%
\pgfsetmiterjoin%
\definecolor{currentfill}{rgb}{1.000000,1.000000,1.000000}%
\pgfsetfillcolor{currentfill}%
\pgfsetlinewidth{0.000000pt}%
\definecolor{currentstroke}{rgb}{0.000000,0.000000,0.000000}%
\pgfsetstrokecolor{currentstroke}%
\pgfsetstrokeopacity{0.000000}%
\pgfsetdash{}{0pt}%
\pgfpathmoveto{\pgfqpoint{1.200000in}{0.900000in}}%
\pgfpathlineto{\pgfqpoint{6.800000in}{0.900000in}}%
\pgfpathlineto{\pgfqpoint{6.800000in}{5.700000in}}%
\pgfpathlineto{\pgfqpoint{1.200000in}{5.700000in}}%
\pgfpathclose%
\pgfusepath{fill}%
\end{pgfscope}%
\begin{pgfscope}%
\pgfpathrectangle{\pgfqpoint{1.200000in}{0.900000in}}{\pgfqpoint{5.600000in}{4.800000in}}%
\pgfusepath{clip}%
\pgfsetbuttcap%
\pgfsetmiterjoin%
\pgfsetlinewidth{1.003750pt}%
\definecolor{currentstroke}{rgb}{0.000000,0.000000,1.000000}%
\pgfsetstrokecolor{currentstroke}%
\pgfsetdash{}{0pt}%
\pgfpathmoveto{\pgfqpoint{1.550000in}{0.890000in}}%
\pgfpathlineto{\pgfqpoint{1.550000in}{3.960274in}}%
\pgfpathlineto{\pgfqpoint{1.878125in}{3.960274in}}%
\pgfpathlineto{\pgfqpoint{1.878125in}{4.404774in}}%
\pgfpathlineto{\pgfqpoint{2.206250in}{4.404774in}}%
\pgfpathlineto{\pgfqpoint{2.206250in}{4.686403in}}%
\pgfpathlineto{\pgfqpoint{2.534375in}{4.686403in}}%
\pgfpathlineto{\pgfqpoint{2.534375in}{4.800797in}}%
\pgfpathlineto{\pgfqpoint{2.862500in}{4.800797in}}%
\pgfpathlineto{\pgfqpoint{2.862500in}{4.789258in}}%
\pgfpathlineto{\pgfqpoint{3.190625in}{4.789258in}}%
\pgfpathlineto{\pgfqpoint{3.190625in}{4.688948in}}%
\pgfpathlineto{\pgfqpoint{3.518750in}{4.688948in}}%
\pgfpathlineto{\pgfqpoint{3.518750in}{4.537205in}}%
\pgfpathlineto{\pgfqpoint{3.846875in}{4.537205in}}%
\pgfpathlineto{\pgfqpoint{3.846875in}{4.295936in}}%
\pgfpathlineto{\pgfqpoint{4.175000in}{4.295936in}}%
\pgfpathlineto{\pgfqpoint{4.175000in}{3.983249in}}%
\pgfpathlineto{\pgfqpoint{4.503125in}{3.983249in}}%
\pgfpathlineto{\pgfqpoint{4.503125in}{3.551418in}}%
\pgfpathlineto{\pgfqpoint{4.831250in}{3.551418in}}%
\pgfpathlineto{\pgfqpoint{4.831250in}{3.202894in}}%
\pgfpathlineto{\pgfqpoint{5.159375in}{3.202894in}}%
\pgfpathlineto{\pgfqpoint{5.159375in}{2.809278in}}%
\pgfpathlineto{\pgfqpoint{5.487500in}{2.809278in}}%
\pgfpathlineto{\pgfqpoint{5.487500in}{2.045091in}}%
\pgfpathlineto{\pgfqpoint{5.815625in}{2.045091in}}%
\pgfpathlineto{\pgfqpoint{5.815625in}{1.914118in}}%
\pgfpathlineto{\pgfqpoint{6.143750in}{1.914118in}}%
\pgfpathlineto{\pgfqpoint{6.143750in}{0.890000in}}%
\pgfpathmoveto{\pgfqpoint{6.471875in}{0.890000in}}%
\pgfpathlineto{\pgfqpoint{6.471875in}{0.900000in}}%
\pgfpathlineto{\pgfqpoint{6.800000in}{0.900000in}}%
\pgfpathlineto{\pgfqpoint{6.800000in}{0.890000in}}%
\pgfusepath{stroke}%
\end{pgfscope}%
\begin{pgfscope}%
\pgfsetrectcap%
\pgfsetmiterjoin%
\pgfsetlinewidth{1.003750pt}%
\definecolor{currentstroke}{rgb}{0.000000,0.000000,0.000000}%
\pgfsetstrokecolor{currentstroke}%
\pgfsetdash{}{0pt}%
\pgfpathmoveto{\pgfqpoint{1.200000in}{0.900000in}}%
\pgfpathlineto{\pgfqpoint{1.200000in}{5.700000in}}%
\pgfusepath{stroke}%
\end{pgfscope}%
\begin{pgfscope}%
\pgfsetrectcap%
\pgfsetmiterjoin%
\pgfsetlinewidth{1.003750pt}%
\definecolor{currentstroke}{rgb}{0.000000,0.000000,0.000000}%
\pgfsetstrokecolor{currentstroke}%
\pgfsetdash{}{0pt}%
\pgfpathmoveto{\pgfqpoint{6.800000in}{0.900000in}}%
\pgfpathlineto{\pgfqpoint{6.800000in}{5.700000in}}%
\pgfusepath{stroke}%
\end{pgfscope}%
\begin{pgfscope}%
\pgfsetrectcap%
\pgfsetmiterjoin%
\pgfsetlinewidth{1.003750pt}%
\definecolor{currentstroke}{rgb}{0.000000,0.000000,0.000000}%
\pgfsetstrokecolor{currentstroke}%
\pgfsetdash{}{0pt}%
\pgfpathmoveto{\pgfqpoint{1.200000in}{0.900000in}}%
\pgfpathlineto{\pgfqpoint{6.800000in}{0.900000in}}%
\pgfusepath{stroke}%
\end{pgfscope}%
\begin{pgfscope}%
\pgfsetrectcap%
\pgfsetmiterjoin%
\pgfsetlinewidth{1.003750pt}%
\definecolor{currentstroke}{rgb}{0.000000,0.000000,0.000000}%
\pgfsetstrokecolor{currentstroke}%
\pgfsetdash{}{0pt}%
\pgfpathmoveto{\pgfqpoint{1.200000in}{5.700000in}}%
\pgfpathlineto{\pgfqpoint{6.800000in}{5.700000in}}%
\pgfusepath{stroke}%
\end{pgfscope}%
\begin{pgfscope}%
\pgfpathrectangle{\pgfqpoint{1.200000in}{0.900000in}}{\pgfqpoint{5.600000in}{4.800000in}}%
\pgfusepath{clip}%
\pgfsetbuttcap%
\pgfsetroundjoin%
\pgfsetlinewidth{0.501875pt}%
\definecolor{currentstroke}{rgb}{0.000000,0.000000,0.000000}%
\pgfsetstrokecolor{currentstroke}%
\pgfsetdash{{1.000000pt}{3.000000pt}}{0.000000pt}%
\pgfpathmoveto{\pgfqpoint{1.200000in}{0.900000in}}%
\pgfpathlineto{\pgfqpoint{1.200000in}{5.700000in}}%
\pgfusepath{stroke}%
\end{pgfscope}%
\begin{pgfscope}%
\pgfsetbuttcap%
\pgfsetroundjoin%
\definecolor{currentfill}{rgb}{0.000000,0.000000,0.000000}%
\pgfsetfillcolor{currentfill}%
\pgfsetlinewidth{0.501875pt}%
\definecolor{currentstroke}{rgb}{0.000000,0.000000,0.000000}%
\pgfsetstrokecolor{currentstroke}%
\pgfsetdash{}{0pt}%
\pgfsys@defobject{currentmarker}{\pgfqpoint{0.000000in}{0.000000in}}{\pgfqpoint{0.000000in}{0.055556in}}{%
\pgfpathmoveto{\pgfqpoint{0.000000in}{0.000000in}}%
\pgfpathlineto{\pgfqpoint{0.000000in}{0.055556in}}%
\pgfusepath{stroke,fill}%
}%
\begin{pgfscope}%
\pgfsys@transformshift{1.200000in}{0.900000in}%
\pgfsys@useobject{currentmarker}{}%
\end{pgfscope}%
\end{pgfscope}%
\begin{pgfscope}%
\pgfsetbuttcap%
\pgfsetroundjoin%
\definecolor{currentfill}{rgb}{0.000000,0.000000,0.000000}%
\pgfsetfillcolor{currentfill}%
\pgfsetlinewidth{0.501875pt}%
\definecolor{currentstroke}{rgb}{0.000000,0.000000,0.000000}%
\pgfsetstrokecolor{currentstroke}%
\pgfsetdash{}{0pt}%
\pgfsys@defobject{currentmarker}{\pgfqpoint{0.000000in}{-0.055556in}}{\pgfqpoint{0.000000in}{0.000000in}}{%
\pgfpathmoveto{\pgfqpoint{0.000000in}{0.000000in}}%
\pgfpathlineto{\pgfqpoint{0.000000in}{-0.055556in}}%
\pgfusepath{stroke,fill}%
}%
\begin{pgfscope}%
\pgfsys@transformshift{1.200000in}{5.700000in}%
\pgfsys@useobject{currentmarker}{}%
\end{pgfscope}%
\end{pgfscope}%
\begin{pgfscope}%
\definecolor{textcolor}{rgb}{0.000000,0.000000,0.000000}%
\pgfsetstrokecolor{textcolor}%
\pgfsetfillcolor{textcolor}%
\pgftext[x=1.200000in,y=0.844444in,,top]{\color{textcolor}\sffamily\fontsize{20.000000}{24.000000}\selectfont \(\displaystyle {0}\)}%
\end{pgfscope}%
\begin{pgfscope}%
\pgfpathrectangle{\pgfqpoint{1.200000in}{0.900000in}}{\pgfqpoint{5.600000in}{4.800000in}}%
\pgfusepath{clip}%
\pgfsetbuttcap%
\pgfsetroundjoin%
\pgfsetlinewidth{0.501875pt}%
\definecolor{currentstroke}{rgb}{0.000000,0.000000,0.000000}%
\pgfsetstrokecolor{currentstroke}%
\pgfsetdash{{1.000000pt}{3.000000pt}}{0.000000pt}%
\pgfpathmoveto{\pgfqpoint{1.900000in}{0.900000in}}%
\pgfpathlineto{\pgfqpoint{1.900000in}{5.700000in}}%
\pgfusepath{stroke}%
\end{pgfscope}%
\begin{pgfscope}%
\pgfsetbuttcap%
\pgfsetroundjoin%
\definecolor{currentfill}{rgb}{0.000000,0.000000,0.000000}%
\pgfsetfillcolor{currentfill}%
\pgfsetlinewidth{0.501875pt}%
\definecolor{currentstroke}{rgb}{0.000000,0.000000,0.000000}%
\pgfsetstrokecolor{currentstroke}%
\pgfsetdash{}{0pt}%
\pgfsys@defobject{currentmarker}{\pgfqpoint{0.000000in}{0.000000in}}{\pgfqpoint{0.000000in}{0.055556in}}{%
\pgfpathmoveto{\pgfqpoint{0.000000in}{0.000000in}}%
\pgfpathlineto{\pgfqpoint{0.000000in}{0.055556in}}%
\pgfusepath{stroke,fill}%
}%
\begin{pgfscope}%
\pgfsys@transformshift{1.900000in}{0.900000in}%
\pgfsys@useobject{currentmarker}{}%
\end{pgfscope}%
\end{pgfscope}%
\begin{pgfscope}%
\pgfsetbuttcap%
\pgfsetroundjoin%
\definecolor{currentfill}{rgb}{0.000000,0.000000,0.000000}%
\pgfsetfillcolor{currentfill}%
\pgfsetlinewidth{0.501875pt}%
\definecolor{currentstroke}{rgb}{0.000000,0.000000,0.000000}%
\pgfsetstrokecolor{currentstroke}%
\pgfsetdash{}{0pt}%
\pgfsys@defobject{currentmarker}{\pgfqpoint{0.000000in}{-0.055556in}}{\pgfqpoint{0.000000in}{0.000000in}}{%
\pgfpathmoveto{\pgfqpoint{0.000000in}{0.000000in}}%
\pgfpathlineto{\pgfqpoint{0.000000in}{-0.055556in}}%
\pgfusepath{stroke,fill}%
}%
\begin{pgfscope}%
\pgfsys@transformshift{1.900000in}{5.700000in}%
\pgfsys@useobject{currentmarker}{}%
\end{pgfscope}%
\end{pgfscope}%
\begin{pgfscope}%
\definecolor{textcolor}{rgb}{0.000000,0.000000,0.000000}%
\pgfsetstrokecolor{textcolor}%
\pgfsetfillcolor{textcolor}%
\pgftext[x=1.900000in,y=0.844444in,,top]{\color{textcolor}\sffamily\fontsize{20.000000}{24.000000}\selectfont \(\displaystyle {2}\)}%
\end{pgfscope}%
\begin{pgfscope}%
\pgfpathrectangle{\pgfqpoint{1.200000in}{0.900000in}}{\pgfqpoint{5.600000in}{4.800000in}}%
\pgfusepath{clip}%
\pgfsetbuttcap%
\pgfsetroundjoin%
\pgfsetlinewidth{0.501875pt}%
\definecolor{currentstroke}{rgb}{0.000000,0.000000,0.000000}%
\pgfsetstrokecolor{currentstroke}%
\pgfsetdash{{1.000000pt}{3.000000pt}}{0.000000pt}%
\pgfpathmoveto{\pgfqpoint{2.600000in}{0.900000in}}%
\pgfpathlineto{\pgfqpoint{2.600000in}{5.700000in}}%
\pgfusepath{stroke}%
\end{pgfscope}%
\begin{pgfscope}%
\pgfsetbuttcap%
\pgfsetroundjoin%
\definecolor{currentfill}{rgb}{0.000000,0.000000,0.000000}%
\pgfsetfillcolor{currentfill}%
\pgfsetlinewidth{0.501875pt}%
\definecolor{currentstroke}{rgb}{0.000000,0.000000,0.000000}%
\pgfsetstrokecolor{currentstroke}%
\pgfsetdash{}{0pt}%
\pgfsys@defobject{currentmarker}{\pgfqpoint{0.000000in}{0.000000in}}{\pgfqpoint{0.000000in}{0.055556in}}{%
\pgfpathmoveto{\pgfqpoint{0.000000in}{0.000000in}}%
\pgfpathlineto{\pgfqpoint{0.000000in}{0.055556in}}%
\pgfusepath{stroke,fill}%
}%
\begin{pgfscope}%
\pgfsys@transformshift{2.600000in}{0.900000in}%
\pgfsys@useobject{currentmarker}{}%
\end{pgfscope}%
\end{pgfscope}%
\begin{pgfscope}%
\pgfsetbuttcap%
\pgfsetroundjoin%
\definecolor{currentfill}{rgb}{0.000000,0.000000,0.000000}%
\pgfsetfillcolor{currentfill}%
\pgfsetlinewidth{0.501875pt}%
\definecolor{currentstroke}{rgb}{0.000000,0.000000,0.000000}%
\pgfsetstrokecolor{currentstroke}%
\pgfsetdash{}{0pt}%
\pgfsys@defobject{currentmarker}{\pgfqpoint{0.000000in}{-0.055556in}}{\pgfqpoint{0.000000in}{0.000000in}}{%
\pgfpathmoveto{\pgfqpoint{0.000000in}{0.000000in}}%
\pgfpathlineto{\pgfqpoint{0.000000in}{-0.055556in}}%
\pgfusepath{stroke,fill}%
}%
\begin{pgfscope}%
\pgfsys@transformshift{2.600000in}{5.700000in}%
\pgfsys@useobject{currentmarker}{}%
\end{pgfscope}%
\end{pgfscope}%
\begin{pgfscope}%
\definecolor{textcolor}{rgb}{0.000000,0.000000,0.000000}%
\pgfsetstrokecolor{textcolor}%
\pgfsetfillcolor{textcolor}%
\pgftext[x=2.600000in,y=0.844444in,,top]{\color{textcolor}\sffamily\fontsize{20.000000}{24.000000}\selectfont \(\displaystyle {4}\)}%
\end{pgfscope}%
\begin{pgfscope}%
\pgfpathrectangle{\pgfqpoint{1.200000in}{0.900000in}}{\pgfqpoint{5.600000in}{4.800000in}}%
\pgfusepath{clip}%
\pgfsetbuttcap%
\pgfsetroundjoin%
\pgfsetlinewidth{0.501875pt}%
\definecolor{currentstroke}{rgb}{0.000000,0.000000,0.000000}%
\pgfsetstrokecolor{currentstroke}%
\pgfsetdash{{1.000000pt}{3.000000pt}}{0.000000pt}%
\pgfpathmoveto{\pgfqpoint{3.300000in}{0.900000in}}%
\pgfpathlineto{\pgfqpoint{3.300000in}{5.700000in}}%
\pgfusepath{stroke}%
\end{pgfscope}%
\begin{pgfscope}%
\pgfsetbuttcap%
\pgfsetroundjoin%
\definecolor{currentfill}{rgb}{0.000000,0.000000,0.000000}%
\pgfsetfillcolor{currentfill}%
\pgfsetlinewidth{0.501875pt}%
\definecolor{currentstroke}{rgb}{0.000000,0.000000,0.000000}%
\pgfsetstrokecolor{currentstroke}%
\pgfsetdash{}{0pt}%
\pgfsys@defobject{currentmarker}{\pgfqpoint{0.000000in}{0.000000in}}{\pgfqpoint{0.000000in}{0.055556in}}{%
\pgfpathmoveto{\pgfqpoint{0.000000in}{0.000000in}}%
\pgfpathlineto{\pgfqpoint{0.000000in}{0.055556in}}%
\pgfusepath{stroke,fill}%
}%
\begin{pgfscope}%
\pgfsys@transformshift{3.300000in}{0.900000in}%
\pgfsys@useobject{currentmarker}{}%
\end{pgfscope}%
\end{pgfscope}%
\begin{pgfscope}%
\pgfsetbuttcap%
\pgfsetroundjoin%
\definecolor{currentfill}{rgb}{0.000000,0.000000,0.000000}%
\pgfsetfillcolor{currentfill}%
\pgfsetlinewidth{0.501875pt}%
\definecolor{currentstroke}{rgb}{0.000000,0.000000,0.000000}%
\pgfsetstrokecolor{currentstroke}%
\pgfsetdash{}{0pt}%
\pgfsys@defobject{currentmarker}{\pgfqpoint{0.000000in}{-0.055556in}}{\pgfqpoint{0.000000in}{0.000000in}}{%
\pgfpathmoveto{\pgfqpoint{0.000000in}{0.000000in}}%
\pgfpathlineto{\pgfqpoint{0.000000in}{-0.055556in}}%
\pgfusepath{stroke,fill}%
}%
\begin{pgfscope}%
\pgfsys@transformshift{3.300000in}{5.700000in}%
\pgfsys@useobject{currentmarker}{}%
\end{pgfscope}%
\end{pgfscope}%
\begin{pgfscope}%
\definecolor{textcolor}{rgb}{0.000000,0.000000,0.000000}%
\pgfsetstrokecolor{textcolor}%
\pgfsetfillcolor{textcolor}%
\pgftext[x=3.300000in,y=0.844444in,,top]{\color{textcolor}\sffamily\fontsize{20.000000}{24.000000}\selectfont \(\displaystyle {6}\)}%
\end{pgfscope}%
\begin{pgfscope}%
\pgfpathrectangle{\pgfqpoint{1.200000in}{0.900000in}}{\pgfqpoint{5.600000in}{4.800000in}}%
\pgfusepath{clip}%
\pgfsetbuttcap%
\pgfsetroundjoin%
\pgfsetlinewidth{0.501875pt}%
\definecolor{currentstroke}{rgb}{0.000000,0.000000,0.000000}%
\pgfsetstrokecolor{currentstroke}%
\pgfsetdash{{1.000000pt}{3.000000pt}}{0.000000pt}%
\pgfpathmoveto{\pgfqpoint{4.000000in}{0.900000in}}%
\pgfpathlineto{\pgfqpoint{4.000000in}{5.700000in}}%
\pgfusepath{stroke}%
\end{pgfscope}%
\begin{pgfscope}%
\pgfsetbuttcap%
\pgfsetroundjoin%
\definecolor{currentfill}{rgb}{0.000000,0.000000,0.000000}%
\pgfsetfillcolor{currentfill}%
\pgfsetlinewidth{0.501875pt}%
\definecolor{currentstroke}{rgb}{0.000000,0.000000,0.000000}%
\pgfsetstrokecolor{currentstroke}%
\pgfsetdash{}{0pt}%
\pgfsys@defobject{currentmarker}{\pgfqpoint{0.000000in}{0.000000in}}{\pgfqpoint{0.000000in}{0.055556in}}{%
\pgfpathmoveto{\pgfqpoint{0.000000in}{0.000000in}}%
\pgfpathlineto{\pgfqpoint{0.000000in}{0.055556in}}%
\pgfusepath{stroke,fill}%
}%
\begin{pgfscope}%
\pgfsys@transformshift{4.000000in}{0.900000in}%
\pgfsys@useobject{currentmarker}{}%
\end{pgfscope}%
\end{pgfscope}%
\begin{pgfscope}%
\pgfsetbuttcap%
\pgfsetroundjoin%
\definecolor{currentfill}{rgb}{0.000000,0.000000,0.000000}%
\pgfsetfillcolor{currentfill}%
\pgfsetlinewidth{0.501875pt}%
\definecolor{currentstroke}{rgb}{0.000000,0.000000,0.000000}%
\pgfsetstrokecolor{currentstroke}%
\pgfsetdash{}{0pt}%
\pgfsys@defobject{currentmarker}{\pgfqpoint{0.000000in}{-0.055556in}}{\pgfqpoint{0.000000in}{0.000000in}}{%
\pgfpathmoveto{\pgfqpoint{0.000000in}{0.000000in}}%
\pgfpathlineto{\pgfqpoint{0.000000in}{-0.055556in}}%
\pgfusepath{stroke,fill}%
}%
\begin{pgfscope}%
\pgfsys@transformshift{4.000000in}{5.700000in}%
\pgfsys@useobject{currentmarker}{}%
\end{pgfscope}%
\end{pgfscope}%
\begin{pgfscope}%
\definecolor{textcolor}{rgb}{0.000000,0.000000,0.000000}%
\pgfsetstrokecolor{textcolor}%
\pgfsetfillcolor{textcolor}%
\pgftext[x=4.000000in,y=0.844444in,,top]{\color{textcolor}\sffamily\fontsize{20.000000}{24.000000}\selectfont \(\displaystyle {8}\)}%
\end{pgfscope}%
\begin{pgfscope}%
\pgfpathrectangle{\pgfqpoint{1.200000in}{0.900000in}}{\pgfqpoint{5.600000in}{4.800000in}}%
\pgfusepath{clip}%
\pgfsetbuttcap%
\pgfsetroundjoin%
\pgfsetlinewidth{0.501875pt}%
\definecolor{currentstroke}{rgb}{0.000000,0.000000,0.000000}%
\pgfsetstrokecolor{currentstroke}%
\pgfsetdash{{1.000000pt}{3.000000pt}}{0.000000pt}%
\pgfpathmoveto{\pgfqpoint{4.700000in}{0.900000in}}%
\pgfpathlineto{\pgfqpoint{4.700000in}{5.700000in}}%
\pgfusepath{stroke}%
\end{pgfscope}%
\begin{pgfscope}%
\pgfsetbuttcap%
\pgfsetroundjoin%
\definecolor{currentfill}{rgb}{0.000000,0.000000,0.000000}%
\pgfsetfillcolor{currentfill}%
\pgfsetlinewidth{0.501875pt}%
\definecolor{currentstroke}{rgb}{0.000000,0.000000,0.000000}%
\pgfsetstrokecolor{currentstroke}%
\pgfsetdash{}{0pt}%
\pgfsys@defobject{currentmarker}{\pgfqpoint{0.000000in}{0.000000in}}{\pgfqpoint{0.000000in}{0.055556in}}{%
\pgfpathmoveto{\pgfqpoint{0.000000in}{0.000000in}}%
\pgfpathlineto{\pgfqpoint{0.000000in}{0.055556in}}%
\pgfusepath{stroke,fill}%
}%
\begin{pgfscope}%
\pgfsys@transformshift{4.700000in}{0.900000in}%
\pgfsys@useobject{currentmarker}{}%
\end{pgfscope}%
\end{pgfscope}%
\begin{pgfscope}%
\pgfsetbuttcap%
\pgfsetroundjoin%
\definecolor{currentfill}{rgb}{0.000000,0.000000,0.000000}%
\pgfsetfillcolor{currentfill}%
\pgfsetlinewidth{0.501875pt}%
\definecolor{currentstroke}{rgb}{0.000000,0.000000,0.000000}%
\pgfsetstrokecolor{currentstroke}%
\pgfsetdash{}{0pt}%
\pgfsys@defobject{currentmarker}{\pgfqpoint{0.000000in}{-0.055556in}}{\pgfqpoint{0.000000in}{0.000000in}}{%
\pgfpathmoveto{\pgfqpoint{0.000000in}{0.000000in}}%
\pgfpathlineto{\pgfqpoint{0.000000in}{-0.055556in}}%
\pgfusepath{stroke,fill}%
}%
\begin{pgfscope}%
\pgfsys@transformshift{4.700000in}{5.700000in}%
\pgfsys@useobject{currentmarker}{}%
\end{pgfscope}%
\end{pgfscope}%
\begin{pgfscope}%
\definecolor{textcolor}{rgb}{0.000000,0.000000,0.000000}%
\pgfsetstrokecolor{textcolor}%
\pgfsetfillcolor{textcolor}%
\pgftext[x=4.700000in,y=0.844444in,,top]{\color{textcolor}\sffamily\fontsize{20.000000}{24.000000}\selectfont \(\displaystyle {10}\)}%
\end{pgfscope}%
\begin{pgfscope}%
\pgfpathrectangle{\pgfqpoint{1.200000in}{0.900000in}}{\pgfqpoint{5.600000in}{4.800000in}}%
\pgfusepath{clip}%
\pgfsetbuttcap%
\pgfsetroundjoin%
\pgfsetlinewidth{0.501875pt}%
\definecolor{currentstroke}{rgb}{0.000000,0.000000,0.000000}%
\pgfsetstrokecolor{currentstroke}%
\pgfsetdash{{1.000000pt}{3.000000pt}}{0.000000pt}%
\pgfpathmoveto{\pgfqpoint{5.400000in}{0.900000in}}%
\pgfpathlineto{\pgfqpoint{5.400000in}{5.700000in}}%
\pgfusepath{stroke}%
\end{pgfscope}%
\begin{pgfscope}%
\pgfsetbuttcap%
\pgfsetroundjoin%
\definecolor{currentfill}{rgb}{0.000000,0.000000,0.000000}%
\pgfsetfillcolor{currentfill}%
\pgfsetlinewidth{0.501875pt}%
\definecolor{currentstroke}{rgb}{0.000000,0.000000,0.000000}%
\pgfsetstrokecolor{currentstroke}%
\pgfsetdash{}{0pt}%
\pgfsys@defobject{currentmarker}{\pgfqpoint{0.000000in}{0.000000in}}{\pgfqpoint{0.000000in}{0.055556in}}{%
\pgfpathmoveto{\pgfqpoint{0.000000in}{0.000000in}}%
\pgfpathlineto{\pgfqpoint{0.000000in}{0.055556in}}%
\pgfusepath{stroke,fill}%
}%
\begin{pgfscope}%
\pgfsys@transformshift{5.400000in}{0.900000in}%
\pgfsys@useobject{currentmarker}{}%
\end{pgfscope}%
\end{pgfscope}%
\begin{pgfscope}%
\pgfsetbuttcap%
\pgfsetroundjoin%
\definecolor{currentfill}{rgb}{0.000000,0.000000,0.000000}%
\pgfsetfillcolor{currentfill}%
\pgfsetlinewidth{0.501875pt}%
\definecolor{currentstroke}{rgb}{0.000000,0.000000,0.000000}%
\pgfsetstrokecolor{currentstroke}%
\pgfsetdash{}{0pt}%
\pgfsys@defobject{currentmarker}{\pgfqpoint{0.000000in}{-0.055556in}}{\pgfqpoint{0.000000in}{0.000000in}}{%
\pgfpathmoveto{\pgfqpoint{0.000000in}{0.000000in}}%
\pgfpathlineto{\pgfqpoint{0.000000in}{-0.055556in}}%
\pgfusepath{stroke,fill}%
}%
\begin{pgfscope}%
\pgfsys@transformshift{5.400000in}{5.700000in}%
\pgfsys@useobject{currentmarker}{}%
\end{pgfscope}%
\end{pgfscope}%
\begin{pgfscope}%
\definecolor{textcolor}{rgb}{0.000000,0.000000,0.000000}%
\pgfsetstrokecolor{textcolor}%
\pgfsetfillcolor{textcolor}%
\pgftext[x=5.400000in,y=0.844444in,,top]{\color{textcolor}\sffamily\fontsize{20.000000}{24.000000}\selectfont \(\displaystyle {12}\)}%
\end{pgfscope}%
\begin{pgfscope}%
\pgfpathrectangle{\pgfqpoint{1.200000in}{0.900000in}}{\pgfqpoint{5.600000in}{4.800000in}}%
\pgfusepath{clip}%
\pgfsetbuttcap%
\pgfsetroundjoin%
\pgfsetlinewidth{0.501875pt}%
\definecolor{currentstroke}{rgb}{0.000000,0.000000,0.000000}%
\pgfsetstrokecolor{currentstroke}%
\pgfsetdash{{1.000000pt}{3.000000pt}}{0.000000pt}%
\pgfpathmoveto{\pgfqpoint{6.100000in}{0.900000in}}%
\pgfpathlineto{\pgfqpoint{6.100000in}{5.700000in}}%
\pgfusepath{stroke}%
\end{pgfscope}%
\begin{pgfscope}%
\pgfsetbuttcap%
\pgfsetroundjoin%
\definecolor{currentfill}{rgb}{0.000000,0.000000,0.000000}%
\pgfsetfillcolor{currentfill}%
\pgfsetlinewidth{0.501875pt}%
\definecolor{currentstroke}{rgb}{0.000000,0.000000,0.000000}%
\pgfsetstrokecolor{currentstroke}%
\pgfsetdash{}{0pt}%
\pgfsys@defobject{currentmarker}{\pgfqpoint{0.000000in}{0.000000in}}{\pgfqpoint{0.000000in}{0.055556in}}{%
\pgfpathmoveto{\pgfqpoint{0.000000in}{0.000000in}}%
\pgfpathlineto{\pgfqpoint{0.000000in}{0.055556in}}%
\pgfusepath{stroke,fill}%
}%
\begin{pgfscope}%
\pgfsys@transformshift{6.100000in}{0.900000in}%
\pgfsys@useobject{currentmarker}{}%
\end{pgfscope}%
\end{pgfscope}%
\begin{pgfscope}%
\pgfsetbuttcap%
\pgfsetroundjoin%
\definecolor{currentfill}{rgb}{0.000000,0.000000,0.000000}%
\pgfsetfillcolor{currentfill}%
\pgfsetlinewidth{0.501875pt}%
\definecolor{currentstroke}{rgb}{0.000000,0.000000,0.000000}%
\pgfsetstrokecolor{currentstroke}%
\pgfsetdash{}{0pt}%
\pgfsys@defobject{currentmarker}{\pgfqpoint{0.000000in}{-0.055556in}}{\pgfqpoint{0.000000in}{0.000000in}}{%
\pgfpathmoveto{\pgfqpoint{0.000000in}{0.000000in}}%
\pgfpathlineto{\pgfqpoint{0.000000in}{-0.055556in}}%
\pgfusepath{stroke,fill}%
}%
\begin{pgfscope}%
\pgfsys@transformshift{6.100000in}{5.700000in}%
\pgfsys@useobject{currentmarker}{}%
\end{pgfscope}%
\end{pgfscope}%
\begin{pgfscope}%
\definecolor{textcolor}{rgb}{0.000000,0.000000,0.000000}%
\pgfsetstrokecolor{textcolor}%
\pgfsetfillcolor{textcolor}%
\pgftext[x=6.100000in,y=0.844444in,,top]{\color{textcolor}\sffamily\fontsize{20.000000}{24.000000}\selectfont \(\displaystyle {14}\)}%
\end{pgfscope}%
\begin{pgfscope}%
\pgfpathrectangle{\pgfqpoint{1.200000in}{0.900000in}}{\pgfqpoint{5.600000in}{4.800000in}}%
\pgfusepath{clip}%
\pgfsetbuttcap%
\pgfsetroundjoin%
\pgfsetlinewidth{0.501875pt}%
\definecolor{currentstroke}{rgb}{0.000000,0.000000,0.000000}%
\pgfsetstrokecolor{currentstroke}%
\pgfsetdash{{1.000000pt}{3.000000pt}}{0.000000pt}%
\pgfpathmoveto{\pgfqpoint{6.800000in}{0.900000in}}%
\pgfpathlineto{\pgfqpoint{6.800000in}{5.700000in}}%
\pgfusepath{stroke}%
\end{pgfscope}%
\begin{pgfscope}%
\pgfsetbuttcap%
\pgfsetroundjoin%
\definecolor{currentfill}{rgb}{0.000000,0.000000,0.000000}%
\pgfsetfillcolor{currentfill}%
\pgfsetlinewidth{0.501875pt}%
\definecolor{currentstroke}{rgb}{0.000000,0.000000,0.000000}%
\pgfsetstrokecolor{currentstroke}%
\pgfsetdash{}{0pt}%
\pgfsys@defobject{currentmarker}{\pgfqpoint{0.000000in}{0.000000in}}{\pgfqpoint{0.000000in}{0.055556in}}{%
\pgfpathmoveto{\pgfqpoint{0.000000in}{0.000000in}}%
\pgfpathlineto{\pgfqpoint{0.000000in}{0.055556in}}%
\pgfusepath{stroke,fill}%
}%
\begin{pgfscope}%
\pgfsys@transformshift{6.800000in}{0.900000in}%
\pgfsys@useobject{currentmarker}{}%
\end{pgfscope}%
\end{pgfscope}%
\begin{pgfscope}%
\pgfsetbuttcap%
\pgfsetroundjoin%
\definecolor{currentfill}{rgb}{0.000000,0.000000,0.000000}%
\pgfsetfillcolor{currentfill}%
\pgfsetlinewidth{0.501875pt}%
\definecolor{currentstroke}{rgb}{0.000000,0.000000,0.000000}%
\pgfsetstrokecolor{currentstroke}%
\pgfsetdash{}{0pt}%
\pgfsys@defobject{currentmarker}{\pgfqpoint{0.000000in}{-0.055556in}}{\pgfqpoint{0.000000in}{0.000000in}}{%
\pgfpathmoveto{\pgfqpoint{0.000000in}{0.000000in}}%
\pgfpathlineto{\pgfqpoint{0.000000in}{-0.055556in}}%
\pgfusepath{stroke,fill}%
}%
\begin{pgfscope}%
\pgfsys@transformshift{6.800000in}{5.700000in}%
\pgfsys@useobject{currentmarker}{}%
\end{pgfscope}%
\end{pgfscope}%
\begin{pgfscope}%
\definecolor{textcolor}{rgb}{0.000000,0.000000,0.000000}%
\pgfsetstrokecolor{textcolor}%
\pgfsetfillcolor{textcolor}%
\pgftext[x=6.800000in,y=0.844444in,,top]{\color{textcolor}\sffamily\fontsize{20.000000}{24.000000}\selectfont \(\displaystyle {16}\)}%
\end{pgfscope}%
\begin{pgfscope}%
\definecolor{textcolor}{rgb}{0.000000,0.000000,0.000000}%
\pgfsetstrokecolor{textcolor}%
\pgfsetfillcolor{textcolor}%
\pgftext[x=4.000000in,y=0.518932in,,top]{\color{textcolor}\sffamily\fontsize{20.000000}{24.000000}\selectfont \(\displaystyle N_{pos}\)}%
\end{pgfscope}%
\begin{pgfscope}%
\pgfpathrectangle{\pgfqpoint{1.200000in}{0.900000in}}{\pgfqpoint{5.600000in}{4.800000in}}%
\pgfusepath{clip}%
\pgfsetbuttcap%
\pgfsetroundjoin%
\pgfsetlinewidth{0.501875pt}%
\definecolor{currentstroke}{rgb}{0.000000,0.000000,0.000000}%
\pgfsetstrokecolor{currentstroke}%
\pgfsetdash{{1.000000pt}{3.000000pt}}{0.000000pt}%
\pgfpathmoveto{\pgfqpoint{1.200000in}{0.900000in}}%
\pgfpathlineto{\pgfqpoint{6.800000in}{0.900000in}}%
\pgfusepath{stroke}%
\end{pgfscope}%
\begin{pgfscope}%
\pgfsetbuttcap%
\pgfsetroundjoin%
\definecolor{currentfill}{rgb}{0.000000,0.000000,0.000000}%
\pgfsetfillcolor{currentfill}%
\pgfsetlinewidth{0.501875pt}%
\definecolor{currentstroke}{rgb}{0.000000,0.000000,0.000000}%
\pgfsetstrokecolor{currentstroke}%
\pgfsetdash{}{0pt}%
\pgfsys@defobject{currentmarker}{\pgfqpoint{0.000000in}{0.000000in}}{\pgfqpoint{0.055556in}{0.000000in}}{%
\pgfpathmoveto{\pgfqpoint{0.000000in}{0.000000in}}%
\pgfpathlineto{\pgfqpoint{0.055556in}{0.000000in}}%
\pgfusepath{stroke,fill}%
}%
\begin{pgfscope}%
\pgfsys@transformshift{1.200000in}{0.900000in}%
\pgfsys@useobject{currentmarker}{}%
\end{pgfscope}%
\end{pgfscope}%
\begin{pgfscope}%
\pgfsetbuttcap%
\pgfsetroundjoin%
\definecolor{currentfill}{rgb}{0.000000,0.000000,0.000000}%
\pgfsetfillcolor{currentfill}%
\pgfsetlinewidth{0.501875pt}%
\definecolor{currentstroke}{rgb}{0.000000,0.000000,0.000000}%
\pgfsetstrokecolor{currentstroke}%
\pgfsetdash{}{0pt}%
\pgfsys@defobject{currentmarker}{\pgfqpoint{-0.055556in}{0.000000in}}{\pgfqpoint{0.000000in}{0.000000in}}{%
\pgfpathmoveto{\pgfqpoint{0.000000in}{0.000000in}}%
\pgfpathlineto{\pgfqpoint{-0.055556in}{0.000000in}}%
\pgfusepath{stroke,fill}%
}%
\begin{pgfscope}%
\pgfsys@transformshift{6.800000in}{0.900000in}%
\pgfsys@useobject{currentmarker}{}%
\end{pgfscope}%
\end{pgfscope}%
\begin{pgfscope}%
\definecolor{textcolor}{rgb}{0.000000,0.000000,0.000000}%
\pgfsetstrokecolor{textcolor}%
\pgfsetfillcolor{textcolor}%
\pgftext[x=1.144444in,y=0.900000in,right,]{\color{textcolor}\sffamily\fontsize{20.000000}{24.000000}\selectfont \(\displaystyle {10^{0}}\)}%
\end{pgfscope}%
\begin{pgfscope}%
\pgfpathrectangle{\pgfqpoint{1.200000in}{0.900000in}}{\pgfqpoint{5.600000in}{4.800000in}}%
\pgfusepath{clip}%
\pgfsetbuttcap%
\pgfsetroundjoin%
\pgfsetlinewidth{0.501875pt}%
\definecolor{currentstroke}{rgb}{0.000000,0.000000,0.000000}%
\pgfsetstrokecolor{currentstroke}%
\pgfsetdash{{1.000000pt}{3.000000pt}}{0.000000pt}%
\pgfpathmoveto{\pgfqpoint{1.200000in}{2.100000in}}%
\pgfpathlineto{\pgfqpoint{6.800000in}{2.100000in}}%
\pgfusepath{stroke}%
\end{pgfscope}%
\begin{pgfscope}%
\pgfsetbuttcap%
\pgfsetroundjoin%
\definecolor{currentfill}{rgb}{0.000000,0.000000,0.000000}%
\pgfsetfillcolor{currentfill}%
\pgfsetlinewidth{0.501875pt}%
\definecolor{currentstroke}{rgb}{0.000000,0.000000,0.000000}%
\pgfsetstrokecolor{currentstroke}%
\pgfsetdash{}{0pt}%
\pgfsys@defobject{currentmarker}{\pgfqpoint{0.000000in}{0.000000in}}{\pgfqpoint{0.055556in}{0.000000in}}{%
\pgfpathmoveto{\pgfqpoint{0.000000in}{0.000000in}}%
\pgfpathlineto{\pgfqpoint{0.055556in}{0.000000in}}%
\pgfusepath{stroke,fill}%
}%
\begin{pgfscope}%
\pgfsys@transformshift{1.200000in}{2.100000in}%
\pgfsys@useobject{currentmarker}{}%
\end{pgfscope}%
\end{pgfscope}%
\begin{pgfscope}%
\pgfsetbuttcap%
\pgfsetroundjoin%
\definecolor{currentfill}{rgb}{0.000000,0.000000,0.000000}%
\pgfsetfillcolor{currentfill}%
\pgfsetlinewidth{0.501875pt}%
\definecolor{currentstroke}{rgb}{0.000000,0.000000,0.000000}%
\pgfsetstrokecolor{currentstroke}%
\pgfsetdash{}{0pt}%
\pgfsys@defobject{currentmarker}{\pgfqpoint{-0.055556in}{0.000000in}}{\pgfqpoint{0.000000in}{0.000000in}}{%
\pgfpathmoveto{\pgfqpoint{0.000000in}{0.000000in}}%
\pgfpathlineto{\pgfqpoint{-0.055556in}{0.000000in}}%
\pgfusepath{stroke,fill}%
}%
\begin{pgfscope}%
\pgfsys@transformshift{6.800000in}{2.100000in}%
\pgfsys@useobject{currentmarker}{}%
\end{pgfscope}%
\end{pgfscope}%
\begin{pgfscope}%
\definecolor{textcolor}{rgb}{0.000000,0.000000,0.000000}%
\pgfsetstrokecolor{textcolor}%
\pgfsetfillcolor{textcolor}%
\pgftext[x=1.144444in,y=2.100000in,right,]{\color{textcolor}\sffamily\fontsize{20.000000}{24.000000}\selectfont \(\displaystyle {10^{1}}\)}%
\end{pgfscope}%
\begin{pgfscope}%
\pgfpathrectangle{\pgfqpoint{1.200000in}{0.900000in}}{\pgfqpoint{5.600000in}{4.800000in}}%
\pgfusepath{clip}%
\pgfsetbuttcap%
\pgfsetroundjoin%
\pgfsetlinewidth{0.501875pt}%
\definecolor{currentstroke}{rgb}{0.000000,0.000000,0.000000}%
\pgfsetstrokecolor{currentstroke}%
\pgfsetdash{{1.000000pt}{3.000000pt}}{0.000000pt}%
\pgfpathmoveto{\pgfqpoint{1.200000in}{3.300000in}}%
\pgfpathlineto{\pgfqpoint{6.800000in}{3.300000in}}%
\pgfusepath{stroke}%
\end{pgfscope}%
\begin{pgfscope}%
\pgfsetbuttcap%
\pgfsetroundjoin%
\definecolor{currentfill}{rgb}{0.000000,0.000000,0.000000}%
\pgfsetfillcolor{currentfill}%
\pgfsetlinewidth{0.501875pt}%
\definecolor{currentstroke}{rgb}{0.000000,0.000000,0.000000}%
\pgfsetstrokecolor{currentstroke}%
\pgfsetdash{}{0pt}%
\pgfsys@defobject{currentmarker}{\pgfqpoint{0.000000in}{0.000000in}}{\pgfqpoint{0.055556in}{0.000000in}}{%
\pgfpathmoveto{\pgfqpoint{0.000000in}{0.000000in}}%
\pgfpathlineto{\pgfqpoint{0.055556in}{0.000000in}}%
\pgfusepath{stroke,fill}%
}%
\begin{pgfscope}%
\pgfsys@transformshift{1.200000in}{3.300000in}%
\pgfsys@useobject{currentmarker}{}%
\end{pgfscope}%
\end{pgfscope}%
\begin{pgfscope}%
\pgfsetbuttcap%
\pgfsetroundjoin%
\definecolor{currentfill}{rgb}{0.000000,0.000000,0.000000}%
\pgfsetfillcolor{currentfill}%
\pgfsetlinewidth{0.501875pt}%
\definecolor{currentstroke}{rgb}{0.000000,0.000000,0.000000}%
\pgfsetstrokecolor{currentstroke}%
\pgfsetdash{}{0pt}%
\pgfsys@defobject{currentmarker}{\pgfqpoint{-0.055556in}{0.000000in}}{\pgfqpoint{0.000000in}{0.000000in}}{%
\pgfpathmoveto{\pgfqpoint{0.000000in}{0.000000in}}%
\pgfpathlineto{\pgfqpoint{-0.055556in}{0.000000in}}%
\pgfusepath{stroke,fill}%
}%
\begin{pgfscope}%
\pgfsys@transformshift{6.800000in}{3.300000in}%
\pgfsys@useobject{currentmarker}{}%
\end{pgfscope}%
\end{pgfscope}%
\begin{pgfscope}%
\definecolor{textcolor}{rgb}{0.000000,0.000000,0.000000}%
\pgfsetstrokecolor{textcolor}%
\pgfsetfillcolor{textcolor}%
\pgftext[x=1.144444in,y=3.300000in,right,]{\color{textcolor}\sffamily\fontsize{20.000000}{24.000000}\selectfont \(\displaystyle {10^{2}}\)}%
\end{pgfscope}%
\begin{pgfscope}%
\pgfpathrectangle{\pgfqpoint{1.200000in}{0.900000in}}{\pgfqpoint{5.600000in}{4.800000in}}%
\pgfusepath{clip}%
\pgfsetbuttcap%
\pgfsetroundjoin%
\pgfsetlinewidth{0.501875pt}%
\definecolor{currentstroke}{rgb}{0.000000,0.000000,0.000000}%
\pgfsetstrokecolor{currentstroke}%
\pgfsetdash{{1.000000pt}{3.000000pt}}{0.000000pt}%
\pgfpathmoveto{\pgfqpoint{1.200000in}{4.500000in}}%
\pgfpathlineto{\pgfqpoint{6.800000in}{4.500000in}}%
\pgfusepath{stroke}%
\end{pgfscope}%
\begin{pgfscope}%
\pgfsetbuttcap%
\pgfsetroundjoin%
\definecolor{currentfill}{rgb}{0.000000,0.000000,0.000000}%
\pgfsetfillcolor{currentfill}%
\pgfsetlinewidth{0.501875pt}%
\definecolor{currentstroke}{rgb}{0.000000,0.000000,0.000000}%
\pgfsetstrokecolor{currentstroke}%
\pgfsetdash{}{0pt}%
\pgfsys@defobject{currentmarker}{\pgfqpoint{0.000000in}{0.000000in}}{\pgfqpoint{0.055556in}{0.000000in}}{%
\pgfpathmoveto{\pgfqpoint{0.000000in}{0.000000in}}%
\pgfpathlineto{\pgfqpoint{0.055556in}{0.000000in}}%
\pgfusepath{stroke,fill}%
}%
\begin{pgfscope}%
\pgfsys@transformshift{1.200000in}{4.500000in}%
\pgfsys@useobject{currentmarker}{}%
\end{pgfscope}%
\end{pgfscope}%
\begin{pgfscope}%
\pgfsetbuttcap%
\pgfsetroundjoin%
\definecolor{currentfill}{rgb}{0.000000,0.000000,0.000000}%
\pgfsetfillcolor{currentfill}%
\pgfsetlinewidth{0.501875pt}%
\definecolor{currentstroke}{rgb}{0.000000,0.000000,0.000000}%
\pgfsetstrokecolor{currentstroke}%
\pgfsetdash{}{0pt}%
\pgfsys@defobject{currentmarker}{\pgfqpoint{-0.055556in}{0.000000in}}{\pgfqpoint{0.000000in}{0.000000in}}{%
\pgfpathmoveto{\pgfqpoint{0.000000in}{0.000000in}}%
\pgfpathlineto{\pgfqpoint{-0.055556in}{0.000000in}}%
\pgfusepath{stroke,fill}%
}%
\begin{pgfscope}%
\pgfsys@transformshift{6.800000in}{4.500000in}%
\pgfsys@useobject{currentmarker}{}%
\end{pgfscope}%
\end{pgfscope}%
\begin{pgfscope}%
\definecolor{textcolor}{rgb}{0.000000,0.000000,0.000000}%
\pgfsetstrokecolor{textcolor}%
\pgfsetfillcolor{textcolor}%
\pgftext[x=1.144444in,y=4.500000in,right,]{\color{textcolor}\sffamily\fontsize{20.000000}{24.000000}\selectfont \(\displaystyle {10^{3}}\)}%
\end{pgfscope}%
\begin{pgfscope}%
\pgfpathrectangle{\pgfqpoint{1.200000in}{0.900000in}}{\pgfqpoint{5.600000in}{4.800000in}}%
\pgfusepath{clip}%
\pgfsetbuttcap%
\pgfsetroundjoin%
\pgfsetlinewidth{0.501875pt}%
\definecolor{currentstroke}{rgb}{0.000000,0.000000,0.000000}%
\pgfsetstrokecolor{currentstroke}%
\pgfsetdash{{1.000000pt}{3.000000pt}}{0.000000pt}%
\pgfpathmoveto{\pgfqpoint{1.200000in}{5.700000in}}%
\pgfpathlineto{\pgfqpoint{6.800000in}{5.700000in}}%
\pgfusepath{stroke}%
\end{pgfscope}%
\begin{pgfscope}%
\pgfsetbuttcap%
\pgfsetroundjoin%
\definecolor{currentfill}{rgb}{0.000000,0.000000,0.000000}%
\pgfsetfillcolor{currentfill}%
\pgfsetlinewidth{0.501875pt}%
\definecolor{currentstroke}{rgb}{0.000000,0.000000,0.000000}%
\pgfsetstrokecolor{currentstroke}%
\pgfsetdash{}{0pt}%
\pgfsys@defobject{currentmarker}{\pgfqpoint{0.000000in}{0.000000in}}{\pgfqpoint{0.055556in}{0.000000in}}{%
\pgfpathmoveto{\pgfqpoint{0.000000in}{0.000000in}}%
\pgfpathlineto{\pgfqpoint{0.055556in}{0.000000in}}%
\pgfusepath{stroke,fill}%
}%
\begin{pgfscope}%
\pgfsys@transformshift{1.200000in}{5.700000in}%
\pgfsys@useobject{currentmarker}{}%
\end{pgfscope}%
\end{pgfscope}%
\begin{pgfscope}%
\pgfsetbuttcap%
\pgfsetroundjoin%
\definecolor{currentfill}{rgb}{0.000000,0.000000,0.000000}%
\pgfsetfillcolor{currentfill}%
\pgfsetlinewidth{0.501875pt}%
\definecolor{currentstroke}{rgb}{0.000000,0.000000,0.000000}%
\pgfsetstrokecolor{currentstroke}%
\pgfsetdash{}{0pt}%
\pgfsys@defobject{currentmarker}{\pgfqpoint{-0.055556in}{0.000000in}}{\pgfqpoint{0.000000in}{0.000000in}}{%
\pgfpathmoveto{\pgfqpoint{0.000000in}{0.000000in}}%
\pgfpathlineto{\pgfqpoint{-0.055556in}{0.000000in}}%
\pgfusepath{stroke,fill}%
}%
\begin{pgfscope}%
\pgfsys@transformshift{6.800000in}{5.700000in}%
\pgfsys@useobject{currentmarker}{}%
\end{pgfscope}%
\end{pgfscope}%
\begin{pgfscope}%
\definecolor{textcolor}{rgb}{0.000000,0.000000,0.000000}%
\pgfsetstrokecolor{textcolor}%
\pgfsetfillcolor{textcolor}%
\pgftext[x=1.144444in,y=5.700000in,right,]{\color{textcolor}\sffamily\fontsize{20.000000}{24.000000}\selectfont \(\displaystyle {10^{4}}\)}%
\end{pgfscope}%
\begin{pgfscope}%
\pgfsetbuttcap%
\pgfsetroundjoin%
\definecolor{currentfill}{rgb}{0.000000,0.000000,0.000000}%
\pgfsetfillcolor{currentfill}%
\pgfsetlinewidth{0.501875pt}%
\definecolor{currentstroke}{rgb}{0.000000,0.000000,0.000000}%
\pgfsetstrokecolor{currentstroke}%
\pgfsetdash{}{0pt}%
\pgfsys@defobject{currentmarker}{\pgfqpoint{0.000000in}{0.000000in}}{\pgfqpoint{0.027778in}{0.000000in}}{%
\pgfpathmoveto{\pgfqpoint{0.000000in}{0.000000in}}%
\pgfpathlineto{\pgfqpoint{0.027778in}{0.000000in}}%
\pgfusepath{stroke,fill}%
}%
\begin{pgfscope}%
\pgfsys@transformshift{1.200000in}{1.261236in}%
\pgfsys@useobject{currentmarker}{}%
\end{pgfscope}%
\end{pgfscope}%
\begin{pgfscope}%
\pgfsetbuttcap%
\pgfsetroundjoin%
\definecolor{currentfill}{rgb}{0.000000,0.000000,0.000000}%
\pgfsetfillcolor{currentfill}%
\pgfsetlinewidth{0.501875pt}%
\definecolor{currentstroke}{rgb}{0.000000,0.000000,0.000000}%
\pgfsetstrokecolor{currentstroke}%
\pgfsetdash{}{0pt}%
\pgfsys@defobject{currentmarker}{\pgfqpoint{-0.027778in}{0.000000in}}{\pgfqpoint{0.000000in}{0.000000in}}{%
\pgfpathmoveto{\pgfqpoint{0.000000in}{0.000000in}}%
\pgfpathlineto{\pgfqpoint{-0.027778in}{0.000000in}}%
\pgfusepath{stroke,fill}%
}%
\begin{pgfscope}%
\pgfsys@transformshift{6.800000in}{1.261236in}%
\pgfsys@useobject{currentmarker}{}%
\end{pgfscope}%
\end{pgfscope}%
\begin{pgfscope}%
\pgfsetbuttcap%
\pgfsetroundjoin%
\definecolor{currentfill}{rgb}{0.000000,0.000000,0.000000}%
\pgfsetfillcolor{currentfill}%
\pgfsetlinewidth{0.501875pt}%
\definecolor{currentstroke}{rgb}{0.000000,0.000000,0.000000}%
\pgfsetstrokecolor{currentstroke}%
\pgfsetdash{}{0pt}%
\pgfsys@defobject{currentmarker}{\pgfqpoint{0.000000in}{0.000000in}}{\pgfqpoint{0.027778in}{0.000000in}}{%
\pgfpathmoveto{\pgfqpoint{0.000000in}{0.000000in}}%
\pgfpathlineto{\pgfqpoint{0.027778in}{0.000000in}}%
\pgfusepath{stroke,fill}%
}%
\begin{pgfscope}%
\pgfsys@transformshift{1.200000in}{1.472546in}%
\pgfsys@useobject{currentmarker}{}%
\end{pgfscope}%
\end{pgfscope}%
\begin{pgfscope}%
\pgfsetbuttcap%
\pgfsetroundjoin%
\definecolor{currentfill}{rgb}{0.000000,0.000000,0.000000}%
\pgfsetfillcolor{currentfill}%
\pgfsetlinewidth{0.501875pt}%
\definecolor{currentstroke}{rgb}{0.000000,0.000000,0.000000}%
\pgfsetstrokecolor{currentstroke}%
\pgfsetdash{}{0pt}%
\pgfsys@defobject{currentmarker}{\pgfqpoint{-0.027778in}{0.000000in}}{\pgfqpoint{0.000000in}{0.000000in}}{%
\pgfpathmoveto{\pgfqpoint{0.000000in}{0.000000in}}%
\pgfpathlineto{\pgfqpoint{-0.027778in}{0.000000in}}%
\pgfusepath{stroke,fill}%
}%
\begin{pgfscope}%
\pgfsys@transformshift{6.800000in}{1.472546in}%
\pgfsys@useobject{currentmarker}{}%
\end{pgfscope}%
\end{pgfscope}%
\begin{pgfscope}%
\pgfsetbuttcap%
\pgfsetroundjoin%
\definecolor{currentfill}{rgb}{0.000000,0.000000,0.000000}%
\pgfsetfillcolor{currentfill}%
\pgfsetlinewidth{0.501875pt}%
\definecolor{currentstroke}{rgb}{0.000000,0.000000,0.000000}%
\pgfsetstrokecolor{currentstroke}%
\pgfsetdash{}{0pt}%
\pgfsys@defobject{currentmarker}{\pgfqpoint{0.000000in}{0.000000in}}{\pgfqpoint{0.027778in}{0.000000in}}{%
\pgfpathmoveto{\pgfqpoint{0.000000in}{0.000000in}}%
\pgfpathlineto{\pgfqpoint{0.027778in}{0.000000in}}%
\pgfusepath{stroke,fill}%
}%
\begin{pgfscope}%
\pgfsys@transformshift{1.200000in}{1.622472in}%
\pgfsys@useobject{currentmarker}{}%
\end{pgfscope}%
\end{pgfscope}%
\begin{pgfscope}%
\pgfsetbuttcap%
\pgfsetroundjoin%
\definecolor{currentfill}{rgb}{0.000000,0.000000,0.000000}%
\pgfsetfillcolor{currentfill}%
\pgfsetlinewidth{0.501875pt}%
\definecolor{currentstroke}{rgb}{0.000000,0.000000,0.000000}%
\pgfsetstrokecolor{currentstroke}%
\pgfsetdash{}{0pt}%
\pgfsys@defobject{currentmarker}{\pgfqpoint{-0.027778in}{0.000000in}}{\pgfqpoint{0.000000in}{0.000000in}}{%
\pgfpathmoveto{\pgfqpoint{0.000000in}{0.000000in}}%
\pgfpathlineto{\pgfqpoint{-0.027778in}{0.000000in}}%
\pgfusepath{stroke,fill}%
}%
\begin{pgfscope}%
\pgfsys@transformshift{6.800000in}{1.622472in}%
\pgfsys@useobject{currentmarker}{}%
\end{pgfscope}%
\end{pgfscope}%
\begin{pgfscope}%
\pgfsetbuttcap%
\pgfsetroundjoin%
\definecolor{currentfill}{rgb}{0.000000,0.000000,0.000000}%
\pgfsetfillcolor{currentfill}%
\pgfsetlinewidth{0.501875pt}%
\definecolor{currentstroke}{rgb}{0.000000,0.000000,0.000000}%
\pgfsetstrokecolor{currentstroke}%
\pgfsetdash{}{0pt}%
\pgfsys@defobject{currentmarker}{\pgfqpoint{0.000000in}{0.000000in}}{\pgfqpoint{0.027778in}{0.000000in}}{%
\pgfpathmoveto{\pgfqpoint{0.000000in}{0.000000in}}%
\pgfpathlineto{\pgfqpoint{0.027778in}{0.000000in}}%
\pgfusepath{stroke,fill}%
}%
\begin{pgfscope}%
\pgfsys@transformshift{1.200000in}{1.738764in}%
\pgfsys@useobject{currentmarker}{}%
\end{pgfscope}%
\end{pgfscope}%
\begin{pgfscope}%
\pgfsetbuttcap%
\pgfsetroundjoin%
\definecolor{currentfill}{rgb}{0.000000,0.000000,0.000000}%
\pgfsetfillcolor{currentfill}%
\pgfsetlinewidth{0.501875pt}%
\definecolor{currentstroke}{rgb}{0.000000,0.000000,0.000000}%
\pgfsetstrokecolor{currentstroke}%
\pgfsetdash{}{0pt}%
\pgfsys@defobject{currentmarker}{\pgfqpoint{-0.027778in}{0.000000in}}{\pgfqpoint{0.000000in}{0.000000in}}{%
\pgfpathmoveto{\pgfqpoint{0.000000in}{0.000000in}}%
\pgfpathlineto{\pgfqpoint{-0.027778in}{0.000000in}}%
\pgfusepath{stroke,fill}%
}%
\begin{pgfscope}%
\pgfsys@transformshift{6.800000in}{1.738764in}%
\pgfsys@useobject{currentmarker}{}%
\end{pgfscope}%
\end{pgfscope}%
\begin{pgfscope}%
\pgfsetbuttcap%
\pgfsetroundjoin%
\definecolor{currentfill}{rgb}{0.000000,0.000000,0.000000}%
\pgfsetfillcolor{currentfill}%
\pgfsetlinewidth{0.501875pt}%
\definecolor{currentstroke}{rgb}{0.000000,0.000000,0.000000}%
\pgfsetstrokecolor{currentstroke}%
\pgfsetdash{}{0pt}%
\pgfsys@defobject{currentmarker}{\pgfqpoint{0.000000in}{0.000000in}}{\pgfqpoint{0.027778in}{0.000000in}}{%
\pgfpathmoveto{\pgfqpoint{0.000000in}{0.000000in}}%
\pgfpathlineto{\pgfqpoint{0.027778in}{0.000000in}}%
\pgfusepath{stroke,fill}%
}%
\begin{pgfscope}%
\pgfsys@transformshift{1.200000in}{1.833782in}%
\pgfsys@useobject{currentmarker}{}%
\end{pgfscope}%
\end{pgfscope}%
\begin{pgfscope}%
\pgfsetbuttcap%
\pgfsetroundjoin%
\definecolor{currentfill}{rgb}{0.000000,0.000000,0.000000}%
\pgfsetfillcolor{currentfill}%
\pgfsetlinewidth{0.501875pt}%
\definecolor{currentstroke}{rgb}{0.000000,0.000000,0.000000}%
\pgfsetstrokecolor{currentstroke}%
\pgfsetdash{}{0pt}%
\pgfsys@defobject{currentmarker}{\pgfqpoint{-0.027778in}{0.000000in}}{\pgfqpoint{0.000000in}{0.000000in}}{%
\pgfpathmoveto{\pgfqpoint{0.000000in}{0.000000in}}%
\pgfpathlineto{\pgfqpoint{-0.027778in}{0.000000in}}%
\pgfusepath{stroke,fill}%
}%
\begin{pgfscope}%
\pgfsys@transformshift{6.800000in}{1.833782in}%
\pgfsys@useobject{currentmarker}{}%
\end{pgfscope}%
\end{pgfscope}%
\begin{pgfscope}%
\pgfsetbuttcap%
\pgfsetroundjoin%
\definecolor{currentfill}{rgb}{0.000000,0.000000,0.000000}%
\pgfsetfillcolor{currentfill}%
\pgfsetlinewidth{0.501875pt}%
\definecolor{currentstroke}{rgb}{0.000000,0.000000,0.000000}%
\pgfsetstrokecolor{currentstroke}%
\pgfsetdash{}{0pt}%
\pgfsys@defobject{currentmarker}{\pgfqpoint{0.000000in}{0.000000in}}{\pgfqpoint{0.027778in}{0.000000in}}{%
\pgfpathmoveto{\pgfqpoint{0.000000in}{0.000000in}}%
\pgfpathlineto{\pgfqpoint{0.027778in}{0.000000in}}%
\pgfusepath{stroke,fill}%
}%
\begin{pgfscope}%
\pgfsys@transformshift{1.200000in}{1.914118in}%
\pgfsys@useobject{currentmarker}{}%
\end{pgfscope}%
\end{pgfscope}%
\begin{pgfscope}%
\pgfsetbuttcap%
\pgfsetroundjoin%
\definecolor{currentfill}{rgb}{0.000000,0.000000,0.000000}%
\pgfsetfillcolor{currentfill}%
\pgfsetlinewidth{0.501875pt}%
\definecolor{currentstroke}{rgb}{0.000000,0.000000,0.000000}%
\pgfsetstrokecolor{currentstroke}%
\pgfsetdash{}{0pt}%
\pgfsys@defobject{currentmarker}{\pgfqpoint{-0.027778in}{0.000000in}}{\pgfqpoint{0.000000in}{0.000000in}}{%
\pgfpathmoveto{\pgfqpoint{0.000000in}{0.000000in}}%
\pgfpathlineto{\pgfqpoint{-0.027778in}{0.000000in}}%
\pgfusepath{stroke,fill}%
}%
\begin{pgfscope}%
\pgfsys@transformshift{6.800000in}{1.914118in}%
\pgfsys@useobject{currentmarker}{}%
\end{pgfscope}%
\end{pgfscope}%
\begin{pgfscope}%
\pgfsetbuttcap%
\pgfsetroundjoin%
\definecolor{currentfill}{rgb}{0.000000,0.000000,0.000000}%
\pgfsetfillcolor{currentfill}%
\pgfsetlinewidth{0.501875pt}%
\definecolor{currentstroke}{rgb}{0.000000,0.000000,0.000000}%
\pgfsetstrokecolor{currentstroke}%
\pgfsetdash{}{0pt}%
\pgfsys@defobject{currentmarker}{\pgfqpoint{0.000000in}{0.000000in}}{\pgfqpoint{0.027778in}{0.000000in}}{%
\pgfpathmoveto{\pgfqpoint{0.000000in}{0.000000in}}%
\pgfpathlineto{\pgfqpoint{0.027778in}{0.000000in}}%
\pgfusepath{stroke,fill}%
}%
\begin{pgfscope}%
\pgfsys@transformshift{1.200000in}{1.983708in}%
\pgfsys@useobject{currentmarker}{}%
\end{pgfscope}%
\end{pgfscope}%
\begin{pgfscope}%
\pgfsetbuttcap%
\pgfsetroundjoin%
\definecolor{currentfill}{rgb}{0.000000,0.000000,0.000000}%
\pgfsetfillcolor{currentfill}%
\pgfsetlinewidth{0.501875pt}%
\definecolor{currentstroke}{rgb}{0.000000,0.000000,0.000000}%
\pgfsetstrokecolor{currentstroke}%
\pgfsetdash{}{0pt}%
\pgfsys@defobject{currentmarker}{\pgfqpoint{-0.027778in}{0.000000in}}{\pgfqpoint{0.000000in}{0.000000in}}{%
\pgfpathmoveto{\pgfqpoint{0.000000in}{0.000000in}}%
\pgfpathlineto{\pgfqpoint{-0.027778in}{0.000000in}}%
\pgfusepath{stroke,fill}%
}%
\begin{pgfscope}%
\pgfsys@transformshift{6.800000in}{1.983708in}%
\pgfsys@useobject{currentmarker}{}%
\end{pgfscope}%
\end{pgfscope}%
\begin{pgfscope}%
\pgfsetbuttcap%
\pgfsetroundjoin%
\definecolor{currentfill}{rgb}{0.000000,0.000000,0.000000}%
\pgfsetfillcolor{currentfill}%
\pgfsetlinewidth{0.501875pt}%
\definecolor{currentstroke}{rgb}{0.000000,0.000000,0.000000}%
\pgfsetstrokecolor{currentstroke}%
\pgfsetdash{}{0pt}%
\pgfsys@defobject{currentmarker}{\pgfqpoint{0.000000in}{0.000000in}}{\pgfqpoint{0.027778in}{0.000000in}}{%
\pgfpathmoveto{\pgfqpoint{0.000000in}{0.000000in}}%
\pgfpathlineto{\pgfqpoint{0.027778in}{0.000000in}}%
\pgfusepath{stroke,fill}%
}%
\begin{pgfscope}%
\pgfsys@transformshift{1.200000in}{2.045091in}%
\pgfsys@useobject{currentmarker}{}%
\end{pgfscope}%
\end{pgfscope}%
\begin{pgfscope}%
\pgfsetbuttcap%
\pgfsetroundjoin%
\definecolor{currentfill}{rgb}{0.000000,0.000000,0.000000}%
\pgfsetfillcolor{currentfill}%
\pgfsetlinewidth{0.501875pt}%
\definecolor{currentstroke}{rgb}{0.000000,0.000000,0.000000}%
\pgfsetstrokecolor{currentstroke}%
\pgfsetdash{}{0pt}%
\pgfsys@defobject{currentmarker}{\pgfqpoint{-0.027778in}{0.000000in}}{\pgfqpoint{0.000000in}{0.000000in}}{%
\pgfpathmoveto{\pgfqpoint{0.000000in}{0.000000in}}%
\pgfpathlineto{\pgfqpoint{-0.027778in}{0.000000in}}%
\pgfusepath{stroke,fill}%
}%
\begin{pgfscope}%
\pgfsys@transformshift{6.800000in}{2.045091in}%
\pgfsys@useobject{currentmarker}{}%
\end{pgfscope}%
\end{pgfscope}%
\begin{pgfscope}%
\pgfsetbuttcap%
\pgfsetroundjoin%
\definecolor{currentfill}{rgb}{0.000000,0.000000,0.000000}%
\pgfsetfillcolor{currentfill}%
\pgfsetlinewidth{0.501875pt}%
\definecolor{currentstroke}{rgb}{0.000000,0.000000,0.000000}%
\pgfsetstrokecolor{currentstroke}%
\pgfsetdash{}{0pt}%
\pgfsys@defobject{currentmarker}{\pgfqpoint{0.000000in}{0.000000in}}{\pgfqpoint{0.027778in}{0.000000in}}{%
\pgfpathmoveto{\pgfqpoint{0.000000in}{0.000000in}}%
\pgfpathlineto{\pgfqpoint{0.027778in}{0.000000in}}%
\pgfusepath{stroke,fill}%
}%
\begin{pgfscope}%
\pgfsys@transformshift{1.200000in}{2.461236in}%
\pgfsys@useobject{currentmarker}{}%
\end{pgfscope}%
\end{pgfscope}%
\begin{pgfscope}%
\pgfsetbuttcap%
\pgfsetroundjoin%
\definecolor{currentfill}{rgb}{0.000000,0.000000,0.000000}%
\pgfsetfillcolor{currentfill}%
\pgfsetlinewidth{0.501875pt}%
\definecolor{currentstroke}{rgb}{0.000000,0.000000,0.000000}%
\pgfsetstrokecolor{currentstroke}%
\pgfsetdash{}{0pt}%
\pgfsys@defobject{currentmarker}{\pgfqpoint{-0.027778in}{0.000000in}}{\pgfqpoint{0.000000in}{0.000000in}}{%
\pgfpathmoveto{\pgfqpoint{0.000000in}{0.000000in}}%
\pgfpathlineto{\pgfqpoint{-0.027778in}{0.000000in}}%
\pgfusepath{stroke,fill}%
}%
\begin{pgfscope}%
\pgfsys@transformshift{6.800000in}{2.461236in}%
\pgfsys@useobject{currentmarker}{}%
\end{pgfscope}%
\end{pgfscope}%
\begin{pgfscope}%
\pgfsetbuttcap%
\pgfsetroundjoin%
\definecolor{currentfill}{rgb}{0.000000,0.000000,0.000000}%
\pgfsetfillcolor{currentfill}%
\pgfsetlinewidth{0.501875pt}%
\definecolor{currentstroke}{rgb}{0.000000,0.000000,0.000000}%
\pgfsetstrokecolor{currentstroke}%
\pgfsetdash{}{0pt}%
\pgfsys@defobject{currentmarker}{\pgfqpoint{0.000000in}{0.000000in}}{\pgfqpoint{0.027778in}{0.000000in}}{%
\pgfpathmoveto{\pgfqpoint{0.000000in}{0.000000in}}%
\pgfpathlineto{\pgfqpoint{0.027778in}{0.000000in}}%
\pgfusepath{stroke,fill}%
}%
\begin{pgfscope}%
\pgfsys@transformshift{1.200000in}{2.672546in}%
\pgfsys@useobject{currentmarker}{}%
\end{pgfscope}%
\end{pgfscope}%
\begin{pgfscope}%
\pgfsetbuttcap%
\pgfsetroundjoin%
\definecolor{currentfill}{rgb}{0.000000,0.000000,0.000000}%
\pgfsetfillcolor{currentfill}%
\pgfsetlinewidth{0.501875pt}%
\definecolor{currentstroke}{rgb}{0.000000,0.000000,0.000000}%
\pgfsetstrokecolor{currentstroke}%
\pgfsetdash{}{0pt}%
\pgfsys@defobject{currentmarker}{\pgfqpoint{-0.027778in}{0.000000in}}{\pgfqpoint{0.000000in}{0.000000in}}{%
\pgfpathmoveto{\pgfqpoint{0.000000in}{0.000000in}}%
\pgfpathlineto{\pgfqpoint{-0.027778in}{0.000000in}}%
\pgfusepath{stroke,fill}%
}%
\begin{pgfscope}%
\pgfsys@transformshift{6.800000in}{2.672546in}%
\pgfsys@useobject{currentmarker}{}%
\end{pgfscope}%
\end{pgfscope}%
\begin{pgfscope}%
\pgfsetbuttcap%
\pgfsetroundjoin%
\definecolor{currentfill}{rgb}{0.000000,0.000000,0.000000}%
\pgfsetfillcolor{currentfill}%
\pgfsetlinewidth{0.501875pt}%
\definecolor{currentstroke}{rgb}{0.000000,0.000000,0.000000}%
\pgfsetstrokecolor{currentstroke}%
\pgfsetdash{}{0pt}%
\pgfsys@defobject{currentmarker}{\pgfqpoint{0.000000in}{0.000000in}}{\pgfqpoint{0.027778in}{0.000000in}}{%
\pgfpathmoveto{\pgfqpoint{0.000000in}{0.000000in}}%
\pgfpathlineto{\pgfqpoint{0.027778in}{0.000000in}}%
\pgfusepath{stroke,fill}%
}%
\begin{pgfscope}%
\pgfsys@transformshift{1.200000in}{2.822472in}%
\pgfsys@useobject{currentmarker}{}%
\end{pgfscope}%
\end{pgfscope}%
\begin{pgfscope}%
\pgfsetbuttcap%
\pgfsetroundjoin%
\definecolor{currentfill}{rgb}{0.000000,0.000000,0.000000}%
\pgfsetfillcolor{currentfill}%
\pgfsetlinewidth{0.501875pt}%
\definecolor{currentstroke}{rgb}{0.000000,0.000000,0.000000}%
\pgfsetstrokecolor{currentstroke}%
\pgfsetdash{}{0pt}%
\pgfsys@defobject{currentmarker}{\pgfqpoint{-0.027778in}{0.000000in}}{\pgfqpoint{0.000000in}{0.000000in}}{%
\pgfpathmoveto{\pgfqpoint{0.000000in}{0.000000in}}%
\pgfpathlineto{\pgfqpoint{-0.027778in}{0.000000in}}%
\pgfusepath{stroke,fill}%
}%
\begin{pgfscope}%
\pgfsys@transformshift{6.800000in}{2.822472in}%
\pgfsys@useobject{currentmarker}{}%
\end{pgfscope}%
\end{pgfscope}%
\begin{pgfscope}%
\pgfsetbuttcap%
\pgfsetroundjoin%
\definecolor{currentfill}{rgb}{0.000000,0.000000,0.000000}%
\pgfsetfillcolor{currentfill}%
\pgfsetlinewidth{0.501875pt}%
\definecolor{currentstroke}{rgb}{0.000000,0.000000,0.000000}%
\pgfsetstrokecolor{currentstroke}%
\pgfsetdash{}{0pt}%
\pgfsys@defobject{currentmarker}{\pgfqpoint{0.000000in}{0.000000in}}{\pgfqpoint{0.027778in}{0.000000in}}{%
\pgfpathmoveto{\pgfqpoint{0.000000in}{0.000000in}}%
\pgfpathlineto{\pgfqpoint{0.027778in}{0.000000in}}%
\pgfusepath{stroke,fill}%
}%
\begin{pgfscope}%
\pgfsys@transformshift{1.200000in}{2.938764in}%
\pgfsys@useobject{currentmarker}{}%
\end{pgfscope}%
\end{pgfscope}%
\begin{pgfscope}%
\pgfsetbuttcap%
\pgfsetroundjoin%
\definecolor{currentfill}{rgb}{0.000000,0.000000,0.000000}%
\pgfsetfillcolor{currentfill}%
\pgfsetlinewidth{0.501875pt}%
\definecolor{currentstroke}{rgb}{0.000000,0.000000,0.000000}%
\pgfsetstrokecolor{currentstroke}%
\pgfsetdash{}{0pt}%
\pgfsys@defobject{currentmarker}{\pgfqpoint{-0.027778in}{0.000000in}}{\pgfqpoint{0.000000in}{0.000000in}}{%
\pgfpathmoveto{\pgfqpoint{0.000000in}{0.000000in}}%
\pgfpathlineto{\pgfqpoint{-0.027778in}{0.000000in}}%
\pgfusepath{stroke,fill}%
}%
\begin{pgfscope}%
\pgfsys@transformshift{6.800000in}{2.938764in}%
\pgfsys@useobject{currentmarker}{}%
\end{pgfscope}%
\end{pgfscope}%
\begin{pgfscope}%
\pgfsetbuttcap%
\pgfsetroundjoin%
\definecolor{currentfill}{rgb}{0.000000,0.000000,0.000000}%
\pgfsetfillcolor{currentfill}%
\pgfsetlinewidth{0.501875pt}%
\definecolor{currentstroke}{rgb}{0.000000,0.000000,0.000000}%
\pgfsetstrokecolor{currentstroke}%
\pgfsetdash{}{0pt}%
\pgfsys@defobject{currentmarker}{\pgfqpoint{0.000000in}{0.000000in}}{\pgfqpoint{0.027778in}{0.000000in}}{%
\pgfpathmoveto{\pgfqpoint{0.000000in}{0.000000in}}%
\pgfpathlineto{\pgfqpoint{0.027778in}{0.000000in}}%
\pgfusepath{stroke,fill}%
}%
\begin{pgfscope}%
\pgfsys@transformshift{1.200000in}{3.033782in}%
\pgfsys@useobject{currentmarker}{}%
\end{pgfscope}%
\end{pgfscope}%
\begin{pgfscope}%
\pgfsetbuttcap%
\pgfsetroundjoin%
\definecolor{currentfill}{rgb}{0.000000,0.000000,0.000000}%
\pgfsetfillcolor{currentfill}%
\pgfsetlinewidth{0.501875pt}%
\definecolor{currentstroke}{rgb}{0.000000,0.000000,0.000000}%
\pgfsetstrokecolor{currentstroke}%
\pgfsetdash{}{0pt}%
\pgfsys@defobject{currentmarker}{\pgfqpoint{-0.027778in}{0.000000in}}{\pgfqpoint{0.000000in}{0.000000in}}{%
\pgfpathmoveto{\pgfqpoint{0.000000in}{0.000000in}}%
\pgfpathlineto{\pgfqpoint{-0.027778in}{0.000000in}}%
\pgfusepath{stroke,fill}%
}%
\begin{pgfscope}%
\pgfsys@transformshift{6.800000in}{3.033782in}%
\pgfsys@useobject{currentmarker}{}%
\end{pgfscope}%
\end{pgfscope}%
\begin{pgfscope}%
\pgfsetbuttcap%
\pgfsetroundjoin%
\definecolor{currentfill}{rgb}{0.000000,0.000000,0.000000}%
\pgfsetfillcolor{currentfill}%
\pgfsetlinewidth{0.501875pt}%
\definecolor{currentstroke}{rgb}{0.000000,0.000000,0.000000}%
\pgfsetstrokecolor{currentstroke}%
\pgfsetdash{}{0pt}%
\pgfsys@defobject{currentmarker}{\pgfqpoint{0.000000in}{0.000000in}}{\pgfqpoint{0.027778in}{0.000000in}}{%
\pgfpathmoveto{\pgfqpoint{0.000000in}{0.000000in}}%
\pgfpathlineto{\pgfqpoint{0.027778in}{0.000000in}}%
\pgfusepath{stroke,fill}%
}%
\begin{pgfscope}%
\pgfsys@transformshift{1.200000in}{3.114118in}%
\pgfsys@useobject{currentmarker}{}%
\end{pgfscope}%
\end{pgfscope}%
\begin{pgfscope}%
\pgfsetbuttcap%
\pgfsetroundjoin%
\definecolor{currentfill}{rgb}{0.000000,0.000000,0.000000}%
\pgfsetfillcolor{currentfill}%
\pgfsetlinewidth{0.501875pt}%
\definecolor{currentstroke}{rgb}{0.000000,0.000000,0.000000}%
\pgfsetstrokecolor{currentstroke}%
\pgfsetdash{}{0pt}%
\pgfsys@defobject{currentmarker}{\pgfqpoint{-0.027778in}{0.000000in}}{\pgfqpoint{0.000000in}{0.000000in}}{%
\pgfpathmoveto{\pgfqpoint{0.000000in}{0.000000in}}%
\pgfpathlineto{\pgfqpoint{-0.027778in}{0.000000in}}%
\pgfusepath{stroke,fill}%
}%
\begin{pgfscope}%
\pgfsys@transformshift{6.800000in}{3.114118in}%
\pgfsys@useobject{currentmarker}{}%
\end{pgfscope}%
\end{pgfscope}%
\begin{pgfscope}%
\pgfsetbuttcap%
\pgfsetroundjoin%
\definecolor{currentfill}{rgb}{0.000000,0.000000,0.000000}%
\pgfsetfillcolor{currentfill}%
\pgfsetlinewidth{0.501875pt}%
\definecolor{currentstroke}{rgb}{0.000000,0.000000,0.000000}%
\pgfsetstrokecolor{currentstroke}%
\pgfsetdash{}{0pt}%
\pgfsys@defobject{currentmarker}{\pgfqpoint{0.000000in}{0.000000in}}{\pgfqpoint{0.027778in}{0.000000in}}{%
\pgfpathmoveto{\pgfqpoint{0.000000in}{0.000000in}}%
\pgfpathlineto{\pgfqpoint{0.027778in}{0.000000in}}%
\pgfusepath{stroke,fill}%
}%
\begin{pgfscope}%
\pgfsys@transformshift{1.200000in}{3.183708in}%
\pgfsys@useobject{currentmarker}{}%
\end{pgfscope}%
\end{pgfscope}%
\begin{pgfscope}%
\pgfsetbuttcap%
\pgfsetroundjoin%
\definecolor{currentfill}{rgb}{0.000000,0.000000,0.000000}%
\pgfsetfillcolor{currentfill}%
\pgfsetlinewidth{0.501875pt}%
\definecolor{currentstroke}{rgb}{0.000000,0.000000,0.000000}%
\pgfsetstrokecolor{currentstroke}%
\pgfsetdash{}{0pt}%
\pgfsys@defobject{currentmarker}{\pgfqpoint{-0.027778in}{0.000000in}}{\pgfqpoint{0.000000in}{0.000000in}}{%
\pgfpathmoveto{\pgfqpoint{0.000000in}{0.000000in}}%
\pgfpathlineto{\pgfqpoint{-0.027778in}{0.000000in}}%
\pgfusepath{stroke,fill}%
}%
\begin{pgfscope}%
\pgfsys@transformshift{6.800000in}{3.183708in}%
\pgfsys@useobject{currentmarker}{}%
\end{pgfscope}%
\end{pgfscope}%
\begin{pgfscope}%
\pgfsetbuttcap%
\pgfsetroundjoin%
\definecolor{currentfill}{rgb}{0.000000,0.000000,0.000000}%
\pgfsetfillcolor{currentfill}%
\pgfsetlinewidth{0.501875pt}%
\definecolor{currentstroke}{rgb}{0.000000,0.000000,0.000000}%
\pgfsetstrokecolor{currentstroke}%
\pgfsetdash{}{0pt}%
\pgfsys@defobject{currentmarker}{\pgfqpoint{0.000000in}{0.000000in}}{\pgfqpoint{0.027778in}{0.000000in}}{%
\pgfpathmoveto{\pgfqpoint{0.000000in}{0.000000in}}%
\pgfpathlineto{\pgfqpoint{0.027778in}{0.000000in}}%
\pgfusepath{stroke,fill}%
}%
\begin{pgfscope}%
\pgfsys@transformshift{1.200000in}{3.245091in}%
\pgfsys@useobject{currentmarker}{}%
\end{pgfscope}%
\end{pgfscope}%
\begin{pgfscope}%
\pgfsetbuttcap%
\pgfsetroundjoin%
\definecolor{currentfill}{rgb}{0.000000,0.000000,0.000000}%
\pgfsetfillcolor{currentfill}%
\pgfsetlinewidth{0.501875pt}%
\definecolor{currentstroke}{rgb}{0.000000,0.000000,0.000000}%
\pgfsetstrokecolor{currentstroke}%
\pgfsetdash{}{0pt}%
\pgfsys@defobject{currentmarker}{\pgfqpoint{-0.027778in}{0.000000in}}{\pgfqpoint{0.000000in}{0.000000in}}{%
\pgfpathmoveto{\pgfqpoint{0.000000in}{0.000000in}}%
\pgfpathlineto{\pgfqpoint{-0.027778in}{0.000000in}}%
\pgfusepath{stroke,fill}%
}%
\begin{pgfscope}%
\pgfsys@transformshift{6.800000in}{3.245091in}%
\pgfsys@useobject{currentmarker}{}%
\end{pgfscope}%
\end{pgfscope}%
\begin{pgfscope}%
\pgfsetbuttcap%
\pgfsetroundjoin%
\definecolor{currentfill}{rgb}{0.000000,0.000000,0.000000}%
\pgfsetfillcolor{currentfill}%
\pgfsetlinewidth{0.501875pt}%
\definecolor{currentstroke}{rgb}{0.000000,0.000000,0.000000}%
\pgfsetstrokecolor{currentstroke}%
\pgfsetdash{}{0pt}%
\pgfsys@defobject{currentmarker}{\pgfqpoint{0.000000in}{0.000000in}}{\pgfqpoint{0.027778in}{0.000000in}}{%
\pgfpathmoveto{\pgfqpoint{0.000000in}{0.000000in}}%
\pgfpathlineto{\pgfqpoint{0.027778in}{0.000000in}}%
\pgfusepath{stroke,fill}%
}%
\begin{pgfscope}%
\pgfsys@transformshift{1.200000in}{3.661236in}%
\pgfsys@useobject{currentmarker}{}%
\end{pgfscope}%
\end{pgfscope}%
\begin{pgfscope}%
\pgfsetbuttcap%
\pgfsetroundjoin%
\definecolor{currentfill}{rgb}{0.000000,0.000000,0.000000}%
\pgfsetfillcolor{currentfill}%
\pgfsetlinewidth{0.501875pt}%
\definecolor{currentstroke}{rgb}{0.000000,0.000000,0.000000}%
\pgfsetstrokecolor{currentstroke}%
\pgfsetdash{}{0pt}%
\pgfsys@defobject{currentmarker}{\pgfqpoint{-0.027778in}{0.000000in}}{\pgfqpoint{0.000000in}{0.000000in}}{%
\pgfpathmoveto{\pgfqpoint{0.000000in}{0.000000in}}%
\pgfpathlineto{\pgfqpoint{-0.027778in}{0.000000in}}%
\pgfusepath{stroke,fill}%
}%
\begin{pgfscope}%
\pgfsys@transformshift{6.800000in}{3.661236in}%
\pgfsys@useobject{currentmarker}{}%
\end{pgfscope}%
\end{pgfscope}%
\begin{pgfscope}%
\pgfsetbuttcap%
\pgfsetroundjoin%
\definecolor{currentfill}{rgb}{0.000000,0.000000,0.000000}%
\pgfsetfillcolor{currentfill}%
\pgfsetlinewidth{0.501875pt}%
\definecolor{currentstroke}{rgb}{0.000000,0.000000,0.000000}%
\pgfsetstrokecolor{currentstroke}%
\pgfsetdash{}{0pt}%
\pgfsys@defobject{currentmarker}{\pgfqpoint{0.000000in}{0.000000in}}{\pgfqpoint{0.027778in}{0.000000in}}{%
\pgfpathmoveto{\pgfqpoint{0.000000in}{0.000000in}}%
\pgfpathlineto{\pgfqpoint{0.027778in}{0.000000in}}%
\pgfusepath{stroke,fill}%
}%
\begin{pgfscope}%
\pgfsys@transformshift{1.200000in}{3.872546in}%
\pgfsys@useobject{currentmarker}{}%
\end{pgfscope}%
\end{pgfscope}%
\begin{pgfscope}%
\pgfsetbuttcap%
\pgfsetroundjoin%
\definecolor{currentfill}{rgb}{0.000000,0.000000,0.000000}%
\pgfsetfillcolor{currentfill}%
\pgfsetlinewidth{0.501875pt}%
\definecolor{currentstroke}{rgb}{0.000000,0.000000,0.000000}%
\pgfsetstrokecolor{currentstroke}%
\pgfsetdash{}{0pt}%
\pgfsys@defobject{currentmarker}{\pgfqpoint{-0.027778in}{0.000000in}}{\pgfqpoint{0.000000in}{0.000000in}}{%
\pgfpathmoveto{\pgfqpoint{0.000000in}{0.000000in}}%
\pgfpathlineto{\pgfqpoint{-0.027778in}{0.000000in}}%
\pgfusepath{stroke,fill}%
}%
\begin{pgfscope}%
\pgfsys@transformshift{6.800000in}{3.872546in}%
\pgfsys@useobject{currentmarker}{}%
\end{pgfscope}%
\end{pgfscope}%
\begin{pgfscope}%
\pgfsetbuttcap%
\pgfsetroundjoin%
\definecolor{currentfill}{rgb}{0.000000,0.000000,0.000000}%
\pgfsetfillcolor{currentfill}%
\pgfsetlinewidth{0.501875pt}%
\definecolor{currentstroke}{rgb}{0.000000,0.000000,0.000000}%
\pgfsetstrokecolor{currentstroke}%
\pgfsetdash{}{0pt}%
\pgfsys@defobject{currentmarker}{\pgfqpoint{0.000000in}{0.000000in}}{\pgfqpoint{0.027778in}{0.000000in}}{%
\pgfpathmoveto{\pgfqpoint{0.000000in}{0.000000in}}%
\pgfpathlineto{\pgfqpoint{0.027778in}{0.000000in}}%
\pgfusepath{stroke,fill}%
}%
\begin{pgfscope}%
\pgfsys@transformshift{1.200000in}{4.022472in}%
\pgfsys@useobject{currentmarker}{}%
\end{pgfscope}%
\end{pgfscope}%
\begin{pgfscope}%
\pgfsetbuttcap%
\pgfsetroundjoin%
\definecolor{currentfill}{rgb}{0.000000,0.000000,0.000000}%
\pgfsetfillcolor{currentfill}%
\pgfsetlinewidth{0.501875pt}%
\definecolor{currentstroke}{rgb}{0.000000,0.000000,0.000000}%
\pgfsetstrokecolor{currentstroke}%
\pgfsetdash{}{0pt}%
\pgfsys@defobject{currentmarker}{\pgfqpoint{-0.027778in}{0.000000in}}{\pgfqpoint{0.000000in}{0.000000in}}{%
\pgfpathmoveto{\pgfqpoint{0.000000in}{0.000000in}}%
\pgfpathlineto{\pgfqpoint{-0.027778in}{0.000000in}}%
\pgfusepath{stroke,fill}%
}%
\begin{pgfscope}%
\pgfsys@transformshift{6.800000in}{4.022472in}%
\pgfsys@useobject{currentmarker}{}%
\end{pgfscope}%
\end{pgfscope}%
\begin{pgfscope}%
\pgfsetbuttcap%
\pgfsetroundjoin%
\definecolor{currentfill}{rgb}{0.000000,0.000000,0.000000}%
\pgfsetfillcolor{currentfill}%
\pgfsetlinewidth{0.501875pt}%
\definecolor{currentstroke}{rgb}{0.000000,0.000000,0.000000}%
\pgfsetstrokecolor{currentstroke}%
\pgfsetdash{}{0pt}%
\pgfsys@defobject{currentmarker}{\pgfqpoint{0.000000in}{0.000000in}}{\pgfqpoint{0.027778in}{0.000000in}}{%
\pgfpathmoveto{\pgfqpoint{0.000000in}{0.000000in}}%
\pgfpathlineto{\pgfqpoint{0.027778in}{0.000000in}}%
\pgfusepath{stroke,fill}%
}%
\begin{pgfscope}%
\pgfsys@transformshift{1.200000in}{4.138764in}%
\pgfsys@useobject{currentmarker}{}%
\end{pgfscope}%
\end{pgfscope}%
\begin{pgfscope}%
\pgfsetbuttcap%
\pgfsetroundjoin%
\definecolor{currentfill}{rgb}{0.000000,0.000000,0.000000}%
\pgfsetfillcolor{currentfill}%
\pgfsetlinewidth{0.501875pt}%
\definecolor{currentstroke}{rgb}{0.000000,0.000000,0.000000}%
\pgfsetstrokecolor{currentstroke}%
\pgfsetdash{}{0pt}%
\pgfsys@defobject{currentmarker}{\pgfqpoint{-0.027778in}{0.000000in}}{\pgfqpoint{0.000000in}{0.000000in}}{%
\pgfpathmoveto{\pgfqpoint{0.000000in}{0.000000in}}%
\pgfpathlineto{\pgfqpoint{-0.027778in}{0.000000in}}%
\pgfusepath{stroke,fill}%
}%
\begin{pgfscope}%
\pgfsys@transformshift{6.800000in}{4.138764in}%
\pgfsys@useobject{currentmarker}{}%
\end{pgfscope}%
\end{pgfscope}%
\begin{pgfscope}%
\pgfsetbuttcap%
\pgfsetroundjoin%
\definecolor{currentfill}{rgb}{0.000000,0.000000,0.000000}%
\pgfsetfillcolor{currentfill}%
\pgfsetlinewidth{0.501875pt}%
\definecolor{currentstroke}{rgb}{0.000000,0.000000,0.000000}%
\pgfsetstrokecolor{currentstroke}%
\pgfsetdash{}{0pt}%
\pgfsys@defobject{currentmarker}{\pgfqpoint{0.000000in}{0.000000in}}{\pgfqpoint{0.027778in}{0.000000in}}{%
\pgfpathmoveto{\pgfqpoint{0.000000in}{0.000000in}}%
\pgfpathlineto{\pgfqpoint{0.027778in}{0.000000in}}%
\pgfusepath{stroke,fill}%
}%
\begin{pgfscope}%
\pgfsys@transformshift{1.200000in}{4.233782in}%
\pgfsys@useobject{currentmarker}{}%
\end{pgfscope}%
\end{pgfscope}%
\begin{pgfscope}%
\pgfsetbuttcap%
\pgfsetroundjoin%
\definecolor{currentfill}{rgb}{0.000000,0.000000,0.000000}%
\pgfsetfillcolor{currentfill}%
\pgfsetlinewidth{0.501875pt}%
\definecolor{currentstroke}{rgb}{0.000000,0.000000,0.000000}%
\pgfsetstrokecolor{currentstroke}%
\pgfsetdash{}{0pt}%
\pgfsys@defobject{currentmarker}{\pgfqpoint{-0.027778in}{0.000000in}}{\pgfqpoint{0.000000in}{0.000000in}}{%
\pgfpathmoveto{\pgfqpoint{0.000000in}{0.000000in}}%
\pgfpathlineto{\pgfqpoint{-0.027778in}{0.000000in}}%
\pgfusepath{stroke,fill}%
}%
\begin{pgfscope}%
\pgfsys@transformshift{6.800000in}{4.233782in}%
\pgfsys@useobject{currentmarker}{}%
\end{pgfscope}%
\end{pgfscope}%
\begin{pgfscope}%
\pgfsetbuttcap%
\pgfsetroundjoin%
\definecolor{currentfill}{rgb}{0.000000,0.000000,0.000000}%
\pgfsetfillcolor{currentfill}%
\pgfsetlinewidth{0.501875pt}%
\definecolor{currentstroke}{rgb}{0.000000,0.000000,0.000000}%
\pgfsetstrokecolor{currentstroke}%
\pgfsetdash{}{0pt}%
\pgfsys@defobject{currentmarker}{\pgfqpoint{0.000000in}{0.000000in}}{\pgfqpoint{0.027778in}{0.000000in}}{%
\pgfpathmoveto{\pgfqpoint{0.000000in}{0.000000in}}%
\pgfpathlineto{\pgfqpoint{0.027778in}{0.000000in}}%
\pgfusepath{stroke,fill}%
}%
\begin{pgfscope}%
\pgfsys@transformshift{1.200000in}{4.314118in}%
\pgfsys@useobject{currentmarker}{}%
\end{pgfscope}%
\end{pgfscope}%
\begin{pgfscope}%
\pgfsetbuttcap%
\pgfsetroundjoin%
\definecolor{currentfill}{rgb}{0.000000,0.000000,0.000000}%
\pgfsetfillcolor{currentfill}%
\pgfsetlinewidth{0.501875pt}%
\definecolor{currentstroke}{rgb}{0.000000,0.000000,0.000000}%
\pgfsetstrokecolor{currentstroke}%
\pgfsetdash{}{0pt}%
\pgfsys@defobject{currentmarker}{\pgfqpoint{-0.027778in}{0.000000in}}{\pgfqpoint{0.000000in}{0.000000in}}{%
\pgfpathmoveto{\pgfqpoint{0.000000in}{0.000000in}}%
\pgfpathlineto{\pgfqpoint{-0.027778in}{0.000000in}}%
\pgfusepath{stroke,fill}%
}%
\begin{pgfscope}%
\pgfsys@transformshift{6.800000in}{4.314118in}%
\pgfsys@useobject{currentmarker}{}%
\end{pgfscope}%
\end{pgfscope}%
\begin{pgfscope}%
\pgfsetbuttcap%
\pgfsetroundjoin%
\definecolor{currentfill}{rgb}{0.000000,0.000000,0.000000}%
\pgfsetfillcolor{currentfill}%
\pgfsetlinewidth{0.501875pt}%
\definecolor{currentstroke}{rgb}{0.000000,0.000000,0.000000}%
\pgfsetstrokecolor{currentstroke}%
\pgfsetdash{}{0pt}%
\pgfsys@defobject{currentmarker}{\pgfqpoint{0.000000in}{0.000000in}}{\pgfqpoint{0.027778in}{0.000000in}}{%
\pgfpathmoveto{\pgfqpoint{0.000000in}{0.000000in}}%
\pgfpathlineto{\pgfqpoint{0.027778in}{0.000000in}}%
\pgfusepath{stroke,fill}%
}%
\begin{pgfscope}%
\pgfsys@transformshift{1.200000in}{4.383708in}%
\pgfsys@useobject{currentmarker}{}%
\end{pgfscope}%
\end{pgfscope}%
\begin{pgfscope}%
\pgfsetbuttcap%
\pgfsetroundjoin%
\definecolor{currentfill}{rgb}{0.000000,0.000000,0.000000}%
\pgfsetfillcolor{currentfill}%
\pgfsetlinewidth{0.501875pt}%
\definecolor{currentstroke}{rgb}{0.000000,0.000000,0.000000}%
\pgfsetstrokecolor{currentstroke}%
\pgfsetdash{}{0pt}%
\pgfsys@defobject{currentmarker}{\pgfqpoint{-0.027778in}{0.000000in}}{\pgfqpoint{0.000000in}{0.000000in}}{%
\pgfpathmoveto{\pgfqpoint{0.000000in}{0.000000in}}%
\pgfpathlineto{\pgfqpoint{-0.027778in}{0.000000in}}%
\pgfusepath{stroke,fill}%
}%
\begin{pgfscope}%
\pgfsys@transformshift{6.800000in}{4.383708in}%
\pgfsys@useobject{currentmarker}{}%
\end{pgfscope}%
\end{pgfscope}%
\begin{pgfscope}%
\pgfsetbuttcap%
\pgfsetroundjoin%
\definecolor{currentfill}{rgb}{0.000000,0.000000,0.000000}%
\pgfsetfillcolor{currentfill}%
\pgfsetlinewidth{0.501875pt}%
\definecolor{currentstroke}{rgb}{0.000000,0.000000,0.000000}%
\pgfsetstrokecolor{currentstroke}%
\pgfsetdash{}{0pt}%
\pgfsys@defobject{currentmarker}{\pgfqpoint{0.000000in}{0.000000in}}{\pgfqpoint{0.027778in}{0.000000in}}{%
\pgfpathmoveto{\pgfqpoint{0.000000in}{0.000000in}}%
\pgfpathlineto{\pgfqpoint{0.027778in}{0.000000in}}%
\pgfusepath{stroke,fill}%
}%
\begin{pgfscope}%
\pgfsys@transformshift{1.200000in}{4.445091in}%
\pgfsys@useobject{currentmarker}{}%
\end{pgfscope}%
\end{pgfscope}%
\begin{pgfscope}%
\pgfsetbuttcap%
\pgfsetroundjoin%
\definecolor{currentfill}{rgb}{0.000000,0.000000,0.000000}%
\pgfsetfillcolor{currentfill}%
\pgfsetlinewidth{0.501875pt}%
\definecolor{currentstroke}{rgb}{0.000000,0.000000,0.000000}%
\pgfsetstrokecolor{currentstroke}%
\pgfsetdash{}{0pt}%
\pgfsys@defobject{currentmarker}{\pgfqpoint{-0.027778in}{0.000000in}}{\pgfqpoint{0.000000in}{0.000000in}}{%
\pgfpathmoveto{\pgfqpoint{0.000000in}{0.000000in}}%
\pgfpathlineto{\pgfqpoint{-0.027778in}{0.000000in}}%
\pgfusepath{stroke,fill}%
}%
\begin{pgfscope}%
\pgfsys@transformshift{6.800000in}{4.445091in}%
\pgfsys@useobject{currentmarker}{}%
\end{pgfscope}%
\end{pgfscope}%
\begin{pgfscope}%
\pgfsetbuttcap%
\pgfsetroundjoin%
\definecolor{currentfill}{rgb}{0.000000,0.000000,0.000000}%
\pgfsetfillcolor{currentfill}%
\pgfsetlinewidth{0.501875pt}%
\definecolor{currentstroke}{rgb}{0.000000,0.000000,0.000000}%
\pgfsetstrokecolor{currentstroke}%
\pgfsetdash{}{0pt}%
\pgfsys@defobject{currentmarker}{\pgfqpoint{0.000000in}{0.000000in}}{\pgfqpoint{0.027778in}{0.000000in}}{%
\pgfpathmoveto{\pgfqpoint{0.000000in}{0.000000in}}%
\pgfpathlineto{\pgfqpoint{0.027778in}{0.000000in}}%
\pgfusepath{stroke,fill}%
}%
\begin{pgfscope}%
\pgfsys@transformshift{1.200000in}{4.861236in}%
\pgfsys@useobject{currentmarker}{}%
\end{pgfscope}%
\end{pgfscope}%
\begin{pgfscope}%
\pgfsetbuttcap%
\pgfsetroundjoin%
\definecolor{currentfill}{rgb}{0.000000,0.000000,0.000000}%
\pgfsetfillcolor{currentfill}%
\pgfsetlinewidth{0.501875pt}%
\definecolor{currentstroke}{rgb}{0.000000,0.000000,0.000000}%
\pgfsetstrokecolor{currentstroke}%
\pgfsetdash{}{0pt}%
\pgfsys@defobject{currentmarker}{\pgfqpoint{-0.027778in}{0.000000in}}{\pgfqpoint{0.000000in}{0.000000in}}{%
\pgfpathmoveto{\pgfqpoint{0.000000in}{0.000000in}}%
\pgfpathlineto{\pgfqpoint{-0.027778in}{0.000000in}}%
\pgfusepath{stroke,fill}%
}%
\begin{pgfscope}%
\pgfsys@transformshift{6.800000in}{4.861236in}%
\pgfsys@useobject{currentmarker}{}%
\end{pgfscope}%
\end{pgfscope}%
\begin{pgfscope}%
\pgfsetbuttcap%
\pgfsetroundjoin%
\definecolor{currentfill}{rgb}{0.000000,0.000000,0.000000}%
\pgfsetfillcolor{currentfill}%
\pgfsetlinewidth{0.501875pt}%
\definecolor{currentstroke}{rgb}{0.000000,0.000000,0.000000}%
\pgfsetstrokecolor{currentstroke}%
\pgfsetdash{}{0pt}%
\pgfsys@defobject{currentmarker}{\pgfqpoint{0.000000in}{0.000000in}}{\pgfqpoint{0.027778in}{0.000000in}}{%
\pgfpathmoveto{\pgfqpoint{0.000000in}{0.000000in}}%
\pgfpathlineto{\pgfqpoint{0.027778in}{0.000000in}}%
\pgfusepath{stroke,fill}%
}%
\begin{pgfscope}%
\pgfsys@transformshift{1.200000in}{5.072546in}%
\pgfsys@useobject{currentmarker}{}%
\end{pgfscope}%
\end{pgfscope}%
\begin{pgfscope}%
\pgfsetbuttcap%
\pgfsetroundjoin%
\definecolor{currentfill}{rgb}{0.000000,0.000000,0.000000}%
\pgfsetfillcolor{currentfill}%
\pgfsetlinewidth{0.501875pt}%
\definecolor{currentstroke}{rgb}{0.000000,0.000000,0.000000}%
\pgfsetstrokecolor{currentstroke}%
\pgfsetdash{}{0pt}%
\pgfsys@defobject{currentmarker}{\pgfqpoint{-0.027778in}{0.000000in}}{\pgfqpoint{0.000000in}{0.000000in}}{%
\pgfpathmoveto{\pgfqpoint{0.000000in}{0.000000in}}%
\pgfpathlineto{\pgfqpoint{-0.027778in}{0.000000in}}%
\pgfusepath{stroke,fill}%
}%
\begin{pgfscope}%
\pgfsys@transformshift{6.800000in}{5.072546in}%
\pgfsys@useobject{currentmarker}{}%
\end{pgfscope}%
\end{pgfscope}%
\begin{pgfscope}%
\pgfsetbuttcap%
\pgfsetroundjoin%
\definecolor{currentfill}{rgb}{0.000000,0.000000,0.000000}%
\pgfsetfillcolor{currentfill}%
\pgfsetlinewidth{0.501875pt}%
\definecolor{currentstroke}{rgb}{0.000000,0.000000,0.000000}%
\pgfsetstrokecolor{currentstroke}%
\pgfsetdash{}{0pt}%
\pgfsys@defobject{currentmarker}{\pgfqpoint{0.000000in}{0.000000in}}{\pgfqpoint{0.027778in}{0.000000in}}{%
\pgfpathmoveto{\pgfqpoint{0.000000in}{0.000000in}}%
\pgfpathlineto{\pgfqpoint{0.027778in}{0.000000in}}%
\pgfusepath{stroke,fill}%
}%
\begin{pgfscope}%
\pgfsys@transformshift{1.200000in}{5.222472in}%
\pgfsys@useobject{currentmarker}{}%
\end{pgfscope}%
\end{pgfscope}%
\begin{pgfscope}%
\pgfsetbuttcap%
\pgfsetroundjoin%
\definecolor{currentfill}{rgb}{0.000000,0.000000,0.000000}%
\pgfsetfillcolor{currentfill}%
\pgfsetlinewidth{0.501875pt}%
\definecolor{currentstroke}{rgb}{0.000000,0.000000,0.000000}%
\pgfsetstrokecolor{currentstroke}%
\pgfsetdash{}{0pt}%
\pgfsys@defobject{currentmarker}{\pgfqpoint{-0.027778in}{0.000000in}}{\pgfqpoint{0.000000in}{0.000000in}}{%
\pgfpathmoveto{\pgfqpoint{0.000000in}{0.000000in}}%
\pgfpathlineto{\pgfqpoint{-0.027778in}{0.000000in}}%
\pgfusepath{stroke,fill}%
}%
\begin{pgfscope}%
\pgfsys@transformshift{6.800000in}{5.222472in}%
\pgfsys@useobject{currentmarker}{}%
\end{pgfscope}%
\end{pgfscope}%
\begin{pgfscope}%
\pgfsetbuttcap%
\pgfsetroundjoin%
\definecolor{currentfill}{rgb}{0.000000,0.000000,0.000000}%
\pgfsetfillcolor{currentfill}%
\pgfsetlinewidth{0.501875pt}%
\definecolor{currentstroke}{rgb}{0.000000,0.000000,0.000000}%
\pgfsetstrokecolor{currentstroke}%
\pgfsetdash{}{0pt}%
\pgfsys@defobject{currentmarker}{\pgfqpoint{0.000000in}{0.000000in}}{\pgfqpoint{0.027778in}{0.000000in}}{%
\pgfpathmoveto{\pgfqpoint{0.000000in}{0.000000in}}%
\pgfpathlineto{\pgfqpoint{0.027778in}{0.000000in}}%
\pgfusepath{stroke,fill}%
}%
\begin{pgfscope}%
\pgfsys@transformshift{1.200000in}{5.338764in}%
\pgfsys@useobject{currentmarker}{}%
\end{pgfscope}%
\end{pgfscope}%
\begin{pgfscope}%
\pgfsetbuttcap%
\pgfsetroundjoin%
\definecolor{currentfill}{rgb}{0.000000,0.000000,0.000000}%
\pgfsetfillcolor{currentfill}%
\pgfsetlinewidth{0.501875pt}%
\definecolor{currentstroke}{rgb}{0.000000,0.000000,0.000000}%
\pgfsetstrokecolor{currentstroke}%
\pgfsetdash{}{0pt}%
\pgfsys@defobject{currentmarker}{\pgfqpoint{-0.027778in}{0.000000in}}{\pgfqpoint{0.000000in}{0.000000in}}{%
\pgfpathmoveto{\pgfqpoint{0.000000in}{0.000000in}}%
\pgfpathlineto{\pgfqpoint{-0.027778in}{0.000000in}}%
\pgfusepath{stroke,fill}%
}%
\begin{pgfscope}%
\pgfsys@transformshift{6.800000in}{5.338764in}%
\pgfsys@useobject{currentmarker}{}%
\end{pgfscope}%
\end{pgfscope}%
\begin{pgfscope}%
\pgfsetbuttcap%
\pgfsetroundjoin%
\definecolor{currentfill}{rgb}{0.000000,0.000000,0.000000}%
\pgfsetfillcolor{currentfill}%
\pgfsetlinewidth{0.501875pt}%
\definecolor{currentstroke}{rgb}{0.000000,0.000000,0.000000}%
\pgfsetstrokecolor{currentstroke}%
\pgfsetdash{}{0pt}%
\pgfsys@defobject{currentmarker}{\pgfqpoint{0.000000in}{0.000000in}}{\pgfqpoint{0.027778in}{0.000000in}}{%
\pgfpathmoveto{\pgfqpoint{0.000000in}{0.000000in}}%
\pgfpathlineto{\pgfqpoint{0.027778in}{0.000000in}}%
\pgfusepath{stroke,fill}%
}%
\begin{pgfscope}%
\pgfsys@transformshift{1.200000in}{5.433782in}%
\pgfsys@useobject{currentmarker}{}%
\end{pgfscope}%
\end{pgfscope}%
\begin{pgfscope}%
\pgfsetbuttcap%
\pgfsetroundjoin%
\definecolor{currentfill}{rgb}{0.000000,0.000000,0.000000}%
\pgfsetfillcolor{currentfill}%
\pgfsetlinewidth{0.501875pt}%
\definecolor{currentstroke}{rgb}{0.000000,0.000000,0.000000}%
\pgfsetstrokecolor{currentstroke}%
\pgfsetdash{}{0pt}%
\pgfsys@defobject{currentmarker}{\pgfqpoint{-0.027778in}{0.000000in}}{\pgfqpoint{0.000000in}{0.000000in}}{%
\pgfpathmoveto{\pgfqpoint{0.000000in}{0.000000in}}%
\pgfpathlineto{\pgfqpoint{-0.027778in}{0.000000in}}%
\pgfusepath{stroke,fill}%
}%
\begin{pgfscope}%
\pgfsys@transformshift{6.800000in}{5.433782in}%
\pgfsys@useobject{currentmarker}{}%
\end{pgfscope}%
\end{pgfscope}%
\begin{pgfscope}%
\pgfsetbuttcap%
\pgfsetroundjoin%
\definecolor{currentfill}{rgb}{0.000000,0.000000,0.000000}%
\pgfsetfillcolor{currentfill}%
\pgfsetlinewidth{0.501875pt}%
\definecolor{currentstroke}{rgb}{0.000000,0.000000,0.000000}%
\pgfsetstrokecolor{currentstroke}%
\pgfsetdash{}{0pt}%
\pgfsys@defobject{currentmarker}{\pgfqpoint{0.000000in}{0.000000in}}{\pgfqpoint{0.027778in}{0.000000in}}{%
\pgfpathmoveto{\pgfqpoint{0.000000in}{0.000000in}}%
\pgfpathlineto{\pgfqpoint{0.027778in}{0.000000in}}%
\pgfusepath{stroke,fill}%
}%
\begin{pgfscope}%
\pgfsys@transformshift{1.200000in}{5.514118in}%
\pgfsys@useobject{currentmarker}{}%
\end{pgfscope}%
\end{pgfscope}%
\begin{pgfscope}%
\pgfsetbuttcap%
\pgfsetroundjoin%
\definecolor{currentfill}{rgb}{0.000000,0.000000,0.000000}%
\pgfsetfillcolor{currentfill}%
\pgfsetlinewidth{0.501875pt}%
\definecolor{currentstroke}{rgb}{0.000000,0.000000,0.000000}%
\pgfsetstrokecolor{currentstroke}%
\pgfsetdash{}{0pt}%
\pgfsys@defobject{currentmarker}{\pgfqpoint{-0.027778in}{0.000000in}}{\pgfqpoint{0.000000in}{0.000000in}}{%
\pgfpathmoveto{\pgfqpoint{0.000000in}{0.000000in}}%
\pgfpathlineto{\pgfqpoint{-0.027778in}{0.000000in}}%
\pgfusepath{stroke,fill}%
}%
\begin{pgfscope}%
\pgfsys@transformshift{6.800000in}{5.514118in}%
\pgfsys@useobject{currentmarker}{}%
\end{pgfscope}%
\end{pgfscope}%
\begin{pgfscope}%
\pgfsetbuttcap%
\pgfsetroundjoin%
\definecolor{currentfill}{rgb}{0.000000,0.000000,0.000000}%
\pgfsetfillcolor{currentfill}%
\pgfsetlinewidth{0.501875pt}%
\definecolor{currentstroke}{rgb}{0.000000,0.000000,0.000000}%
\pgfsetstrokecolor{currentstroke}%
\pgfsetdash{}{0pt}%
\pgfsys@defobject{currentmarker}{\pgfqpoint{0.000000in}{0.000000in}}{\pgfqpoint{0.027778in}{0.000000in}}{%
\pgfpathmoveto{\pgfqpoint{0.000000in}{0.000000in}}%
\pgfpathlineto{\pgfqpoint{0.027778in}{0.000000in}}%
\pgfusepath{stroke,fill}%
}%
\begin{pgfscope}%
\pgfsys@transformshift{1.200000in}{5.583708in}%
\pgfsys@useobject{currentmarker}{}%
\end{pgfscope}%
\end{pgfscope}%
\begin{pgfscope}%
\pgfsetbuttcap%
\pgfsetroundjoin%
\definecolor{currentfill}{rgb}{0.000000,0.000000,0.000000}%
\pgfsetfillcolor{currentfill}%
\pgfsetlinewidth{0.501875pt}%
\definecolor{currentstroke}{rgb}{0.000000,0.000000,0.000000}%
\pgfsetstrokecolor{currentstroke}%
\pgfsetdash{}{0pt}%
\pgfsys@defobject{currentmarker}{\pgfqpoint{-0.027778in}{0.000000in}}{\pgfqpoint{0.000000in}{0.000000in}}{%
\pgfpathmoveto{\pgfqpoint{0.000000in}{0.000000in}}%
\pgfpathlineto{\pgfqpoint{-0.027778in}{0.000000in}}%
\pgfusepath{stroke,fill}%
}%
\begin{pgfscope}%
\pgfsys@transformshift{6.800000in}{5.583708in}%
\pgfsys@useobject{currentmarker}{}%
\end{pgfscope}%
\end{pgfscope}%
\begin{pgfscope}%
\pgfsetbuttcap%
\pgfsetroundjoin%
\definecolor{currentfill}{rgb}{0.000000,0.000000,0.000000}%
\pgfsetfillcolor{currentfill}%
\pgfsetlinewidth{0.501875pt}%
\definecolor{currentstroke}{rgb}{0.000000,0.000000,0.000000}%
\pgfsetstrokecolor{currentstroke}%
\pgfsetdash{}{0pt}%
\pgfsys@defobject{currentmarker}{\pgfqpoint{0.000000in}{0.000000in}}{\pgfqpoint{0.027778in}{0.000000in}}{%
\pgfpathmoveto{\pgfqpoint{0.000000in}{0.000000in}}%
\pgfpathlineto{\pgfqpoint{0.027778in}{0.000000in}}%
\pgfusepath{stroke,fill}%
}%
\begin{pgfscope}%
\pgfsys@transformshift{1.200000in}{5.645091in}%
\pgfsys@useobject{currentmarker}{}%
\end{pgfscope}%
\end{pgfscope}%
\begin{pgfscope}%
\pgfsetbuttcap%
\pgfsetroundjoin%
\definecolor{currentfill}{rgb}{0.000000,0.000000,0.000000}%
\pgfsetfillcolor{currentfill}%
\pgfsetlinewidth{0.501875pt}%
\definecolor{currentstroke}{rgb}{0.000000,0.000000,0.000000}%
\pgfsetstrokecolor{currentstroke}%
\pgfsetdash{}{0pt}%
\pgfsys@defobject{currentmarker}{\pgfqpoint{-0.027778in}{0.000000in}}{\pgfqpoint{0.000000in}{0.000000in}}{%
\pgfpathmoveto{\pgfqpoint{0.000000in}{0.000000in}}%
\pgfpathlineto{\pgfqpoint{-0.027778in}{0.000000in}}%
\pgfusepath{stroke,fill}%
}%
\begin{pgfscope}%
\pgfsys@transformshift{6.800000in}{5.645091in}%
\pgfsys@useobject{currentmarker}{}%
\end{pgfscope}%
\end{pgfscope}%
\begin{pgfscope}%
\definecolor{textcolor}{rgb}{0.000000,0.000000,0.000000}%
\pgfsetstrokecolor{textcolor}%
\pgfsetfillcolor{textcolor}%
\pgftext[x=0.705925in,y=3.300000in,,bottom,rotate=90.000000]{\color{textcolor}\sffamily\fontsize{20.000000}{24.000000}\selectfont \(\displaystyle Count\)}%
\end{pgfscope}%
\end{pgfpicture}%
\makeatother%
\endgroup%
}
    \caption{\label{fig:penum} $N_{pe}$ histogram}
\end{figure}

% section Towards Timing Resolution (end)