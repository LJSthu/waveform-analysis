A figure of merit to distinguish the difference between reconstruction result and the truth is crucial. In many cases, the reconstruction result and truth can both be regarded as sampling result from some form of probability distribution. Here we introduce Wasserstein distance as a figure of merit which can contribute to the improvement of timing resolution. 

Suppose $P(X)$ and $Q(X)$ are 2 probability density function of random variable $X$. There are many ways to measure the difference (distance) between $P(X)$ and $Q(X)$: 

\begin{minipage}{.3\textwidth}
\begin{align*}
    L_{1} &= \int|p-q| \ud t \\
    L_{2} &= \int(p-q)^{2} \ud t \\
    \chi^{2} &= \int\frac{(p-q)^{2}}{q} \ud t \\
    \cdots
\end{align*}
\end{minipage}
\begin{minipage}{.7\textwidth}
\begin{figure}[H]
    \centering
    \scalebox{0.4}{%% Creator: Matplotlib, PGF backend
%%
%% To include the figure in your LaTeX document, write
%%   \input{<filename>.pgf}
%%
%% Make sure the required packages are loaded in your preamble
%%   \usepackage{pgf}
%%
%% and, on pdftex
%%   \usepackage[utf8]{inputenc}\DeclareUnicodeCharacter{2212}{-}
%%
%% or, on luatex and xetex
%%   \usepackage{unicode-math}
%%
%% Figures using additional raster images can only be included by \input if
%% they are in the same directory as the main LaTeX file. For loading figures
%% from other directories you can use the `import` package
%%   \usepackage{import}
%%
%% and then include the figures with
%%   \import{<path to file>}{<filename>.pgf}
%%
%% Matplotlib used the following preamble
%%   \usepackage[detect-all,locale=DE]{siunitx}
%%
\begingroup%
\makeatletter%
\begin{pgfpicture}%
\pgfpathrectangle{\pgfpointorigin}{\pgfqpoint{6.400000in}{4.800000in}}%
\pgfusepath{use as bounding box, clip}%
\begin{pgfscope}%
\pgfsetbuttcap%
\pgfsetmiterjoin%
\definecolor{currentfill}{rgb}{1.000000,1.000000,1.000000}%
\pgfsetfillcolor{currentfill}%
\pgfsetlinewidth{0.000000pt}%
\definecolor{currentstroke}{rgb}{1.000000,1.000000,1.000000}%
\pgfsetstrokecolor{currentstroke}%
\pgfsetdash{}{0pt}%
\pgfpathmoveto{\pgfqpoint{0.000000in}{0.000000in}}%
\pgfpathlineto{\pgfqpoint{6.400000in}{0.000000in}}%
\pgfpathlineto{\pgfqpoint{6.400000in}{4.800000in}}%
\pgfpathlineto{\pgfqpoint{0.000000in}{4.800000in}}%
\pgfpathclose%
\pgfusepath{fill}%
\end{pgfscope}%
\begin{pgfscope}%
\pgfsetbuttcap%
\pgfsetmiterjoin%
\definecolor{currentfill}{rgb}{1.000000,1.000000,1.000000}%
\pgfsetfillcolor{currentfill}%
\pgfsetlinewidth{0.000000pt}%
\definecolor{currentstroke}{rgb}{0.000000,0.000000,0.000000}%
\pgfsetstrokecolor{currentstroke}%
\pgfsetstrokeopacity{0.000000}%
\pgfsetdash{}{0pt}%
\pgfpathmoveto{\pgfqpoint{0.768000in}{3.081600in}}%
\pgfpathlineto{\pgfqpoint{2.981333in}{3.081600in}}%
\pgfpathlineto{\pgfqpoint{2.981333in}{4.656000in}}%
\pgfpathlineto{\pgfqpoint{0.768000in}{4.656000in}}%
\pgfpathclose%
\pgfusepath{fill}%
\end{pgfscope}%
\begin{pgfscope}%
\pgfpathrectangle{\pgfqpoint{0.768000in}{3.081600in}}{\pgfqpoint{2.213333in}{1.574400in}}%
\pgfusepath{clip}%
\pgfsetbuttcap%
\pgfsetmiterjoin%
\definecolor{currentfill}{rgb}{1.000000,0.000000,0.000000}%
\pgfsetfillcolor{currentfill}%
\pgfsetlinewidth{0.000000pt}%
\definecolor{currentstroke}{rgb}{0.000000,0.000000,0.000000}%
\pgfsetstrokecolor{currentstroke}%
\pgfsetstrokeopacity{0.000000}%
\pgfsetdash{}{0pt}%
\pgfpathmoveto{\pgfqpoint{0.868606in}{3.081600in}}%
\pgfpathlineto{\pgfqpoint{1.292211in}{3.081600in}}%
\pgfpathlineto{\pgfqpoint{1.292211in}{3.795614in}}%
\pgfpathlineto{\pgfqpoint{0.868606in}{3.795614in}}%
\pgfpathclose%
\pgfusepath{fill}%
\end{pgfscope}%
\begin{pgfscope}%
\pgfpathrectangle{\pgfqpoint{0.768000in}{3.081600in}}{\pgfqpoint{2.213333in}{1.574400in}}%
\pgfusepath{clip}%
\pgfsetbuttcap%
\pgfsetmiterjoin%
\definecolor{currentfill}{rgb}{1.000000,0.000000,0.000000}%
\pgfsetfillcolor{currentfill}%
\pgfsetlinewidth{0.000000pt}%
\definecolor{currentstroke}{rgb}{0.000000,0.000000,0.000000}%
\pgfsetstrokecolor{currentstroke}%
\pgfsetstrokeopacity{0.000000}%
\pgfsetdash{}{0pt}%
\pgfpathmoveto{\pgfqpoint{1.398112in}{3.081600in}}%
\pgfpathlineto{\pgfqpoint{1.821716in}{3.081600in}}%
\pgfpathlineto{\pgfqpoint{1.821716in}{3.795614in}}%
\pgfpathlineto{\pgfqpoint{1.398112in}{3.795614in}}%
\pgfpathclose%
\pgfusepath{fill}%
\end{pgfscope}%
\begin{pgfscope}%
\pgfpathrectangle{\pgfqpoint{0.768000in}{3.081600in}}{\pgfqpoint{2.213333in}{1.574400in}}%
\pgfusepath{clip}%
\pgfsetbuttcap%
\pgfsetmiterjoin%
\definecolor{currentfill}{rgb}{1.000000,0.000000,0.000000}%
\pgfsetfillcolor{currentfill}%
\pgfsetlinewidth{0.000000pt}%
\definecolor{currentstroke}{rgb}{0.000000,0.000000,0.000000}%
\pgfsetstrokecolor{currentstroke}%
\pgfsetstrokeopacity{0.000000}%
\pgfsetdash{}{0pt}%
\pgfpathmoveto{\pgfqpoint{1.927617in}{3.081600in}}%
\pgfpathlineto{\pgfqpoint{2.351222in}{3.081600in}}%
\pgfpathlineto{\pgfqpoint{2.351222in}{3.795614in}}%
\pgfpathlineto{\pgfqpoint{1.927617in}{3.795614in}}%
\pgfpathclose%
\pgfusepath{fill}%
\end{pgfscope}%
\begin{pgfscope}%
\pgfpathrectangle{\pgfqpoint{0.768000in}{3.081600in}}{\pgfqpoint{2.213333in}{1.574400in}}%
\pgfusepath{clip}%
\pgfsetbuttcap%
\pgfsetmiterjoin%
\definecolor{currentfill}{rgb}{1.000000,0.000000,0.000000}%
\pgfsetfillcolor{currentfill}%
\pgfsetlinewidth{0.000000pt}%
\definecolor{currentstroke}{rgb}{0.000000,0.000000,0.000000}%
\pgfsetstrokecolor{currentstroke}%
\pgfsetstrokeopacity{0.000000}%
\pgfsetdash{}{0pt}%
\pgfpathmoveto{\pgfqpoint{2.457123in}{3.081600in}}%
\pgfpathlineto{\pgfqpoint{2.880727in}{3.081600in}}%
\pgfpathlineto{\pgfqpoint{2.880727in}{4.509627in}}%
\pgfpathlineto{\pgfqpoint{2.457123in}{4.509627in}}%
\pgfpathclose%
\pgfusepath{fill}%
\end{pgfscope}%
\begin{pgfscope}%
\pgfsetbuttcap%
\pgfsetroundjoin%
\definecolor{currentfill}{rgb}{0.000000,0.000000,0.000000}%
\pgfsetfillcolor{currentfill}%
\pgfsetlinewidth{0.803000pt}%
\definecolor{currentstroke}{rgb}{0.000000,0.000000,0.000000}%
\pgfsetstrokecolor{currentstroke}%
\pgfsetdash{}{0pt}%
\pgfsys@defobject{currentmarker}{\pgfqpoint{0.000000in}{-0.048611in}}{\pgfqpoint{0.000000in}{0.000000in}}{%
\pgfpathmoveto{\pgfqpoint{0.000000in}{0.000000in}}%
\pgfpathlineto{\pgfqpoint{0.000000in}{-0.048611in}}%
\pgfusepath{stroke,fill}%
}%
\begin{pgfscope}%
\pgfsys@transformshift{1.080408in}{3.081600in}%
\pgfsys@useobject{currentmarker}{}%
\end{pgfscope}%
\end{pgfscope}%
\begin{pgfscope}%
\definecolor{textcolor}{rgb}{0.000000,0.000000,0.000000}%
\pgfsetstrokecolor{textcolor}%
\pgfsetfillcolor{textcolor}%
\pgftext[x=1.080408in,y=2.984378in,,top]{\color{textcolor}\sffamily\fontsize{20.000000}{24.000000}\selectfont \(\displaystyle {0}\)}%
\end{pgfscope}%
\begin{pgfscope}%
\pgfsetbuttcap%
\pgfsetroundjoin%
\definecolor{currentfill}{rgb}{0.000000,0.000000,0.000000}%
\pgfsetfillcolor{currentfill}%
\pgfsetlinewidth{0.803000pt}%
\definecolor{currentstroke}{rgb}{0.000000,0.000000,0.000000}%
\pgfsetstrokecolor{currentstroke}%
\pgfsetdash{}{0pt}%
\pgfsys@defobject{currentmarker}{\pgfqpoint{0.000000in}{-0.048611in}}{\pgfqpoint{0.000000in}{0.000000in}}{%
\pgfpathmoveto{\pgfqpoint{0.000000in}{0.000000in}}%
\pgfpathlineto{\pgfqpoint{0.000000in}{-0.048611in}}%
\pgfusepath{stroke,fill}%
}%
\begin{pgfscope}%
\pgfsys@transformshift{1.609914in}{3.081600in}%
\pgfsys@useobject{currentmarker}{}%
\end{pgfscope}%
\end{pgfscope}%
\begin{pgfscope}%
\definecolor{textcolor}{rgb}{0.000000,0.000000,0.000000}%
\pgfsetstrokecolor{textcolor}%
\pgfsetfillcolor{textcolor}%
\pgftext[x=1.609914in,y=2.984378in,,top]{\color{textcolor}\sffamily\fontsize{20.000000}{24.000000}\selectfont \(\displaystyle {1}\)}%
\end{pgfscope}%
\begin{pgfscope}%
\pgfsetbuttcap%
\pgfsetroundjoin%
\definecolor{currentfill}{rgb}{0.000000,0.000000,0.000000}%
\pgfsetfillcolor{currentfill}%
\pgfsetlinewidth{0.803000pt}%
\definecolor{currentstroke}{rgb}{0.000000,0.000000,0.000000}%
\pgfsetstrokecolor{currentstroke}%
\pgfsetdash{}{0pt}%
\pgfsys@defobject{currentmarker}{\pgfqpoint{0.000000in}{-0.048611in}}{\pgfqpoint{0.000000in}{0.000000in}}{%
\pgfpathmoveto{\pgfqpoint{0.000000in}{0.000000in}}%
\pgfpathlineto{\pgfqpoint{0.000000in}{-0.048611in}}%
\pgfusepath{stroke,fill}%
}%
\begin{pgfscope}%
\pgfsys@transformshift{2.139419in}{3.081600in}%
\pgfsys@useobject{currentmarker}{}%
\end{pgfscope}%
\end{pgfscope}%
\begin{pgfscope}%
\definecolor{textcolor}{rgb}{0.000000,0.000000,0.000000}%
\pgfsetstrokecolor{textcolor}%
\pgfsetfillcolor{textcolor}%
\pgftext[x=2.139419in,y=2.984378in,,top]{\color{textcolor}\sffamily\fontsize{20.000000}{24.000000}\selectfont \(\displaystyle {2}\)}%
\end{pgfscope}%
\begin{pgfscope}%
\pgfsetbuttcap%
\pgfsetroundjoin%
\definecolor{currentfill}{rgb}{0.000000,0.000000,0.000000}%
\pgfsetfillcolor{currentfill}%
\pgfsetlinewidth{0.803000pt}%
\definecolor{currentstroke}{rgb}{0.000000,0.000000,0.000000}%
\pgfsetstrokecolor{currentstroke}%
\pgfsetdash{}{0pt}%
\pgfsys@defobject{currentmarker}{\pgfqpoint{0.000000in}{-0.048611in}}{\pgfqpoint{0.000000in}{0.000000in}}{%
\pgfpathmoveto{\pgfqpoint{0.000000in}{0.000000in}}%
\pgfpathlineto{\pgfqpoint{0.000000in}{-0.048611in}}%
\pgfusepath{stroke,fill}%
}%
\begin{pgfscope}%
\pgfsys@transformshift{2.668925in}{3.081600in}%
\pgfsys@useobject{currentmarker}{}%
\end{pgfscope}%
\end{pgfscope}%
\begin{pgfscope}%
\definecolor{textcolor}{rgb}{0.000000,0.000000,0.000000}%
\pgfsetstrokecolor{textcolor}%
\pgfsetfillcolor{textcolor}%
\pgftext[x=2.668925in,y=2.984378in,,top]{\color{textcolor}\sffamily\fontsize{20.000000}{24.000000}\selectfont \(\displaystyle {3}\)}%
\end{pgfscope}%
\begin{pgfscope}%
\definecolor{textcolor}{rgb}{0.000000,0.000000,0.000000}%
\pgfsetstrokecolor{textcolor}%
\pgfsetfillcolor{textcolor}%
\pgftext[x=1.874667in,y=2.672755in,,top]{\color{textcolor}\sffamily\fontsize{20.000000}{24.000000}\selectfont \(\displaystyle a\)}%
\end{pgfscope}%
\begin{pgfscope}%
\pgfsetbuttcap%
\pgfsetroundjoin%
\definecolor{currentfill}{rgb}{0.000000,0.000000,0.000000}%
\pgfsetfillcolor{currentfill}%
\pgfsetlinewidth{0.803000pt}%
\definecolor{currentstroke}{rgb}{0.000000,0.000000,0.000000}%
\pgfsetstrokecolor{currentstroke}%
\pgfsetdash{}{0pt}%
\pgfsys@defobject{currentmarker}{\pgfqpoint{-0.048611in}{0.000000in}}{\pgfqpoint{-0.000000in}{0.000000in}}{%
\pgfpathmoveto{\pgfqpoint{-0.000000in}{0.000000in}}%
\pgfpathlineto{\pgfqpoint{-0.048611in}{0.000000in}}%
\pgfusepath{stroke,fill}%
}%
\begin{pgfscope}%
\pgfsys@transformshift{0.768000in}{3.081600in}%
\pgfsys@useobject{currentmarker}{}%
\end{pgfscope}%
\end{pgfscope}%
\begin{pgfscope}%
\definecolor{textcolor}{rgb}{0.000000,0.000000,0.000000}%
\pgfsetstrokecolor{textcolor}%
\pgfsetfillcolor{textcolor}%
\pgftext[x=0.538670in, y=2.981581in, left, base]{\color{textcolor}\sffamily\fontsize{20.000000}{24.000000}\selectfont \(\displaystyle {0}\)}%
\end{pgfscope}%
\begin{pgfscope}%
\pgfsetbuttcap%
\pgfsetroundjoin%
\definecolor{currentfill}{rgb}{0.000000,0.000000,0.000000}%
\pgfsetfillcolor{currentfill}%
\pgfsetlinewidth{0.803000pt}%
\definecolor{currentstroke}{rgb}{0.000000,0.000000,0.000000}%
\pgfsetstrokecolor{currentstroke}%
\pgfsetdash{}{0pt}%
\pgfsys@defobject{currentmarker}{\pgfqpoint{-0.048611in}{0.000000in}}{\pgfqpoint{-0.000000in}{0.000000in}}{%
\pgfpathmoveto{\pgfqpoint{-0.000000in}{0.000000in}}%
\pgfpathlineto{\pgfqpoint{-0.048611in}{0.000000in}}%
\pgfusepath{stroke,fill}%
}%
\begin{pgfscope}%
\pgfsys@transformshift{0.768000in}{4.509627in}%
\pgfsys@useobject{currentmarker}{}%
\end{pgfscope}%
\end{pgfscope}%
\begin{pgfscope}%
\definecolor{textcolor}{rgb}{0.000000,0.000000,0.000000}%
\pgfsetstrokecolor{textcolor}%
\pgfsetfillcolor{textcolor}%
\pgftext[x=0.538670in, y=4.409608in, left, base]{\color{textcolor}\sffamily\fontsize{20.000000}{24.000000}\selectfont \(\displaystyle {1}\)}%
\end{pgfscope}%
\begin{pgfscope}%
\definecolor{textcolor}{rgb}{0.000000,0.000000,0.000000}%
\pgfsetstrokecolor{textcolor}%
\pgfsetfillcolor{textcolor}%
\pgftext[x=0.483115in,y=3.868800in,,bottom,rotate=90.000000]{\color{textcolor}\sffamily\fontsize{20.000000}{24.000000}\selectfont \(\displaystyle \mathrm{weight}\ a\)}%
\end{pgfscope}%
\begin{pgfscope}%
\pgfsetrectcap%
\pgfsetmiterjoin%
\pgfsetlinewidth{0.803000pt}%
\definecolor{currentstroke}{rgb}{0.000000,0.000000,0.000000}%
\pgfsetstrokecolor{currentstroke}%
\pgfsetdash{}{0pt}%
\pgfpathmoveto{\pgfqpoint{0.768000in}{3.081600in}}%
\pgfpathlineto{\pgfqpoint{0.768000in}{4.656000in}}%
\pgfusepath{stroke}%
\end{pgfscope}%
\begin{pgfscope}%
\pgfsetrectcap%
\pgfsetmiterjoin%
\pgfsetlinewidth{0.803000pt}%
\definecolor{currentstroke}{rgb}{0.000000,0.000000,0.000000}%
\pgfsetstrokecolor{currentstroke}%
\pgfsetdash{}{0pt}%
\pgfpathmoveto{\pgfqpoint{2.981333in}{3.081600in}}%
\pgfpathlineto{\pgfqpoint{2.981333in}{4.656000in}}%
\pgfusepath{stroke}%
\end{pgfscope}%
\begin{pgfscope}%
\pgfsetrectcap%
\pgfsetmiterjoin%
\pgfsetlinewidth{0.803000pt}%
\definecolor{currentstroke}{rgb}{0.000000,0.000000,0.000000}%
\pgfsetstrokecolor{currentstroke}%
\pgfsetdash{}{0pt}%
\pgfpathmoveto{\pgfqpoint{0.768000in}{3.081600in}}%
\pgfpathlineto{\pgfqpoint{2.981333in}{3.081600in}}%
\pgfusepath{stroke}%
\end{pgfscope}%
\begin{pgfscope}%
\pgfsetrectcap%
\pgfsetmiterjoin%
\pgfsetlinewidth{0.803000pt}%
\definecolor{currentstroke}{rgb}{0.000000,0.000000,0.000000}%
\pgfsetstrokecolor{currentstroke}%
\pgfsetdash{}{0pt}%
\pgfpathmoveto{\pgfqpoint{0.768000in}{4.656000in}}%
\pgfpathlineto{\pgfqpoint{2.981333in}{4.656000in}}%
\pgfusepath{stroke}%
\end{pgfscope}%
\begin{pgfscope}%
\pgfsetbuttcap%
\pgfsetmiterjoin%
\definecolor{currentfill}{rgb}{1.000000,1.000000,1.000000}%
\pgfsetfillcolor{currentfill}%
\pgfsetlinewidth{0.000000pt}%
\definecolor{currentstroke}{rgb}{0.000000,0.000000,0.000000}%
\pgfsetstrokecolor{currentstroke}%
\pgfsetstrokeopacity{0.000000}%
\pgfsetdash{}{0pt}%
\pgfpathmoveto{\pgfqpoint{3.866667in}{3.081600in}}%
\pgfpathlineto{\pgfqpoint{6.080000in}{3.081600in}}%
\pgfpathlineto{\pgfqpoint{6.080000in}{4.656000in}}%
\pgfpathlineto{\pgfqpoint{3.866667in}{4.656000in}}%
\pgfpathclose%
\pgfusepath{fill}%
\end{pgfscope}%
\begin{pgfscope}%
\pgfpathrectangle{\pgfqpoint{3.866667in}{3.081600in}}{\pgfqpoint{2.213333in}{1.574400in}}%
\pgfusepath{clip}%
\pgfsetbuttcap%
\pgfsetmiterjoin%
\definecolor{currentfill}{rgb}{0.000000,0.000000,1.000000}%
\pgfsetfillcolor{currentfill}%
\pgfsetlinewidth{0.000000pt}%
\definecolor{currentstroke}{rgb}{0.000000,0.000000,0.000000}%
\pgfsetstrokecolor{currentstroke}%
\pgfsetstrokeopacity{0.000000}%
\pgfsetdash{}{0pt}%
\pgfpathmoveto{\pgfqpoint{3.967273in}{3.081600in}}%
\pgfpathlineto{\pgfqpoint{4.390877in}{3.081600in}}%
\pgfpathlineto{\pgfqpoint{4.390877in}{3.795614in}}%
\pgfpathlineto{\pgfqpoint{3.967273in}{3.795614in}}%
\pgfpathclose%
\pgfusepath{fill}%
\end{pgfscope}%
\begin{pgfscope}%
\pgfpathrectangle{\pgfqpoint{3.866667in}{3.081600in}}{\pgfqpoint{2.213333in}{1.574400in}}%
\pgfusepath{clip}%
\pgfsetbuttcap%
\pgfsetmiterjoin%
\definecolor{currentfill}{rgb}{0.000000,0.000000,1.000000}%
\pgfsetfillcolor{currentfill}%
\pgfsetlinewidth{0.000000pt}%
\definecolor{currentstroke}{rgb}{0.000000,0.000000,0.000000}%
\pgfsetstrokecolor{currentstroke}%
\pgfsetstrokeopacity{0.000000}%
\pgfsetdash{}{0pt}%
\pgfpathmoveto{\pgfqpoint{4.496778in}{3.081600in}}%
\pgfpathlineto{\pgfqpoint{4.920383in}{3.081600in}}%
\pgfpathlineto{\pgfqpoint{4.920383in}{3.795614in}}%
\pgfpathlineto{\pgfqpoint{4.496778in}{3.795614in}}%
\pgfpathclose%
\pgfusepath{fill}%
\end{pgfscope}%
\begin{pgfscope}%
\pgfpathrectangle{\pgfqpoint{3.866667in}{3.081600in}}{\pgfqpoint{2.213333in}{1.574400in}}%
\pgfusepath{clip}%
\pgfsetbuttcap%
\pgfsetmiterjoin%
\definecolor{currentfill}{rgb}{0.000000,0.000000,1.000000}%
\pgfsetfillcolor{currentfill}%
\pgfsetlinewidth{0.000000pt}%
\definecolor{currentstroke}{rgb}{0.000000,0.000000,0.000000}%
\pgfsetstrokecolor{currentstroke}%
\pgfsetstrokeopacity{0.000000}%
\pgfsetdash{}{0pt}%
\pgfpathmoveto{\pgfqpoint{5.026284in}{3.081600in}}%
\pgfpathlineto{\pgfqpoint{5.449888in}{3.081600in}}%
\pgfpathlineto{\pgfqpoint{5.449888in}{4.509627in}}%
\pgfpathlineto{\pgfqpoint{5.026284in}{4.509627in}}%
\pgfpathclose%
\pgfusepath{fill}%
\end{pgfscope}%
\begin{pgfscope}%
\pgfpathrectangle{\pgfqpoint{3.866667in}{3.081600in}}{\pgfqpoint{2.213333in}{1.574400in}}%
\pgfusepath{clip}%
\pgfsetbuttcap%
\pgfsetmiterjoin%
\definecolor{currentfill}{rgb}{0.000000,0.000000,1.000000}%
\pgfsetfillcolor{currentfill}%
\pgfsetlinewidth{0.000000pt}%
\definecolor{currentstroke}{rgb}{0.000000,0.000000,0.000000}%
\pgfsetstrokecolor{currentstroke}%
\pgfsetstrokeopacity{0.000000}%
\pgfsetdash{}{0pt}%
\pgfpathmoveto{\pgfqpoint{5.555789in}{3.081600in}}%
\pgfpathlineto{\pgfqpoint{5.979394in}{3.081600in}}%
\pgfpathlineto{\pgfqpoint{5.979394in}{3.795614in}}%
\pgfpathlineto{\pgfqpoint{5.555789in}{3.795614in}}%
\pgfpathclose%
\pgfusepath{fill}%
\end{pgfscope}%
\begin{pgfscope}%
\pgfsetbuttcap%
\pgfsetroundjoin%
\definecolor{currentfill}{rgb}{0.000000,0.000000,0.000000}%
\pgfsetfillcolor{currentfill}%
\pgfsetlinewidth{0.803000pt}%
\definecolor{currentstroke}{rgb}{0.000000,0.000000,0.000000}%
\pgfsetstrokecolor{currentstroke}%
\pgfsetdash{}{0pt}%
\pgfsys@defobject{currentmarker}{\pgfqpoint{0.000000in}{-0.048611in}}{\pgfqpoint{0.000000in}{0.000000in}}{%
\pgfpathmoveto{\pgfqpoint{0.000000in}{0.000000in}}%
\pgfpathlineto{\pgfqpoint{0.000000in}{-0.048611in}}%
\pgfusepath{stroke,fill}%
}%
\begin{pgfscope}%
\pgfsys@transformshift{4.179075in}{3.081600in}%
\pgfsys@useobject{currentmarker}{}%
\end{pgfscope}%
\end{pgfscope}%
\begin{pgfscope}%
\definecolor{textcolor}{rgb}{0.000000,0.000000,0.000000}%
\pgfsetstrokecolor{textcolor}%
\pgfsetfillcolor{textcolor}%
\pgftext[x=4.179075in,y=2.984378in,,top]{\color{textcolor}\sffamily\fontsize{20.000000}{24.000000}\selectfont \(\displaystyle {0}\)}%
\end{pgfscope}%
\begin{pgfscope}%
\pgfsetbuttcap%
\pgfsetroundjoin%
\definecolor{currentfill}{rgb}{0.000000,0.000000,0.000000}%
\pgfsetfillcolor{currentfill}%
\pgfsetlinewidth{0.803000pt}%
\definecolor{currentstroke}{rgb}{0.000000,0.000000,0.000000}%
\pgfsetstrokecolor{currentstroke}%
\pgfsetdash{}{0pt}%
\pgfsys@defobject{currentmarker}{\pgfqpoint{0.000000in}{-0.048611in}}{\pgfqpoint{0.000000in}{0.000000in}}{%
\pgfpathmoveto{\pgfqpoint{0.000000in}{0.000000in}}%
\pgfpathlineto{\pgfqpoint{0.000000in}{-0.048611in}}%
\pgfusepath{stroke,fill}%
}%
\begin{pgfscope}%
\pgfsys@transformshift{4.708581in}{3.081600in}%
\pgfsys@useobject{currentmarker}{}%
\end{pgfscope}%
\end{pgfscope}%
\begin{pgfscope}%
\definecolor{textcolor}{rgb}{0.000000,0.000000,0.000000}%
\pgfsetstrokecolor{textcolor}%
\pgfsetfillcolor{textcolor}%
\pgftext[x=4.708581in,y=2.984378in,,top]{\color{textcolor}\sffamily\fontsize{20.000000}{24.000000}\selectfont \(\displaystyle {1}\)}%
\end{pgfscope}%
\begin{pgfscope}%
\pgfsetbuttcap%
\pgfsetroundjoin%
\definecolor{currentfill}{rgb}{0.000000,0.000000,0.000000}%
\pgfsetfillcolor{currentfill}%
\pgfsetlinewidth{0.803000pt}%
\definecolor{currentstroke}{rgb}{0.000000,0.000000,0.000000}%
\pgfsetstrokecolor{currentstroke}%
\pgfsetdash{}{0pt}%
\pgfsys@defobject{currentmarker}{\pgfqpoint{0.000000in}{-0.048611in}}{\pgfqpoint{0.000000in}{0.000000in}}{%
\pgfpathmoveto{\pgfqpoint{0.000000in}{0.000000in}}%
\pgfpathlineto{\pgfqpoint{0.000000in}{-0.048611in}}%
\pgfusepath{stroke,fill}%
}%
\begin{pgfscope}%
\pgfsys@transformshift{5.238086in}{3.081600in}%
\pgfsys@useobject{currentmarker}{}%
\end{pgfscope}%
\end{pgfscope}%
\begin{pgfscope}%
\definecolor{textcolor}{rgb}{0.000000,0.000000,0.000000}%
\pgfsetstrokecolor{textcolor}%
\pgfsetfillcolor{textcolor}%
\pgftext[x=5.238086in,y=2.984378in,,top]{\color{textcolor}\sffamily\fontsize{20.000000}{24.000000}\selectfont \(\displaystyle {2}\)}%
\end{pgfscope}%
\begin{pgfscope}%
\pgfsetbuttcap%
\pgfsetroundjoin%
\definecolor{currentfill}{rgb}{0.000000,0.000000,0.000000}%
\pgfsetfillcolor{currentfill}%
\pgfsetlinewidth{0.803000pt}%
\definecolor{currentstroke}{rgb}{0.000000,0.000000,0.000000}%
\pgfsetstrokecolor{currentstroke}%
\pgfsetdash{}{0pt}%
\pgfsys@defobject{currentmarker}{\pgfqpoint{0.000000in}{-0.048611in}}{\pgfqpoint{0.000000in}{0.000000in}}{%
\pgfpathmoveto{\pgfqpoint{0.000000in}{0.000000in}}%
\pgfpathlineto{\pgfqpoint{0.000000in}{-0.048611in}}%
\pgfusepath{stroke,fill}%
}%
\begin{pgfscope}%
\pgfsys@transformshift{5.767592in}{3.081600in}%
\pgfsys@useobject{currentmarker}{}%
\end{pgfscope}%
\end{pgfscope}%
\begin{pgfscope}%
\definecolor{textcolor}{rgb}{0.000000,0.000000,0.000000}%
\pgfsetstrokecolor{textcolor}%
\pgfsetfillcolor{textcolor}%
\pgftext[x=5.767592in,y=2.984378in,,top]{\color{textcolor}\sffamily\fontsize{20.000000}{24.000000}\selectfont \(\displaystyle {3}\)}%
\end{pgfscope}%
\begin{pgfscope}%
\definecolor{textcolor}{rgb}{0.000000,0.000000,0.000000}%
\pgfsetstrokecolor{textcolor}%
\pgfsetfillcolor{textcolor}%
\pgftext[x=4.973333in,y=2.672755in,,top]{\color{textcolor}\sffamily\fontsize{20.000000}{24.000000}\selectfont \(\displaystyle b_{1}\)}%
\end{pgfscope}%
\begin{pgfscope}%
\pgfsetbuttcap%
\pgfsetroundjoin%
\definecolor{currentfill}{rgb}{0.000000,0.000000,0.000000}%
\pgfsetfillcolor{currentfill}%
\pgfsetlinewidth{0.803000pt}%
\definecolor{currentstroke}{rgb}{0.000000,0.000000,0.000000}%
\pgfsetstrokecolor{currentstroke}%
\pgfsetdash{}{0pt}%
\pgfsys@defobject{currentmarker}{\pgfqpoint{-0.048611in}{0.000000in}}{\pgfqpoint{-0.000000in}{0.000000in}}{%
\pgfpathmoveto{\pgfqpoint{-0.000000in}{0.000000in}}%
\pgfpathlineto{\pgfqpoint{-0.048611in}{0.000000in}}%
\pgfusepath{stroke,fill}%
}%
\begin{pgfscope}%
\pgfsys@transformshift{3.866667in}{3.081600in}%
\pgfsys@useobject{currentmarker}{}%
\end{pgfscope}%
\end{pgfscope}%
\begin{pgfscope}%
\definecolor{textcolor}{rgb}{0.000000,0.000000,0.000000}%
\pgfsetstrokecolor{textcolor}%
\pgfsetfillcolor{textcolor}%
\pgftext[x=3.637337in, y=2.981581in, left, base]{\color{textcolor}\sffamily\fontsize{20.000000}{24.000000}\selectfont \(\displaystyle {0}\)}%
\end{pgfscope}%
\begin{pgfscope}%
\pgfsetbuttcap%
\pgfsetroundjoin%
\definecolor{currentfill}{rgb}{0.000000,0.000000,0.000000}%
\pgfsetfillcolor{currentfill}%
\pgfsetlinewidth{0.803000pt}%
\definecolor{currentstroke}{rgb}{0.000000,0.000000,0.000000}%
\pgfsetstrokecolor{currentstroke}%
\pgfsetdash{}{0pt}%
\pgfsys@defobject{currentmarker}{\pgfqpoint{-0.048611in}{0.000000in}}{\pgfqpoint{-0.000000in}{0.000000in}}{%
\pgfpathmoveto{\pgfqpoint{-0.000000in}{0.000000in}}%
\pgfpathlineto{\pgfqpoint{-0.048611in}{0.000000in}}%
\pgfusepath{stroke,fill}%
}%
\begin{pgfscope}%
\pgfsys@transformshift{3.866667in}{4.509627in}%
\pgfsys@useobject{currentmarker}{}%
\end{pgfscope}%
\end{pgfscope}%
\begin{pgfscope}%
\definecolor{textcolor}{rgb}{0.000000,0.000000,0.000000}%
\pgfsetstrokecolor{textcolor}%
\pgfsetfillcolor{textcolor}%
\pgftext[x=3.637337in, y=4.409608in, left, base]{\color{textcolor}\sffamily\fontsize{20.000000}{24.000000}\selectfont \(\displaystyle {1}\)}%
\end{pgfscope}%
\begin{pgfscope}%
\definecolor{textcolor}{rgb}{0.000000,0.000000,0.000000}%
\pgfsetstrokecolor{textcolor}%
\pgfsetfillcolor{textcolor}%
\pgftext[x=3.581782in,y=3.868800in,,bottom,rotate=90.000000]{\color{textcolor}\sffamily\fontsize{20.000000}{24.000000}\selectfont \(\displaystyle \mathrm{weight}\ b_{1}\)}%
\end{pgfscope}%
\begin{pgfscope}%
\pgfsetrectcap%
\pgfsetmiterjoin%
\pgfsetlinewidth{0.803000pt}%
\definecolor{currentstroke}{rgb}{0.000000,0.000000,0.000000}%
\pgfsetstrokecolor{currentstroke}%
\pgfsetdash{}{0pt}%
\pgfpathmoveto{\pgfqpoint{3.866667in}{3.081600in}}%
\pgfpathlineto{\pgfqpoint{3.866667in}{4.656000in}}%
\pgfusepath{stroke}%
\end{pgfscope}%
\begin{pgfscope}%
\pgfsetrectcap%
\pgfsetmiterjoin%
\pgfsetlinewidth{0.803000pt}%
\definecolor{currentstroke}{rgb}{0.000000,0.000000,0.000000}%
\pgfsetstrokecolor{currentstroke}%
\pgfsetdash{}{0pt}%
\pgfpathmoveto{\pgfqpoint{6.080000in}{3.081600in}}%
\pgfpathlineto{\pgfqpoint{6.080000in}{4.656000in}}%
\pgfusepath{stroke}%
\end{pgfscope}%
\begin{pgfscope}%
\pgfsetrectcap%
\pgfsetmiterjoin%
\pgfsetlinewidth{0.803000pt}%
\definecolor{currentstroke}{rgb}{0.000000,0.000000,0.000000}%
\pgfsetstrokecolor{currentstroke}%
\pgfsetdash{}{0pt}%
\pgfpathmoveto{\pgfqpoint{3.866667in}{3.081600in}}%
\pgfpathlineto{\pgfqpoint{6.080000in}{3.081600in}}%
\pgfusepath{stroke}%
\end{pgfscope}%
\begin{pgfscope}%
\pgfsetrectcap%
\pgfsetmiterjoin%
\pgfsetlinewidth{0.803000pt}%
\definecolor{currentstroke}{rgb}{0.000000,0.000000,0.000000}%
\pgfsetstrokecolor{currentstroke}%
\pgfsetdash{}{0pt}%
\pgfpathmoveto{\pgfqpoint{3.866667in}{4.656000in}}%
\pgfpathlineto{\pgfqpoint{6.080000in}{4.656000in}}%
\pgfusepath{stroke}%
\end{pgfscope}%
\begin{pgfscope}%
\pgfsetbuttcap%
\pgfsetmiterjoin%
\definecolor{currentfill}{rgb}{1.000000,1.000000,1.000000}%
\pgfsetfillcolor{currentfill}%
\pgfsetlinewidth{0.000000pt}%
\definecolor{currentstroke}{rgb}{0.000000,0.000000,0.000000}%
\pgfsetstrokecolor{currentstroke}%
\pgfsetstrokeopacity{0.000000}%
\pgfsetdash{}{0pt}%
\pgfpathmoveto{\pgfqpoint{0.768000in}{0.720000in}}%
\pgfpathlineto{\pgfqpoint{2.981333in}{0.720000in}}%
\pgfpathlineto{\pgfqpoint{2.981333in}{2.294400in}}%
\pgfpathlineto{\pgfqpoint{0.768000in}{2.294400in}}%
\pgfpathclose%
\pgfusepath{fill}%
\end{pgfscope}%
\begin{pgfscope}%
\pgfpathrectangle{\pgfqpoint{0.768000in}{0.720000in}}{\pgfqpoint{2.213333in}{1.574400in}}%
\pgfusepath{clip}%
\pgfsetbuttcap%
\pgfsetmiterjoin%
\definecolor{currentfill}{rgb}{0.000000,0.000000,1.000000}%
\pgfsetfillcolor{currentfill}%
\pgfsetlinewidth{0.000000pt}%
\definecolor{currentstroke}{rgb}{0.000000,0.000000,0.000000}%
\pgfsetstrokecolor{currentstroke}%
\pgfsetstrokeopacity{0.000000}%
\pgfsetdash{}{0pt}%
\pgfpathmoveto{\pgfqpoint{0.868606in}{0.720000in}}%
\pgfpathlineto{\pgfqpoint{1.292211in}{0.720000in}}%
\pgfpathlineto{\pgfqpoint{1.292211in}{2.148027in}}%
\pgfpathlineto{\pgfqpoint{0.868606in}{2.148027in}}%
\pgfpathclose%
\pgfusepath{fill}%
\end{pgfscope}%
\begin{pgfscope}%
\pgfpathrectangle{\pgfqpoint{0.768000in}{0.720000in}}{\pgfqpoint{2.213333in}{1.574400in}}%
\pgfusepath{clip}%
\pgfsetbuttcap%
\pgfsetmiterjoin%
\definecolor{currentfill}{rgb}{0.000000,0.000000,1.000000}%
\pgfsetfillcolor{currentfill}%
\pgfsetlinewidth{0.000000pt}%
\definecolor{currentstroke}{rgb}{0.000000,0.000000,0.000000}%
\pgfsetstrokecolor{currentstroke}%
\pgfsetstrokeopacity{0.000000}%
\pgfsetdash{}{0pt}%
\pgfpathmoveto{\pgfqpoint{1.398112in}{0.720000in}}%
\pgfpathlineto{\pgfqpoint{1.821716in}{0.720000in}}%
\pgfpathlineto{\pgfqpoint{1.821716in}{1.434014in}}%
\pgfpathlineto{\pgfqpoint{1.398112in}{1.434014in}}%
\pgfpathclose%
\pgfusepath{fill}%
\end{pgfscope}%
\begin{pgfscope}%
\pgfpathrectangle{\pgfqpoint{0.768000in}{0.720000in}}{\pgfqpoint{2.213333in}{1.574400in}}%
\pgfusepath{clip}%
\pgfsetbuttcap%
\pgfsetmiterjoin%
\definecolor{currentfill}{rgb}{0.000000,0.000000,1.000000}%
\pgfsetfillcolor{currentfill}%
\pgfsetlinewidth{0.000000pt}%
\definecolor{currentstroke}{rgb}{0.000000,0.000000,0.000000}%
\pgfsetstrokecolor{currentstroke}%
\pgfsetstrokeopacity{0.000000}%
\pgfsetdash{}{0pt}%
\pgfpathmoveto{\pgfqpoint{1.927617in}{0.720000in}}%
\pgfpathlineto{\pgfqpoint{2.351222in}{0.720000in}}%
\pgfpathlineto{\pgfqpoint{2.351222in}{1.434014in}}%
\pgfpathlineto{\pgfqpoint{1.927617in}{1.434014in}}%
\pgfpathclose%
\pgfusepath{fill}%
\end{pgfscope}%
\begin{pgfscope}%
\pgfpathrectangle{\pgfqpoint{0.768000in}{0.720000in}}{\pgfqpoint{2.213333in}{1.574400in}}%
\pgfusepath{clip}%
\pgfsetbuttcap%
\pgfsetmiterjoin%
\definecolor{currentfill}{rgb}{0.000000,0.000000,1.000000}%
\pgfsetfillcolor{currentfill}%
\pgfsetlinewidth{0.000000pt}%
\definecolor{currentstroke}{rgb}{0.000000,0.000000,0.000000}%
\pgfsetstrokecolor{currentstroke}%
\pgfsetstrokeopacity{0.000000}%
\pgfsetdash{}{0pt}%
\pgfpathmoveto{\pgfqpoint{2.457123in}{0.720000in}}%
\pgfpathlineto{\pgfqpoint{2.880727in}{0.720000in}}%
\pgfpathlineto{\pgfqpoint{2.880727in}{1.434014in}}%
\pgfpathlineto{\pgfqpoint{2.457123in}{1.434014in}}%
\pgfpathclose%
\pgfusepath{fill}%
\end{pgfscope}%
\begin{pgfscope}%
\pgfsetbuttcap%
\pgfsetroundjoin%
\definecolor{currentfill}{rgb}{0.000000,0.000000,0.000000}%
\pgfsetfillcolor{currentfill}%
\pgfsetlinewidth{0.803000pt}%
\definecolor{currentstroke}{rgb}{0.000000,0.000000,0.000000}%
\pgfsetstrokecolor{currentstroke}%
\pgfsetdash{}{0pt}%
\pgfsys@defobject{currentmarker}{\pgfqpoint{0.000000in}{-0.048611in}}{\pgfqpoint{0.000000in}{0.000000in}}{%
\pgfpathmoveto{\pgfqpoint{0.000000in}{0.000000in}}%
\pgfpathlineto{\pgfqpoint{0.000000in}{-0.048611in}}%
\pgfusepath{stroke,fill}%
}%
\begin{pgfscope}%
\pgfsys@transformshift{1.080408in}{0.720000in}%
\pgfsys@useobject{currentmarker}{}%
\end{pgfscope}%
\end{pgfscope}%
\begin{pgfscope}%
\definecolor{textcolor}{rgb}{0.000000,0.000000,0.000000}%
\pgfsetstrokecolor{textcolor}%
\pgfsetfillcolor{textcolor}%
\pgftext[x=1.080408in,y=0.622778in,,top]{\color{textcolor}\sffamily\fontsize{20.000000}{24.000000}\selectfont \(\displaystyle {0}\)}%
\end{pgfscope}%
\begin{pgfscope}%
\pgfsetbuttcap%
\pgfsetroundjoin%
\definecolor{currentfill}{rgb}{0.000000,0.000000,0.000000}%
\pgfsetfillcolor{currentfill}%
\pgfsetlinewidth{0.803000pt}%
\definecolor{currentstroke}{rgb}{0.000000,0.000000,0.000000}%
\pgfsetstrokecolor{currentstroke}%
\pgfsetdash{}{0pt}%
\pgfsys@defobject{currentmarker}{\pgfqpoint{0.000000in}{-0.048611in}}{\pgfqpoint{0.000000in}{0.000000in}}{%
\pgfpathmoveto{\pgfqpoint{0.000000in}{0.000000in}}%
\pgfpathlineto{\pgfqpoint{0.000000in}{-0.048611in}}%
\pgfusepath{stroke,fill}%
}%
\begin{pgfscope}%
\pgfsys@transformshift{1.609914in}{0.720000in}%
\pgfsys@useobject{currentmarker}{}%
\end{pgfscope}%
\end{pgfscope}%
\begin{pgfscope}%
\definecolor{textcolor}{rgb}{0.000000,0.000000,0.000000}%
\pgfsetstrokecolor{textcolor}%
\pgfsetfillcolor{textcolor}%
\pgftext[x=1.609914in,y=0.622778in,,top]{\color{textcolor}\sffamily\fontsize{20.000000}{24.000000}\selectfont \(\displaystyle {1}\)}%
\end{pgfscope}%
\begin{pgfscope}%
\pgfsetbuttcap%
\pgfsetroundjoin%
\definecolor{currentfill}{rgb}{0.000000,0.000000,0.000000}%
\pgfsetfillcolor{currentfill}%
\pgfsetlinewidth{0.803000pt}%
\definecolor{currentstroke}{rgb}{0.000000,0.000000,0.000000}%
\pgfsetstrokecolor{currentstroke}%
\pgfsetdash{}{0pt}%
\pgfsys@defobject{currentmarker}{\pgfqpoint{0.000000in}{-0.048611in}}{\pgfqpoint{0.000000in}{0.000000in}}{%
\pgfpathmoveto{\pgfqpoint{0.000000in}{0.000000in}}%
\pgfpathlineto{\pgfqpoint{0.000000in}{-0.048611in}}%
\pgfusepath{stroke,fill}%
}%
\begin{pgfscope}%
\pgfsys@transformshift{2.139419in}{0.720000in}%
\pgfsys@useobject{currentmarker}{}%
\end{pgfscope}%
\end{pgfscope}%
\begin{pgfscope}%
\definecolor{textcolor}{rgb}{0.000000,0.000000,0.000000}%
\pgfsetstrokecolor{textcolor}%
\pgfsetfillcolor{textcolor}%
\pgftext[x=2.139419in,y=0.622778in,,top]{\color{textcolor}\sffamily\fontsize{20.000000}{24.000000}\selectfont \(\displaystyle {2}\)}%
\end{pgfscope}%
\begin{pgfscope}%
\pgfsetbuttcap%
\pgfsetroundjoin%
\definecolor{currentfill}{rgb}{0.000000,0.000000,0.000000}%
\pgfsetfillcolor{currentfill}%
\pgfsetlinewidth{0.803000pt}%
\definecolor{currentstroke}{rgb}{0.000000,0.000000,0.000000}%
\pgfsetstrokecolor{currentstroke}%
\pgfsetdash{}{0pt}%
\pgfsys@defobject{currentmarker}{\pgfqpoint{0.000000in}{-0.048611in}}{\pgfqpoint{0.000000in}{0.000000in}}{%
\pgfpathmoveto{\pgfqpoint{0.000000in}{0.000000in}}%
\pgfpathlineto{\pgfqpoint{0.000000in}{-0.048611in}}%
\pgfusepath{stroke,fill}%
}%
\begin{pgfscope}%
\pgfsys@transformshift{2.668925in}{0.720000in}%
\pgfsys@useobject{currentmarker}{}%
\end{pgfscope}%
\end{pgfscope}%
\begin{pgfscope}%
\definecolor{textcolor}{rgb}{0.000000,0.000000,0.000000}%
\pgfsetstrokecolor{textcolor}%
\pgfsetfillcolor{textcolor}%
\pgftext[x=2.668925in,y=0.622778in,,top]{\color{textcolor}\sffamily\fontsize{20.000000}{24.000000}\selectfont \(\displaystyle {3}\)}%
\end{pgfscope}%
\begin{pgfscope}%
\definecolor{textcolor}{rgb}{0.000000,0.000000,0.000000}%
\pgfsetstrokecolor{textcolor}%
\pgfsetfillcolor{textcolor}%
\pgftext[x=1.874667in,y=0.311155in,,top]{\color{textcolor}\sffamily\fontsize{20.000000}{24.000000}\selectfont \(\displaystyle b_{2}\)}%
\end{pgfscope}%
\begin{pgfscope}%
\pgfsetbuttcap%
\pgfsetroundjoin%
\definecolor{currentfill}{rgb}{0.000000,0.000000,0.000000}%
\pgfsetfillcolor{currentfill}%
\pgfsetlinewidth{0.803000pt}%
\definecolor{currentstroke}{rgb}{0.000000,0.000000,0.000000}%
\pgfsetstrokecolor{currentstroke}%
\pgfsetdash{}{0pt}%
\pgfsys@defobject{currentmarker}{\pgfqpoint{-0.048611in}{0.000000in}}{\pgfqpoint{-0.000000in}{0.000000in}}{%
\pgfpathmoveto{\pgfqpoint{-0.000000in}{0.000000in}}%
\pgfpathlineto{\pgfqpoint{-0.048611in}{0.000000in}}%
\pgfusepath{stroke,fill}%
}%
\begin{pgfscope}%
\pgfsys@transformshift{0.768000in}{0.720000in}%
\pgfsys@useobject{currentmarker}{}%
\end{pgfscope}%
\end{pgfscope}%
\begin{pgfscope}%
\definecolor{textcolor}{rgb}{0.000000,0.000000,0.000000}%
\pgfsetstrokecolor{textcolor}%
\pgfsetfillcolor{textcolor}%
\pgftext[x=0.538670in, y=0.619981in, left, base]{\color{textcolor}\sffamily\fontsize{20.000000}{24.000000}\selectfont \(\displaystyle {0}\)}%
\end{pgfscope}%
\begin{pgfscope}%
\pgfsetbuttcap%
\pgfsetroundjoin%
\definecolor{currentfill}{rgb}{0.000000,0.000000,0.000000}%
\pgfsetfillcolor{currentfill}%
\pgfsetlinewidth{0.803000pt}%
\definecolor{currentstroke}{rgb}{0.000000,0.000000,0.000000}%
\pgfsetstrokecolor{currentstroke}%
\pgfsetdash{}{0pt}%
\pgfsys@defobject{currentmarker}{\pgfqpoint{-0.048611in}{0.000000in}}{\pgfqpoint{-0.000000in}{0.000000in}}{%
\pgfpathmoveto{\pgfqpoint{-0.000000in}{0.000000in}}%
\pgfpathlineto{\pgfqpoint{-0.048611in}{0.000000in}}%
\pgfusepath{stroke,fill}%
}%
\begin{pgfscope}%
\pgfsys@transformshift{0.768000in}{2.148027in}%
\pgfsys@useobject{currentmarker}{}%
\end{pgfscope}%
\end{pgfscope}%
\begin{pgfscope}%
\definecolor{textcolor}{rgb}{0.000000,0.000000,0.000000}%
\pgfsetstrokecolor{textcolor}%
\pgfsetfillcolor{textcolor}%
\pgftext[x=0.538670in, y=2.048008in, left, base]{\color{textcolor}\sffamily\fontsize{20.000000}{24.000000}\selectfont \(\displaystyle {1}\)}%
\end{pgfscope}%
\begin{pgfscope}%
\definecolor{textcolor}{rgb}{0.000000,0.000000,0.000000}%
\pgfsetstrokecolor{textcolor}%
\pgfsetfillcolor{textcolor}%
\pgftext[x=0.483115in,y=1.507200in,,bottom,rotate=90.000000]{\color{textcolor}\sffamily\fontsize{20.000000}{24.000000}\selectfont \(\displaystyle \mathrm{weight}\ b_{2}\)}%
\end{pgfscope}%
\begin{pgfscope}%
\pgfsetrectcap%
\pgfsetmiterjoin%
\pgfsetlinewidth{0.803000pt}%
\definecolor{currentstroke}{rgb}{0.000000,0.000000,0.000000}%
\pgfsetstrokecolor{currentstroke}%
\pgfsetdash{}{0pt}%
\pgfpathmoveto{\pgfqpoint{0.768000in}{0.720000in}}%
\pgfpathlineto{\pgfqpoint{0.768000in}{2.294400in}}%
\pgfusepath{stroke}%
\end{pgfscope}%
\begin{pgfscope}%
\pgfsetrectcap%
\pgfsetmiterjoin%
\pgfsetlinewidth{0.803000pt}%
\definecolor{currentstroke}{rgb}{0.000000,0.000000,0.000000}%
\pgfsetstrokecolor{currentstroke}%
\pgfsetdash{}{0pt}%
\pgfpathmoveto{\pgfqpoint{2.981333in}{0.720000in}}%
\pgfpathlineto{\pgfqpoint{2.981333in}{2.294400in}}%
\pgfusepath{stroke}%
\end{pgfscope}%
\begin{pgfscope}%
\pgfsetrectcap%
\pgfsetmiterjoin%
\pgfsetlinewidth{0.803000pt}%
\definecolor{currentstroke}{rgb}{0.000000,0.000000,0.000000}%
\pgfsetstrokecolor{currentstroke}%
\pgfsetdash{}{0pt}%
\pgfpathmoveto{\pgfqpoint{0.768000in}{0.720000in}}%
\pgfpathlineto{\pgfqpoint{2.981333in}{0.720000in}}%
\pgfusepath{stroke}%
\end{pgfscope}%
\begin{pgfscope}%
\pgfsetrectcap%
\pgfsetmiterjoin%
\pgfsetlinewidth{0.803000pt}%
\definecolor{currentstroke}{rgb}{0.000000,0.000000,0.000000}%
\pgfsetstrokecolor{currentstroke}%
\pgfsetdash{}{0pt}%
\pgfpathmoveto{\pgfqpoint{0.768000in}{2.294400in}}%
\pgfpathlineto{\pgfqpoint{2.981333in}{2.294400in}}%
\pgfusepath{stroke}%
\end{pgfscope}%
\begin{pgfscope}%
\pgfsetbuttcap%
\pgfsetmiterjoin%
\definecolor{currentfill}{rgb}{1.000000,1.000000,1.000000}%
\pgfsetfillcolor{currentfill}%
\pgfsetlinewidth{0.000000pt}%
\definecolor{currentstroke}{rgb}{0.000000,0.000000,0.000000}%
\pgfsetstrokecolor{currentstroke}%
\pgfsetstrokeopacity{0.000000}%
\pgfsetdash{}{0pt}%
\pgfpathmoveto{\pgfqpoint{3.866667in}{0.720000in}}%
\pgfpathlineto{\pgfqpoint{6.080000in}{0.720000in}}%
\pgfpathlineto{\pgfqpoint{6.080000in}{2.294400in}}%
\pgfpathlineto{\pgfqpoint{3.866667in}{2.294400in}}%
\pgfpathclose%
\pgfusepath{fill}%
\end{pgfscope}%
\begin{pgfscope}%
\definecolor{textcolor}{rgb}{0.000000,0.000000,0.000000}%
\pgfsetstrokecolor{textcolor}%
\pgfsetfillcolor{textcolor}%
\pgftext[x=3.866667in,y=1.507200in,left,base]{\color{textcolor}\sffamily\fontsize{20.000000}{24.000000}\selectfont \(\displaystyle L_{2}\ =\ 0.25\)}%
\end{pgfscope}%
\end{pgfpicture}%
\makeatother%
\endgroup%
}
    \caption{\label{fig:l2} L2 distance demo}
\end{figure}
\end{minipage}

But these distances have certain shortcomings. First, they cannot compare discrete distribution with continuous distribution. Second, they are not sensitive to timing: as the graph shows, the L2 distances between the distribution of $a,b_{1}$ and $a,b_{2}$ is both 0.25, but the shape of the $b_{1}$ and $b_{2}$ distribution is very different (see figure~\ref{fig:l2}). 

\label{sub:Wasserstein distance}
\begin{minipage}{.45\textwidth}
\begin{equation}
    W_{p}(\mu,\nu):=\left(\inf_{\gamma\in\Phi(\mu,\nu)}\int_{\chi}d(x,y)^{p}\,\mathrm{d}\gamma(x,y)\right)^{1/p} \label{eq:w-dist-def}
\end{equation}
\end{minipage}
\begin{minipage}{.55\textwidth}
\begin{figure}[H]
    \centering
    \scalebox{0.4}{%% Creator: Matplotlib, PGF backend
%%
%% To include the figure in your LaTeX document, write
%%   \input{<filename>.pgf}
%%
%% Make sure the required packages are loaded in your preamble
%%   \usepackage{pgf}
%%
%% and, on pdftex
%%   \usepackage[utf8]{inputenc}\DeclareUnicodeCharacter{2212}{-}
%%
%% or, on luatex and xetex
%%   \usepackage{unicode-math}
%%
%% Figures using additional raster images can only be included by \input if
%% they are in the same directory as the main LaTeX file. For loading figures
%% from other directories you can use the `import` package
%%   \usepackage{import}
%%
%% and then include the figures with
%%   \import{<path to file>}{<filename>.pgf}
%%
%% Matplotlib used the following preamble
%%   \usepackage[detect-all,locale=DE]{siunitx}
%%
\begingroup%
\makeatletter%
\begin{pgfpicture}%
\pgfpathrectangle{\pgfpointorigin}{\pgfqpoint{5.000000in}{5.000000in}}%
\pgfusepath{use as bounding box, clip}%
\begin{pgfscope}%
\pgfsetbuttcap%
\pgfsetmiterjoin%
\definecolor{currentfill}{rgb}{1.000000,1.000000,1.000000}%
\pgfsetfillcolor{currentfill}%
\pgfsetlinewidth{0.000000pt}%
\definecolor{currentstroke}{rgb}{1.000000,1.000000,1.000000}%
\pgfsetstrokecolor{currentstroke}%
\pgfsetdash{}{0pt}%
\pgfpathmoveto{\pgfqpoint{0.000000in}{0.000000in}}%
\pgfpathlineto{\pgfqpoint{5.000000in}{0.000000in}}%
\pgfpathlineto{\pgfqpoint{5.000000in}{5.000000in}}%
\pgfpathlineto{\pgfqpoint{0.000000in}{5.000000in}}%
\pgfpathclose%
\pgfusepath{fill}%
\end{pgfscope}%
\begin{pgfscope}%
\pgfpathrectangle{\pgfqpoint{0.500000in}{4.250000in}}{\pgfqpoint{3.750000in}{0.600000in}}%
\pgfusepath{clip}%
\pgfsetbuttcap%
\pgfsetroundjoin%
\definecolor{currentfill}{rgb}{0.000000,0.000000,1.000000}%
\pgfsetfillcolor{currentfill}%
\pgfsetfillopacity{0.500000}%
\pgfsetlinewidth{1.003750pt}%
\definecolor{currentstroke}{rgb}{0.000000,0.000000,0.000000}%
\pgfsetstrokecolor{currentstroke}%
\pgfsetstrokeopacity{0.500000}%
\pgfsetdash{}{0pt}%
\pgfsys@defobject{currentmarker}{\pgfqpoint{0.500000in}{4.250000in}}{\pgfqpoint{4.250000in}{4.850000in}}{%
\pgfpathmoveto{\pgfqpoint{0.500000in}{4.250000in}}%
\pgfpathlineto{\pgfqpoint{0.500000in}{4.250130in}}%
\pgfpathlineto{\pgfqpoint{0.506260in}{4.250139in}}%
\pgfpathlineto{\pgfqpoint{0.512521in}{4.250149in}}%
\pgfpathlineto{\pgfqpoint{0.518781in}{4.250159in}}%
\pgfpathlineto{\pgfqpoint{0.525042in}{4.250170in}}%
\pgfpathlineto{\pgfqpoint{0.531302in}{4.250181in}}%
\pgfpathlineto{\pgfqpoint{0.537563in}{4.250194in}}%
\pgfpathlineto{\pgfqpoint{0.543823in}{4.250207in}}%
\pgfpathlineto{\pgfqpoint{0.550083in}{4.250220in}}%
\pgfpathlineto{\pgfqpoint{0.556344in}{4.250235in}}%
\pgfpathlineto{\pgfqpoint{0.562604in}{4.250251in}}%
\pgfpathlineto{\pgfqpoint{0.568865in}{4.250267in}}%
\pgfpathlineto{\pgfqpoint{0.575125in}{4.250285in}}%
\pgfpathlineto{\pgfqpoint{0.581386in}{4.250303in}}%
\pgfpathlineto{\pgfqpoint{0.587646in}{4.250323in}}%
\pgfpathlineto{\pgfqpoint{0.593907in}{4.250344in}}%
\pgfpathlineto{\pgfqpoint{0.600167in}{4.250366in}}%
\pgfpathlineto{\pgfqpoint{0.606427in}{4.250389in}}%
\pgfpathlineto{\pgfqpoint{0.612688in}{4.250414in}}%
\pgfpathlineto{\pgfqpoint{0.618948in}{4.250441in}}%
\pgfpathlineto{\pgfqpoint{0.625209in}{4.250468in}}%
\pgfpathlineto{\pgfqpoint{0.631469in}{4.250498in}}%
\pgfpathlineto{\pgfqpoint{0.637730in}{4.250529in}}%
\pgfpathlineto{\pgfqpoint{0.643990in}{4.250562in}}%
\pgfpathlineto{\pgfqpoint{0.650250in}{4.250597in}}%
\pgfpathlineto{\pgfqpoint{0.656511in}{4.250634in}}%
\pgfpathlineto{\pgfqpoint{0.662771in}{4.250673in}}%
\pgfpathlineto{\pgfqpoint{0.669032in}{4.250714in}}%
\pgfpathlineto{\pgfqpoint{0.675292in}{4.250758in}}%
\pgfpathlineto{\pgfqpoint{0.681553in}{4.250804in}}%
\pgfpathlineto{\pgfqpoint{0.687813in}{4.250852in}}%
\pgfpathlineto{\pgfqpoint{0.694073in}{4.250903in}}%
\pgfpathlineto{\pgfqpoint{0.700334in}{4.250957in}}%
\pgfpathlineto{\pgfqpoint{0.706594in}{4.251014in}}%
\pgfpathlineto{\pgfqpoint{0.712855in}{4.251074in}}%
\pgfpathlineto{\pgfqpoint{0.719115in}{4.251137in}}%
\pgfpathlineto{\pgfqpoint{0.725376in}{4.251204in}}%
\pgfpathlineto{\pgfqpoint{0.731636in}{4.251274in}}%
\pgfpathlineto{\pgfqpoint{0.737896in}{4.251348in}}%
\pgfpathlineto{\pgfqpoint{0.744157in}{4.251425in}}%
\pgfpathlineto{\pgfqpoint{0.750417in}{4.251507in}}%
\pgfpathlineto{\pgfqpoint{0.756678in}{4.251593in}}%
\pgfpathlineto{\pgfqpoint{0.762938in}{4.251683in}}%
\pgfpathlineto{\pgfqpoint{0.769199in}{4.251778in}}%
\pgfpathlineto{\pgfqpoint{0.775459in}{4.251878in}}%
\pgfpathlineto{\pgfqpoint{0.781720in}{4.251983in}}%
\pgfpathlineto{\pgfqpoint{0.787980in}{4.252094in}}%
\pgfpathlineto{\pgfqpoint{0.794240in}{4.252209in}}%
\pgfpathlineto{\pgfqpoint{0.800501in}{4.252331in}}%
\pgfpathlineto{\pgfqpoint{0.806761in}{4.252458in}}%
\pgfpathlineto{\pgfqpoint{0.813022in}{4.252592in}}%
\pgfpathlineto{\pgfqpoint{0.819282in}{4.252732in}}%
\pgfpathlineto{\pgfqpoint{0.825543in}{4.252879in}}%
\pgfpathlineto{\pgfqpoint{0.831803in}{4.253034in}}%
\pgfpathlineto{\pgfqpoint{0.838063in}{4.253195in}}%
\pgfpathlineto{\pgfqpoint{0.844324in}{4.253364in}}%
\pgfpathlineto{\pgfqpoint{0.850584in}{4.253541in}}%
\pgfpathlineto{\pgfqpoint{0.856845in}{4.253727in}}%
\pgfpathlineto{\pgfqpoint{0.863105in}{4.253921in}}%
\pgfpathlineto{\pgfqpoint{0.869366in}{4.254124in}}%
\pgfpathlineto{\pgfqpoint{0.875626in}{4.254336in}}%
\pgfpathlineto{\pgfqpoint{0.881886in}{4.254558in}}%
\pgfpathlineto{\pgfqpoint{0.888147in}{4.254790in}}%
\pgfpathlineto{\pgfqpoint{0.894407in}{4.255032in}}%
\pgfpathlineto{\pgfqpoint{0.900668in}{4.255285in}}%
\pgfpathlineto{\pgfqpoint{0.906928in}{4.255550in}}%
\pgfpathlineto{\pgfqpoint{0.913189in}{4.255826in}}%
\pgfpathlineto{\pgfqpoint{0.919449in}{4.256114in}}%
\pgfpathlineto{\pgfqpoint{0.925710in}{4.256414in}}%
\pgfpathlineto{\pgfqpoint{0.931970in}{4.256727in}}%
\pgfpathlineto{\pgfqpoint{0.938230in}{4.257054in}}%
\pgfpathlineto{\pgfqpoint{0.944491in}{4.257394in}}%
\pgfpathlineto{\pgfqpoint{0.950751in}{4.257749in}}%
\pgfpathlineto{\pgfqpoint{0.957012in}{4.258118in}}%
\pgfpathlineto{\pgfqpoint{0.963272in}{4.258503in}}%
\pgfpathlineto{\pgfqpoint{0.969533in}{4.258903in}}%
\pgfpathlineto{\pgfqpoint{0.975793in}{4.259320in}}%
\pgfpathlineto{\pgfqpoint{0.982053in}{4.259753in}}%
\pgfpathlineto{\pgfqpoint{0.988314in}{4.260204in}}%
\pgfpathlineto{\pgfqpoint{0.994574in}{4.260673in}}%
\pgfpathlineto{\pgfqpoint{1.000835in}{4.261160in}}%
\pgfpathlineto{\pgfqpoint{1.007095in}{4.261666in}}%
\pgfpathlineto{\pgfqpoint{1.013356in}{4.262191in}}%
\pgfpathlineto{\pgfqpoint{1.019616in}{4.262737in}}%
\pgfpathlineto{\pgfqpoint{1.025876in}{4.263303in}}%
\pgfpathlineto{\pgfqpoint{1.032137in}{4.263891in}}%
\pgfpathlineto{\pgfqpoint{1.038397in}{4.264501in}}%
\pgfpathlineto{\pgfqpoint{1.044658in}{4.265133in}}%
\pgfpathlineto{\pgfqpoint{1.050918in}{4.265788in}}%
\pgfpathlineto{\pgfqpoint{1.057179in}{4.266467in}}%
\pgfpathlineto{\pgfqpoint{1.063439in}{4.267171in}}%
\pgfpathlineto{\pgfqpoint{1.069699in}{4.267899in}}%
\pgfpathlineto{\pgfqpoint{1.075960in}{4.268653in}}%
\pgfpathlineto{\pgfqpoint{1.082220in}{4.269434in}}%
\pgfpathlineto{\pgfqpoint{1.088481in}{4.270242in}}%
\pgfpathlineto{\pgfqpoint{1.094741in}{4.271077in}}%
\pgfpathlineto{\pgfqpoint{1.101002in}{4.271941in}}%
\pgfpathlineto{\pgfqpoint{1.107262in}{4.272834in}}%
\pgfpathlineto{\pgfqpoint{1.113523in}{4.273756in}}%
\pgfpathlineto{\pgfqpoint{1.119783in}{4.274709in}}%
\pgfpathlineto{\pgfqpoint{1.126043in}{4.275693in}}%
\pgfpathlineto{\pgfqpoint{1.132304in}{4.276708in}}%
\pgfpathlineto{\pgfqpoint{1.138564in}{4.277756in}}%
\pgfpathlineto{\pgfqpoint{1.144825in}{4.278837in}}%
\pgfpathlineto{\pgfqpoint{1.151085in}{4.279952in}}%
\pgfpathlineto{\pgfqpoint{1.157346in}{4.281102in}}%
\pgfpathlineto{\pgfqpoint{1.163606in}{4.282286in}}%
\pgfpathlineto{\pgfqpoint{1.169866in}{4.283506in}}%
\pgfpathlineto{\pgfqpoint{1.176127in}{4.284763in}}%
\pgfpathlineto{\pgfqpoint{1.182387in}{4.286056in}}%
\pgfpathlineto{\pgfqpoint{1.188648in}{4.287388in}}%
\pgfpathlineto{\pgfqpoint{1.194908in}{4.288758in}}%
\pgfpathlineto{\pgfqpoint{1.201169in}{4.290166in}}%
\pgfpathlineto{\pgfqpoint{1.207429in}{4.291615in}}%
\pgfpathlineto{\pgfqpoint{1.213689in}{4.293103in}}%
\pgfpathlineto{\pgfqpoint{1.219950in}{4.294632in}}%
\pgfpathlineto{\pgfqpoint{1.226210in}{4.296203in}}%
\pgfpathlineto{\pgfqpoint{1.232471in}{4.297816in}}%
\pgfpathlineto{\pgfqpoint{1.238731in}{4.299471in}}%
\pgfpathlineto{\pgfqpoint{1.244992in}{4.301169in}}%
\pgfpathlineto{\pgfqpoint{1.251252in}{4.302911in}}%
\pgfpathlineto{\pgfqpoint{1.257513in}{4.304697in}}%
\pgfpathlineto{\pgfqpoint{1.263773in}{4.306527in}}%
\pgfpathlineto{\pgfqpoint{1.270033in}{4.308402in}}%
\pgfpathlineto{\pgfqpoint{1.276294in}{4.310323in}}%
\pgfpathlineto{\pgfqpoint{1.282554in}{4.312290in}}%
\pgfpathlineto{\pgfqpoint{1.288815in}{4.314302in}}%
\pgfpathlineto{\pgfqpoint{1.295075in}{4.316361in}}%
\pgfpathlineto{\pgfqpoint{1.301336in}{4.318467in}}%
\pgfpathlineto{\pgfqpoint{1.307596in}{4.320621in}}%
\pgfpathlineto{\pgfqpoint{1.313856in}{4.322821in}}%
\pgfpathlineto{\pgfqpoint{1.320117in}{4.325070in}}%
\pgfpathlineto{\pgfqpoint{1.326377in}{4.327366in}}%
\pgfpathlineto{\pgfqpoint{1.332638in}{4.329710in}}%
\pgfpathlineto{\pgfqpoint{1.338898in}{4.332102in}}%
\pgfpathlineto{\pgfqpoint{1.345159in}{4.334543in}}%
\pgfpathlineto{\pgfqpoint{1.351419in}{4.337031in}}%
\pgfpathlineto{\pgfqpoint{1.357679in}{4.339568in}}%
\pgfpathlineto{\pgfqpoint{1.363940in}{4.342154in}}%
\pgfpathlineto{\pgfqpoint{1.370200in}{4.344787in}}%
\pgfpathlineto{\pgfqpoint{1.376461in}{4.347469in}}%
\pgfpathlineto{\pgfqpoint{1.382721in}{4.350199in}}%
\pgfpathlineto{\pgfqpoint{1.388982in}{4.352976in}}%
\pgfpathlineto{\pgfqpoint{1.395242in}{4.355801in}}%
\pgfpathlineto{\pgfqpoint{1.401503in}{4.358673in}}%
\pgfpathlineto{\pgfqpoint{1.407763in}{4.361592in}}%
\pgfpathlineto{\pgfqpoint{1.414023in}{4.364557in}}%
\pgfpathlineto{\pgfqpoint{1.420284in}{4.367568in}}%
\pgfpathlineto{\pgfqpoint{1.426544in}{4.370625in}}%
\pgfpathlineto{\pgfqpoint{1.432805in}{4.373727in}}%
\pgfpathlineto{\pgfqpoint{1.439065in}{4.376874in}}%
\pgfpathlineto{\pgfqpoint{1.445326in}{4.380064in}}%
\pgfpathlineto{\pgfqpoint{1.451586in}{4.383297in}}%
\pgfpathlineto{\pgfqpoint{1.457846in}{4.386572in}}%
\pgfpathlineto{\pgfqpoint{1.464107in}{4.389889in}}%
\pgfpathlineto{\pgfqpoint{1.470367in}{4.393247in}}%
\pgfpathlineto{\pgfqpoint{1.476628in}{4.396644in}}%
\pgfpathlineto{\pgfqpoint{1.482888in}{4.400080in}}%
\pgfpathlineto{\pgfqpoint{1.489149in}{4.403553in}}%
\pgfpathlineto{\pgfqpoint{1.495409in}{4.407064in}}%
\pgfpathlineto{\pgfqpoint{1.501669in}{4.410609in}}%
\pgfpathlineto{\pgfqpoint{1.507930in}{4.414190in}}%
\pgfpathlineto{\pgfqpoint{1.514190in}{4.417803in}}%
\pgfpathlineto{\pgfqpoint{1.520451in}{4.421447in}}%
\pgfpathlineto{\pgfqpoint{1.526711in}{4.425123in}}%
\pgfpathlineto{\pgfqpoint{1.532972in}{4.428827in}}%
\pgfpathlineto{\pgfqpoint{1.539232in}{4.432558in}}%
\pgfpathlineto{\pgfqpoint{1.545492in}{4.436316in}}%
\pgfpathlineto{\pgfqpoint{1.551753in}{4.440098in}}%
\pgfpathlineto{\pgfqpoint{1.558013in}{4.443902in}}%
\pgfpathlineto{\pgfqpoint{1.564274in}{4.447728in}}%
\pgfpathlineto{\pgfqpoint{1.570534in}{4.451573in}}%
\pgfpathlineto{\pgfqpoint{1.576795in}{4.455435in}}%
\pgfpathlineto{\pgfqpoint{1.583055in}{4.459313in}}%
\pgfpathlineto{\pgfqpoint{1.589316in}{4.463205in}}%
\pgfpathlineto{\pgfqpoint{1.595576in}{4.467108in}}%
\pgfpathlineto{\pgfqpoint{1.601836in}{4.471022in}}%
\pgfpathlineto{\pgfqpoint{1.608097in}{4.474943in}}%
\pgfpathlineto{\pgfqpoint{1.614357in}{4.478870in}}%
\pgfpathlineto{\pgfqpoint{1.620618in}{4.482801in}}%
\pgfpathlineto{\pgfqpoint{1.626878in}{4.486733in}}%
\pgfpathlineto{\pgfqpoint{1.633139in}{4.490665in}}%
\pgfpathlineto{\pgfqpoint{1.639399in}{4.494594in}}%
\pgfpathlineto{\pgfqpoint{1.645659in}{4.498517in}}%
\pgfpathlineto{\pgfqpoint{1.651920in}{4.502434in}}%
\pgfpathlineto{\pgfqpoint{1.658180in}{4.506340in}}%
\pgfpathlineto{\pgfqpoint{1.664441in}{4.510235in}}%
\pgfpathlineto{\pgfqpoint{1.670701in}{4.514114in}}%
\pgfpathlineto{\pgfqpoint{1.676962in}{4.517978in}}%
\pgfpathlineto{\pgfqpoint{1.683222in}{4.521821in}}%
\pgfpathlineto{\pgfqpoint{1.689482in}{4.525643in}}%
\pgfpathlineto{\pgfqpoint{1.695743in}{4.529441in}}%
\pgfpathlineto{\pgfqpoint{1.702003in}{4.533213in}}%
\pgfpathlineto{\pgfqpoint{1.708264in}{4.536955in}}%
\pgfpathlineto{\pgfqpoint{1.714524in}{4.540666in}}%
\pgfpathlineto{\pgfqpoint{1.720785in}{4.544342in}}%
\pgfpathlineto{\pgfqpoint{1.727045in}{4.547982in}}%
\pgfpathlineto{\pgfqpoint{1.733306in}{4.551583in}}%
\pgfpathlineto{\pgfqpoint{1.739566in}{4.555143in}}%
\pgfpathlineto{\pgfqpoint{1.745826in}{4.558658in}}%
\pgfpathlineto{\pgfqpoint{1.752087in}{4.562127in}}%
\pgfpathlineto{\pgfqpoint{1.758347in}{4.565547in}}%
\pgfpathlineto{\pgfqpoint{1.764608in}{4.568916in}}%
\pgfpathlineto{\pgfqpoint{1.770868in}{4.572231in}}%
\pgfpathlineto{\pgfqpoint{1.777129in}{4.575489in}}%
\pgfpathlineto{\pgfqpoint{1.783389in}{4.578689in}}%
\pgfpathlineto{\pgfqpoint{1.789649in}{4.581827in}}%
\pgfpathlineto{\pgfqpoint{1.795910in}{4.584903in}}%
\pgfpathlineto{\pgfqpoint{1.802170in}{4.587912in}}%
\pgfpathlineto{\pgfqpoint{1.808431in}{4.590854in}}%
\pgfpathlineto{\pgfqpoint{1.814691in}{4.593726in}}%
\pgfpathlineto{\pgfqpoint{1.820952in}{4.596525in}}%
\pgfpathlineto{\pgfqpoint{1.827212in}{4.599249in}}%
\pgfpathlineto{\pgfqpoint{1.833472in}{4.601897in}}%
\pgfpathlineto{\pgfqpoint{1.839733in}{4.604466in}}%
\pgfpathlineto{\pgfqpoint{1.845993in}{4.606954in}}%
\pgfpathlineto{\pgfqpoint{1.852254in}{4.609360in}}%
\pgfpathlineto{\pgfqpoint{1.858514in}{4.611681in}}%
\pgfpathlineto{\pgfqpoint{1.864775in}{4.613916in}}%
\pgfpathlineto{\pgfqpoint{1.871035in}{4.616062in}}%
\pgfpathlineto{\pgfqpoint{1.877295in}{4.618119in}}%
\pgfpathlineto{\pgfqpoint{1.883556in}{4.620083in}}%
\pgfpathlineto{\pgfqpoint{1.889816in}{4.621955in}}%
\pgfpathlineto{\pgfqpoint{1.896077in}{4.623732in}}%
\pgfpathlineto{\pgfqpoint{1.902337in}{4.625413in}}%
\pgfpathlineto{\pgfqpoint{1.908598in}{4.626996in}}%
\pgfpathlineto{\pgfqpoint{1.914858in}{4.628481in}}%
\pgfpathlineto{\pgfqpoint{1.921119in}{4.629865in}}%
\pgfpathlineto{\pgfqpoint{1.927379in}{4.631148in}}%
\pgfpathlineto{\pgfqpoint{1.933639in}{4.632329in}}%
\pgfpathlineto{\pgfqpoint{1.939900in}{4.633407in}}%
\pgfpathlineto{\pgfqpoint{1.946160in}{4.634381in}}%
\pgfpathlineto{\pgfqpoint{1.952421in}{4.635250in}}%
\pgfpathlineto{\pgfqpoint{1.958681in}{4.636013in}}%
\pgfpathlineto{\pgfqpoint{1.964942in}{4.636670in}}%
\pgfpathlineto{\pgfqpoint{1.971202in}{4.637220in}}%
\pgfpathlineto{\pgfqpoint{1.977462in}{4.637663in}}%
\pgfpathlineto{\pgfqpoint{1.983723in}{4.637998in}}%
\pgfpathlineto{\pgfqpoint{1.989983in}{4.638226in}}%
\pgfpathlineto{\pgfqpoint{1.996244in}{4.638345in}}%
\pgfpathlineto{\pgfqpoint{2.002504in}{4.638356in}}%
\pgfpathlineto{\pgfqpoint{2.008765in}{4.638259in}}%
\pgfpathlineto{\pgfqpoint{2.015025in}{4.638054in}}%
\pgfpathlineto{\pgfqpoint{2.021285in}{4.637741in}}%
\pgfpathlineto{\pgfqpoint{2.027546in}{4.637321in}}%
\pgfpathlineto{\pgfqpoint{2.033806in}{4.636793in}}%
\pgfpathlineto{\pgfqpoint{2.040067in}{4.636158in}}%
\pgfpathlineto{\pgfqpoint{2.046327in}{4.635417in}}%
\pgfpathlineto{\pgfqpoint{2.052588in}{4.634571in}}%
\pgfpathlineto{\pgfqpoint{2.058848in}{4.633620in}}%
\pgfpathlineto{\pgfqpoint{2.065109in}{4.632564in}}%
\pgfpathlineto{\pgfqpoint{2.071369in}{4.631406in}}%
\pgfpathlineto{\pgfqpoint{2.077629in}{4.630145in}}%
\pgfpathlineto{\pgfqpoint{2.083890in}{4.628784in}}%
\pgfpathlineto{\pgfqpoint{2.090150in}{4.627322in}}%
\pgfpathlineto{\pgfqpoint{2.096411in}{4.625762in}}%
\pgfpathlineto{\pgfqpoint{2.102671in}{4.624105in}}%
\pgfpathlineto{\pgfqpoint{2.108932in}{4.622352in}}%
\pgfpathlineto{\pgfqpoint{2.115192in}{4.620504in}}%
\pgfpathlineto{\pgfqpoint{2.121452in}{4.618564in}}%
\pgfpathlineto{\pgfqpoint{2.127713in}{4.616533in}}%
\pgfpathlineto{\pgfqpoint{2.133973in}{4.614412in}}%
\pgfpathlineto{\pgfqpoint{2.140234in}{4.612204in}}%
\pgfpathlineto{\pgfqpoint{2.146494in}{4.609911in}}%
\pgfpathlineto{\pgfqpoint{2.152755in}{4.607534in}}%
\pgfpathlineto{\pgfqpoint{2.159015in}{4.605075in}}%
\pgfpathlineto{\pgfqpoint{2.165275in}{4.602538in}}%
\pgfpathlineto{\pgfqpoint{2.171536in}{4.599923in}}%
\pgfpathlineto{\pgfqpoint{2.177796in}{4.597234in}}%
\pgfpathlineto{\pgfqpoint{2.184057in}{4.594473in}}%
\pgfpathlineto{\pgfqpoint{2.190317in}{4.591642in}}%
\pgfpathlineto{\pgfqpoint{2.196578in}{4.588743in}}%
\pgfpathlineto{\pgfqpoint{2.202838in}{4.585780in}}%
\pgfpathlineto{\pgfqpoint{2.209098in}{4.582755in}}%
\pgfpathlineto{\pgfqpoint{2.215359in}{4.579672in}}%
\pgfpathlineto{\pgfqpoint{2.221619in}{4.576532in}}%
\pgfpathlineto{\pgfqpoint{2.227880in}{4.573339in}}%
\pgfpathlineto{\pgfqpoint{2.234140in}{4.570096in}}%
\pgfpathlineto{\pgfqpoint{2.240401in}{4.566806in}}%
\pgfpathlineto{\pgfqpoint{2.246661in}{4.563472in}}%
\pgfpathlineto{\pgfqpoint{2.252922in}{4.560098in}}%
\pgfpathlineto{\pgfqpoint{2.259182in}{4.556688in}}%
\pgfpathlineto{\pgfqpoint{2.265442in}{4.553244in}}%
\pgfpathlineto{\pgfqpoint{2.271703in}{4.549771in}}%
\pgfpathlineto{\pgfqpoint{2.277963in}{4.546272in}}%
\pgfpathlineto{\pgfqpoint{2.284224in}{4.542751in}}%
\pgfpathlineto{\pgfqpoint{2.290484in}{4.539212in}}%
\pgfpathlineto{\pgfqpoint{2.296745in}{4.535660in}}%
\pgfpathlineto{\pgfqpoint{2.303005in}{4.532098in}}%
\pgfpathlineto{\pgfqpoint{2.309265in}{4.528531in}}%
\pgfpathlineto{\pgfqpoint{2.315526in}{4.524964in}}%
\pgfpathlineto{\pgfqpoint{2.321786in}{4.521401in}}%
\pgfpathlineto{\pgfqpoint{2.328047in}{4.517847in}}%
\pgfpathlineto{\pgfqpoint{2.334307in}{4.514308in}}%
\pgfpathlineto{\pgfqpoint{2.340568in}{4.510788in}}%
\pgfpathlineto{\pgfqpoint{2.346828in}{4.507293in}}%
\pgfpathlineto{\pgfqpoint{2.353088in}{4.503828in}}%
\pgfpathlineto{\pgfqpoint{2.359349in}{4.500399in}}%
\pgfpathlineto{\pgfqpoint{2.365609in}{4.497013in}}%
\pgfpathlineto{\pgfqpoint{2.371870in}{4.493674in}}%
\pgfpathlineto{\pgfqpoint{2.378130in}{4.490390in}}%
\pgfpathlineto{\pgfqpoint{2.384391in}{4.487167in}}%
\pgfpathlineto{\pgfqpoint{2.390651in}{4.484012in}}%
\pgfpathlineto{\pgfqpoint{2.396912in}{4.480932in}}%
\pgfpathlineto{\pgfqpoint{2.403172in}{4.477933in}}%
\pgfpathlineto{\pgfqpoint{2.409432in}{4.475024in}}%
\pgfpathlineto{\pgfqpoint{2.415693in}{4.472211in}}%
\pgfpathlineto{\pgfqpoint{2.421953in}{4.469503in}}%
\pgfpathlineto{\pgfqpoint{2.428214in}{4.466908in}}%
\pgfpathlineto{\pgfqpoint{2.434474in}{4.464433in}}%
\pgfpathlineto{\pgfqpoint{2.440735in}{4.462087in}}%
\pgfpathlineto{\pgfqpoint{2.446995in}{4.459879in}}%
\pgfpathlineto{\pgfqpoint{2.453255in}{4.457817in}}%
\pgfpathlineto{\pgfqpoint{2.459516in}{4.455910in}}%
\pgfpathlineto{\pgfqpoint{2.465776in}{4.454168in}}%
\pgfpathlineto{\pgfqpoint{2.472037in}{4.452599in}}%
\pgfpathlineto{\pgfqpoint{2.478297in}{4.451212in}}%
\pgfpathlineto{\pgfqpoint{2.484558in}{4.450017in}}%
\pgfpathlineto{\pgfqpoint{2.490818in}{4.449024in}}%
\pgfpathlineto{\pgfqpoint{2.497078in}{4.448241in}}%
\pgfpathlineto{\pgfqpoint{2.503339in}{4.447678in}}%
\pgfpathlineto{\pgfqpoint{2.509599in}{4.447344in}}%
\pgfpathlineto{\pgfqpoint{2.515860in}{4.447249in}}%
\pgfpathlineto{\pgfqpoint{2.522120in}{4.447402in}}%
\pgfpathlineto{\pgfqpoint{2.528381in}{4.447811in}}%
\pgfpathlineto{\pgfqpoint{2.534641in}{4.448485in}}%
\pgfpathlineto{\pgfqpoint{2.540902in}{4.449434in}}%
\pgfpathlineto{\pgfqpoint{2.547162in}{4.450665in}}%
\pgfpathlineto{\pgfqpoint{2.553422in}{4.452186in}}%
\pgfpathlineto{\pgfqpoint{2.559683in}{4.454004in}}%
\pgfpathlineto{\pgfqpoint{2.565943in}{4.456127in}}%
\pgfpathlineto{\pgfqpoint{2.572204in}{4.458561in}}%
\pgfpathlineto{\pgfqpoint{2.578464in}{4.461312in}}%
\pgfpathlineto{\pgfqpoint{2.584725in}{4.464386in}}%
\pgfpathlineto{\pgfqpoint{2.590985in}{4.467786in}}%
\pgfpathlineto{\pgfqpoint{2.597245in}{4.471517in}}%
\pgfpathlineto{\pgfqpoint{2.603506in}{4.475581in}}%
\pgfpathlineto{\pgfqpoint{2.609766in}{4.479982in}}%
\pgfpathlineto{\pgfqpoint{2.616027in}{4.484719in}}%
\pgfpathlineto{\pgfqpoint{2.622287in}{4.489794in}}%
\pgfpathlineto{\pgfqpoint{2.628548in}{4.495205in}}%
\pgfpathlineto{\pgfqpoint{2.634808in}{4.500951in}}%
\pgfpathlineto{\pgfqpoint{2.641068in}{4.507028in}}%
\pgfpathlineto{\pgfqpoint{2.647329in}{4.513433in}}%
\pgfpathlineto{\pgfqpoint{2.653589in}{4.520159in}}%
\pgfpathlineto{\pgfqpoint{2.659850in}{4.527200in}}%
\pgfpathlineto{\pgfqpoint{2.666110in}{4.534548in}}%
\pgfpathlineto{\pgfqpoint{2.672371in}{4.542194in}}%
\pgfpathlineto{\pgfqpoint{2.678631in}{4.550127in}}%
\pgfpathlineto{\pgfqpoint{2.684891in}{4.558334in}}%
\pgfpathlineto{\pgfqpoint{2.691152in}{4.566802in}}%
\pgfpathlineto{\pgfqpoint{2.697412in}{4.575517in}}%
\pgfpathlineto{\pgfqpoint{2.703673in}{4.584461in}}%
\pgfpathlineto{\pgfqpoint{2.709933in}{4.593619in}}%
\pgfpathlineto{\pgfqpoint{2.716194in}{4.602969in}}%
\pgfpathlineto{\pgfqpoint{2.722454in}{4.612494in}}%
\pgfpathlineto{\pgfqpoint{2.728715in}{4.622170in}}%
\pgfpathlineto{\pgfqpoint{2.734975in}{4.631975in}}%
\pgfpathlineto{\pgfqpoint{2.741235in}{4.641885in}}%
\pgfpathlineto{\pgfqpoint{2.747496in}{4.651877in}}%
\pgfpathlineto{\pgfqpoint{2.753756in}{4.661922in}}%
\pgfpathlineto{\pgfqpoint{2.760017in}{4.671995in}}%
\pgfpathlineto{\pgfqpoint{2.766277in}{4.682068in}}%
\pgfpathlineto{\pgfqpoint{2.772538in}{4.692113in}}%
\pgfpathlineto{\pgfqpoint{2.778798in}{4.702100in}}%
\pgfpathlineto{\pgfqpoint{2.785058in}{4.712000in}}%
\pgfpathlineto{\pgfqpoint{2.791319in}{4.721782in}}%
\pgfpathlineto{\pgfqpoint{2.797579in}{4.731417in}}%
\pgfpathlineto{\pgfqpoint{2.803840in}{4.740874in}}%
\pgfpathlineto{\pgfqpoint{2.810100in}{4.750122in}}%
\pgfpathlineto{\pgfqpoint{2.816361in}{4.759131in}}%
\pgfpathlineto{\pgfqpoint{2.822621in}{4.767871in}}%
\pgfpathlineto{\pgfqpoint{2.828881in}{4.776312in}}%
\pgfpathlineto{\pgfqpoint{2.835142in}{4.784424in}}%
\pgfpathlineto{\pgfqpoint{2.841402in}{4.792180in}}%
\pgfpathlineto{\pgfqpoint{2.847663in}{4.799551in}}%
\pgfpathlineto{\pgfqpoint{2.853923in}{4.806511in}}%
\pgfpathlineto{\pgfqpoint{2.860184in}{4.813034in}}%
\pgfpathlineto{\pgfqpoint{2.866444in}{4.819095in}}%
\pgfpathlineto{\pgfqpoint{2.872705in}{4.824670in}}%
\pgfpathlineto{\pgfqpoint{2.878965in}{4.829740in}}%
\pgfpathlineto{\pgfqpoint{2.885225in}{4.834282in}}%
\pgfpathlineto{\pgfqpoint{2.891486in}{4.838279in}}%
\pgfpathlineto{\pgfqpoint{2.897746in}{4.841713in}}%
\pgfpathlineto{\pgfqpoint{2.904007in}{4.844571in}}%
\pgfpathlineto{\pgfqpoint{2.910267in}{4.846838in}}%
\pgfpathlineto{\pgfqpoint{2.916528in}{4.848504in}}%
\pgfpathlineto{\pgfqpoint{2.922788in}{4.849560in}}%
\pgfpathlineto{\pgfqpoint{2.929048in}{4.850000in}}%
\pgfpathlineto{\pgfqpoint{2.935309in}{4.849818in}}%
\pgfpathlineto{\pgfqpoint{2.941569in}{4.849012in}}%
\pgfpathlineto{\pgfqpoint{2.947830in}{4.847582in}}%
\pgfpathlineto{\pgfqpoint{2.954090in}{4.845530in}}%
\pgfpathlineto{\pgfqpoint{2.960351in}{4.842859in}}%
\pgfpathlineto{\pgfqpoint{2.966611in}{4.839576in}}%
\pgfpathlineto{\pgfqpoint{2.972871in}{4.835689in}}%
\pgfpathlineto{\pgfqpoint{2.979132in}{4.831208in}}%
\pgfpathlineto{\pgfqpoint{2.985392in}{4.826146in}}%
\pgfpathlineto{\pgfqpoint{2.991653in}{4.820516in}}%
\pgfpathlineto{\pgfqpoint{2.997913in}{4.814336in}}%
\pgfpathlineto{\pgfqpoint{3.004174in}{4.807622in}}%
\pgfpathlineto{\pgfqpoint{3.010434in}{4.800394in}}%
\pgfpathlineto{\pgfqpoint{3.016694in}{4.792674in}}%
\pgfpathlineto{\pgfqpoint{3.022955in}{4.784483in}}%
\pgfpathlineto{\pgfqpoint{3.029215in}{4.775845in}}%
\pgfpathlineto{\pgfqpoint{3.035476in}{4.766785in}}%
\pgfpathlineto{\pgfqpoint{3.041736in}{4.757329in}}%
\pgfpathlineto{\pgfqpoint{3.047997in}{4.747504in}}%
\pgfpathlineto{\pgfqpoint{3.054257in}{4.737338in}}%
\pgfpathlineto{\pgfqpoint{3.060518in}{4.726859in}}%
\pgfpathlineto{\pgfqpoint{3.066778in}{4.716095in}}%
\pgfpathlineto{\pgfqpoint{3.073038in}{4.705076in}}%
\pgfpathlineto{\pgfqpoint{3.079299in}{4.693831in}}%
\pgfpathlineto{\pgfqpoint{3.085559in}{4.682390in}}%
\pgfpathlineto{\pgfqpoint{3.091820in}{4.670782in}}%
\pgfpathlineto{\pgfqpoint{3.098080in}{4.659037in}}%
\pgfpathlineto{\pgfqpoint{3.104341in}{4.647182in}}%
\pgfpathlineto{\pgfqpoint{3.110601in}{4.635248in}}%
\pgfpathlineto{\pgfqpoint{3.116861in}{4.623261in}}%
\pgfpathlineto{\pgfqpoint{3.123122in}{4.611249in}}%
\pgfpathlineto{\pgfqpoint{3.129382in}{4.599239in}}%
\pgfpathlineto{\pgfqpoint{3.135643in}{4.587257in}}%
\pgfpathlineto{\pgfqpoint{3.141903in}{4.575327in}}%
\pgfpathlineto{\pgfqpoint{3.148164in}{4.563473in}}%
\pgfpathlineto{\pgfqpoint{3.154424in}{4.551718in}}%
\pgfpathlineto{\pgfqpoint{3.160684in}{4.540083in}}%
\pgfpathlineto{\pgfqpoint{3.166945in}{4.528589in}}%
\pgfpathlineto{\pgfqpoint{3.173205in}{4.517256in}}%
\pgfpathlineto{\pgfqpoint{3.179466in}{4.506100in}}%
\pgfpathlineto{\pgfqpoint{3.185726in}{4.495140in}}%
\pgfpathlineto{\pgfqpoint{3.191987in}{4.484389in}}%
\pgfpathlineto{\pgfqpoint{3.198247in}{4.473862in}}%
\pgfpathlineto{\pgfqpoint{3.204508in}{4.463571in}}%
\pgfpathlineto{\pgfqpoint{3.210768in}{4.453529in}}%
\pgfpathlineto{\pgfqpoint{3.217028in}{4.443744in}}%
\pgfpathlineto{\pgfqpoint{3.223289in}{4.434226in}}%
\pgfpathlineto{\pgfqpoint{3.229549in}{4.424981in}}%
\pgfpathlineto{\pgfqpoint{3.235810in}{4.416017in}}%
\pgfpathlineto{\pgfqpoint{3.242070in}{4.407338in}}%
\pgfpathlineto{\pgfqpoint{3.248331in}{4.398948in}}%
\pgfpathlineto{\pgfqpoint{3.254591in}{4.390849in}}%
\pgfpathlineto{\pgfqpoint{3.260851in}{4.383044in}}%
\pgfpathlineto{\pgfqpoint{3.267112in}{4.375532in}}%
\pgfpathlineto{\pgfqpoint{3.273372in}{4.368313in}}%
\pgfpathlineto{\pgfqpoint{3.279633in}{4.361386in}}%
\pgfpathlineto{\pgfqpoint{3.285893in}{4.354749in}}%
\pgfpathlineto{\pgfqpoint{3.292154in}{4.348398in}}%
\pgfpathlineto{\pgfqpoint{3.298414in}{4.342330in}}%
\pgfpathlineto{\pgfqpoint{3.304674in}{4.336541in}}%
\pgfpathlineto{\pgfqpoint{3.310935in}{4.331025in}}%
\pgfpathlineto{\pgfqpoint{3.317195in}{4.325777in}}%
\pgfpathlineto{\pgfqpoint{3.323456in}{4.320790in}}%
\pgfpathlineto{\pgfqpoint{3.329716in}{4.316059in}}%
\pgfpathlineto{\pgfqpoint{3.335977in}{4.311576in}}%
\pgfpathlineto{\pgfqpoint{3.342237in}{4.307334in}}%
\pgfpathlineto{\pgfqpoint{3.348497in}{4.303325in}}%
\pgfpathlineto{\pgfqpoint{3.354758in}{4.299541in}}%
\pgfpathlineto{\pgfqpoint{3.361018in}{4.295976in}}%
\pgfpathlineto{\pgfqpoint{3.367279in}{4.292620in}}%
\pgfpathlineto{\pgfqpoint{3.373539in}{4.289465in}}%
\pgfpathlineto{\pgfqpoint{3.379800in}{4.286504in}}%
\pgfpathlineto{\pgfqpoint{3.386060in}{4.283728in}}%
\pgfpathlineto{\pgfqpoint{3.392321in}{4.281128in}}%
\pgfpathlineto{\pgfqpoint{3.398581in}{4.278698in}}%
\pgfpathlineto{\pgfqpoint{3.404841in}{4.276428in}}%
\pgfpathlineto{\pgfqpoint{3.411102in}{4.274311in}}%
\pgfpathlineto{\pgfqpoint{3.417362in}{4.272339in}}%
\pgfpathlineto{\pgfqpoint{3.423623in}{4.270504in}}%
\pgfpathlineto{\pgfqpoint{3.429883in}{4.268800in}}%
\pgfpathlineto{\pgfqpoint{3.436144in}{4.267218in}}%
\pgfpathlineto{\pgfqpoint{3.442404in}{4.265752in}}%
\pgfpathlineto{\pgfqpoint{3.448664in}{4.264396in}}%
\pgfpathlineto{\pgfqpoint{3.454925in}{4.263141in}}%
\pgfpathlineto{\pgfqpoint{3.461185in}{4.261983in}}%
\pgfpathlineto{\pgfqpoint{3.467446in}{4.260915in}}%
\pgfpathlineto{\pgfqpoint{3.473706in}{4.259932in}}%
\pgfpathlineto{\pgfqpoint{3.479967in}{4.259027in}}%
\pgfpathlineto{\pgfqpoint{3.486227in}{4.258196in}}%
\pgfpathlineto{\pgfqpoint{3.492487in}{4.257433in}}%
\pgfpathlineto{\pgfqpoint{3.498748in}{4.256734in}}%
\pgfpathlineto{\pgfqpoint{3.505008in}{4.256094in}}%
\pgfpathlineto{\pgfqpoint{3.511269in}{4.255510in}}%
\pgfpathlineto{\pgfqpoint{3.517529in}{4.254975in}}%
\pgfpathlineto{\pgfqpoint{3.523790in}{4.254488in}}%
\pgfpathlineto{\pgfqpoint{3.530050in}{4.254044in}}%
\pgfpathlineto{\pgfqpoint{3.536311in}{4.253641in}}%
\pgfpathlineto{\pgfqpoint{3.542571in}{4.253274in}}%
\pgfpathlineto{\pgfqpoint{3.548831in}{4.252941in}}%
\pgfpathlineto{\pgfqpoint{3.555092in}{4.252639in}}%
\pgfpathlineto{\pgfqpoint{3.561352in}{4.252365in}}%
\pgfpathlineto{\pgfqpoint{3.567613in}{4.252118in}}%
\pgfpathlineto{\pgfqpoint{3.573873in}{4.251895in}}%
\pgfpathlineto{\pgfqpoint{3.580134in}{4.251693in}}%
\pgfpathlineto{\pgfqpoint{3.586394in}{4.251511in}}%
\pgfpathlineto{\pgfqpoint{3.592654in}{4.251348in}}%
\pgfpathlineto{\pgfqpoint{3.598915in}{4.251201in}}%
\pgfpathlineto{\pgfqpoint{3.605175in}{4.251068in}}%
\pgfpathlineto{\pgfqpoint{3.611436in}{4.250950in}}%
\pgfpathlineto{\pgfqpoint{3.617696in}{4.250844in}}%
\pgfpathlineto{\pgfqpoint{3.623957in}{4.250749in}}%
\pgfpathlineto{\pgfqpoint{3.630217in}{4.250664in}}%
\pgfpathlineto{\pgfqpoint{3.636477in}{4.250588in}}%
\pgfpathlineto{\pgfqpoint{3.642738in}{4.250520in}}%
\pgfpathlineto{\pgfqpoint{3.648998in}{4.250460in}}%
\pgfpathlineto{\pgfqpoint{3.655259in}{4.250406in}}%
\pgfpathlineto{\pgfqpoint{3.661519in}{4.250358in}}%
\pgfpathlineto{\pgfqpoint{3.667780in}{4.250316in}}%
\pgfpathlineto{\pgfqpoint{3.674040in}{4.250278in}}%
\pgfpathlineto{\pgfqpoint{3.680301in}{4.250245in}}%
\pgfpathlineto{\pgfqpoint{3.686561in}{4.250215in}}%
\pgfpathlineto{\pgfqpoint{3.692821in}{4.250189in}}%
\pgfpathlineto{\pgfqpoint{3.699082in}{4.250166in}}%
\pgfpathlineto{\pgfqpoint{3.705342in}{4.250145in}}%
\pgfpathlineto{\pgfqpoint{3.711603in}{4.250128in}}%
\pgfpathlineto{\pgfqpoint{3.717863in}{4.250112in}}%
\pgfpathlineto{\pgfqpoint{3.724124in}{4.250098in}}%
\pgfpathlineto{\pgfqpoint{3.730384in}{4.250085in}}%
\pgfpathlineto{\pgfqpoint{3.736644in}{4.250075in}}%
\pgfpathlineto{\pgfqpoint{3.742905in}{4.250065in}}%
\pgfpathlineto{\pgfqpoint{3.749165in}{4.250057in}}%
\pgfpathlineto{\pgfqpoint{3.755426in}{4.250050in}}%
\pgfpathlineto{\pgfqpoint{3.761686in}{4.250043in}}%
\pgfpathlineto{\pgfqpoint{3.767947in}{4.250038in}}%
\pgfpathlineto{\pgfqpoint{3.774207in}{4.250033in}}%
\pgfpathlineto{\pgfqpoint{3.780467in}{4.250029in}}%
\pgfpathlineto{\pgfqpoint{3.786728in}{4.250025in}}%
\pgfpathlineto{\pgfqpoint{3.792988in}{4.250022in}}%
\pgfpathlineto{\pgfqpoint{3.799249in}{4.250019in}}%
\pgfpathlineto{\pgfqpoint{3.805509in}{4.250017in}}%
\pgfpathlineto{\pgfqpoint{3.811770in}{4.250014in}}%
\pgfpathlineto{\pgfqpoint{3.818030in}{4.250013in}}%
\pgfpathlineto{\pgfqpoint{3.824290in}{4.250011in}}%
\pgfpathlineto{\pgfqpoint{3.830551in}{4.250010in}}%
\pgfpathlineto{\pgfqpoint{3.836811in}{4.250008in}}%
\pgfpathlineto{\pgfqpoint{3.843072in}{4.250007in}}%
\pgfpathlineto{\pgfqpoint{3.849332in}{4.250006in}}%
\pgfpathlineto{\pgfqpoint{3.855593in}{4.250005in}}%
\pgfpathlineto{\pgfqpoint{3.861853in}{4.250005in}}%
\pgfpathlineto{\pgfqpoint{3.868114in}{4.250004in}}%
\pgfpathlineto{\pgfqpoint{3.874374in}{4.250004in}}%
\pgfpathlineto{\pgfqpoint{3.880634in}{4.250003in}}%
\pgfpathlineto{\pgfqpoint{3.886895in}{4.250003in}}%
\pgfpathlineto{\pgfqpoint{3.893155in}{4.250002in}}%
\pgfpathlineto{\pgfqpoint{3.899416in}{4.250002in}}%
\pgfpathlineto{\pgfqpoint{3.905676in}{4.250002in}}%
\pgfpathlineto{\pgfqpoint{3.911937in}{4.250002in}}%
\pgfpathlineto{\pgfqpoint{3.918197in}{4.250001in}}%
\pgfpathlineto{\pgfqpoint{3.924457in}{4.250001in}}%
\pgfpathlineto{\pgfqpoint{3.930718in}{4.250001in}}%
\pgfpathlineto{\pgfqpoint{3.936978in}{4.250001in}}%
\pgfpathlineto{\pgfqpoint{3.943239in}{4.250001in}}%
\pgfpathlineto{\pgfqpoint{3.949499in}{4.250001in}}%
\pgfpathlineto{\pgfqpoint{3.955760in}{4.250001in}}%
\pgfpathlineto{\pgfqpoint{3.962020in}{4.250001in}}%
\pgfpathlineto{\pgfqpoint{3.968280in}{4.250001in}}%
\pgfpathlineto{\pgfqpoint{3.974541in}{4.250001in}}%
\pgfpathlineto{\pgfqpoint{3.980801in}{4.250000in}}%
\pgfpathlineto{\pgfqpoint{3.987062in}{4.250000in}}%
\pgfpathlineto{\pgfqpoint{3.993322in}{4.250000in}}%
\pgfpathlineto{\pgfqpoint{3.999583in}{4.250000in}}%
\pgfpathlineto{\pgfqpoint{4.005843in}{4.250000in}}%
\pgfpathlineto{\pgfqpoint{4.012104in}{4.250000in}}%
\pgfpathlineto{\pgfqpoint{4.018364in}{4.250000in}}%
\pgfpathlineto{\pgfqpoint{4.024624in}{4.250000in}}%
\pgfpathlineto{\pgfqpoint{4.030885in}{4.250000in}}%
\pgfpathlineto{\pgfqpoint{4.037145in}{4.250000in}}%
\pgfpathlineto{\pgfqpoint{4.043406in}{4.250000in}}%
\pgfpathlineto{\pgfqpoint{4.049666in}{4.250000in}}%
\pgfpathlineto{\pgfqpoint{4.055927in}{4.250000in}}%
\pgfpathlineto{\pgfqpoint{4.062187in}{4.250000in}}%
\pgfpathlineto{\pgfqpoint{4.068447in}{4.250000in}}%
\pgfpathlineto{\pgfqpoint{4.074708in}{4.250000in}}%
\pgfpathlineto{\pgfqpoint{4.080968in}{4.250000in}}%
\pgfpathlineto{\pgfqpoint{4.087229in}{4.250000in}}%
\pgfpathlineto{\pgfqpoint{4.093489in}{4.250000in}}%
\pgfpathlineto{\pgfqpoint{4.099750in}{4.250000in}}%
\pgfpathlineto{\pgfqpoint{4.106010in}{4.250000in}}%
\pgfpathlineto{\pgfqpoint{4.112270in}{4.250000in}}%
\pgfpathlineto{\pgfqpoint{4.118531in}{4.250000in}}%
\pgfpathlineto{\pgfqpoint{4.124791in}{4.250000in}}%
\pgfpathlineto{\pgfqpoint{4.131052in}{4.250000in}}%
\pgfpathlineto{\pgfqpoint{4.137312in}{4.250000in}}%
\pgfpathlineto{\pgfqpoint{4.143573in}{4.250000in}}%
\pgfpathlineto{\pgfqpoint{4.149833in}{4.250000in}}%
\pgfpathlineto{\pgfqpoint{4.156093in}{4.250000in}}%
\pgfpathlineto{\pgfqpoint{4.162354in}{4.250000in}}%
\pgfpathlineto{\pgfqpoint{4.168614in}{4.250000in}}%
\pgfpathlineto{\pgfqpoint{4.174875in}{4.250000in}}%
\pgfpathlineto{\pgfqpoint{4.181135in}{4.250000in}}%
\pgfpathlineto{\pgfqpoint{4.187396in}{4.250000in}}%
\pgfpathlineto{\pgfqpoint{4.193656in}{4.250000in}}%
\pgfpathlineto{\pgfqpoint{4.199917in}{4.250000in}}%
\pgfpathlineto{\pgfqpoint{4.206177in}{4.250000in}}%
\pgfpathlineto{\pgfqpoint{4.212437in}{4.250000in}}%
\pgfpathlineto{\pgfqpoint{4.218698in}{4.250000in}}%
\pgfpathlineto{\pgfqpoint{4.224958in}{4.250000in}}%
\pgfpathlineto{\pgfqpoint{4.231219in}{4.250000in}}%
\pgfpathlineto{\pgfqpoint{4.237479in}{4.250000in}}%
\pgfpathlineto{\pgfqpoint{4.243740in}{4.250000in}}%
\pgfpathlineto{\pgfqpoint{4.250000in}{4.250000in}}%
\pgfpathlineto{\pgfqpoint{4.250000in}{4.250000in}}%
\pgfpathlineto{\pgfqpoint{4.250000in}{4.250000in}}%
\pgfpathlineto{\pgfqpoint{4.243740in}{4.250000in}}%
\pgfpathlineto{\pgfqpoint{4.237479in}{4.250000in}}%
\pgfpathlineto{\pgfqpoint{4.231219in}{4.250000in}}%
\pgfpathlineto{\pgfqpoint{4.224958in}{4.250000in}}%
\pgfpathlineto{\pgfqpoint{4.218698in}{4.250000in}}%
\pgfpathlineto{\pgfqpoint{4.212437in}{4.250000in}}%
\pgfpathlineto{\pgfqpoint{4.206177in}{4.250000in}}%
\pgfpathlineto{\pgfqpoint{4.199917in}{4.250000in}}%
\pgfpathlineto{\pgfqpoint{4.193656in}{4.250000in}}%
\pgfpathlineto{\pgfqpoint{4.187396in}{4.250000in}}%
\pgfpathlineto{\pgfqpoint{4.181135in}{4.250000in}}%
\pgfpathlineto{\pgfqpoint{4.174875in}{4.250000in}}%
\pgfpathlineto{\pgfqpoint{4.168614in}{4.250000in}}%
\pgfpathlineto{\pgfqpoint{4.162354in}{4.250000in}}%
\pgfpathlineto{\pgfqpoint{4.156093in}{4.250000in}}%
\pgfpathlineto{\pgfqpoint{4.149833in}{4.250000in}}%
\pgfpathlineto{\pgfqpoint{4.143573in}{4.250000in}}%
\pgfpathlineto{\pgfqpoint{4.137312in}{4.250000in}}%
\pgfpathlineto{\pgfqpoint{4.131052in}{4.250000in}}%
\pgfpathlineto{\pgfqpoint{4.124791in}{4.250000in}}%
\pgfpathlineto{\pgfqpoint{4.118531in}{4.250000in}}%
\pgfpathlineto{\pgfqpoint{4.112270in}{4.250000in}}%
\pgfpathlineto{\pgfqpoint{4.106010in}{4.250000in}}%
\pgfpathlineto{\pgfqpoint{4.099750in}{4.250000in}}%
\pgfpathlineto{\pgfqpoint{4.093489in}{4.250000in}}%
\pgfpathlineto{\pgfqpoint{4.087229in}{4.250000in}}%
\pgfpathlineto{\pgfqpoint{4.080968in}{4.250000in}}%
\pgfpathlineto{\pgfqpoint{4.074708in}{4.250000in}}%
\pgfpathlineto{\pgfqpoint{4.068447in}{4.250000in}}%
\pgfpathlineto{\pgfqpoint{4.062187in}{4.250000in}}%
\pgfpathlineto{\pgfqpoint{4.055927in}{4.250000in}}%
\pgfpathlineto{\pgfqpoint{4.049666in}{4.250000in}}%
\pgfpathlineto{\pgfqpoint{4.043406in}{4.250000in}}%
\pgfpathlineto{\pgfqpoint{4.037145in}{4.250000in}}%
\pgfpathlineto{\pgfqpoint{4.030885in}{4.250000in}}%
\pgfpathlineto{\pgfqpoint{4.024624in}{4.250000in}}%
\pgfpathlineto{\pgfqpoint{4.018364in}{4.250000in}}%
\pgfpathlineto{\pgfqpoint{4.012104in}{4.250000in}}%
\pgfpathlineto{\pgfqpoint{4.005843in}{4.250000in}}%
\pgfpathlineto{\pgfqpoint{3.999583in}{4.250000in}}%
\pgfpathlineto{\pgfqpoint{3.993322in}{4.250000in}}%
\pgfpathlineto{\pgfqpoint{3.987062in}{4.250000in}}%
\pgfpathlineto{\pgfqpoint{3.980801in}{4.250000in}}%
\pgfpathlineto{\pgfqpoint{3.974541in}{4.250000in}}%
\pgfpathlineto{\pgfqpoint{3.968280in}{4.250000in}}%
\pgfpathlineto{\pgfqpoint{3.962020in}{4.250000in}}%
\pgfpathlineto{\pgfqpoint{3.955760in}{4.250000in}}%
\pgfpathlineto{\pgfqpoint{3.949499in}{4.250000in}}%
\pgfpathlineto{\pgfqpoint{3.943239in}{4.250000in}}%
\pgfpathlineto{\pgfqpoint{3.936978in}{4.250000in}}%
\pgfpathlineto{\pgfqpoint{3.930718in}{4.250000in}}%
\pgfpathlineto{\pgfqpoint{3.924457in}{4.250000in}}%
\pgfpathlineto{\pgfqpoint{3.918197in}{4.250000in}}%
\pgfpathlineto{\pgfqpoint{3.911937in}{4.250000in}}%
\pgfpathlineto{\pgfqpoint{3.905676in}{4.250000in}}%
\pgfpathlineto{\pgfqpoint{3.899416in}{4.250000in}}%
\pgfpathlineto{\pgfqpoint{3.893155in}{4.250000in}}%
\pgfpathlineto{\pgfqpoint{3.886895in}{4.250000in}}%
\pgfpathlineto{\pgfqpoint{3.880634in}{4.250000in}}%
\pgfpathlineto{\pgfqpoint{3.874374in}{4.250000in}}%
\pgfpathlineto{\pgfqpoint{3.868114in}{4.250000in}}%
\pgfpathlineto{\pgfqpoint{3.861853in}{4.250000in}}%
\pgfpathlineto{\pgfqpoint{3.855593in}{4.250000in}}%
\pgfpathlineto{\pgfqpoint{3.849332in}{4.250000in}}%
\pgfpathlineto{\pgfqpoint{3.843072in}{4.250000in}}%
\pgfpathlineto{\pgfqpoint{3.836811in}{4.250000in}}%
\pgfpathlineto{\pgfqpoint{3.830551in}{4.250000in}}%
\pgfpathlineto{\pgfqpoint{3.824290in}{4.250000in}}%
\pgfpathlineto{\pgfqpoint{3.818030in}{4.250000in}}%
\pgfpathlineto{\pgfqpoint{3.811770in}{4.250000in}}%
\pgfpathlineto{\pgfqpoint{3.805509in}{4.250000in}}%
\pgfpathlineto{\pgfqpoint{3.799249in}{4.250000in}}%
\pgfpathlineto{\pgfqpoint{3.792988in}{4.250000in}}%
\pgfpathlineto{\pgfqpoint{3.786728in}{4.250000in}}%
\pgfpathlineto{\pgfqpoint{3.780467in}{4.250000in}}%
\pgfpathlineto{\pgfqpoint{3.774207in}{4.250000in}}%
\pgfpathlineto{\pgfqpoint{3.767947in}{4.250000in}}%
\pgfpathlineto{\pgfqpoint{3.761686in}{4.250000in}}%
\pgfpathlineto{\pgfqpoint{3.755426in}{4.250000in}}%
\pgfpathlineto{\pgfqpoint{3.749165in}{4.250000in}}%
\pgfpathlineto{\pgfqpoint{3.742905in}{4.250000in}}%
\pgfpathlineto{\pgfqpoint{3.736644in}{4.250000in}}%
\pgfpathlineto{\pgfqpoint{3.730384in}{4.250000in}}%
\pgfpathlineto{\pgfqpoint{3.724124in}{4.250000in}}%
\pgfpathlineto{\pgfqpoint{3.717863in}{4.250000in}}%
\pgfpathlineto{\pgfqpoint{3.711603in}{4.250000in}}%
\pgfpathlineto{\pgfqpoint{3.705342in}{4.250000in}}%
\pgfpathlineto{\pgfqpoint{3.699082in}{4.250000in}}%
\pgfpathlineto{\pgfqpoint{3.692821in}{4.250000in}}%
\pgfpathlineto{\pgfqpoint{3.686561in}{4.250000in}}%
\pgfpathlineto{\pgfqpoint{3.680301in}{4.250000in}}%
\pgfpathlineto{\pgfqpoint{3.674040in}{4.250000in}}%
\pgfpathlineto{\pgfqpoint{3.667780in}{4.250000in}}%
\pgfpathlineto{\pgfqpoint{3.661519in}{4.250000in}}%
\pgfpathlineto{\pgfqpoint{3.655259in}{4.250000in}}%
\pgfpathlineto{\pgfqpoint{3.648998in}{4.250000in}}%
\pgfpathlineto{\pgfqpoint{3.642738in}{4.250000in}}%
\pgfpathlineto{\pgfqpoint{3.636477in}{4.250000in}}%
\pgfpathlineto{\pgfqpoint{3.630217in}{4.250000in}}%
\pgfpathlineto{\pgfqpoint{3.623957in}{4.250000in}}%
\pgfpathlineto{\pgfqpoint{3.617696in}{4.250000in}}%
\pgfpathlineto{\pgfqpoint{3.611436in}{4.250000in}}%
\pgfpathlineto{\pgfqpoint{3.605175in}{4.250000in}}%
\pgfpathlineto{\pgfqpoint{3.598915in}{4.250000in}}%
\pgfpathlineto{\pgfqpoint{3.592654in}{4.250000in}}%
\pgfpathlineto{\pgfqpoint{3.586394in}{4.250000in}}%
\pgfpathlineto{\pgfqpoint{3.580134in}{4.250000in}}%
\pgfpathlineto{\pgfqpoint{3.573873in}{4.250000in}}%
\pgfpathlineto{\pgfqpoint{3.567613in}{4.250000in}}%
\pgfpathlineto{\pgfqpoint{3.561352in}{4.250000in}}%
\pgfpathlineto{\pgfqpoint{3.555092in}{4.250000in}}%
\pgfpathlineto{\pgfqpoint{3.548831in}{4.250000in}}%
\pgfpathlineto{\pgfqpoint{3.542571in}{4.250000in}}%
\pgfpathlineto{\pgfqpoint{3.536311in}{4.250000in}}%
\pgfpathlineto{\pgfqpoint{3.530050in}{4.250000in}}%
\pgfpathlineto{\pgfqpoint{3.523790in}{4.250000in}}%
\pgfpathlineto{\pgfqpoint{3.517529in}{4.250000in}}%
\pgfpathlineto{\pgfqpoint{3.511269in}{4.250000in}}%
\pgfpathlineto{\pgfqpoint{3.505008in}{4.250000in}}%
\pgfpathlineto{\pgfqpoint{3.498748in}{4.250000in}}%
\pgfpathlineto{\pgfqpoint{3.492487in}{4.250000in}}%
\pgfpathlineto{\pgfqpoint{3.486227in}{4.250000in}}%
\pgfpathlineto{\pgfqpoint{3.479967in}{4.250000in}}%
\pgfpathlineto{\pgfqpoint{3.473706in}{4.250000in}}%
\pgfpathlineto{\pgfqpoint{3.467446in}{4.250000in}}%
\pgfpathlineto{\pgfqpoint{3.461185in}{4.250000in}}%
\pgfpathlineto{\pgfqpoint{3.454925in}{4.250000in}}%
\pgfpathlineto{\pgfqpoint{3.448664in}{4.250000in}}%
\pgfpathlineto{\pgfqpoint{3.442404in}{4.250000in}}%
\pgfpathlineto{\pgfqpoint{3.436144in}{4.250000in}}%
\pgfpathlineto{\pgfqpoint{3.429883in}{4.250000in}}%
\pgfpathlineto{\pgfqpoint{3.423623in}{4.250000in}}%
\pgfpathlineto{\pgfqpoint{3.417362in}{4.250000in}}%
\pgfpathlineto{\pgfqpoint{3.411102in}{4.250000in}}%
\pgfpathlineto{\pgfqpoint{3.404841in}{4.250000in}}%
\pgfpathlineto{\pgfqpoint{3.398581in}{4.250000in}}%
\pgfpathlineto{\pgfqpoint{3.392321in}{4.250000in}}%
\pgfpathlineto{\pgfqpoint{3.386060in}{4.250000in}}%
\pgfpathlineto{\pgfqpoint{3.379800in}{4.250000in}}%
\pgfpathlineto{\pgfqpoint{3.373539in}{4.250000in}}%
\pgfpathlineto{\pgfqpoint{3.367279in}{4.250000in}}%
\pgfpathlineto{\pgfqpoint{3.361018in}{4.250000in}}%
\pgfpathlineto{\pgfqpoint{3.354758in}{4.250000in}}%
\pgfpathlineto{\pgfqpoint{3.348497in}{4.250000in}}%
\pgfpathlineto{\pgfqpoint{3.342237in}{4.250000in}}%
\pgfpathlineto{\pgfqpoint{3.335977in}{4.250000in}}%
\pgfpathlineto{\pgfqpoint{3.329716in}{4.250000in}}%
\pgfpathlineto{\pgfqpoint{3.323456in}{4.250000in}}%
\pgfpathlineto{\pgfqpoint{3.317195in}{4.250000in}}%
\pgfpathlineto{\pgfqpoint{3.310935in}{4.250000in}}%
\pgfpathlineto{\pgfqpoint{3.304674in}{4.250000in}}%
\pgfpathlineto{\pgfqpoint{3.298414in}{4.250000in}}%
\pgfpathlineto{\pgfqpoint{3.292154in}{4.250000in}}%
\pgfpathlineto{\pgfqpoint{3.285893in}{4.250000in}}%
\pgfpathlineto{\pgfqpoint{3.279633in}{4.250000in}}%
\pgfpathlineto{\pgfqpoint{3.273372in}{4.250000in}}%
\pgfpathlineto{\pgfqpoint{3.267112in}{4.250000in}}%
\pgfpathlineto{\pgfqpoint{3.260851in}{4.250000in}}%
\pgfpathlineto{\pgfqpoint{3.254591in}{4.250000in}}%
\pgfpathlineto{\pgfqpoint{3.248331in}{4.250000in}}%
\pgfpathlineto{\pgfqpoint{3.242070in}{4.250000in}}%
\pgfpathlineto{\pgfqpoint{3.235810in}{4.250000in}}%
\pgfpathlineto{\pgfqpoint{3.229549in}{4.250000in}}%
\pgfpathlineto{\pgfqpoint{3.223289in}{4.250000in}}%
\pgfpathlineto{\pgfqpoint{3.217028in}{4.250000in}}%
\pgfpathlineto{\pgfqpoint{3.210768in}{4.250000in}}%
\pgfpathlineto{\pgfqpoint{3.204508in}{4.250000in}}%
\pgfpathlineto{\pgfqpoint{3.198247in}{4.250000in}}%
\pgfpathlineto{\pgfqpoint{3.191987in}{4.250000in}}%
\pgfpathlineto{\pgfqpoint{3.185726in}{4.250000in}}%
\pgfpathlineto{\pgfqpoint{3.179466in}{4.250000in}}%
\pgfpathlineto{\pgfqpoint{3.173205in}{4.250000in}}%
\pgfpathlineto{\pgfqpoint{3.166945in}{4.250000in}}%
\pgfpathlineto{\pgfqpoint{3.160684in}{4.250000in}}%
\pgfpathlineto{\pgfqpoint{3.154424in}{4.250000in}}%
\pgfpathlineto{\pgfqpoint{3.148164in}{4.250000in}}%
\pgfpathlineto{\pgfqpoint{3.141903in}{4.250000in}}%
\pgfpathlineto{\pgfqpoint{3.135643in}{4.250000in}}%
\pgfpathlineto{\pgfqpoint{3.129382in}{4.250000in}}%
\pgfpathlineto{\pgfqpoint{3.123122in}{4.250000in}}%
\pgfpathlineto{\pgfqpoint{3.116861in}{4.250000in}}%
\pgfpathlineto{\pgfqpoint{3.110601in}{4.250000in}}%
\pgfpathlineto{\pgfqpoint{3.104341in}{4.250000in}}%
\pgfpathlineto{\pgfqpoint{3.098080in}{4.250000in}}%
\pgfpathlineto{\pgfqpoint{3.091820in}{4.250000in}}%
\pgfpathlineto{\pgfqpoint{3.085559in}{4.250000in}}%
\pgfpathlineto{\pgfqpoint{3.079299in}{4.250000in}}%
\pgfpathlineto{\pgfqpoint{3.073038in}{4.250000in}}%
\pgfpathlineto{\pgfqpoint{3.066778in}{4.250000in}}%
\pgfpathlineto{\pgfqpoint{3.060518in}{4.250000in}}%
\pgfpathlineto{\pgfqpoint{3.054257in}{4.250000in}}%
\pgfpathlineto{\pgfqpoint{3.047997in}{4.250000in}}%
\pgfpathlineto{\pgfqpoint{3.041736in}{4.250000in}}%
\pgfpathlineto{\pgfqpoint{3.035476in}{4.250000in}}%
\pgfpathlineto{\pgfqpoint{3.029215in}{4.250000in}}%
\pgfpathlineto{\pgfqpoint{3.022955in}{4.250000in}}%
\pgfpathlineto{\pgfqpoint{3.016694in}{4.250000in}}%
\pgfpathlineto{\pgfqpoint{3.010434in}{4.250000in}}%
\pgfpathlineto{\pgfqpoint{3.004174in}{4.250000in}}%
\pgfpathlineto{\pgfqpoint{2.997913in}{4.250000in}}%
\pgfpathlineto{\pgfqpoint{2.991653in}{4.250000in}}%
\pgfpathlineto{\pgfqpoint{2.985392in}{4.250000in}}%
\pgfpathlineto{\pgfqpoint{2.979132in}{4.250000in}}%
\pgfpathlineto{\pgfqpoint{2.972871in}{4.250000in}}%
\pgfpathlineto{\pgfqpoint{2.966611in}{4.250000in}}%
\pgfpathlineto{\pgfqpoint{2.960351in}{4.250000in}}%
\pgfpathlineto{\pgfqpoint{2.954090in}{4.250000in}}%
\pgfpathlineto{\pgfqpoint{2.947830in}{4.250000in}}%
\pgfpathlineto{\pgfqpoint{2.941569in}{4.250000in}}%
\pgfpathlineto{\pgfqpoint{2.935309in}{4.250000in}}%
\pgfpathlineto{\pgfqpoint{2.929048in}{4.250000in}}%
\pgfpathlineto{\pgfqpoint{2.922788in}{4.250000in}}%
\pgfpathlineto{\pgfqpoint{2.916528in}{4.250000in}}%
\pgfpathlineto{\pgfqpoint{2.910267in}{4.250000in}}%
\pgfpathlineto{\pgfqpoint{2.904007in}{4.250000in}}%
\pgfpathlineto{\pgfqpoint{2.897746in}{4.250000in}}%
\pgfpathlineto{\pgfqpoint{2.891486in}{4.250000in}}%
\pgfpathlineto{\pgfqpoint{2.885225in}{4.250000in}}%
\pgfpathlineto{\pgfqpoint{2.878965in}{4.250000in}}%
\pgfpathlineto{\pgfqpoint{2.872705in}{4.250000in}}%
\pgfpathlineto{\pgfqpoint{2.866444in}{4.250000in}}%
\pgfpathlineto{\pgfqpoint{2.860184in}{4.250000in}}%
\pgfpathlineto{\pgfqpoint{2.853923in}{4.250000in}}%
\pgfpathlineto{\pgfqpoint{2.847663in}{4.250000in}}%
\pgfpathlineto{\pgfqpoint{2.841402in}{4.250000in}}%
\pgfpathlineto{\pgfqpoint{2.835142in}{4.250000in}}%
\pgfpathlineto{\pgfqpoint{2.828881in}{4.250000in}}%
\pgfpathlineto{\pgfqpoint{2.822621in}{4.250000in}}%
\pgfpathlineto{\pgfqpoint{2.816361in}{4.250000in}}%
\pgfpathlineto{\pgfqpoint{2.810100in}{4.250000in}}%
\pgfpathlineto{\pgfqpoint{2.803840in}{4.250000in}}%
\pgfpathlineto{\pgfqpoint{2.797579in}{4.250000in}}%
\pgfpathlineto{\pgfqpoint{2.791319in}{4.250000in}}%
\pgfpathlineto{\pgfqpoint{2.785058in}{4.250000in}}%
\pgfpathlineto{\pgfqpoint{2.778798in}{4.250000in}}%
\pgfpathlineto{\pgfqpoint{2.772538in}{4.250000in}}%
\pgfpathlineto{\pgfqpoint{2.766277in}{4.250000in}}%
\pgfpathlineto{\pgfqpoint{2.760017in}{4.250000in}}%
\pgfpathlineto{\pgfqpoint{2.753756in}{4.250000in}}%
\pgfpathlineto{\pgfqpoint{2.747496in}{4.250000in}}%
\pgfpathlineto{\pgfqpoint{2.741235in}{4.250000in}}%
\pgfpathlineto{\pgfqpoint{2.734975in}{4.250000in}}%
\pgfpathlineto{\pgfqpoint{2.728715in}{4.250000in}}%
\pgfpathlineto{\pgfqpoint{2.722454in}{4.250000in}}%
\pgfpathlineto{\pgfqpoint{2.716194in}{4.250000in}}%
\pgfpathlineto{\pgfqpoint{2.709933in}{4.250000in}}%
\pgfpathlineto{\pgfqpoint{2.703673in}{4.250000in}}%
\pgfpathlineto{\pgfqpoint{2.697412in}{4.250000in}}%
\pgfpathlineto{\pgfqpoint{2.691152in}{4.250000in}}%
\pgfpathlineto{\pgfqpoint{2.684891in}{4.250000in}}%
\pgfpathlineto{\pgfqpoint{2.678631in}{4.250000in}}%
\pgfpathlineto{\pgfqpoint{2.672371in}{4.250000in}}%
\pgfpathlineto{\pgfqpoint{2.666110in}{4.250000in}}%
\pgfpathlineto{\pgfqpoint{2.659850in}{4.250000in}}%
\pgfpathlineto{\pgfqpoint{2.653589in}{4.250000in}}%
\pgfpathlineto{\pgfqpoint{2.647329in}{4.250000in}}%
\pgfpathlineto{\pgfqpoint{2.641068in}{4.250000in}}%
\pgfpathlineto{\pgfqpoint{2.634808in}{4.250000in}}%
\pgfpathlineto{\pgfqpoint{2.628548in}{4.250000in}}%
\pgfpathlineto{\pgfqpoint{2.622287in}{4.250000in}}%
\pgfpathlineto{\pgfqpoint{2.616027in}{4.250000in}}%
\pgfpathlineto{\pgfqpoint{2.609766in}{4.250000in}}%
\pgfpathlineto{\pgfqpoint{2.603506in}{4.250000in}}%
\pgfpathlineto{\pgfqpoint{2.597245in}{4.250000in}}%
\pgfpathlineto{\pgfqpoint{2.590985in}{4.250000in}}%
\pgfpathlineto{\pgfqpoint{2.584725in}{4.250000in}}%
\pgfpathlineto{\pgfqpoint{2.578464in}{4.250000in}}%
\pgfpathlineto{\pgfqpoint{2.572204in}{4.250000in}}%
\pgfpathlineto{\pgfqpoint{2.565943in}{4.250000in}}%
\pgfpathlineto{\pgfqpoint{2.559683in}{4.250000in}}%
\pgfpathlineto{\pgfqpoint{2.553422in}{4.250000in}}%
\pgfpathlineto{\pgfqpoint{2.547162in}{4.250000in}}%
\pgfpathlineto{\pgfqpoint{2.540902in}{4.250000in}}%
\pgfpathlineto{\pgfqpoint{2.534641in}{4.250000in}}%
\pgfpathlineto{\pgfqpoint{2.528381in}{4.250000in}}%
\pgfpathlineto{\pgfqpoint{2.522120in}{4.250000in}}%
\pgfpathlineto{\pgfqpoint{2.515860in}{4.250000in}}%
\pgfpathlineto{\pgfqpoint{2.509599in}{4.250000in}}%
\pgfpathlineto{\pgfqpoint{2.503339in}{4.250000in}}%
\pgfpathlineto{\pgfqpoint{2.497078in}{4.250000in}}%
\pgfpathlineto{\pgfqpoint{2.490818in}{4.250000in}}%
\pgfpathlineto{\pgfqpoint{2.484558in}{4.250000in}}%
\pgfpathlineto{\pgfqpoint{2.478297in}{4.250000in}}%
\pgfpathlineto{\pgfqpoint{2.472037in}{4.250000in}}%
\pgfpathlineto{\pgfqpoint{2.465776in}{4.250000in}}%
\pgfpathlineto{\pgfqpoint{2.459516in}{4.250000in}}%
\pgfpathlineto{\pgfqpoint{2.453255in}{4.250000in}}%
\pgfpathlineto{\pgfqpoint{2.446995in}{4.250000in}}%
\pgfpathlineto{\pgfqpoint{2.440735in}{4.250000in}}%
\pgfpathlineto{\pgfqpoint{2.434474in}{4.250000in}}%
\pgfpathlineto{\pgfqpoint{2.428214in}{4.250000in}}%
\pgfpathlineto{\pgfqpoint{2.421953in}{4.250000in}}%
\pgfpathlineto{\pgfqpoint{2.415693in}{4.250000in}}%
\pgfpathlineto{\pgfqpoint{2.409432in}{4.250000in}}%
\pgfpathlineto{\pgfqpoint{2.403172in}{4.250000in}}%
\pgfpathlineto{\pgfqpoint{2.396912in}{4.250000in}}%
\pgfpathlineto{\pgfqpoint{2.390651in}{4.250000in}}%
\pgfpathlineto{\pgfqpoint{2.384391in}{4.250000in}}%
\pgfpathlineto{\pgfqpoint{2.378130in}{4.250000in}}%
\pgfpathlineto{\pgfqpoint{2.371870in}{4.250000in}}%
\pgfpathlineto{\pgfqpoint{2.365609in}{4.250000in}}%
\pgfpathlineto{\pgfqpoint{2.359349in}{4.250000in}}%
\pgfpathlineto{\pgfqpoint{2.353088in}{4.250000in}}%
\pgfpathlineto{\pgfqpoint{2.346828in}{4.250000in}}%
\pgfpathlineto{\pgfqpoint{2.340568in}{4.250000in}}%
\pgfpathlineto{\pgfqpoint{2.334307in}{4.250000in}}%
\pgfpathlineto{\pgfqpoint{2.328047in}{4.250000in}}%
\pgfpathlineto{\pgfqpoint{2.321786in}{4.250000in}}%
\pgfpathlineto{\pgfqpoint{2.315526in}{4.250000in}}%
\pgfpathlineto{\pgfqpoint{2.309265in}{4.250000in}}%
\pgfpathlineto{\pgfqpoint{2.303005in}{4.250000in}}%
\pgfpathlineto{\pgfqpoint{2.296745in}{4.250000in}}%
\pgfpathlineto{\pgfqpoint{2.290484in}{4.250000in}}%
\pgfpathlineto{\pgfqpoint{2.284224in}{4.250000in}}%
\pgfpathlineto{\pgfqpoint{2.277963in}{4.250000in}}%
\pgfpathlineto{\pgfqpoint{2.271703in}{4.250000in}}%
\pgfpathlineto{\pgfqpoint{2.265442in}{4.250000in}}%
\pgfpathlineto{\pgfqpoint{2.259182in}{4.250000in}}%
\pgfpathlineto{\pgfqpoint{2.252922in}{4.250000in}}%
\pgfpathlineto{\pgfqpoint{2.246661in}{4.250000in}}%
\pgfpathlineto{\pgfqpoint{2.240401in}{4.250000in}}%
\pgfpathlineto{\pgfqpoint{2.234140in}{4.250000in}}%
\pgfpathlineto{\pgfqpoint{2.227880in}{4.250000in}}%
\pgfpathlineto{\pgfqpoint{2.221619in}{4.250000in}}%
\pgfpathlineto{\pgfqpoint{2.215359in}{4.250000in}}%
\pgfpathlineto{\pgfqpoint{2.209098in}{4.250000in}}%
\pgfpathlineto{\pgfqpoint{2.202838in}{4.250000in}}%
\pgfpathlineto{\pgfqpoint{2.196578in}{4.250000in}}%
\pgfpathlineto{\pgfqpoint{2.190317in}{4.250000in}}%
\pgfpathlineto{\pgfqpoint{2.184057in}{4.250000in}}%
\pgfpathlineto{\pgfqpoint{2.177796in}{4.250000in}}%
\pgfpathlineto{\pgfqpoint{2.171536in}{4.250000in}}%
\pgfpathlineto{\pgfqpoint{2.165275in}{4.250000in}}%
\pgfpathlineto{\pgfqpoint{2.159015in}{4.250000in}}%
\pgfpathlineto{\pgfqpoint{2.152755in}{4.250000in}}%
\pgfpathlineto{\pgfqpoint{2.146494in}{4.250000in}}%
\pgfpathlineto{\pgfqpoint{2.140234in}{4.250000in}}%
\pgfpathlineto{\pgfqpoint{2.133973in}{4.250000in}}%
\pgfpathlineto{\pgfqpoint{2.127713in}{4.250000in}}%
\pgfpathlineto{\pgfqpoint{2.121452in}{4.250000in}}%
\pgfpathlineto{\pgfqpoint{2.115192in}{4.250000in}}%
\pgfpathlineto{\pgfqpoint{2.108932in}{4.250000in}}%
\pgfpathlineto{\pgfqpoint{2.102671in}{4.250000in}}%
\pgfpathlineto{\pgfqpoint{2.096411in}{4.250000in}}%
\pgfpathlineto{\pgfqpoint{2.090150in}{4.250000in}}%
\pgfpathlineto{\pgfqpoint{2.083890in}{4.250000in}}%
\pgfpathlineto{\pgfqpoint{2.077629in}{4.250000in}}%
\pgfpathlineto{\pgfqpoint{2.071369in}{4.250000in}}%
\pgfpathlineto{\pgfqpoint{2.065109in}{4.250000in}}%
\pgfpathlineto{\pgfqpoint{2.058848in}{4.250000in}}%
\pgfpathlineto{\pgfqpoint{2.052588in}{4.250000in}}%
\pgfpathlineto{\pgfqpoint{2.046327in}{4.250000in}}%
\pgfpathlineto{\pgfqpoint{2.040067in}{4.250000in}}%
\pgfpathlineto{\pgfqpoint{2.033806in}{4.250000in}}%
\pgfpathlineto{\pgfqpoint{2.027546in}{4.250000in}}%
\pgfpathlineto{\pgfqpoint{2.021285in}{4.250000in}}%
\pgfpathlineto{\pgfqpoint{2.015025in}{4.250000in}}%
\pgfpathlineto{\pgfqpoint{2.008765in}{4.250000in}}%
\pgfpathlineto{\pgfqpoint{2.002504in}{4.250000in}}%
\pgfpathlineto{\pgfqpoint{1.996244in}{4.250000in}}%
\pgfpathlineto{\pgfqpoint{1.989983in}{4.250000in}}%
\pgfpathlineto{\pgfqpoint{1.983723in}{4.250000in}}%
\pgfpathlineto{\pgfqpoint{1.977462in}{4.250000in}}%
\pgfpathlineto{\pgfqpoint{1.971202in}{4.250000in}}%
\pgfpathlineto{\pgfqpoint{1.964942in}{4.250000in}}%
\pgfpathlineto{\pgfqpoint{1.958681in}{4.250000in}}%
\pgfpathlineto{\pgfqpoint{1.952421in}{4.250000in}}%
\pgfpathlineto{\pgfqpoint{1.946160in}{4.250000in}}%
\pgfpathlineto{\pgfqpoint{1.939900in}{4.250000in}}%
\pgfpathlineto{\pgfqpoint{1.933639in}{4.250000in}}%
\pgfpathlineto{\pgfqpoint{1.927379in}{4.250000in}}%
\pgfpathlineto{\pgfqpoint{1.921119in}{4.250000in}}%
\pgfpathlineto{\pgfqpoint{1.914858in}{4.250000in}}%
\pgfpathlineto{\pgfqpoint{1.908598in}{4.250000in}}%
\pgfpathlineto{\pgfqpoint{1.902337in}{4.250000in}}%
\pgfpathlineto{\pgfqpoint{1.896077in}{4.250000in}}%
\pgfpathlineto{\pgfqpoint{1.889816in}{4.250000in}}%
\pgfpathlineto{\pgfqpoint{1.883556in}{4.250000in}}%
\pgfpathlineto{\pgfqpoint{1.877295in}{4.250000in}}%
\pgfpathlineto{\pgfqpoint{1.871035in}{4.250000in}}%
\pgfpathlineto{\pgfqpoint{1.864775in}{4.250000in}}%
\pgfpathlineto{\pgfqpoint{1.858514in}{4.250000in}}%
\pgfpathlineto{\pgfqpoint{1.852254in}{4.250000in}}%
\pgfpathlineto{\pgfqpoint{1.845993in}{4.250000in}}%
\pgfpathlineto{\pgfqpoint{1.839733in}{4.250000in}}%
\pgfpathlineto{\pgfqpoint{1.833472in}{4.250000in}}%
\pgfpathlineto{\pgfqpoint{1.827212in}{4.250000in}}%
\pgfpathlineto{\pgfqpoint{1.820952in}{4.250000in}}%
\pgfpathlineto{\pgfqpoint{1.814691in}{4.250000in}}%
\pgfpathlineto{\pgfqpoint{1.808431in}{4.250000in}}%
\pgfpathlineto{\pgfqpoint{1.802170in}{4.250000in}}%
\pgfpathlineto{\pgfqpoint{1.795910in}{4.250000in}}%
\pgfpathlineto{\pgfqpoint{1.789649in}{4.250000in}}%
\pgfpathlineto{\pgfqpoint{1.783389in}{4.250000in}}%
\pgfpathlineto{\pgfqpoint{1.777129in}{4.250000in}}%
\pgfpathlineto{\pgfqpoint{1.770868in}{4.250000in}}%
\pgfpathlineto{\pgfqpoint{1.764608in}{4.250000in}}%
\pgfpathlineto{\pgfqpoint{1.758347in}{4.250000in}}%
\pgfpathlineto{\pgfqpoint{1.752087in}{4.250000in}}%
\pgfpathlineto{\pgfqpoint{1.745826in}{4.250000in}}%
\pgfpathlineto{\pgfqpoint{1.739566in}{4.250000in}}%
\pgfpathlineto{\pgfqpoint{1.733306in}{4.250000in}}%
\pgfpathlineto{\pgfqpoint{1.727045in}{4.250000in}}%
\pgfpathlineto{\pgfqpoint{1.720785in}{4.250000in}}%
\pgfpathlineto{\pgfqpoint{1.714524in}{4.250000in}}%
\pgfpathlineto{\pgfqpoint{1.708264in}{4.250000in}}%
\pgfpathlineto{\pgfqpoint{1.702003in}{4.250000in}}%
\pgfpathlineto{\pgfqpoint{1.695743in}{4.250000in}}%
\pgfpathlineto{\pgfqpoint{1.689482in}{4.250000in}}%
\pgfpathlineto{\pgfqpoint{1.683222in}{4.250000in}}%
\pgfpathlineto{\pgfqpoint{1.676962in}{4.250000in}}%
\pgfpathlineto{\pgfqpoint{1.670701in}{4.250000in}}%
\pgfpathlineto{\pgfqpoint{1.664441in}{4.250000in}}%
\pgfpathlineto{\pgfqpoint{1.658180in}{4.250000in}}%
\pgfpathlineto{\pgfqpoint{1.651920in}{4.250000in}}%
\pgfpathlineto{\pgfqpoint{1.645659in}{4.250000in}}%
\pgfpathlineto{\pgfqpoint{1.639399in}{4.250000in}}%
\pgfpathlineto{\pgfqpoint{1.633139in}{4.250000in}}%
\pgfpathlineto{\pgfqpoint{1.626878in}{4.250000in}}%
\pgfpathlineto{\pgfqpoint{1.620618in}{4.250000in}}%
\pgfpathlineto{\pgfqpoint{1.614357in}{4.250000in}}%
\pgfpathlineto{\pgfqpoint{1.608097in}{4.250000in}}%
\pgfpathlineto{\pgfqpoint{1.601836in}{4.250000in}}%
\pgfpathlineto{\pgfqpoint{1.595576in}{4.250000in}}%
\pgfpathlineto{\pgfqpoint{1.589316in}{4.250000in}}%
\pgfpathlineto{\pgfqpoint{1.583055in}{4.250000in}}%
\pgfpathlineto{\pgfqpoint{1.576795in}{4.250000in}}%
\pgfpathlineto{\pgfqpoint{1.570534in}{4.250000in}}%
\pgfpathlineto{\pgfqpoint{1.564274in}{4.250000in}}%
\pgfpathlineto{\pgfqpoint{1.558013in}{4.250000in}}%
\pgfpathlineto{\pgfqpoint{1.551753in}{4.250000in}}%
\pgfpathlineto{\pgfqpoint{1.545492in}{4.250000in}}%
\pgfpathlineto{\pgfqpoint{1.539232in}{4.250000in}}%
\pgfpathlineto{\pgfqpoint{1.532972in}{4.250000in}}%
\pgfpathlineto{\pgfqpoint{1.526711in}{4.250000in}}%
\pgfpathlineto{\pgfqpoint{1.520451in}{4.250000in}}%
\pgfpathlineto{\pgfqpoint{1.514190in}{4.250000in}}%
\pgfpathlineto{\pgfqpoint{1.507930in}{4.250000in}}%
\pgfpathlineto{\pgfqpoint{1.501669in}{4.250000in}}%
\pgfpathlineto{\pgfqpoint{1.495409in}{4.250000in}}%
\pgfpathlineto{\pgfqpoint{1.489149in}{4.250000in}}%
\pgfpathlineto{\pgfqpoint{1.482888in}{4.250000in}}%
\pgfpathlineto{\pgfqpoint{1.476628in}{4.250000in}}%
\pgfpathlineto{\pgfqpoint{1.470367in}{4.250000in}}%
\pgfpathlineto{\pgfqpoint{1.464107in}{4.250000in}}%
\pgfpathlineto{\pgfqpoint{1.457846in}{4.250000in}}%
\pgfpathlineto{\pgfqpoint{1.451586in}{4.250000in}}%
\pgfpathlineto{\pgfqpoint{1.445326in}{4.250000in}}%
\pgfpathlineto{\pgfqpoint{1.439065in}{4.250000in}}%
\pgfpathlineto{\pgfqpoint{1.432805in}{4.250000in}}%
\pgfpathlineto{\pgfqpoint{1.426544in}{4.250000in}}%
\pgfpathlineto{\pgfqpoint{1.420284in}{4.250000in}}%
\pgfpathlineto{\pgfqpoint{1.414023in}{4.250000in}}%
\pgfpathlineto{\pgfqpoint{1.407763in}{4.250000in}}%
\pgfpathlineto{\pgfqpoint{1.401503in}{4.250000in}}%
\pgfpathlineto{\pgfqpoint{1.395242in}{4.250000in}}%
\pgfpathlineto{\pgfqpoint{1.388982in}{4.250000in}}%
\pgfpathlineto{\pgfqpoint{1.382721in}{4.250000in}}%
\pgfpathlineto{\pgfqpoint{1.376461in}{4.250000in}}%
\pgfpathlineto{\pgfqpoint{1.370200in}{4.250000in}}%
\pgfpathlineto{\pgfqpoint{1.363940in}{4.250000in}}%
\pgfpathlineto{\pgfqpoint{1.357679in}{4.250000in}}%
\pgfpathlineto{\pgfqpoint{1.351419in}{4.250000in}}%
\pgfpathlineto{\pgfqpoint{1.345159in}{4.250000in}}%
\pgfpathlineto{\pgfqpoint{1.338898in}{4.250000in}}%
\pgfpathlineto{\pgfqpoint{1.332638in}{4.250000in}}%
\pgfpathlineto{\pgfqpoint{1.326377in}{4.250000in}}%
\pgfpathlineto{\pgfqpoint{1.320117in}{4.250000in}}%
\pgfpathlineto{\pgfqpoint{1.313856in}{4.250000in}}%
\pgfpathlineto{\pgfqpoint{1.307596in}{4.250000in}}%
\pgfpathlineto{\pgfqpoint{1.301336in}{4.250000in}}%
\pgfpathlineto{\pgfqpoint{1.295075in}{4.250000in}}%
\pgfpathlineto{\pgfqpoint{1.288815in}{4.250000in}}%
\pgfpathlineto{\pgfqpoint{1.282554in}{4.250000in}}%
\pgfpathlineto{\pgfqpoint{1.276294in}{4.250000in}}%
\pgfpathlineto{\pgfqpoint{1.270033in}{4.250000in}}%
\pgfpathlineto{\pgfqpoint{1.263773in}{4.250000in}}%
\pgfpathlineto{\pgfqpoint{1.257513in}{4.250000in}}%
\pgfpathlineto{\pgfqpoint{1.251252in}{4.250000in}}%
\pgfpathlineto{\pgfqpoint{1.244992in}{4.250000in}}%
\pgfpathlineto{\pgfqpoint{1.238731in}{4.250000in}}%
\pgfpathlineto{\pgfqpoint{1.232471in}{4.250000in}}%
\pgfpathlineto{\pgfqpoint{1.226210in}{4.250000in}}%
\pgfpathlineto{\pgfqpoint{1.219950in}{4.250000in}}%
\pgfpathlineto{\pgfqpoint{1.213689in}{4.250000in}}%
\pgfpathlineto{\pgfqpoint{1.207429in}{4.250000in}}%
\pgfpathlineto{\pgfqpoint{1.201169in}{4.250000in}}%
\pgfpathlineto{\pgfqpoint{1.194908in}{4.250000in}}%
\pgfpathlineto{\pgfqpoint{1.188648in}{4.250000in}}%
\pgfpathlineto{\pgfqpoint{1.182387in}{4.250000in}}%
\pgfpathlineto{\pgfqpoint{1.176127in}{4.250000in}}%
\pgfpathlineto{\pgfqpoint{1.169866in}{4.250000in}}%
\pgfpathlineto{\pgfqpoint{1.163606in}{4.250000in}}%
\pgfpathlineto{\pgfqpoint{1.157346in}{4.250000in}}%
\pgfpathlineto{\pgfqpoint{1.151085in}{4.250000in}}%
\pgfpathlineto{\pgfqpoint{1.144825in}{4.250000in}}%
\pgfpathlineto{\pgfqpoint{1.138564in}{4.250000in}}%
\pgfpathlineto{\pgfqpoint{1.132304in}{4.250000in}}%
\pgfpathlineto{\pgfqpoint{1.126043in}{4.250000in}}%
\pgfpathlineto{\pgfqpoint{1.119783in}{4.250000in}}%
\pgfpathlineto{\pgfqpoint{1.113523in}{4.250000in}}%
\pgfpathlineto{\pgfqpoint{1.107262in}{4.250000in}}%
\pgfpathlineto{\pgfqpoint{1.101002in}{4.250000in}}%
\pgfpathlineto{\pgfqpoint{1.094741in}{4.250000in}}%
\pgfpathlineto{\pgfqpoint{1.088481in}{4.250000in}}%
\pgfpathlineto{\pgfqpoint{1.082220in}{4.250000in}}%
\pgfpathlineto{\pgfqpoint{1.075960in}{4.250000in}}%
\pgfpathlineto{\pgfqpoint{1.069699in}{4.250000in}}%
\pgfpathlineto{\pgfqpoint{1.063439in}{4.250000in}}%
\pgfpathlineto{\pgfqpoint{1.057179in}{4.250000in}}%
\pgfpathlineto{\pgfqpoint{1.050918in}{4.250000in}}%
\pgfpathlineto{\pgfqpoint{1.044658in}{4.250000in}}%
\pgfpathlineto{\pgfqpoint{1.038397in}{4.250000in}}%
\pgfpathlineto{\pgfqpoint{1.032137in}{4.250000in}}%
\pgfpathlineto{\pgfqpoint{1.025876in}{4.250000in}}%
\pgfpathlineto{\pgfqpoint{1.019616in}{4.250000in}}%
\pgfpathlineto{\pgfqpoint{1.013356in}{4.250000in}}%
\pgfpathlineto{\pgfqpoint{1.007095in}{4.250000in}}%
\pgfpathlineto{\pgfqpoint{1.000835in}{4.250000in}}%
\pgfpathlineto{\pgfqpoint{0.994574in}{4.250000in}}%
\pgfpathlineto{\pgfqpoint{0.988314in}{4.250000in}}%
\pgfpathlineto{\pgfqpoint{0.982053in}{4.250000in}}%
\pgfpathlineto{\pgfqpoint{0.975793in}{4.250000in}}%
\pgfpathlineto{\pgfqpoint{0.969533in}{4.250000in}}%
\pgfpathlineto{\pgfqpoint{0.963272in}{4.250000in}}%
\pgfpathlineto{\pgfqpoint{0.957012in}{4.250000in}}%
\pgfpathlineto{\pgfqpoint{0.950751in}{4.250000in}}%
\pgfpathlineto{\pgfqpoint{0.944491in}{4.250000in}}%
\pgfpathlineto{\pgfqpoint{0.938230in}{4.250000in}}%
\pgfpathlineto{\pgfqpoint{0.931970in}{4.250000in}}%
\pgfpathlineto{\pgfqpoint{0.925710in}{4.250000in}}%
\pgfpathlineto{\pgfqpoint{0.919449in}{4.250000in}}%
\pgfpathlineto{\pgfqpoint{0.913189in}{4.250000in}}%
\pgfpathlineto{\pgfqpoint{0.906928in}{4.250000in}}%
\pgfpathlineto{\pgfqpoint{0.900668in}{4.250000in}}%
\pgfpathlineto{\pgfqpoint{0.894407in}{4.250000in}}%
\pgfpathlineto{\pgfqpoint{0.888147in}{4.250000in}}%
\pgfpathlineto{\pgfqpoint{0.881886in}{4.250000in}}%
\pgfpathlineto{\pgfqpoint{0.875626in}{4.250000in}}%
\pgfpathlineto{\pgfqpoint{0.869366in}{4.250000in}}%
\pgfpathlineto{\pgfqpoint{0.863105in}{4.250000in}}%
\pgfpathlineto{\pgfqpoint{0.856845in}{4.250000in}}%
\pgfpathlineto{\pgfqpoint{0.850584in}{4.250000in}}%
\pgfpathlineto{\pgfqpoint{0.844324in}{4.250000in}}%
\pgfpathlineto{\pgfqpoint{0.838063in}{4.250000in}}%
\pgfpathlineto{\pgfqpoint{0.831803in}{4.250000in}}%
\pgfpathlineto{\pgfqpoint{0.825543in}{4.250000in}}%
\pgfpathlineto{\pgfqpoint{0.819282in}{4.250000in}}%
\pgfpathlineto{\pgfqpoint{0.813022in}{4.250000in}}%
\pgfpathlineto{\pgfqpoint{0.806761in}{4.250000in}}%
\pgfpathlineto{\pgfqpoint{0.800501in}{4.250000in}}%
\pgfpathlineto{\pgfqpoint{0.794240in}{4.250000in}}%
\pgfpathlineto{\pgfqpoint{0.787980in}{4.250000in}}%
\pgfpathlineto{\pgfqpoint{0.781720in}{4.250000in}}%
\pgfpathlineto{\pgfqpoint{0.775459in}{4.250000in}}%
\pgfpathlineto{\pgfqpoint{0.769199in}{4.250000in}}%
\pgfpathlineto{\pgfqpoint{0.762938in}{4.250000in}}%
\pgfpathlineto{\pgfqpoint{0.756678in}{4.250000in}}%
\pgfpathlineto{\pgfqpoint{0.750417in}{4.250000in}}%
\pgfpathlineto{\pgfqpoint{0.744157in}{4.250000in}}%
\pgfpathlineto{\pgfqpoint{0.737896in}{4.250000in}}%
\pgfpathlineto{\pgfqpoint{0.731636in}{4.250000in}}%
\pgfpathlineto{\pgfqpoint{0.725376in}{4.250000in}}%
\pgfpathlineto{\pgfqpoint{0.719115in}{4.250000in}}%
\pgfpathlineto{\pgfqpoint{0.712855in}{4.250000in}}%
\pgfpathlineto{\pgfqpoint{0.706594in}{4.250000in}}%
\pgfpathlineto{\pgfqpoint{0.700334in}{4.250000in}}%
\pgfpathlineto{\pgfqpoint{0.694073in}{4.250000in}}%
\pgfpathlineto{\pgfqpoint{0.687813in}{4.250000in}}%
\pgfpathlineto{\pgfqpoint{0.681553in}{4.250000in}}%
\pgfpathlineto{\pgfqpoint{0.675292in}{4.250000in}}%
\pgfpathlineto{\pgfqpoint{0.669032in}{4.250000in}}%
\pgfpathlineto{\pgfqpoint{0.662771in}{4.250000in}}%
\pgfpathlineto{\pgfqpoint{0.656511in}{4.250000in}}%
\pgfpathlineto{\pgfqpoint{0.650250in}{4.250000in}}%
\pgfpathlineto{\pgfqpoint{0.643990in}{4.250000in}}%
\pgfpathlineto{\pgfqpoint{0.637730in}{4.250000in}}%
\pgfpathlineto{\pgfqpoint{0.631469in}{4.250000in}}%
\pgfpathlineto{\pgfqpoint{0.625209in}{4.250000in}}%
\pgfpathlineto{\pgfqpoint{0.618948in}{4.250000in}}%
\pgfpathlineto{\pgfqpoint{0.612688in}{4.250000in}}%
\pgfpathlineto{\pgfqpoint{0.606427in}{4.250000in}}%
\pgfpathlineto{\pgfqpoint{0.600167in}{4.250000in}}%
\pgfpathlineto{\pgfqpoint{0.593907in}{4.250000in}}%
\pgfpathlineto{\pgfqpoint{0.587646in}{4.250000in}}%
\pgfpathlineto{\pgfqpoint{0.581386in}{4.250000in}}%
\pgfpathlineto{\pgfqpoint{0.575125in}{4.250000in}}%
\pgfpathlineto{\pgfqpoint{0.568865in}{4.250000in}}%
\pgfpathlineto{\pgfqpoint{0.562604in}{4.250000in}}%
\pgfpathlineto{\pgfqpoint{0.556344in}{4.250000in}}%
\pgfpathlineto{\pgfqpoint{0.550083in}{4.250000in}}%
\pgfpathlineto{\pgfqpoint{0.543823in}{4.250000in}}%
\pgfpathlineto{\pgfqpoint{0.537563in}{4.250000in}}%
\pgfpathlineto{\pgfqpoint{0.531302in}{4.250000in}}%
\pgfpathlineto{\pgfqpoint{0.525042in}{4.250000in}}%
\pgfpathlineto{\pgfqpoint{0.518781in}{4.250000in}}%
\pgfpathlineto{\pgfqpoint{0.512521in}{4.250000in}}%
\pgfpathlineto{\pgfqpoint{0.506260in}{4.250000in}}%
\pgfpathlineto{\pgfqpoint{0.500000in}{4.250000in}}%
\pgfpathclose%
\pgfusepath{stroke,fill}%
}%
\begin{pgfscope}%
\pgfsys@transformshift{0.000000in}{0.000000in}%
\pgfsys@useobject{currentmarker}{}%
\end{pgfscope}%
\end{pgfscope}%
\begin{pgfscope}%
\pgfpathrectangle{\pgfqpoint{0.500000in}{4.250000in}}{\pgfqpoint{3.750000in}{0.600000in}}%
\pgfusepath{clip}%
\pgfsetrectcap%
\pgfsetroundjoin%
\pgfsetlinewidth{2.007500pt}%
\definecolor{currentstroke}{rgb}{0.000000,0.000000,1.000000}%
\pgfsetstrokecolor{currentstroke}%
\pgfsetdash{}{0pt}%
\pgfpathmoveto{\pgfqpoint{0.500000in}{4.250130in}}%
\pgfpathlineto{\pgfqpoint{0.756678in}{4.251593in}}%
\pgfpathlineto{\pgfqpoint{0.875626in}{4.254336in}}%
\pgfpathlineto{\pgfqpoint{0.963272in}{4.258503in}}%
\pgfpathlineto{\pgfqpoint{1.032137in}{4.263891in}}%
\pgfpathlineto{\pgfqpoint{1.094741in}{4.271077in}}%
\pgfpathlineto{\pgfqpoint{1.151085in}{4.279952in}}%
\pgfpathlineto{\pgfqpoint{1.201169in}{4.290166in}}%
\pgfpathlineto{\pgfqpoint{1.244992in}{4.301169in}}%
\pgfpathlineto{\pgfqpoint{1.288815in}{4.314302in}}%
\pgfpathlineto{\pgfqpoint{1.332638in}{4.329710in}}%
\pgfpathlineto{\pgfqpoint{1.376461in}{4.347469in}}%
\pgfpathlineto{\pgfqpoint{1.420284in}{4.367568in}}%
\pgfpathlineto{\pgfqpoint{1.470367in}{4.393247in}}%
\pgfpathlineto{\pgfqpoint{1.520451in}{4.421447in}}%
\pgfpathlineto{\pgfqpoint{1.583055in}{4.459313in}}%
\pgfpathlineto{\pgfqpoint{1.733306in}{4.551583in}}%
\pgfpathlineto{\pgfqpoint{1.777129in}{4.575489in}}%
\pgfpathlineto{\pgfqpoint{1.814691in}{4.593726in}}%
\pgfpathlineto{\pgfqpoint{1.852254in}{4.609360in}}%
\pgfpathlineto{\pgfqpoint{1.883556in}{4.620083in}}%
\pgfpathlineto{\pgfqpoint{1.914858in}{4.628481in}}%
\pgfpathlineto{\pgfqpoint{1.946160in}{4.634381in}}%
\pgfpathlineto{\pgfqpoint{1.977462in}{4.637663in}}%
\pgfpathlineto{\pgfqpoint{2.008765in}{4.638259in}}%
\pgfpathlineto{\pgfqpoint{2.040067in}{4.636158in}}%
\pgfpathlineto{\pgfqpoint{2.071369in}{4.631406in}}%
\pgfpathlineto{\pgfqpoint{2.102671in}{4.624105in}}%
\pgfpathlineto{\pgfqpoint{2.133973in}{4.614412in}}%
\pgfpathlineto{\pgfqpoint{2.165275in}{4.602538in}}%
\pgfpathlineto{\pgfqpoint{2.202838in}{4.585780in}}%
\pgfpathlineto{\pgfqpoint{2.246661in}{4.563472in}}%
\pgfpathlineto{\pgfqpoint{2.309265in}{4.528531in}}%
\pgfpathlineto{\pgfqpoint{2.378130in}{4.490390in}}%
\pgfpathlineto{\pgfqpoint{2.415693in}{4.472211in}}%
\pgfpathlineto{\pgfqpoint{2.446995in}{4.459879in}}%
\pgfpathlineto{\pgfqpoint{2.472037in}{4.452599in}}%
\pgfpathlineto{\pgfqpoint{2.490818in}{4.449024in}}%
\pgfpathlineto{\pgfqpoint{2.509599in}{4.447344in}}%
\pgfpathlineto{\pgfqpoint{2.528381in}{4.447811in}}%
\pgfpathlineto{\pgfqpoint{2.547162in}{4.450665in}}%
\pgfpathlineto{\pgfqpoint{2.565943in}{4.456127in}}%
\pgfpathlineto{\pgfqpoint{2.584725in}{4.464386in}}%
\pgfpathlineto{\pgfqpoint{2.603506in}{4.475581in}}%
\pgfpathlineto{\pgfqpoint{2.622287in}{4.489794in}}%
\pgfpathlineto{\pgfqpoint{2.641068in}{4.507028in}}%
\pgfpathlineto{\pgfqpoint{2.659850in}{4.527200in}}%
\pgfpathlineto{\pgfqpoint{2.678631in}{4.550127in}}%
\pgfpathlineto{\pgfqpoint{2.703673in}{4.584461in}}%
\pgfpathlineto{\pgfqpoint{2.734975in}{4.631975in}}%
\pgfpathlineto{\pgfqpoint{2.816361in}{4.759131in}}%
\pgfpathlineto{\pgfqpoint{2.835142in}{4.784424in}}%
\pgfpathlineto{\pgfqpoint{2.853923in}{4.806511in}}%
\pgfpathlineto{\pgfqpoint{2.872705in}{4.824670in}}%
\pgfpathlineto{\pgfqpoint{2.885225in}{4.834282in}}%
\pgfpathlineto{\pgfqpoint{2.897746in}{4.841713in}}%
\pgfpathlineto{\pgfqpoint{2.910267in}{4.846838in}}%
\pgfpathlineto{\pgfqpoint{2.922788in}{4.849560in}}%
\pgfpathlineto{\pgfqpoint{2.935309in}{4.849818in}}%
\pgfpathlineto{\pgfqpoint{2.947830in}{4.847582in}}%
\pgfpathlineto{\pgfqpoint{2.960351in}{4.842859in}}%
\pgfpathlineto{\pgfqpoint{2.972871in}{4.835689in}}%
\pgfpathlineto{\pgfqpoint{2.985392in}{4.826146in}}%
\pgfpathlineto{\pgfqpoint{2.997913in}{4.814336in}}%
\pgfpathlineto{\pgfqpoint{3.016694in}{4.792674in}}%
\pgfpathlineto{\pgfqpoint{3.035476in}{4.766785in}}%
\pgfpathlineto{\pgfqpoint{3.054257in}{4.737338in}}%
\pgfpathlineto{\pgfqpoint{3.079299in}{4.693831in}}%
\pgfpathlineto{\pgfqpoint{3.116861in}{4.623261in}}%
\pgfpathlineto{\pgfqpoint{3.173205in}{4.517256in}}%
\pgfpathlineto{\pgfqpoint{3.204508in}{4.463571in}}%
\pgfpathlineto{\pgfqpoint{3.229549in}{4.424981in}}%
\pgfpathlineto{\pgfqpoint{3.254591in}{4.390849in}}%
\pgfpathlineto{\pgfqpoint{3.279633in}{4.361386in}}%
\pgfpathlineto{\pgfqpoint{3.298414in}{4.342330in}}%
\pgfpathlineto{\pgfqpoint{3.317195in}{4.325777in}}%
\pgfpathlineto{\pgfqpoint{3.335977in}{4.311576in}}%
\pgfpathlineto{\pgfqpoint{3.354758in}{4.299541in}}%
\pgfpathlineto{\pgfqpoint{3.379800in}{4.286504in}}%
\pgfpathlineto{\pgfqpoint{3.404841in}{4.276428in}}%
\pgfpathlineto{\pgfqpoint{3.429883in}{4.268800in}}%
\pgfpathlineto{\pgfqpoint{3.461185in}{4.261983in}}%
\pgfpathlineto{\pgfqpoint{3.492487in}{4.257433in}}%
\pgfpathlineto{\pgfqpoint{3.536311in}{4.253641in}}%
\pgfpathlineto{\pgfqpoint{3.592654in}{4.251348in}}%
\pgfpathlineto{\pgfqpoint{3.686561in}{4.250215in}}%
\pgfpathlineto{\pgfqpoint{3.999583in}{4.250000in}}%
\pgfpathlineto{\pgfqpoint{4.250000in}{4.250000in}}%
\pgfpathlineto{\pgfqpoint{4.250000in}{4.250000in}}%
\pgfusepath{stroke}%
\end{pgfscope}%
\begin{pgfscope}%
\pgfpathrectangle{\pgfqpoint{4.250000in}{0.500000in}}{\pgfqpoint{0.600000in}{3.750000in}}%
\pgfusepath{clip}%
\pgfsetbuttcap%
\pgfsetroundjoin%
\definecolor{currentfill}{rgb}{0.000000,0.000000,1.000000}%
\pgfsetfillcolor{currentfill}%
\pgfsetfillopacity{0.500000}%
\pgfsetlinewidth{1.003750pt}%
\definecolor{currentstroke}{rgb}{0.000000,0.000000,0.000000}%
\pgfsetstrokecolor{currentstroke}%
\pgfsetstrokeopacity{0.500000}%
\pgfsetdash{}{0pt}%
\pgfsys@defobject{currentmarker}{\pgfqpoint{4.250000in}{0.500000in}}{\pgfqpoint{4.834081in}{4.250000in}}{%
\pgfpathmoveto{\pgfqpoint{4.250000in}{4.250000in}}%
\pgfpathlineto{\pgfqpoint{4.250000in}{4.250000in}}%
\pgfpathlineto{\pgfqpoint{4.250000in}{4.243740in}}%
\pgfpathlineto{\pgfqpoint{4.250000in}{4.237479in}}%
\pgfpathlineto{\pgfqpoint{4.250000in}{4.231219in}}%
\pgfpathlineto{\pgfqpoint{4.250000in}{4.224958in}}%
\pgfpathlineto{\pgfqpoint{4.250000in}{4.218698in}}%
\pgfpathlineto{\pgfqpoint{4.250000in}{4.212437in}}%
\pgfpathlineto{\pgfqpoint{4.250000in}{4.206177in}}%
\pgfpathlineto{\pgfqpoint{4.250000in}{4.199917in}}%
\pgfpathlineto{\pgfqpoint{4.250000in}{4.193656in}}%
\pgfpathlineto{\pgfqpoint{4.250000in}{4.187396in}}%
\pgfpathlineto{\pgfqpoint{4.250000in}{4.181135in}}%
\pgfpathlineto{\pgfqpoint{4.250000in}{4.174875in}}%
\pgfpathlineto{\pgfqpoint{4.250000in}{4.168614in}}%
\pgfpathlineto{\pgfqpoint{4.250000in}{4.162354in}}%
\pgfpathlineto{\pgfqpoint{4.250000in}{4.156093in}}%
\pgfpathlineto{\pgfqpoint{4.250000in}{4.149833in}}%
\pgfpathlineto{\pgfqpoint{4.250000in}{4.143573in}}%
\pgfpathlineto{\pgfqpoint{4.250000in}{4.137312in}}%
\pgfpathlineto{\pgfqpoint{4.250000in}{4.131052in}}%
\pgfpathlineto{\pgfqpoint{4.250000in}{4.124791in}}%
\pgfpathlineto{\pgfqpoint{4.250000in}{4.118531in}}%
\pgfpathlineto{\pgfqpoint{4.250000in}{4.112270in}}%
\pgfpathlineto{\pgfqpoint{4.250000in}{4.106010in}}%
\pgfpathlineto{\pgfqpoint{4.250000in}{4.099750in}}%
\pgfpathlineto{\pgfqpoint{4.250000in}{4.093489in}}%
\pgfpathlineto{\pgfqpoint{4.250000in}{4.087229in}}%
\pgfpathlineto{\pgfqpoint{4.250000in}{4.080968in}}%
\pgfpathlineto{\pgfqpoint{4.250000in}{4.074708in}}%
\pgfpathlineto{\pgfqpoint{4.250000in}{4.068447in}}%
\pgfpathlineto{\pgfqpoint{4.250000in}{4.062187in}}%
\pgfpathlineto{\pgfqpoint{4.250000in}{4.055927in}}%
\pgfpathlineto{\pgfqpoint{4.250000in}{4.049666in}}%
\pgfpathlineto{\pgfqpoint{4.250000in}{4.043406in}}%
\pgfpathlineto{\pgfqpoint{4.250000in}{4.037145in}}%
\pgfpathlineto{\pgfqpoint{4.250000in}{4.030885in}}%
\pgfpathlineto{\pgfqpoint{4.250000in}{4.024624in}}%
\pgfpathlineto{\pgfqpoint{4.250000in}{4.018364in}}%
\pgfpathlineto{\pgfqpoint{4.250000in}{4.012104in}}%
\pgfpathlineto{\pgfqpoint{4.250000in}{4.005843in}}%
\pgfpathlineto{\pgfqpoint{4.250000in}{3.999583in}}%
\pgfpathlineto{\pgfqpoint{4.250000in}{3.993322in}}%
\pgfpathlineto{\pgfqpoint{4.250000in}{3.987062in}}%
\pgfpathlineto{\pgfqpoint{4.250000in}{3.980801in}}%
\pgfpathlineto{\pgfqpoint{4.250000in}{3.974541in}}%
\pgfpathlineto{\pgfqpoint{4.250000in}{3.968280in}}%
\pgfpathlineto{\pgfqpoint{4.250000in}{3.962020in}}%
\pgfpathlineto{\pgfqpoint{4.250000in}{3.955760in}}%
\pgfpathlineto{\pgfqpoint{4.250000in}{3.949499in}}%
\pgfpathlineto{\pgfqpoint{4.250000in}{3.943239in}}%
\pgfpathlineto{\pgfqpoint{4.250000in}{3.936978in}}%
\pgfpathlineto{\pgfqpoint{4.250000in}{3.930718in}}%
\pgfpathlineto{\pgfqpoint{4.250000in}{3.924457in}}%
\pgfpathlineto{\pgfqpoint{4.250001in}{3.918197in}}%
\pgfpathlineto{\pgfqpoint{4.250001in}{3.911937in}}%
\pgfpathlineto{\pgfqpoint{4.250001in}{3.905676in}}%
\pgfpathlineto{\pgfqpoint{4.250001in}{3.899416in}}%
\pgfpathlineto{\pgfqpoint{4.250001in}{3.893155in}}%
\pgfpathlineto{\pgfqpoint{4.250001in}{3.886895in}}%
\pgfpathlineto{\pgfqpoint{4.250001in}{3.880634in}}%
\pgfpathlineto{\pgfqpoint{4.250001in}{3.874374in}}%
\pgfpathlineto{\pgfqpoint{4.250002in}{3.868114in}}%
\pgfpathlineto{\pgfqpoint{4.250002in}{3.861853in}}%
\pgfpathlineto{\pgfqpoint{4.250002in}{3.855593in}}%
\pgfpathlineto{\pgfqpoint{4.250003in}{3.849332in}}%
\pgfpathlineto{\pgfqpoint{4.250003in}{3.843072in}}%
\pgfpathlineto{\pgfqpoint{4.250003in}{3.836811in}}%
\pgfpathlineto{\pgfqpoint{4.250004in}{3.830551in}}%
\pgfpathlineto{\pgfqpoint{4.250004in}{3.824290in}}%
\pgfpathlineto{\pgfqpoint{4.250005in}{3.818030in}}%
\pgfpathlineto{\pgfqpoint{4.250006in}{3.811770in}}%
\pgfpathlineto{\pgfqpoint{4.250006in}{3.805509in}}%
\pgfpathlineto{\pgfqpoint{4.250007in}{3.799249in}}%
\pgfpathlineto{\pgfqpoint{4.250008in}{3.792988in}}%
\pgfpathlineto{\pgfqpoint{4.250010in}{3.786728in}}%
\pgfpathlineto{\pgfqpoint{4.250011in}{3.780467in}}%
\pgfpathlineto{\pgfqpoint{4.250012in}{3.774207in}}%
\pgfpathlineto{\pgfqpoint{4.250014in}{3.767947in}}%
\pgfpathlineto{\pgfqpoint{4.250016in}{3.761686in}}%
\pgfpathlineto{\pgfqpoint{4.250018in}{3.755426in}}%
\pgfpathlineto{\pgfqpoint{4.250020in}{3.749165in}}%
\pgfpathlineto{\pgfqpoint{4.250023in}{3.742905in}}%
\pgfpathlineto{\pgfqpoint{4.250026in}{3.736644in}}%
\pgfpathlineto{\pgfqpoint{4.250029in}{3.730384in}}%
\pgfpathlineto{\pgfqpoint{4.250033in}{3.724124in}}%
\pgfpathlineto{\pgfqpoint{4.250037in}{3.717863in}}%
\pgfpathlineto{\pgfqpoint{4.250042in}{3.711603in}}%
\pgfpathlineto{\pgfqpoint{4.250047in}{3.705342in}}%
\pgfpathlineto{\pgfqpoint{4.250053in}{3.699082in}}%
\pgfpathlineto{\pgfqpoint{4.250060in}{3.692821in}}%
\pgfpathlineto{\pgfqpoint{4.250067in}{3.686561in}}%
\pgfpathlineto{\pgfqpoint{4.250075in}{3.680301in}}%
\pgfpathlineto{\pgfqpoint{4.250085in}{3.674040in}}%
\pgfpathlineto{\pgfqpoint{4.250095in}{3.667780in}}%
\pgfpathlineto{\pgfqpoint{4.250106in}{3.661519in}}%
\pgfpathlineto{\pgfqpoint{4.250119in}{3.655259in}}%
\pgfpathlineto{\pgfqpoint{4.250133in}{3.648998in}}%
\pgfpathlineto{\pgfqpoint{4.250148in}{3.642738in}}%
\pgfpathlineto{\pgfqpoint{4.250166in}{3.636477in}}%
\pgfpathlineto{\pgfqpoint{4.250185in}{3.630217in}}%
\pgfpathlineto{\pgfqpoint{4.250206in}{3.623957in}}%
\pgfpathlineto{\pgfqpoint{4.250229in}{3.617696in}}%
\pgfpathlineto{\pgfqpoint{4.250255in}{3.611436in}}%
\pgfpathlineto{\pgfqpoint{4.250284in}{3.605175in}}%
\pgfpathlineto{\pgfqpoint{4.250315in}{3.598915in}}%
\pgfpathlineto{\pgfqpoint{4.250350in}{3.592654in}}%
\pgfpathlineto{\pgfqpoint{4.250388in}{3.586394in}}%
\pgfpathlineto{\pgfqpoint{4.250430in}{3.580134in}}%
\pgfpathlineto{\pgfqpoint{4.250477in}{3.573873in}}%
\pgfpathlineto{\pgfqpoint{4.250528in}{3.567613in}}%
\pgfpathlineto{\pgfqpoint{4.250584in}{3.561352in}}%
\pgfpathlineto{\pgfqpoint{4.250645in}{3.555092in}}%
\pgfpathlineto{\pgfqpoint{4.250712in}{3.548831in}}%
\pgfpathlineto{\pgfqpoint{4.250786in}{3.542571in}}%
\pgfpathlineto{\pgfqpoint{4.250867in}{3.536311in}}%
\pgfpathlineto{\pgfqpoint{4.250955in}{3.530050in}}%
\pgfpathlineto{\pgfqpoint{4.251051in}{3.523790in}}%
\pgfpathlineto{\pgfqpoint{4.251156in}{3.517529in}}%
\pgfpathlineto{\pgfqpoint{4.251271in}{3.511269in}}%
\pgfpathlineto{\pgfqpoint{4.251396in}{3.505008in}}%
\pgfpathlineto{\pgfqpoint{4.251532in}{3.498748in}}%
\pgfpathlineto{\pgfqpoint{4.251680in}{3.492487in}}%
\pgfpathlineto{\pgfqpoint{4.251840in}{3.486227in}}%
\pgfpathlineto{\pgfqpoint{4.252015in}{3.479967in}}%
\pgfpathlineto{\pgfqpoint{4.252205in}{3.473706in}}%
\pgfpathlineto{\pgfqpoint{4.252410in}{3.467446in}}%
\pgfpathlineto{\pgfqpoint{4.252633in}{3.461185in}}%
\pgfpathlineto{\pgfqpoint{4.252874in}{3.454925in}}%
\pgfpathlineto{\pgfqpoint{4.253135in}{3.448664in}}%
\pgfpathlineto{\pgfqpoint{4.253416in}{3.442404in}}%
\pgfpathlineto{\pgfqpoint{4.253720in}{3.436144in}}%
\pgfpathlineto{\pgfqpoint{4.254048in}{3.429883in}}%
\pgfpathlineto{\pgfqpoint{4.254402in}{3.423623in}}%
\pgfpathlineto{\pgfqpoint{4.254782in}{3.417362in}}%
\pgfpathlineto{\pgfqpoint{4.255192in}{3.411102in}}%
\pgfpathlineto{\pgfqpoint{4.255632in}{3.404841in}}%
\pgfpathlineto{\pgfqpoint{4.256105in}{3.398581in}}%
\pgfpathlineto{\pgfqpoint{4.256613in}{3.392321in}}%
\pgfpathlineto{\pgfqpoint{4.257157in}{3.386060in}}%
\pgfpathlineto{\pgfqpoint{4.257740in}{3.379800in}}%
\pgfpathlineto{\pgfqpoint{4.258364in}{3.373539in}}%
\pgfpathlineto{\pgfqpoint{4.259031in}{3.367279in}}%
\pgfpathlineto{\pgfqpoint{4.259744in}{3.361018in}}%
\pgfpathlineto{\pgfqpoint{4.260505in}{3.354758in}}%
\pgfpathlineto{\pgfqpoint{4.261317in}{3.348497in}}%
\pgfpathlineto{\pgfqpoint{4.262182in}{3.342237in}}%
\pgfpathlineto{\pgfqpoint{4.263103in}{3.335977in}}%
\pgfpathlineto{\pgfqpoint{4.264083in}{3.329716in}}%
\pgfpathlineto{\pgfqpoint{4.265124in}{3.323456in}}%
\pgfpathlineto{\pgfqpoint{4.266230in}{3.317195in}}%
\pgfpathlineto{\pgfqpoint{4.267404in}{3.310935in}}%
\pgfpathlineto{\pgfqpoint{4.268648in}{3.304674in}}%
\pgfpathlineto{\pgfqpoint{4.269965in}{3.298414in}}%
\pgfpathlineto{\pgfqpoint{4.271359in}{3.292154in}}%
\pgfpathlineto{\pgfqpoint{4.272833in}{3.285893in}}%
\pgfpathlineto{\pgfqpoint{4.274389in}{3.279633in}}%
\pgfpathlineto{\pgfqpoint{4.276032in}{3.273372in}}%
\pgfpathlineto{\pgfqpoint{4.277763in}{3.267112in}}%
\pgfpathlineto{\pgfqpoint{4.279587in}{3.260851in}}%
\pgfpathlineto{\pgfqpoint{4.281507in}{3.254591in}}%
\pgfpathlineto{\pgfqpoint{4.283525in}{3.248331in}}%
\pgfpathlineto{\pgfqpoint{4.285645in}{3.242070in}}%
\pgfpathlineto{\pgfqpoint{4.287870in}{3.235810in}}%
\pgfpathlineto{\pgfqpoint{4.290202in}{3.229549in}}%
\pgfpathlineto{\pgfqpoint{4.292646in}{3.223289in}}%
\pgfpathlineto{\pgfqpoint{4.295203in}{3.217028in}}%
\pgfpathlineto{\pgfqpoint{4.297876in}{3.210768in}}%
\pgfpathlineto{\pgfqpoint{4.300669in}{3.204508in}}%
\pgfpathlineto{\pgfqpoint{4.303583in}{3.198247in}}%
\pgfpathlineto{\pgfqpoint{4.306621in}{3.191987in}}%
\pgfpathlineto{\pgfqpoint{4.309785in}{3.185726in}}%
\pgfpathlineto{\pgfqpoint{4.313078in}{3.179466in}}%
\pgfpathlineto{\pgfqpoint{4.316501in}{3.173205in}}%
\pgfpathlineto{\pgfqpoint{4.320056in}{3.166945in}}%
\pgfpathlineto{\pgfqpoint{4.323743in}{3.160684in}}%
\pgfpathlineto{\pgfqpoint{4.327566in}{3.154424in}}%
\pgfpathlineto{\pgfqpoint{4.331524in}{3.148164in}}%
\pgfpathlineto{\pgfqpoint{4.335618in}{3.141903in}}%
\pgfpathlineto{\pgfqpoint{4.339848in}{3.135643in}}%
\pgfpathlineto{\pgfqpoint{4.344216in}{3.129382in}}%
\pgfpathlineto{\pgfqpoint{4.348719in}{3.123122in}}%
\pgfpathlineto{\pgfqpoint{4.353359in}{3.116861in}}%
\pgfpathlineto{\pgfqpoint{4.358134in}{3.110601in}}%
\pgfpathlineto{\pgfqpoint{4.363043in}{3.104341in}}%
\pgfpathlineto{\pgfqpoint{4.368084in}{3.098080in}}%
\pgfpathlineto{\pgfqpoint{4.373256in}{3.091820in}}%
\pgfpathlineto{\pgfqpoint{4.378556in}{3.085559in}}%
\pgfpathlineto{\pgfqpoint{4.383981in}{3.079299in}}%
\pgfpathlineto{\pgfqpoint{4.389528in}{3.073038in}}%
\pgfpathlineto{\pgfqpoint{4.395195in}{3.066778in}}%
\pgfpathlineto{\pgfqpoint{4.400976in}{3.060518in}}%
\pgfpathlineto{\pgfqpoint{4.406868in}{3.054257in}}%
\pgfpathlineto{\pgfqpoint{4.412865in}{3.047997in}}%
\pgfpathlineto{\pgfqpoint{4.418964in}{3.041736in}}%
\pgfpathlineto{\pgfqpoint{4.425157in}{3.035476in}}%
\pgfpathlineto{\pgfqpoint{4.431439in}{3.029215in}}%
\pgfpathlineto{\pgfqpoint{4.437804in}{3.022955in}}%
\pgfpathlineto{\pgfqpoint{4.444246in}{3.016694in}}%
\pgfpathlineto{\pgfqpoint{4.450755in}{3.010434in}}%
\pgfpathlineto{\pgfqpoint{4.457327in}{3.004174in}}%
\pgfpathlineto{\pgfqpoint{4.463951in}{2.997913in}}%
\pgfpathlineto{\pgfqpoint{4.470621in}{2.991653in}}%
\pgfpathlineto{\pgfqpoint{4.477327in}{2.985392in}}%
\pgfpathlineto{\pgfqpoint{4.484061in}{2.979132in}}%
\pgfpathlineto{\pgfqpoint{4.490814in}{2.972871in}}%
\pgfpathlineto{\pgfqpoint{4.497576in}{2.966611in}}%
\pgfpathlineto{\pgfqpoint{4.504337in}{2.960351in}}%
\pgfpathlineto{\pgfqpoint{4.511087in}{2.954090in}}%
\pgfpathlineto{\pgfqpoint{4.517817in}{2.947830in}}%
\pgfpathlineto{\pgfqpoint{4.524516in}{2.941569in}}%
\pgfpathlineto{\pgfqpoint{4.531174in}{2.935309in}}%
\pgfpathlineto{\pgfqpoint{4.537780in}{2.929048in}}%
\pgfpathlineto{\pgfqpoint{4.544323in}{2.922788in}}%
\pgfpathlineto{\pgfqpoint{4.550794in}{2.916528in}}%
\pgfpathlineto{\pgfqpoint{4.557181in}{2.910267in}}%
\pgfpathlineto{\pgfqpoint{4.563474in}{2.904007in}}%
\pgfpathlineto{\pgfqpoint{4.569663in}{2.897746in}}%
\pgfpathlineto{\pgfqpoint{4.575737in}{2.891486in}}%
\pgfpathlineto{\pgfqpoint{4.581685in}{2.885225in}}%
\pgfpathlineto{\pgfqpoint{4.587499in}{2.878965in}}%
\pgfpathlineto{\pgfqpoint{4.593167in}{2.872705in}}%
\pgfpathlineto{\pgfqpoint{4.598681in}{2.866444in}}%
\pgfpathlineto{\pgfqpoint{4.604031in}{2.860184in}}%
\pgfpathlineto{\pgfqpoint{4.609209in}{2.853923in}}%
\pgfpathlineto{\pgfqpoint{4.614205in}{2.847663in}}%
\pgfpathlineto{\pgfqpoint{4.619012in}{2.841402in}}%
\pgfpathlineto{\pgfqpoint{4.623622in}{2.835142in}}%
\pgfpathlineto{\pgfqpoint{4.628028in}{2.828881in}}%
\pgfpathlineto{\pgfqpoint{4.632222in}{2.822621in}}%
\pgfpathlineto{\pgfqpoint{4.636200in}{2.816361in}}%
\pgfpathlineto{\pgfqpoint{4.639955in}{2.810100in}}%
\pgfpathlineto{\pgfqpoint{4.643482in}{2.803840in}}%
\pgfpathlineto{\pgfqpoint{4.646778in}{2.797579in}}%
\pgfpathlineto{\pgfqpoint{4.649837in}{2.791319in}}%
\pgfpathlineto{\pgfqpoint{4.652657in}{2.785058in}}%
\pgfpathlineto{\pgfqpoint{4.655236in}{2.778798in}}%
\pgfpathlineto{\pgfqpoint{4.657572in}{2.772538in}}%
\pgfpathlineto{\pgfqpoint{4.659663in}{2.766277in}}%
\pgfpathlineto{\pgfqpoint{4.661510in}{2.760017in}}%
\pgfpathlineto{\pgfqpoint{4.663113in}{2.753756in}}%
\pgfpathlineto{\pgfqpoint{4.664472in}{2.747496in}}%
\pgfpathlineto{\pgfqpoint{4.665591in}{2.741235in}}%
\pgfpathlineto{\pgfqpoint{4.666470in}{2.734975in}}%
\pgfpathlineto{\pgfqpoint{4.667114in}{2.728715in}}%
\pgfpathlineto{\pgfqpoint{4.667527in}{2.722454in}}%
\pgfpathlineto{\pgfqpoint{4.667713in}{2.716194in}}%
\pgfpathlineto{\pgfqpoint{4.667677in}{2.709933in}}%
\pgfpathlineto{\pgfqpoint{4.667425in}{2.703673in}}%
\pgfpathlineto{\pgfqpoint{4.666965in}{2.697412in}}%
\pgfpathlineto{\pgfqpoint{4.666303in}{2.691152in}}%
\pgfpathlineto{\pgfqpoint{4.665448in}{2.684891in}}%
\pgfpathlineto{\pgfqpoint{4.664407in}{2.678631in}}%
\pgfpathlineto{\pgfqpoint{4.663190in}{2.672371in}}%
\pgfpathlineto{\pgfqpoint{4.661806in}{2.666110in}}%
\pgfpathlineto{\pgfqpoint{4.660266in}{2.659850in}}%
\pgfpathlineto{\pgfqpoint{4.658579in}{2.653589in}}%
\pgfpathlineto{\pgfqpoint{4.656757in}{2.647329in}}%
\pgfpathlineto{\pgfqpoint{4.654811in}{2.641068in}}%
\pgfpathlineto{\pgfqpoint{4.652752in}{2.634808in}}%
\pgfpathlineto{\pgfqpoint{4.650593in}{2.628548in}}%
\pgfpathlineto{\pgfqpoint{4.648346in}{2.622287in}}%
\pgfpathlineto{\pgfqpoint{4.646022in}{2.616027in}}%
\pgfpathlineto{\pgfqpoint{4.643635in}{2.609766in}}%
\pgfpathlineto{\pgfqpoint{4.641197in}{2.603506in}}%
\pgfpathlineto{\pgfqpoint{4.638722in}{2.597245in}}%
\pgfpathlineto{\pgfqpoint{4.636221in}{2.590985in}}%
\pgfpathlineto{\pgfqpoint{4.633708in}{2.584725in}}%
\pgfpathlineto{\pgfqpoint{4.631196in}{2.578464in}}%
\pgfpathlineto{\pgfqpoint{4.628697in}{2.572204in}}%
\pgfpathlineto{\pgfqpoint{4.626223in}{2.565943in}}%
\pgfpathlineto{\pgfqpoint{4.623789in}{2.559683in}}%
\pgfpathlineto{\pgfqpoint{4.621404in}{2.553422in}}%
\pgfpathlineto{\pgfqpoint{4.619083in}{2.547162in}}%
\pgfpathlineto{\pgfqpoint{4.616835in}{2.540902in}}%
\pgfpathlineto{\pgfqpoint{4.614673in}{2.534641in}}%
\pgfpathlineto{\pgfqpoint{4.612609in}{2.528381in}}%
\pgfpathlineto{\pgfqpoint{4.610651in}{2.522120in}}%
\pgfpathlineto{\pgfqpoint{4.608812in}{2.515860in}}%
\pgfpathlineto{\pgfqpoint{4.607100in}{2.509599in}}%
\pgfpathlineto{\pgfqpoint{4.605525in}{2.503339in}}%
\pgfpathlineto{\pgfqpoint{4.604096in}{2.497078in}}%
\pgfpathlineto{\pgfqpoint{4.602821in}{2.490818in}}%
\pgfpathlineto{\pgfqpoint{4.601709in}{2.484558in}}%
\pgfpathlineto{\pgfqpoint{4.600766in}{2.478297in}}%
\pgfpathlineto{\pgfqpoint{4.599999in}{2.472037in}}%
\pgfpathlineto{\pgfqpoint{4.599414in}{2.465776in}}%
\pgfpathlineto{\pgfqpoint{4.599017in}{2.459516in}}%
\pgfpathlineto{\pgfqpoint{4.598813in}{2.453255in}}%
\pgfpathlineto{\pgfqpoint{4.598806in}{2.446995in}}%
\pgfpathlineto{\pgfqpoint{4.599000in}{2.440735in}}%
\pgfpathlineto{\pgfqpoint{4.599397in}{2.434474in}}%
\pgfpathlineto{\pgfqpoint{4.600001in}{2.428214in}}%
\pgfpathlineto{\pgfqpoint{4.600813in}{2.421953in}}%
\pgfpathlineto{\pgfqpoint{4.601833in}{2.415693in}}%
\pgfpathlineto{\pgfqpoint{4.603064in}{2.409432in}}%
\pgfpathlineto{\pgfqpoint{4.604504in}{2.403172in}}%
\pgfpathlineto{\pgfqpoint{4.606153in}{2.396912in}}%
\pgfpathlineto{\pgfqpoint{4.608010in}{2.390651in}}%
\pgfpathlineto{\pgfqpoint{4.610073in}{2.384391in}}%
\pgfpathlineto{\pgfqpoint{4.612339in}{2.378130in}}%
\pgfpathlineto{\pgfqpoint{4.614805in}{2.371870in}}%
\pgfpathlineto{\pgfqpoint{4.617468in}{2.365609in}}%
\pgfpathlineto{\pgfqpoint{4.620323in}{2.359349in}}%
\pgfpathlineto{\pgfqpoint{4.623367in}{2.353088in}}%
\pgfpathlineto{\pgfqpoint{4.626593in}{2.346828in}}%
\pgfpathlineto{\pgfqpoint{4.629996in}{2.340568in}}%
\pgfpathlineto{\pgfqpoint{4.633571in}{2.334307in}}%
\pgfpathlineto{\pgfqpoint{4.637310in}{2.328047in}}%
\pgfpathlineto{\pgfqpoint{4.641207in}{2.321786in}}%
\pgfpathlineto{\pgfqpoint{4.645254in}{2.315526in}}%
\pgfpathlineto{\pgfqpoint{4.649444in}{2.309265in}}%
\pgfpathlineto{\pgfqpoint{4.653769in}{2.303005in}}%
\pgfpathlineto{\pgfqpoint{4.658219in}{2.296745in}}%
\pgfpathlineto{\pgfqpoint{4.662787in}{2.290484in}}%
\pgfpathlineto{\pgfqpoint{4.667464in}{2.284224in}}%
\pgfpathlineto{\pgfqpoint{4.672240in}{2.277963in}}%
\pgfpathlineto{\pgfqpoint{4.677106in}{2.271703in}}%
\pgfpathlineto{\pgfqpoint{4.682052in}{2.265442in}}%
\pgfpathlineto{\pgfqpoint{4.687068in}{2.259182in}}%
\pgfpathlineto{\pgfqpoint{4.692144in}{2.252922in}}%
\pgfpathlineto{\pgfqpoint{4.697271in}{2.246661in}}%
\pgfpathlineto{\pgfqpoint{4.702438in}{2.240401in}}%
\pgfpathlineto{\pgfqpoint{4.707635in}{2.234140in}}%
\pgfpathlineto{\pgfqpoint{4.712851in}{2.227880in}}%
\pgfpathlineto{\pgfqpoint{4.718075in}{2.221619in}}%
\pgfpathlineto{\pgfqpoint{4.723298in}{2.215359in}}%
\pgfpathlineto{\pgfqpoint{4.728509in}{2.209098in}}%
\pgfpathlineto{\pgfqpoint{4.733698in}{2.202838in}}%
\pgfpathlineto{\pgfqpoint{4.738854in}{2.196578in}}%
\pgfpathlineto{\pgfqpoint{4.743967in}{2.190317in}}%
\pgfpathlineto{\pgfqpoint{4.749026in}{2.184057in}}%
\pgfpathlineto{\pgfqpoint{4.754023in}{2.177796in}}%
\pgfpathlineto{\pgfqpoint{4.758946in}{2.171536in}}%
\pgfpathlineto{\pgfqpoint{4.763785in}{2.165275in}}%
\pgfpathlineto{\pgfqpoint{4.768533in}{2.159015in}}%
\pgfpathlineto{\pgfqpoint{4.773178in}{2.152755in}}%
\pgfpathlineto{\pgfqpoint{4.777711in}{2.146494in}}%
\pgfpathlineto{\pgfqpoint{4.782124in}{2.140234in}}%
\pgfpathlineto{\pgfqpoint{4.786408in}{2.133973in}}%
\pgfpathlineto{\pgfqpoint{4.790555in}{2.127713in}}%
\pgfpathlineto{\pgfqpoint{4.794555in}{2.121452in}}%
\pgfpathlineto{\pgfqpoint{4.798401in}{2.115192in}}%
\pgfpathlineto{\pgfqpoint{4.802085in}{2.108932in}}%
\pgfpathlineto{\pgfqpoint{4.805600in}{2.102671in}}%
\pgfpathlineto{\pgfqpoint{4.808939in}{2.096411in}}%
\pgfpathlineto{\pgfqpoint{4.812095in}{2.090150in}}%
\pgfpathlineto{\pgfqpoint{4.815061in}{2.083890in}}%
\pgfpathlineto{\pgfqpoint{4.817831in}{2.077629in}}%
\pgfpathlineto{\pgfqpoint{4.820400in}{2.071369in}}%
\pgfpathlineto{\pgfqpoint{4.822762in}{2.065109in}}%
\pgfpathlineto{\pgfqpoint{4.824912in}{2.058848in}}%
\pgfpathlineto{\pgfqpoint{4.826846in}{2.052588in}}%
\pgfpathlineto{\pgfqpoint{4.828559in}{2.046327in}}%
\pgfpathlineto{\pgfqpoint{4.830047in}{2.040067in}}%
\pgfpathlineto{\pgfqpoint{4.831306in}{2.033806in}}%
\pgfpathlineto{\pgfqpoint{4.832334in}{2.027546in}}%
\pgfpathlineto{\pgfqpoint{4.833128in}{2.021285in}}%
\pgfpathlineto{\pgfqpoint{4.833685in}{2.015025in}}%
\pgfpathlineto{\pgfqpoint{4.834003in}{2.008765in}}%
\pgfpathlineto{\pgfqpoint{4.834081in}{2.002504in}}%
\pgfpathlineto{\pgfqpoint{4.833918in}{1.996244in}}%
\pgfpathlineto{\pgfqpoint{4.833512in}{1.989983in}}%
\pgfpathlineto{\pgfqpoint{4.832863in}{1.983723in}}%
\pgfpathlineto{\pgfqpoint{4.831972in}{1.977462in}}%
\pgfpathlineto{\pgfqpoint{4.830838in}{1.971202in}}%
\pgfpathlineto{\pgfqpoint{4.829462in}{1.964942in}}%
\pgfpathlineto{\pgfqpoint{4.827845in}{1.958681in}}%
\pgfpathlineto{\pgfqpoint{4.825989in}{1.952421in}}%
\pgfpathlineto{\pgfqpoint{4.823895in}{1.946160in}}%
\pgfpathlineto{\pgfqpoint{4.821565in}{1.939900in}}%
\pgfpathlineto{\pgfqpoint{4.819002in}{1.933639in}}%
\pgfpathlineto{\pgfqpoint{4.816209in}{1.927379in}}%
\pgfpathlineto{\pgfqpoint{4.813189in}{1.921119in}}%
\pgfpathlineto{\pgfqpoint{4.809945in}{1.914858in}}%
\pgfpathlineto{\pgfqpoint{4.806481in}{1.908598in}}%
\pgfpathlineto{\pgfqpoint{4.802800in}{1.902337in}}%
\pgfpathlineto{\pgfqpoint{4.798908in}{1.896077in}}%
\pgfpathlineto{\pgfqpoint{4.794809in}{1.889816in}}%
\pgfpathlineto{\pgfqpoint{4.790507in}{1.883556in}}%
\pgfpathlineto{\pgfqpoint{4.786008in}{1.877295in}}%
\pgfpathlineto{\pgfqpoint{4.781317in}{1.871035in}}%
\pgfpathlineto{\pgfqpoint{4.776440in}{1.864775in}}%
\pgfpathlineto{\pgfqpoint{4.771382in}{1.858514in}}%
\pgfpathlineto{\pgfqpoint{4.766149in}{1.852254in}}%
\pgfpathlineto{\pgfqpoint{4.760747in}{1.845993in}}%
\pgfpathlineto{\pgfqpoint{4.755183in}{1.839733in}}%
\pgfpathlineto{\pgfqpoint{4.749463in}{1.833472in}}%
\pgfpathlineto{\pgfqpoint{4.743594in}{1.827212in}}%
\pgfpathlineto{\pgfqpoint{4.737583in}{1.820952in}}%
\pgfpathlineto{\pgfqpoint{4.731436in}{1.814691in}}%
\pgfpathlineto{\pgfqpoint{4.725160in}{1.808431in}}%
\pgfpathlineto{\pgfqpoint{4.718762in}{1.802170in}}%
\pgfpathlineto{\pgfqpoint{4.712249in}{1.795910in}}%
\pgfpathlineto{\pgfqpoint{4.705629in}{1.789649in}}%
\pgfpathlineto{\pgfqpoint{4.698909in}{1.783389in}}%
\pgfpathlineto{\pgfqpoint{4.692096in}{1.777129in}}%
\pgfpathlineto{\pgfqpoint{4.685197in}{1.770868in}}%
\pgfpathlineto{\pgfqpoint{4.678219in}{1.764608in}}%
\pgfpathlineto{\pgfqpoint{4.671170in}{1.758347in}}%
\pgfpathlineto{\pgfqpoint{4.664057in}{1.752087in}}%
\pgfpathlineto{\pgfqpoint{4.656888in}{1.745826in}}%
\pgfpathlineto{\pgfqpoint{4.649668in}{1.739566in}}%
\pgfpathlineto{\pgfqpoint{4.642406in}{1.733306in}}%
\pgfpathlineto{\pgfqpoint{4.635108in}{1.727045in}}%
\pgfpathlineto{\pgfqpoint{4.627781in}{1.720785in}}%
\pgfpathlineto{\pgfqpoint{4.620433in}{1.714524in}}%
\pgfpathlineto{\pgfqpoint{4.613069in}{1.708264in}}%
\pgfpathlineto{\pgfqpoint{4.605698in}{1.702003in}}%
\pgfpathlineto{\pgfqpoint{4.598324in}{1.695743in}}%
\pgfpathlineto{\pgfqpoint{4.590954in}{1.689482in}}%
\pgfpathlineto{\pgfqpoint{4.583595in}{1.683222in}}%
\pgfpathlineto{\pgfqpoint{4.576253in}{1.676962in}}%
\pgfpathlineto{\pgfqpoint{4.568934in}{1.670701in}}%
\pgfpathlineto{\pgfqpoint{4.561643in}{1.664441in}}%
\pgfpathlineto{\pgfqpoint{4.554387in}{1.658180in}}%
\pgfpathlineto{\pgfqpoint{4.547170in}{1.651920in}}%
\pgfpathlineto{\pgfqpoint{4.539997in}{1.645659in}}%
\pgfpathlineto{\pgfqpoint{4.532875in}{1.639399in}}%
\pgfpathlineto{\pgfqpoint{4.525807in}{1.633139in}}%
\pgfpathlineto{\pgfqpoint{4.518799in}{1.626878in}}%
\pgfpathlineto{\pgfqpoint{4.511855in}{1.620618in}}%
\pgfpathlineto{\pgfqpoint{4.504980in}{1.614357in}}%
\pgfpathlineto{\pgfqpoint{4.498177in}{1.608097in}}%
\pgfpathlineto{\pgfqpoint{4.491450in}{1.601836in}}%
\pgfpathlineto{\pgfqpoint{4.484803in}{1.595576in}}%
\pgfpathlineto{\pgfqpoint{4.478240in}{1.589316in}}%
\pgfpathlineto{\pgfqpoint{4.471764in}{1.583055in}}%
\pgfpathlineto{\pgfqpoint{4.465377in}{1.576795in}}%
\pgfpathlineto{\pgfqpoint{4.459084in}{1.570534in}}%
\pgfpathlineto{\pgfqpoint{4.452886in}{1.564274in}}%
\pgfpathlineto{\pgfqpoint{4.446786in}{1.558013in}}%
\pgfpathlineto{\pgfqpoint{4.440786in}{1.551753in}}%
\pgfpathlineto{\pgfqpoint{4.434889in}{1.545492in}}%
\pgfpathlineto{\pgfqpoint{4.429096in}{1.539232in}}%
\pgfpathlineto{\pgfqpoint{4.423409in}{1.532972in}}%
\pgfpathlineto{\pgfqpoint{4.417829in}{1.526711in}}%
\pgfpathlineto{\pgfqpoint{4.412359in}{1.520451in}}%
\pgfpathlineto{\pgfqpoint{4.406998in}{1.514190in}}%
\pgfpathlineto{\pgfqpoint{4.401748in}{1.507930in}}%
\pgfpathlineto{\pgfqpoint{4.396610in}{1.501669in}}%
\pgfpathlineto{\pgfqpoint{4.391584in}{1.495409in}}%
\pgfpathlineto{\pgfqpoint{4.386671in}{1.489149in}}%
\pgfpathlineto{\pgfqpoint{4.381871in}{1.482888in}}%
\pgfpathlineto{\pgfqpoint{4.377184in}{1.476628in}}%
\pgfpathlineto{\pgfqpoint{4.372610in}{1.470367in}}%
\pgfpathlineto{\pgfqpoint{4.368150in}{1.464107in}}%
\pgfpathlineto{\pgfqpoint{4.363802in}{1.457846in}}%
\pgfpathlineto{\pgfqpoint{4.359566in}{1.451586in}}%
\pgfpathlineto{\pgfqpoint{4.355442in}{1.445326in}}%
\pgfpathlineto{\pgfqpoint{4.351429in}{1.439065in}}%
\pgfpathlineto{\pgfqpoint{4.347526in}{1.432805in}}%
\pgfpathlineto{\pgfqpoint{4.343733in}{1.426544in}}%
\pgfpathlineto{\pgfqpoint{4.340048in}{1.420284in}}%
\pgfpathlineto{\pgfqpoint{4.336470in}{1.414023in}}%
\pgfpathlineto{\pgfqpoint{4.332998in}{1.407763in}}%
\pgfpathlineto{\pgfqpoint{4.329631in}{1.401503in}}%
\pgfpathlineto{\pgfqpoint{4.326368in}{1.395242in}}%
\pgfpathlineto{\pgfqpoint{4.323206in}{1.388982in}}%
\pgfpathlineto{\pgfqpoint{4.320144in}{1.382721in}}%
\pgfpathlineto{\pgfqpoint{4.317181in}{1.376461in}}%
\pgfpathlineto{\pgfqpoint{4.314316in}{1.370200in}}%
\pgfpathlineto{\pgfqpoint{4.311546in}{1.363940in}}%
\pgfpathlineto{\pgfqpoint{4.308869in}{1.357679in}}%
\pgfpathlineto{\pgfqpoint{4.306285in}{1.351419in}}%
\pgfpathlineto{\pgfqpoint{4.303790in}{1.345159in}}%
\pgfpathlineto{\pgfqpoint{4.301384in}{1.338898in}}%
\pgfpathlineto{\pgfqpoint{4.299063in}{1.332638in}}%
\pgfpathlineto{\pgfqpoint{4.296828in}{1.326377in}}%
\pgfpathlineto{\pgfqpoint{4.294674in}{1.320117in}}%
\pgfpathlineto{\pgfqpoint{4.292601in}{1.313856in}}%
\pgfpathlineto{\pgfqpoint{4.290607in}{1.307596in}}%
\pgfpathlineto{\pgfqpoint{4.288689in}{1.301336in}}%
\pgfpathlineto{\pgfqpoint{4.286846in}{1.295075in}}%
\pgfpathlineto{\pgfqpoint{4.285075in}{1.288815in}}%
\pgfpathlineto{\pgfqpoint{4.283375in}{1.282554in}}%
\pgfpathlineto{\pgfqpoint{4.281743in}{1.276294in}}%
\pgfpathlineto{\pgfqpoint{4.280178in}{1.270033in}}%
\pgfpathlineto{\pgfqpoint{4.278678in}{1.263773in}}%
\pgfpathlineto{\pgfqpoint{4.277240in}{1.257513in}}%
\pgfpathlineto{\pgfqpoint{4.275863in}{1.251252in}}%
\pgfpathlineto{\pgfqpoint{4.274545in}{1.244992in}}%
\pgfpathlineto{\pgfqpoint{4.273285in}{1.238731in}}%
\pgfpathlineto{\pgfqpoint{4.272079in}{1.232471in}}%
\pgfpathlineto{\pgfqpoint{4.270926in}{1.226210in}}%
\pgfpathlineto{\pgfqpoint{4.269826in}{1.219950in}}%
\pgfpathlineto{\pgfqpoint{4.268774in}{1.213689in}}%
\pgfpathlineto{\pgfqpoint{4.267771in}{1.207429in}}%
\pgfpathlineto{\pgfqpoint{4.266814in}{1.201169in}}%
\pgfpathlineto{\pgfqpoint{4.265902in}{1.194908in}}%
\pgfpathlineto{\pgfqpoint{4.265033in}{1.188648in}}%
\pgfpathlineto{\pgfqpoint{4.264205in}{1.182387in}}%
\pgfpathlineto{\pgfqpoint{4.263416in}{1.176127in}}%
\pgfpathlineto{\pgfqpoint{4.262666in}{1.169866in}}%
\pgfpathlineto{\pgfqpoint{4.261953in}{1.163606in}}%
\pgfpathlineto{\pgfqpoint{4.261275in}{1.157346in}}%
\pgfpathlineto{\pgfqpoint{4.260631in}{1.151085in}}%
\pgfpathlineto{\pgfqpoint{4.260019in}{1.144825in}}%
\pgfpathlineto{\pgfqpoint{4.259438in}{1.138564in}}%
\pgfpathlineto{\pgfqpoint{4.258887in}{1.132304in}}%
\pgfpathlineto{\pgfqpoint{4.258365in}{1.126043in}}%
\pgfpathlineto{\pgfqpoint{4.257870in}{1.119783in}}%
\pgfpathlineto{\pgfqpoint{4.257401in}{1.113523in}}%
\pgfpathlineto{\pgfqpoint{4.256957in}{1.107262in}}%
\pgfpathlineto{\pgfqpoint{4.256537in}{1.101002in}}%
\pgfpathlineto{\pgfqpoint{4.256139in}{1.094741in}}%
\pgfpathlineto{\pgfqpoint{4.255763in}{1.088481in}}%
\pgfpathlineto{\pgfqpoint{4.255408in}{1.082220in}}%
\pgfpathlineto{\pgfqpoint{4.255072in}{1.075960in}}%
\pgfpathlineto{\pgfqpoint{4.254755in}{1.069699in}}%
\pgfpathlineto{\pgfqpoint{4.254457in}{1.063439in}}%
\pgfpathlineto{\pgfqpoint{4.254175in}{1.057179in}}%
\pgfpathlineto{\pgfqpoint{4.253909in}{1.050918in}}%
\pgfpathlineto{\pgfqpoint{4.253658in}{1.044658in}}%
\pgfpathlineto{\pgfqpoint{4.253422in}{1.038397in}}%
\pgfpathlineto{\pgfqpoint{4.253200in}{1.032137in}}%
\pgfpathlineto{\pgfqpoint{4.252991in}{1.025876in}}%
\pgfpathlineto{\pgfqpoint{4.252795in}{1.019616in}}%
\pgfpathlineto{\pgfqpoint{4.252610in}{1.013356in}}%
\pgfpathlineto{\pgfqpoint{4.252436in}{1.007095in}}%
\pgfpathlineto{\pgfqpoint{4.252273in}{1.000835in}}%
\pgfpathlineto{\pgfqpoint{4.252120in}{0.994574in}}%
\pgfpathlineto{\pgfqpoint{4.251976in}{0.988314in}}%
\pgfpathlineto{\pgfqpoint{4.251842in}{0.982053in}}%
\pgfpathlineto{\pgfqpoint{4.251715in}{0.975793in}}%
\pgfpathlineto{\pgfqpoint{4.251597in}{0.969533in}}%
\pgfpathlineto{\pgfqpoint{4.251486in}{0.963272in}}%
\pgfpathlineto{\pgfqpoint{4.251383in}{0.957012in}}%
\pgfpathlineto{\pgfqpoint{4.251286in}{0.950751in}}%
\pgfpathlineto{\pgfqpoint{4.251195in}{0.944491in}}%
\pgfpathlineto{\pgfqpoint{4.251110in}{0.938230in}}%
\pgfpathlineto{\pgfqpoint{4.251031in}{0.931970in}}%
\pgfpathlineto{\pgfqpoint{4.250957in}{0.925710in}}%
\pgfpathlineto{\pgfqpoint{4.250888in}{0.919449in}}%
\pgfpathlineto{\pgfqpoint{4.250823in}{0.913189in}}%
\pgfpathlineto{\pgfqpoint{4.250763in}{0.906928in}}%
\pgfpathlineto{\pgfqpoint{4.250707in}{0.900668in}}%
\pgfpathlineto{\pgfqpoint{4.250655in}{0.894407in}}%
\pgfpathlineto{\pgfqpoint{4.250606in}{0.888147in}}%
\pgfpathlineto{\pgfqpoint{4.250561in}{0.881886in}}%
\pgfpathlineto{\pgfqpoint{4.250519in}{0.875626in}}%
\pgfpathlineto{\pgfqpoint{4.250480in}{0.869366in}}%
\pgfpathlineto{\pgfqpoint{4.250443in}{0.863105in}}%
\pgfpathlineto{\pgfqpoint{4.250410in}{0.856845in}}%
\pgfpathlineto{\pgfqpoint{4.250378in}{0.850584in}}%
\pgfpathlineto{\pgfqpoint{4.250349in}{0.844324in}}%
\pgfpathlineto{\pgfqpoint{4.250322in}{0.838063in}}%
\pgfpathlineto{\pgfqpoint{4.250297in}{0.831803in}}%
\pgfpathlineto{\pgfqpoint{4.250274in}{0.825543in}}%
\pgfpathlineto{\pgfqpoint{4.250252in}{0.819282in}}%
\pgfpathlineto{\pgfqpoint{4.250232in}{0.813022in}}%
\pgfpathlineto{\pgfqpoint{4.250214in}{0.806761in}}%
\pgfpathlineto{\pgfqpoint{4.250197in}{0.800501in}}%
\pgfpathlineto{\pgfqpoint{4.250181in}{0.794240in}}%
\pgfpathlineto{\pgfqpoint{4.250166in}{0.787980in}}%
\pgfpathlineto{\pgfqpoint{4.250153in}{0.781720in}}%
\pgfpathlineto{\pgfqpoint{4.250140in}{0.775459in}}%
\pgfpathlineto{\pgfqpoint{4.250129in}{0.769199in}}%
\pgfpathlineto{\pgfqpoint{4.250118in}{0.762938in}}%
\pgfpathlineto{\pgfqpoint{4.250109in}{0.756678in}}%
\pgfpathlineto{\pgfqpoint{4.250100in}{0.750417in}}%
\pgfpathlineto{\pgfqpoint{4.250091in}{0.744157in}}%
\pgfpathlineto{\pgfqpoint{4.250084in}{0.737896in}}%
\pgfpathlineto{\pgfqpoint{4.250077in}{0.731636in}}%
\pgfpathlineto{\pgfqpoint{4.250070in}{0.725376in}}%
\pgfpathlineto{\pgfqpoint{4.250064in}{0.719115in}}%
\pgfpathlineto{\pgfqpoint{4.250059in}{0.712855in}}%
\pgfpathlineto{\pgfqpoint{4.250054in}{0.706594in}}%
\pgfpathlineto{\pgfqpoint{4.250049in}{0.700334in}}%
\pgfpathlineto{\pgfqpoint{4.250045in}{0.694073in}}%
\pgfpathlineto{\pgfqpoint{4.250041in}{0.687813in}}%
\pgfpathlineto{\pgfqpoint{4.250037in}{0.681553in}}%
\pgfpathlineto{\pgfqpoint{4.250034in}{0.675292in}}%
\pgfpathlineto{\pgfqpoint{4.250031in}{0.669032in}}%
\pgfpathlineto{\pgfqpoint{4.250028in}{0.662771in}}%
\pgfpathlineto{\pgfqpoint{4.250026in}{0.656511in}}%
\pgfpathlineto{\pgfqpoint{4.250023in}{0.650250in}}%
\pgfpathlineto{\pgfqpoint{4.250021in}{0.643990in}}%
\pgfpathlineto{\pgfqpoint{4.250019in}{0.637730in}}%
\pgfpathlineto{\pgfqpoint{4.250018in}{0.631469in}}%
\pgfpathlineto{\pgfqpoint{4.250016in}{0.625209in}}%
\pgfpathlineto{\pgfqpoint{4.250015in}{0.618948in}}%
\pgfpathlineto{\pgfqpoint{4.250013in}{0.612688in}}%
\pgfpathlineto{\pgfqpoint{4.250012in}{0.606427in}}%
\pgfpathlineto{\pgfqpoint{4.250011in}{0.600167in}}%
\pgfpathlineto{\pgfqpoint{4.250010in}{0.593907in}}%
\pgfpathlineto{\pgfqpoint{4.250009in}{0.587646in}}%
\pgfpathlineto{\pgfqpoint{4.250008in}{0.581386in}}%
\pgfpathlineto{\pgfqpoint{4.250007in}{0.575125in}}%
\pgfpathlineto{\pgfqpoint{4.250007in}{0.568865in}}%
\pgfpathlineto{\pgfqpoint{4.250006in}{0.562604in}}%
\pgfpathlineto{\pgfqpoint{4.250005in}{0.556344in}}%
\pgfpathlineto{\pgfqpoint{4.250005in}{0.550083in}}%
\pgfpathlineto{\pgfqpoint{4.250004in}{0.543823in}}%
\pgfpathlineto{\pgfqpoint{4.250004in}{0.537563in}}%
\pgfpathlineto{\pgfqpoint{4.250004in}{0.531302in}}%
\pgfpathlineto{\pgfqpoint{4.250003in}{0.525042in}}%
\pgfpathlineto{\pgfqpoint{4.250003in}{0.518781in}}%
\pgfpathlineto{\pgfqpoint{4.250003in}{0.512521in}}%
\pgfpathlineto{\pgfqpoint{4.250002in}{0.506260in}}%
\pgfpathlineto{\pgfqpoint{4.250002in}{0.500000in}}%
\pgfpathlineto{\pgfqpoint{4.250000in}{0.500000in}}%
\pgfpathlineto{\pgfqpoint{4.250000in}{0.500000in}}%
\pgfpathlineto{\pgfqpoint{4.250000in}{0.506260in}}%
\pgfpathlineto{\pgfqpoint{4.250000in}{0.512521in}}%
\pgfpathlineto{\pgfqpoint{4.250000in}{0.518781in}}%
\pgfpathlineto{\pgfqpoint{4.250000in}{0.525042in}}%
\pgfpathlineto{\pgfqpoint{4.250000in}{0.531302in}}%
\pgfpathlineto{\pgfqpoint{4.250000in}{0.537563in}}%
\pgfpathlineto{\pgfqpoint{4.250000in}{0.543823in}}%
\pgfpathlineto{\pgfqpoint{4.250000in}{0.550083in}}%
\pgfpathlineto{\pgfqpoint{4.250000in}{0.556344in}}%
\pgfpathlineto{\pgfqpoint{4.250000in}{0.562604in}}%
\pgfpathlineto{\pgfqpoint{4.250000in}{0.568865in}}%
\pgfpathlineto{\pgfqpoint{4.250000in}{0.575125in}}%
\pgfpathlineto{\pgfqpoint{4.250000in}{0.581386in}}%
\pgfpathlineto{\pgfqpoint{4.250000in}{0.587646in}}%
\pgfpathlineto{\pgfqpoint{4.250000in}{0.593907in}}%
\pgfpathlineto{\pgfqpoint{4.250000in}{0.600167in}}%
\pgfpathlineto{\pgfqpoint{4.250000in}{0.606427in}}%
\pgfpathlineto{\pgfqpoint{4.250000in}{0.612688in}}%
\pgfpathlineto{\pgfqpoint{4.250000in}{0.618948in}}%
\pgfpathlineto{\pgfqpoint{4.250000in}{0.625209in}}%
\pgfpathlineto{\pgfqpoint{4.250000in}{0.631469in}}%
\pgfpathlineto{\pgfqpoint{4.250000in}{0.637730in}}%
\pgfpathlineto{\pgfqpoint{4.250000in}{0.643990in}}%
\pgfpathlineto{\pgfqpoint{4.250000in}{0.650250in}}%
\pgfpathlineto{\pgfqpoint{4.250000in}{0.656511in}}%
\pgfpathlineto{\pgfqpoint{4.250000in}{0.662771in}}%
\pgfpathlineto{\pgfqpoint{4.250000in}{0.669032in}}%
\pgfpathlineto{\pgfqpoint{4.250000in}{0.675292in}}%
\pgfpathlineto{\pgfqpoint{4.250000in}{0.681553in}}%
\pgfpathlineto{\pgfqpoint{4.250000in}{0.687813in}}%
\pgfpathlineto{\pgfqpoint{4.250000in}{0.694073in}}%
\pgfpathlineto{\pgfqpoint{4.250000in}{0.700334in}}%
\pgfpathlineto{\pgfqpoint{4.250000in}{0.706594in}}%
\pgfpathlineto{\pgfqpoint{4.250000in}{0.712855in}}%
\pgfpathlineto{\pgfqpoint{4.250000in}{0.719115in}}%
\pgfpathlineto{\pgfqpoint{4.250000in}{0.725376in}}%
\pgfpathlineto{\pgfqpoint{4.250000in}{0.731636in}}%
\pgfpathlineto{\pgfqpoint{4.250000in}{0.737896in}}%
\pgfpathlineto{\pgfqpoint{4.250000in}{0.744157in}}%
\pgfpathlineto{\pgfqpoint{4.250000in}{0.750417in}}%
\pgfpathlineto{\pgfqpoint{4.250000in}{0.756678in}}%
\pgfpathlineto{\pgfqpoint{4.250000in}{0.762938in}}%
\pgfpathlineto{\pgfqpoint{4.250000in}{0.769199in}}%
\pgfpathlineto{\pgfqpoint{4.250000in}{0.775459in}}%
\pgfpathlineto{\pgfqpoint{4.250000in}{0.781720in}}%
\pgfpathlineto{\pgfqpoint{4.250000in}{0.787980in}}%
\pgfpathlineto{\pgfqpoint{4.250000in}{0.794240in}}%
\pgfpathlineto{\pgfqpoint{4.250000in}{0.800501in}}%
\pgfpathlineto{\pgfqpoint{4.250000in}{0.806761in}}%
\pgfpathlineto{\pgfqpoint{4.250000in}{0.813022in}}%
\pgfpathlineto{\pgfqpoint{4.250000in}{0.819282in}}%
\pgfpathlineto{\pgfqpoint{4.250000in}{0.825543in}}%
\pgfpathlineto{\pgfqpoint{4.250000in}{0.831803in}}%
\pgfpathlineto{\pgfqpoint{4.250000in}{0.838063in}}%
\pgfpathlineto{\pgfqpoint{4.250000in}{0.844324in}}%
\pgfpathlineto{\pgfqpoint{4.250000in}{0.850584in}}%
\pgfpathlineto{\pgfqpoint{4.250000in}{0.856845in}}%
\pgfpathlineto{\pgfqpoint{4.250000in}{0.863105in}}%
\pgfpathlineto{\pgfqpoint{4.250000in}{0.869366in}}%
\pgfpathlineto{\pgfqpoint{4.250000in}{0.875626in}}%
\pgfpathlineto{\pgfqpoint{4.250000in}{0.881886in}}%
\pgfpathlineto{\pgfqpoint{4.250000in}{0.888147in}}%
\pgfpathlineto{\pgfqpoint{4.250000in}{0.894407in}}%
\pgfpathlineto{\pgfqpoint{4.250000in}{0.900668in}}%
\pgfpathlineto{\pgfqpoint{4.250000in}{0.906928in}}%
\pgfpathlineto{\pgfqpoint{4.250000in}{0.913189in}}%
\pgfpathlineto{\pgfqpoint{4.250000in}{0.919449in}}%
\pgfpathlineto{\pgfqpoint{4.250000in}{0.925710in}}%
\pgfpathlineto{\pgfqpoint{4.250000in}{0.931970in}}%
\pgfpathlineto{\pgfqpoint{4.250000in}{0.938230in}}%
\pgfpathlineto{\pgfqpoint{4.250000in}{0.944491in}}%
\pgfpathlineto{\pgfqpoint{4.250000in}{0.950751in}}%
\pgfpathlineto{\pgfqpoint{4.250000in}{0.957012in}}%
\pgfpathlineto{\pgfqpoint{4.250000in}{0.963272in}}%
\pgfpathlineto{\pgfqpoint{4.250000in}{0.969533in}}%
\pgfpathlineto{\pgfqpoint{4.250000in}{0.975793in}}%
\pgfpathlineto{\pgfqpoint{4.250000in}{0.982053in}}%
\pgfpathlineto{\pgfqpoint{4.250000in}{0.988314in}}%
\pgfpathlineto{\pgfqpoint{4.250000in}{0.994574in}}%
\pgfpathlineto{\pgfqpoint{4.250000in}{1.000835in}}%
\pgfpathlineto{\pgfqpoint{4.250000in}{1.007095in}}%
\pgfpathlineto{\pgfqpoint{4.250000in}{1.013356in}}%
\pgfpathlineto{\pgfqpoint{4.250000in}{1.019616in}}%
\pgfpathlineto{\pgfqpoint{4.250000in}{1.025876in}}%
\pgfpathlineto{\pgfqpoint{4.250000in}{1.032137in}}%
\pgfpathlineto{\pgfqpoint{4.250000in}{1.038397in}}%
\pgfpathlineto{\pgfqpoint{4.250000in}{1.044658in}}%
\pgfpathlineto{\pgfqpoint{4.250000in}{1.050918in}}%
\pgfpathlineto{\pgfqpoint{4.250000in}{1.057179in}}%
\pgfpathlineto{\pgfqpoint{4.250000in}{1.063439in}}%
\pgfpathlineto{\pgfqpoint{4.250000in}{1.069699in}}%
\pgfpathlineto{\pgfqpoint{4.250000in}{1.075960in}}%
\pgfpathlineto{\pgfqpoint{4.250000in}{1.082220in}}%
\pgfpathlineto{\pgfqpoint{4.250000in}{1.088481in}}%
\pgfpathlineto{\pgfqpoint{4.250000in}{1.094741in}}%
\pgfpathlineto{\pgfqpoint{4.250000in}{1.101002in}}%
\pgfpathlineto{\pgfqpoint{4.250000in}{1.107262in}}%
\pgfpathlineto{\pgfqpoint{4.250000in}{1.113523in}}%
\pgfpathlineto{\pgfqpoint{4.250000in}{1.119783in}}%
\pgfpathlineto{\pgfqpoint{4.250000in}{1.126043in}}%
\pgfpathlineto{\pgfqpoint{4.250000in}{1.132304in}}%
\pgfpathlineto{\pgfqpoint{4.250000in}{1.138564in}}%
\pgfpathlineto{\pgfqpoint{4.250000in}{1.144825in}}%
\pgfpathlineto{\pgfqpoint{4.250000in}{1.151085in}}%
\pgfpathlineto{\pgfqpoint{4.250000in}{1.157346in}}%
\pgfpathlineto{\pgfqpoint{4.250000in}{1.163606in}}%
\pgfpathlineto{\pgfqpoint{4.250000in}{1.169866in}}%
\pgfpathlineto{\pgfqpoint{4.250000in}{1.176127in}}%
\pgfpathlineto{\pgfqpoint{4.250000in}{1.182387in}}%
\pgfpathlineto{\pgfqpoint{4.250000in}{1.188648in}}%
\pgfpathlineto{\pgfqpoint{4.250000in}{1.194908in}}%
\pgfpathlineto{\pgfqpoint{4.250000in}{1.201169in}}%
\pgfpathlineto{\pgfqpoint{4.250000in}{1.207429in}}%
\pgfpathlineto{\pgfqpoint{4.250000in}{1.213689in}}%
\pgfpathlineto{\pgfqpoint{4.250000in}{1.219950in}}%
\pgfpathlineto{\pgfqpoint{4.250000in}{1.226210in}}%
\pgfpathlineto{\pgfqpoint{4.250000in}{1.232471in}}%
\pgfpathlineto{\pgfqpoint{4.250000in}{1.238731in}}%
\pgfpathlineto{\pgfqpoint{4.250000in}{1.244992in}}%
\pgfpathlineto{\pgfqpoint{4.250000in}{1.251252in}}%
\pgfpathlineto{\pgfqpoint{4.250000in}{1.257513in}}%
\pgfpathlineto{\pgfqpoint{4.250000in}{1.263773in}}%
\pgfpathlineto{\pgfqpoint{4.250000in}{1.270033in}}%
\pgfpathlineto{\pgfqpoint{4.250000in}{1.276294in}}%
\pgfpathlineto{\pgfqpoint{4.250000in}{1.282554in}}%
\pgfpathlineto{\pgfqpoint{4.250000in}{1.288815in}}%
\pgfpathlineto{\pgfqpoint{4.250000in}{1.295075in}}%
\pgfpathlineto{\pgfqpoint{4.250000in}{1.301336in}}%
\pgfpathlineto{\pgfqpoint{4.250000in}{1.307596in}}%
\pgfpathlineto{\pgfqpoint{4.250000in}{1.313856in}}%
\pgfpathlineto{\pgfqpoint{4.250000in}{1.320117in}}%
\pgfpathlineto{\pgfqpoint{4.250000in}{1.326377in}}%
\pgfpathlineto{\pgfqpoint{4.250000in}{1.332638in}}%
\pgfpathlineto{\pgfqpoint{4.250000in}{1.338898in}}%
\pgfpathlineto{\pgfqpoint{4.250000in}{1.345159in}}%
\pgfpathlineto{\pgfqpoint{4.250000in}{1.351419in}}%
\pgfpathlineto{\pgfqpoint{4.250000in}{1.357679in}}%
\pgfpathlineto{\pgfqpoint{4.250000in}{1.363940in}}%
\pgfpathlineto{\pgfqpoint{4.250000in}{1.370200in}}%
\pgfpathlineto{\pgfqpoint{4.250000in}{1.376461in}}%
\pgfpathlineto{\pgfqpoint{4.250000in}{1.382721in}}%
\pgfpathlineto{\pgfqpoint{4.250000in}{1.388982in}}%
\pgfpathlineto{\pgfqpoint{4.250000in}{1.395242in}}%
\pgfpathlineto{\pgfqpoint{4.250000in}{1.401503in}}%
\pgfpathlineto{\pgfqpoint{4.250000in}{1.407763in}}%
\pgfpathlineto{\pgfqpoint{4.250000in}{1.414023in}}%
\pgfpathlineto{\pgfqpoint{4.250000in}{1.420284in}}%
\pgfpathlineto{\pgfqpoint{4.250000in}{1.426544in}}%
\pgfpathlineto{\pgfqpoint{4.250000in}{1.432805in}}%
\pgfpathlineto{\pgfqpoint{4.250000in}{1.439065in}}%
\pgfpathlineto{\pgfqpoint{4.250000in}{1.445326in}}%
\pgfpathlineto{\pgfqpoint{4.250000in}{1.451586in}}%
\pgfpathlineto{\pgfqpoint{4.250000in}{1.457846in}}%
\pgfpathlineto{\pgfqpoint{4.250000in}{1.464107in}}%
\pgfpathlineto{\pgfqpoint{4.250000in}{1.470367in}}%
\pgfpathlineto{\pgfqpoint{4.250000in}{1.476628in}}%
\pgfpathlineto{\pgfqpoint{4.250000in}{1.482888in}}%
\pgfpathlineto{\pgfqpoint{4.250000in}{1.489149in}}%
\pgfpathlineto{\pgfqpoint{4.250000in}{1.495409in}}%
\pgfpathlineto{\pgfqpoint{4.250000in}{1.501669in}}%
\pgfpathlineto{\pgfqpoint{4.250000in}{1.507930in}}%
\pgfpathlineto{\pgfqpoint{4.250000in}{1.514190in}}%
\pgfpathlineto{\pgfqpoint{4.250000in}{1.520451in}}%
\pgfpathlineto{\pgfqpoint{4.250000in}{1.526711in}}%
\pgfpathlineto{\pgfqpoint{4.250000in}{1.532972in}}%
\pgfpathlineto{\pgfqpoint{4.250000in}{1.539232in}}%
\pgfpathlineto{\pgfqpoint{4.250000in}{1.545492in}}%
\pgfpathlineto{\pgfqpoint{4.250000in}{1.551753in}}%
\pgfpathlineto{\pgfqpoint{4.250000in}{1.558013in}}%
\pgfpathlineto{\pgfqpoint{4.250000in}{1.564274in}}%
\pgfpathlineto{\pgfqpoint{4.250000in}{1.570534in}}%
\pgfpathlineto{\pgfqpoint{4.250000in}{1.576795in}}%
\pgfpathlineto{\pgfqpoint{4.250000in}{1.583055in}}%
\pgfpathlineto{\pgfqpoint{4.250000in}{1.589316in}}%
\pgfpathlineto{\pgfqpoint{4.250000in}{1.595576in}}%
\pgfpathlineto{\pgfqpoint{4.250000in}{1.601836in}}%
\pgfpathlineto{\pgfqpoint{4.250000in}{1.608097in}}%
\pgfpathlineto{\pgfqpoint{4.250000in}{1.614357in}}%
\pgfpathlineto{\pgfqpoint{4.250000in}{1.620618in}}%
\pgfpathlineto{\pgfqpoint{4.250000in}{1.626878in}}%
\pgfpathlineto{\pgfqpoint{4.250000in}{1.633139in}}%
\pgfpathlineto{\pgfqpoint{4.250000in}{1.639399in}}%
\pgfpathlineto{\pgfqpoint{4.250000in}{1.645659in}}%
\pgfpathlineto{\pgfqpoint{4.250000in}{1.651920in}}%
\pgfpathlineto{\pgfqpoint{4.250000in}{1.658180in}}%
\pgfpathlineto{\pgfqpoint{4.250000in}{1.664441in}}%
\pgfpathlineto{\pgfqpoint{4.250000in}{1.670701in}}%
\pgfpathlineto{\pgfqpoint{4.250000in}{1.676962in}}%
\pgfpathlineto{\pgfqpoint{4.250000in}{1.683222in}}%
\pgfpathlineto{\pgfqpoint{4.250000in}{1.689482in}}%
\pgfpathlineto{\pgfqpoint{4.250000in}{1.695743in}}%
\pgfpathlineto{\pgfqpoint{4.250000in}{1.702003in}}%
\pgfpathlineto{\pgfqpoint{4.250000in}{1.708264in}}%
\pgfpathlineto{\pgfqpoint{4.250000in}{1.714524in}}%
\pgfpathlineto{\pgfqpoint{4.250000in}{1.720785in}}%
\pgfpathlineto{\pgfqpoint{4.250000in}{1.727045in}}%
\pgfpathlineto{\pgfqpoint{4.250000in}{1.733306in}}%
\pgfpathlineto{\pgfqpoint{4.250000in}{1.739566in}}%
\pgfpathlineto{\pgfqpoint{4.250000in}{1.745826in}}%
\pgfpathlineto{\pgfqpoint{4.250000in}{1.752087in}}%
\pgfpathlineto{\pgfqpoint{4.250000in}{1.758347in}}%
\pgfpathlineto{\pgfqpoint{4.250000in}{1.764608in}}%
\pgfpathlineto{\pgfqpoint{4.250000in}{1.770868in}}%
\pgfpathlineto{\pgfqpoint{4.250000in}{1.777129in}}%
\pgfpathlineto{\pgfqpoint{4.250000in}{1.783389in}}%
\pgfpathlineto{\pgfqpoint{4.250000in}{1.789649in}}%
\pgfpathlineto{\pgfqpoint{4.250000in}{1.795910in}}%
\pgfpathlineto{\pgfqpoint{4.250000in}{1.802170in}}%
\pgfpathlineto{\pgfqpoint{4.250000in}{1.808431in}}%
\pgfpathlineto{\pgfqpoint{4.250000in}{1.814691in}}%
\pgfpathlineto{\pgfqpoint{4.250000in}{1.820952in}}%
\pgfpathlineto{\pgfqpoint{4.250000in}{1.827212in}}%
\pgfpathlineto{\pgfqpoint{4.250000in}{1.833472in}}%
\pgfpathlineto{\pgfqpoint{4.250000in}{1.839733in}}%
\pgfpathlineto{\pgfqpoint{4.250000in}{1.845993in}}%
\pgfpathlineto{\pgfqpoint{4.250000in}{1.852254in}}%
\pgfpathlineto{\pgfqpoint{4.250000in}{1.858514in}}%
\pgfpathlineto{\pgfqpoint{4.250000in}{1.864775in}}%
\pgfpathlineto{\pgfqpoint{4.250000in}{1.871035in}}%
\pgfpathlineto{\pgfqpoint{4.250000in}{1.877295in}}%
\pgfpathlineto{\pgfqpoint{4.250000in}{1.883556in}}%
\pgfpathlineto{\pgfqpoint{4.250000in}{1.889816in}}%
\pgfpathlineto{\pgfqpoint{4.250000in}{1.896077in}}%
\pgfpathlineto{\pgfqpoint{4.250000in}{1.902337in}}%
\pgfpathlineto{\pgfqpoint{4.250000in}{1.908598in}}%
\pgfpathlineto{\pgfqpoint{4.250000in}{1.914858in}}%
\pgfpathlineto{\pgfqpoint{4.250000in}{1.921119in}}%
\pgfpathlineto{\pgfqpoint{4.250000in}{1.927379in}}%
\pgfpathlineto{\pgfqpoint{4.250000in}{1.933639in}}%
\pgfpathlineto{\pgfqpoint{4.250000in}{1.939900in}}%
\pgfpathlineto{\pgfqpoint{4.250000in}{1.946160in}}%
\pgfpathlineto{\pgfqpoint{4.250000in}{1.952421in}}%
\pgfpathlineto{\pgfqpoint{4.250000in}{1.958681in}}%
\pgfpathlineto{\pgfqpoint{4.250000in}{1.964942in}}%
\pgfpathlineto{\pgfqpoint{4.250000in}{1.971202in}}%
\pgfpathlineto{\pgfqpoint{4.250000in}{1.977462in}}%
\pgfpathlineto{\pgfqpoint{4.250000in}{1.983723in}}%
\pgfpathlineto{\pgfqpoint{4.250000in}{1.989983in}}%
\pgfpathlineto{\pgfqpoint{4.250000in}{1.996244in}}%
\pgfpathlineto{\pgfqpoint{4.250000in}{2.002504in}}%
\pgfpathlineto{\pgfqpoint{4.250000in}{2.008765in}}%
\pgfpathlineto{\pgfqpoint{4.250000in}{2.015025in}}%
\pgfpathlineto{\pgfqpoint{4.250000in}{2.021285in}}%
\pgfpathlineto{\pgfqpoint{4.250000in}{2.027546in}}%
\pgfpathlineto{\pgfqpoint{4.250000in}{2.033806in}}%
\pgfpathlineto{\pgfqpoint{4.250000in}{2.040067in}}%
\pgfpathlineto{\pgfqpoint{4.250000in}{2.046327in}}%
\pgfpathlineto{\pgfqpoint{4.250000in}{2.052588in}}%
\pgfpathlineto{\pgfqpoint{4.250000in}{2.058848in}}%
\pgfpathlineto{\pgfqpoint{4.250000in}{2.065109in}}%
\pgfpathlineto{\pgfqpoint{4.250000in}{2.071369in}}%
\pgfpathlineto{\pgfqpoint{4.250000in}{2.077629in}}%
\pgfpathlineto{\pgfqpoint{4.250000in}{2.083890in}}%
\pgfpathlineto{\pgfqpoint{4.250000in}{2.090150in}}%
\pgfpathlineto{\pgfqpoint{4.250000in}{2.096411in}}%
\pgfpathlineto{\pgfqpoint{4.250000in}{2.102671in}}%
\pgfpathlineto{\pgfqpoint{4.250000in}{2.108932in}}%
\pgfpathlineto{\pgfqpoint{4.250000in}{2.115192in}}%
\pgfpathlineto{\pgfqpoint{4.250000in}{2.121452in}}%
\pgfpathlineto{\pgfqpoint{4.250000in}{2.127713in}}%
\pgfpathlineto{\pgfqpoint{4.250000in}{2.133973in}}%
\pgfpathlineto{\pgfqpoint{4.250000in}{2.140234in}}%
\pgfpathlineto{\pgfqpoint{4.250000in}{2.146494in}}%
\pgfpathlineto{\pgfqpoint{4.250000in}{2.152755in}}%
\pgfpathlineto{\pgfqpoint{4.250000in}{2.159015in}}%
\pgfpathlineto{\pgfqpoint{4.250000in}{2.165275in}}%
\pgfpathlineto{\pgfqpoint{4.250000in}{2.171536in}}%
\pgfpathlineto{\pgfqpoint{4.250000in}{2.177796in}}%
\pgfpathlineto{\pgfqpoint{4.250000in}{2.184057in}}%
\pgfpathlineto{\pgfqpoint{4.250000in}{2.190317in}}%
\pgfpathlineto{\pgfqpoint{4.250000in}{2.196578in}}%
\pgfpathlineto{\pgfqpoint{4.250000in}{2.202838in}}%
\pgfpathlineto{\pgfqpoint{4.250000in}{2.209098in}}%
\pgfpathlineto{\pgfqpoint{4.250000in}{2.215359in}}%
\pgfpathlineto{\pgfqpoint{4.250000in}{2.221619in}}%
\pgfpathlineto{\pgfqpoint{4.250000in}{2.227880in}}%
\pgfpathlineto{\pgfqpoint{4.250000in}{2.234140in}}%
\pgfpathlineto{\pgfqpoint{4.250000in}{2.240401in}}%
\pgfpathlineto{\pgfqpoint{4.250000in}{2.246661in}}%
\pgfpathlineto{\pgfqpoint{4.250000in}{2.252922in}}%
\pgfpathlineto{\pgfqpoint{4.250000in}{2.259182in}}%
\pgfpathlineto{\pgfqpoint{4.250000in}{2.265442in}}%
\pgfpathlineto{\pgfqpoint{4.250000in}{2.271703in}}%
\pgfpathlineto{\pgfqpoint{4.250000in}{2.277963in}}%
\pgfpathlineto{\pgfqpoint{4.250000in}{2.284224in}}%
\pgfpathlineto{\pgfqpoint{4.250000in}{2.290484in}}%
\pgfpathlineto{\pgfqpoint{4.250000in}{2.296745in}}%
\pgfpathlineto{\pgfqpoint{4.250000in}{2.303005in}}%
\pgfpathlineto{\pgfqpoint{4.250000in}{2.309265in}}%
\pgfpathlineto{\pgfqpoint{4.250000in}{2.315526in}}%
\pgfpathlineto{\pgfqpoint{4.250000in}{2.321786in}}%
\pgfpathlineto{\pgfqpoint{4.250000in}{2.328047in}}%
\pgfpathlineto{\pgfqpoint{4.250000in}{2.334307in}}%
\pgfpathlineto{\pgfqpoint{4.250000in}{2.340568in}}%
\pgfpathlineto{\pgfqpoint{4.250000in}{2.346828in}}%
\pgfpathlineto{\pgfqpoint{4.250000in}{2.353088in}}%
\pgfpathlineto{\pgfqpoint{4.250000in}{2.359349in}}%
\pgfpathlineto{\pgfqpoint{4.250000in}{2.365609in}}%
\pgfpathlineto{\pgfqpoint{4.250000in}{2.371870in}}%
\pgfpathlineto{\pgfqpoint{4.250000in}{2.378130in}}%
\pgfpathlineto{\pgfqpoint{4.250000in}{2.384391in}}%
\pgfpathlineto{\pgfqpoint{4.250000in}{2.390651in}}%
\pgfpathlineto{\pgfqpoint{4.250000in}{2.396912in}}%
\pgfpathlineto{\pgfqpoint{4.250000in}{2.403172in}}%
\pgfpathlineto{\pgfqpoint{4.250000in}{2.409432in}}%
\pgfpathlineto{\pgfqpoint{4.250000in}{2.415693in}}%
\pgfpathlineto{\pgfqpoint{4.250000in}{2.421953in}}%
\pgfpathlineto{\pgfqpoint{4.250000in}{2.428214in}}%
\pgfpathlineto{\pgfqpoint{4.250000in}{2.434474in}}%
\pgfpathlineto{\pgfqpoint{4.250000in}{2.440735in}}%
\pgfpathlineto{\pgfqpoint{4.250000in}{2.446995in}}%
\pgfpathlineto{\pgfqpoint{4.250000in}{2.453255in}}%
\pgfpathlineto{\pgfqpoint{4.250000in}{2.459516in}}%
\pgfpathlineto{\pgfqpoint{4.250000in}{2.465776in}}%
\pgfpathlineto{\pgfqpoint{4.250000in}{2.472037in}}%
\pgfpathlineto{\pgfqpoint{4.250000in}{2.478297in}}%
\pgfpathlineto{\pgfqpoint{4.250000in}{2.484558in}}%
\pgfpathlineto{\pgfqpoint{4.250000in}{2.490818in}}%
\pgfpathlineto{\pgfqpoint{4.250000in}{2.497078in}}%
\pgfpathlineto{\pgfqpoint{4.250000in}{2.503339in}}%
\pgfpathlineto{\pgfqpoint{4.250000in}{2.509599in}}%
\pgfpathlineto{\pgfqpoint{4.250000in}{2.515860in}}%
\pgfpathlineto{\pgfqpoint{4.250000in}{2.522120in}}%
\pgfpathlineto{\pgfqpoint{4.250000in}{2.528381in}}%
\pgfpathlineto{\pgfqpoint{4.250000in}{2.534641in}}%
\pgfpathlineto{\pgfqpoint{4.250000in}{2.540902in}}%
\pgfpathlineto{\pgfqpoint{4.250000in}{2.547162in}}%
\pgfpathlineto{\pgfqpoint{4.250000in}{2.553422in}}%
\pgfpathlineto{\pgfqpoint{4.250000in}{2.559683in}}%
\pgfpathlineto{\pgfqpoint{4.250000in}{2.565943in}}%
\pgfpathlineto{\pgfqpoint{4.250000in}{2.572204in}}%
\pgfpathlineto{\pgfqpoint{4.250000in}{2.578464in}}%
\pgfpathlineto{\pgfqpoint{4.250000in}{2.584725in}}%
\pgfpathlineto{\pgfqpoint{4.250000in}{2.590985in}}%
\pgfpathlineto{\pgfqpoint{4.250000in}{2.597245in}}%
\pgfpathlineto{\pgfqpoint{4.250000in}{2.603506in}}%
\pgfpathlineto{\pgfqpoint{4.250000in}{2.609766in}}%
\pgfpathlineto{\pgfqpoint{4.250000in}{2.616027in}}%
\pgfpathlineto{\pgfqpoint{4.250000in}{2.622287in}}%
\pgfpathlineto{\pgfqpoint{4.250000in}{2.628548in}}%
\pgfpathlineto{\pgfqpoint{4.250000in}{2.634808in}}%
\pgfpathlineto{\pgfqpoint{4.250000in}{2.641068in}}%
\pgfpathlineto{\pgfqpoint{4.250000in}{2.647329in}}%
\pgfpathlineto{\pgfqpoint{4.250000in}{2.653589in}}%
\pgfpathlineto{\pgfqpoint{4.250000in}{2.659850in}}%
\pgfpathlineto{\pgfqpoint{4.250000in}{2.666110in}}%
\pgfpathlineto{\pgfqpoint{4.250000in}{2.672371in}}%
\pgfpathlineto{\pgfqpoint{4.250000in}{2.678631in}}%
\pgfpathlineto{\pgfqpoint{4.250000in}{2.684891in}}%
\pgfpathlineto{\pgfqpoint{4.250000in}{2.691152in}}%
\pgfpathlineto{\pgfqpoint{4.250000in}{2.697412in}}%
\pgfpathlineto{\pgfqpoint{4.250000in}{2.703673in}}%
\pgfpathlineto{\pgfqpoint{4.250000in}{2.709933in}}%
\pgfpathlineto{\pgfqpoint{4.250000in}{2.716194in}}%
\pgfpathlineto{\pgfqpoint{4.250000in}{2.722454in}}%
\pgfpathlineto{\pgfqpoint{4.250000in}{2.728715in}}%
\pgfpathlineto{\pgfqpoint{4.250000in}{2.734975in}}%
\pgfpathlineto{\pgfqpoint{4.250000in}{2.741235in}}%
\pgfpathlineto{\pgfqpoint{4.250000in}{2.747496in}}%
\pgfpathlineto{\pgfqpoint{4.250000in}{2.753756in}}%
\pgfpathlineto{\pgfqpoint{4.250000in}{2.760017in}}%
\pgfpathlineto{\pgfqpoint{4.250000in}{2.766277in}}%
\pgfpathlineto{\pgfqpoint{4.250000in}{2.772538in}}%
\pgfpathlineto{\pgfqpoint{4.250000in}{2.778798in}}%
\pgfpathlineto{\pgfqpoint{4.250000in}{2.785058in}}%
\pgfpathlineto{\pgfqpoint{4.250000in}{2.791319in}}%
\pgfpathlineto{\pgfqpoint{4.250000in}{2.797579in}}%
\pgfpathlineto{\pgfqpoint{4.250000in}{2.803840in}}%
\pgfpathlineto{\pgfqpoint{4.250000in}{2.810100in}}%
\pgfpathlineto{\pgfqpoint{4.250000in}{2.816361in}}%
\pgfpathlineto{\pgfqpoint{4.250000in}{2.822621in}}%
\pgfpathlineto{\pgfqpoint{4.250000in}{2.828881in}}%
\pgfpathlineto{\pgfqpoint{4.250000in}{2.835142in}}%
\pgfpathlineto{\pgfqpoint{4.250000in}{2.841402in}}%
\pgfpathlineto{\pgfqpoint{4.250000in}{2.847663in}}%
\pgfpathlineto{\pgfqpoint{4.250000in}{2.853923in}}%
\pgfpathlineto{\pgfqpoint{4.250000in}{2.860184in}}%
\pgfpathlineto{\pgfqpoint{4.250000in}{2.866444in}}%
\pgfpathlineto{\pgfqpoint{4.250000in}{2.872705in}}%
\pgfpathlineto{\pgfqpoint{4.250000in}{2.878965in}}%
\pgfpathlineto{\pgfqpoint{4.250000in}{2.885225in}}%
\pgfpathlineto{\pgfqpoint{4.250000in}{2.891486in}}%
\pgfpathlineto{\pgfqpoint{4.250000in}{2.897746in}}%
\pgfpathlineto{\pgfqpoint{4.250000in}{2.904007in}}%
\pgfpathlineto{\pgfqpoint{4.250000in}{2.910267in}}%
\pgfpathlineto{\pgfqpoint{4.250000in}{2.916528in}}%
\pgfpathlineto{\pgfqpoint{4.250000in}{2.922788in}}%
\pgfpathlineto{\pgfqpoint{4.250000in}{2.929048in}}%
\pgfpathlineto{\pgfqpoint{4.250000in}{2.935309in}}%
\pgfpathlineto{\pgfqpoint{4.250000in}{2.941569in}}%
\pgfpathlineto{\pgfqpoint{4.250000in}{2.947830in}}%
\pgfpathlineto{\pgfqpoint{4.250000in}{2.954090in}}%
\pgfpathlineto{\pgfqpoint{4.250000in}{2.960351in}}%
\pgfpathlineto{\pgfqpoint{4.250000in}{2.966611in}}%
\pgfpathlineto{\pgfqpoint{4.250000in}{2.972871in}}%
\pgfpathlineto{\pgfqpoint{4.250000in}{2.979132in}}%
\pgfpathlineto{\pgfqpoint{4.250000in}{2.985392in}}%
\pgfpathlineto{\pgfqpoint{4.250000in}{2.991653in}}%
\pgfpathlineto{\pgfqpoint{4.250000in}{2.997913in}}%
\pgfpathlineto{\pgfqpoint{4.250000in}{3.004174in}}%
\pgfpathlineto{\pgfqpoint{4.250000in}{3.010434in}}%
\pgfpathlineto{\pgfqpoint{4.250000in}{3.016694in}}%
\pgfpathlineto{\pgfqpoint{4.250000in}{3.022955in}}%
\pgfpathlineto{\pgfqpoint{4.250000in}{3.029215in}}%
\pgfpathlineto{\pgfqpoint{4.250000in}{3.035476in}}%
\pgfpathlineto{\pgfqpoint{4.250000in}{3.041736in}}%
\pgfpathlineto{\pgfqpoint{4.250000in}{3.047997in}}%
\pgfpathlineto{\pgfqpoint{4.250000in}{3.054257in}}%
\pgfpathlineto{\pgfqpoint{4.250000in}{3.060518in}}%
\pgfpathlineto{\pgfqpoint{4.250000in}{3.066778in}}%
\pgfpathlineto{\pgfqpoint{4.250000in}{3.073038in}}%
\pgfpathlineto{\pgfqpoint{4.250000in}{3.079299in}}%
\pgfpathlineto{\pgfqpoint{4.250000in}{3.085559in}}%
\pgfpathlineto{\pgfqpoint{4.250000in}{3.091820in}}%
\pgfpathlineto{\pgfqpoint{4.250000in}{3.098080in}}%
\pgfpathlineto{\pgfqpoint{4.250000in}{3.104341in}}%
\pgfpathlineto{\pgfqpoint{4.250000in}{3.110601in}}%
\pgfpathlineto{\pgfqpoint{4.250000in}{3.116861in}}%
\pgfpathlineto{\pgfqpoint{4.250000in}{3.123122in}}%
\pgfpathlineto{\pgfqpoint{4.250000in}{3.129382in}}%
\pgfpathlineto{\pgfqpoint{4.250000in}{3.135643in}}%
\pgfpathlineto{\pgfqpoint{4.250000in}{3.141903in}}%
\pgfpathlineto{\pgfqpoint{4.250000in}{3.148164in}}%
\pgfpathlineto{\pgfqpoint{4.250000in}{3.154424in}}%
\pgfpathlineto{\pgfqpoint{4.250000in}{3.160684in}}%
\pgfpathlineto{\pgfqpoint{4.250000in}{3.166945in}}%
\pgfpathlineto{\pgfqpoint{4.250000in}{3.173205in}}%
\pgfpathlineto{\pgfqpoint{4.250000in}{3.179466in}}%
\pgfpathlineto{\pgfqpoint{4.250000in}{3.185726in}}%
\pgfpathlineto{\pgfqpoint{4.250000in}{3.191987in}}%
\pgfpathlineto{\pgfqpoint{4.250000in}{3.198247in}}%
\pgfpathlineto{\pgfqpoint{4.250000in}{3.204508in}}%
\pgfpathlineto{\pgfqpoint{4.250000in}{3.210768in}}%
\pgfpathlineto{\pgfqpoint{4.250000in}{3.217028in}}%
\pgfpathlineto{\pgfqpoint{4.250000in}{3.223289in}}%
\pgfpathlineto{\pgfqpoint{4.250000in}{3.229549in}}%
\pgfpathlineto{\pgfqpoint{4.250000in}{3.235810in}}%
\pgfpathlineto{\pgfqpoint{4.250000in}{3.242070in}}%
\pgfpathlineto{\pgfqpoint{4.250000in}{3.248331in}}%
\pgfpathlineto{\pgfqpoint{4.250000in}{3.254591in}}%
\pgfpathlineto{\pgfqpoint{4.250000in}{3.260851in}}%
\pgfpathlineto{\pgfqpoint{4.250000in}{3.267112in}}%
\pgfpathlineto{\pgfqpoint{4.250000in}{3.273372in}}%
\pgfpathlineto{\pgfqpoint{4.250000in}{3.279633in}}%
\pgfpathlineto{\pgfqpoint{4.250000in}{3.285893in}}%
\pgfpathlineto{\pgfqpoint{4.250000in}{3.292154in}}%
\pgfpathlineto{\pgfqpoint{4.250000in}{3.298414in}}%
\pgfpathlineto{\pgfqpoint{4.250000in}{3.304674in}}%
\pgfpathlineto{\pgfqpoint{4.250000in}{3.310935in}}%
\pgfpathlineto{\pgfqpoint{4.250000in}{3.317195in}}%
\pgfpathlineto{\pgfqpoint{4.250000in}{3.323456in}}%
\pgfpathlineto{\pgfqpoint{4.250000in}{3.329716in}}%
\pgfpathlineto{\pgfqpoint{4.250000in}{3.335977in}}%
\pgfpathlineto{\pgfqpoint{4.250000in}{3.342237in}}%
\pgfpathlineto{\pgfqpoint{4.250000in}{3.348497in}}%
\pgfpathlineto{\pgfqpoint{4.250000in}{3.354758in}}%
\pgfpathlineto{\pgfqpoint{4.250000in}{3.361018in}}%
\pgfpathlineto{\pgfqpoint{4.250000in}{3.367279in}}%
\pgfpathlineto{\pgfqpoint{4.250000in}{3.373539in}}%
\pgfpathlineto{\pgfqpoint{4.250000in}{3.379800in}}%
\pgfpathlineto{\pgfqpoint{4.250000in}{3.386060in}}%
\pgfpathlineto{\pgfqpoint{4.250000in}{3.392321in}}%
\pgfpathlineto{\pgfqpoint{4.250000in}{3.398581in}}%
\pgfpathlineto{\pgfqpoint{4.250000in}{3.404841in}}%
\pgfpathlineto{\pgfqpoint{4.250000in}{3.411102in}}%
\pgfpathlineto{\pgfqpoint{4.250000in}{3.417362in}}%
\pgfpathlineto{\pgfqpoint{4.250000in}{3.423623in}}%
\pgfpathlineto{\pgfqpoint{4.250000in}{3.429883in}}%
\pgfpathlineto{\pgfqpoint{4.250000in}{3.436144in}}%
\pgfpathlineto{\pgfqpoint{4.250000in}{3.442404in}}%
\pgfpathlineto{\pgfqpoint{4.250000in}{3.448664in}}%
\pgfpathlineto{\pgfqpoint{4.250000in}{3.454925in}}%
\pgfpathlineto{\pgfqpoint{4.250000in}{3.461185in}}%
\pgfpathlineto{\pgfqpoint{4.250000in}{3.467446in}}%
\pgfpathlineto{\pgfqpoint{4.250000in}{3.473706in}}%
\pgfpathlineto{\pgfqpoint{4.250000in}{3.479967in}}%
\pgfpathlineto{\pgfqpoint{4.250000in}{3.486227in}}%
\pgfpathlineto{\pgfqpoint{4.250000in}{3.492487in}}%
\pgfpathlineto{\pgfqpoint{4.250000in}{3.498748in}}%
\pgfpathlineto{\pgfqpoint{4.250000in}{3.505008in}}%
\pgfpathlineto{\pgfqpoint{4.250000in}{3.511269in}}%
\pgfpathlineto{\pgfqpoint{4.250000in}{3.517529in}}%
\pgfpathlineto{\pgfqpoint{4.250000in}{3.523790in}}%
\pgfpathlineto{\pgfqpoint{4.250000in}{3.530050in}}%
\pgfpathlineto{\pgfqpoint{4.250000in}{3.536311in}}%
\pgfpathlineto{\pgfqpoint{4.250000in}{3.542571in}}%
\pgfpathlineto{\pgfqpoint{4.250000in}{3.548831in}}%
\pgfpathlineto{\pgfqpoint{4.250000in}{3.555092in}}%
\pgfpathlineto{\pgfqpoint{4.250000in}{3.561352in}}%
\pgfpathlineto{\pgfqpoint{4.250000in}{3.567613in}}%
\pgfpathlineto{\pgfqpoint{4.250000in}{3.573873in}}%
\pgfpathlineto{\pgfqpoint{4.250000in}{3.580134in}}%
\pgfpathlineto{\pgfqpoint{4.250000in}{3.586394in}}%
\pgfpathlineto{\pgfqpoint{4.250000in}{3.592654in}}%
\pgfpathlineto{\pgfqpoint{4.250000in}{3.598915in}}%
\pgfpathlineto{\pgfqpoint{4.250000in}{3.605175in}}%
\pgfpathlineto{\pgfqpoint{4.250000in}{3.611436in}}%
\pgfpathlineto{\pgfqpoint{4.250000in}{3.617696in}}%
\pgfpathlineto{\pgfqpoint{4.250000in}{3.623957in}}%
\pgfpathlineto{\pgfqpoint{4.250000in}{3.630217in}}%
\pgfpathlineto{\pgfqpoint{4.250000in}{3.636477in}}%
\pgfpathlineto{\pgfqpoint{4.250000in}{3.642738in}}%
\pgfpathlineto{\pgfqpoint{4.250000in}{3.648998in}}%
\pgfpathlineto{\pgfqpoint{4.250000in}{3.655259in}}%
\pgfpathlineto{\pgfqpoint{4.250000in}{3.661519in}}%
\pgfpathlineto{\pgfqpoint{4.250000in}{3.667780in}}%
\pgfpathlineto{\pgfqpoint{4.250000in}{3.674040in}}%
\pgfpathlineto{\pgfqpoint{4.250000in}{3.680301in}}%
\pgfpathlineto{\pgfqpoint{4.250000in}{3.686561in}}%
\pgfpathlineto{\pgfqpoint{4.250000in}{3.692821in}}%
\pgfpathlineto{\pgfqpoint{4.250000in}{3.699082in}}%
\pgfpathlineto{\pgfqpoint{4.250000in}{3.705342in}}%
\pgfpathlineto{\pgfqpoint{4.250000in}{3.711603in}}%
\pgfpathlineto{\pgfqpoint{4.250000in}{3.717863in}}%
\pgfpathlineto{\pgfqpoint{4.250000in}{3.724124in}}%
\pgfpathlineto{\pgfqpoint{4.250000in}{3.730384in}}%
\pgfpathlineto{\pgfqpoint{4.250000in}{3.736644in}}%
\pgfpathlineto{\pgfqpoint{4.250000in}{3.742905in}}%
\pgfpathlineto{\pgfqpoint{4.250000in}{3.749165in}}%
\pgfpathlineto{\pgfqpoint{4.250000in}{3.755426in}}%
\pgfpathlineto{\pgfqpoint{4.250000in}{3.761686in}}%
\pgfpathlineto{\pgfqpoint{4.250000in}{3.767947in}}%
\pgfpathlineto{\pgfqpoint{4.250000in}{3.774207in}}%
\pgfpathlineto{\pgfqpoint{4.250000in}{3.780467in}}%
\pgfpathlineto{\pgfqpoint{4.250000in}{3.786728in}}%
\pgfpathlineto{\pgfqpoint{4.250000in}{3.792988in}}%
\pgfpathlineto{\pgfqpoint{4.250000in}{3.799249in}}%
\pgfpathlineto{\pgfqpoint{4.250000in}{3.805509in}}%
\pgfpathlineto{\pgfqpoint{4.250000in}{3.811770in}}%
\pgfpathlineto{\pgfqpoint{4.250000in}{3.818030in}}%
\pgfpathlineto{\pgfqpoint{4.250000in}{3.824290in}}%
\pgfpathlineto{\pgfqpoint{4.250000in}{3.830551in}}%
\pgfpathlineto{\pgfqpoint{4.250000in}{3.836811in}}%
\pgfpathlineto{\pgfqpoint{4.250000in}{3.843072in}}%
\pgfpathlineto{\pgfqpoint{4.250000in}{3.849332in}}%
\pgfpathlineto{\pgfqpoint{4.250000in}{3.855593in}}%
\pgfpathlineto{\pgfqpoint{4.250000in}{3.861853in}}%
\pgfpathlineto{\pgfqpoint{4.250000in}{3.868114in}}%
\pgfpathlineto{\pgfqpoint{4.250000in}{3.874374in}}%
\pgfpathlineto{\pgfqpoint{4.250000in}{3.880634in}}%
\pgfpathlineto{\pgfqpoint{4.250000in}{3.886895in}}%
\pgfpathlineto{\pgfqpoint{4.250000in}{3.893155in}}%
\pgfpathlineto{\pgfqpoint{4.250000in}{3.899416in}}%
\pgfpathlineto{\pgfqpoint{4.250000in}{3.905676in}}%
\pgfpathlineto{\pgfqpoint{4.250000in}{3.911937in}}%
\pgfpathlineto{\pgfqpoint{4.250000in}{3.918197in}}%
\pgfpathlineto{\pgfqpoint{4.250000in}{3.924457in}}%
\pgfpathlineto{\pgfqpoint{4.250000in}{3.930718in}}%
\pgfpathlineto{\pgfqpoint{4.250000in}{3.936978in}}%
\pgfpathlineto{\pgfqpoint{4.250000in}{3.943239in}}%
\pgfpathlineto{\pgfqpoint{4.250000in}{3.949499in}}%
\pgfpathlineto{\pgfqpoint{4.250000in}{3.955760in}}%
\pgfpathlineto{\pgfqpoint{4.250000in}{3.962020in}}%
\pgfpathlineto{\pgfqpoint{4.250000in}{3.968280in}}%
\pgfpathlineto{\pgfqpoint{4.250000in}{3.974541in}}%
\pgfpathlineto{\pgfqpoint{4.250000in}{3.980801in}}%
\pgfpathlineto{\pgfqpoint{4.250000in}{3.987062in}}%
\pgfpathlineto{\pgfqpoint{4.250000in}{3.993322in}}%
\pgfpathlineto{\pgfqpoint{4.250000in}{3.999583in}}%
\pgfpathlineto{\pgfqpoint{4.250000in}{4.005843in}}%
\pgfpathlineto{\pgfqpoint{4.250000in}{4.012104in}}%
\pgfpathlineto{\pgfqpoint{4.250000in}{4.018364in}}%
\pgfpathlineto{\pgfqpoint{4.250000in}{4.024624in}}%
\pgfpathlineto{\pgfqpoint{4.250000in}{4.030885in}}%
\pgfpathlineto{\pgfqpoint{4.250000in}{4.037145in}}%
\pgfpathlineto{\pgfqpoint{4.250000in}{4.043406in}}%
\pgfpathlineto{\pgfqpoint{4.250000in}{4.049666in}}%
\pgfpathlineto{\pgfqpoint{4.250000in}{4.055927in}}%
\pgfpathlineto{\pgfqpoint{4.250000in}{4.062187in}}%
\pgfpathlineto{\pgfqpoint{4.250000in}{4.068447in}}%
\pgfpathlineto{\pgfqpoint{4.250000in}{4.074708in}}%
\pgfpathlineto{\pgfqpoint{4.250000in}{4.080968in}}%
\pgfpathlineto{\pgfqpoint{4.250000in}{4.087229in}}%
\pgfpathlineto{\pgfqpoint{4.250000in}{4.093489in}}%
\pgfpathlineto{\pgfqpoint{4.250000in}{4.099750in}}%
\pgfpathlineto{\pgfqpoint{4.250000in}{4.106010in}}%
\pgfpathlineto{\pgfqpoint{4.250000in}{4.112270in}}%
\pgfpathlineto{\pgfqpoint{4.250000in}{4.118531in}}%
\pgfpathlineto{\pgfqpoint{4.250000in}{4.124791in}}%
\pgfpathlineto{\pgfqpoint{4.250000in}{4.131052in}}%
\pgfpathlineto{\pgfqpoint{4.250000in}{4.137312in}}%
\pgfpathlineto{\pgfqpoint{4.250000in}{4.143573in}}%
\pgfpathlineto{\pgfqpoint{4.250000in}{4.149833in}}%
\pgfpathlineto{\pgfqpoint{4.250000in}{4.156093in}}%
\pgfpathlineto{\pgfqpoint{4.250000in}{4.162354in}}%
\pgfpathlineto{\pgfqpoint{4.250000in}{4.168614in}}%
\pgfpathlineto{\pgfqpoint{4.250000in}{4.174875in}}%
\pgfpathlineto{\pgfqpoint{4.250000in}{4.181135in}}%
\pgfpathlineto{\pgfqpoint{4.250000in}{4.187396in}}%
\pgfpathlineto{\pgfqpoint{4.250000in}{4.193656in}}%
\pgfpathlineto{\pgfqpoint{4.250000in}{4.199917in}}%
\pgfpathlineto{\pgfqpoint{4.250000in}{4.206177in}}%
\pgfpathlineto{\pgfqpoint{4.250000in}{4.212437in}}%
\pgfpathlineto{\pgfqpoint{4.250000in}{4.218698in}}%
\pgfpathlineto{\pgfqpoint{4.250000in}{4.224958in}}%
\pgfpathlineto{\pgfqpoint{4.250000in}{4.231219in}}%
\pgfpathlineto{\pgfqpoint{4.250000in}{4.237479in}}%
\pgfpathlineto{\pgfqpoint{4.250000in}{4.243740in}}%
\pgfpathlineto{\pgfqpoint{4.250000in}{4.250000in}}%
\pgfpathclose%
\pgfusepath{stroke,fill}%
}%
\begin{pgfscope}%
\pgfsys@transformshift{0.000000in}{0.000000in}%
\pgfsys@useobject{currentmarker}{}%
\end{pgfscope}%
\end{pgfscope}%
\begin{pgfscope}%
\pgfpathrectangle{\pgfqpoint{4.250000in}{0.500000in}}{\pgfqpoint{0.600000in}{3.750000in}}%
\pgfusepath{clip}%
\pgfsetrectcap%
\pgfsetroundjoin%
\pgfsetlinewidth{2.007500pt}%
\definecolor{currentstroke}{rgb}{0.000000,0.000000,1.000000}%
\pgfsetstrokecolor{currentstroke}%
\pgfsetdash{}{0pt}%
\pgfpathmoveto{\pgfqpoint{4.250000in}{4.250000in}}%
\pgfpathlineto{\pgfqpoint{4.251051in}{3.523790in}}%
\pgfpathlineto{\pgfqpoint{4.253416in}{3.442404in}}%
\pgfpathlineto{\pgfqpoint{4.257157in}{3.386060in}}%
\pgfpathlineto{\pgfqpoint{4.262182in}{3.342237in}}%
\pgfpathlineto{\pgfqpoint{4.268648in}{3.304674in}}%
\pgfpathlineto{\pgfqpoint{4.276032in}{3.273372in}}%
\pgfpathlineto{\pgfqpoint{4.285645in}{3.242070in}}%
\pgfpathlineto{\pgfqpoint{4.297876in}{3.210768in}}%
\pgfpathlineto{\pgfqpoint{4.309785in}{3.185726in}}%
\pgfpathlineto{\pgfqpoint{4.323743in}{3.160684in}}%
\pgfpathlineto{\pgfqpoint{4.339848in}{3.135643in}}%
\pgfpathlineto{\pgfqpoint{4.358134in}{3.110601in}}%
\pgfpathlineto{\pgfqpoint{4.383981in}{3.079299in}}%
\pgfpathlineto{\pgfqpoint{4.412865in}{3.047997in}}%
\pgfpathlineto{\pgfqpoint{4.450755in}{3.010434in}}%
\pgfpathlineto{\pgfqpoint{4.587499in}{2.878965in}}%
\pgfpathlineto{\pgfqpoint{4.609209in}{2.853923in}}%
\pgfpathlineto{\pgfqpoint{4.628028in}{2.828881in}}%
\pgfpathlineto{\pgfqpoint{4.643482in}{2.803840in}}%
\pgfpathlineto{\pgfqpoint{4.652657in}{2.785058in}}%
\pgfpathlineto{\pgfqpoint{4.659663in}{2.766277in}}%
\pgfpathlineto{\pgfqpoint{4.664472in}{2.747496in}}%
\pgfpathlineto{\pgfqpoint{4.667114in}{2.728715in}}%
\pgfpathlineto{\pgfqpoint{4.667677in}{2.709933in}}%
\pgfpathlineto{\pgfqpoint{4.666303in}{2.691152in}}%
\pgfpathlineto{\pgfqpoint{4.661806in}{2.666110in}}%
\pgfpathlineto{\pgfqpoint{4.654811in}{2.641068in}}%
\pgfpathlineto{\pgfqpoint{4.643635in}{2.609766in}}%
\pgfpathlineto{\pgfqpoint{4.610651in}{2.522120in}}%
\pgfpathlineto{\pgfqpoint{4.604096in}{2.497078in}}%
\pgfpathlineto{\pgfqpoint{4.599999in}{2.472037in}}%
\pgfpathlineto{\pgfqpoint{4.598813in}{2.453255in}}%
\pgfpathlineto{\pgfqpoint{4.599397in}{2.434474in}}%
\pgfpathlineto{\pgfqpoint{4.601833in}{2.415693in}}%
\pgfpathlineto{\pgfqpoint{4.606153in}{2.396912in}}%
\pgfpathlineto{\pgfqpoint{4.612339in}{2.378130in}}%
\pgfpathlineto{\pgfqpoint{4.623367in}{2.353088in}}%
\pgfpathlineto{\pgfqpoint{4.637310in}{2.328047in}}%
\pgfpathlineto{\pgfqpoint{4.653769in}{2.303005in}}%
\pgfpathlineto{\pgfqpoint{4.677106in}{2.271703in}}%
\pgfpathlineto{\pgfqpoint{4.718075in}{2.221619in}}%
\pgfpathlineto{\pgfqpoint{4.763785in}{2.165275in}}%
\pgfpathlineto{\pgfqpoint{4.786408in}{2.133973in}}%
\pgfpathlineto{\pgfqpoint{4.802085in}{2.108932in}}%
\pgfpathlineto{\pgfqpoint{4.815061in}{2.083890in}}%
\pgfpathlineto{\pgfqpoint{4.822762in}{2.065109in}}%
\pgfpathlineto{\pgfqpoint{4.828559in}{2.046327in}}%
\pgfpathlineto{\pgfqpoint{4.832334in}{2.027546in}}%
\pgfpathlineto{\pgfqpoint{4.834003in}{2.008765in}}%
\pgfpathlineto{\pgfqpoint{4.833512in}{1.989983in}}%
\pgfpathlineto{\pgfqpoint{4.830838in}{1.971202in}}%
\pgfpathlineto{\pgfqpoint{4.825989in}{1.952421in}}%
\pgfpathlineto{\pgfqpoint{4.819002in}{1.933639in}}%
\pgfpathlineto{\pgfqpoint{4.809945in}{1.914858in}}%
\pgfpathlineto{\pgfqpoint{4.798908in}{1.896077in}}%
\pgfpathlineto{\pgfqpoint{4.781317in}{1.871035in}}%
\pgfpathlineto{\pgfqpoint{4.760747in}{1.845993in}}%
\pgfpathlineto{\pgfqpoint{4.737583in}{1.820952in}}%
\pgfpathlineto{\pgfqpoint{4.705629in}{1.789649in}}%
\pgfpathlineto{\pgfqpoint{4.664057in}{1.752087in}}%
\pgfpathlineto{\pgfqpoint{4.583595in}{1.683222in}}%
\pgfpathlineto{\pgfqpoint{4.511855in}{1.620618in}}%
\pgfpathlineto{\pgfqpoint{4.471764in}{1.583055in}}%
\pgfpathlineto{\pgfqpoint{4.434889in}{1.545492in}}%
\pgfpathlineto{\pgfqpoint{4.406998in}{1.514190in}}%
\pgfpathlineto{\pgfqpoint{4.381871in}{1.482888in}}%
\pgfpathlineto{\pgfqpoint{4.359566in}{1.451586in}}%
\pgfpathlineto{\pgfqpoint{4.340048in}{1.420284in}}%
\pgfpathlineto{\pgfqpoint{4.323206in}{1.388982in}}%
\pgfpathlineto{\pgfqpoint{4.308869in}{1.357679in}}%
\pgfpathlineto{\pgfqpoint{4.296828in}{1.326377in}}%
\pgfpathlineto{\pgfqpoint{4.285075in}{1.288815in}}%
\pgfpathlineto{\pgfqpoint{4.275863in}{1.251252in}}%
\pgfpathlineto{\pgfqpoint{4.267771in}{1.207429in}}%
\pgfpathlineto{\pgfqpoint{4.261953in}{1.163606in}}%
\pgfpathlineto{\pgfqpoint{4.256957in}{1.107262in}}%
\pgfpathlineto{\pgfqpoint{4.253422in}{1.038397in}}%
\pgfpathlineto{\pgfqpoint{4.251195in}{0.944491in}}%
\pgfpathlineto{\pgfqpoint{4.250166in}{0.787980in}}%
\pgfpathlineto{\pgfqpoint{4.250002in}{0.500000in}}%
\pgfpathlineto{\pgfqpoint{4.250002in}{0.500000in}}%
\pgfusepath{stroke}%
\end{pgfscope}%
\begin{pgfscope}%
\pgfsetbuttcap%
\pgfsetmiterjoin%
\definecolor{currentfill}{rgb}{1.000000,1.000000,1.000000}%
\pgfsetfillcolor{currentfill}%
\pgfsetlinewidth{0.000000pt}%
\definecolor{currentstroke}{rgb}{0.000000,0.000000,0.000000}%
\pgfsetstrokecolor{currentstroke}%
\pgfsetstrokeopacity{0.000000}%
\pgfsetdash{}{0pt}%
\pgfpathmoveto{\pgfqpoint{0.500000in}{0.500000in}}%
\pgfpathlineto{\pgfqpoint{4.250000in}{0.500000in}}%
\pgfpathlineto{\pgfqpoint{4.250000in}{4.250000in}}%
\pgfpathlineto{\pgfqpoint{0.500000in}{4.250000in}}%
\pgfpathclose%
\pgfusepath{fill}%
\end{pgfscope}%
\begin{pgfscope}%
\pgfpathrectangle{\pgfqpoint{0.500000in}{0.500000in}}{\pgfqpoint{3.750000in}{3.750000in}}%
\pgfusepath{clip}%
\pgfsetbuttcap%
\pgfsetroundjoin%
\definecolor{currentfill}{rgb}{0.957924,0.933195,0.964383}%
\pgfsetfillcolor{currentfill}%
\pgfsetlinewidth{0.000000pt}%
\definecolor{currentstroke}{rgb}{0.000000,0.000000,0.000000}%
\pgfsetstrokecolor{currentstroke}%
\pgfsetdash{}{0pt}%
\pgfpathmoveto{\pgfqpoint{0.506260in}{0.500000in}}%
\pgfpathlineto{\pgfqpoint{0.512521in}{0.500000in}}%
\pgfpathlineto{\pgfqpoint{0.518781in}{0.500000in}}%
\pgfpathlineto{\pgfqpoint{0.525042in}{0.500000in}}%
\pgfpathlineto{\pgfqpoint{0.531302in}{0.500000in}}%
\pgfpathlineto{\pgfqpoint{0.537563in}{0.500000in}}%
\pgfpathlineto{\pgfqpoint{0.543823in}{0.500000in}}%
\pgfpathlineto{\pgfqpoint{0.550083in}{0.500000in}}%
\pgfpathlineto{\pgfqpoint{0.556344in}{0.500000in}}%
\pgfpathlineto{\pgfqpoint{0.562604in}{0.500000in}}%
\pgfpathlineto{\pgfqpoint{0.568865in}{0.500000in}}%
\pgfpathlineto{\pgfqpoint{0.575125in}{0.500000in}}%
\pgfpathlineto{\pgfqpoint{0.581386in}{0.500000in}}%
\pgfpathlineto{\pgfqpoint{0.587646in}{0.500000in}}%
\pgfpathlineto{\pgfqpoint{0.593907in}{0.500000in}}%
\pgfpathlineto{\pgfqpoint{0.600167in}{0.500000in}}%
\pgfpathlineto{\pgfqpoint{0.606427in}{0.500000in}}%
\pgfpathlineto{\pgfqpoint{0.612688in}{0.500000in}}%
\pgfpathlineto{\pgfqpoint{0.618948in}{0.500000in}}%
\pgfpathlineto{\pgfqpoint{0.625209in}{0.500000in}}%
\pgfpathlineto{\pgfqpoint{0.631469in}{0.500000in}}%
\pgfpathlineto{\pgfqpoint{0.637730in}{0.500000in}}%
\pgfpathlineto{\pgfqpoint{0.643990in}{0.500000in}}%
\pgfpathlineto{\pgfqpoint{0.650250in}{0.500000in}}%
\pgfpathlineto{\pgfqpoint{0.656511in}{0.500000in}}%
\pgfpathlineto{\pgfqpoint{0.662771in}{0.500000in}}%
\pgfpathlineto{\pgfqpoint{0.669032in}{0.500000in}}%
\pgfpathlineto{\pgfqpoint{0.675292in}{0.500000in}}%
\pgfpathlineto{\pgfqpoint{0.681553in}{0.500000in}}%
\pgfpathlineto{\pgfqpoint{0.687813in}{0.500000in}}%
\pgfpathlineto{\pgfqpoint{0.694073in}{0.500000in}}%
\pgfpathlineto{\pgfqpoint{0.700334in}{0.500000in}}%
\pgfpathlineto{\pgfqpoint{0.706594in}{0.500000in}}%
\pgfpathlineto{\pgfqpoint{0.712855in}{0.500000in}}%
\pgfpathlineto{\pgfqpoint{0.719115in}{0.500000in}}%
\pgfpathlineto{\pgfqpoint{0.725376in}{0.500000in}}%
\pgfpathlineto{\pgfqpoint{0.731636in}{0.500000in}}%
\pgfpathlineto{\pgfqpoint{0.737896in}{0.500000in}}%
\pgfpathlineto{\pgfqpoint{0.744157in}{0.500000in}}%
\pgfpathlineto{\pgfqpoint{0.750417in}{0.500000in}}%
\pgfpathlineto{\pgfqpoint{0.756678in}{0.500000in}}%
\pgfpathlineto{\pgfqpoint{0.762938in}{0.500000in}}%
\pgfpathlineto{\pgfqpoint{0.769199in}{0.500000in}}%
\pgfpathlineto{\pgfqpoint{0.775459in}{0.500000in}}%
\pgfpathlineto{\pgfqpoint{0.781720in}{0.500000in}}%
\pgfpathlineto{\pgfqpoint{0.787980in}{0.500000in}}%
\pgfpathlineto{\pgfqpoint{0.794240in}{0.500000in}}%
\pgfpathlineto{\pgfqpoint{0.800501in}{0.500000in}}%
\pgfpathlineto{\pgfqpoint{0.806761in}{0.500000in}}%
\pgfpathlineto{\pgfqpoint{0.813022in}{0.500000in}}%
\pgfpathlineto{\pgfqpoint{0.819282in}{0.500000in}}%
\pgfpathlineto{\pgfqpoint{0.825543in}{0.500000in}}%
\pgfpathlineto{\pgfqpoint{0.831803in}{0.500000in}}%
\pgfpathlineto{\pgfqpoint{0.838063in}{0.500000in}}%
\pgfpathlineto{\pgfqpoint{0.844324in}{0.500000in}}%
\pgfpathlineto{\pgfqpoint{0.850584in}{0.500000in}}%
\pgfpathlineto{\pgfqpoint{0.856845in}{0.500000in}}%
\pgfpathlineto{\pgfqpoint{0.863105in}{0.500000in}}%
\pgfpathlineto{\pgfqpoint{0.869366in}{0.500000in}}%
\pgfpathlineto{\pgfqpoint{0.875626in}{0.500000in}}%
\pgfpathlineto{\pgfqpoint{0.881886in}{0.500000in}}%
\pgfpathlineto{\pgfqpoint{0.888147in}{0.500000in}}%
\pgfpathlineto{\pgfqpoint{0.894407in}{0.500000in}}%
\pgfpathlineto{\pgfqpoint{0.900668in}{0.500000in}}%
\pgfpathlineto{\pgfqpoint{0.906928in}{0.500000in}}%
\pgfpathlineto{\pgfqpoint{0.913189in}{0.500000in}}%
\pgfpathlineto{\pgfqpoint{0.919449in}{0.500000in}}%
\pgfpathlineto{\pgfqpoint{0.925710in}{0.500000in}}%
\pgfpathlineto{\pgfqpoint{0.931970in}{0.500000in}}%
\pgfpathlineto{\pgfqpoint{0.938230in}{0.500000in}}%
\pgfpathlineto{\pgfqpoint{0.944491in}{0.500000in}}%
\pgfpathlineto{\pgfqpoint{0.950751in}{0.500000in}}%
\pgfpathlineto{\pgfqpoint{0.957012in}{0.500000in}}%
\pgfpathlineto{\pgfqpoint{0.963272in}{0.500000in}}%
\pgfpathlineto{\pgfqpoint{0.969533in}{0.500000in}}%
\pgfpathlineto{\pgfqpoint{0.975793in}{0.500000in}}%
\pgfpathlineto{\pgfqpoint{0.982053in}{0.500000in}}%
\pgfpathlineto{\pgfqpoint{0.988314in}{0.500000in}}%
\pgfpathlineto{\pgfqpoint{0.994574in}{0.500000in}}%
\pgfpathlineto{\pgfqpoint{1.000835in}{0.500000in}}%
\pgfpathlineto{\pgfqpoint{1.007095in}{0.500000in}}%
\pgfpathlineto{\pgfqpoint{1.013356in}{0.500000in}}%
\pgfpathlineto{\pgfqpoint{1.019616in}{0.500000in}}%
\pgfpathlineto{\pgfqpoint{1.025876in}{0.500000in}}%
\pgfpathlineto{\pgfqpoint{1.032137in}{0.500000in}}%
\pgfpathlineto{\pgfqpoint{1.038397in}{0.500000in}}%
\pgfpathlineto{\pgfqpoint{1.044658in}{0.500000in}}%
\pgfpathlineto{\pgfqpoint{1.050918in}{0.500000in}}%
\pgfpathlineto{\pgfqpoint{1.057179in}{0.500000in}}%
\pgfpathlineto{\pgfqpoint{1.063439in}{0.500000in}}%
\pgfpathlineto{\pgfqpoint{1.069699in}{0.500000in}}%
\pgfpathlineto{\pgfqpoint{1.075960in}{0.500000in}}%
\pgfpathlineto{\pgfqpoint{1.082220in}{0.500000in}}%
\pgfpathlineto{\pgfqpoint{1.088481in}{0.500000in}}%
\pgfpathlineto{\pgfqpoint{1.094741in}{0.500000in}}%
\pgfpathlineto{\pgfqpoint{1.101002in}{0.500000in}}%
\pgfpathlineto{\pgfqpoint{1.107262in}{0.500000in}}%
\pgfpathlineto{\pgfqpoint{1.113523in}{0.500000in}}%
\pgfpathlineto{\pgfqpoint{1.119783in}{0.500000in}}%
\pgfpathlineto{\pgfqpoint{1.126043in}{0.500000in}}%
\pgfpathlineto{\pgfqpoint{1.132304in}{0.500000in}}%
\pgfpathlineto{\pgfqpoint{1.138564in}{0.500000in}}%
\pgfpathlineto{\pgfqpoint{1.144825in}{0.500000in}}%
\pgfpathlineto{\pgfqpoint{1.151085in}{0.500000in}}%
\pgfpathlineto{\pgfqpoint{1.157346in}{0.500000in}}%
\pgfpathlineto{\pgfqpoint{1.163606in}{0.500000in}}%
\pgfpathlineto{\pgfqpoint{1.169866in}{0.500000in}}%
\pgfpathlineto{\pgfqpoint{1.176127in}{0.500000in}}%
\pgfpathlineto{\pgfqpoint{1.182387in}{0.500000in}}%
\pgfpathlineto{\pgfqpoint{1.188648in}{0.500000in}}%
\pgfpathlineto{\pgfqpoint{1.194908in}{0.500000in}}%
\pgfpathlineto{\pgfqpoint{1.201169in}{0.500000in}}%
\pgfpathlineto{\pgfqpoint{1.207429in}{0.500000in}}%
\pgfpathlineto{\pgfqpoint{1.213689in}{0.500000in}}%
\pgfpathlineto{\pgfqpoint{1.219950in}{0.500000in}}%
\pgfpathlineto{\pgfqpoint{1.226210in}{0.500000in}}%
\pgfpathlineto{\pgfqpoint{1.232471in}{0.500000in}}%
\pgfpathlineto{\pgfqpoint{1.238731in}{0.500000in}}%
\pgfpathlineto{\pgfqpoint{1.244992in}{0.500000in}}%
\pgfpathlineto{\pgfqpoint{1.251252in}{0.500000in}}%
\pgfpathlineto{\pgfqpoint{1.257513in}{0.500000in}}%
\pgfpathlineto{\pgfqpoint{1.263773in}{0.500000in}}%
\pgfpathlineto{\pgfqpoint{1.270033in}{0.500000in}}%
\pgfpathlineto{\pgfqpoint{1.276294in}{0.500000in}}%
\pgfpathlineto{\pgfqpoint{1.282554in}{0.500000in}}%
\pgfpathlineto{\pgfqpoint{1.288815in}{0.500000in}}%
\pgfpathlineto{\pgfqpoint{1.295075in}{0.500000in}}%
\pgfpathlineto{\pgfqpoint{1.301336in}{0.500000in}}%
\pgfpathlineto{\pgfqpoint{1.307596in}{0.500000in}}%
\pgfpathlineto{\pgfqpoint{1.313856in}{0.500000in}}%
\pgfpathlineto{\pgfqpoint{1.320117in}{0.500000in}}%
\pgfpathlineto{\pgfqpoint{1.326377in}{0.500000in}}%
\pgfpathlineto{\pgfqpoint{1.332638in}{0.500000in}}%
\pgfpathlineto{\pgfqpoint{1.338898in}{0.500000in}}%
\pgfpathlineto{\pgfqpoint{1.345159in}{0.500000in}}%
\pgfpathlineto{\pgfqpoint{1.351419in}{0.500000in}}%
\pgfpathlineto{\pgfqpoint{1.357679in}{0.500000in}}%
\pgfpathlineto{\pgfqpoint{1.363940in}{0.500000in}}%
\pgfpathlineto{\pgfqpoint{1.370200in}{0.500000in}}%
\pgfpathlineto{\pgfqpoint{1.376461in}{0.500000in}}%
\pgfpathlineto{\pgfqpoint{1.382721in}{0.500000in}}%
\pgfpathlineto{\pgfqpoint{1.388982in}{0.500000in}}%
\pgfpathlineto{\pgfqpoint{1.395242in}{0.500000in}}%
\pgfpathlineto{\pgfqpoint{1.401503in}{0.500000in}}%
\pgfpathlineto{\pgfqpoint{1.407763in}{0.500000in}}%
\pgfpathlineto{\pgfqpoint{1.414023in}{0.500000in}}%
\pgfpathlineto{\pgfqpoint{1.420284in}{0.500000in}}%
\pgfpathlineto{\pgfqpoint{1.426544in}{0.500000in}}%
\pgfpathlineto{\pgfqpoint{1.432805in}{0.500000in}}%
\pgfpathlineto{\pgfqpoint{1.439065in}{0.500000in}}%
\pgfpathlineto{\pgfqpoint{1.445326in}{0.500000in}}%
\pgfpathlineto{\pgfqpoint{1.451586in}{0.500000in}}%
\pgfpathlineto{\pgfqpoint{1.457846in}{0.500000in}}%
\pgfpathlineto{\pgfqpoint{1.464107in}{0.500000in}}%
\pgfpathlineto{\pgfqpoint{1.470367in}{0.500000in}}%
\pgfpathlineto{\pgfqpoint{1.476628in}{0.500000in}}%
\pgfpathlineto{\pgfqpoint{1.482888in}{0.500000in}}%
\pgfpathlineto{\pgfqpoint{1.489149in}{0.500000in}}%
\pgfpathlineto{\pgfqpoint{1.495409in}{0.500000in}}%
\pgfpathlineto{\pgfqpoint{1.501669in}{0.500000in}}%
\pgfpathlineto{\pgfqpoint{1.507930in}{0.500000in}}%
\pgfpathlineto{\pgfqpoint{1.514190in}{0.500000in}}%
\pgfpathlineto{\pgfqpoint{1.520451in}{0.500000in}}%
\pgfpathlineto{\pgfqpoint{1.526711in}{0.500000in}}%
\pgfpathlineto{\pgfqpoint{1.532972in}{0.500000in}}%
\pgfpathlineto{\pgfqpoint{1.539232in}{0.500000in}}%
\pgfpathlineto{\pgfqpoint{1.545492in}{0.500000in}}%
\pgfpathlineto{\pgfqpoint{1.551753in}{0.500000in}}%
\pgfpathlineto{\pgfqpoint{1.558013in}{0.500000in}}%
\pgfpathlineto{\pgfqpoint{1.564274in}{0.500000in}}%
\pgfpathlineto{\pgfqpoint{1.570534in}{0.500000in}}%
\pgfpathlineto{\pgfqpoint{1.576795in}{0.500000in}}%
\pgfpathlineto{\pgfqpoint{1.583055in}{0.500000in}}%
\pgfpathlineto{\pgfqpoint{1.589316in}{0.500000in}}%
\pgfpathlineto{\pgfqpoint{1.595576in}{0.500000in}}%
\pgfpathlineto{\pgfqpoint{1.601836in}{0.500000in}}%
\pgfpathlineto{\pgfqpoint{1.608097in}{0.500000in}}%
\pgfpathlineto{\pgfqpoint{1.614357in}{0.500000in}}%
\pgfpathlineto{\pgfqpoint{1.620618in}{0.500000in}}%
\pgfpathlineto{\pgfqpoint{1.626878in}{0.500000in}}%
\pgfpathlineto{\pgfqpoint{1.633139in}{0.500000in}}%
\pgfpathlineto{\pgfqpoint{1.639399in}{0.500000in}}%
\pgfpathlineto{\pgfqpoint{1.645659in}{0.500000in}}%
\pgfpathlineto{\pgfqpoint{1.651920in}{0.500000in}}%
\pgfpathlineto{\pgfqpoint{1.658180in}{0.500000in}}%
\pgfpathlineto{\pgfqpoint{1.664441in}{0.500000in}}%
\pgfpathlineto{\pgfqpoint{1.670701in}{0.500000in}}%
\pgfpathlineto{\pgfqpoint{1.676962in}{0.500000in}}%
\pgfpathlineto{\pgfqpoint{1.683222in}{0.500000in}}%
\pgfpathlineto{\pgfqpoint{1.689482in}{0.500000in}}%
\pgfpathlineto{\pgfqpoint{1.695743in}{0.500000in}}%
\pgfpathlineto{\pgfqpoint{1.702003in}{0.500000in}}%
\pgfpathlineto{\pgfqpoint{1.708264in}{0.500000in}}%
\pgfpathlineto{\pgfqpoint{1.714524in}{0.500000in}}%
\pgfpathlineto{\pgfqpoint{1.720785in}{0.500000in}}%
\pgfpathlineto{\pgfqpoint{1.727045in}{0.500000in}}%
\pgfpathlineto{\pgfqpoint{1.733306in}{0.500000in}}%
\pgfpathlineto{\pgfqpoint{1.739566in}{0.500000in}}%
\pgfpathlineto{\pgfqpoint{1.745826in}{0.500000in}}%
\pgfpathlineto{\pgfqpoint{1.752087in}{0.500000in}}%
\pgfpathlineto{\pgfqpoint{1.758347in}{0.500000in}}%
\pgfpathlineto{\pgfqpoint{1.764608in}{0.500000in}}%
\pgfpathlineto{\pgfqpoint{1.770868in}{0.500000in}}%
\pgfpathlineto{\pgfqpoint{1.777129in}{0.500000in}}%
\pgfpathlineto{\pgfqpoint{1.783389in}{0.500000in}}%
\pgfpathlineto{\pgfqpoint{1.789649in}{0.500000in}}%
\pgfpathlineto{\pgfqpoint{1.795910in}{0.500000in}}%
\pgfpathlineto{\pgfqpoint{1.802170in}{0.500000in}}%
\pgfpathlineto{\pgfqpoint{1.808431in}{0.500000in}}%
\pgfpathlineto{\pgfqpoint{1.814691in}{0.500000in}}%
\pgfpathlineto{\pgfqpoint{1.820952in}{0.500000in}}%
\pgfpathlineto{\pgfqpoint{1.827212in}{0.500000in}}%
\pgfpathlineto{\pgfqpoint{1.833472in}{0.500000in}}%
\pgfpathlineto{\pgfqpoint{1.839733in}{0.500000in}}%
\pgfpathlineto{\pgfqpoint{1.845993in}{0.500000in}}%
\pgfpathlineto{\pgfqpoint{1.852254in}{0.500000in}}%
\pgfpathlineto{\pgfqpoint{1.858514in}{0.500000in}}%
\pgfpathlineto{\pgfqpoint{1.864775in}{0.500000in}}%
\pgfpathlineto{\pgfqpoint{1.871035in}{0.500000in}}%
\pgfpathlineto{\pgfqpoint{1.877295in}{0.500000in}}%
\pgfpathlineto{\pgfqpoint{1.883556in}{0.500000in}}%
\pgfpathlineto{\pgfqpoint{1.889816in}{0.500000in}}%
\pgfpathlineto{\pgfqpoint{1.896077in}{0.500000in}}%
\pgfpathlineto{\pgfqpoint{1.902337in}{0.500000in}}%
\pgfpathlineto{\pgfqpoint{1.908598in}{0.500000in}}%
\pgfpathlineto{\pgfqpoint{1.914858in}{0.500000in}}%
\pgfpathlineto{\pgfqpoint{1.921119in}{0.500000in}}%
\pgfpathlineto{\pgfqpoint{1.927379in}{0.500000in}}%
\pgfpathlineto{\pgfqpoint{1.933639in}{0.500000in}}%
\pgfpathlineto{\pgfqpoint{1.939900in}{0.500000in}}%
\pgfpathlineto{\pgfqpoint{1.946160in}{0.500000in}}%
\pgfpathlineto{\pgfqpoint{1.952421in}{0.500000in}}%
\pgfpathlineto{\pgfqpoint{1.958681in}{0.500000in}}%
\pgfpathlineto{\pgfqpoint{1.964942in}{0.500000in}}%
\pgfpathlineto{\pgfqpoint{1.971202in}{0.500000in}}%
\pgfpathlineto{\pgfqpoint{1.977462in}{0.500000in}}%
\pgfpathlineto{\pgfqpoint{1.983723in}{0.500000in}}%
\pgfpathlineto{\pgfqpoint{1.989983in}{0.500000in}}%
\pgfpathlineto{\pgfqpoint{1.996244in}{0.500000in}}%
\pgfpathlineto{\pgfqpoint{2.002504in}{0.500000in}}%
\pgfpathlineto{\pgfqpoint{2.008765in}{0.500000in}}%
\pgfpathlineto{\pgfqpoint{2.015025in}{0.500000in}}%
\pgfpathlineto{\pgfqpoint{2.021285in}{0.500000in}}%
\pgfpathlineto{\pgfqpoint{2.027546in}{0.500000in}}%
\pgfpathlineto{\pgfqpoint{2.033806in}{0.500000in}}%
\pgfpathlineto{\pgfqpoint{2.040067in}{0.500000in}}%
\pgfpathlineto{\pgfqpoint{2.046327in}{0.500000in}}%
\pgfpathlineto{\pgfqpoint{2.052588in}{0.500000in}}%
\pgfpathlineto{\pgfqpoint{2.058848in}{0.500000in}}%
\pgfpathlineto{\pgfqpoint{2.065109in}{0.500000in}}%
\pgfpathlineto{\pgfqpoint{2.071369in}{0.500000in}}%
\pgfpathlineto{\pgfqpoint{2.077629in}{0.500000in}}%
\pgfpathlineto{\pgfqpoint{2.083890in}{0.500000in}}%
\pgfpathlineto{\pgfqpoint{2.090150in}{0.500000in}}%
\pgfpathlineto{\pgfqpoint{2.096411in}{0.500000in}}%
\pgfpathlineto{\pgfqpoint{2.102671in}{0.500000in}}%
\pgfpathlineto{\pgfqpoint{2.108932in}{0.500000in}}%
\pgfpathlineto{\pgfqpoint{2.115192in}{0.500000in}}%
\pgfpathlineto{\pgfqpoint{2.121452in}{0.500000in}}%
\pgfpathlineto{\pgfqpoint{2.127713in}{0.500000in}}%
\pgfpathlineto{\pgfqpoint{2.133973in}{0.500000in}}%
\pgfpathlineto{\pgfqpoint{2.140234in}{0.500000in}}%
\pgfpathlineto{\pgfqpoint{2.146494in}{0.500000in}}%
\pgfpathlineto{\pgfqpoint{2.152755in}{0.500000in}}%
\pgfpathlineto{\pgfqpoint{2.159015in}{0.500000in}}%
\pgfpathlineto{\pgfqpoint{2.165275in}{0.500000in}}%
\pgfpathlineto{\pgfqpoint{2.171536in}{0.500000in}}%
\pgfpathlineto{\pgfqpoint{2.177796in}{0.500000in}}%
\pgfpathlineto{\pgfqpoint{2.184057in}{0.500000in}}%
\pgfpathlineto{\pgfqpoint{2.190317in}{0.500000in}}%
\pgfpathlineto{\pgfqpoint{2.196578in}{0.500000in}}%
\pgfpathlineto{\pgfqpoint{2.202838in}{0.500000in}}%
\pgfpathlineto{\pgfqpoint{2.209098in}{0.500000in}}%
\pgfpathlineto{\pgfqpoint{2.215359in}{0.500000in}}%
\pgfpathlineto{\pgfqpoint{2.221619in}{0.500000in}}%
\pgfpathlineto{\pgfqpoint{2.227880in}{0.500000in}}%
\pgfpathlineto{\pgfqpoint{2.234140in}{0.500000in}}%
\pgfpathlineto{\pgfqpoint{2.240401in}{0.500000in}}%
\pgfpathlineto{\pgfqpoint{2.246661in}{0.500000in}}%
\pgfpathlineto{\pgfqpoint{2.252922in}{0.500000in}}%
\pgfpathlineto{\pgfqpoint{2.259182in}{0.500000in}}%
\pgfpathlineto{\pgfqpoint{2.265442in}{0.500000in}}%
\pgfpathlineto{\pgfqpoint{2.271703in}{0.500000in}}%
\pgfpathlineto{\pgfqpoint{2.277963in}{0.500000in}}%
\pgfpathlineto{\pgfqpoint{2.284224in}{0.500000in}}%
\pgfpathlineto{\pgfqpoint{2.290484in}{0.500000in}}%
\pgfpathlineto{\pgfqpoint{2.296745in}{0.500000in}}%
\pgfpathlineto{\pgfqpoint{2.303005in}{0.500000in}}%
\pgfpathlineto{\pgfqpoint{2.309265in}{0.500000in}}%
\pgfpathlineto{\pgfqpoint{2.315526in}{0.500000in}}%
\pgfpathlineto{\pgfqpoint{2.321786in}{0.500000in}}%
\pgfpathlineto{\pgfqpoint{2.328047in}{0.500000in}}%
\pgfpathlineto{\pgfqpoint{2.334307in}{0.500000in}}%
\pgfpathlineto{\pgfqpoint{2.340568in}{0.500000in}}%
\pgfpathlineto{\pgfqpoint{2.346828in}{0.500000in}}%
\pgfpathlineto{\pgfqpoint{2.353088in}{0.500000in}}%
\pgfpathlineto{\pgfqpoint{2.359349in}{0.500000in}}%
\pgfpathlineto{\pgfqpoint{2.365609in}{0.500000in}}%
\pgfpathlineto{\pgfqpoint{2.371870in}{0.500000in}}%
\pgfpathlineto{\pgfqpoint{2.378130in}{0.500000in}}%
\pgfpathlineto{\pgfqpoint{2.384391in}{0.500000in}}%
\pgfpathlineto{\pgfqpoint{2.390651in}{0.500000in}}%
\pgfpathlineto{\pgfqpoint{2.396912in}{0.500000in}}%
\pgfpathlineto{\pgfqpoint{2.403172in}{0.500000in}}%
\pgfpathlineto{\pgfqpoint{2.409432in}{0.500000in}}%
\pgfpathlineto{\pgfqpoint{2.415693in}{0.500000in}}%
\pgfpathlineto{\pgfqpoint{2.421953in}{0.500000in}}%
\pgfpathlineto{\pgfqpoint{2.428214in}{0.500000in}}%
\pgfpathlineto{\pgfqpoint{2.434474in}{0.500000in}}%
\pgfpathlineto{\pgfqpoint{2.440735in}{0.500000in}}%
\pgfpathlineto{\pgfqpoint{2.446995in}{0.500000in}}%
\pgfpathlineto{\pgfqpoint{2.453255in}{0.500000in}}%
\pgfpathlineto{\pgfqpoint{2.459516in}{0.500000in}}%
\pgfpathlineto{\pgfqpoint{2.465776in}{0.500000in}}%
\pgfpathlineto{\pgfqpoint{2.472037in}{0.500000in}}%
\pgfpathlineto{\pgfqpoint{2.478297in}{0.500000in}}%
\pgfpathlineto{\pgfqpoint{2.484558in}{0.500000in}}%
\pgfpathlineto{\pgfqpoint{2.490818in}{0.500000in}}%
\pgfpathlineto{\pgfqpoint{2.497078in}{0.500000in}}%
\pgfpathlineto{\pgfqpoint{2.503339in}{0.500000in}}%
\pgfpathlineto{\pgfqpoint{2.509599in}{0.500000in}}%
\pgfpathlineto{\pgfqpoint{2.515860in}{0.500000in}}%
\pgfpathlineto{\pgfqpoint{2.522120in}{0.500000in}}%
\pgfpathlineto{\pgfqpoint{2.528381in}{0.500000in}}%
\pgfpathlineto{\pgfqpoint{2.534641in}{0.500000in}}%
\pgfpathlineto{\pgfqpoint{2.540902in}{0.500000in}}%
\pgfpathlineto{\pgfqpoint{2.547162in}{0.500000in}}%
\pgfpathlineto{\pgfqpoint{2.553422in}{0.500000in}}%
\pgfpathlineto{\pgfqpoint{2.559683in}{0.500000in}}%
\pgfpathlineto{\pgfqpoint{2.565943in}{0.500000in}}%
\pgfpathlineto{\pgfqpoint{2.572204in}{0.500000in}}%
\pgfpathlineto{\pgfqpoint{2.578464in}{0.500000in}}%
\pgfpathlineto{\pgfqpoint{2.584725in}{0.500000in}}%
\pgfpathlineto{\pgfqpoint{2.590985in}{0.500000in}}%
\pgfpathlineto{\pgfqpoint{2.597245in}{0.500000in}}%
\pgfpathlineto{\pgfqpoint{2.603506in}{0.500000in}}%
\pgfpathlineto{\pgfqpoint{2.609766in}{0.500000in}}%
\pgfpathlineto{\pgfqpoint{2.616027in}{0.500000in}}%
\pgfpathlineto{\pgfqpoint{2.622287in}{0.500000in}}%
\pgfpathlineto{\pgfqpoint{2.628548in}{0.500000in}}%
\pgfpathlineto{\pgfqpoint{2.634808in}{0.500000in}}%
\pgfpathlineto{\pgfqpoint{2.641068in}{0.500000in}}%
\pgfpathlineto{\pgfqpoint{2.647329in}{0.500000in}}%
\pgfpathlineto{\pgfqpoint{2.653589in}{0.500000in}}%
\pgfpathlineto{\pgfqpoint{2.659850in}{0.500000in}}%
\pgfpathlineto{\pgfqpoint{2.666110in}{0.500000in}}%
\pgfpathlineto{\pgfqpoint{2.672371in}{0.500000in}}%
\pgfpathlineto{\pgfqpoint{2.678631in}{0.500000in}}%
\pgfpathlineto{\pgfqpoint{2.684891in}{0.500000in}}%
\pgfpathlineto{\pgfqpoint{2.691152in}{0.500000in}}%
\pgfpathlineto{\pgfqpoint{2.697412in}{0.500000in}}%
\pgfpathlineto{\pgfqpoint{2.703673in}{0.500000in}}%
\pgfpathlineto{\pgfqpoint{2.709933in}{0.500000in}}%
\pgfpathlineto{\pgfqpoint{2.716194in}{0.500000in}}%
\pgfpathlineto{\pgfqpoint{2.722454in}{0.500000in}}%
\pgfpathlineto{\pgfqpoint{2.728715in}{0.500000in}}%
\pgfpathlineto{\pgfqpoint{2.734975in}{0.500000in}}%
\pgfpathlineto{\pgfqpoint{2.741235in}{0.500000in}}%
\pgfpathlineto{\pgfqpoint{2.747496in}{0.500000in}}%
\pgfpathlineto{\pgfqpoint{2.753756in}{0.500000in}}%
\pgfpathlineto{\pgfqpoint{2.760017in}{0.500000in}}%
\pgfpathlineto{\pgfqpoint{2.766277in}{0.500000in}}%
\pgfpathlineto{\pgfqpoint{2.772538in}{0.500000in}}%
\pgfpathlineto{\pgfqpoint{2.778798in}{0.500000in}}%
\pgfpathlineto{\pgfqpoint{2.785058in}{0.500000in}}%
\pgfpathlineto{\pgfqpoint{2.791319in}{0.500000in}}%
\pgfpathlineto{\pgfqpoint{2.797579in}{0.500000in}}%
\pgfpathlineto{\pgfqpoint{2.803840in}{0.500000in}}%
\pgfpathlineto{\pgfqpoint{2.810100in}{0.500000in}}%
\pgfpathlineto{\pgfqpoint{2.816361in}{0.500000in}}%
\pgfpathlineto{\pgfqpoint{2.822621in}{0.500000in}}%
\pgfpathlineto{\pgfqpoint{2.828881in}{0.500000in}}%
\pgfpathlineto{\pgfqpoint{2.835142in}{0.500000in}}%
\pgfpathlineto{\pgfqpoint{2.841402in}{0.500000in}}%
\pgfpathlineto{\pgfqpoint{2.847663in}{0.500000in}}%
\pgfpathlineto{\pgfqpoint{2.853923in}{0.500000in}}%
\pgfpathlineto{\pgfqpoint{2.860184in}{0.500000in}}%
\pgfpathlineto{\pgfqpoint{2.866444in}{0.500000in}}%
\pgfpathlineto{\pgfqpoint{2.872705in}{0.500000in}}%
\pgfpathlineto{\pgfqpoint{2.878965in}{0.500000in}}%
\pgfpathlineto{\pgfqpoint{2.885225in}{0.500000in}}%
\pgfpathlineto{\pgfqpoint{2.891486in}{0.500000in}}%
\pgfpathlineto{\pgfqpoint{2.897746in}{0.500000in}}%
\pgfpathlineto{\pgfqpoint{2.904007in}{0.500000in}}%
\pgfpathlineto{\pgfqpoint{2.910267in}{0.500000in}}%
\pgfpathlineto{\pgfqpoint{2.916528in}{0.500000in}}%
\pgfpathlineto{\pgfqpoint{2.922788in}{0.500000in}}%
\pgfpathlineto{\pgfqpoint{2.929048in}{0.500000in}}%
\pgfpathlineto{\pgfqpoint{2.935309in}{0.500000in}}%
\pgfpathlineto{\pgfqpoint{2.941569in}{0.500000in}}%
\pgfpathlineto{\pgfqpoint{2.947830in}{0.500000in}}%
\pgfpathlineto{\pgfqpoint{2.954090in}{0.500000in}}%
\pgfpathlineto{\pgfqpoint{2.960351in}{0.500000in}}%
\pgfpathlineto{\pgfqpoint{2.966611in}{0.500000in}}%
\pgfpathlineto{\pgfqpoint{2.972871in}{0.500000in}}%
\pgfpathlineto{\pgfqpoint{2.979132in}{0.500000in}}%
\pgfpathlineto{\pgfqpoint{2.985392in}{0.500000in}}%
\pgfpathlineto{\pgfqpoint{2.991653in}{0.500000in}}%
\pgfpathlineto{\pgfqpoint{2.997913in}{0.500000in}}%
\pgfpathlineto{\pgfqpoint{3.004174in}{0.500000in}}%
\pgfpathlineto{\pgfqpoint{3.010434in}{0.500000in}}%
\pgfpathlineto{\pgfqpoint{3.016694in}{0.500000in}}%
\pgfpathlineto{\pgfqpoint{3.022955in}{0.500000in}}%
\pgfpathlineto{\pgfqpoint{3.029215in}{0.500000in}}%
\pgfpathlineto{\pgfqpoint{3.035476in}{0.500000in}}%
\pgfpathlineto{\pgfqpoint{3.041736in}{0.500000in}}%
\pgfpathlineto{\pgfqpoint{3.047997in}{0.500000in}}%
\pgfpathlineto{\pgfqpoint{3.054257in}{0.500000in}}%
\pgfpathlineto{\pgfqpoint{3.060518in}{0.500000in}}%
\pgfpathlineto{\pgfqpoint{3.066778in}{0.500000in}}%
\pgfpathlineto{\pgfqpoint{3.073038in}{0.500000in}}%
\pgfpathlineto{\pgfqpoint{3.079299in}{0.500000in}}%
\pgfpathlineto{\pgfqpoint{3.085559in}{0.500000in}}%
\pgfpathlineto{\pgfqpoint{3.091820in}{0.500000in}}%
\pgfpathlineto{\pgfqpoint{3.098080in}{0.500000in}}%
\pgfpathlineto{\pgfqpoint{3.104341in}{0.500000in}}%
\pgfpathlineto{\pgfqpoint{3.110601in}{0.500000in}}%
\pgfpathlineto{\pgfqpoint{3.116861in}{0.500000in}}%
\pgfpathlineto{\pgfqpoint{3.123122in}{0.500000in}}%
\pgfpathlineto{\pgfqpoint{3.129382in}{0.500000in}}%
\pgfpathlineto{\pgfqpoint{3.135643in}{0.500000in}}%
\pgfpathlineto{\pgfqpoint{3.141903in}{0.500000in}}%
\pgfpathlineto{\pgfqpoint{3.148164in}{0.500000in}}%
\pgfpathlineto{\pgfqpoint{3.154424in}{0.500000in}}%
\pgfpathlineto{\pgfqpoint{3.160684in}{0.500000in}}%
\pgfpathlineto{\pgfqpoint{3.166945in}{0.500000in}}%
\pgfpathlineto{\pgfqpoint{3.173205in}{0.500000in}}%
\pgfpathlineto{\pgfqpoint{3.179466in}{0.500000in}}%
\pgfpathlineto{\pgfqpoint{3.185726in}{0.500000in}}%
\pgfpathlineto{\pgfqpoint{3.191987in}{0.500000in}}%
\pgfpathlineto{\pgfqpoint{3.198247in}{0.500000in}}%
\pgfpathlineto{\pgfqpoint{3.204508in}{0.500000in}}%
\pgfpathlineto{\pgfqpoint{3.210768in}{0.500000in}}%
\pgfpathlineto{\pgfqpoint{3.217028in}{0.500000in}}%
\pgfpathlineto{\pgfqpoint{3.223289in}{0.500000in}}%
\pgfpathlineto{\pgfqpoint{3.229549in}{0.500000in}}%
\pgfpathlineto{\pgfqpoint{3.235810in}{0.500000in}}%
\pgfpathlineto{\pgfqpoint{3.242070in}{0.500000in}}%
\pgfpathlineto{\pgfqpoint{3.248331in}{0.500000in}}%
\pgfpathlineto{\pgfqpoint{3.254591in}{0.500000in}}%
\pgfpathlineto{\pgfqpoint{3.260851in}{0.500000in}}%
\pgfpathlineto{\pgfqpoint{3.267112in}{0.500000in}}%
\pgfpathlineto{\pgfqpoint{3.273372in}{0.500000in}}%
\pgfpathlineto{\pgfqpoint{3.279633in}{0.500000in}}%
\pgfpathlineto{\pgfqpoint{3.285893in}{0.500000in}}%
\pgfpathlineto{\pgfqpoint{3.292154in}{0.500000in}}%
\pgfpathlineto{\pgfqpoint{3.298414in}{0.500000in}}%
\pgfpathlineto{\pgfqpoint{3.304674in}{0.500000in}}%
\pgfpathlineto{\pgfqpoint{3.310935in}{0.500000in}}%
\pgfpathlineto{\pgfqpoint{3.317195in}{0.500000in}}%
\pgfpathlineto{\pgfqpoint{3.323456in}{0.500000in}}%
\pgfpathlineto{\pgfqpoint{3.329716in}{0.500000in}}%
\pgfpathlineto{\pgfqpoint{3.335977in}{0.500000in}}%
\pgfpathlineto{\pgfqpoint{3.342237in}{0.500000in}}%
\pgfpathlineto{\pgfqpoint{3.348497in}{0.500000in}}%
\pgfpathlineto{\pgfqpoint{3.354758in}{0.500000in}}%
\pgfpathlineto{\pgfqpoint{3.361018in}{0.500000in}}%
\pgfpathlineto{\pgfqpoint{3.367279in}{0.500000in}}%
\pgfpathlineto{\pgfqpoint{3.373539in}{0.500000in}}%
\pgfpathlineto{\pgfqpoint{3.379800in}{0.500000in}}%
\pgfpathlineto{\pgfqpoint{3.386060in}{0.500000in}}%
\pgfpathlineto{\pgfqpoint{3.392321in}{0.500000in}}%
\pgfpathlineto{\pgfqpoint{3.398581in}{0.500000in}}%
\pgfpathlineto{\pgfqpoint{3.404841in}{0.500000in}}%
\pgfpathlineto{\pgfqpoint{3.411102in}{0.500000in}}%
\pgfpathlineto{\pgfqpoint{3.417362in}{0.500000in}}%
\pgfpathlineto{\pgfqpoint{3.423623in}{0.500000in}}%
\pgfpathlineto{\pgfqpoint{3.429883in}{0.500000in}}%
\pgfpathlineto{\pgfqpoint{3.436144in}{0.500000in}}%
\pgfpathlineto{\pgfqpoint{3.442404in}{0.500000in}}%
\pgfpathlineto{\pgfqpoint{3.448664in}{0.500000in}}%
\pgfpathlineto{\pgfqpoint{3.454925in}{0.500000in}}%
\pgfpathlineto{\pgfqpoint{3.461185in}{0.500000in}}%
\pgfpathlineto{\pgfqpoint{3.467446in}{0.500000in}}%
\pgfpathlineto{\pgfqpoint{3.473706in}{0.500000in}}%
\pgfpathlineto{\pgfqpoint{3.479967in}{0.500000in}}%
\pgfpathlineto{\pgfqpoint{3.486227in}{0.500000in}}%
\pgfpathlineto{\pgfqpoint{3.492487in}{0.500000in}}%
\pgfpathlineto{\pgfqpoint{3.498748in}{0.500000in}}%
\pgfpathlineto{\pgfqpoint{3.505008in}{0.500000in}}%
\pgfpathlineto{\pgfqpoint{3.511269in}{0.500000in}}%
\pgfpathlineto{\pgfqpoint{3.517529in}{0.500000in}}%
\pgfpathlineto{\pgfqpoint{3.523790in}{0.500000in}}%
\pgfpathlineto{\pgfqpoint{3.530050in}{0.500000in}}%
\pgfpathlineto{\pgfqpoint{3.536311in}{0.500000in}}%
\pgfpathlineto{\pgfqpoint{3.542571in}{0.500000in}}%
\pgfpathlineto{\pgfqpoint{3.548831in}{0.500000in}}%
\pgfpathlineto{\pgfqpoint{3.555092in}{0.500000in}}%
\pgfpathlineto{\pgfqpoint{3.561352in}{0.500000in}}%
\pgfpathlineto{\pgfqpoint{3.567613in}{0.500000in}}%
\pgfpathlineto{\pgfqpoint{3.573873in}{0.500000in}}%
\pgfpathlineto{\pgfqpoint{3.580134in}{0.500000in}}%
\pgfpathlineto{\pgfqpoint{3.586394in}{0.500000in}}%
\pgfpathlineto{\pgfqpoint{3.592654in}{0.500000in}}%
\pgfpathlineto{\pgfqpoint{3.598915in}{0.500000in}}%
\pgfpathlineto{\pgfqpoint{3.605175in}{0.500000in}}%
\pgfpathlineto{\pgfqpoint{3.611436in}{0.500000in}}%
\pgfpathlineto{\pgfqpoint{3.617696in}{0.500000in}}%
\pgfpathlineto{\pgfqpoint{3.623957in}{0.500000in}}%
\pgfpathlineto{\pgfqpoint{3.630217in}{0.500000in}}%
\pgfpathlineto{\pgfqpoint{3.636477in}{0.500000in}}%
\pgfpathlineto{\pgfqpoint{3.642738in}{0.500000in}}%
\pgfpathlineto{\pgfqpoint{3.648998in}{0.500000in}}%
\pgfpathlineto{\pgfqpoint{3.655259in}{0.500000in}}%
\pgfpathlineto{\pgfqpoint{3.661519in}{0.500000in}}%
\pgfpathlineto{\pgfqpoint{3.667780in}{0.500000in}}%
\pgfpathlineto{\pgfqpoint{3.674040in}{0.500000in}}%
\pgfpathlineto{\pgfqpoint{3.680301in}{0.500000in}}%
\pgfpathlineto{\pgfqpoint{3.686561in}{0.500000in}}%
\pgfpathlineto{\pgfqpoint{3.692821in}{0.500000in}}%
\pgfpathlineto{\pgfqpoint{3.699082in}{0.500000in}}%
\pgfpathlineto{\pgfqpoint{3.705342in}{0.500000in}}%
\pgfpathlineto{\pgfqpoint{3.711603in}{0.500000in}}%
\pgfpathlineto{\pgfqpoint{3.717863in}{0.500000in}}%
\pgfpathlineto{\pgfqpoint{3.724124in}{0.500000in}}%
\pgfpathlineto{\pgfqpoint{3.730384in}{0.500000in}}%
\pgfpathlineto{\pgfqpoint{3.736644in}{0.500000in}}%
\pgfpathlineto{\pgfqpoint{3.742905in}{0.500000in}}%
\pgfpathlineto{\pgfqpoint{3.749165in}{0.500000in}}%
\pgfpathlineto{\pgfqpoint{3.755426in}{0.500000in}}%
\pgfpathlineto{\pgfqpoint{3.761686in}{0.500000in}}%
\pgfpathlineto{\pgfqpoint{3.767947in}{0.500000in}}%
\pgfpathlineto{\pgfqpoint{3.774207in}{0.500000in}}%
\pgfpathlineto{\pgfqpoint{3.780467in}{0.500000in}}%
\pgfpathlineto{\pgfqpoint{3.786728in}{0.500000in}}%
\pgfpathlineto{\pgfqpoint{3.792988in}{0.500000in}}%
\pgfpathlineto{\pgfqpoint{3.799249in}{0.500000in}}%
\pgfpathlineto{\pgfqpoint{3.805509in}{0.500000in}}%
\pgfpathlineto{\pgfqpoint{3.811770in}{0.500000in}}%
\pgfpathlineto{\pgfqpoint{3.818030in}{0.500000in}}%
\pgfpathlineto{\pgfqpoint{3.824290in}{0.500000in}}%
\pgfpathlineto{\pgfqpoint{3.830551in}{0.500000in}}%
\pgfpathlineto{\pgfqpoint{3.836811in}{0.500000in}}%
\pgfpathlineto{\pgfqpoint{3.843072in}{0.500000in}}%
\pgfpathlineto{\pgfqpoint{3.849332in}{0.500000in}}%
\pgfpathlineto{\pgfqpoint{3.855593in}{0.500000in}}%
\pgfpathlineto{\pgfqpoint{3.861853in}{0.500000in}}%
\pgfpathlineto{\pgfqpoint{3.868114in}{0.500000in}}%
\pgfpathlineto{\pgfqpoint{3.874374in}{0.500000in}}%
\pgfpathlineto{\pgfqpoint{3.880634in}{0.500000in}}%
\pgfpathlineto{\pgfqpoint{3.886895in}{0.500000in}}%
\pgfpathlineto{\pgfqpoint{3.893155in}{0.500000in}}%
\pgfpathlineto{\pgfqpoint{3.899416in}{0.500000in}}%
\pgfpathlineto{\pgfqpoint{3.905676in}{0.500000in}}%
\pgfpathlineto{\pgfqpoint{3.911937in}{0.500000in}}%
\pgfpathlineto{\pgfqpoint{3.918197in}{0.500000in}}%
\pgfpathlineto{\pgfqpoint{3.924457in}{0.500000in}}%
\pgfpathlineto{\pgfqpoint{3.930718in}{0.500000in}}%
\pgfpathlineto{\pgfqpoint{3.936978in}{0.500000in}}%
\pgfpathlineto{\pgfqpoint{3.943239in}{0.500000in}}%
\pgfpathlineto{\pgfqpoint{3.949499in}{0.500000in}}%
\pgfpathlineto{\pgfqpoint{3.955760in}{0.500000in}}%
\pgfpathlineto{\pgfqpoint{3.962020in}{0.500000in}}%
\pgfpathlineto{\pgfqpoint{3.968280in}{0.500000in}}%
\pgfpathlineto{\pgfqpoint{3.974541in}{0.500000in}}%
\pgfpathlineto{\pgfqpoint{3.980801in}{0.500000in}}%
\pgfpathlineto{\pgfqpoint{3.987062in}{0.500000in}}%
\pgfpathlineto{\pgfqpoint{3.993322in}{0.500000in}}%
\pgfpathlineto{\pgfqpoint{3.999583in}{0.500000in}}%
\pgfpathlineto{\pgfqpoint{4.005843in}{0.500000in}}%
\pgfpathlineto{\pgfqpoint{4.012104in}{0.500000in}}%
\pgfpathlineto{\pgfqpoint{4.018364in}{0.500000in}}%
\pgfpathlineto{\pgfqpoint{4.024624in}{0.500000in}}%
\pgfpathlineto{\pgfqpoint{4.030885in}{0.500000in}}%
\pgfpathlineto{\pgfqpoint{4.037145in}{0.500000in}}%
\pgfpathlineto{\pgfqpoint{4.043406in}{0.500000in}}%
\pgfpathlineto{\pgfqpoint{4.049666in}{0.500000in}}%
\pgfpathlineto{\pgfqpoint{4.055927in}{0.500000in}}%
\pgfpathlineto{\pgfqpoint{4.062187in}{0.500000in}}%
\pgfpathlineto{\pgfqpoint{4.068447in}{0.500000in}}%
\pgfpathlineto{\pgfqpoint{4.074708in}{0.500000in}}%
\pgfpathlineto{\pgfqpoint{4.080968in}{0.500000in}}%
\pgfpathlineto{\pgfqpoint{4.087229in}{0.500000in}}%
\pgfpathlineto{\pgfqpoint{4.093489in}{0.500000in}}%
\pgfpathlineto{\pgfqpoint{4.099750in}{0.500000in}}%
\pgfpathlineto{\pgfqpoint{4.106010in}{0.500000in}}%
\pgfpathlineto{\pgfqpoint{4.112270in}{0.500000in}}%
\pgfpathlineto{\pgfqpoint{4.118531in}{0.500000in}}%
\pgfpathlineto{\pgfqpoint{4.124791in}{0.500000in}}%
\pgfpathlineto{\pgfqpoint{4.131052in}{0.500000in}}%
\pgfpathlineto{\pgfqpoint{4.137312in}{0.500000in}}%
\pgfpathlineto{\pgfqpoint{4.143573in}{0.500000in}}%
\pgfpathlineto{\pgfqpoint{4.149833in}{0.500000in}}%
\pgfpathlineto{\pgfqpoint{4.156093in}{0.500000in}}%
\pgfpathlineto{\pgfqpoint{4.162354in}{0.500000in}}%
\pgfpathlineto{\pgfqpoint{4.168614in}{0.500000in}}%
\pgfpathlineto{\pgfqpoint{4.174875in}{0.500000in}}%
\pgfpathlineto{\pgfqpoint{4.181135in}{0.500000in}}%
\pgfpathlineto{\pgfqpoint{4.187396in}{0.500000in}}%
\pgfpathlineto{\pgfqpoint{4.193656in}{0.500000in}}%
\pgfpathlineto{\pgfqpoint{4.199917in}{0.500000in}}%
\pgfpathlineto{\pgfqpoint{4.206177in}{0.500000in}}%
\pgfpathlineto{\pgfqpoint{4.212437in}{0.500000in}}%
\pgfpathlineto{\pgfqpoint{4.218698in}{0.500000in}}%
\pgfpathlineto{\pgfqpoint{4.224958in}{0.500000in}}%
\pgfpathlineto{\pgfqpoint{4.231219in}{0.500000in}}%
\pgfpathlineto{\pgfqpoint{4.237479in}{0.500000in}}%
\pgfpathlineto{\pgfqpoint{4.243740in}{0.500000in}}%
\pgfpathlineto{\pgfqpoint{4.250000in}{0.500000in}}%
\pgfpathlineto{\pgfqpoint{4.250000in}{0.506260in}}%
\pgfpathlineto{\pgfqpoint{4.250000in}{0.512521in}}%
\pgfpathlineto{\pgfqpoint{4.250000in}{0.518781in}}%
\pgfpathlineto{\pgfqpoint{4.250000in}{0.525042in}}%
\pgfpathlineto{\pgfqpoint{4.250000in}{0.531302in}}%
\pgfpathlineto{\pgfqpoint{4.250000in}{0.537563in}}%
\pgfpathlineto{\pgfqpoint{4.250000in}{0.543823in}}%
\pgfpathlineto{\pgfqpoint{4.250000in}{0.550083in}}%
\pgfpathlineto{\pgfqpoint{4.250000in}{0.556344in}}%
\pgfpathlineto{\pgfqpoint{4.250000in}{0.562604in}}%
\pgfpathlineto{\pgfqpoint{4.250000in}{0.568865in}}%
\pgfpathlineto{\pgfqpoint{4.250000in}{0.575125in}}%
\pgfpathlineto{\pgfqpoint{4.250000in}{0.581386in}}%
\pgfpathlineto{\pgfqpoint{4.250000in}{0.587646in}}%
\pgfpathlineto{\pgfqpoint{4.250000in}{0.593907in}}%
\pgfpathlineto{\pgfqpoint{4.250000in}{0.600167in}}%
\pgfpathlineto{\pgfqpoint{4.250000in}{0.606427in}}%
\pgfpathlineto{\pgfqpoint{4.250000in}{0.612688in}}%
\pgfpathlineto{\pgfqpoint{4.250000in}{0.618948in}}%
\pgfpathlineto{\pgfqpoint{4.250000in}{0.625209in}}%
\pgfpathlineto{\pgfqpoint{4.250000in}{0.631469in}}%
\pgfpathlineto{\pgfqpoint{4.250000in}{0.637730in}}%
\pgfpathlineto{\pgfqpoint{4.250000in}{0.643990in}}%
\pgfpathlineto{\pgfqpoint{4.250000in}{0.650250in}}%
\pgfpathlineto{\pgfqpoint{4.250000in}{0.656511in}}%
\pgfpathlineto{\pgfqpoint{4.250000in}{0.662771in}}%
\pgfpathlineto{\pgfqpoint{4.250000in}{0.669032in}}%
\pgfpathlineto{\pgfqpoint{4.250000in}{0.675292in}}%
\pgfpathlineto{\pgfqpoint{4.250000in}{0.681553in}}%
\pgfpathlineto{\pgfqpoint{4.250000in}{0.687813in}}%
\pgfpathlineto{\pgfqpoint{4.250000in}{0.694073in}}%
\pgfpathlineto{\pgfqpoint{4.250000in}{0.700334in}}%
\pgfpathlineto{\pgfqpoint{4.250000in}{0.706594in}}%
\pgfpathlineto{\pgfqpoint{4.250000in}{0.712855in}}%
\pgfpathlineto{\pgfqpoint{4.250000in}{0.719115in}}%
\pgfpathlineto{\pgfqpoint{4.250000in}{0.725376in}}%
\pgfpathlineto{\pgfqpoint{4.250000in}{0.731636in}}%
\pgfpathlineto{\pgfqpoint{4.250000in}{0.737896in}}%
\pgfpathlineto{\pgfqpoint{4.250000in}{0.744157in}}%
\pgfpathlineto{\pgfqpoint{4.250000in}{0.750417in}}%
\pgfpathlineto{\pgfqpoint{4.250000in}{0.756678in}}%
\pgfpathlineto{\pgfqpoint{4.250000in}{0.762938in}}%
\pgfpathlineto{\pgfqpoint{4.250000in}{0.769199in}}%
\pgfpathlineto{\pgfqpoint{4.250000in}{0.775459in}}%
\pgfpathlineto{\pgfqpoint{4.250000in}{0.781720in}}%
\pgfpathlineto{\pgfqpoint{4.250000in}{0.787980in}}%
\pgfpathlineto{\pgfqpoint{4.250000in}{0.794240in}}%
\pgfpathlineto{\pgfqpoint{4.250000in}{0.800501in}}%
\pgfpathlineto{\pgfqpoint{4.250000in}{0.806761in}}%
\pgfpathlineto{\pgfqpoint{4.250000in}{0.813022in}}%
\pgfpathlineto{\pgfqpoint{4.250000in}{0.819282in}}%
\pgfpathlineto{\pgfqpoint{4.250000in}{0.825543in}}%
\pgfpathlineto{\pgfqpoint{4.250000in}{0.831803in}}%
\pgfpathlineto{\pgfqpoint{4.250000in}{0.838063in}}%
\pgfpathlineto{\pgfqpoint{4.250000in}{0.844324in}}%
\pgfpathlineto{\pgfqpoint{4.250000in}{0.850584in}}%
\pgfpathlineto{\pgfqpoint{4.250000in}{0.856845in}}%
\pgfpathlineto{\pgfqpoint{4.250000in}{0.863105in}}%
\pgfpathlineto{\pgfqpoint{4.250000in}{0.869366in}}%
\pgfpathlineto{\pgfqpoint{4.250000in}{0.875626in}}%
\pgfpathlineto{\pgfqpoint{4.250000in}{0.881886in}}%
\pgfpathlineto{\pgfqpoint{4.250000in}{0.888147in}}%
\pgfpathlineto{\pgfqpoint{4.250000in}{0.894407in}}%
\pgfpathlineto{\pgfqpoint{4.250000in}{0.900668in}}%
\pgfpathlineto{\pgfqpoint{4.250000in}{0.906928in}}%
\pgfpathlineto{\pgfqpoint{4.250000in}{0.913189in}}%
\pgfpathlineto{\pgfqpoint{4.250000in}{0.919449in}}%
\pgfpathlineto{\pgfqpoint{4.250000in}{0.925710in}}%
\pgfpathlineto{\pgfqpoint{4.250000in}{0.931970in}}%
\pgfpathlineto{\pgfqpoint{4.250000in}{0.938230in}}%
\pgfpathlineto{\pgfqpoint{4.250000in}{0.944491in}}%
\pgfpathlineto{\pgfqpoint{4.250000in}{0.950751in}}%
\pgfpathlineto{\pgfqpoint{4.250000in}{0.957012in}}%
\pgfpathlineto{\pgfqpoint{4.250000in}{0.963272in}}%
\pgfpathlineto{\pgfqpoint{4.250000in}{0.969533in}}%
\pgfpathlineto{\pgfqpoint{4.250000in}{0.975793in}}%
\pgfpathlineto{\pgfqpoint{4.250000in}{0.982053in}}%
\pgfpathlineto{\pgfqpoint{4.250000in}{0.988314in}}%
\pgfpathlineto{\pgfqpoint{4.250000in}{0.994574in}}%
\pgfpathlineto{\pgfqpoint{4.250000in}{1.000835in}}%
\pgfpathlineto{\pgfqpoint{4.250000in}{1.007095in}}%
\pgfpathlineto{\pgfqpoint{4.250000in}{1.013356in}}%
\pgfpathlineto{\pgfqpoint{4.250000in}{1.019616in}}%
\pgfpathlineto{\pgfqpoint{4.250000in}{1.025876in}}%
\pgfpathlineto{\pgfqpoint{4.250000in}{1.032137in}}%
\pgfpathlineto{\pgfqpoint{4.250000in}{1.038397in}}%
\pgfpathlineto{\pgfqpoint{4.250000in}{1.044658in}}%
\pgfpathlineto{\pgfqpoint{4.250000in}{1.050918in}}%
\pgfpathlineto{\pgfqpoint{4.250000in}{1.057179in}}%
\pgfpathlineto{\pgfqpoint{4.250000in}{1.063439in}}%
\pgfpathlineto{\pgfqpoint{4.250000in}{1.069699in}}%
\pgfpathlineto{\pgfqpoint{4.250000in}{1.075960in}}%
\pgfpathlineto{\pgfqpoint{4.250000in}{1.082220in}}%
\pgfpathlineto{\pgfqpoint{4.250000in}{1.088481in}}%
\pgfpathlineto{\pgfqpoint{4.250000in}{1.094741in}}%
\pgfpathlineto{\pgfqpoint{4.250000in}{1.101002in}}%
\pgfpathlineto{\pgfqpoint{4.250000in}{1.107262in}}%
\pgfpathlineto{\pgfqpoint{4.250000in}{1.113523in}}%
\pgfpathlineto{\pgfqpoint{4.250000in}{1.119783in}}%
\pgfpathlineto{\pgfqpoint{4.250000in}{1.126043in}}%
\pgfpathlineto{\pgfqpoint{4.250000in}{1.132304in}}%
\pgfpathlineto{\pgfqpoint{4.250000in}{1.138564in}}%
\pgfpathlineto{\pgfqpoint{4.250000in}{1.144825in}}%
\pgfpathlineto{\pgfqpoint{4.250000in}{1.151085in}}%
\pgfpathlineto{\pgfqpoint{4.250000in}{1.157346in}}%
\pgfpathlineto{\pgfqpoint{4.250000in}{1.163606in}}%
\pgfpathlineto{\pgfqpoint{4.250000in}{1.169866in}}%
\pgfpathlineto{\pgfqpoint{4.250000in}{1.176127in}}%
\pgfpathlineto{\pgfqpoint{4.250000in}{1.182387in}}%
\pgfpathlineto{\pgfqpoint{4.250000in}{1.188648in}}%
\pgfpathlineto{\pgfqpoint{4.250000in}{1.194908in}}%
\pgfpathlineto{\pgfqpoint{4.250000in}{1.201169in}}%
\pgfpathlineto{\pgfqpoint{4.250000in}{1.207429in}}%
\pgfpathlineto{\pgfqpoint{4.250000in}{1.213689in}}%
\pgfpathlineto{\pgfqpoint{4.250000in}{1.219950in}}%
\pgfpathlineto{\pgfqpoint{4.250000in}{1.226210in}}%
\pgfpathlineto{\pgfqpoint{4.250000in}{1.232471in}}%
\pgfpathlineto{\pgfqpoint{4.250000in}{1.238731in}}%
\pgfpathlineto{\pgfqpoint{4.250000in}{1.244992in}}%
\pgfpathlineto{\pgfqpoint{4.250000in}{1.251252in}}%
\pgfpathlineto{\pgfqpoint{4.250000in}{1.257513in}}%
\pgfpathlineto{\pgfqpoint{4.250000in}{1.263773in}}%
\pgfpathlineto{\pgfqpoint{4.250000in}{1.270033in}}%
\pgfpathlineto{\pgfqpoint{4.250000in}{1.276294in}}%
\pgfpathlineto{\pgfqpoint{4.250000in}{1.282554in}}%
\pgfpathlineto{\pgfqpoint{4.250000in}{1.288815in}}%
\pgfpathlineto{\pgfqpoint{4.250000in}{1.295075in}}%
\pgfpathlineto{\pgfqpoint{4.250000in}{1.301336in}}%
\pgfpathlineto{\pgfqpoint{4.250000in}{1.307596in}}%
\pgfpathlineto{\pgfqpoint{4.250000in}{1.313856in}}%
\pgfpathlineto{\pgfqpoint{4.250000in}{1.320117in}}%
\pgfpathlineto{\pgfqpoint{4.250000in}{1.326377in}}%
\pgfpathlineto{\pgfqpoint{4.250000in}{1.332638in}}%
\pgfpathlineto{\pgfqpoint{4.250000in}{1.338898in}}%
\pgfpathlineto{\pgfqpoint{4.250000in}{1.345159in}}%
\pgfpathlineto{\pgfqpoint{4.250000in}{1.351419in}}%
\pgfpathlineto{\pgfqpoint{4.250000in}{1.357679in}}%
\pgfpathlineto{\pgfqpoint{4.250000in}{1.363940in}}%
\pgfpathlineto{\pgfqpoint{4.250000in}{1.370200in}}%
\pgfpathlineto{\pgfqpoint{4.250000in}{1.376461in}}%
\pgfpathlineto{\pgfqpoint{4.250000in}{1.382721in}}%
\pgfpathlineto{\pgfqpoint{4.250000in}{1.388982in}}%
\pgfpathlineto{\pgfqpoint{4.250000in}{1.395242in}}%
\pgfpathlineto{\pgfqpoint{4.250000in}{1.401503in}}%
\pgfpathlineto{\pgfqpoint{4.250000in}{1.407763in}}%
\pgfpathlineto{\pgfqpoint{4.250000in}{1.414023in}}%
\pgfpathlineto{\pgfqpoint{4.250000in}{1.420284in}}%
\pgfpathlineto{\pgfqpoint{4.250000in}{1.426544in}}%
\pgfpathlineto{\pgfqpoint{4.250000in}{1.432805in}}%
\pgfpathlineto{\pgfqpoint{4.250000in}{1.439065in}}%
\pgfpathlineto{\pgfqpoint{4.250000in}{1.445326in}}%
\pgfpathlineto{\pgfqpoint{4.250000in}{1.451586in}}%
\pgfpathlineto{\pgfqpoint{4.250000in}{1.457846in}}%
\pgfpathlineto{\pgfqpoint{4.250000in}{1.464107in}}%
\pgfpathlineto{\pgfqpoint{4.250000in}{1.470367in}}%
\pgfpathlineto{\pgfqpoint{4.250000in}{1.476628in}}%
\pgfpathlineto{\pgfqpoint{4.250000in}{1.482888in}}%
\pgfpathlineto{\pgfqpoint{4.250000in}{1.489149in}}%
\pgfpathlineto{\pgfqpoint{4.250000in}{1.495409in}}%
\pgfpathlineto{\pgfqpoint{4.250000in}{1.501669in}}%
\pgfpathlineto{\pgfqpoint{4.250000in}{1.507930in}}%
\pgfpathlineto{\pgfqpoint{4.250000in}{1.514190in}}%
\pgfpathlineto{\pgfqpoint{4.250000in}{1.520451in}}%
\pgfpathlineto{\pgfqpoint{4.250000in}{1.526711in}}%
\pgfpathlineto{\pgfqpoint{4.250000in}{1.532972in}}%
\pgfpathlineto{\pgfqpoint{4.250000in}{1.539232in}}%
\pgfpathlineto{\pgfqpoint{4.250000in}{1.545492in}}%
\pgfpathlineto{\pgfqpoint{4.250000in}{1.551753in}}%
\pgfpathlineto{\pgfqpoint{4.250000in}{1.558013in}}%
\pgfpathlineto{\pgfqpoint{4.250000in}{1.564274in}}%
\pgfpathlineto{\pgfqpoint{4.250000in}{1.570534in}}%
\pgfpathlineto{\pgfqpoint{4.250000in}{1.576795in}}%
\pgfpathlineto{\pgfqpoint{4.250000in}{1.583055in}}%
\pgfpathlineto{\pgfqpoint{4.250000in}{1.589316in}}%
\pgfpathlineto{\pgfqpoint{4.250000in}{1.595576in}}%
\pgfpathlineto{\pgfqpoint{4.250000in}{1.601836in}}%
\pgfpathlineto{\pgfqpoint{4.250000in}{1.608097in}}%
\pgfpathlineto{\pgfqpoint{4.250000in}{1.614357in}}%
\pgfpathlineto{\pgfqpoint{4.250000in}{1.620618in}}%
\pgfpathlineto{\pgfqpoint{4.250000in}{1.626878in}}%
\pgfpathlineto{\pgfqpoint{4.250000in}{1.633139in}}%
\pgfpathlineto{\pgfqpoint{4.250000in}{1.639399in}}%
\pgfpathlineto{\pgfqpoint{4.250000in}{1.645659in}}%
\pgfpathlineto{\pgfqpoint{4.250000in}{1.651920in}}%
\pgfpathlineto{\pgfqpoint{4.250000in}{1.658180in}}%
\pgfpathlineto{\pgfqpoint{4.250000in}{1.664441in}}%
\pgfpathlineto{\pgfqpoint{4.250000in}{1.670701in}}%
\pgfpathlineto{\pgfqpoint{4.250000in}{1.676962in}}%
\pgfpathlineto{\pgfqpoint{4.250000in}{1.683222in}}%
\pgfpathlineto{\pgfqpoint{4.250000in}{1.689482in}}%
\pgfpathlineto{\pgfqpoint{4.250000in}{1.695743in}}%
\pgfpathlineto{\pgfqpoint{4.250000in}{1.702003in}}%
\pgfpathlineto{\pgfqpoint{4.250000in}{1.708264in}}%
\pgfpathlineto{\pgfqpoint{4.250000in}{1.714524in}}%
\pgfpathlineto{\pgfqpoint{4.250000in}{1.720785in}}%
\pgfpathlineto{\pgfqpoint{4.250000in}{1.727045in}}%
\pgfpathlineto{\pgfqpoint{4.250000in}{1.733306in}}%
\pgfpathlineto{\pgfqpoint{4.250000in}{1.739566in}}%
\pgfpathlineto{\pgfqpoint{4.250000in}{1.745826in}}%
\pgfpathlineto{\pgfqpoint{4.250000in}{1.752087in}}%
\pgfpathlineto{\pgfqpoint{4.250000in}{1.758347in}}%
\pgfpathlineto{\pgfqpoint{4.250000in}{1.764608in}}%
\pgfpathlineto{\pgfqpoint{4.250000in}{1.770868in}}%
\pgfpathlineto{\pgfqpoint{4.250000in}{1.777129in}}%
\pgfpathlineto{\pgfqpoint{4.250000in}{1.783389in}}%
\pgfpathlineto{\pgfqpoint{4.250000in}{1.789649in}}%
\pgfpathlineto{\pgfqpoint{4.250000in}{1.795910in}}%
\pgfpathlineto{\pgfqpoint{4.250000in}{1.802170in}}%
\pgfpathlineto{\pgfqpoint{4.250000in}{1.808431in}}%
\pgfpathlineto{\pgfqpoint{4.250000in}{1.814691in}}%
\pgfpathlineto{\pgfqpoint{4.250000in}{1.820952in}}%
\pgfpathlineto{\pgfqpoint{4.250000in}{1.827212in}}%
\pgfpathlineto{\pgfqpoint{4.250000in}{1.833472in}}%
\pgfpathlineto{\pgfqpoint{4.250000in}{1.839733in}}%
\pgfpathlineto{\pgfqpoint{4.250000in}{1.845993in}}%
\pgfpathlineto{\pgfqpoint{4.250000in}{1.852254in}}%
\pgfpathlineto{\pgfqpoint{4.250000in}{1.858514in}}%
\pgfpathlineto{\pgfqpoint{4.250000in}{1.864775in}}%
\pgfpathlineto{\pgfqpoint{4.250000in}{1.871035in}}%
\pgfpathlineto{\pgfqpoint{4.250000in}{1.877295in}}%
\pgfpathlineto{\pgfqpoint{4.250000in}{1.883556in}}%
\pgfpathlineto{\pgfqpoint{4.250000in}{1.889816in}}%
\pgfpathlineto{\pgfqpoint{4.250000in}{1.896077in}}%
\pgfpathlineto{\pgfqpoint{4.250000in}{1.902337in}}%
\pgfpathlineto{\pgfqpoint{4.250000in}{1.908598in}}%
\pgfpathlineto{\pgfqpoint{4.250000in}{1.914858in}}%
\pgfpathlineto{\pgfqpoint{4.250000in}{1.921119in}}%
\pgfpathlineto{\pgfqpoint{4.250000in}{1.927379in}}%
\pgfpathlineto{\pgfqpoint{4.250000in}{1.933639in}}%
\pgfpathlineto{\pgfqpoint{4.250000in}{1.939900in}}%
\pgfpathlineto{\pgfqpoint{4.250000in}{1.946160in}}%
\pgfpathlineto{\pgfqpoint{4.250000in}{1.952421in}}%
\pgfpathlineto{\pgfqpoint{4.250000in}{1.958681in}}%
\pgfpathlineto{\pgfqpoint{4.250000in}{1.964942in}}%
\pgfpathlineto{\pgfqpoint{4.250000in}{1.971202in}}%
\pgfpathlineto{\pgfqpoint{4.250000in}{1.977462in}}%
\pgfpathlineto{\pgfqpoint{4.250000in}{1.983723in}}%
\pgfpathlineto{\pgfqpoint{4.250000in}{1.989983in}}%
\pgfpathlineto{\pgfqpoint{4.250000in}{1.996244in}}%
\pgfpathlineto{\pgfqpoint{4.250000in}{2.002504in}}%
\pgfpathlineto{\pgfqpoint{4.250000in}{2.008765in}}%
\pgfpathlineto{\pgfqpoint{4.250000in}{2.015025in}}%
\pgfpathlineto{\pgfqpoint{4.250000in}{2.021285in}}%
\pgfpathlineto{\pgfqpoint{4.250000in}{2.027546in}}%
\pgfpathlineto{\pgfqpoint{4.250000in}{2.033806in}}%
\pgfpathlineto{\pgfqpoint{4.250000in}{2.040067in}}%
\pgfpathlineto{\pgfqpoint{4.250000in}{2.046327in}}%
\pgfpathlineto{\pgfqpoint{4.250000in}{2.052588in}}%
\pgfpathlineto{\pgfqpoint{4.250000in}{2.058848in}}%
\pgfpathlineto{\pgfqpoint{4.250000in}{2.065109in}}%
\pgfpathlineto{\pgfqpoint{4.250000in}{2.071369in}}%
\pgfpathlineto{\pgfqpoint{4.250000in}{2.077629in}}%
\pgfpathlineto{\pgfqpoint{4.250000in}{2.083890in}}%
\pgfpathlineto{\pgfqpoint{4.250000in}{2.090150in}}%
\pgfpathlineto{\pgfqpoint{4.250000in}{2.096411in}}%
\pgfpathlineto{\pgfqpoint{4.250000in}{2.102671in}}%
\pgfpathlineto{\pgfqpoint{4.250000in}{2.108932in}}%
\pgfpathlineto{\pgfqpoint{4.250000in}{2.115192in}}%
\pgfpathlineto{\pgfqpoint{4.250000in}{2.121452in}}%
\pgfpathlineto{\pgfqpoint{4.250000in}{2.127713in}}%
\pgfpathlineto{\pgfqpoint{4.250000in}{2.133973in}}%
\pgfpathlineto{\pgfqpoint{4.250000in}{2.140234in}}%
\pgfpathlineto{\pgfqpoint{4.250000in}{2.146494in}}%
\pgfpathlineto{\pgfqpoint{4.250000in}{2.152755in}}%
\pgfpathlineto{\pgfqpoint{4.250000in}{2.159015in}}%
\pgfpathlineto{\pgfqpoint{4.250000in}{2.165275in}}%
\pgfpathlineto{\pgfqpoint{4.250000in}{2.171536in}}%
\pgfpathlineto{\pgfqpoint{4.250000in}{2.177796in}}%
\pgfpathlineto{\pgfqpoint{4.250000in}{2.184057in}}%
\pgfpathlineto{\pgfqpoint{4.250000in}{2.190317in}}%
\pgfpathlineto{\pgfqpoint{4.250000in}{2.196578in}}%
\pgfpathlineto{\pgfqpoint{4.250000in}{2.202838in}}%
\pgfpathlineto{\pgfqpoint{4.250000in}{2.209098in}}%
\pgfpathlineto{\pgfqpoint{4.250000in}{2.215359in}}%
\pgfpathlineto{\pgfqpoint{4.250000in}{2.221619in}}%
\pgfpathlineto{\pgfqpoint{4.250000in}{2.227880in}}%
\pgfpathlineto{\pgfqpoint{4.250000in}{2.234140in}}%
\pgfpathlineto{\pgfqpoint{4.250000in}{2.240401in}}%
\pgfpathlineto{\pgfqpoint{4.250000in}{2.246661in}}%
\pgfpathlineto{\pgfqpoint{4.250000in}{2.252922in}}%
\pgfpathlineto{\pgfqpoint{4.250000in}{2.259182in}}%
\pgfpathlineto{\pgfqpoint{4.250000in}{2.265442in}}%
\pgfpathlineto{\pgfqpoint{4.250000in}{2.271703in}}%
\pgfpathlineto{\pgfqpoint{4.250000in}{2.277963in}}%
\pgfpathlineto{\pgfqpoint{4.250000in}{2.284224in}}%
\pgfpathlineto{\pgfqpoint{4.250000in}{2.290484in}}%
\pgfpathlineto{\pgfqpoint{4.250000in}{2.296745in}}%
\pgfpathlineto{\pgfqpoint{4.250000in}{2.303005in}}%
\pgfpathlineto{\pgfqpoint{4.250000in}{2.309265in}}%
\pgfpathlineto{\pgfqpoint{4.250000in}{2.315526in}}%
\pgfpathlineto{\pgfqpoint{4.250000in}{2.321786in}}%
\pgfpathlineto{\pgfqpoint{4.250000in}{2.328047in}}%
\pgfpathlineto{\pgfqpoint{4.250000in}{2.334307in}}%
\pgfpathlineto{\pgfqpoint{4.250000in}{2.340568in}}%
\pgfpathlineto{\pgfqpoint{4.250000in}{2.346828in}}%
\pgfpathlineto{\pgfqpoint{4.250000in}{2.353088in}}%
\pgfpathlineto{\pgfqpoint{4.250000in}{2.359349in}}%
\pgfpathlineto{\pgfqpoint{4.250000in}{2.365609in}}%
\pgfpathlineto{\pgfqpoint{4.250000in}{2.371870in}}%
\pgfpathlineto{\pgfqpoint{4.250000in}{2.378130in}}%
\pgfpathlineto{\pgfqpoint{4.250000in}{2.384391in}}%
\pgfpathlineto{\pgfqpoint{4.250000in}{2.390651in}}%
\pgfpathlineto{\pgfqpoint{4.250000in}{2.396912in}}%
\pgfpathlineto{\pgfqpoint{4.250000in}{2.403172in}}%
\pgfpathlineto{\pgfqpoint{4.250000in}{2.409432in}}%
\pgfpathlineto{\pgfqpoint{4.250000in}{2.415693in}}%
\pgfpathlineto{\pgfqpoint{4.250000in}{2.421953in}}%
\pgfpathlineto{\pgfqpoint{4.250000in}{2.428214in}}%
\pgfpathlineto{\pgfqpoint{4.250000in}{2.434474in}}%
\pgfpathlineto{\pgfqpoint{4.250000in}{2.440735in}}%
\pgfpathlineto{\pgfqpoint{4.250000in}{2.446995in}}%
\pgfpathlineto{\pgfqpoint{4.250000in}{2.453255in}}%
\pgfpathlineto{\pgfqpoint{4.250000in}{2.459516in}}%
\pgfpathlineto{\pgfqpoint{4.250000in}{2.465776in}}%
\pgfpathlineto{\pgfqpoint{4.250000in}{2.472037in}}%
\pgfpathlineto{\pgfqpoint{4.250000in}{2.478297in}}%
\pgfpathlineto{\pgfqpoint{4.250000in}{2.484558in}}%
\pgfpathlineto{\pgfqpoint{4.250000in}{2.490818in}}%
\pgfpathlineto{\pgfqpoint{4.250000in}{2.497078in}}%
\pgfpathlineto{\pgfqpoint{4.250000in}{2.503339in}}%
\pgfpathlineto{\pgfqpoint{4.250000in}{2.509599in}}%
\pgfpathlineto{\pgfqpoint{4.250000in}{2.515860in}}%
\pgfpathlineto{\pgfqpoint{4.250000in}{2.522120in}}%
\pgfpathlineto{\pgfqpoint{4.250000in}{2.528381in}}%
\pgfpathlineto{\pgfqpoint{4.250000in}{2.534641in}}%
\pgfpathlineto{\pgfqpoint{4.250000in}{2.540902in}}%
\pgfpathlineto{\pgfqpoint{4.250000in}{2.547162in}}%
\pgfpathlineto{\pgfqpoint{4.250000in}{2.553422in}}%
\pgfpathlineto{\pgfqpoint{4.250000in}{2.559683in}}%
\pgfpathlineto{\pgfqpoint{4.250000in}{2.565943in}}%
\pgfpathlineto{\pgfqpoint{4.250000in}{2.572204in}}%
\pgfpathlineto{\pgfqpoint{4.250000in}{2.578464in}}%
\pgfpathlineto{\pgfqpoint{4.250000in}{2.584725in}}%
\pgfpathlineto{\pgfqpoint{4.250000in}{2.590985in}}%
\pgfpathlineto{\pgfqpoint{4.250000in}{2.597245in}}%
\pgfpathlineto{\pgfqpoint{4.250000in}{2.603506in}}%
\pgfpathlineto{\pgfqpoint{4.250000in}{2.609766in}}%
\pgfpathlineto{\pgfqpoint{4.250000in}{2.616027in}}%
\pgfpathlineto{\pgfqpoint{4.250000in}{2.622287in}}%
\pgfpathlineto{\pgfqpoint{4.250000in}{2.628548in}}%
\pgfpathlineto{\pgfqpoint{4.250000in}{2.634808in}}%
\pgfpathlineto{\pgfqpoint{4.250000in}{2.641068in}}%
\pgfpathlineto{\pgfqpoint{4.250000in}{2.647329in}}%
\pgfpathlineto{\pgfqpoint{4.250000in}{2.653589in}}%
\pgfpathlineto{\pgfqpoint{4.250000in}{2.659850in}}%
\pgfpathlineto{\pgfqpoint{4.250000in}{2.666110in}}%
\pgfpathlineto{\pgfqpoint{4.250000in}{2.672371in}}%
\pgfpathlineto{\pgfqpoint{4.250000in}{2.678631in}}%
\pgfpathlineto{\pgfqpoint{4.250000in}{2.684891in}}%
\pgfpathlineto{\pgfqpoint{4.250000in}{2.691152in}}%
\pgfpathlineto{\pgfqpoint{4.250000in}{2.697412in}}%
\pgfpathlineto{\pgfqpoint{4.250000in}{2.703673in}}%
\pgfpathlineto{\pgfqpoint{4.250000in}{2.709933in}}%
\pgfpathlineto{\pgfqpoint{4.250000in}{2.716194in}}%
\pgfpathlineto{\pgfqpoint{4.250000in}{2.722454in}}%
\pgfpathlineto{\pgfqpoint{4.250000in}{2.728715in}}%
\pgfpathlineto{\pgfqpoint{4.250000in}{2.734975in}}%
\pgfpathlineto{\pgfqpoint{4.250000in}{2.741235in}}%
\pgfpathlineto{\pgfqpoint{4.250000in}{2.747496in}}%
\pgfpathlineto{\pgfqpoint{4.250000in}{2.753756in}}%
\pgfpathlineto{\pgfqpoint{4.250000in}{2.760017in}}%
\pgfpathlineto{\pgfqpoint{4.250000in}{2.766277in}}%
\pgfpathlineto{\pgfqpoint{4.250000in}{2.772538in}}%
\pgfpathlineto{\pgfqpoint{4.250000in}{2.778798in}}%
\pgfpathlineto{\pgfqpoint{4.250000in}{2.785058in}}%
\pgfpathlineto{\pgfqpoint{4.250000in}{2.791319in}}%
\pgfpathlineto{\pgfqpoint{4.250000in}{2.797579in}}%
\pgfpathlineto{\pgfqpoint{4.250000in}{2.803840in}}%
\pgfpathlineto{\pgfqpoint{4.250000in}{2.810100in}}%
\pgfpathlineto{\pgfqpoint{4.250000in}{2.816361in}}%
\pgfpathlineto{\pgfqpoint{4.250000in}{2.822621in}}%
\pgfpathlineto{\pgfqpoint{4.250000in}{2.828881in}}%
\pgfpathlineto{\pgfqpoint{4.250000in}{2.835142in}}%
\pgfpathlineto{\pgfqpoint{4.250000in}{2.841402in}}%
\pgfpathlineto{\pgfqpoint{4.250000in}{2.847663in}}%
\pgfpathlineto{\pgfqpoint{4.250000in}{2.853923in}}%
\pgfpathlineto{\pgfqpoint{4.250000in}{2.860184in}}%
\pgfpathlineto{\pgfqpoint{4.250000in}{2.866444in}}%
\pgfpathlineto{\pgfqpoint{4.250000in}{2.872705in}}%
\pgfpathlineto{\pgfqpoint{4.250000in}{2.878965in}}%
\pgfpathlineto{\pgfqpoint{4.250000in}{2.885225in}}%
\pgfpathlineto{\pgfqpoint{4.250000in}{2.891486in}}%
\pgfpathlineto{\pgfqpoint{4.250000in}{2.897746in}}%
\pgfpathlineto{\pgfqpoint{4.250000in}{2.904007in}}%
\pgfpathlineto{\pgfqpoint{4.250000in}{2.910267in}}%
\pgfpathlineto{\pgfqpoint{4.250000in}{2.916528in}}%
\pgfpathlineto{\pgfqpoint{4.250000in}{2.922788in}}%
\pgfpathlineto{\pgfqpoint{4.250000in}{2.929048in}}%
\pgfpathlineto{\pgfqpoint{4.250000in}{2.935309in}}%
\pgfpathlineto{\pgfqpoint{4.250000in}{2.941569in}}%
\pgfpathlineto{\pgfqpoint{4.250000in}{2.947830in}}%
\pgfpathlineto{\pgfqpoint{4.250000in}{2.954090in}}%
\pgfpathlineto{\pgfqpoint{4.250000in}{2.960351in}}%
\pgfpathlineto{\pgfqpoint{4.250000in}{2.966611in}}%
\pgfpathlineto{\pgfqpoint{4.250000in}{2.972871in}}%
\pgfpathlineto{\pgfqpoint{4.250000in}{2.979132in}}%
\pgfpathlineto{\pgfqpoint{4.250000in}{2.985392in}}%
\pgfpathlineto{\pgfqpoint{4.250000in}{2.991653in}}%
\pgfpathlineto{\pgfqpoint{4.250000in}{2.997913in}}%
\pgfpathlineto{\pgfqpoint{4.250000in}{3.004174in}}%
\pgfpathlineto{\pgfqpoint{4.250000in}{3.010434in}}%
\pgfpathlineto{\pgfqpoint{4.250000in}{3.016694in}}%
\pgfpathlineto{\pgfqpoint{4.250000in}{3.022955in}}%
\pgfpathlineto{\pgfqpoint{4.250000in}{3.029215in}}%
\pgfpathlineto{\pgfqpoint{4.250000in}{3.035476in}}%
\pgfpathlineto{\pgfqpoint{4.250000in}{3.041736in}}%
\pgfpathlineto{\pgfqpoint{4.250000in}{3.047997in}}%
\pgfpathlineto{\pgfqpoint{4.250000in}{3.054257in}}%
\pgfpathlineto{\pgfqpoint{4.250000in}{3.060518in}}%
\pgfpathlineto{\pgfqpoint{4.250000in}{3.066778in}}%
\pgfpathlineto{\pgfqpoint{4.250000in}{3.073038in}}%
\pgfpathlineto{\pgfqpoint{4.250000in}{3.079299in}}%
\pgfpathlineto{\pgfqpoint{4.250000in}{3.085559in}}%
\pgfpathlineto{\pgfqpoint{4.250000in}{3.091820in}}%
\pgfpathlineto{\pgfqpoint{4.250000in}{3.098080in}}%
\pgfpathlineto{\pgfqpoint{4.250000in}{3.104341in}}%
\pgfpathlineto{\pgfqpoint{4.250000in}{3.110601in}}%
\pgfpathlineto{\pgfqpoint{4.250000in}{3.116861in}}%
\pgfpathlineto{\pgfqpoint{4.250000in}{3.123122in}}%
\pgfpathlineto{\pgfqpoint{4.250000in}{3.129382in}}%
\pgfpathlineto{\pgfqpoint{4.250000in}{3.135643in}}%
\pgfpathlineto{\pgfqpoint{4.250000in}{3.141903in}}%
\pgfpathlineto{\pgfqpoint{4.250000in}{3.148164in}}%
\pgfpathlineto{\pgfqpoint{4.250000in}{3.154424in}}%
\pgfpathlineto{\pgfqpoint{4.250000in}{3.160684in}}%
\pgfpathlineto{\pgfqpoint{4.250000in}{3.166945in}}%
\pgfpathlineto{\pgfqpoint{4.250000in}{3.173205in}}%
\pgfpathlineto{\pgfqpoint{4.250000in}{3.179466in}}%
\pgfpathlineto{\pgfqpoint{4.250000in}{3.185726in}}%
\pgfpathlineto{\pgfqpoint{4.250000in}{3.191987in}}%
\pgfpathlineto{\pgfqpoint{4.250000in}{3.198247in}}%
\pgfpathlineto{\pgfqpoint{4.250000in}{3.204508in}}%
\pgfpathlineto{\pgfqpoint{4.250000in}{3.210768in}}%
\pgfpathlineto{\pgfqpoint{4.250000in}{3.217028in}}%
\pgfpathlineto{\pgfqpoint{4.250000in}{3.223289in}}%
\pgfpathlineto{\pgfqpoint{4.250000in}{3.229549in}}%
\pgfpathlineto{\pgfqpoint{4.250000in}{3.235810in}}%
\pgfpathlineto{\pgfqpoint{4.250000in}{3.242070in}}%
\pgfpathlineto{\pgfqpoint{4.250000in}{3.248331in}}%
\pgfpathlineto{\pgfqpoint{4.250000in}{3.254591in}}%
\pgfpathlineto{\pgfqpoint{4.250000in}{3.260851in}}%
\pgfpathlineto{\pgfqpoint{4.250000in}{3.267112in}}%
\pgfpathlineto{\pgfqpoint{4.250000in}{3.273372in}}%
\pgfpathlineto{\pgfqpoint{4.250000in}{3.279633in}}%
\pgfpathlineto{\pgfqpoint{4.250000in}{3.285893in}}%
\pgfpathlineto{\pgfqpoint{4.250000in}{3.292154in}}%
\pgfpathlineto{\pgfqpoint{4.250000in}{3.298414in}}%
\pgfpathlineto{\pgfqpoint{4.250000in}{3.304674in}}%
\pgfpathlineto{\pgfqpoint{4.250000in}{3.310935in}}%
\pgfpathlineto{\pgfqpoint{4.250000in}{3.317195in}}%
\pgfpathlineto{\pgfqpoint{4.250000in}{3.323456in}}%
\pgfpathlineto{\pgfqpoint{4.250000in}{3.329716in}}%
\pgfpathlineto{\pgfqpoint{4.250000in}{3.335977in}}%
\pgfpathlineto{\pgfqpoint{4.250000in}{3.342237in}}%
\pgfpathlineto{\pgfqpoint{4.250000in}{3.348497in}}%
\pgfpathlineto{\pgfqpoint{4.250000in}{3.354758in}}%
\pgfpathlineto{\pgfqpoint{4.250000in}{3.361018in}}%
\pgfpathlineto{\pgfqpoint{4.250000in}{3.367279in}}%
\pgfpathlineto{\pgfqpoint{4.250000in}{3.373539in}}%
\pgfpathlineto{\pgfqpoint{4.250000in}{3.379800in}}%
\pgfpathlineto{\pgfqpoint{4.250000in}{3.386060in}}%
\pgfpathlineto{\pgfqpoint{4.250000in}{3.392321in}}%
\pgfpathlineto{\pgfqpoint{4.250000in}{3.398581in}}%
\pgfpathlineto{\pgfqpoint{4.250000in}{3.404841in}}%
\pgfpathlineto{\pgfqpoint{4.250000in}{3.411102in}}%
\pgfpathlineto{\pgfqpoint{4.250000in}{3.417362in}}%
\pgfpathlineto{\pgfqpoint{4.250000in}{3.423623in}}%
\pgfpathlineto{\pgfqpoint{4.250000in}{3.429883in}}%
\pgfpathlineto{\pgfqpoint{4.250000in}{3.436144in}}%
\pgfpathlineto{\pgfqpoint{4.250000in}{3.442404in}}%
\pgfpathlineto{\pgfqpoint{4.250000in}{3.448664in}}%
\pgfpathlineto{\pgfqpoint{4.250000in}{3.454925in}}%
\pgfpathlineto{\pgfqpoint{4.250000in}{3.461185in}}%
\pgfpathlineto{\pgfqpoint{4.250000in}{3.467446in}}%
\pgfpathlineto{\pgfqpoint{4.250000in}{3.473706in}}%
\pgfpathlineto{\pgfqpoint{4.250000in}{3.479967in}}%
\pgfpathlineto{\pgfqpoint{4.250000in}{3.486227in}}%
\pgfpathlineto{\pgfqpoint{4.250000in}{3.492487in}}%
\pgfpathlineto{\pgfqpoint{4.250000in}{3.498748in}}%
\pgfpathlineto{\pgfqpoint{4.250000in}{3.505008in}}%
\pgfpathlineto{\pgfqpoint{4.250000in}{3.511269in}}%
\pgfpathlineto{\pgfqpoint{4.250000in}{3.517529in}}%
\pgfpathlineto{\pgfqpoint{4.250000in}{3.523790in}}%
\pgfpathlineto{\pgfqpoint{4.250000in}{3.530050in}}%
\pgfpathlineto{\pgfqpoint{4.250000in}{3.536311in}}%
\pgfpathlineto{\pgfqpoint{4.250000in}{3.542571in}}%
\pgfpathlineto{\pgfqpoint{4.250000in}{3.548831in}}%
\pgfpathlineto{\pgfqpoint{4.250000in}{3.555092in}}%
\pgfpathlineto{\pgfqpoint{4.250000in}{3.561352in}}%
\pgfpathlineto{\pgfqpoint{4.250000in}{3.567613in}}%
\pgfpathlineto{\pgfqpoint{4.250000in}{3.573873in}}%
\pgfpathlineto{\pgfqpoint{4.250000in}{3.580134in}}%
\pgfpathlineto{\pgfqpoint{4.250000in}{3.586394in}}%
\pgfpathlineto{\pgfqpoint{4.250000in}{3.592654in}}%
\pgfpathlineto{\pgfqpoint{4.250000in}{3.598915in}}%
\pgfpathlineto{\pgfqpoint{4.250000in}{3.605175in}}%
\pgfpathlineto{\pgfqpoint{4.250000in}{3.611436in}}%
\pgfpathlineto{\pgfqpoint{4.250000in}{3.617696in}}%
\pgfpathlineto{\pgfqpoint{4.250000in}{3.623957in}}%
\pgfpathlineto{\pgfqpoint{4.250000in}{3.630217in}}%
\pgfpathlineto{\pgfqpoint{4.250000in}{3.636477in}}%
\pgfpathlineto{\pgfqpoint{4.250000in}{3.642738in}}%
\pgfpathlineto{\pgfqpoint{4.250000in}{3.648998in}}%
\pgfpathlineto{\pgfqpoint{4.250000in}{3.655259in}}%
\pgfpathlineto{\pgfqpoint{4.250000in}{3.661519in}}%
\pgfpathlineto{\pgfqpoint{4.250000in}{3.667780in}}%
\pgfpathlineto{\pgfqpoint{4.250000in}{3.674040in}}%
\pgfpathlineto{\pgfqpoint{4.250000in}{3.680301in}}%
\pgfpathlineto{\pgfqpoint{4.250000in}{3.686561in}}%
\pgfpathlineto{\pgfqpoint{4.250000in}{3.692821in}}%
\pgfpathlineto{\pgfqpoint{4.250000in}{3.699082in}}%
\pgfpathlineto{\pgfqpoint{4.250000in}{3.705342in}}%
\pgfpathlineto{\pgfqpoint{4.250000in}{3.711603in}}%
\pgfpathlineto{\pgfqpoint{4.250000in}{3.717863in}}%
\pgfpathlineto{\pgfqpoint{4.250000in}{3.724124in}}%
\pgfpathlineto{\pgfqpoint{4.250000in}{3.730384in}}%
\pgfpathlineto{\pgfqpoint{4.250000in}{3.736644in}}%
\pgfpathlineto{\pgfqpoint{4.250000in}{3.742905in}}%
\pgfpathlineto{\pgfqpoint{4.250000in}{3.749165in}}%
\pgfpathlineto{\pgfqpoint{4.250000in}{3.755426in}}%
\pgfpathlineto{\pgfqpoint{4.250000in}{3.761686in}}%
\pgfpathlineto{\pgfqpoint{4.250000in}{3.767947in}}%
\pgfpathlineto{\pgfqpoint{4.250000in}{3.774207in}}%
\pgfpathlineto{\pgfqpoint{4.250000in}{3.780467in}}%
\pgfpathlineto{\pgfqpoint{4.250000in}{3.786728in}}%
\pgfpathlineto{\pgfqpoint{4.250000in}{3.792988in}}%
\pgfpathlineto{\pgfqpoint{4.250000in}{3.799249in}}%
\pgfpathlineto{\pgfqpoint{4.250000in}{3.805509in}}%
\pgfpathlineto{\pgfqpoint{4.250000in}{3.811770in}}%
\pgfpathlineto{\pgfqpoint{4.250000in}{3.818030in}}%
\pgfpathlineto{\pgfqpoint{4.250000in}{3.824290in}}%
\pgfpathlineto{\pgfqpoint{4.250000in}{3.830551in}}%
\pgfpathlineto{\pgfqpoint{4.250000in}{3.836811in}}%
\pgfpathlineto{\pgfqpoint{4.250000in}{3.843072in}}%
\pgfpathlineto{\pgfqpoint{4.250000in}{3.849332in}}%
\pgfpathlineto{\pgfqpoint{4.250000in}{3.855593in}}%
\pgfpathlineto{\pgfqpoint{4.250000in}{3.861853in}}%
\pgfpathlineto{\pgfqpoint{4.250000in}{3.868114in}}%
\pgfpathlineto{\pgfqpoint{4.250000in}{3.874374in}}%
\pgfpathlineto{\pgfqpoint{4.250000in}{3.880634in}}%
\pgfpathlineto{\pgfqpoint{4.250000in}{3.886895in}}%
\pgfpathlineto{\pgfqpoint{4.250000in}{3.893155in}}%
\pgfpathlineto{\pgfqpoint{4.250000in}{3.899416in}}%
\pgfpathlineto{\pgfqpoint{4.250000in}{3.905676in}}%
\pgfpathlineto{\pgfqpoint{4.250000in}{3.911937in}}%
\pgfpathlineto{\pgfqpoint{4.250000in}{3.918197in}}%
\pgfpathlineto{\pgfqpoint{4.250000in}{3.924457in}}%
\pgfpathlineto{\pgfqpoint{4.250000in}{3.930718in}}%
\pgfpathlineto{\pgfqpoint{4.250000in}{3.936978in}}%
\pgfpathlineto{\pgfqpoint{4.250000in}{3.943239in}}%
\pgfpathlineto{\pgfqpoint{4.250000in}{3.949499in}}%
\pgfpathlineto{\pgfqpoint{4.250000in}{3.955760in}}%
\pgfpathlineto{\pgfqpoint{4.250000in}{3.962020in}}%
\pgfpathlineto{\pgfqpoint{4.250000in}{3.968280in}}%
\pgfpathlineto{\pgfqpoint{4.250000in}{3.974541in}}%
\pgfpathlineto{\pgfqpoint{4.250000in}{3.980801in}}%
\pgfpathlineto{\pgfqpoint{4.250000in}{3.987062in}}%
\pgfpathlineto{\pgfqpoint{4.250000in}{3.993322in}}%
\pgfpathlineto{\pgfqpoint{4.250000in}{3.999583in}}%
\pgfpathlineto{\pgfqpoint{4.250000in}{4.005843in}}%
\pgfpathlineto{\pgfqpoint{4.250000in}{4.012104in}}%
\pgfpathlineto{\pgfqpoint{4.250000in}{4.018364in}}%
\pgfpathlineto{\pgfqpoint{4.250000in}{4.024624in}}%
\pgfpathlineto{\pgfqpoint{4.250000in}{4.030885in}}%
\pgfpathlineto{\pgfqpoint{4.250000in}{4.037145in}}%
\pgfpathlineto{\pgfqpoint{4.250000in}{4.043406in}}%
\pgfpathlineto{\pgfqpoint{4.250000in}{4.049666in}}%
\pgfpathlineto{\pgfqpoint{4.250000in}{4.055927in}}%
\pgfpathlineto{\pgfqpoint{4.250000in}{4.062187in}}%
\pgfpathlineto{\pgfqpoint{4.250000in}{4.068447in}}%
\pgfpathlineto{\pgfqpoint{4.250000in}{4.074708in}}%
\pgfpathlineto{\pgfqpoint{4.250000in}{4.080968in}}%
\pgfpathlineto{\pgfqpoint{4.250000in}{4.087229in}}%
\pgfpathlineto{\pgfqpoint{4.250000in}{4.093489in}}%
\pgfpathlineto{\pgfqpoint{4.250000in}{4.099750in}}%
\pgfpathlineto{\pgfqpoint{4.250000in}{4.106010in}}%
\pgfpathlineto{\pgfqpoint{4.250000in}{4.112270in}}%
\pgfpathlineto{\pgfqpoint{4.250000in}{4.118531in}}%
\pgfpathlineto{\pgfqpoint{4.250000in}{4.124791in}}%
\pgfpathlineto{\pgfqpoint{4.250000in}{4.131052in}}%
\pgfpathlineto{\pgfqpoint{4.250000in}{4.137312in}}%
\pgfpathlineto{\pgfqpoint{4.250000in}{4.143573in}}%
\pgfpathlineto{\pgfqpoint{4.250000in}{4.149833in}}%
\pgfpathlineto{\pgfqpoint{4.250000in}{4.156093in}}%
\pgfpathlineto{\pgfqpoint{4.250000in}{4.162354in}}%
\pgfpathlineto{\pgfqpoint{4.250000in}{4.168614in}}%
\pgfpathlineto{\pgfqpoint{4.250000in}{4.174875in}}%
\pgfpathlineto{\pgfqpoint{4.250000in}{4.181135in}}%
\pgfpathlineto{\pgfqpoint{4.250000in}{4.187396in}}%
\pgfpathlineto{\pgfqpoint{4.250000in}{4.193656in}}%
\pgfpathlineto{\pgfqpoint{4.250000in}{4.199917in}}%
\pgfpathlineto{\pgfqpoint{4.250000in}{4.206177in}}%
\pgfpathlineto{\pgfqpoint{4.250000in}{4.212437in}}%
\pgfpathlineto{\pgfqpoint{4.250000in}{4.218698in}}%
\pgfpathlineto{\pgfqpoint{4.250000in}{4.224958in}}%
\pgfpathlineto{\pgfqpoint{4.250000in}{4.231219in}}%
\pgfpathlineto{\pgfqpoint{4.250000in}{4.237479in}}%
\pgfpathlineto{\pgfqpoint{4.250000in}{4.243740in}}%
\pgfpathlineto{\pgfqpoint{4.250000in}{4.250000in}}%
\pgfpathlineto{\pgfqpoint{4.243740in}{4.250000in}}%
\pgfpathlineto{\pgfqpoint{4.237479in}{4.250000in}}%
\pgfpathlineto{\pgfqpoint{4.231219in}{4.250000in}}%
\pgfpathlineto{\pgfqpoint{4.224958in}{4.250000in}}%
\pgfpathlineto{\pgfqpoint{4.218698in}{4.250000in}}%
\pgfpathlineto{\pgfqpoint{4.212437in}{4.250000in}}%
\pgfpathlineto{\pgfqpoint{4.206177in}{4.250000in}}%
\pgfpathlineto{\pgfqpoint{4.199917in}{4.250000in}}%
\pgfpathlineto{\pgfqpoint{4.193656in}{4.250000in}}%
\pgfpathlineto{\pgfqpoint{4.187396in}{4.250000in}}%
\pgfpathlineto{\pgfqpoint{4.181135in}{4.250000in}}%
\pgfpathlineto{\pgfqpoint{4.174875in}{4.250000in}}%
\pgfpathlineto{\pgfqpoint{4.168614in}{4.250000in}}%
\pgfpathlineto{\pgfqpoint{4.162354in}{4.250000in}}%
\pgfpathlineto{\pgfqpoint{4.156093in}{4.250000in}}%
\pgfpathlineto{\pgfqpoint{4.149833in}{4.250000in}}%
\pgfpathlineto{\pgfqpoint{4.143573in}{4.250000in}}%
\pgfpathlineto{\pgfqpoint{4.137312in}{4.250000in}}%
\pgfpathlineto{\pgfqpoint{4.131052in}{4.250000in}}%
\pgfpathlineto{\pgfqpoint{4.124791in}{4.250000in}}%
\pgfpathlineto{\pgfqpoint{4.118531in}{4.250000in}}%
\pgfpathlineto{\pgfqpoint{4.112270in}{4.250000in}}%
\pgfpathlineto{\pgfqpoint{4.106010in}{4.250000in}}%
\pgfpathlineto{\pgfqpoint{4.099750in}{4.250000in}}%
\pgfpathlineto{\pgfqpoint{4.093489in}{4.250000in}}%
\pgfpathlineto{\pgfqpoint{4.087229in}{4.250000in}}%
\pgfpathlineto{\pgfqpoint{4.080968in}{4.250000in}}%
\pgfpathlineto{\pgfqpoint{4.074708in}{4.250000in}}%
\pgfpathlineto{\pgfqpoint{4.068447in}{4.250000in}}%
\pgfpathlineto{\pgfqpoint{4.062187in}{4.250000in}}%
\pgfpathlineto{\pgfqpoint{4.055927in}{4.250000in}}%
\pgfpathlineto{\pgfqpoint{4.049666in}{4.250000in}}%
\pgfpathlineto{\pgfqpoint{4.043406in}{4.250000in}}%
\pgfpathlineto{\pgfqpoint{4.037145in}{4.250000in}}%
\pgfpathlineto{\pgfqpoint{4.030885in}{4.250000in}}%
\pgfpathlineto{\pgfqpoint{4.024624in}{4.250000in}}%
\pgfpathlineto{\pgfqpoint{4.018364in}{4.250000in}}%
\pgfpathlineto{\pgfqpoint{4.012104in}{4.250000in}}%
\pgfpathlineto{\pgfqpoint{4.005843in}{4.250000in}}%
\pgfpathlineto{\pgfqpoint{3.999583in}{4.250000in}}%
\pgfpathlineto{\pgfqpoint{3.993322in}{4.250000in}}%
\pgfpathlineto{\pgfqpoint{3.987062in}{4.250000in}}%
\pgfpathlineto{\pgfqpoint{3.980801in}{4.250000in}}%
\pgfpathlineto{\pgfqpoint{3.974541in}{4.250000in}}%
\pgfpathlineto{\pgfqpoint{3.968280in}{4.250000in}}%
\pgfpathlineto{\pgfqpoint{3.962020in}{4.250000in}}%
\pgfpathlineto{\pgfqpoint{3.955760in}{4.250000in}}%
\pgfpathlineto{\pgfqpoint{3.949499in}{4.250000in}}%
\pgfpathlineto{\pgfqpoint{3.943239in}{4.250000in}}%
\pgfpathlineto{\pgfqpoint{3.936978in}{4.250000in}}%
\pgfpathlineto{\pgfqpoint{3.930718in}{4.250000in}}%
\pgfpathlineto{\pgfqpoint{3.924457in}{4.250000in}}%
\pgfpathlineto{\pgfqpoint{3.918197in}{4.250000in}}%
\pgfpathlineto{\pgfqpoint{3.911937in}{4.250000in}}%
\pgfpathlineto{\pgfqpoint{3.905676in}{4.250000in}}%
\pgfpathlineto{\pgfqpoint{3.899416in}{4.250000in}}%
\pgfpathlineto{\pgfqpoint{3.893155in}{4.250000in}}%
\pgfpathlineto{\pgfqpoint{3.886895in}{4.250000in}}%
\pgfpathlineto{\pgfqpoint{3.880634in}{4.250000in}}%
\pgfpathlineto{\pgfqpoint{3.874374in}{4.250000in}}%
\pgfpathlineto{\pgfqpoint{3.868114in}{4.250000in}}%
\pgfpathlineto{\pgfqpoint{3.861853in}{4.250000in}}%
\pgfpathlineto{\pgfqpoint{3.855593in}{4.250000in}}%
\pgfpathlineto{\pgfqpoint{3.849332in}{4.250000in}}%
\pgfpathlineto{\pgfqpoint{3.843072in}{4.250000in}}%
\pgfpathlineto{\pgfqpoint{3.836811in}{4.250000in}}%
\pgfpathlineto{\pgfqpoint{3.830551in}{4.250000in}}%
\pgfpathlineto{\pgfqpoint{3.824290in}{4.250000in}}%
\pgfpathlineto{\pgfqpoint{3.818030in}{4.250000in}}%
\pgfpathlineto{\pgfqpoint{3.811770in}{4.250000in}}%
\pgfpathlineto{\pgfqpoint{3.805509in}{4.250000in}}%
\pgfpathlineto{\pgfqpoint{3.799249in}{4.250000in}}%
\pgfpathlineto{\pgfqpoint{3.792988in}{4.250000in}}%
\pgfpathlineto{\pgfqpoint{3.786728in}{4.250000in}}%
\pgfpathlineto{\pgfqpoint{3.780467in}{4.250000in}}%
\pgfpathlineto{\pgfqpoint{3.774207in}{4.250000in}}%
\pgfpathlineto{\pgfqpoint{3.767947in}{4.250000in}}%
\pgfpathlineto{\pgfqpoint{3.761686in}{4.250000in}}%
\pgfpathlineto{\pgfqpoint{3.755426in}{4.250000in}}%
\pgfpathlineto{\pgfqpoint{3.749165in}{4.250000in}}%
\pgfpathlineto{\pgfqpoint{3.742905in}{4.250000in}}%
\pgfpathlineto{\pgfqpoint{3.736644in}{4.250000in}}%
\pgfpathlineto{\pgfqpoint{3.730384in}{4.250000in}}%
\pgfpathlineto{\pgfqpoint{3.724124in}{4.250000in}}%
\pgfpathlineto{\pgfqpoint{3.717863in}{4.250000in}}%
\pgfpathlineto{\pgfqpoint{3.711603in}{4.250000in}}%
\pgfpathlineto{\pgfqpoint{3.705342in}{4.250000in}}%
\pgfpathlineto{\pgfqpoint{3.699082in}{4.250000in}}%
\pgfpathlineto{\pgfqpoint{3.692821in}{4.250000in}}%
\pgfpathlineto{\pgfqpoint{3.686561in}{4.250000in}}%
\pgfpathlineto{\pgfqpoint{3.680301in}{4.250000in}}%
\pgfpathlineto{\pgfqpoint{3.674040in}{4.250000in}}%
\pgfpathlineto{\pgfqpoint{3.667780in}{4.250000in}}%
\pgfpathlineto{\pgfqpoint{3.661519in}{4.250000in}}%
\pgfpathlineto{\pgfqpoint{3.655259in}{4.250000in}}%
\pgfpathlineto{\pgfqpoint{3.648998in}{4.250000in}}%
\pgfpathlineto{\pgfqpoint{3.642738in}{4.250000in}}%
\pgfpathlineto{\pgfqpoint{3.636477in}{4.250000in}}%
\pgfpathlineto{\pgfqpoint{3.630217in}{4.250000in}}%
\pgfpathlineto{\pgfqpoint{3.623957in}{4.250000in}}%
\pgfpathlineto{\pgfqpoint{3.617696in}{4.250000in}}%
\pgfpathlineto{\pgfqpoint{3.611436in}{4.250000in}}%
\pgfpathlineto{\pgfqpoint{3.605175in}{4.250000in}}%
\pgfpathlineto{\pgfqpoint{3.598915in}{4.250000in}}%
\pgfpathlineto{\pgfqpoint{3.592654in}{4.250000in}}%
\pgfpathlineto{\pgfqpoint{3.586394in}{4.250000in}}%
\pgfpathlineto{\pgfqpoint{3.580134in}{4.250000in}}%
\pgfpathlineto{\pgfqpoint{3.573873in}{4.250000in}}%
\pgfpathlineto{\pgfqpoint{3.567613in}{4.250000in}}%
\pgfpathlineto{\pgfqpoint{3.561352in}{4.250000in}}%
\pgfpathlineto{\pgfqpoint{3.555092in}{4.250000in}}%
\pgfpathlineto{\pgfqpoint{3.548831in}{4.250000in}}%
\pgfpathlineto{\pgfqpoint{3.542571in}{4.250000in}}%
\pgfpathlineto{\pgfqpoint{3.536311in}{4.250000in}}%
\pgfpathlineto{\pgfqpoint{3.530050in}{4.250000in}}%
\pgfpathlineto{\pgfqpoint{3.523790in}{4.250000in}}%
\pgfpathlineto{\pgfqpoint{3.517529in}{4.250000in}}%
\pgfpathlineto{\pgfqpoint{3.511269in}{4.250000in}}%
\pgfpathlineto{\pgfqpoint{3.505008in}{4.250000in}}%
\pgfpathlineto{\pgfqpoint{3.498748in}{4.250000in}}%
\pgfpathlineto{\pgfqpoint{3.492487in}{4.250000in}}%
\pgfpathlineto{\pgfqpoint{3.486227in}{4.250000in}}%
\pgfpathlineto{\pgfqpoint{3.479967in}{4.250000in}}%
\pgfpathlineto{\pgfqpoint{3.473706in}{4.250000in}}%
\pgfpathlineto{\pgfqpoint{3.467446in}{4.250000in}}%
\pgfpathlineto{\pgfqpoint{3.461185in}{4.250000in}}%
\pgfpathlineto{\pgfqpoint{3.454925in}{4.250000in}}%
\pgfpathlineto{\pgfqpoint{3.448664in}{4.250000in}}%
\pgfpathlineto{\pgfqpoint{3.442404in}{4.250000in}}%
\pgfpathlineto{\pgfqpoint{3.436144in}{4.250000in}}%
\pgfpathlineto{\pgfqpoint{3.429883in}{4.250000in}}%
\pgfpathlineto{\pgfqpoint{3.423623in}{4.250000in}}%
\pgfpathlineto{\pgfqpoint{3.417362in}{4.250000in}}%
\pgfpathlineto{\pgfqpoint{3.411102in}{4.250000in}}%
\pgfpathlineto{\pgfqpoint{3.404841in}{4.250000in}}%
\pgfpathlineto{\pgfqpoint{3.398581in}{4.250000in}}%
\pgfpathlineto{\pgfqpoint{3.392321in}{4.250000in}}%
\pgfpathlineto{\pgfqpoint{3.386060in}{4.250000in}}%
\pgfpathlineto{\pgfqpoint{3.379800in}{4.250000in}}%
\pgfpathlineto{\pgfqpoint{3.373539in}{4.250000in}}%
\pgfpathlineto{\pgfqpoint{3.367279in}{4.250000in}}%
\pgfpathlineto{\pgfqpoint{3.361018in}{4.250000in}}%
\pgfpathlineto{\pgfqpoint{3.354758in}{4.250000in}}%
\pgfpathlineto{\pgfqpoint{3.348497in}{4.250000in}}%
\pgfpathlineto{\pgfqpoint{3.342237in}{4.250000in}}%
\pgfpathlineto{\pgfqpoint{3.335977in}{4.250000in}}%
\pgfpathlineto{\pgfqpoint{3.329716in}{4.250000in}}%
\pgfpathlineto{\pgfqpoint{3.323456in}{4.250000in}}%
\pgfpathlineto{\pgfqpoint{3.317195in}{4.250000in}}%
\pgfpathlineto{\pgfqpoint{3.310935in}{4.250000in}}%
\pgfpathlineto{\pgfqpoint{3.304674in}{4.250000in}}%
\pgfpathlineto{\pgfqpoint{3.298414in}{4.250000in}}%
\pgfpathlineto{\pgfqpoint{3.292154in}{4.250000in}}%
\pgfpathlineto{\pgfqpoint{3.285893in}{4.250000in}}%
\pgfpathlineto{\pgfqpoint{3.279633in}{4.250000in}}%
\pgfpathlineto{\pgfqpoint{3.273372in}{4.250000in}}%
\pgfpathlineto{\pgfqpoint{3.267112in}{4.250000in}}%
\pgfpathlineto{\pgfqpoint{3.260851in}{4.250000in}}%
\pgfpathlineto{\pgfqpoint{3.254591in}{4.250000in}}%
\pgfpathlineto{\pgfqpoint{3.248331in}{4.250000in}}%
\pgfpathlineto{\pgfqpoint{3.242070in}{4.250000in}}%
\pgfpathlineto{\pgfqpoint{3.235810in}{4.250000in}}%
\pgfpathlineto{\pgfqpoint{3.229549in}{4.250000in}}%
\pgfpathlineto{\pgfqpoint{3.223289in}{4.250000in}}%
\pgfpathlineto{\pgfqpoint{3.217028in}{4.250000in}}%
\pgfpathlineto{\pgfqpoint{3.210768in}{4.250000in}}%
\pgfpathlineto{\pgfqpoint{3.204508in}{4.250000in}}%
\pgfpathlineto{\pgfqpoint{3.198247in}{4.250000in}}%
\pgfpathlineto{\pgfqpoint{3.191987in}{4.250000in}}%
\pgfpathlineto{\pgfqpoint{3.185726in}{4.250000in}}%
\pgfpathlineto{\pgfqpoint{3.179466in}{4.250000in}}%
\pgfpathlineto{\pgfqpoint{3.173205in}{4.250000in}}%
\pgfpathlineto{\pgfqpoint{3.166945in}{4.250000in}}%
\pgfpathlineto{\pgfqpoint{3.160684in}{4.250000in}}%
\pgfpathlineto{\pgfqpoint{3.154424in}{4.250000in}}%
\pgfpathlineto{\pgfqpoint{3.148164in}{4.250000in}}%
\pgfpathlineto{\pgfqpoint{3.141903in}{4.250000in}}%
\pgfpathlineto{\pgfqpoint{3.135643in}{4.250000in}}%
\pgfpathlineto{\pgfqpoint{3.129382in}{4.250000in}}%
\pgfpathlineto{\pgfqpoint{3.123122in}{4.250000in}}%
\pgfpathlineto{\pgfqpoint{3.116861in}{4.250000in}}%
\pgfpathlineto{\pgfqpoint{3.110601in}{4.250000in}}%
\pgfpathlineto{\pgfqpoint{3.104341in}{4.250000in}}%
\pgfpathlineto{\pgfqpoint{3.098080in}{4.250000in}}%
\pgfpathlineto{\pgfqpoint{3.091820in}{4.250000in}}%
\pgfpathlineto{\pgfqpoint{3.085559in}{4.250000in}}%
\pgfpathlineto{\pgfqpoint{3.079299in}{4.250000in}}%
\pgfpathlineto{\pgfqpoint{3.073038in}{4.250000in}}%
\pgfpathlineto{\pgfqpoint{3.066778in}{4.250000in}}%
\pgfpathlineto{\pgfqpoint{3.060518in}{4.250000in}}%
\pgfpathlineto{\pgfqpoint{3.054257in}{4.250000in}}%
\pgfpathlineto{\pgfqpoint{3.047997in}{4.250000in}}%
\pgfpathlineto{\pgfqpoint{3.041736in}{4.250000in}}%
\pgfpathlineto{\pgfqpoint{3.035476in}{4.250000in}}%
\pgfpathlineto{\pgfqpoint{3.029215in}{4.250000in}}%
\pgfpathlineto{\pgfqpoint{3.022955in}{4.250000in}}%
\pgfpathlineto{\pgfqpoint{3.016694in}{4.250000in}}%
\pgfpathlineto{\pgfqpoint{3.010434in}{4.250000in}}%
\pgfpathlineto{\pgfqpoint{3.004174in}{4.250000in}}%
\pgfpathlineto{\pgfqpoint{2.997913in}{4.250000in}}%
\pgfpathlineto{\pgfqpoint{2.991653in}{4.250000in}}%
\pgfpathlineto{\pgfqpoint{2.985392in}{4.250000in}}%
\pgfpathlineto{\pgfqpoint{2.979132in}{4.250000in}}%
\pgfpathlineto{\pgfqpoint{2.972871in}{4.250000in}}%
\pgfpathlineto{\pgfqpoint{2.966611in}{4.250000in}}%
\pgfpathlineto{\pgfqpoint{2.960351in}{4.250000in}}%
\pgfpathlineto{\pgfqpoint{2.954090in}{4.250000in}}%
\pgfpathlineto{\pgfqpoint{2.947830in}{4.250000in}}%
\pgfpathlineto{\pgfqpoint{2.941569in}{4.250000in}}%
\pgfpathlineto{\pgfqpoint{2.935309in}{4.250000in}}%
\pgfpathlineto{\pgfqpoint{2.929048in}{4.250000in}}%
\pgfpathlineto{\pgfqpoint{2.922788in}{4.250000in}}%
\pgfpathlineto{\pgfqpoint{2.916528in}{4.250000in}}%
\pgfpathlineto{\pgfqpoint{2.910267in}{4.250000in}}%
\pgfpathlineto{\pgfqpoint{2.904007in}{4.250000in}}%
\pgfpathlineto{\pgfqpoint{2.897746in}{4.250000in}}%
\pgfpathlineto{\pgfqpoint{2.891486in}{4.250000in}}%
\pgfpathlineto{\pgfqpoint{2.885225in}{4.250000in}}%
\pgfpathlineto{\pgfqpoint{2.878965in}{4.250000in}}%
\pgfpathlineto{\pgfqpoint{2.872705in}{4.250000in}}%
\pgfpathlineto{\pgfqpoint{2.866444in}{4.250000in}}%
\pgfpathlineto{\pgfqpoint{2.860184in}{4.250000in}}%
\pgfpathlineto{\pgfqpoint{2.853923in}{4.250000in}}%
\pgfpathlineto{\pgfqpoint{2.847663in}{4.250000in}}%
\pgfpathlineto{\pgfqpoint{2.841402in}{4.250000in}}%
\pgfpathlineto{\pgfqpoint{2.835142in}{4.250000in}}%
\pgfpathlineto{\pgfqpoint{2.828881in}{4.250000in}}%
\pgfpathlineto{\pgfqpoint{2.822621in}{4.250000in}}%
\pgfpathlineto{\pgfqpoint{2.816361in}{4.250000in}}%
\pgfpathlineto{\pgfqpoint{2.810100in}{4.250000in}}%
\pgfpathlineto{\pgfqpoint{2.803840in}{4.250000in}}%
\pgfpathlineto{\pgfqpoint{2.797579in}{4.250000in}}%
\pgfpathlineto{\pgfqpoint{2.791319in}{4.250000in}}%
\pgfpathlineto{\pgfqpoint{2.785058in}{4.250000in}}%
\pgfpathlineto{\pgfqpoint{2.778798in}{4.250000in}}%
\pgfpathlineto{\pgfqpoint{2.772538in}{4.250000in}}%
\pgfpathlineto{\pgfqpoint{2.766277in}{4.250000in}}%
\pgfpathlineto{\pgfqpoint{2.760017in}{4.250000in}}%
\pgfpathlineto{\pgfqpoint{2.753756in}{4.250000in}}%
\pgfpathlineto{\pgfqpoint{2.747496in}{4.250000in}}%
\pgfpathlineto{\pgfqpoint{2.741235in}{4.250000in}}%
\pgfpathlineto{\pgfqpoint{2.734975in}{4.250000in}}%
\pgfpathlineto{\pgfqpoint{2.728715in}{4.250000in}}%
\pgfpathlineto{\pgfqpoint{2.722454in}{4.250000in}}%
\pgfpathlineto{\pgfqpoint{2.716194in}{4.250000in}}%
\pgfpathlineto{\pgfqpoint{2.709933in}{4.250000in}}%
\pgfpathlineto{\pgfqpoint{2.703673in}{4.250000in}}%
\pgfpathlineto{\pgfqpoint{2.697412in}{4.250000in}}%
\pgfpathlineto{\pgfqpoint{2.691152in}{4.250000in}}%
\pgfpathlineto{\pgfqpoint{2.684891in}{4.250000in}}%
\pgfpathlineto{\pgfqpoint{2.678631in}{4.250000in}}%
\pgfpathlineto{\pgfqpoint{2.672371in}{4.250000in}}%
\pgfpathlineto{\pgfqpoint{2.666110in}{4.250000in}}%
\pgfpathlineto{\pgfqpoint{2.659850in}{4.250000in}}%
\pgfpathlineto{\pgfqpoint{2.653589in}{4.250000in}}%
\pgfpathlineto{\pgfqpoint{2.647329in}{4.250000in}}%
\pgfpathlineto{\pgfqpoint{2.641068in}{4.250000in}}%
\pgfpathlineto{\pgfqpoint{2.634808in}{4.250000in}}%
\pgfpathlineto{\pgfqpoint{2.628548in}{4.250000in}}%
\pgfpathlineto{\pgfqpoint{2.622287in}{4.250000in}}%
\pgfpathlineto{\pgfqpoint{2.616027in}{4.250000in}}%
\pgfpathlineto{\pgfqpoint{2.609766in}{4.250000in}}%
\pgfpathlineto{\pgfqpoint{2.603506in}{4.250000in}}%
\pgfpathlineto{\pgfqpoint{2.597245in}{4.250000in}}%
\pgfpathlineto{\pgfqpoint{2.590985in}{4.250000in}}%
\pgfpathlineto{\pgfqpoint{2.584725in}{4.250000in}}%
\pgfpathlineto{\pgfqpoint{2.578464in}{4.250000in}}%
\pgfpathlineto{\pgfqpoint{2.572204in}{4.250000in}}%
\pgfpathlineto{\pgfqpoint{2.565943in}{4.250000in}}%
\pgfpathlineto{\pgfqpoint{2.559683in}{4.250000in}}%
\pgfpathlineto{\pgfqpoint{2.553422in}{4.250000in}}%
\pgfpathlineto{\pgfqpoint{2.547162in}{4.250000in}}%
\pgfpathlineto{\pgfqpoint{2.540902in}{4.250000in}}%
\pgfpathlineto{\pgfqpoint{2.534641in}{4.250000in}}%
\pgfpathlineto{\pgfqpoint{2.528381in}{4.250000in}}%
\pgfpathlineto{\pgfqpoint{2.522120in}{4.250000in}}%
\pgfpathlineto{\pgfqpoint{2.515860in}{4.250000in}}%
\pgfpathlineto{\pgfqpoint{2.509599in}{4.250000in}}%
\pgfpathlineto{\pgfqpoint{2.503339in}{4.250000in}}%
\pgfpathlineto{\pgfqpoint{2.497078in}{4.250000in}}%
\pgfpathlineto{\pgfqpoint{2.490818in}{4.250000in}}%
\pgfpathlineto{\pgfqpoint{2.484558in}{4.250000in}}%
\pgfpathlineto{\pgfqpoint{2.478297in}{4.250000in}}%
\pgfpathlineto{\pgfqpoint{2.472037in}{4.250000in}}%
\pgfpathlineto{\pgfqpoint{2.465776in}{4.250000in}}%
\pgfpathlineto{\pgfqpoint{2.459516in}{4.250000in}}%
\pgfpathlineto{\pgfqpoint{2.453255in}{4.250000in}}%
\pgfpathlineto{\pgfqpoint{2.446995in}{4.250000in}}%
\pgfpathlineto{\pgfqpoint{2.440735in}{4.250000in}}%
\pgfpathlineto{\pgfqpoint{2.434474in}{4.250000in}}%
\pgfpathlineto{\pgfqpoint{2.428214in}{4.250000in}}%
\pgfpathlineto{\pgfqpoint{2.421953in}{4.250000in}}%
\pgfpathlineto{\pgfqpoint{2.415693in}{4.250000in}}%
\pgfpathlineto{\pgfqpoint{2.409432in}{4.250000in}}%
\pgfpathlineto{\pgfqpoint{2.403172in}{4.250000in}}%
\pgfpathlineto{\pgfqpoint{2.396912in}{4.250000in}}%
\pgfpathlineto{\pgfqpoint{2.390651in}{4.250000in}}%
\pgfpathlineto{\pgfqpoint{2.384391in}{4.250000in}}%
\pgfpathlineto{\pgfqpoint{2.378130in}{4.250000in}}%
\pgfpathlineto{\pgfqpoint{2.371870in}{4.250000in}}%
\pgfpathlineto{\pgfqpoint{2.365609in}{4.250000in}}%
\pgfpathlineto{\pgfqpoint{2.359349in}{4.250000in}}%
\pgfpathlineto{\pgfqpoint{2.353088in}{4.250000in}}%
\pgfpathlineto{\pgfqpoint{2.346828in}{4.250000in}}%
\pgfpathlineto{\pgfqpoint{2.340568in}{4.250000in}}%
\pgfpathlineto{\pgfqpoint{2.334307in}{4.250000in}}%
\pgfpathlineto{\pgfqpoint{2.328047in}{4.250000in}}%
\pgfpathlineto{\pgfqpoint{2.321786in}{4.250000in}}%
\pgfpathlineto{\pgfqpoint{2.315526in}{4.250000in}}%
\pgfpathlineto{\pgfqpoint{2.309265in}{4.250000in}}%
\pgfpathlineto{\pgfqpoint{2.303005in}{4.250000in}}%
\pgfpathlineto{\pgfqpoint{2.296745in}{4.250000in}}%
\pgfpathlineto{\pgfqpoint{2.290484in}{4.250000in}}%
\pgfpathlineto{\pgfqpoint{2.284224in}{4.250000in}}%
\pgfpathlineto{\pgfqpoint{2.277963in}{4.250000in}}%
\pgfpathlineto{\pgfqpoint{2.271703in}{4.250000in}}%
\pgfpathlineto{\pgfqpoint{2.265442in}{4.250000in}}%
\pgfpathlineto{\pgfqpoint{2.259182in}{4.250000in}}%
\pgfpathlineto{\pgfqpoint{2.252922in}{4.250000in}}%
\pgfpathlineto{\pgfqpoint{2.246661in}{4.250000in}}%
\pgfpathlineto{\pgfqpoint{2.240401in}{4.250000in}}%
\pgfpathlineto{\pgfqpoint{2.234140in}{4.250000in}}%
\pgfpathlineto{\pgfqpoint{2.227880in}{4.250000in}}%
\pgfpathlineto{\pgfqpoint{2.221619in}{4.250000in}}%
\pgfpathlineto{\pgfqpoint{2.215359in}{4.250000in}}%
\pgfpathlineto{\pgfqpoint{2.209098in}{4.250000in}}%
\pgfpathlineto{\pgfqpoint{2.202838in}{4.250000in}}%
\pgfpathlineto{\pgfqpoint{2.196578in}{4.250000in}}%
\pgfpathlineto{\pgfqpoint{2.190317in}{4.250000in}}%
\pgfpathlineto{\pgfqpoint{2.184057in}{4.250000in}}%
\pgfpathlineto{\pgfqpoint{2.177796in}{4.250000in}}%
\pgfpathlineto{\pgfqpoint{2.171536in}{4.250000in}}%
\pgfpathlineto{\pgfqpoint{2.165275in}{4.250000in}}%
\pgfpathlineto{\pgfqpoint{2.159015in}{4.250000in}}%
\pgfpathlineto{\pgfqpoint{2.152755in}{4.250000in}}%
\pgfpathlineto{\pgfqpoint{2.146494in}{4.250000in}}%
\pgfpathlineto{\pgfqpoint{2.140234in}{4.250000in}}%
\pgfpathlineto{\pgfqpoint{2.133973in}{4.250000in}}%
\pgfpathlineto{\pgfqpoint{2.127713in}{4.250000in}}%
\pgfpathlineto{\pgfqpoint{2.121452in}{4.250000in}}%
\pgfpathlineto{\pgfqpoint{2.115192in}{4.250000in}}%
\pgfpathlineto{\pgfqpoint{2.108932in}{4.250000in}}%
\pgfpathlineto{\pgfqpoint{2.102671in}{4.250000in}}%
\pgfpathlineto{\pgfqpoint{2.096411in}{4.250000in}}%
\pgfpathlineto{\pgfqpoint{2.090150in}{4.250000in}}%
\pgfpathlineto{\pgfqpoint{2.083890in}{4.250000in}}%
\pgfpathlineto{\pgfqpoint{2.077629in}{4.250000in}}%
\pgfpathlineto{\pgfqpoint{2.071369in}{4.250000in}}%
\pgfpathlineto{\pgfqpoint{2.065109in}{4.250000in}}%
\pgfpathlineto{\pgfqpoint{2.058848in}{4.250000in}}%
\pgfpathlineto{\pgfqpoint{2.052588in}{4.250000in}}%
\pgfpathlineto{\pgfqpoint{2.046327in}{4.250000in}}%
\pgfpathlineto{\pgfqpoint{2.040067in}{4.250000in}}%
\pgfpathlineto{\pgfqpoint{2.033806in}{4.250000in}}%
\pgfpathlineto{\pgfqpoint{2.027546in}{4.250000in}}%
\pgfpathlineto{\pgfqpoint{2.021285in}{4.250000in}}%
\pgfpathlineto{\pgfqpoint{2.015025in}{4.250000in}}%
\pgfpathlineto{\pgfqpoint{2.008765in}{4.250000in}}%
\pgfpathlineto{\pgfqpoint{2.002504in}{4.250000in}}%
\pgfpathlineto{\pgfqpoint{1.996244in}{4.250000in}}%
\pgfpathlineto{\pgfqpoint{1.989983in}{4.250000in}}%
\pgfpathlineto{\pgfqpoint{1.983723in}{4.250000in}}%
\pgfpathlineto{\pgfqpoint{1.977462in}{4.250000in}}%
\pgfpathlineto{\pgfqpoint{1.971202in}{4.250000in}}%
\pgfpathlineto{\pgfqpoint{1.964942in}{4.250000in}}%
\pgfpathlineto{\pgfqpoint{1.958681in}{4.250000in}}%
\pgfpathlineto{\pgfqpoint{1.952421in}{4.250000in}}%
\pgfpathlineto{\pgfqpoint{1.946160in}{4.250000in}}%
\pgfpathlineto{\pgfqpoint{1.939900in}{4.250000in}}%
\pgfpathlineto{\pgfqpoint{1.933639in}{4.250000in}}%
\pgfpathlineto{\pgfqpoint{1.927379in}{4.250000in}}%
\pgfpathlineto{\pgfqpoint{1.921119in}{4.250000in}}%
\pgfpathlineto{\pgfqpoint{1.914858in}{4.250000in}}%
\pgfpathlineto{\pgfqpoint{1.908598in}{4.250000in}}%
\pgfpathlineto{\pgfqpoint{1.902337in}{4.250000in}}%
\pgfpathlineto{\pgfqpoint{1.896077in}{4.250000in}}%
\pgfpathlineto{\pgfqpoint{1.889816in}{4.250000in}}%
\pgfpathlineto{\pgfqpoint{1.883556in}{4.250000in}}%
\pgfpathlineto{\pgfqpoint{1.877295in}{4.250000in}}%
\pgfpathlineto{\pgfqpoint{1.871035in}{4.250000in}}%
\pgfpathlineto{\pgfqpoint{1.864775in}{4.250000in}}%
\pgfpathlineto{\pgfqpoint{1.858514in}{4.250000in}}%
\pgfpathlineto{\pgfqpoint{1.852254in}{4.250000in}}%
\pgfpathlineto{\pgfqpoint{1.845993in}{4.250000in}}%
\pgfpathlineto{\pgfqpoint{1.839733in}{4.250000in}}%
\pgfpathlineto{\pgfqpoint{1.833472in}{4.250000in}}%
\pgfpathlineto{\pgfqpoint{1.827212in}{4.250000in}}%
\pgfpathlineto{\pgfqpoint{1.820952in}{4.250000in}}%
\pgfpathlineto{\pgfqpoint{1.814691in}{4.250000in}}%
\pgfpathlineto{\pgfqpoint{1.808431in}{4.250000in}}%
\pgfpathlineto{\pgfqpoint{1.802170in}{4.250000in}}%
\pgfpathlineto{\pgfqpoint{1.795910in}{4.250000in}}%
\pgfpathlineto{\pgfqpoint{1.789649in}{4.250000in}}%
\pgfpathlineto{\pgfqpoint{1.783389in}{4.250000in}}%
\pgfpathlineto{\pgfqpoint{1.777129in}{4.250000in}}%
\pgfpathlineto{\pgfqpoint{1.770868in}{4.250000in}}%
\pgfpathlineto{\pgfqpoint{1.764608in}{4.250000in}}%
\pgfpathlineto{\pgfqpoint{1.758347in}{4.250000in}}%
\pgfpathlineto{\pgfqpoint{1.752087in}{4.250000in}}%
\pgfpathlineto{\pgfqpoint{1.745826in}{4.250000in}}%
\pgfpathlineto{\pgfqpoint{1.739566in}{4.250000in}}%
\pgfpathlineto{\pgfqpoint{1.733306in}{4.250000in}}%
\pgfpathlineto{\pgfqpoint{1.727045in}{4.250000in}}%
\pgfpathlineto{\pgfqpoint{1.720785in}{4.250000in}}%
\pgfpathlineto{\pgfqpoint{1.714524in}{4.250000in}}%
\pgfpathlineto{\pgfqpoint{1.708264in}{4.250000in}}%
\pgfpathlineto{\pgfqpoint{1.702003in}{4.250000in}}%
\pgfpathlineto{\pgfqpoint{1.695743in}{4.250000in}}%
\pgfpathlineto{\pgfqpoint{1.689482in}{4.250000in}}%
\pgfpathlineto{\pgfqpoint{1.683222in}{4.250000in}}%
\pgfpathlineto{\pgfqpoint{1.676962in}{4.250000in}}%
\pgfpathlineto{\pgfqpoint{1.670701in}{4.250000in}}%
\pgfpathlineto{\pgfqpoint{1.664441in}{4.250000in}}%
\pgfpathlineto{\pgfqpoint{1.658180in}{4.250000in}}%
\pgfpathlineto{\pgfqpoint{1.651920in}{4.250000in}}%
\pgfpathlineto{\pgfqpoint{1.645659in}{4.250000in}}%
\pgfpathlineto{\pgfqpoint{1.639399in}{4.250000in}}%
\pgfpathlineto{\pgfqpoint{1.633139in}{4.250000in}}%
\pgfpathlineto{\pgfqpoint{1.626878in}{4.250000in}}%
\pgfpathlineto{\pgfqpoint{1.620618in}{4.250000in}}%
\pgfpathlineto{\pgfqpoint{1.614357in}{4.250000in}}%
\pgfpathlineto{\pgfqpoint{1.608097in}{4.250000in}}%
\pgfpathlineto{\pgfqpoint{1.601836in}{4.250000in}}%
\pgfpathlineto{\pgfqpoint{1.595576in}{4.250000in}}%
\pgfpathlineto{\pgfqpoint{1.589316in}{4.250000in}}%
\pgfpathlineto{\pgfqpoint{1.583055in}{4.250000in}}%
\pgfpathlineto{\pgfqpoint{1.576795in}{4.250000in}}%
\pgfpathlineto{\pgfqpoint{1.570534in}{4.250000in}}%
\pgfpathlineto{\pgfqpoint{1.564274in}{4.250000in}}%
\pgfpathlineto{\pgfqpoint{1.558013in}{4.250000in}}%
\pgfpathlineto{\pgfqpoint{1.551753in}{4.250000in}}%
\pgfpathlineto{\pgfqpoint{1.545492in}{4.250000in}}%
\pgfpathlineto{\pgfqpoint{1.539232in}{4.250000in}}%
\pgfpathlineto{\pgfqpoint{1.532972in}{4.250000in}}%
\pgfpathlineto{\pgfqpoint{1.526711in}{4.250000in}}%
\pgfpathlineto{\pgfqpoint{1.520451in}{4.250000in}}%
\pgfpathlineto{\pgfqpoint{1.514190in}{4.250000in}}%
\pgfpathlineto{\pgfqpoint{1.507930in}{4.250000in}}%
\pgfpathlineto{\pgfqpoint{1.501669in}{4.250000in}}%
\pgfpathlineto{\pgfqpoint{1.495409in}{4.250000in}}%
\pgfpathlineto{\pgfqpoint{1.489149in}{4.250000in}}%
\pgfpathlineto{\pgfqpoint{1.482888in}{4.250000in}}%
\pgfpathlineto{\pgfqpoint{1.476628in}{4.250000in}}%
\pgfpathlineto{\pgfqpoint{1.470367in}{4.250000in}}%
\pgfpathlineto{\pgfqpoint{1.464107in}{4.250000in}}%
\pgfpathlineto{\pgfqpoint{1.457846in}{4.250000in}}%
\pgfpathlineto{\pgfqpoint{1.451586in}{4.250000in}}%
\pgfpathlineto{\pgfqpoint{1.445326in}{4.250000in}}%
\pgfpathlineto{\pgfqpoint{1.439065in}{4.250000in}}%
\pgfpathlineto{\pgfqpoint{1.432805in}{4.250000in}}%
\pgfpathlineto{\pgfqpoint{1.426544in}{4.250000in}}%
\pgfpathlineto{\pgfqpoint{1.420284in}{4.250000in}}%
\pgfpathlineto{\pgfqpoint{1.414023in}{4.250000in}}%
\pgfpathlineto{\pgfqpoint{1.407763in}{4.250000in}}%
\pgfpathlineto{\pgfqpoint{1.401503in}{4.250000in}}%
\pgfpathlineto{\pgfqpoint{1.395242in}{4.250000in}}%
\pgfpathlineto{\pgfqpoint{1.388982in}{4.250000in}}%
\pgfpathlineto{\pgfqpoint{1.382721in}{4.250000in}}%
\pgfpathlineto{\pgfqpoint{1.376461in}{4.250000in}}%
\pgfpathlineto{\pgfqpoint{1.370200in}{4.250000in}}%
\pgfpathlineto{\pgfqpoint{1.363940in}{4.250000in}}%
\pgfpathlineto{\pgfqpoint{1.357679in}{4.250000in}}%
\pgfpathlineto{\pgfqpoint{1.351419in}{4.250000in}}%
\pgfpathlineto{\pgfqpoint{1.345159in}{4.250000in}}%
\pgfpathlineto{\pgfqpoint{1.338898in}{4.250000in}}%
\pgfpathlineto{\pgfqpoint{1.332638in}{4.250000in}}%
\pgfpathlineto{\pgfqpoint{1.326377in}{4.250000in}}%
\pgfpathlineto{\pgfqpoint{1.320117in}{4.250000in}}%
\pgfpathlineto{\pgfqpoint{1.313856in}{4.250000in}}%
\pgfpathlineto{\pgfqpoint{1.307596in}{4.250000in}}%
\pgfpathlineto{\pgfqpoint{1.301336in}{4.250000in}}%
\pgfpathlineto{\pgfqpoint{1.295075in}{4.250000in}}%
\pgfpathlineto{\pgfqpoint{1.288815in}{4.250000in}}%
\pgfpathlineto{\pgfqpoint{1.282554in}{4.250000in}}%
\pgfpathlineto{\pgfqpoint{1.276294in}{4.250000in}}%
\pgfpathlineto{\pgfqpoint{1.270033in}{4.250000in}}%
\pgfpathlineto{\pgfqpoint{1.263773in}{4.250000in}}%
\pgfpathlineto{\pgfqpoint{1.257513in}{4.250000in}}%
\pgfpathlineto{\pgfqpoint{1.251252in}{4.250000in}}%
\pgfpathlineto{\pgfqpoint{1.244992in}{4.250000in}}%
\pgfpathlineto{\pgfqpoint{1.238731in}{4.250000in}}%
\pgfpathlineto{\pgfqpoint{1.232471in}{4.250000in}}%
\pgfpathlineto{\pgfqpoint{1.226210in}{4.250000in}}%
\pgfpathlineto{\pgfqpoint{1.219950in}{4.250000in}}%
\pgfpathlineto{\pgfqpoint{1.213689in}{4.250000in}}%
\pgfpathlineto{\pgfqpoint{1.207429in}{4.250000in}}%
\pgfpathlineto{\pgfqpoint{1.201169in}{4.250000in}}%
\pgfpathlineto{\pgfqpoint{1.194908in}{4.250000in}}%
\pgfpathlineto{\pgfqpoint{1.188648in}{4.250000in}}%
\pgfpathlineto{\pgfqpoint{1.182387in}{4.250000in}}%
\pgfpathlineto{\pgfqpoint{1.176127in}{4.250000in}}%
\pgfpathlineto{\pgfqpoint{1.169866in}{4.250000in}}%
\pgfpathlineto{\pgfqpoint{1.163606in}{4.250000in}}%
\pgfpathlineto{\pgfqpoint{1.157346in}{4.250000in}}%
\pgfpathlineto{\pgfqpoint{1.151085in}{4.250000in}}%
\pgfpathlineto{\pgfqpoint{1.144825in}{4.250000in}}%
\pgfpathlineto{\pgfqpoint{1.138564in}{4.250000in}}%
\pgfpathlineto{\pgfqpoint{1.132304in}{4.250000in}}%
\pgfpathlineto{\pgfqpoint{1.126043in}{4.250000in}}%
\pgfpathlineto{\pgfqpoint{1.119783in}{4.250000in}}%
\pgfpathlineto{\pgfqpoint{1.113523in}{4.250000in}}%
\pgfpathlineto{\pgfqpoint{1.107262in}{4.250000in}}%
\pgfpathlineto{\pgfqpoint{1.101002in}{4.250000in}}%
\pgfpathlineto{\pgfqpoint{1.094741in}{4.250000in}}%
\pgfpathlineto{\pgfqpoint{1.088481in}{4.250000in}}%
\pgfpathlineto{\pgfqpoint{1.082220in}{4.250000in}}%
\pgfpathlineto{\pgfqpoint{1.075960in}{4.250000in}}%
\pgfpathlineto{\pgfqpoint{1.069699in}{4.250000in}}%
\pgfpathlineto{\pgfqpoint{1.063439in}{4.250000in}}%
\pgfpathlineto{\pgfqpoint{1.057179in}{4.250000in}}%
\pgfpathlineto{\pgfqpoint{1.050918in}{4.250000in}}%
\pgfpathlineto{\pgfqpoint{1.044658in}{4.250000in}}%
\pgfpathlineto{\pgfqpoint{1.038397in}{4.250000in}}%
\pgfpathlineto{\pgfqpoint{1.032137in}{4.250000in}}%
\pgfpathlineto{\pgfqpoint{1.025876in}{4.250000in}}%
\pgfpathlineto{\pgfqpoint{1.019616in}{4.250000in}}%
\pgfpathlineto{\pgfqpoint{1.013356in}{4.250000in}}%
\pgfpathlineto{\pgfqpoint{1.007095in}{4.250000in}}%
\pgfpathlineto{\pgfqpoint{1.000835in}{4.250000in}}%
\pgfpathlineto{\pgfqpoint{0.994574in}{4.250000in}}%
\pgfpathlineto{\pgfqpoint{0.988314in}{4.250000in}}%
\pgfpathlineto{\pgfqpoint{0.982053in}{4.250000in}}%
\pgfpathlineto{\pgfqpoint{0.975793in}{4.250000in}}%
\pgfpathlineto{\pgfqpoint{0.969533in}{4.250000in}}%
\pgfpathlineto{\pgfqpoint{0.963272in}{4.250000in}}%
\pgfpathlineto{\pgfqpoint{0.957012in}{4.250000in}}%
\pgfpathlineto{\pgfqpoint{0.950751in}{4.250000in}}%
\pgfpathlineto{\pgfqpoint{0.944491in}{4.250000in}}%
\pgfpathlineto{\pgfqpoint{0.938230in}{4.250000in}}%
\pgfpathlineto{\pgfqpoint{0.931970in}{4.250000in}}%
\pgfpathlineto{\pgfqpoint{0.925710in}{4.250000in}}%
\pgfpathlineto{\pgfqpoint{0.919449in}{4.250000in}}%
\pgfpathlineto{\pgfqpoint{0.913189in}{4.250000in}}%
\pgfpathlineto{\pgfqpoint{0.906928in}{4.250000in}}%
\pgfpathlineto{\pgfqpoint{0.900668in}{4.250000in}}%
\pgfpathlineto{\pgfqpoint{0.894407in}{4.250000in}}%
\pgfpathlineto{\pgfqpoint{0.888147in}{4.250000in}}%
\pgfpathlineto{\pgfqpoint{0.881886in}{4.250000in}}%
\pgfpathlineto{\pgfqpoint{0.875626in}{4.250000in}}%
\pgfpathlineto{\pgfqpoint{0.869366in}{4.250000in}}%
\pgfpathlineto{\pgfqpoint{0.863105in}{4.250000in}}%
\pgfpathlineto{\pgfqpoint{0.856845in}{4.250000in}}%
\pgfpathlineto{\pgfqpoint{0.850584in}{4.250000in}}%
\pgfpathlineto{\pgfqpoint{0.844324in}{4.250000in}}%
\pgfpathlineto{\pgfqpoint{0.838063in}{4.250000in}}%
\pgfpathlineto{\pgfqpoint{0.831803in}{4.250000in}}%
\pgfpathlineto{\pgfqpoint{0.825543in}{4.250000in}}%
\pgfpathlineto{\pgfqpoint{0.819282in}{4.250000in}}%
\pgfpathlineto{\pgfqpoint{0.813022in}{4.250000in}}%
\pgfpathlineto{\pgfqpoint{0.806761in}{4.250000in}}%
\pgfpathlineto{\pgfqpoint{0.800501in}{4.250000in}}%
\pgfpathlineto{\pgfqpoint{0.794240in}{4.250000in}}%
\pgfpathlineto{\pgfqpoint{0.787980in}{4.250000in}}%
\pgfpathlineto{\pgfqpoint{0.781720in}{4.250000in}}%
\pgfpathlineto{\pgfqpoint{0.775459in}{4.250000in}}%
\pgfpathlineto{\pgfqpoint{0.769199in}{4.250000in}}%
\pgfpathlineto{\pgfqpoint{0.762938in}{4.250000in}}%
\pgfpathlineto{\pgfqpoint{0.756678in}{4.250000in}}%
\pgfpathlineto{\pgfqpoint{0.750417in}{4.250000in}}%
\pgfpathlineto{\pgfqpoint{0.744157in}{4.250000in}}%
\pgfpathlineto{\pgfqpoint{0.737896in}{4.250000in}}%
\pgfpathlineto{\pgfqpoint{0.731636in}{4.250000in}}%
\pgfpathlineto{\pgfqpoint{0.725376in}{4.250000in}}%
\pgfpathlineto{\pgfqpoint{0.719115in}{4.250000in}}%
\pgfpathlineto{\pgfqpoint{0.712855in}{4.250000in}}%
\pgfpathlineto{\pgfqpoint{0.706594in}{4.250000in}}%
\pgfpathlineto{\pgfqpoint{0.700334in}{4.250000in}}%
\pgfpathlineto{\pgfqpoint{0.694073in}{4.250000in}}%
\pgfpathlineto{\pgfqpoint{0.687813in}{4.250000in}}%
\pgfpathlineto{\pgfqpoint{0.681553in}{4.250000in}}%
\pgfpathlineto{\pgfqpoint{0.675292in}{4.250000in}}%
\pgfpathlineto{\pgfqpoint{0.669032in}{4.250000in}}%
\pgfpathlineto{\pgfqpoint{0.662771in}{4.250000in}}%
\pgfpathlineto{\pgfqpoint{0.656511in}{4.250000in}}%
\pgfpathlineto{\pgfqpoint{0.650250in}{4.250000in}}%
\pgfpathlineto{\pgfqpoint{0.643990in}{4.250000in}}%
\pgfpathlineto{\pgfqpoint{0.637730in}{4.250000in}}%
\pgfpathlineto{\pgfqpoint{0.631469in}{4.250000in}}%
\pgfpathlineto{\pgfqpoint{0.625209in}{4.250000in}}%
\pgfpathlineto{\pgfqpoint{0.618948in}{4.250000in}}%
\pgfpathlineto{\pgfqpoint{0.612688in}{4.250000in}}%
\pgfpathlineto{\pgfqpoint{0.606427in}{4.250000in}}%
\pgfpathlineto{\pgfqpoint{0.600167in}{4.250000in}}%
\pgfpathlineto{\pgfqpoint{0.593907in}{4.250000in}}%
\pgfpathlineto{\pgfqpoint{0.587646in}{4.250000in}}%
\pgfpathlineto{\pgfqpoint{0.581386in}{4.250000in}}%
\pgfpathlineto{\pgfqpoint{0.575125in}{4.250000in}}%
\pgfpathlineto{\pgfqpoint{0.568865in}{4.250000in}}%
\pgfpathlineto{\pgfqpoint{0.562604in}{4.250000in}}%
\pgfpathlineto{\pgfqpoint{0.556344in}{4.250000in}}%
\pgfpathlineto{\pgfqpoint{0.550083in}{4.250000in}}%
\pgfpathlineto{\pgfqpoint{0.543823in}{4.250000in}}%
\pgfpathlineto{\pgfqpoint{0.537563in}{4.250000in}}%
\pgfpathlineto{\pgfqpoint{0.531302in}{4.250000in}}%
\pgfpathlineto{\pgfqpoint{0.525042in}{4.250000in}}%
\pgfpathlineto{\pgfqpoint{0.518781in}{4.250000in}}%
\pgfpathlineto{\pgfqpoint{0.512521in}{4.250000in}}%
\pgfpathlineto{\pgfqpoint{0.506260in}{4.250000in}}%
\pgfpathlineto{\pgfqpoint{0.500000in}{4.250000in}}%
\pgfpathlineto{\pgfqpoint{0.500000in}{4.243740in}}%
\pgfpathlineto{\pgfqpoint{0.500000in}{4.237479in}}%
\pgfpathlineto{\pgfqpoint{0.500000in}{4.231219in}}%
\pgfpathlineto{\pgfqpoint{0.500000in}{4.224958in}}%
\pgfpathlineto{\pgfqpoint{0.500000in}{4.218698in}}%
\pgfpathlineto{\pgfqpoint{0.500000in}{4.212437in}}%
\pgfpathlineto{\pgfqpoint{0.500000in}{4.206177in}}%
\pgfpathlineto{\pgfqpoint{0.500000in}{4.199917in}}%
\pgfpathlineto{\pgfqpoint{0.500000in}{4.193656in}}%
\pgfpathlineto{\pgfqpoint{0.500000in}{4.187396in}}%
\pgfpathlineto{\pgfqpoint{0.500000in}{4.181135in}}%
\pgfpathlineto{\pgfqpoint{0.500000in}{4.174875in}}%
\pgfpathlineto{\pgfqpoint{0.500000in}{4.168614in}}%
\pgfpathlineto{\pgfqpoint{0.500000in}{4.162354in}}%
\pgfpathlineto{\pgfqpoint{0.500000in}{4.156093in}}%
\pgfpathlineto{\pgfqpoint{0.500000in}{4.149833in}}%
\pgfpathlineto{\pgfqpoint{0.500000in}{4.143573in}}%
\pgfpathlineto{\pgfqpoint{0.500000in}{4.137312in}}%
\pgfpathlineto{\pgfqpoint{0.500000in}{4.131052in}}%
\pgfpathlineto{\pgfqpoint{0.500000in}{4.124791in}}%
\pgfpathlineto{\pgfqpoint{0.500000in}{4.118531in}}%
\pgfpathlineto{\pgfqpoint{0.500000in}{4.112270in}}%
\pgfpathlineto{\pgfqpoint{0.500000in}{4.106010in}}%
\pgfpathlineto{\pgfqpoint{0.500000in}{4.099750in}}%
\pgfpathlineto{\pgfqpoint{0.500000in}{4.093489in}}%
\pgfpathlineto{\pgfqpoint{0.500000in}{4.087229in}}%
\pgfpathlineto{\pgfqpoint{0.500000in}{4.080968in}}%
\pgfpathlineto{\pgfqpoint{0.500000in}{4.074708in}}%
\pgfpathlineto{\pgfqpoint{0.500000in}{4.068447in}}%
\pgfpathlineto{\pgfqpoint{0.500000in}{4.062187in}}%
\pgfpathlineto{\pgfqpoint{0.500000in}{4.055927in}}%
\pgfpathlineto{\pgfqpoint{0.500000in}{4.049666in}}%
\pgfpathlineto{\pgfqpoint{0.500000in}{4.043406in}}%
\pgfpathlineto{\pgfqpoint{0.500000in}{4.037145in}}%
\pgfpathlineto{\pgfqpoint{0.500000in}{4.030885in}}%
\pgfpathlineto{\pgfqpoint{0.500000in}{4.024624in}}%
\pgfpathlineto{\pgfqpoint{0.500000in}{4.018364in}}%
\pgfpathlineto{\pgfqpoint{0.500000in}{4.012104in}}%
\pgfpathlineto{\pgfqpoint{0.500000in}{4.005843in}}%
\pgfpathlineto{\pgfqpoint{0.500000in}{3.999583in}}%
\pgfpathlineto{\pgfqpoint{0.500000in}{3.993322in}}%
\pgfpathlineto{\pgfqpoint{0.500000in}{3.987062in}}%
\pgfpathlineto{\pgfqpoint{0.500000in}{3.980801in}}%
\pgfpathlineto{\pgfqpoint{0.500000in}{3.974541in}}%
\pgfpathlineto{\pgfqpoint{0.500000in}{3.968280in}}%
\pgfpathlineto{\pgfqpoint{0.500000in}{3.962020in}}%
\pgfpathlineto{\pgfqpoint{0.500000in}{3.955760in}}%
\pgfpathlineto{\pgfqpoint{0.500000in}{3.949499in}}%
\pgfpathlineto{\pgfqpoint{0.500000in}{3.943239in}}%
\pgfpathlineto{\pgfqpoint{0.500000in}{3.936978in}}%
\pgfpathlineto{\pgfqpoint{0.500000in}{3.930718in}}%
\pgfpathlineto{\pgfqpoint{0.500000in}{3.924457in}}%
\pgfpathlineto{\pgfqpoint{0.500000in}{3.918197in}}%
\pgfpathlineto{\pgfqpoint{0.500000in}{3.911937in}}%
\pgfpathlineto{\pgfqpoint{0.500000in}{3.905676in}}%
\pgfpathlineto{\pgfqpoint{0.500000in}{3.899416in}}%
\pgfpathlineto{\pgfqpoint{0.500000in}{3.893155in}}%
\pgfpathlineto{\pgfqpoint{0.500000in}{3.886895in}}%
\pgfpathlineto{\pgfqpoint{0.500000in}{3.880634in}}%
\pgfpathlineto{\pgfqpoint{0.500000in}{3.874374in}}%
\pgfpathlineto{\pgfqpoint{0.500000in}{3.868114in}}%
\pgfpathlineto{\pgfqpoint{0.500000in}{3.861853in}}%
\pgfpathlineto{\pgfqpoint{0.500000in}{3.855593in}}%
\pgfpathlineto{\pgfqpoint{0.500000in}{3.849332in}}%
\pgfpathlineto{\pgfqpoint{0.500000in}{3.843072in}}%
\pgfpathlineto{\pgfqpoint{0.500000in}{3.836811in}}%
\pgfpathlineto{\pgfqpoint{0.500000in}{3.830551in}}%
\pgfpathlineto{\pgfqpoint{0.500000in}{3.824290in}}%
\pgfpathlineto{\pgfqpoint{0.500000in}{3.818030in}}%
\pgfpathlineto{\pgfqpoint{0.500000in}{3.811770in}}%
\pgfpathlineto{\pgfqpoint{0.500000in}{3.805509in}}%
\pgfpathlineto{\pgfqpoint{0.500000in}{3.799249in}}%
\pgfpathlineto{\pgfqpoint{0.500000in}{3.792988in}}%
\pgfpathlineto{\pgfqpoint{0.500000in}{3.786728in}}%
\pgfpathlineto{\pgfqpoint{0.500000in}{3.780467in}}%
\pgfpathlineto{\pgfqpoint{0.500000in}{3.774207in}}%
\pgfpathlineto{\pgfqpoint{0.500000in}{3.767947in}}%
\pgfpathlineto{\pgfqpoint{0.500000in}{3.761686in}}%
\pgfpathlineto{\pgfqpoint{0.500000in}{3.755426in}}%
\pgfpathlineto{\pgfqpoint{0.500000in}{3.749165in}}%
\pgfpathlineto{\pgfqpoint{0.500000in}{3.742905in}}%
\pgfpathlineto{\pgfqpoint{0.500000in}{3.736644in}}%
\pgfpathlineto{\pgfqpoint{0.500000in}{3.730384in}}%
\pgfpathlineto{\pgfqpoint{0.500000in}{3.724124in}}%
\pgfpathlineto{\pgfqpoint{0.500000in}{3.717863in}}%
\pgfpathlineto{\pgfqpoint{0.500000in}{3.711603in}}%
\pgfpathlineto{\pgfqpoint{0.500000in}{3.705342in}}%
\pgfpathlineto{\pgfqpoint{0.500000in}{3.699082in}}%
\pgfpathlineto{\pgfqpoint{0.500000in}{3.692821in}}%
\pgfpathlineto{\pgfqpoint{0.500000in}{3.686561in}}%
\pgfpathlineto{\pgfqpoint{0.500000in}{3.680301in}}%
\pgfpathlineto{\pgfqpoint{0.500000in}{3.674040in}}%
\pgfpathlineto{\pgfqpoint{0.500000in}{3.667780in}}%
\pgfpathlineto{\pgfqpoint{0.500000in}{3.661519in}}%
\pgfpathlineto{\pgfqpoint{0.500000in}{3.655259in}}%
\pgfpathlineto{\pgfqpoint{0.500000in}{3.648998in}}%
\pgfpathlineto{\pgfqpoint{0.500000in}{3.642738in}}%
\pgfpathlineto{\pgfqpoint{0.500000in}{3.636477in}}%
\pgfpathlineto{\pgfqpoint{0.500000in}{3.630217in}}%
\pgfpathlineto{\pgfqpoint{0.500000in}{3.623957in}}%
\pgfpathlineto{\pgfqpoint{0.500000in}{3.617696in}}%
\pgfpathlineto{\pgfqpoint{0.500000in}{3.611436in}}%
\pgfpathlineto{\pgfqpoint{0.500000in}{3.605175in}}%
\pgfpathlineto{\pgfqpoint{0.500000in}{3.598915in}}%
\pgfpathlineto{\pgfqpoint{0.500000in}{3.592654in}}%
\pgfpathlineto{\pgfqpoint{0.500000in}{3.586394in}}%
\pgfpathlineto{\pgfqpoint{0.500000in}{3.580134in}}%
\pgfpathlineto{\pgfqpoint{0.500000in}{3.573873in}}%
\pgfpathlineto{\pgfqpoint{0.500000in}{3.567613in}}%
\pgfpathlineto{\pgfqpoint{0.500000in}{3.561352in}}%
\pgfpathlineto{\pgfqpoint{0.500000in}{3.555092in}}%
\pgfpathlineto{\pgfqpoint{0.500000in}{3.548831in}}%
\pgfpathlineto{\pgfqpoint{0.500000in}{3.542571in}}%
\pgfpathlineto{\pgfqpoint{0.500000in}{3.536311in}}%
\pgfpathlineto{\pgfqpoint{0.500000in}{3.530050in}}%
\pgfpathlineto{\pgfqpoint{0.500000in}{3.523790in}}%
\pgfpathlineto{\pgfqpoint{0.500000in}{3.517529in}}%
\pgfpathlineto{\pgfqpoint{0.500000in}{3.511269in}}%
\pgfpathlineto{\pgfqpoint{0.500000in}{3.505008in}}%
\pgfpathlineto{\pgfqpoint{0.500000in}{3.498748in}}%
\pgfpathlineto{\pgfqpoint{0.500000in}{3.492487in}}%
\pgfpathlineto{\pgfqpoint{0.500000in}{3.486227in}}%
\pgfpathlineto{\pgfqpoint{0.500000in}{3.479967in}}%
\pgfpathlineto{\pgfqpoint{0.500000in}{3.473706in}}%
\pgfpathlineto{\pgfqpoint{0.500000in}{3.467446in}}%
\pgfpathlineto{\pgfqpoint{0.500000in}{3.461185in}}%
\pgfpathlineto{\pgfqpoint{0.500000in}{3.454925in}}%
\pgfpathlineto{\pgfqpoint{0.500000in}{3.448664in}}%
\pgfpathlineto{\pgfqpoint{0.500000in}{3.442404in}}%
\pgfpathlineto{\pgfqpoint{0.500000in}{3.436144in}}%
\pgfpathlineto{\pgfqpoint{0.500000in}{3.429883in}}%
\pgfpathlineto{\pgfqpoint{0.500000in}{3.423623in}}%
\pgfpathlineto{\pgfqpoint{0.500000in}{3.417362in}}%
\pgfpathlineto{\pgfqpoint{0.500000in}{3.411102in}}%
\pgfpathlineto{\pgfqpoint{0.500000in}{3.404841in}}%
\pgfpathlineto{\pgfqpoint{0.500000in}{3.398581in}}%
\pgfpathlineto{\pgfqpoint{0.500000in}{3.392321in}}%
\pgfpathlineto{\pgfqpoint{0.500000in}{3.386060in}}%
\pgfpathlineto{\pgfqpoint{0.500000in}{3.379800in}}%
\pgfpathlineto{\pgfqpoint{0.500000in}{3.373539in}}%
\pgfpathlineto{\pgfqpoint{0.500000in}{3.367279in}}%
\pgfpathlineto{\pgfqpoint{0.500000in}{3.361018in}}%
\pgfpathlineto{\pgfqpoint{0.500000in}{3.354758in}}%
\pgfpathlineto{\pgfqpoint{0.500000in}{3.348497in}}%
\pgfpathlineto{\pgfqpoint{0.500000in}{3.342237in}}%
\pgfpathlineto{\pgfqpoint{0.500000in}{3.335977in}}%
\pgfpathlineto{\pgfqpoint{0.500000in}{3.329716in}}%
\pgfpathlineto{\pgfqpoint{0.500000in}{3.323456in}}%
\pgfpathlineto{\pgfqpoint{0.500000in}{3.317195in}}%
\pgfpathlineto{\pgfqpoint{0.500000in}{3.310935in}}%
\pgfpathlineto{\pgfqpoint{0.500000in}{3.304674in}}%
\pgfpathlineto{\pgfqpoint{0.500000in}{3.298414in}}%
\pgfpathlineto{\pgfqpoint{0.500000in}{3.292154in}}%
\pgfpathlineto{\pgfqpoint{0.500000in}{3.285893in}}%
\pgfpathlineto{\pgfqpoint{0.500000in}{3.279633in}}%
\pgfpathlineto{\pgfqpoint{0.500000in}{3.273372in}}%
\pgfpathlineto{\pgfqpoint{0.500000in}{3.267112in}}%
\pgfpathlineto{\pgfqpoint{0.500000in}{3.260851in}}%
\pgfpathlineto{\pgfqpoint{0.500000in}{3.254591in}}%
\pgfpathlineto{\pgfqpoint{0.500000in}{3.248331in}}%
\pgfpathlineto{\pgfqpoint{0.500000in}{3.242070in}}%
\pgfpathlineto{\pgfqpoint{0.500000in}{3.235810in}}%
\pgfpathlineto{\pgfqpoint{0.500000in}{3.229549in}}%
\pgfpathlineto{\pgfqpoint{0.500000in}{3.223289in}}%
\pgfpathlineto{\pgfqpoint{0.500000in}{3.217028in}}%
\pgfpathlineto{\pgfqpoint{0.500000in}{3.210768in}}%
\pgfpathlineto{\pgfqpoint{0.500000in}{3.204508in}}%
\pgfpathlineto{\pgfqpoint{0.500000in}{3.198247in}}%
\pgfpathlineto{\pgfqpoint{0.500000in}{3.191987in}}%
\pgfpathlineto{\pgfqpoint{0.500000in}{3.185726in}}%
\pgfpathlineto{\pgfqpoint{0.500000in}{3.179466in}}%
\pgfpathlineto{\pgfqpoint{0.500000in}{3.173205in}}%
\pgfpathlineto{\pgfqpoint{0.500000in}{3.166945in}}%
\pgfpathlineto{\pgfqpoint{0.500000in}{3.160684in}}%
\pgfpathlineto{\pgfqpoint{0.500000in}{3.154424in}}%
\pgfpathlineto{\pgfqpoint{0.500000in}{3.148164in}}%
\pgfpathlineto{\pgfqpoint{0.500000in}{3.141903in}}%
\pgfpathlineto{\pgfqpoint{0.500000in}{3.135643in}}%
\pgfpathlineto{\pgfqpoint{0.500000in}{3.129382in}}%
\pgfpathlineto{\pgfqpoint{0.500000in}{3.123122in}}%
\pgfpathlineto{\pgfqpoint{0.500000in}{3.116861in}}%
\pgfpathlineto{\pgfqpoint{0.500000in}{3.110601in}}%
\pgfpathlineto{\pgfqpoint{0.500000in}{3.104341in}}%
\pgfpathlineto{\pgfqpoint{0.500000in}{3.098080in}}%
\pgfpathlineto{\pgfqpoint{0.500000in}{3.091820in}}%
\pgfpathlineto{\pgfqpoint{0.500000in}{3.085559in}}%
\pgfpathlineto{\pgfqpoint{0.500000in}{3.079299in}}%
\pgfpathlineto{\pgfqpoint{0.500000in}{3.073038in}}%
\pgfpathlineto{\pgfqpoint{0.500000in}{3.066778in}}%
\pgfpathlineto{\pgfqpoint{0.500000in}{3.060518in}}%
\pgfpathlineto{\pgfqpoint{0.500000in}{3.054257in}}%
\pgfpathlineto{\pgfqpoint{0.500000in}{3.047997in}}%
\pgfpathlineto{\pgfqpoint{0.500000in}{3.041736in}}%
\pgfpathlineto{\pgfqpoint{0.500000in}{3.035476in}}%
\pgfpathlineto{\pgfqpoint{0.500000in}{3.029215in}}%
\pgfpathlineto{\pgfqpoint{0.500000in}{3.022955in}}%
\pgfpathlineto{\pgfqpoint{0.500000in}{3.016694in}}%
\pgfpathlineto{\pgfqpoint{0.500000in}{3.010434in}}%
\pgfpathlineto{\pgfqpoint{0.500000in}{3.004174in}}%
\pgfpathlineto{\pgfqpoint{0.500000in}{2.997913in}}%
\pgfpathlineto{\pgfqpoint{0.500000in}{2.991653in}}%
\pgfpathlineto{\pgfqpoint{0.500000in}{2.985392in}}%
\pgfpathlineto{\pgfqpoint{0.500000in}{2.979132in}}%
\pgfpathlineto{\pgfqpoint{0.500000in}{2.972871in}}%
\pgfpathlineto{\pgfqpoint{0.500000in}{2.966611in}}%
\pgfpathlineto{\pgfqpoint{0.500000in}{2.960351in}}%
\pgfpathlineto{\pgfqpoint{0.500000in}{2.954090in}}%
\pgfpathlineto{\pgfqpoint{0.500000in}{2.947830in}}%
\pgfpathlineto{\pgfqpoint{0.500000in}{2.941569in}}%
\pgfpathlineto{\pgfqpoint{0.500000in}{2.935309in}}%
\pgfpathlineto{\pgfqpoint{0.500000in}{2.929048in}}%
\pgfpathlineto{\pgfqpoint{0.500000in}{2.922788in}}%
\pgfpathlineto{\pgfqpoint{0.500000in}{2.916528in}}%
\pgfpathlineto{\pgfqpoint{0.500000in}{2.910267in}}%
\pgfpathlineto{\pgfqpoint{0.500000in}{2.904007in}}%
\pgfpathlineto{\pgfqpoint{0.500000in}{2.897746in}}%
\pgfpathlineto{\pgfqpoint{0.500000in}{2.891486in}}%
\pgfpathlineto{\pgfqpoint{0.500000in}{2.885225in}}%
\pgfpathlineto{\pgfqpoint{0.500000in}{2.878965in}}%
\pgfpathlineto{\pgfqpoint{0.500000in}{2.872705in}}%
\pgfpathlineto{\pgfqpoint{0.500000in}{2.866444in}}%
\pgfpathlineto{\pgfqpoint{0.500000in}{2.860184in}}%
\pgfpathlineto{\pgfqpoint{0.500000in}{2.853923in}}%
\pgfpathlineto{\pgfqpoint{0.500000in}{2.847663in}}%
\pgfpathlineto{\pgfqpoint{0.500000in}{2.841402in}}%
\pgfpathlineto{\pgfqpoint{0.500000in}{2.835142in}}%
\pgfpathlineto{\pgfqpoint{0.500000in}{2.828881in}}%
\pgfpathlineto{\pgfqpoint{0.500000in}{2.822621in}}%
\pgfpathlineto{\pgfqpoint{0.500000in}{2.816361in}}%
\pgfpathlineto{\pgfqpoint{0.500000in}{2.810100in}}%
\pgfpathlineto{\pgfqpoint{0.500000in}{2.803840in}}%
\pgfpathlineto{\pgfqpoint{0.500000in}{2.797579in}}%
\pgfpathlineto{\pgfqpoint{0.500000in}{2.791319in}}%
\pgfpathlineto{\pgfqpoint{0.500000in}{2.785058in}}%
\pgfpathlineto{\pgfqpoint{0.500000in}{2.778798in}}%
\pgfpathlineto{\pgfqpoint{0.500000in}{2.772538in}}%
\pgfpathlineto{\pgfqpoint{0.500000in}{2.766277in}}%
\pgfpathlineto{\pgfqpoint{0.500000in}{2.760017in}}%
\pgfpathlineto{\pgfqpoint{0.500000in}{2.753756in}}%
\pgfpathlineto{\pgfqpoint{0.500000in}{2.747496in}}%
\pgfpathlineto{\pgfqpoint{0.500000in}{2.741235in}}%
\pgfpathlineto{\pgfqpoint{0.500000in}{2.734975in}}%
\pgfpathlineto{\pgfqpoint{0.500000in}{2.728715in}}%
\pgfpathlineto{\pgfqpoint{0.500000in}{2.722454in}}%
\pgfpathlineto{\pgfqpoint{0.500000in}{2.716194in}}%
\pgfpathlineto{\pgfqpoint{0.500000in}{2.709933in}}%
\pgfpathlineto{\pgfqpoint{0.500000in}{2.703673in}}%
\pgfpathlineto{\pgfqpoint{0.500000in}{2.697412in}}%
\pgfpathlineto{\pgfqpoint{0.500000in}{2.691152in}}%
\pgfpathlineto{\pgfqpoint{0.500000in}{2.684891in}}%
\pgfpathlineto{\pgfqpoint{0.500000in}{2.678631in}}%
\pgfpathlineto{\pgfqpoint{0.500000in}{2.672371in}}%
\pgfpathlineto{\pgfqpoint{0.500000in}{2.666110in}}%
\pgfpathlineto{\pgfqpoint{0.500000in}{2.659850in}}%
\pgfpathlineto{\pgfqpoint{0.500000in}{2.653589in}}%
\pgfpathlineto{\pgfqpoint{0.500000in}{2.647329in}}%
\pgfpathlineto{\pgfqpoint{0.500000in}{2.641068in}}%
\pgfpathlineto{\pgfqpoint{0.500000in}{2.634808in}}%
\pgfpathlineto{\pgfqpoint{0.500000in}{2.628548in}}%
\pgfpathlineto{\pgfqpoint{0.500000in}{2.622287in}}%
\pgfpathlineto{\pgfqpoint{0.500000in}{2.616027in}}%
\pgfpathlineto{\pgfqpoint{0.500000in}{2.609766in}}%
\pgfpathlineto{\pgfqpoint{0.500000in}{2.603506in}}%
\pgfpathlineto{\pgfqpoint{0.500000in}{2.597245in}}%
\pgfpathlineto{\pgfqpoint{0.500000in}{2.590985in}}%
\pgfpathlineto{\pgfqpoint{0.500000in}{2.584725in}}%
\pgfpathlineto{\pgfqpoint{0.500000in}{2.578464in}}%
\pgfpathlineto{\pgfqpoint{0.500000in}{2.572204in}}%
\pgfpathlineto{\pgfqpoint{0.500000in}{2.565943in}}%
\pgfpathlineto{\pgfqpoint{0.500000in}{2.559683in}}%
\pgfpathlineto{\pgfqpoint{0.500000in}{2.553422in}}%
\pgfpathlineto{\pgfqpoint{0.500000in}{2.547162in}}%
\pgfpathlineto{\pgfqpoint{0.500000in}{2.540902in}}%
\pgfpathlineto{\pgfqpoint{0.500000in}{2.534641in}}%
\pgfpathlineto{\pgfqpoint{0.500000in}{2.528381in}}%
\pgfpathlineto{\pgfqpoint{0.500000in}{2.522120in}}%
\pgfpathlineto{\pgfqpoint{0.500000in}{2.515860in}}%
\pgfpathlineto{\pgfqpoint{0.500000in}{2.509599in}}%
\pgfpathlineto{\pgfqpoint{0.500000in}{2.503339in}}%
\pgfpathlineto{\pgfqpoint{0.500000in}{2.497078in}}%
\pgfpathlineto{\pgfqpoint{0.500000in}{2.490818in}}%
\pgfpathlineto{\pgfqpoint{0.500000in}{2.484558in}}%
\pgfpathlineto{\pgfqpoint{0.500000in}{2.478297in}}%
\pgfpathlineto{\pgfqpoint{0.500000in}{2.472037in}}%
\pgfpathlineto{\pgfqpoint{0.500000in}{2.465776in}}%
\pgfpathlineto{\pgfqpoint{0.500000in}{2.459516in}}%
\pgfpathlineto{\pgfqpoint{0.500000in}{2.453255in}}%
\pgfpathlineto{\pgfqpoint{0.500000in}{2.446995in}}%
\pgfpathlineto{\pgfqpoint{0.500000in}{2.440735in}}%
\pgfpathlineto{\pgfqpoint{0.500000in}{2.434474in}}%
\pgfpathlineto{\pgfqpoint{0.500000in}{2.428214in}}%
\pgfpathlineto{\pgfqpoint{0.500000in}{2.421953in}}%
\pgfpathlineto{\pgfqpoint{0.500000in}{2.415693in}}%
\pgfpathlineto{\pgfqpoint{0.500000in}{2.409432in}}%
\pgfpathlineto{\pgfqpoint{0.500000in}{2.403172in}}%
\pgfpathlineto{\pgfqpoint{0.500000in}{2.396912in}}%
\pgfpathlineto{\pgfqpoint{0.500000in}{2.390651in}}%
\pgfpathlineto{\pgfqpoint{0.500000in}{2.384391in}}%
\pgfpathlineto{\pgfqpoint{0.500000in}{2.378130in}}%
\pgfpathlineto{\pgfqpoint{0.500000in}{2.371870in}}%
\pgfpathlineto{\pgfqpoint{0.500000in}{2.365609in}}%
\pgfpathlineto{\pgfqpoint{0.500000in}{2.359349in}}%
\pgfpathlineto{\pgfqpoint{0.500000in}{2.353088in}}%
\pgfpathlineto{\pgfqpoint{0.500000in}{2.346828in}}%
\pgfpathlineto{\pgfqpoint{0.500000in}{2.340568in}}%
\pgfpathlineto{\pgfqpoint{0.500000in}{2.334307in}}%
\pgfpathlineto{\pgfqpoint{0.500000in}{2.328047in}}%
\pgfpathlineto{\pgfqpoint{0.500000in}{2.321786in}}%
\pgfpathlineto{\pgfqpoint{0.500000in}{2.315526in}}%
\pgfpathlineto{\pgfqpoint{0.500000in}{2.309265in}}%
\pgfpathlineto{\pgfqpoint{0.500000in}{2.303005in}}%
\pgfpathlineto{\pgfqpoint{0.500000in}{2.296745in}}%
\pgfpathlineto{\pgfqpoint{0.500000in}{2.290484in}}%
\pgfpathlineto{\pgfqpoint{0.500000in}{2.284224in}}%
\pgfpathlineto{\pgfqpoint{0.500000in}{2.277963in}}%
\pgfpathlineto{\pgfqpoint{0.500000in}{2.271703in}}%
\pgfpathlineto{\pgfqpoint{0.500000in}{2.265442in}}%
\pgfpathlineto{\pgfqpoint{0.500000in}{2.259182in}}%
\pgfpathlineto{\pgfqpoint{0.500000in}{2.252922in}}%
\pgfpathlineto{\pgfqpoint{0.500000in}{2.246661in}}%
\pgfpathlineto{\pgfqpoint{0.500000in}{2.240401in}}%
\pgfpathlineto{\pgfqpoint{0.500000in}{2.234140in}}%
\pgfpathlineto{\pgfqpoint{0.500000in}{2.227880in}}%
\pgfpathlineto{\pgfqpoint{0.500000in}{2.221619in}}%
\pgfpathlineto{\pgfqpoint{0.500000in}{2.215359in}}%
\pgfpathlineto{\pgfqpoint{0.500000in}{2.209098in}}%
\pgfpathlineto{\pgfqpoint{0.500000in}{2.202838in}}%
\pgfpathlineto{\pgfqpoint{0.500000in}{2.196578in}}%
\pgfpathlineto{\pgfqpoint{0.500000in}{2.190317in}}%
\pgfpathlineto{\pgfqpoint{0.500000in}{2.184057in}}%
\pgfpathlineto{\pgfqpoint{0.500000in}{2.177796in}}%
\pgfpathlineto{\pgfqpoint{0.500000in}{2.171536in}}%
\pgfpathlineto{\pgfqpoint{0.500000in}{2.165275in}}%
\pgfpathlineto{\pgfqpoint{0.500000in}{2.159015in}}%
\pgfpathlineto{\pgfqpoint{0.500000in}{2.152755in}}%
\pgfpathlineto{\pgfqpoint{0.500000in}{2.146494in}}%
\pgfpathlineto{\pgfqpoint{0.500000in}{2.140234in}}%
\pgfpathlineto{\pgfqpoint{0.500000in}{2.133973in}}%
\pgfpathlineto{\pgfqpoint{0.500000in}{2.127713in}}%
\pgfpathlineto{\pgfqpoint{0.500000in}{2.121452in}}%
\pgfpathlineto{\pgfqpoint{0.500000in}{2.115192in}}%
\pgfpathlineto{\pgfqpoint{0.500000in}{2.108932in}}%
\pgfpathlineto{\pgfqpoint{0.500000in}{2.102671in}}%
\pgfpathlineto{\pgfqpoint{0.500000in}{2.096411in}}%
\pgfpathlineto{\pgfqpoint{0.500000in}{2.090150in}}%
\pgfpathlineto{\pgfqpoint{0.500000in}{2.083890in}}%
\pgfpathlineto{\pgfqpoint{0.500000in}{2.077629in}}%
\pgfpathlineto{\pgfqpoint{0.500000in}{2.071369in}}%
\pgfpathlineto{\pgfqpoint{0.500000in}{2.065109in}}%
\pgfpathlineto{\pgfqpoint{0.500000in}{2.058848in}}%
\pgfpathlineto{\pgfqpoint{0.500000in}{2.052588in}}%
\pgfpathlineto{\pgfqpoint{0.500000in}{2.046327in}}%
\pgfpathlineto{\pgfqpoint{0.500000in}{2.040067in}}%
\pgfpathlineto{\pgfqpoint{0.500000in}{2.033806in}}%
\pgfpathlineto{\pgfqpoint{0.500000in}{2.027546in}}%
\pgfpathlineto{\pgfqpoint{0.500000in}{2.021285in}}%
\pgfpathlineto{\pgfqpoint{0.500000in}{2.015025in}}%
\pgfpathlineto{\pgfqpoint{0.500000in}{2.008765in}}%
\pgfpathlineto{\pgfqpoint{0.500000in}{2.002504in}}%
\pgfpathlineto{\pgfqpoint{0.500000in}{1.996244in}}%
\pgfpathlineto{\pgfqpoint{0.500000in}{1.989983in}}%
\pgfpathlineto{\pgfqpoint{0.500000in}{1.983723in}}%
\pgfpathlineto{\pgfqpoint{0.500000in}{1.977462in}}%
\pgfpathlineto{\pgfqpoint{0.500000in}{1.971202in}}%
\pgfpathlineto{\pgfqpoint{0.500000in}{1.964942in}}%
\pgfpathlineto{\pgfqpoint{0.500000in}{1.958681in}}%
\pgfpathlineto{\pgfqpoint{0.500000in}{1.952421in}}%
\pgfpathlineto{\pgfqpoint{0.500000in}{1.946160in}}%
\pgfpathlineto{\pgfqpoint{0.500000in}{1.939900in}}%
\pgfpathlineto{\pgfqpoint{0.500000in}{1.933639in}}%
\pgfpathlineto{\pgfqpoint{0.500000in}{1.927379in}}%
\pgfpathlineto{\pgfqpoint{0.500000in}{1.921119in}}%
\pgfpathlineto{\pgfqpoint{0.500000in}{1.914858in}}%
\pgfpathlineto{\pgfqpoint{0.500000in}{1.908598in}}%
\pgfpathlineto{\pgfqpoint{0.500000in}{1.902337in}}%
\pgfpathlineto{\pgfqpoint{0.500000in}{1.896077in}}%
\pgfpathlineto{\pgfqpoint{0.500000in}{1.889816in}}%
\pgfpathlineto{\pgfqpoint{0.500000in}{1.883556in}}%
\pgfpathlineto{\pgfqpoint{0.500000in}{1.877295in}}%
\pgfpathlineto{\pgfqpoint{0.500000in}{1.871035in}}%
\pgfpathlineto{\pgfqpoint{0.500000in}{1.864775in}}%
\pgfpathlineto{\pgfqpoint{0.500000in}{1.858514in}}%
\pgfpathlineto{\pgfqpoint{0.500000in}{1.852254in}}%
\pgfpathlineto{\pgfqpoint{0.500000in}{1.845993in}}%
\pgfpathlineto{\pgfqpoint{0.500000in}{1.839733in}}%
\pgfpathlineto{\pgfqpoint{0.500000in}{1.833472in}}%
\pgfpathlineto{\pgfqpoint{0.500000in}{1.827212in}}%
\pgfpathlineto{\pgfqpoint{0.500000in}{1.820952in}}%
\pgfpathlineto{\pgfqpoint{0.500000in}{1.814691in}}%
\pgfpathlineto{\pgfqpoint{0.500000in}{1.808431in}}%
\pgfpathlineto{\pgfqpoint{0.500000in}{1.802170in}}%
\pgfpathlineto{\pgfqpoint{0.500000in}{1.795910in}}%
\pgfpathlineto{\pgfqpoint{0.500000in}{1.789649in}}%
\pgfpathlineto{\pgfqpoint{0.500000in}{1.783389in}}%
\pgfpathlineto{\pgfqpoint{0.500000in}{1.777129in}}%
\pgfpathlineto{\pgfqpoint{0.500000in}{1.770868in}}%
\pgfpathlineto{\pgfqpoint{0.500000in}{1.764608in}}%
\pgfpathlineto{\pgfqpoint{0.500000in}{1.758347in}}%
\pgfpathlineto{\pgfqpoint{0.500000in}{1.752087in}}%
\pgfpathlineto{\pgfqpoint{0.500000in}{1.745826in}}%
\pgfpathlineto{\pgfqpoint{0.500000in}{1.739566in}}%
\pgfpathlineto{\pgfqpoint{0.500000in}{1.733306in}}%
\pgfpathlineto{\pgfqpoint{0.500000in}{1.727045in}}%
\pgfpathlineto{\pgfqpoint{0.500000in}{1.720785in}}%
\pgfpathlineto{\pgfqpoint{0.500000in}{1.714524in}}%
\pgfpathlineto{\pgfqpoint{0.500000in}{1.708264in}}%
\pgfpathlineto{\pgfqpoint{0.500000in}{1.702003in}}%
\pgfpathlineto{\pgfqpoint{0.500000in}{1.695743in}}%
\pgfpathlineto{\pgfqpoint{0.500000in}{1.689482in}}%
\pgfpathlineto{\pgfqpoint{0.500000in}{1.683222in}}%
\pgfpathlineto{\pgfqpoint{0.500000in}{1.676962in}}%
\pgfpathlineto{\pgfqpoint{0.500000in}{1.670701in}}%
\pgfpathlineto{\pgfqpoint{0.500000in}{1.664441in}}%
\pgfpathlineto{\pgfqpoint{0.500000in}{1.658180in}}%
\pgfpathlineto{\pgfqpoint{0.500000in}{1.651920in}}%
\pgfpathlineto{\pgfqpoint{0.500000in}{1.645659in}}%
\pgfpathlineto{\pgfqpoint{0.500000in}{1.639399in}}%
\pgfpathlineto{\pgfqpoint{0.500000in}{1.633139in}}%
\pgfpathlineto{\pgfqpoint{0.500000in}{1.626878in}}%
\pgfpathlineto{\pgfqpoint{0.500000in}{1.620618in}}%
\pgfpathlineto{\pgfqpoint{0.500000in}{1.614357in}}%
\pgfpathlineto{\pgfqpoint{0.500000in}{1.608097in}}%
\pgfpathlineto{\pgfqpoint{0.500000in}{1.601836in}}%
\pgfpathlineto{\pgfqpoint{0.500000in}{1.595576in}}%
\pgfpathlineto{\pgfqpoint{0.500000in}{1.589316in}}%
\pgfpathlineto{\pgfqpoint{0.500000in}{1.583055in}}%
\pgfpathlineto{\pgfqpoint{0.500000in}{1.576795in}}%
\pgfpathlineto{\pgfqpoint{0.500000in}{1.570534in}}%
\pgfpathlineto{\pgfqpoint{0.500000in}{1.564274in}}%
\pgfpathlineto{\pgfqpoint{0.500000in}{1.558013in}}%
\pgfpathlineto{\pgfqpoint{0.500000in}{1.551753in}}%
\pgfpathlineto{\pgfqpoint{0.500000in}{1.545492in}}%
\pgfpathlineto{\pgfqpoint{0.500000in}{1.539232in}}%
\pgfpathlineto{\pgfqpoint{0.500000in}{1.532972in}}%
\pgfpathlineto{\pgfqpoint{0.500000in}{1.526711in}}%
\pgfpathlineto{\pgfqpoint{0.500000in}{1.520451in}}%
\pgfpathlineto{\pgfqpoint{0.500000in}{1.514190in}}%
\pgfpathlineto{\pgfqpoint{0.500000in}{1.507930in}}%
\pgfpathlineto{\pgfqpoint{0.500000in}{1.501669in}}%
\pgfpathlineto{\pgfqpoint{0.500000in}{1.495409in}}%
\pgfpathlineto{\pgfqpoint{0.500000in}{1.489149in}}%
\pgfpathlineto{\pgfqpoint{0.500000in}{1.482888in}}%
\pgfpathlineto{\pgfqpoint{0.500000in}{1.476628in}}%
\pgfpathlineto{\pgfqpoint{0.500000in}{1.470367in}}%
\pgfpathlineto{\pgfqpoint{0.500000in}{1.464107in}}%
\pgfpathlineto{\pgfqpoint{0.500000in}{1.457846in}}%
\pgfpathlineto{\pgfqpoint{0.500000in}{1.451586in}}%
\pgfpathlineto{\pgfqpoint{0.500000in}{1.445326in}}%
\pgfpathlineto{\pgfqpoint{0.500000in}{1.439065in}}%
\pgfpathlineto{\pgfqpoint{0.500000in}{1.432805in}}%
\pgfpathlineto{\pgfqpoint{0.500000in}{1.426544in}}%
\pgfpathlineto{\pgfqpoint{0.500000in}{1.420284in}}%
\pgfpathlineto{\pgfqpoint{0.500000in}{1.414023in}}%
\pgfpathlineto{\pgfqpoint{0.500000in}{1.407763in}}%
\pgfpathlineto{\pgfqpoint{0.500000in}{1.401503in}}%
\pgfpathlineto{\pgfqpoint{0.500000in}{1.395242in}}%
\pgfpathlineto{\pgfqpoint{0.500000in}{1.388982in}}%
\pgfpathlineto{\pgfqpoint{0.500000in}{1.382721in}}%
\pgfpathlineto{\pgfqpoint{0.500000in}{1.376461in}}%
\pgfpathlineto{\pgfqpoint{0.500000in}{1.370200in}}%
\pgfpathlineto{\pgfqpoint{0.500000in}{1.363940in}}%
\pgfpathlineto{\pgfqpoint{0.500000in}{1.357679in}}%
\pgfpathlineto{\pgfqpoint{0.500000in}{1.351419in}}%
\pgfpathlineto{\pgfqpoint{0.500000in}{1.345159in}}%
\pgfpathlineto{\pgfqpoint{0.500000in}{1.338898in}}%
\pgfpathlineto{\pgfqpoint{0.500000in}{1.332638in}}%
\pgfpathlineto{\pgfqpoint{0.500000in}{1.326377in}}%
\pgfpathlineto{\pgfqpoint{0.500000in}{1.320117in}}%
\pgfpathlineto{\pgfqpoint{0.500000in}{1.313856in}}%
\pgfpathlineto{\pgfqpoint{0.500000in}{1.307596in}}%
\pgfpathlineto{\pgfqpoint{0.500000in}{1.301336in}}%
\pgfpathlineto{\pgfqpoint{0.500000in}{1.295075in}}%
\pgfpathlineto{\pgfqpoint{0.500000in}{1.288815in}}%
\pgfpathlineto{\pgfqpoint{0.500000in}{1.282554in}}%
\pgfpathlineto{\pgfqpoint{0.500000in}{1.276294in}}%
\pgfpathlineto{\pgfqpoint{0.500000in}{1.270033in}}%
\pgfpathlineto{\pgfqpoint{0.500000in}{1.263773in}}%
\pgfpathlineto{\pgfqpoint{0.500000in}{1.257513in}}%
\pgfpathlineto{\pgfqpoint{0.500000in}{1.251252in}}%
\pgfpathlineto{\pgfqpoint{0.500000in}{1.244992in}}%
\pgfpathlineto{\pgfqpoint{0.500000in}{1.238731in}}%
\pgfpathlineto{\pgfqpoint{0.500000in}{1.232471in}}%
\pgfpathlineto{\pgfqpoint{0.500000in}{1.226210in}}%
\pgfpathlineto{\pgfqpoint{0.500000in}{1.219950in}}%
\pgfpathlineto{\pgfqpoint{0.500000in}{1.213689in}}%
\pgfpathlineto{\pgfqpoint{0.500000in}{1.207429in}}%
\pgfpathlineto{\pgfqpoint{0.500000in}{1.201169in}}%
\pgfpathlineto{\pgfqpoint{0.500000in}{1.194908in}}%
\pgfpathlineto{\pgfqpoint{0.500000in}{1.188648in}}%
\pgfpathlineto{\pgfqpoint{0.500000in}{1.182387in}}%
\pgfpathlineto{\pgfqpoint{0.500000in}{1.176127in}}%
\pgfpathlineto{\pgfqpoint{0.500000in}{1.169866in}}%
\pgfpathlineto{\pgfqpoint{0.500000in}{1.163606in}}%
\pgfpathlineto{\pgfqpoint{0.500000in}{1.157346in}}%
\pgfpathlineto{\pgfqpoint{0.500000in}{1.151085in}}%
\pgfpathlineto{\pgfqpoint{0.500000in}{1.144825in}}%
\pgfpathlineto{\pgfqpoint{0.500000in}{1.138564in}}%
\pgfpathlineto{\pgfqpoint{0.500000in}{1.132304in}}%
\pgfpathlineto{\pgfqpoint{0.500000in}{1.126043in}}%
\pgfpathlineto{\pgfqpoint{0.500000in}{1.119783in}}%
\pgfpathlineto{\pgfqpoint{0.500000in}{1.113523in}}%
\pgfpathlineto{\pgfqpoint{0.500000in}{1.107262in}}%
\pgfpathlineto{\pgfqpoint{0.500000in}{1.101002in}}%
\pgfpathlineto{\pgfqpoint{0.500000in}{1.094741in}}%
\pgfpathlineto{\pgfqpoint{0.500000in}{1.088481in}}%
\pgfpathlineto{\pgfqpoint{0.500000in}{1.082220in}}%
\pgfpathlineto{\pgfqpoint{0.500000in}{1.075960in}}%
\pgfpathlineto{\pgfqpoint{0.500000in}{1.069699in}}%
\pgfpathlineto{\pgfqpoint{0.500000in}{1.063439in}}%
\pgfpathlineto{\pgfqpoint{0.500000in}{1.057179in}}%
\pgfpathlineto{\pgfqpoint{0.500000in}{1.050918in}}%
\pgfpathlineto{\pgfqpoint{0.500000in}{1.044658in}}%
\pgfpathlineto{\pgfqpoint{0.500000in}{1.038397in}}%
\pgfpathlineto{\pgfqpoint{0.500000in}{1.032137in}}%
\pgfpathlineto{\pgfqpoint{0.500000in}{1.025876in}}%
\pgfpathlineto{\pgfqpoint{0.500000in}{1.019616in}}%
\pgfpathlineto{\pgfqpoint{0.500000in}{1.013356in}}%
\pgfpathlineto{\pgfqpoint{0.500000in}{1.007095in}}%
\pgfpathlineto{\pgfqpoint{0.500000in}{1.000835in}}%
\pgfpathlineto{\pgfqpoint{0.500000in}{0.994574in}}%
\pgfpathlineto{\pgfqpoint{0.500000in}{0.988314in}}%
\pgfpathlineto{\pgfqpoint{0.500000in}{0.982053in}}%
\pgfpathlineto{\pgfqpoint{0.500000in}{0.975793in}}%
\pgfpathlineto{\pgfqpoint{0.500000in}{0.969533in}}%
\pgfpathlineto{\pgfqpoint{0.500000in}{0.963272in}}%
\pgfpathlineto{\pgfqpoint{0.500000in}{0.957012in}}%
\pgfpathlineto{\pgfqpoint{0.500000in}{0.950751in}}%
\pgfpathlineto{\pgfqpoint{0.500000in}{0.944491in}}%
\pgfpathlineto{\pgfqpoint{0.500000in}{0.938230in}}%
\pgfpathlineto{\pgfqpoint{0.500000in}{0.931970in}}%
\pgfpathlineto{\pgfqpoint{0.500000in}{0.925710in}}%
\pgfpathlineto{\pgfqpoint{0.500000in}{0.919449in}}%
\pgfpathlineto{\pgfqpoint{0.500000in}{0.913189in}}%
\pgfpathlineto{\pgfqpoint{0.500000in}{0.906928in}}%
\pgfpathlineto{\pgfqpoint{0.500000in}{0.900668in}}%
\pgfpathlineto{\pgfqpoint{0.500000in}{0.894407in}}%
\pgfpathlineto{\pgfqpoint{0.500000in}{0.888147in}}%
\pgfpathlineto{\pgfqpoint{0.500000in}{0.881886in}}%
\pgfpathlineto{\pgfqpoint{0.500000in}{0.875626in}}%
\pgfpathlineto{\pgfqpoint{0.500000in}{0.869366in}}%
\pgfpathlineto{\pgfqpoint{0.500000in}{0.863105in}}%
\pgfpathlineto{\pgfqpoint{0.500000in}{0.856845in}}%
\pgfpathlineto{\pgfqpoint{0.500000in}{0.850584in}}%
\pgfpathlineto{\pgfqpoint{0.500000in}{0.844324in}}%
\pgfpathlineto{\pgfqpoint{0.500000in}{0.838063in}}%
\pgfpathlineto{\pgfqpoint{0.500000in}{0.831803in}}%
\pgfpathlineto{\pgfqpoint{0.500000in}{0.825543in}}%
\pgfpathlineto{\pgfqpoint{0.500000in}{0.819282in}}%
\pgfpathlineto{\pgfqpoint{0.500000in}{0.813022in}}%
\pgfpathlineto{\pgfqpoint{0.500000in}{0.806761in}}%
\pgfpathlineto{\pgfqpoint{0.500000in}{0.800501in}}%
\pgfpathlineto{\pgfqpoint{0.500000in}{0.794240in}}%
\pgfpathlineto{\pgfqpoint{0.500000in}{0.787980in}}%
\pgfpathlineto{\pgfqpoint{0.500000in}{0.781720in}}%
\pgfpathlineto{\pgfqpoint{0.500000in}{0.775459in}}%
\pgfpathlineto{\pgfqpoint{0.500000in}{0.769199in}}%
\pgfpathlineto{\pgfqpoint{0.500000in}{0.762938in}}%
\pgfpathlineto{\pgfqpoint{0.500000in}{0.756678in}}%
\pgfpathlineto{\pgfqpoint{0.500000in}{0.750417in}}%
\pgfpathlineto{\pgfqpoint{0.500000in}{0.744157in}}%
\pgfpathlineto{\pgfqpoint{0.500000in}{0.737896in}}%
\pgfpathlineto{\pgfqpoint{0.500000in}{0.731636in}}%
\pgfpathlineto{\pgfqpoint{0.500000in}{0.725376in}}%
\pgfpathlineto{\pgfqpoint{0.500000in}{0.719115in}}%
\pgfpathlineto{\pgfqpoint{0.500000in}{0.712855in}}%
\pgfpathlineto{\pgfqpoint{0.500000in}{0.706594in}}%
\pgfpathlineto{\pgfqpoint{0.500000in}{0.700334in}}%
\pgfpathlineto{\pgfqpoint{0.500000in}{0.694073in}}%
\pgfpathlineto{\pgfqpoint{0.500000in}{0.687813in}}%
\pgfpathlineto{\pgfqpoint{0.500000in}{0.681553in}}%
\pgfpathlineto{\pgfqpoint{0.500000in}{0.675292in}}%
\pgfpathlineto{\pgfqpoint{0.500000in}{0.669032in}}%
\pgfpathlineto{\pgfqpoint{0.500000in}{0.662771in}}%
\pgfpathlineto{\pgfqpoint{0.500000in}{0.656511in}}%
\pgfpathlineto{\pgfqpoint{0.500000in}{0.650250in}}%
\pgfpathlineto{\pgfqpoint{0.500000in}{0.643990in}}%
\pgfpathlineto{\pgfqpoint{0.500000in}{0.637730in}}%
\pgfpathlineto{\pgfqpoint{0.500000in}{0.631469in}}%
\pgfpathlineto{\pgfqpoint{0.500000in}{0.625209in}}%
\pgfpathlineto{\pgfqpoint{0.500000in}{0.618948in}}%
\pgfpathlineto{\pgfqpoint{0.500000in}{0.612688in}}%
\pgfpathlineto{\pgfqpoint{0.500000in}{0.606427in}}%
\pgfpathlineto{\pgfqpoint{0.500000in}{0.600167in}}%
\pgfpathlineto{\pgfqpoint{0.500000in}{0.593907in}}%
\pgfpathlineto{\pgfqpoint{0.500000in}{0.587646in}}%
\pgfpathlineto{\pgfqpoint{0.500000in}{0.581386in}}%
\pgfpathlineto{\pgfqpoint{0.500000in}{0.575125in}}%
\pgfpathlineto{\pgfqpoint{0.500000in}{0.568865in}}%
\pgfpathlineto{\pgfqpoint{0.500000in}{0.562604in}}%
\pgfpathlineto{\pgfqpoint{0.500000in}{0.556344in}}%
\pgfpathlineto{\pgfqpoint{0.500000in}{0.550083in}}%
\pgfpathlineto{\pgfqpoint{0.500000in}{0.543823in}}%
\pgfpathlineto{\pgfqpoint{0.500000in}{0.537563in}}%
\pgfpathlineto{\pgfqpoint{0.500000in}{0.531302in}}%
\pgfpathlineto{\pgfqpoint{0.500000in}{0.525042in}}%
\pgfpathlineto{\pgfqpoint{0.500000in}{0.518781in}}%
\pgfpathlineto{\pgfqpoint{0.500000in}{0.512521in}}%
\pgfpathlineto{\pgfqpoint{0.500000in}{0.506260in}}%
\pgfpathlineto{\pgfqpoint{0.500000in}{0.500000in}}%
\pgfpathclose%
\pgfpathmoveto{\pgfqpoint{2.747042in}{1.583055in}}%
\pgfpathlineto{\pgfqpoint{2.741235in}{1.584410in}}%
\pgfpathlineto{\pgfqpoint{2.734975in}{1.586087in}}%
\pgfpathlineto{\pgfqpoint{2.728715in}{1.587996in}}%
\pgfpathlineto{\pgfqpoint{2.724841in}{1.589316in}}%
\pgfpathlineto{\pgfqpoint{2.722454in}{1.590128in}}%
\pgfpathlineto{\pgfqpoint{2.716194in}{1.592482in}}%
\pgfpathlineto{\pgfqpoint{2.709933in}{1.595109in}}%
\pgfpathlineto{\pgfqpoint{2.708906in}{1.595576in}}%
\pgfpathlineto{\pgfqpoint{2.703673in}{1.597972in}}%
\pgfpathlineto{\pgfqpoint{2.697412in}{1.601133in}}%
\pgfpathlineto{\pgfqpoint{2.696118in}{1.601836in}}%
\pgfpathlineto{\pgfqpoint{2.691152in}{1.604568in}}%
\pgfpathlineto{\pgfqpoint{2.685281in}{1.608097in}}%
\pgfpathlineto{\pgfqpoint{2.684891in}{1.608335in}}%
\pgfpathlineto{\pgfqpoint{2.678631in}{1.612411in}}%
\pgfpathlineto{\pgfqpoint{2.675852in}{1.614357in}}%
\pgfpathlineto{\pgfqpoint{2.672371in}{1.616850in}}%
\pgfpathlineto{\pgfqpoint{2.667455in}{1.620618in}}%
\pgfpathlineto{\pgfqpoint{2.666110in}{1.621675in}}%
\pgfpathlineto{\pgfqpoint{2.659886in}{1.626878in}}%
\pgfpathlineto{\pgfqpoint{2.659850in}{1.626910in}}%
\pgfpathlineto{\pgfqpoint{2.653589in}{1.632581in}}%
\pgfpathlineto{\pgfqpoint{2.653004in}{1.633139in}}%
\pgfpathlineto{\pgfqpoint{2.647329in}{1.638738in}}%
\pgfpathlineto{\pgfqpoint{2.646688in}{1.639399in}}%
\pgfpathlineto{\pgfqpoint{2.641068in}{1.645425in}}%
\pgfpathlineto{\pgfqpoint{2.640858in}{1.645659in}}%
\pgfpathlineto{\pgfqpoint{2.635447in}{1.651920in}}%
\pgfpathlineto{\pgfqpoint{2.634808in}{1.652690in}}%
\pgfpathlineto{\pgfqpoint{2.630400in}{1.658180in}}%
\pgfpathlineto{\pgfqpoint{2.628548in}{1.660597in}}%
\pgfpathlineto{\pgfqpoint{2.625680in}{1.664441in}}%
\pgfpathlineto{\pgfqpoint{2.622287in}{1.669218in}}%
\pgfpathlineto{\pgfqpoint{2.621257in}{1.670701in}}%
\pgfpathlineto{\pgfqpoint{2.617090in}{1.676962in}}%
\pgfpathlineto{\pgfqpoint{2.616027in}{1.678633in}}%
\pgfpathlineto{\pgfqpoint{2.613150in}{1.683222in}}%
\pgfpathlineto{\pgfqpoint{2.609766in}{1.688931in}}%
\pgfpathlineto{\pgfqpoint{2.609442in}{1.689482in}}%
\pgfpathlineto{\pgfqpoint{2.605901in}{1.695743in}}%
\pgfpathlineto{\pgfqpoint{2.603506in}{1.700222in}}%
\pgfpathlineto{\pgfqpoint{2.602556in}{1.702003in}}%
\pgfpathlineto{\pgfqpoint{2.599355in}{1.708264in}}%
\pgfpathlineto{\pgfqpoint{2.597245in}{1.712613in}}%
\pgfpathlineto{\pgfqpoint{2.596315in}{1.714524in}}%
\pgfpathlineto{\pgfqpoint{2.593387in}{1.720785in}}%
\pgfpathlineto{\pgfqpoint{2.590985in}{1.726205in}}%
\pgfpathlineto{\pgfqpoint{2.590609in}{1.727045in}}%
\pgfpathlineto{\pgfqpoint{2.587899in}{1.733306in}}%
\pgfpathlineto{\pgfqpoint{2.585326in}{1.739566in}}%
\pgfpathlineto{\pgfqpoint{2.584725in}{1.741083in}}%
\pgfpathlineto{\pgfqpoint{2.582807in}{1.745826in}}%
\pgfpathlineto{\pgfqpoint{2.580380in}{1.752087in}}%
\pgfpathlineto{\pgfqpoint{2.578464in}{1.757236in}}%
\pgfpathlineto{\pgfqpoint{2.578039in}{1.758347in}}%
\pgfpathlineto{\pgfqpoint{2.575713in}{1.764608in}}%
\pgfpathlineto{\pgfqpoint{2.573471in}{1.770868in}}%
\pgfpathlineto{\pgfqpoint{2.572204in}{1.774513in}}%
\pgfpathlineto{\pgfqpoint{2.571261in}{1.777129in}}%
\pgfpathlineto{\pgfqpoint{2.569055in}{1.783389in}}%
\pgfpathlineto{\pgfqpoint{2.566906in}{1.789649in}}%
\pgfpathlineto{\pgfqpoint{2.565943in}{1.792505in}}%
\pgfpathlineto{\pgfqpoint{2.564741in}{1.795910in}}%
\pgfpathlineto{\pgfqpoint{2.562560in}{1.802170in}}%
\pgfpathlineto{\pgfqpoint{2.560405in}{1.808431in}}%
\pgfpathlineto{\pgfqpoint{2.559683in}{1.810542in}}%
\pgfpathlineto{\pgfqpoint{2.558179in}{1.814691in}}%
\pgfpathlineto{\pgfqpoint{2.555912in}{1.820952in}}%
\pgfpathlineto{\pgfqpoint{2.553637in}{1.827212in}}%
\pgfpathlineto{\pgfqpoint{2.553422in}{1.827798in}}%
\pgfpathlineto{\pgfqpoint{2.551197in}{1.833472in}}%
\pgfpathlineto{\pgfqpoint{2.548703in}{1.839733in}}%
\pgfpathlineto{\pgfqpoint{2.547162in}{1.843532in}}%
\pgfpathlineto{\pgfqpoint{2.546079in}{1.845993in}}%
\pgfpathlineto{\pgfqpoint{2.543258in}{1.852254in}}%
\pgfpathlineto{\pgfqpoint{2.540902in}{1.857322in}}%
\pgfpathlineto{\pgfqpoint{2.540292in}{1.858514in}}%
\pgfpathlineto{\pgfqpoint{2.536990in}{1.864775in}}%
\pgfpathlineto{\pgfqpoint{2.534641in}{1.869045in}}%
\pgfpathlineto{\pgfqpoint{2.533420in}{1.871035in}}%
\pgfpathlineto{\pgfqpoint{2.529404in}{1.877295in}}%
\pgfpathlineto{\pgfqpoint{2.528381in}{1.878822in}}%
\pgfpathlineto{\pgfqpoint{2.524770in}{1.883556in}}%
\pgfpathlineto{\pgfqpoint{2.522120in}{1.886851in}}%
\pgfpathlineto{\pgfqpoint{2.519358in}{1.889816in}}%
\pgfpathlineto{\pgfqpoint{2.515860in}{1.893374in}}%
\pgfpathlineto{\pgfqpoint{2.512707in}{1.896077in}}%
\pgfpathlineto{\pgfqpoint{2.509599in}{1.898601in}}%
\pgfpathlineto{\pgfqpoint{2.503966in}{1.902337in}}%
\pgfpathlineto{\pgfqpoint{2.503339in}{1.902732in}}%
\pgfpathlineto{\pgfqpoint{2.497078in}{1.905888in}}%
\pgfpathlineto{\pgfqpoint{2.490818in}{1.908249in}}%
\pgfpathlineto{\pgfqpoint{2.489530in}{1.908598in}}%
\pgfpathlineto{\pgfqpoint{2.484558in}{1.909880in}}%
\pgfpathlineto{\pgfqpoint{2.478297in}{1.910894in}}%
\pgfpathlineto{\pgfqpoint{2.472037in}{1.911368in}}%
\pgfpathlineto{\pgfqpoint{2.465776in}{1.911362in}}%
\pgfpathlineto{\pgfqpoint{2.459516in}{1.910933in}}%
\pgfpathlineto{\pgfqpoint{2.453255in}{1.910128in}}%
\pgfpathlineto{\pgfqpoint{2.446995in}{1.908991in}}%
\pgfpathlineto{\pgfqpoint{2.445305in}{1.908598in}}%
\pgfpathlineto{\pgfqpoint{2.440735in}{1.907543in}}%
\pgfpathlineto{\pgfqpoint{2.434474in}{1.905826in}}%
\pgfpathlineto{\pgfqpoint{2.428214in}{1.903878in}}%
\pgfpathlineto{\pgfqpoint{2.423757in}{1.902337in}}%
\pgfpathlineto{\pgfqpoint{2.421953in}{1.901718in}}%
\pgfpathlineto{\pgfqpoint{2.415693in}{1.899355in}}%
\pgfpathlineto{\pgfqpoint{2.409432in}{1.896836in}}%
\pgfpathlineto{\pgfqpoint{2.407667in}{1.896077in}}%
\pgfpathlineto{\pgfqpoint{2.403172in}{1.894154in}}%
\pgfpathlineto{\pgfqpoint{2.396912in}{1.891346in}}%
\pgfpathlineto{\pgfqpoint{2.393648in}{1.889816in}}%
\pgfpathlineto{\pgfqpoint{2.390651in}{1.888419in}}%
\pgfpathlineto{\pgfqpoint{2.384391in}{1.885387in}}%
\pgfpathlineto{\pgfqpoint{2.380722in}{1.883556in}}%
\pgfpathlineto{\pgfqpoint{2.378130in}{1.882268in}}%
\pgfpathlineto{\pgfqpoint{2.371870in}{1.879068in}}%
\pgfpathlineto{\pgfqpoint{2.368476in}{1.877295in}}%
\pgfpathlineto{\pgfqpoint{2.365609in}{1.875804in}}%
\pgfpathlineto{\pgfqpoint{2.359349in}{1.872483in}}%
\pgfpathlineto{\pgfqpoint{2.356664in}{1.871035in}}%
\pgfpathlineto{\pgfqpoint{2.353088in}{1.869113in}}%
\pgfpathlineto{\pgfqpoint{2.346828in}{1.865712in}}%
\pgfpathlineto{\pgfqpoint{2.345127in}{1.864775in}}%
\pgfpathlineto{\pgfqpoint{2.340568in}{1.862269in}}%
\pgfpathlineto{\pgfqpoint{2.334307in}{1.858819in}}%
\pgfpathlineto{\pgfqpoint{2.333760in}{1.858514in}}%
\pgfpathlineto{\pgfqpoint{2.328047in}{1.855333in}}%
\pgfpathlineto{\pgfqpoint{2.322504in}{1.852254in}}%
\pgfpathlineto{\pgfqpoint{2.321786in}{1.851856in}}%
\pgfpathlineto{\pgfqpoint{2.315526in}{1.848354in}}%
\pgfpathlineto{\pgfqpoint{2.311289in}{1.845993in}}%
\pgfpathlineto{\pgfqpoint{2.309265in}{1.844866in}}%
\pgfpathlineto{\pgfqpoint{2.303005in}{1.841376in}}%
\pgfpathlineto{\pgfqpoint{2.300050in}{1.839733in}}%
\pgfpathlineto{\pgfqpoint{2.296745in}{1.837895in}}%
\pgfpathlineto{\pgfqpoint{2.290484in}{1.834431in}}%
\pgfpathlineto{\pgfqpoint{2.288749in}{1.833472in}}%
\pgfpathlineto{\pgfqpoint{2.284224in}{1.830971in}}%
\pgfpathlineto{\pgfqpoint{2.277963in}{1.827547in}}%
\pgfpathlineto{\pgfqpoint{2.277351in}{1.827212in}}%
\pgfpathlineto{\pgfqpoint{2.271703in}{1.824120in}}%
\pgfpathlineto{\pgfqpoint{2.265831in}{1.820952in}}%
\pgfpathlineto{\pgfqpoint{2.265442in}{1.820741in}}%
\pgfpathlineto{\pgfqpoint{2.259182in}{1.817361in}}%
\pgfpathlineto{\pgfqpoint{2.254161in}{1.814691in}}%
\pgfpathlineto{\pgfqpoint{2.252922in}{1.814030in}}%
\pgfpathlineto{\pgfqpoint{2.246661in}{1.810710in}}%
\pgfpathlineto{\pgfqpoint{2.242301in}{1.808431in}}%
\pgfpathlineto{\pgfqpoint{2.240401in}{1.807434in}}%
\pgfpathlineto{\pgfqpoint{2.234140in}{1.804177in}}%
\pgfpathlineto{\pgfqpoint{2.230225in}{1.802170in}}%
\pgfpathlineto{\pgfqpoint{2.227880in}{1.800962in}}%
\pgfpathlineto{\pgfqpoint{2.221619in}{1.797773in}}%
\pgfpathlineto{\pgfqpoint{2.217908in}{1.795910in}}%
\pgfpathlineto{\pgfqpoint{2.215359in}{1.794623in}}%
\pgfpathlineto{\pgfqpoint{2.209098in}{1.791503in}}%
\pgfpathlineto{\pgfqpoint{2.205320in}{1.789649in}}%
\pgfpathlineto{\pgfqpoint{2.202838in}{1.788424in}}%
\pgfpathlineto{\pgfqpoint{2.196578in}{1.785373in}}%
\pgfpathlineto{\pgfqpoint{2.192435in}{1.783389in}}%
\pgfpathlineto{\pgfqpoint{2.190317in}{1.782367in}}%
\pgfpathlineto{\pgfqpoint{2.184057in}{1.779386in}}%
\pgfpathlineto{\pgfqpoint{2.179222in}{1.777129in}}%
\pgfpathlineto{\pgfqpoint{2.177796in}{1.776457in}}%
\pgfpathlineto{\pgfqpoint{2.171536in}{1.773545in}}%
\pgfpathlineto{\pgfqpoint{2.165648in}{1.770868in}}%
\pgfpathlineto{\pgfqpoint{2.165275in}{1.770697in}}%
\pgfpathlineto{\pgfqpoint{2.159015in}{1.767854in}}%
\pgfpathlineto{\pgfqpoint{2.152755in}{1.765077in}}%
\pgfpathlineto{\pgfqpoint{2.151682in}{1.764608in}}%
\pgfpathlineto{\pgfqpoint{2.146494in}{1.762312in}}%
\pgfpathlineto{\pgfqpoint{2.140234in}{1.759602in}}%
\pgfpathlineto{\pgfqpoint{2.137284in}{1.758347in}}%
\pgfpathlineto{\pgfqpoint{2.133973in}{1.756923in}}%
\pgfpathlineto{\pgfqpoint{2.127713in}{1.754278in}}%
\pgfpathlineto{\pgfqpoint{2.122409in}{1.752087in}}%
\pgfpathlineto{\pgfqpoint{2.121452in}{1.751687in}}%
\pgfpathlineto{\pgfqpoint{2.115192in}{1.749105in}}%
\pgfpathlineto{\pgfqpoint{2.108932in}{1.746587in}}%
\pgfpathlineto{\pgfqpoint{2.107010in}{1.745826in}}%
\pgfpathlineto{\pgfqpoint{2.102671in}{1.744087in}}%
\pgfpathlineto{\pgfqpoint{2.096411in}{1.741630in}}%
\pgfpathlineto{\pgfqpoint{2.091029in}{1.739566in}}%
\pgfpathlineto{\pgfqpoint{2.090150in}{1.739224in}}%
\pgfpathlineto{\pgfqpoint{2.083890in}{1.736826in}}%
\pgfpathlineto{\pgfqpoint{2.077629in}{1.734489in}}%
\pgfpathlineto{\pgfqpoint{2.074397in}{1.733306in}}%
\pgfpathlineto{\pgfqpoint{2.071369in}{1.732180in}}%
\pgfpathlineto{\pgfqpoint{2.065109in}{1.729900in}}%
\pgfpathlineto{\pgfqpoint{2.058848in}{1.727677in}}%
\pgfpathlineto{\pgfqpoint{2.057036in}{1.727045in}}%
\pgfpathlineto{\pgfqpoint{2.052588in}{1.725469in}}%
\pgfpathlineto{\pgfqpoint{2.046327in}{1.723302in}}%
\pgfpathlineto{\pgfqpoint{2.040067in}{1.721190in}}%
\pgfpathlineto{\pgfqpoint{2.038842in}{1.720785in}}%
\pgfpathlineto{\pgfqpoint{2.033806in}{1.719088in}}%
\pgfpathlineto{\pgfqpoint{2.027546in}{1.717031in}}%
\pgfpathlineto{\pgfqpoint{2.021285in}{1.715026in}}%
\pgfpathlineto{\pgfqpoint{2.019686in}{1.714524in}}%
\pgfpathlineto{\pgfqpoint{2.015025in}{1.713035in}}%
\pgfpathlineto{\pgfqpoint{2.008765in}{1.711084in}}%
\pgfpathlineto{\pgfqpoint{2.002504in}{1.709184in}}%
\pgfpathlineto{\pgfqpoint{1.999399in}{1.708264in}}%
\pgfpathlineto{\pgfqpoint{1.996244in}{1.707311in}}%
\pgfpathlineto{\pgfqpoint{1.989983in}{1.705463in}}%
\pgfpathlineto{\pgfqpoint{1.983723in}{1.703667in}}%
\pgfpathlineto{\pgfqpoint{1.977766in}{1.702003in}}%
\pgfpathlineto{\pgfqpoint{1.977462in}{1.701917in}}%
\pgfpathlineto{\pgfqpoint{1.971202in}{1.700172in}}%
\pgfpathlineto{\pgfqpoint{1.964942in}{1.698477in}}%
\pgfpathlineto{\pgfqpoint{1.958681in}{1.696829in}}%
\pgfpathlineto{\pgfqpoint{1.954438in}{1.695743in}}%
\pgfpathlineto{\pgfqpoint{1.952421in}{1.695216in}}%
\pgfpathlineto{\pgfqpoint{1.946160in}{1.693621in}}%
\pgfpathlineto{\pgfqpoint{1.939900in}{1.692074in}}%
\pgfpathlineto{\pgfqpoint{1.933639in}{1.690573in}}%
\pgfpathlineto{\pgfqpoint{1.928954in}{1.689482in}}%
\pgfpathlineto{\pgfqpoint{1.927379in}{1.689108in}}%
\pgfpathlineto{\pgfqpoint{1.921119in}{1.687661in}}%
\pgfpathlineto{\pgfqpoint{1.914858in}{1.686260in}}%
\pgfpathlineto{\pgfqpoint{1.908598in}{1.684905in}}%
\pgfpathlineto{\pgfqpoint{1.902337in}{1.683594in}}%
\pgfpathlineto{\pgfqpoint{1.900500in}{1.683222in}}%
\pgfpathlineto{\pgfqpoint{1.896077in}{1.682305in}}%
\pgfpathlineto{\pgfqpoint{1.889816in}{1.681051in}}%
\pgfpathlineto{\pgfqpoint{1.883556in}{1.679842in}}%
\pgfpathlineto{\pgfqpoint{1.877295in}{1.678678in}}%
\pgfpathlineto{\pgfqpoint{1.871035in}{1.677557in}}%
\pgfpathlineto{\pgfqpoint{1.867575in}{1.676962in}}%
\pgfpathlineto{\pgfqpoint{1.864775in}{1.676468in}}%
\pgfpathlineto{\pgfqpoint{1.858514in}{1.675409in}}%
\pgfpathlineto{\pgfqpoint{1.852254in}{1.674395in}}%
\pgfpathlineto{\pgfqpoint{1.845993in}{1.673425in}}%
\pgfpathlineto{\pgfqpoint{1.839733in}{1.672499in}}%
\pgfpathlineto{\pgfqpoint{1.833472in}{1.671617in}}%
\pgfpathlineto{\pgfqpoint{1.827212in}{1.670779in}}%
\pgfpathlineto{\pgfqpoint{1.826596in}{1.670701in}}%
\pgfpathlineto{\pgfqpoint{1.820952in}{1.669969in}}%
\pgfpathlineto{\pgfqpoint{1.814691in}{1.669202in}}%
\pgfpathlineto{\pgfqpoint{1.808431in}{1.668480in}}%
\pgfpathlineto{\pgfqpoint{1.802170in}{1.667805in}}%
\pgfpathlineto{\pgfqpoint{1.795910in}{1.667175in}}%
\pgfpathlineto{\pgfqpoint{1.789649in}{1.666591in}}%
\pgfpathlineto{\pgfqpoint{1.783389in}{1.666054in}}%
\pgfpathlineto{\pgfqpoint{1.777129in}{1.665564in}}%
\pgfpathlineto{\pgfqpoint{1.770868in}{1.665121in}}%
\pgfpathlineto{\pgfqpoint{1.764608in}{1.664725in}}%
\pgfpathlineto{\pgfqpoint{1.759471in}{1.664441in}}%
\pgfpathlineto{\pgfqpoint{1.758347in}{1.664377in}}%
\pgfpathlineto{\pgfqpoint{1.752087in}{1.664073in}}%
\pgfpathlineto{\pgfqpoint{1.745826in}{1.663821in}}%
\pgfpathlineto{\pgfqpoint{1.739566in}{1.663620in}}%
\pgfpathlineto{\pgfqpoint{1.733306in}{1.663472in}}%
\pgfpathlineto{\pgfqpoint{1.727045in}{1.663378in}}%
\pgfpathlineto{\pgfqpoint{1.720785in}{1.663338in}}%
\pgfpathlineto{\pgfqpoint{1.714524in}{1.663355in}}%
\pgfpathlineto{\pgfqpoint{1.708264in}{1.663428in}}%
\pgfpathlineto{\pgfqpoint{1.702003in}{1.663561in}}%
\pgfpathlineto{\pgfqpoint{1.695743in}{1.663753in}}%
\pgfpathlineto{\pgfqpoint{1.689482in}{1.664007in}}%
\pgfpathlineto{\pgfqpoint{1.683222in}{1.664324in}}%
\pgfpathlineto{\pgfqpoint{1.681297in}{1.664441in}}%
\pgfpathlineto{\pgfqpoint{1.676962in}{1.664703in}}%
\pgfpathlineto{\pgfqpoint{1.670701in}{1.665147in}}%
\pgfpathlineto{\pgfqpoint{1.664441in}{1.665659in}}%
\pgfpathlineto{\pgfqpoint{1.658180in}{1.666243in}}%
\pgfpathlineto{\pgfqpoint{1.651920in}{1.666900in}}%
\pgfpathlineto{\pgfqpoint{1.645659in}{1.667633in}}%
\pgfpathlineto{\pgfqpoint{1.639399in}{1.668445in}}%
\pgfpathlineto{\pgfqpoint{1.633139in}{1.669338in}}%
\pgfpathlineto{\pgfqpoint{1.626878in}{1.670316in}}%
\pgfpathlineto{\pgfqpoint{1.624599in}{1.670701in}}%
\pgfpathlineto{\pgfqpoint{1.620618in}{1.671379in}}%
\pgfpathlineto{\pgfqpoint{1.614357in}{1.672530in}}%
\pgfpathlineto{\pgfqpoint{1.608097in}{1.673778in}}%
\pgfpathlineto{\pgfqpoint{1.601836in}{1.675126in}}%
\pgfpathlineto{\pgfqpoint{1.595576in}{1.676579in}}%
\pgfpathlineto{\pgfqpoint{1.594023in}{1.676962in}}%
\pgfpathlineto{\pgfqpoint{1.589316in}{1.678139in}}%
\pgfpathlineto{\pgfqpoint{1.583055in}{1.679814in}}%
\pgfpathlineto{\pgfqpoint{1.576795in}{1.681611in}}%
\pgfpathlineto{\pgfqpoint{1.571534in}{1.683222in}}%
\pgfpathlineto{\pgfqpoint{1.570534in}{1.683535in}}%
\pgfpathlineto{\pgfqpoint{1.564274in}{1.685598in}}%
\pgfpathlineto{\pgfqpoint{1.558013in}{1.687803in}}%
\pgfpathlineto{\pgfqpoint{1.553524in}{1.689482in}}%
\pgfpathlineto{\pgfqpoint{1.551753in}{1.690164in}}%
\pgfpathlineto{\pgfqpoint{1.545492in}{1.692696in}}%
\pgfpathlineto{\pgfqpoint{1.539232in}{1.695399in}}%
\pgfpathlineto{\pgfqpoint{1.538470in}{1.695743in}}%
\pgfpathlineto{\pgfqpoint{1.532972in}{1.698311in}}%
\pgfpathlineto{\pgfqpoint{1.526711in}{1.701422in}}%
\pgfpathlineto{\pgfqpoint{1.525592in}{1.702003in}}%
\pgfpathlineto{\pgfqpoint{1.520451in}{1.704783in}}%
\pgfpathlineto{\pgfqpoint{1.514389in}{1.708264in}}%
\pgfpathlineto{\pgfqpoint{1.514190in}{1.708383in}}%
\pgfpathlineto{\pgfqpoint{1.507930in}{1.712299in}}%
\pgfpathlineto{\pgfqpoint{1.504557in}{1.714524in}}%
\pgfpathlineto{\pgfqpoint{1.501669in}{1.716531in}}%
\pgfpathlineto{\pgfqpoint{1.495847in}{1.720785in}}%
\pgfpathlineto{\pgfqpoint{1.495409in}{1.721123in}}%
\pgfpathlineto{\pgfqpoint{1.489149in}{1.726171in}}%
\pgfpathlineto{\pgfqpoint{1.488109in}{1.727045in}}%
\pgfpathlineto{\pgfqpoint{1.482888in}{1.731736in}}%
\pgfpathlineto{\pgfqpoint{1.481212in}{1.733306in}}%
\pgfpathlineto{\pgfqpoint{1.476628in}{1.737921in}}%
\pgfpathlineto{\pgfqpoint{1.475054in}{1.739566in}}%
\pgfpathlineto{\pgfqpoint{1.470367in}{1.744877in}}%
\pgfpathlineto{\pgfqpoint{1.469557in}{1.745826in}}%
\pgfpathlineto{\pgfqpoint{1.464659in}{1.752087in}}%
\pgfpathlineto{\pgfqpoint{1.464107in}{1.752860in}}%
\pgfpathlineto{\pgfqpoint{1.460303in}{1.758347in}}%
\pgfpathlineto{\pgfqpoint{1.457846in}{1.762297in}}%
\pgfpathlineto{\pgfqpoint{1.456446in}{1.764608in}}%
\pgfpathlineto{\pgfqpoint{1.453049in}{1.770868in}}%
\pgfpathlineto{\pgfqpoint{1.451586in}{1.773920in}}%
\pgfpathlineto{\pgfqpoint{1.450080in}{1.777129in}}%
\pgfpathlineto{\pgfqpoint{1.447509in}{1.783389in}}%
\pgfpathlineto{\pgfqpoint{1.445326in}{1.789630in}}%
\pgfpathlineto{\pgfqpoint{1.445319in}{1.789649in}}%
\pgfpathlineto{\pgfqpoint{1.443473in}{1.795910in}}%
\pgfpathlineto{\pgfqpoint{1.441965in}{1.802170in}}%
\pgfpathlineto{\pgfqpoint{1.440776in}{1.808431in}}%
\pgfpathlineto{\pgfqpoint{1.439891in}{1.814691in}}%
\pgfpathlineto{\pgfqpoint{1.439295in}{1.820952in}}%
\pgfpathlineto{\pgfqpoint{1.439065in}{1.825427in}}%
\pgfpathlineto{\pgfqpoint{1.438974in}{1.827212in}}%
\pgfpathlineto{\pgfqpoint{1.438918in}{1.833472in}}%
\pgfpathlineto{\pgfqpoint{1.439065in}{1.838030in}}%
\pgfpathlineto{\pgfqpoint{1.439119in}{1.839733in}}%
\pgfpathlineto{\pgfqpoint{1.439564in}{1.845993in}}%
\pgfpathlineto{\pgfqpoint{1.440248in}{1.852254in}}%
\pgfpathlineto{\pgfqpoint{1.441163in}{1.858514in}}%
\pgfpathlineto{\pgfqpoint{1.442305in}{1.864775in}}%
\pgfpathlineto{\pgfqpoint{1.443669in}{1.871035in}}%
\pgfpathlineto{\pgfqpoint{1.445252in}{1.877295in}}%
\pgfpathlineto{\pgfqpoint{1.445326in}{1.877551in}}%
\pgfpathlineto{\pgfqpoint{1.447028in}{1.883556in}}%
\pgfpathlineto{\pgfqpoint{1.449012in}{1.889816in}}%
\pgfpathlineto{\pgfqpoint{1.451206in}{1.896077in}}%
\pgfpathlineto{\pgfqpoint{1.451586in}{1.897065in}}%
\pgfpathlineto{\pgfqpoint{1.453581in}{1.902337in}}%
\pgfpathlineto{\pgfqpoint{1.456155in}{1.908598in}}%
\pgfpathlineto{\pgfqpoint{1.457846in}{1.912410in}}%
\pgfpathlineto{\pgfqpoint{1.458919in}{1.914858in}}%
\pgfpathlineto{\pgfqpoint{1.461862in}{1.921119in}}%
\pgfpathlineto{\pgfqpoint{1.464107in}{1.925587in}}%
\pgfpathlineto{\pgfqpoint{1.464997in}{1.927379in}}%
\pgfpathlineto{\pgfqpoint{1.468303in}{1.933639in}}%
\pgfpathlineto{\pgfqpoint{1.470367in}{1.937326in}}%
\pgfpathlineto{\pgfqpoint{1.471794in}{1.939900in}}%
\pgfpathlineto{\pgfqpoint{1.475460in}{1.946160in}}%
\pgfpathlineto{\pgfqpoint{1.476628in}{1.948055in}}%
\pgfpathlineto{\pgfqpoint{1.479295in}{1.952421in}}%
\pgfpathlineto{\pgfqpoint{1.482888in}{1.958008in}}%
\pgfpathlineto{\pgfqpoint{1.483317in}{1.958681in}}%
\pgfpathlineto{\pgfqpoint{1.487496in}{1.964942in}}%
\pgfpathlineto{\pgfqpoint{1.489149in}{1.967309in}}%
\pgfpathlineto{\pgfqpoint{1.491849in}{1.971202in}}%
\pgfpathlineto{\pgfqpoint{1.495409in}{1.976114in}}%
\pgfpathlineto{\pgfqpoint{1.496380in}{1.977462in}}%
\pgfpathlineto{\pgfqpoint{1.501077in}{1.983723in}}%
\pgfpathlineto{\pgfqpoint{1.501669in}{1.984483in}}%
\pgfpathlineto{\pgfqpoint{1.505934in}{1.989983in}}%
\pgfpathlineto{\pgfqpoint{1.507930in}{1.992462in}}%
\pgfpathlineto{\pgfqpoint{1.510964in}{1.996244in}}%
\pgfpathlineto{\pgfqpoint{1.514190in}{2.000123in}}%
\pgfpathlineto{\pgfqpoint{1.516165in}{2.002504in}}%
\pgfpathlineto{\pgfqpoint{1.520451in}{2.007498in}}%
\pgfpathlineto{\pgfqpoint{1.521535in}{2.008765in}}%
\pgfpathlineto{\pgfqpoint{1.526711in}{2.014614in}}%
\pgfpathlineto{\pgfqpoint{1.527075in}{2.015025in}}%
\pgfpathlineto{\pgfqpoint{1.532779in}{2.021285in}}%
\pgfpathlineto{\pgfqpoint{1.532972in}{2.021491in}}%
\pgfpathlineto{\pgfqpoint{1.538648in}{2.027546in}}%
\pgfpathlineto{\pgfqpoint{1.539232in}{2.028152in}}%
\pgfpathlineto{\pgfqpoint{1.544688in}{2.033806in}}%
\pgfpathlineto{\pgfqpoint{1.545492in}{2.034619in}}%
\pgfpathlineto{\pgfqpoint{1.550898in}{2.040067in}}%
\pgfpathlineto{\pgfqpoint{1.551753in}{2.040907in}}%
\pgfpathlineto{\pgfqpoint{1.557280in}{2.046327in}}%
\pgfpathlineto{\pgfqpoint{1.558013in}{2.047030in}}%
\pgfpathlineto{\pgfqpoint{1.563835in}{2.052588in}}%
\pgfpathlineto{\pgfqpoint{1.564274in}{2.052997in}}%
\pgfpathlineto{\pgfqpoint{1.570534in}{2.058820in}}%
\pgfpathlineto{\pgfqpoint{1.570564in}{2.058848in}}%
\pgfpathlineto{\pgfqpoint{1.576795in}{2.064508in}}%
\pgfpathlineto{\pgfqpoint{1.577460in}{2.065109in}}%
\pgfpathlineto{\pgfqpoint{1.583055in}{2.070068in}}%
\pgfpathlineto{\pgfqpoint{1.584531in}{2.071369in}}%
\pgfpathlineto{\pgfqpoint{1.589316in}{2.075510in}}%
\pgfpathlineto{\pgfqpoint{1.591781in}{2.077629in}}%
\pgfpathlineto{\pgfqpoint{1.595576in}{2.080838in}}%
\pgfpathlineto{\pgfqpoint{1.599211in}{2.083890in}}%
\pgfpathlineto{\pgfqpoint{1.601836in}{2.086059in}}%
\pgfpathlineto{\pgfqpoint{1.606826in}{2.090150in}}%
\pgfpathlineto{\pgfqpoint{1.608097in}{2.091177in}}%
\pgfpathlineto{\pgfqpoint{1.614357in}{2.096198in}}%
\pgfpathlineto{\pgfqpoint{1.614624in}{2.096411in}}%
\pgfpathlineto{\pgfqpoint{1.620618in}{2.101131in}}%
\pgfpathlineto{\pgfqpoint{1.622592in}{2.102671in}}%
\pgfpathlineto{\pgfqpoint{1.626878in}{2.105975in}}%
\pgfpathlineto{\pgfqpoint{1.630752in}{2.108932in}}%
\pgfpathlineto{\pgfqpoint{1.633139in}{2.110734in}}%
\pgfpathlineto{\pgfqpoint{1.639107in}{2.115192in}}%
\pgfpathlineto{\pgfqpoint{1.639399in}{2.115408in}}%
\pgfpathlineto{\pgfqpoint{1.645659in}{2.120015in}}%
\pgfpathlineto{\pgfqpoint{1.647633in}{2.121452in}}%
\pgfpathlineto{\pgfqpoint{1.651920in}{2.124549in}}%
\pgfpathlineto{\pgfqpoint{1.656355in}{2.127713in}}%
\pgfpathlineto{\pgfqpoint{1.658180in}{2.129007in}}%
\pgfpathlineto{\pgfqpoint{1.664441in}{2.133397in}}%
\pgfpathlineto{\pgfqpoint{1.665268in}{2.133973in}}%
\pgfpathlineto{\pgfqpoint{1.670701in}{2.137735in}}%
\pgfpathlineto{\pgfqpoint{1.674359in}{2.140234in}}%
\pgfpathlineto{\pgfqpoint{1.676962in}{2.142004in}}%
\pgfpathlineto{\pgfqpoint{1.683222in}{2.146209in}}%
\pgfpathlineto{\pgfqpoint{1.683649in}{2.146494in}}%
\pgfpathlineto{\pgfqpoint{1.689482in}{2.150377in}}%
\pgfpathlineto{\pgfqpoint{1.693107in}{2.152755in}}%
\pgfpathlineto{\pgfqpoint{1.695743in}{2.154481in}}%
\pgfpathlineto{\pgfqpoint{1.702003in}{2.158531in}}%
\pgfpathlineto{\pgfqpoint{1.702757in}{2.159015in}}%
\pgfpathlineto{\pgfqpoint{1.708264in}{2.162551in}}%
\pgfpathlineto{\pgfqpoint{1.712570in}{2.165275in}}%
\pgfpathlineto{\pgfqpoint{1.714524in}{2.166514in}}%
\pgfpathlineto{\pgfqpoint{1.720785in}{2.170439in}}%
\pgfpathlineto{\pgfqpoint{1.722552in}{2.171536in}}%
\pgfpathlineto{\pgfqpoint{1.727045in}{2.174334in}}%
\pgfpathlineto{\pgfqpoint{1.732691in}{2.177796in}}%
\pgfpathlineto{\pgfqpoint{1.733306in}{2.178175in}}%
\pgfpathlineto{\pgfqpoint{1.739566in}{2.182008in}}%
\pgfpathlineto{\pgfqpoint{1.742959in}{2.184057in}}%
\pgfpathlineto{\pgfqpoint{1.745826in}{2.185800in}}%
\pgfpathlineto{\pgfqpoint{1.752087in}{2.189561in}}%
\pgfpathlineto{\pgfqpoint{1.753355in}{2.190317in}}%
\pgfpathlineto{\pgfqpoint{1.758347in}{2.193319in}}%
\pgfpathlineto{\pgfqpoint{1.763847in}{2.196578in}}%
\pgfpathlineto{\pgfqpoint{1.764608in}{2.197033in}}%
\pgfpathlineto{\pgfqpoint{1.770868in}{2.200760in}}%
\pgfpathlineto{\pgfqpoint{1.774402in}{2.202838in}}%
\pgfpathlineto{\pgfqpoint{1.777129in}{2.204462in}}%
\pgfpathlineto{\pgfqpoint{1.783389in}{2.208155in}}%
\pgfpathlineto{\pgfqpoint{1.784999in}{2.209098in}}%
\pgfpathlineto{\pgfqpoint{1.789649in}{2.211865in}}%
\pgfpathlineto{\pgfqpoint{1.795596in}{2.215359in}}%
\pgfpathlineto{\pgfqpoint{1.795910in}{2.215546in}}%
\pgfpathlineto{\pgfqpoint{1.802170in}{2.219279in}}%
\pgfpathlineto{\pgfqpoint{1.806135in}{2.221619in}}%
\pgfpathlineto{\pgfqpoint{1.808431in}{2.223003in}}%
\pgfpathlineto{\pgfqpoint{1.814691in}{2.226752in}}%
\pgfpathlineto{\pgfqpoint{1.816583in}{2.227880in}}%
\pgfpathlineto{\pgfqpoint{1.820952in}{2.230547in}}%
\pgfpathlineto{\pgfqpoint{1.826883in}{2.234140in}}%
\pgfpathlineto{\pgfqpoint{1.827212in}{2.234345in}}%
\pgfpathlineto{\pgfqpoint{1.833472in}{2.238241in}}%
\pgfpathlineto{\pgfqpoint{1.836961in}{2.240401in}}%
\pgfpathlineto{\pgfqpoint{1.839733in}{2.242172in}}%
\pgfpathlineto{\pgfqpoint{1.845993in}{2.246160in}}%
\pgfpathlineto{\pgfqpoint{1.846779in}{2.246661in}}%
\pgfpathlineto{\pgfqpoint{1.852254in}{2.250281in}}%
\pgfpathlineto{\pgfqpoint{1.856255in}{2.252922in}}%
\pgfpathlineto{\pgfqpoint{1.858514in}{2.254475in}}%
\pgfpathlineto{\pgfqpoint{1.864775in}{2.258785in}}%
\pgfpathlineto{\pgfqpoint{1.865349in}{2.259182in}}%
\pgfpathlineto{\pgfqpoint{1.871035in}{2.263301in}}%
\pgfpathlineto{\pgfqpoint{1.873982in}{2.265442in}}%
\pgfpathlineto{\pgfqpoint{1.877295in}{2.267983in}}%
\pgfpathlineto{\pgfqpoint{1.882122in}{2.271703in}}%
\pgfpathlineto{\pgfqpoint{1.883556in}{2.272877in}}%
\pgfpathlineto{\pgfqpoint{1.889720in}{2.277963in}}%
\pgfpathlineto{\pgfqpoint{1.889816in}{2.278048in}}%
\pgfpathlineto{\pgfqpoint{1.896077in}{2.283640in}}%
\pgfpathlineto{\pgfqpoint{1.896724in}{2.284224in}}%
\pgfpathlineto{\pgfqpoint{1.902337in}{2.289714in}}%
\pgfpathlineto{\pgfqpoint{1.903115in}{2.290484in}}%
\pgfpathlineto{\pgfqpoint{1.908598in}{2.296444in}}%
\pgfpathlineto{\pgfqpoint{1.908870in}{2.296745in}}%
\pgfpathlineto{\pgfqpoint{1.913966in}{2.303005in}}%
\pgfpathlineto{\pgfqpoint{1.914858in}{2.304247in}}%
\pgfpathlineto{\pgfqpoint{1.918397in}{2.309265in}}%
\pgfpathlineto{\pgfqpoint{1.921119in}{2.313751in}}%
\pgfpathlineto{\pgfqpoint{1.922174in}{2.315526in}}%
\pgfpathlineto{\pgfqpoint{1.925281in}{2.321786in}}%
\pgfpathlineto{\pgfqpoint{1.927379in}{2.327097in}}%
\pgfpathlineto{\pgfqpoint{1.927746in}{2.328047in}}%
\pgfpathlineto{\pgfqpoint{1.929552in}{2.334307in}}%
\pgfpathlineto{\pgfqpoint{1.930737in}{2.340568in}}%
\pgfpathlineto{\pgfqpoint{1.931312in}{2.346828in}}%
\pgfpathlineto{\pgfqpoint{1.931293in}{2.353088in}}%
\pgfpathlineto{\pgfqpoint{1.930699in}{2.359349in}}%
\pgfpathlineto{\pgfqpoint{1.929555in}{2.365609in}}%
\pgfpathlineto{\pgfqpoint{1.927886in}{2.371870in}}%
\pgfpathlineto{\pgfqpoint{1.927379in}{2.373320in}}%
\pgfpathlineto{\pgfqpoint{1.925696in}{2.378130in}}%
\pgfpathlineto{\pgfqpoint{1.923027in}{2.384391in}}%
\pgfpathlineto{\pgfqpoint{1.921119in}{2.388208in}}%
\pgfpathlineto{\pgfqpoint{1.919899in}{2.390651in}}%
\pgfpathlineto{\pgfqpoint{1.916331in}{2.396912in}}%
\pgfpathlineto{\pgfqpoint{1.914858in}{2.399221in}}%
\pgfpathlineto{\pgfqpoint{1.912347in}{2.403172in}}%
\pgfpathlineto{\pgfqpoint{1.908598in}{2.408551in}}%
\pgfpathlineto{\pgfqpoint{1.907986in}{2.409432in}}%
\pgfpathlineto{\pgfqpoint{1.903251in}{2.415693in}}%
\pgfpathlineto{\pgfqpoint{1.902337in}{2.416815in}}%
\pgfpathlineto{\pgfqpoint{1.898173in}{2.421953in}}%
\pgfpathlineto{\pgfqpoint{1.896077in}{2.424382in}}%
\pgfpathlineto{\pgfqpoint{1.892785in}{2.428214in}}%
\pgfpathlineto{\pgfqpoint{1.889816in}{2.431480in}}%
\pgfpathlineto{\pgfqpoint{1.887109in}{2.434474in}}%
\pgfpathlineto{\pgfqpoint{1.883556in}{2.438209in}}%
\pgfpathlineto{\pgfqpoint{1.881167in}{2.440735in}}%
\pgfpathlineto{\pgfqpoint{1.877295in}{2.444644in}}%
\pgfpathlineto{\pgfqpoint{1.874983in}{2.446995in}}%
\pgfpathlineto{\pgfqpoint{1.871035in}{2.450844in}}%
\pgfpathlineto{\pgfqpoint{1.868579in}{2.453255in}}%
\pgfpathlineto{\pgfqpoint{1.864775in}{2.456854in}}%
\pgfpathlineto{\pgfqpoint{1.861980in}{2.459516in}}%
\pgfpathlineto{\pgfqpoint{1.858514in}{2.462708in}}%
\pgfpathlineto{\pgfqpoint{1.855208in}{2.465776in}}%
\pgfpathlineto{\pgfqpoint{1.852254in}{2.468435in}}%
\pgfpathlineto{\pgfqpoint{1.848284in}{2.472037in}}%
\pgfpathlineto{\pgfqpoint{1.845993in}{2.474058in}}%
\pgfpathlineto{\pgfqpoint{1.841230in}{2.478297in}}%
\pgfpathlineto{\pgfqpoint{1.839733in}{2.479596in}}%
\pgfpathlineto{\pgfqpoint{1.834065in}{2.484558in}}%
\pgfpathlineto{\pgfqpoint{1.833472in}{2.485065in}}%
\pgfpathlineto{\pgfqpoint{1.827212in}{2.490462in}}%
\pgfpathlineto{\pgfqpoint{1.826801in}{2.490818in}}%
\pgfpathlineto{\pgfqpoint{1.820952in}{2.495791in}}%
\pgfpathlineto{\pgfqpoint{1.819451in}{2.497078in}}%
\pgfpathlineto{\pgfqpoint{1.814691in}{2.501088in}}%
\pgfpathlineto{\pgfqpoint{1.812042in}{2.503339in}}%
\pgfpathlineto{\pgfqpoint{1.808431in}{2.506359in}}%
\pgfpathlineto{\pgfqpoint{1.804591in}{2.509599in}}%
\pgfpathlineto{\pgfqpoint{1.802170in}{2.511612in}}%
\pgfpathlineto{\pgfqpoint{1.797110in}{2.515860in}}%
\pgfpathlineto{\pgfqpoint{1.795910in}{2.516855in}}%
\pgfpathlineto{\pgfqpoint{1.789649in}{2.522090in}}%
\pgfpathlineto{\pgfqpoint{1.789613in}{2.522120in}}%
\pgfpathlineto{\pgfqpoint{1.783389in}{2.527289in}}%
\pgfpathlineto{\pgfqpoint{1.782086in}{2.528381in}}%
\pgfpathlineto{\pgfqpoint{1.777129in}{2.532494in}}%
\pgfpathlineto{\pgfqpoint{1.774565in}{2.534641in}}%
\pgfpathlineto{\pgfqpoint{1.770868in}{2.537710in}}%
\pgfpathlineto{\pgfqpoint{1.767058in}{2.540902in}}%
\pgfpathlineto{\pgfqpoint{1.764608in}{2.542939in}}%
\pgfpathlineto{\pgfqpoint{1.759575in}{2.547162in}}%
\pgfpathlineto{\pgfqpoint{1.758347in}{2.548186in}}%
\pgfpathlineto{\pgfqpoint{1.752122in}{2.553422in}}%
\pgfpathlineto{\pgfqpoint{1.752087in}{2.553452in}}%
\pgfpathlineto{\pgfqpoint{1.745826in}{2.558712in}}%
\pgfpathlineto{\pgfqpoint{1.744680in}{2.559683in}}%
\pgfpathlineto{\pgfqpoint{1.739566in}{2.563998in}}%
\pgfpathlineto{\pgfqpoint{1.737280in}{2.565943in}}%
\pgfpathlineto{\pgfqpoint{1.733306in}{2.569314in}}%
\pgfpathlineto{\pgfqpoint{1.729927in}{2.572204in}}%
\pgfpathlineto{\pgfqpoint{1.727045in}{2.574662in}}%
\pgfpathlineto{\pgfqpoint{1.722624in}{2.578464in}}%
\pgfpathlineto{\pgfqpoint{1.720785in}{2.580044in}}%
\pgfpathlineto{\pgfqpoint{1.715376in}{2.584725in}}%
\pgfpathlineto{\pgfqpoint{1.714524in}{2.585461in}}%
\pgfpathlineto{\pgfqpoint{1.708264in}{2.590914in}}%
\pgfpathlineto{\pgfqpoint{1.708183in}{2.590985in}}%
\pgfpathlineto{\pgfqpoint{1.702003in}{2.596388in}}%
\pgfpathlineto{\pgfqpoint{1.701029in}{2.597245in}}%
\pgfpathlineto{\pgfqpoint{1.695743in}{2.601905in}}%
\pgfpathlineto{\pgfqpoint{1.693938in}{2.603506in}}%
\pgfpathlineto{\pgfqpoint{1.689482in}{2.607466in}}%
\pgfpathlineto{\pgfqpoint{1.686911in}{2.609766in}}%
\pgfpathlineto{\pgfqpoint{1.683222in}{2.613074in}}%
\pgfpathlineto{\pgfqpoint{1.679948in}{2.616027in}}%
\pgfpathlineto{\pgfqpoint{1.676962in}{2.618729in}}%
\pgfpathlineto{\pgfqpoint{1.673051in}{2.622287in}}%
\pgfpathlineto{\pgfqpoint{1.670701in}{2.624433in}}%
\pgfpathlineto{\pgfqpoint{1.666220in}{2.628548in}}%
\pgfpathlineto{\pgfqpoint{1.664441in}{2.630188in}}%
\pgfpathlineto{\pgfqpoint{1.659455in}{2.634808in}}%
\pgfpathlineto{\pgfqpoint{1.658180in}{2.635995in}}%
\pgfpathlineto{\pgfqpoint{1.652758in}{2.641068in}}%
\pgfpathlineto{\pgfqpoint{1.651920in}{2.641857in}}%
\pgfpathlineto{\pgfqpoint{1.646128in}{2.647329in}}%
\pgfpathlineto{\pgfqpoint{1.645659in}{2.647774in}}%
\pgfpathlineto{\pgfqpoint{1.639565in}{2.653589in}}%
\pgfpathlineto{\pgfqpoint{1.639399in}{2.653749in}}%
\pgfpathlineto{\pgfqpoint{1.633139in}{2.659782in}}%
\pgfpathlineto{\pgfqpoint{1.633068in}{2.659850in}}%
\pgfpathlineto{\pgfqpoint{1.626878in}{2.665874in}}%
\pgfpathlineto{\pgfqpoint{1.626636in}{2.666110in}}%
\pgfpathlineto{\pgfqpoint{1.620618in}{2.672032in}}%
\pgfpathlineto{\pgfqpoint{1.620274in}{2.672371in}}%
\pgfpathlineto{\pgfqpoint{1.614357in}{2.678256in}}%
\pgfpathlineto{\pgfqpoint{1.613981in}{2.678631in}}%
\pgfpathlineto{\pgfqpoint{1.608097in}{2.684549in}}%
\pgfpathlineto{\pgfqpoint{1.607757in}{2.684891in}}%
\pgfpathlineto{\pgfqpoint{1.601836in}{2.690914in}}%
\pgfpathlineto{\pgfqpoint{1.601603in}{2.691152in}}%
\pgfpathlineto{\pgfqpoint{1.595576in}{2.697353in}}%
\pgfpathlineto{\pgfqpoint{1.595518in}{2.697412in}}%
\pgfpathlineto{\pgfqpoint{1.589498in}{2.703673in}}%
\pgfpathlineto{\pgfqpoint{1.589316in}{2.703865in}}%
\pgfpathlineto{\pgfqpoint{1.583545in}{2.709933in}}%
\pgfpathlineto{\pgfqpoint{1.583055in}{2.710455in}}%
\pgfpathlineto{\pgfqpoint{1.577660in}{2.716194in}}%
\pgfpathlineto{\pgfqpoint{1.576795in}{2.717126in}}%
\pgfpathlineto{\pgfqpoint{1.571844in}{2.722454in}}%
\pgfpathlineto{\pgfqpoint{1.570534in}{2.723882in}}%
\pgfpathlineto{\pgfqpoint{1.566097in}{2.728715in}}%
\pgfpathlineto{\pgfqpoint{1.564274in}{2.730727in}}%
\pgfpathlineto{\pgfqpoint{1.560419in}{2.734975in}}%
\pgfpathlineto{\pgfqpoint{1.558013in}{2.737664in}}%
\pgfpathlineto{\pgfqpoint{1.554811in}{2.741235in}}%
\pgfpathlineto{\pgfqpoint{1.551753in}{2.744698in}}%
\pgfpathlineto{\pgfqpoint{1.549274in}{2.747496in}}%
\pgfpathlineto{\pgfqpoint{1.545492in}{2.751831in}}%
\pgfpathlineto{\pgfqpoint{1.543808in}{2.753756in}}%
\pgfpathlineto{\pgfqpoint{1.539232in}{2.759070in}}%
\pgfpathlineto{\pgfqpoint{1.538414in}{2.760017in}}%
\pgfpathlineto{\pgfqpoint{1.533090in}{2.766277in}}%
\pgfpathlineto{\pgfqpoint{1.532972in}{2.766418in}}%
\pgfpathlineto{\pgfqpoint{1.527816in}{2.772538in}}%
\pgfpathlineto{\pgfqpoint{1.526711in}{2.773873in}}%
\pgfpathlineto{\pgfqpoint{1.522616in}{2.778798in}}%
\pgfpathlineto{\pgfqpoint{1.520451in}{2.781450in}}%
\pgfpathlineto{\pgfqpoint{1.517489in}{2.785058in}}%
\pgfpathlineto{\pgfqpoint{1.514190in}{2.789155in}}%
\pgfpathlineto{\pgfqpoint{1.512437in}{2.791319in}}%
\pgfpathlineto{\pgfqpoint{1.507930in}{2.796994in}}%
\pgfpathlineto{\pgfqpoint{1.507462in}{2.797579in}}%
\pgfpathlineto{\pgfqpoint{1.502542in}{2.803840in}}%
\pgfpathlineto{\pgfqpoint{1.501669in}{2.804971in}}%
\pgfpathlineto{\pgfqpoint{1.497687in}{2.810100in}}%
\pgfpathlineto{\pgfqpoint{1.495409in}{2.813097in}}%
\pgfpathlineto{\pgfqpoint{1.492910in}{2.816361in}}%
\pgfpathlineto{\pgfqpoint{1.489149in}{2.821384in}}%
\pgfpathlineto{\pgfqpoint{1.488214in}{2.822621in}}%
\pgfpathlineto{\pgfqpoint{1.483583in}{2.828881in}}%
\pgfpathlineto{\pgfqpoint{1.482888in}{2.829840in}}%
\pgfpathlineto{\pgfqpoint{1.479010in}{2.835142in}}%
\pgfpathlineto{\pgfqpoint{1.476628in}{2.838478in}}%
\pgfpathlineto{\pgfqpoint{1.474521in}{2.841402in}}%
\pgfpathlineto{\pgfqpoint{1.470367in}{2.847310in}}%
\pgfpathlineto{\pgfqpoint{1.470117in}{2.847663in}}%
\pgfpathlineto{\pgfqpoint{1.465760in}{2.853923in}}%
\pgfpathlineto{\pgfqpoint{1.464107in}{2.856360in}}%
\pgfpathlineto{\pgfqpoint{1.461485in}{2.860184in}}%
\pgfpathlineto{\pgfqpoint{1.457846in}{2.865635in}}%
\pgfpathlineto{\pgfqpoint{1.457300in}{2.866444in}}%
\pgfpathlineto{\pgfqpoint{1.453170in}{2.872705in}}%
\pgfpathlineto{\pgfqpoint{1.451586in}{2.875170in}}%
\pgfpathlineto{\pgfqpoint{1.449119in}{2.878965in}}%
\pgfpathlineto{\pgfqpoint{1.445326in}{2.884973in}}%
\pgfpathlineto{\pgfqpoint{1.445164in}{2.885225in}}%
\pgfpathlineto{\pgfqpoint{1.441255in}{2.891486in}}%
\pgfpathlineto{\pgfqpoint{1.439065in}{2.895102in}}%
\pgfpathlineto{\pgfqpoint{1.437442in}{2.897746in}}%
\pgfpathlineto{\pgfqpoint{1.433710in}{2.904007in}}%
\pgfpathlineto{\pgfqpoint{1.432805in}{2.905569in}}%
\pgfpathlineto{\pgfqpoint{1.430044in}{2.910267in}}%
\pgfpathlineto{\pgfqpoint{1.426544in}{2.916424in}}%
\pgfpathlineto{\pgfqpoint{1.426485in}{2.916528in}}%
\pgfpathlineto{\pgfqpoint{1.422973in}{2.922788in}}%
\pgfpathlineto{\pgfqpoint{1.420284in}{2.927753in}}%
\pgfpathlineto{\pgfqpoint{1.419571in}{2.929048in}}%
\pgfpathlineto{\pgfqpoint{1.416235in}{2.935309in}}%
\pgfpathlineto{\pgfqpoint{1.414023in}{2.939609in}}%
\pgfpathlineto{\pgfqpoint{1.412999in}{2.941569in}}%
\pgfpathlineto{\pgfqpoint{1.409839in}{2.947830in}}%
\pgfpathlineto{\pgfqpoint{1.407763in}{2.952098in}}%
\pgfpathlineto{\pgfqpoint{1.406778in}{2.954090in}}%
\pgfpathlineto{\pgfqpoint{1.403794in}{2.960351in}}%
\pgfpathlineto{\pgfqpoint{1.401503in}{2.965357in}}%
\pgfpathlineto{\pgfqpoint{1.400918in}{2.966611in}}%
\pgfpathlineto{\pgfqpoint{1.398116in}{2.972871in}}%
\pgfpathlineto{\pgfqpoint{1.395432in}{2.979132in}}%
\pgfpathlineto{\pgfqpoint{1.395242in}{2.979596in}}%
\pgfpathlineto{\pgfqpoint{1.392820in}{2.985392in}}%
\pgfpathlineto{\pgfqpoint{1.390323in}{2.991653in}}%
\pgfpathlineto{\pgfqpoint{1.388982in}{2.995186in}}%
\pgfpathlineto{\pgfqpoint{1.387926in}{2.997913in}}%
\pgfpathlineto{\pgfqpoint{1.385624in}{3.004174in}}%
\pgfpathlineto{\pgfqpoint{1.383443in}{3.010434in}}%
\pgfpathlineto{\pgfqpoint{1.382721in}{3.012633in}}%
\pgfpathlineto{\pgfqpoint{1.381360in}{3.016694in}}%
\pgfpathlineto{\pgfqpoint{1.379388in}{3.022955in}}%
\pgfpathlineto{\pgfqpoint{1.377541in}{3.029215in}}%
\pgfpathlineto{\pgfqpoint{1.376461in}{3.033151in}}%
\pgfpathlineto{\pgfqpoint{1.375809in}{3.035476in}}%
\pgfpathlineto{\pgfqpoint{1.374190in}{3.041736in}}%
\pgfpathlineto{\pgfqpoint{1.372704in}{3.047997in}}%
\pgfpathlineto{\pgfqpoint{1.371352in}{3.054257in}}%
\pgfpathlineto{\pgfqpoint{1.370200in}{3.060198in}}%
\pgfpathlineto{\pgfqpoint{1.370137in}{3.060518in}}%
\pgfpathlineto{\pgfqpoint{1.369049in}{3.066778in}}%
\pgfpathlineto{\pgfqpoint{1.368109in}{3.073038in}}%
\pgfpathlineto{\pgfqpoint{1.367322in}{3.079299in}}%
\pgfpathlineto{\pgfqpoint{1.366692in}{3.085559in}}%
\pgfpathlineto{\pgfqpoint{1.366226in}{3.091820in}}%
\pgfpathlineto{\pgfqpoint{1.365932in}{3.098080in}}%
\pgfpathlineto{\pgfqpoint{1.365816in}{3.104341in}}%
\pgfpathlineto{\pgfqpoint{1.365888in}{3.110601in}}%
\pgfpathlineto{\pgfqpoint{1.366158in}{3.116861in}}%
\pgfpathlineto{\pgfqpoint{1.366637in}{3.123122in}}%
\pgfpathlineto{\pgfqpoint{1.367339in}{3.129382in}}%
\pgfpathlineto{\pgfqpoint{1.368276in}{3.135643in}}%
\pgfpathlineto{\pgfqpoint{1.369467in}{3.141903in}}%
\pgfpathlineto{\pgfqpoint{1.370200in}{3.145088in}}%
\pgfpathlineto{\pgfqpoint{1.370925in}{3.148164in}}%
\pgfpathlineto{\pgfqpoint{1.372672in}{3.154424in}}%
\pgfpathlineto{\pgfqpoint{1.374734in}{3.160684in}}%
\pgfpathlineto{\pgfqpoint{1.376461in}{3.165221in}}%
\pgfpathlineto{\pgfqpoint{1.377137in}{3.166945in}}%
\pgfpathlineto{\pgfqpoint{1.379915in}{3.173205in}}%
\pgfpathlineto{\pgfqpoint{1.382721in}{3.178744in}}%
\pgfpathlineto{\pgfqpoint{1.383101in}{3.179466in}}%
\pgfpathlineto{\pgfqpoint{1.386750in}{3.185726in}}%
\pgfpathlineto{\pgfqpoint{1.388982in}{3.189155in}}%
\pgfpathlineto{\pgfqpoint{1.390910in}{3.191987in}}%
\pgfpathlineto{\pgfqpoint{1.395242in}{3.197732in}}%
\pgfpathlineto{\pgfqpoint{1.395650in}{3.198247in}}%
\pgfpathlineto{\pgfqpoint{1.401068in}{3.204508in}}%
\pgfpathlineto{\pgfqpoint{1.401503in}{3.204971in}}%
\pgfpathlineto{\pgfqpoint{1.407275in}{3.210768in}}%
\pgfpathlineto{\pgfqpoint{1.407763in}{3.211222in}}%
\pgfpathlineto{\pgfqpoint{1.414023in}{3.216698in}}%
\pgfpathlineto{\pgfqpoint{1.414427in}{3.217028in}}%
\pgfpathlineto{\pgfqpoint{1.420284in}{3.221526in}}%
\pgfpathlineto{\pgfqpoint{1.422758in}{3.223289in}}%
\pgfpathlineto{\pgfqpoint{1.426544in}{3.225833in}}%
\pgfpathlineto{\pgfqpoint{1.432551in}{3.229549in}}%
\pgfpathlineto{\pgfqpoint{1.432805in}{3.229698in}}%
\pgfpathlineto{\pgfqpoint{1.439065in}{3.233145in}}%
\pgfpathlineto{\pgfqpoint{1.444383in}{3.235810in}}%
\pgfpathlineto{\pgfqpoint{1.445326in}{3.236261in}}%
\pgfpathlineto{\pgfqpoint{1.451586in}{3.239054in}}%
\pgfpathlineto{\pgfqpoint{1.457846in}{3.241578in}}%
\pgfpathlineto{\pgfqpoint{1.459172in}{3.242070in}}%
\pgfpathlineto{\pgfqpoint{1.464107in}{3.243841in}}%
\pgfpathlineto{\pgfqpoint{1.470367in}{3.245875in}}%
\pgfpathlineto{\pgfqpoint{1.476628in}{3.247699in}}%
\pgfpathlineto{\pgfqpoint{1.479019in}{3.248331in}}%
\pgfpathlineto{\pgfqpoint{1.482888in}{3.249325in}}%
\pgfpathlineto{\pgfqpoint{1.489149in}{3.250769in}}%
\pgfpathlineto{\pgfqpoint{1.495409in}{3.252043in}}%
\pgfpathlineto{\pgfqpoint{1.501669in}{3.253158in}}%
\pgfpathlineto{\pgfqpoint{1.507930in}{3.254122in}}%
\pgfpathlineto{\pgfqpoint{1.511465in}{3.254591in}}%
\pgfpathlineto{\pgfqpoint{1.514190in}{3.254947in}}%
\pgfpathlineto{\pgfqpoint{1.520451in}{3.255640in}}%
\pgfpathlineto{\pgfqpoint{1.526711in}{3.256207in}}%
\pgfpathlineto{\pgfqpoint{1.532972in}{3.256654in}}%
\pgfpathlineto{\pgfqpoint{1.539232in}{3.256986in}}%
\pgfpathlineto{\pgfqpoint{1.545492in}{3.257210in}}%
\pgfpathlineto{\pgfqpoint{1.551753in}{3.257329in}}%
\pgfpathlineto{\pgfqpoint{1.558013in}{3.257348in}}%
\pgfpathlineto{\pgfqpoint{1.564274in}{3.257272in}}%
\pgfpathlineto{\pgfqpoint{1.570534in}{3.257103in}}%
\pgfpathlineto{\pgfqpoint{1.576795in}{3.256845in}}%
\pgfpathlineto{\pgfqpoint{1.583055in}{3.256502in}}%
\pgfpathlineto{\pgfqpoint{1.589316in}{3.256075in}}%
\pgfpathlineto{\pgfqpoint{1.595576in}{3.255567in}}%
\pgfpathlineto{\pgfqpoint{1.601836in}{3.254981in}}%
\pgfpathlineto{\pgfqpoint{1.605502in}{3.254591in}}%
\pgfpathlineto{\pgfqpoint{1.608097in}{3.254322in}}%
\pgfpathlineto{\pgfqpoint{1.614357in}{3.253597in}}%
\pgfpathlineto{\pgfqpoint{1.620618in}{3.252800in}}%
\pgfpathlineto{\pgfqpoint{1.626878in}{3.251933in}}%
\pgfpathlineto{\pgfqpoint{1.633139in}{3.250997in}}%
\pgfpathlineto{\pgfqpoint{1.639399in}{3.249993in}}%
\pgfpathlineto{\pgfqpoint{1.645659in}{3.248921in}}%
\pgfpathlineto{\pgfqpoint{1.648900in}{3.248331in}}%
\pgfpathlineto{\pgfqpoint{1.651920in}{3.247794in}}%
\pgfpathlineto{\pgfqpoint{1.658180in}{3.246614in}}%
\pgfpathlineto{\pgfqpoint{1.664441in}{3.245372in}}%
\pgfpathlineto{\pgfqpoint{1.670701in}{3.244066in}}%
\pgfpathlineto{\pgfqpoint{1.676962in}{3.242696in}}%
\pgfpathlineto{\pgfqpoint{1.679694in}{3.242070in}}%
\pgfpathlineto{\pgfqpoint{1.683222in}{3.241281in}}%
\pgfpathlineto{\pgfqpoint{1.689482in}{3.239820in}}%
\pgfpathlineto{\pgfqpoint{1.695743in}{3.238297in}}%
\pgfpathlineto{\pgfqpoint{1.702003in}{3.236712in}}%
\pgfpathlineto{\pgfqpoint{1.705436in}{3.235810in}}%
\pgfpathlineto{\pgfqpoint{1.708264in}{3.235083in}}%
\pgfpathlineto{\pgfqpoint{1.714524in}{3.233416in}}%
\pgfpathlineto{\pgfqpoint{1.720785in}{3.231688in}}%
\pgfpathlineto{\pgfqpoint{1.727045in}{3.229898in}}%
\pgfpathlineto{\pgfqpoint{1.728227in}{3.229549in}}%
\pgfpathlineto{\pgfqpoint{1.733306in}{3.228084in}}%
\pgfpathlineto{\pgfqpoint{1.739566in}{3.226219in}}%
\pgfpathlineto{\pgfqpoint{1.745826in}{3.224291in}}%
\pgfpathlineto{\pgfqpoint{1.748986in}{3.223289in}}%
\pgfpathlineto{\pgfqpoint{1.752087in}{3.222327in}}%
\pgfpathlineto{\pgfqpoint{1.758347in}{3.220328in}}%
\pgfpathlineto{\pgfqpoint{1.764608in}{3.218267in}}%
\pgfpathlineto{\pgfqpoint{1.768262in}{3.217028in}}%
\pgfpathlineto{\pgfqpoint{1.770868in}{3.216165in}}%
\pgfpathlineto{\pgfqpoint{1.777129in}{3.214035in}}%
\pgfpathlineto{\pgfqpoint{1.783389in}{3.211841in}}%
\pgfpathlineto{\pgfqpoint{1.786372in}{3.210768in}}%
\pgfpathlineto{\pgfqpoint{1.789649in}{3.209615in}}%
\pgfpathlineto{\pgfqpoint{1.795910in}{3.207354in}}%
\pgfpathlineto{\pgfqpoint{1.802170in}{3.205028in}}%
\pgfpathlineto{\pgfqpoint{1.803540in}{3.204508in}}%
\pgfpathlineto{\pgfqpoint{1.808431in}{3.202688in}}%
\pgfpathlineto{\pgfqpoint{1.814691in}{3.200297in}}%
\pgfpathlineto{\pgfqpoint{1.819915in}{3.198247in}}%
\pgfpathlineto{\pgfqpoint{1.820952in}{3.197849in}}%
\pgfpathlineto{\pgfqpoint{1.827212in}{3.195393in}}%
\pgfpathlineto{\pgfqpoint{1.833472in}{3.192869in}}%
\pgfpathlineto{\pgfqpoint{1.835613in}{3.191987in}}%
\pgfpathlineto{\pgfqpoint{1.839733in}{3.190323in}}%
\pgfpathlineto{\pgfqpoint{1.845993in}{3.187735in}}%
\pgfpathlineto{\pgfqpoint{1.850730in}{3.185726in}}%
\pgfpathlineto{\pgfqpoint{1.852254in}{3.185093in}}%
\pgfpathlineto{\pgfqpoint{1.858514in}{3.182440in}}%
\pgfpathlineto{\pgfqpoint{1.864775in}{3.179714in}}%
\pgfpathlineto{\pgfqpoint{1.865334in}{3.179466in}}%
\pgfpathlineto{\pgfqpoint{1.871035in}{3.176989in}}%
\pgfpathlineto{\pgfqpoint{1.877295in}{3.174197in}}%
\pgfpathlineto{\pgfqpoint{1.879476in}{3.173205in}}%
\pgfpathlineto{\pgfqpoint{1.883556in}{3.171385in}}%
\pgfpathlineto{\pgfqpoint{1.889816in}{3.168527in}}%
\pgfpathlineto{\pgfqpoint{1.893208in}{3.166945in}}%
\pgfpathlineto{\pgfqpoint{1.896077in}{3.165632in}}%
\pgfpathlineto{\pgfqpoint{1.902337in}{3.162707in}}%
\pgfpathlineto{\pgfqpoint{1.906568in}{3.160684in}}%
\pgfpathlineto{\pgfqpoint{1.908598in}{3.159731in}}%
\pgfpathlineto{\pgfqpoint{1.914858in}{3.156739in}}%
\pgfpathlineto{\pgfqpoint{1.919585in}{3.154424in}}%
\pgfpathlineto{\pgfqpoint{1.921119in}{3.153686in}}%
\pgfpathlineto{\pgfqpoint{1.927379in}{3.150624in}}%
\pgfpathlineto{\pgfqpoint{1.932289in}{3.148164in}}%
\pgfpathlineto{\pgfqpoint{1.933639in}{3.147498in}}%
\pgfpathlineto{\pgfqpoint{1.939900in}{3.144365in}}%
\pgfpathlineto{\pgfqpoint{1.944703in}{3.141903in}}%
\pgfpathlineto{\pgfqpoint{1.946160in}{3.141169in}}%
\pgfpathlineto{\pgfqpoint{1.952421in}{3.137963in}}%
\pgfpathlineto{\pgfqpoint{1.956848in}{3.135643in}}%
\pgfpathlineto{\pgfqpoint{1.958681in}{3.134698in}}%
\pgfpathlineto{\pgfqpoint{1.964942in}{3.131417in}}%
\pgfpathlineto{\pgfqpoint{1.968742in}{3.129382in}}%
\pgfpathlineto{\pgfqpoint{1.971202in}{3.128086in}}%
\pgfpathlineto{\pgfqpoint{1.977462in}{3.124728in}}%
\pgfpathlineto{\pgfqpoint{1.980400in}{3.123122in}}%
\pgfpathlineto{\pgfqpoint{1.983723in}{3.121333in}}%
\pgfpathlineto{\pgfqpoint{1.989983in}{3.117894in}}%
\pgfpathlineto{\pgfqpoint{1.991835in}{3.116861in}}%
\pgfpathlineto{\pgfqpoint{1.996244in}{3.114437in}}%
\pgfpathlineto{\pgfqpoint{2.002504in}{3.110916in}}%
\pgfpathlineto{\pgfqpoint{2.003057in}{3.110601in}}%
\pgfpathlineto{\pgfqpoint{2.008765in}{3.107399in}}%
\pgfpathlineto{\pgfqpoint{2.014086in}{3.104341in}}%
\pgfpathlineto{\pgfqpoint{2.015025in}{3.103808in}}%
\pgfpathlineto{\pgfqpoint{2.021285in}{3.100215in}}%
\pgfpathlineto{\pgfqpoint{2.024929in}{3.098080in}}%
\pgfpathlineto{\pgfqpoint{2.027546in}{3.096567in}}%
\pgfpathlineto{\pgfqpoint{2.033806in}{3.092884in}}%
\pgfpathlineto{\pgfqpoint{2.035589in}{3.091820in}}%
\pgfpathlineto{\pgfqpoint{2.040067in}{3.089179in}}%
\pgfpathlineto{\pgfqpoint{2.046072in}{3.085559in}}%
\pgfpathlineto{\pgfqpoint{2.046327in}{3.085407in}}%
\pgfpathlineto{\pgfqpoint{2.052588in}{3.081643in}}%
\pgfpathlineto{\pgfqpoint{2.056405in}{3.079299in}}%
\pgfpathlineto{\pgfqpoint{2.058848in}{3.077816in}}%
\pgfpathlineto{\pgfqpoint{2.065109in}{3.073954in}}%
\pgfpathlineto{\pgfqpoint{2.066573in}{3.073038in}}%
\pgfpathlineto{\pgfqpoint{2.071369in}{3.070073in}}%
\pgfpathlineto{\pgfqpoint{2.076590in}{3.066778in}}%
\pgfpathlineto{\pgfqpoint{2.077629in}{3.066129in}}%
\pgfpathlineto{\pgfqpoint{2.083890in}{3.062174in}}%
\pgfpathlineto{\pgfqpoint{2.086468in}{3.060518in}}%
\pgfpathlineto{\pgfqpoint{2.090150in}{3.058176in}}%
\pgfpathlineto{\pgfqpoint{2.096194in}{3.054257in}}%
\pgfpathlineto{\pgfqpoint{2.096411in}{3.054118in}}%
\pgfpathlineto{\pgfqpoint{2.102671in}{3.050063in}}%
\pgfpathlineto{\pgfqpoint{2.105804in}{3.047997in}}%
\pgfpathlineto{\pgfqpoint{2.108932in}{3.045952in}}%
\pgfpathlineto{\pgfqpoint{2.115192in}{3.041784in}}%
\pgfpathlineto{\pgfqpoint{2.115263in}{3.041736in}}%
\pgfpathlineto{\pgfqpoint{2.121452in}{3.037620in}}%
\pgfpathlineto{\pgfqpoint{2.124619in}{3.035476in}}%
\pgfpathlineto{\pgfqpoint{2.127713in}{3.033397in}}%
\pgfpathlineto{\pgfqpoint{2.133830in}{3.029215in}}%
\pgfpathlineto{\pgfqpoint{2.133973in}{3.029118in}}%
\pgfpathlineto{\pgfqpoint{2.140234in}{3.024839in}}%
\pgfpathlineto{\pgfqpoint{2.142945in}{3.022955in}}%
\pgfpathlineto{\pgfqpoint{2.146494in}{3.020506in}}%
\pgfpathlineto{\pgfqpoint{2.151928in}{3.016694in}}%
\pgfpathlineto{\pgfqpoint{2.152755in}{3.016118in}}%
\pgfpathlineto{\pgfqpoint{2.159015in}{3.011714in}}%
\pgfpathlineto{\pgfqpoint{2.160811in}{3.010434in}}%
\pgfpathlineto{\pgfqpoint{2.165275in}{3.007271in}}%
\pgfpathlineto{\pgfqpoint{2.169580in}{3.004174in}}%
\pgfpathlineto{\pgfqpoint{2.171536in}{3.002774in}}%
\pgfpathlineto{\pgfqpoint{2.177796in}{2.998234in}}%
\pgfpathlineto{\pgfqpoint{2.178235in}{2.997913in}}%
\pgfpathlineto{\pgfqpoint{2.184057in}{2.993681in}}%
\pgfpathlineto{\pgfqpoint{2.186806in}{2.991653in}}%
\pgfpathlineto{\pgfqpoint{2.190317in}{2.989075in}}%
\pgfpathlineto{\pgfqpoint{2.195262in}{2.985392in}}%
\pgfpathlineto{\pgfqpoint{2.196578in}{2.984416in}}%
\pgfpathlineto{\pgfqpoint{2.202838in}{2.979723in}}%
\pgfpathlineto{\pgfqpoint{2.203621in}{2.979132in}}%
\pgfpathlineto{\pgfqpoint{2.209098in}{2.975006in}}%
\pgfpathlineto{\pgfqpoint{2.211896in}{2.972871in}}%
\pgfpathlineto{\pgfqpoint{2.215359in}{2.970237in}}%
\pgfpathlineto{\pgfqpoint{2.220066in}{2.966611in}}%
\pgfpathlineto{\pgfqpoint{2.221619in}{2.965418in}}%
\pgfpathlineto{\pgfqpoint{2.227880in}{2.960554in}}%
\pgfpathlineto{\pgfqpoint{2.228140in}{2.960351in}}%
\pgfpathlineto{\pgfqpoint{2.234140in}{2.955673in}}%
\pgfpathlineto{\pgfqpoint{2.236147in}{2.954090in}}%
\pgfpathlineto{\pgfqpoint{2.240401in}{2.950741in}}%
\pgfpathlineto{\pgfqpoint{2.244058in}{2.947830in}}%
\pgfpathlineto{\pgfqpoint{2.246661in}{2.945760in}}%
\pgfpathlineto{\pgfqpoint{2.251877in}{2.941569in}}%
\pgfpathlineto{\pgfqpoint{2.252922in}{2.940730in}}%
\pgfpathlineto{\pgfqpoint{2.259182in}{2.935662in}}%
\pgfpathlineto{\pgfqpoint{2.259616in}{2.935309in}}%
\pgfpathlineto{\pgfqpoint{2.265442in}{2.930568in}}%
\pgfpathlineto{\pgfqpoint{2.267293in}{2.929048in}}%
\pgfpathlineto{\pgfqpoint{2.271703in}{2.925426in}}%
\pgfpathlineto{\pgfqpoint{2.274887in}{2.922788in}}%
\pgfpathlineto{\pgfqpoint{2.277963in}{2.920237in}}%
\pgfpathlineto{\pgfqpoint{2.282402in}{2.916528in}}%
\pgfpathlineto{\pgfqpoint{2.284224in}{2.915003in}}%
\pgfpathlineto{\pgfqpoint{2.289842in}{2.910267in}}%
\pgfpathlineto{\pgfqpoint{2.290484in}{2.909725in}}%
\pgfpathlineto{\pgfqpoint{2.296745in}{2.904414in}}%
\pgfpathlineto{\pgfqpoint{2.297222in}{2.904007in}}%
\pgfpathlineto{\pgfqpoint{2.303005in}{2.899073in}}%
\pgfpathlineto{\pgfqpoint{2.304551in}{2.897746in}}%
\pgfpathlineto{\pgfqpoint{2.309265in}{2.893692in}}%
\pgfpathlineto{\pgfqpoint{2.311817in}{2.891486in}}%
\pgfpathlineto{\pgfqpoint{2.315526in}{2.888271in}}%
\pgfpathlineto{\pgfqpoint{2.319025in}{2.885225in}}%
\pgfpathlineto{\pgfqpoint{2.321786in}{2.882814in}}%
\pgfpathlineto{\pgfqpoint{2.326180in}{2.878965in}}%
\pgfpathlineto{\pgfqpoint{2.328047in}{2.877324in}}%
\pgfpathlineto{\pgfqpoint{2.333289in}{2.872705in}}%
\pgfpathlineto{\pgfqpoint{2.334307in}{2.871804in}}%
\pgfpathlineto{\pgfqpoint{2.340358in}{2.866444in}}%
\pgfpathlineto{\pgfqpoint{2.340568in}{2.866257in}}%
\pgfpathlineto{\pgfqpoint{2.346828in}{2.860699in}}%
\pgfpathlineto{\pgfqpoint{2.347410in}{2.860184in}}%
\pgfpathlineto{\pgfqpoint{2.353088in}{2.855127in}}%
\pgfpathlineto{\pgfqpoint{2.354443in}{2.853923in}}%
\pgfpathlineto{\pgfqpoint{2.359349in}{2.849543in}}%
\pgfpathlineto{\pgfqpoint{2.361461in}{2.847663in}}%
\pgfpathlineto{\pgfqpoint{2.365609in}{2.843953in}}%
\pgfpathlineto{\pgfqpoint{2.368475in}{2.841402in}}%
\pgfpathlineto{\pgfqpoint{2.371870in}{2.838365in}}%
\pgfpathlineto{\pgfqpoint{2.375495in}{2.835142in}}%
\pgfpathlineto{\pgfqpoint{2.378130in}{2.832787in}}%
\pgfpathlineto{\pgfqpoint{2.382536in}{2.828881in}}%
\pgfpathlineto{\pgfqpoint{2.384391in}{2.827229in}}%
\pgfpathlineto{\pgfqpoint{2.389615in}{2.822621in}}%
\pgfpathlineto{\pgfqpoint{2.390651in}{2.821702in}}%
\pgfpathlineto{\pgfqpoint{2.396749in}{2.816361in}}%
\pgfpathlineto{\pgfqpoint{2.396912in}{2.816218in}}%
\pgfpathlineto{\pgfqpoint{2.403172in}{2.810802in}}%
\pgfpathlineto{\pgfqpoint{2.403997in}{2.810100in}}%
\pgfpathlineto{\pgfqpoint{2.409432in}{2.805459in}}%
\pgfpathlineto{\pgfqpoint{2.411367in}{2.803840in}}%
\pgfpathlineto{\pgfqpoint{2.415693in}{2.800204in}}%
\pgfpathlineto{\pgfqpoint{2.418889in}{2.797579in}}%
\pgfpathlineto{\pgfqpoint{2.421953in}{2.795052in}}%
\pgfpathlineto{\pgfqpoint{2.426607in}{2.791319in}}%
\pgfpathlineto{\pgfqpoint{2.428214in}{2.790025in}}%
\pgfpathlineto{\pgfqpoint{2.434474in}{2.785141in}}%
\pgfpathlineto{\pgfqpoint{2.434585in}{2.785058in}}%
\pgfpathlineto{\pgfqpoint{2.440735in}{2.780447in}}%
\pgfpathlineto{\pgfqpoint{2.443029in}{2.778798in}}%
\pgfpathlineto{\pgfqpoint{2.446995in}{2.775940in}}%
\pgfpathlineto{\pgfqpoint{2.451958in}{2.772538in}}%
\pgfpathlineto{\pgfqpoint{2.453255in}{2.771646in}}%
\pgfpathlineto{\pgfqpoint{2.459516in}{2.767609in}}%
\pgfpathlineto{\pgfqpoint{2.461734in}{2.766277in}}%
\pgfpathlineto{\pgfqpoint{2.465776in}{2.763849in}}%
\pgfpathlineto{\pgfqpoint{2.472037in}{2.760383in}}%
\pgfpathlineto{\pgfqpoint{2.472772in}{2.760017in}}%
\pgfpathlineto{\pgfqpoint{2.478297in}{2.757270in}}%
\pgfpathlineto{\pgfqpoint{2.484558in}{2.754510in}}%
\pgfpathlineto{\pgfqpoint{2.486562in}{2.753756in}}%
\pgfpathlineto{\pgfqpoint{2.490818in}{2.752160in}}%
\pgfpathlineto{\pgfqpoint{2.497078in}{2.750241in}}%
\pgfpathlineto{\pgfqpoint{2.503339in}{2.748781in}}%
\pgfpathlineto{\pgfqpoint{2.509599in}{2.747818in}}%
\pgfpathlineto{\pgfqpoint{2.514336in}{2.747496in}}%
\pgfpathlineto{\pgfqpoint{2.515860in}{2.747393in}}%
\pgfpathlineto{\pgfqpoint{2.520386in}{2.747496in}}%
\pgfpathlineto{\pgfqpoint{2.522120in}{2.747536in}}%
\pgfpathlineto{\pgfqpoint{2.528381in}{2.748287in}}%
\pgfpathlineto{\pgfqpoint{2.534641in}{2.749674in}}%
\pgfpathlineto{\pgfqpoint{2.540902in}{2.751724in}}%
\pgfpathlineto{\pgfqpoint{2.545558in}{2.753756in}}%
\pgfpathlineto{\pgfqpoint{2.547162in}{2.754470in}}%
\pgfpathlineto{\pgfqpoint{2.553422in}{2.757960in}}%
\pgfpathlineto{\pgfqpoint{2.556483in}{2.760017in}}%
\pgfpathlineto{\pgfqpoint{2.559683in}{2.762204in}}%
\pgfpathlineto{\pgfqpoint{2.564775in}{2.766277in}}%
\pgfpathlineto{\pgfqpoint{2.565943in}{2.767226in}}%
\pgfpathlineto{\pgfqpoint{2.571656in}{2.772538in}}%
\pgfpathlineto{\pgfqpoint{2.572204in}{2.773053in}}%
\pgfpathlineto{\pgfqpoint{2.577623in}{2.778798in}}%
\pgfpathlineto{\pgfqpoint{2.578464in}{2.779700in}}%
\pgfpathlineto{\pgfqpoint{2.582969in}{2.785058in}}%
\pgfpathlineto{\pgfqpoint{2.584725in}{2.787164in}}%
\pgfpathlineto{\pgfqpoint{2.587879in}{2.791319in}}%
\pgfpathlineto{\pgfqpoint{2.590985in}{2.795429in}}%
\pgfpathlineto{\pgfqpoint{2.592478in}{2.797579in}}%
\pgfpathlineto{\pgfqpoint{2.596811in}{2.803840in}}%
\pgfpathlineto{\pgfqpoint{2.597245in}{2.804471in}}%
\pgfpathlineto{\pgfqpoint{2.600840in}{2.810100in}}%
\pgfpathlineto{\pgfqpoint{2.603506in}{2.814269in}}%
\pgfpathlineto{\pgfqpoint{2.604754in}{2.816361in}}%
\pgfpathlineto{\pgfqpoint{2.608493in}{2.822621in}}%
\pgfpathlineto{\pgfqpoint{2.609766in}{2.824751in}}%
\pgfpathlineto{\pgfqpoint{2.612086in}{2.828881in}}%
\pgfpathlineto{\pgfqpoint{2.615628in}{2.835142in}}%
\pgfpathlineto{\pgfqpoint{2.616027in}{2.835846in}}%
\pgfpathlineto{\pgfqpoint{2.618997in}{2.841402in}}%
\pgfpathlineto{\pgfqpoint{2.622287in}{2.847481in}}%
\pgfpathlineto{\pgfqpoint{2.622380in}{2.847663in}}%
\pgfpathlineto{\pgfqpoint{2.625600in}{2.853923in}}%
\pgfpathlineto{\pgfqpoint{2.628548in}{2.859566in}}%
\pgfpathlineto{\pgfqpoint{2.628855in}{2.860184in}}%
\pgfpathlineto{\pgfqpoint{2.631985in}{2.866444in}}%
\pgfpathlineto{\pgfqpoint{2.634808in}{2.871993in}}%
\pgfpathlineto{\pgfqpoint{2.635155in}{2.872705in}}%
\pgfpathlineto{\pgfqpoint{2.638220in}{2.878965in}}%
\pgfpathlineto{\pgfqpoint{2.641068in}{2.884663in}}%
\pgfpathlineto{\pgfqpoint{2.641338in}{2.885225in}}%
\pgfpathlineto{\pgfqpoint{2.644363in}{2.891486in}}%
\pgfpathlineto{\pgfqpoint{2.647329in}{2.897482in}}%
\pgfpathlineto{\pgfqpoint{2.647455in}{2.897746in}}%
\pgfpathlineto{\pgfqpoint{2.650458in}{2.904007in}}%
\pgfpathlineto{\pgfqpoint{2.653541in}{2.910267in}}%
\pgfpathlineto{\pgfqpoint{2.653589in}{2.910364in}}%
\pgfpathlineto{\pgfqpoint{2.656543in}{2.916528in}}%
\pgfpathlineto{\pgfqpoint{2.659628in}{2.922788in}}%
\pgfpathlineto{\pgfqpoint{2.659850in}{2.923234in}}%
\pgfpathlineto{\pgfqpoint{2.662651in}{2.929048in}}%
\pgfpathlineto{\pgfqpoint{2.665753in}{2.935309in}}%
\pgfpathlineto{\pgfqpoint{2.666110in}{2.936022in}}%
\pgfpathlineto{\pgfqpoint{2.668809in}{2.941569in}}%
\pgfpathlineto{\pgfqpoint{2.671943in}{2.947830in}}%
\pgfpathlineto{\pgfqpoint{2.672371in}{2.948674in}}%
\pgfpathlineto{\pgfqpoint{2.675042in}{2.954090in}}%
\pgfpathlineto{\pgfqpoint{2.678222in}{2.960351in}}%
\pgfpathlineto{\pgfqpoint{2.678631in}{2.961148in}}%
\pgfpathlineto{\pgfqpoint{2.681372in}{2.966611in}}%
\pgfpathlineto{\pgfqpoint{2.684609in}{2.972871in}}%
\pgfpathlineto{\pgfqpoint{2.684891in}{2.973411in}}%
\pgfpathlineto{\pgfqpoint{2.687818in}{2.979132in}}%
\pgfpathlineto{\pgfqpoint{2.691127in}{2.985392in}}%
\pgfpathlineto{\pgfqpoint{2.691152in}{2.985440in}}%
\pgfpathlineto{\pgfqpoint{2.694401in}{2.991653in}}%
\pgfpathlineto{\pgfqpoint{2.697412in}{2.997229in}}%
\pgfpathlineto{\pgfqpoint{2.697775in}{2.997913in}}%
\pgfpathlineto{\pgfqpoint{2.701137in}{3.004174in}}%
\pgfpathlineto{\pgfqpoint{2.703673in}{3.008762in}}%
\pgfpathlineto{\pgfqpoint{2.704583in}{3.010434in}}%
\pgfpathlineto{\pgfqpoint{2.708045in}{3.016694in}}%
\pgfpathlineto{\pgfqpoint{2.709933in}{3.020029in}}%
\pgfpathlineto{\pgfqpoint{2.711568in}{3.022955in}}%
\pgfpathlineto{\pgfqpoint{2.715143in}{3.029215in}}%
\pgfpathlineto{\pgfqpoint{2.716194in}{3.031025in}}%
\pgfpathlineto{\pgfqpoint{2.718748in}{3.035476in}}%
\pgfpathlineto{\pgfqpoint{2.722449in}{3.041736in}}%
\pgfpathlineto{\pgfqpoint{2.722454in}{3.041745in}}%
\pgfpathlineto{\pgfqpoint{2.726143in}{3.047997in}}%
\pgfpathlineto{\pgfqpoint{2.728715in}{3.052232in}}%
\pgfpathlineto{\pgfqpoint{2.729934in}{3.054257in}}%
\pgfpathlineto{\pgfqpoint{2.733772in}{3.060518in}}%
\pgfpathlineto{\pgfqpoint{2.734975in}{3.062446in}}%
\pgfpathlineto{\pgfqpoint{2.737659in}{3.066778in}}%
\pgfpathlineto{\pgfqpoint{2.741235in}{3.072400in}}%
\pgfpathlineto{\pgfqpoint{2.741640in}{3.073038in}}%
\pgfpathlineto{\pgfqpoint{2.745647in}{3.079299in}}%
\pgfpathlineto{\pgfqpoint{2.747496in}{3.082126in}}%
\pgfpathlineto{\pgfqpoint{2.749734in}{3.085559in}}%
\pgfpathlineto{\pgfqpoint{2.753756in}{3.091583in}}%
\pgfpathlineto{\pgfqpoint{2.753914in}{3.091820in}}%
\pgfpathlineto{\pgfqpoint{2.758124in}{3.098080in}}%
\pgfpathlineto{\pgfqpoint{2.760017in}{3.100836in}}%
\pgfpathlineto{\pgfqpoint{2.762424in}{3.104341in}}%
\pgfpathlineto{\pgfqpoint{2.766277in}{3.109832in}}%
\pgfpathlineto{\pgfqpoint{2.766818in}{3.110601in}}%
\pgfpathlineto{\pgfqpoint{2.771266in}{3.116861in}}%
\pgfpathlineto{\pgfqpoint{2.772538in}{3.118622in}}%
\pgfpathlineto{\pgfqpoint{2.775797in}{3.123122in}}%
\pgfpathlineto{\pgfqpoint{2.778798in}{3.127187in}}%
\pgfpathlineto{\pgfqpoint{2.780426in}{3.129382in}}%
\pgfpathlineto{\pgfqpoint{2.785058in}{3.135519in}}%
\pgfpathlineto{\pgfqpoint{2.785152in}{3.135643in}}%
\pgfpathlineto{\pgfqpoint{2.789942in}{3.141903in}}%
\pgfpathlineto{\pgfqpoint{2.791319in}{3.143675in}}%
\pgfpathlineto{\pgfqpoint{2.794834in}{3.148164in}}%
\pgfpathlineto{\pgfqpoint{2.797579in}{3.151615in}}%
\pgfpathlineto{\pgfqpoint{2.799834in}{3.154424in}}%
\pgfpathlineto{\pgfqpoint{2.803840in}{3.159345in}}%
\pgfpathlineto{\pgfqpoint{2.804942in}{3.160684in}}%
\pgfpathlineto{\pgfqpoint{2.810100in}{3.166871in}}%
\pgfpathlineto{\pgfqpoint{2.810162in}{3.166945in}}%
\pgfpathlineto{\pgfqpoint{2.815480in}{3.173205in}}%
\pgfpathlineto{\pgfqpoint{2.816361in}{3.174231in}}%
\pgfpathlineto{\pgfqpoint{2.820920in}{3.179466in}}%
\pgfpathlineto{\pgfqpoint{2.822621in}{3.181399in}}%
\pgfpathlineto{\pgfqpoint{2.826490in}{3.185726in}}%
\pgfpathlineto{\pgfqpoint{2.828881in}{3.188377in}}%
\pgfpathlineto{\pgfqpoint{2.832195in}{3.191987in}}%
\pgfpathlineto{\pgfqpoint{2.835142in}{3.195171in}}%
\pgfpathlineto{\pgfqpoint{2.838042in}{3.198247in}}%
\pgfpathlineto{\pgfqpoint{2.841402in}{3.201786in}}%
\pgfpathlineto{\pgfqpoint{2.844040in}{3.204508in}}%
\pgfpathlineto{\pgfqpoint{2.847663in}{3.208224in}}%
\pgfpathlineto{\pgfqpoint{2.850198in}{3.210768in}}%
\pgfpathlineto{\pgfqpoint{2.853923in}{3.214489in}}%
\pgfpathlineto{\pgfqpoint{2.856528in}{3.217028in}}%
\pgfpathlineto{\pgfqpoint{2.860184in}{3.220582in}}%
\pgfpathlineto{\pgfqpoint{2.863041in}{3.223289in}}%
\pgfpathlineto{\pgfqpoint{2.866444in}{3.226506in}}%
\pgfpathlineto{\pgfqpoint{2.869754in}{3.229549in}}%
\pgfpathlineto{\pgfqpoint{2.872705in}{3.232261in}}%
\pgfpathlineto{\pgfqpoint{2.876682in}{3.235810in}}%
\pgfpathlineto{\pgfqpoint{2.878965in}{3.237847in}}%
\pgfpathlineto{\pgfqpoint{2.883846in}{3.242070in}}%
\pgfpathlineto{\pgfqpoint{2.885225in}{3.243266in}}%
\pgfpathlineto{\pgfqpoint{2.891267in}{3.248331in}}%
\pgfpathlineto{\pgfqpoint{2.891486in}{3.248514in}}%
\pgfpathlineto{\pgfqpoint{2.897746in}{3.253623in}}%
\pgfpathlineto{\pgfqpoint{2.898975in}{3.254591in}}%
\pgfpathlineto{\pgfqpoint{2.904007in}{3.258571in}}%
\pgfpathlineto{\pgfqpoint{2.907004in}{3.260851in}}%
\pgfpathlineto{\pgfqpoint{2.910267in}{3.263349in}}%
\pgfpathlineto{\pgfqpoint{2.915389in}{3.267112in}}%
\pgfpathlineto{\pgfqpoint{2.916528in}{3.267955in}}%
\pgfpathlineto{\pgfqpoint{2.922788in}{3.272414in}}%
\pgfpathlineto{\pgfqpoint{2.924189in}{3.273372in}}%
\pgfpathlineto{\pgfqpoint{2.929048in}{3.276726in}}%
\pgfpathlineto{\pgfqpoint{2.933466in}{3.279633in}}%
\pgfpathlineto{\pgfqpoint{2.935309in}{3.280858in}}%
\pgfpathlineto{\pgfqpoint{2.941569in}{3.284839in}}%
\pgfpathlineto{\pgfqpoint{2.943306in}{3.285893in}}%
\pgfpathlineto{\pgfqpoint{2.947830in}{3.288675in}}%
\pgfpathlineto{\pgfqpoint{2.953811in}{3.292154in}}%
\pgfpathlineto{\pgfqpoint{2.954090in}{3.292318in}}%
\pgfpathlineto{\pgfqpoint{2.960351in}{3.295847in}}%
\pgfpathlineto{\pgfqpoint{2.965188in}{3.298414in}}%
\pgfpathlineto{\pgfqpoint{2.966611in}{3.299181in}}%
\pgfpathlineto{\pgfqpoint{2.972871in}{3.302382in}}%
\pgfpathlineto{\pgfqpoint{2.977654in}{3.304674in}}%
\pgfpathlineto{\pgfqpoint{2.979132in}{3.305396in}}%
\pgfpathlineto{\pgfqpoint{2.985392in}{3.308276in}}%
\pgfpathlineto{\pgfqpoint{2.991607in}{3.310935in}}%
\pgfpathlineto{\pgfqpoint{2.991653in}{3.310955in}}%
\pgfpathlineto{\pgfqpoint{2.997913in}{3.313518in}}%
\pgfpathlineto{\pgfqpoint{3.004174in}{3.315884in}}%
\pgfpathlineto{\pgfqpoint{3.007934in}{3.317195in}}%
\pgfpathlineto{\pgfqpoint{3.010434in}{3.318088in}}%
\pgfpathlineto{\pgfqpoint{3.016694in}{3.320139in}}%
\pgfpathlineto{\pgfqpoint{3.022955in}{3.322000in}}%
\pgfpathlineto{\pgfqpoint{3.028390in}{3.323456in}}%
\pgfpathlineto{\pgfqpoint{3.029215in}{3.323683in}}%
\pgfpathlineto{\pgfqpoint{3.035476in}{3.325220in}}%
\pgfpathlineto{\pgfqpoint{3.041736in}{3.326568in}}%
\pgfpathlineto{\pgfqpoint{3.047997in}{3.327728in}}%
\pgfpathlineto{\pgfqpoint{3.054257in}{3.328701in}}%
\pgfpathlineto{\pgfqpoint{3.060518in}{3.329486in}}%
\pgfpathlineto{\pgfqpoint{3.062950in}{3.329716in}}%
\pgfpathlineto{\pgfqpoint{3.066778in}{3.330091in}}%
\pgfpathlineto{\pgfqpoint{3.073038in}{3.330505in}}%
\pgfpathlineto{\pgfqpoint{3.079299in}{3.330719in}}%
\pgfpathlineto{\pgfqpoint{3.085559in}{3.330727in}}%
\pgfpathlineto{\pgfqpoint{3.091820in}{3.330525in}}%
\pgfpathlineto{\pgfqpoint{3.098080in}{3.330107in}}%
\pgfpathlineto{\pgfqpoint{3.101909in}{3.329716in}}%
\pgfpathlineto{\pgfqpoint{3.104341in}{3.329472in}}%
\pgfpathlineto{\pgfqpoint{3.110601in}{3.328621in}}%
\pgfpathlineto{\pgfqpoint{3.116861in}{3.327535in}}%
\pgfpathlineto{\pgfqpoint{3.123122in}{3.326202in}}%
\pgfpathlineto{\pgfqpoint{3.129382in}{3.324611in}}%
\pgfpathlineto{\pgfqpoint{3.133296in}{3.323456in}}%
\pgfpathlineto{\pgfqpoint{3.135643in}{3.322765in}}%
\pgfpathlineto{\pgfqpoint{3.141903in}{3.320664in}}%
\pgfpathlineto{\pgfqpoint{3.148164in}{3.318264in}}%
\pgfpathlineto{\pgfqpoint{3.150663in}{3.317195in}}%
\pgfpathlineto{\pgfqpoint{3.154424in}{3.315582in}}%
\pgfpathlineto{\pgfqpoint{3.160684in}{3.312588in}}%
\pgfpathlineto{\pgfqpoint{3.163819in}{3.310935in}}%
\pgfpathlineto{\pgfqpoint{3.166945in}{3.309272in}}%
\pgfpathlineto{\pgfqpoint{3.173205in}{3.305611in}}%
\pgfpathlineto{\pgfqpoint{3.174683in}{3.304674in}}%
\pgfpathlineto{\pgfqpoint{3.179466in}{3.301601in}}%
\pgfpathlineto{\pgfqpoint{3.184016in}{3.298414in}}%
\pgfpathlineto{\pgfqpoint{3.185726in}{3.297196in}}%
\pgfpathlineto{\pgfqpoint{3.191987in}{3.292382in}}%
\pgfpathlineto{\pgfqpoint{3.192266in}{3.292154in}}%
\pgfpathlineto{\pgfqpoint{3.198247in}{3.287143in}}%
\pgfpathlineto{\pgfqpoint{3.199643in}{3.285893in}}%
\pgfpathlineto{\pgfqpoint{3.204508in}{3.281429in}}%
\pgfpathlineto{\pgfqpoint{3.206350in}{3.279633in}}%
\pgfpathlineto{\pgfqpoint{3.210768in}{3.275202in}}%
\pgfpathlineto{\pgfqpoint{3.212496in}{3.273372in}}%
\pgfpathlineto{\pgfqpoint{3.217028in}{3.268419in}}%
\pgfpathlineto{\pgfqpoint{3.218167in}{3.267112in}}%
\pgfpathlineto{\pgfqpoint{3.223289in}{3.261022in}}%
\pgfpathlineto{\pgfqpoint{3.223426in}{3.260851in}}%
\pgfpathlineto{\pgfqpoint{3.228320in}{3.254591in}}%
\pgfpathlineto{\pgfqpoint{3.229549in}{3.252955in}}%
\pgfpathlineto{\pgfqpoint{3.232897in}{3.248331in}}%
\pgfpathlineto{\pgfqpoint{3.235810in}{3.244127in}}%
\pgfpathlineto{\pgfqpoint{3.237188in}{3.242070in}}%
\pgfpathlineto{\pgfqpoint{3.241220in}{3.235810in}}%
\pgfpathlineto{\pgfqpoint{3.242070in}{3.234438in}}%
\pgfpathlineto{\pgfqpoint{3.245020in}{3.229549in}}%
\pgfpathlineto{\pgfqpoint{3.248331in}{3.223764in}}%
\pgfpathlineto{\pgfqpoint{3.248596in}{3.223289in}}%
\pgfpathlineto{\pgfqpoint{3.251984in}{3.217028in}}%
\pgfpathlineto{\pgfqpoint{3.254591in}{3.211934in}}%
\pgfpathlineto{\pgfqpoint{3.255177in}{3.210768in}}%
\pgfpathlineto{\pgfqpoint{3.258210in}{3.204508in}}%
\pgfpathlineto{\pgfqpoint{3.260851in}{3.198724in}}%
\pgfpathlineto{\pgfqpoint{3.261066in}{3.198247in}}%
\pgfpathlineto{\pgfqpoint{3.263793in}{3.191987in}}%
\pgfpathlineto{\pgfqpoint{3.266359in}{3.185726in}}%
\pgfpathlineto{\pgfqpoint{3.267112in}{3.183807in}}%
\pgfpathlineto{\pgfqpoint{3.268804in}{3.179466in}}%
\pgfpathlineto{\pgfqpoint{3.271120in}{3.173205in}}%
\pgfpathlineto{\pgfqpoint{3.273298in}{3.166945in}}%
\pgfpathlineto{\pgfqpoint{3.273372in}{3.166722in}}%
\pgfpathlineto{\pgfqpoint{3.275390in}{3.160684in}}%
\pgfpathlineto{\pgfqpoint{3.277360in}{3.154424in}}%
\pgfpathlineto{\pgfqpoint{3.279211in}{3.148164in}}%
\pgfpathlineto{\pgfqpoint{3.279633in}{3.146668in}}%
\pgfpathlineto{\pgfqpoint{3.280983in}{3.141903in}}%
\pgfpathlineto{\pgfqpoint{3.282658in}{3.135643in}}%
\pgfpathlineto{\pgfqpoint{3.284228in}{3.129382in}}%
\pgfpathlineto{\pgfqpoint{3.285699in}{3.123122in}}%
\pgfpathlineto{\pgfqpoint{3.285893in}{3.122252in}}%
\pgfpathlineto{\pgfqpoint{3.287111in}{3.116861in}}%
\pgfpathlineto{\pgfqpoint{3.288436in}{3.110601in}}%
\pgfpathlineto{\pgfqpoint{3.289673in}{3.104341in}}%
\pgfpathlineto{\pgfqpoint{3.290825in}{3.098080in}}%
\pgfpathlineto{\pgfqpoint{3.291896in}{3.091820in}}%
\pgfpathlineto{\pgfqpoint{3.292154in}{3.090210in}}%
\pgfpathlineto{\pgfqpoint{3.292913in}{3.085559in}}%
\pgfpathlineto{\pgfqpoint{3.293863in}{3.079299in}}%
\pgfpathlineto{\pgfqpoint{3.294739in}{3.073038in}}%
\pgfpathlineto{\pgfqpoint{3.295543in}{3.066778in}}%
\pgfpathlineto{\pgfqpoint{3.296278in}{3.060518in}}%
\pgfpathlineto{\pgfqpoint{3.296947in}{3.054257in}}%
\pgfpathlineto{\pgfqpoint{3.297551in}{3.047997in}}%
\pgfpathlineto{\pgfqpoint{3.298093in}{3.041736in}}%
\pgfpathlineto{\pgfqpoint{3.298414in}{3.037580in}}%
\pgfpathlineto{\pgfqpoint{3.298581in}{3.035476in}}%
\pgfpathlineto{\pgfqpoint{3.299021in}{3.029215in}}%
\pgfpathlineto{\pgfqpoint{3.299403in}{3.022955in}}%
\pgfpathlineto{\pgfqpoint{3.299726in}{3.016694in}}%
\pgfpathlineto{\pgfqpoint{3.299994in}{3.010434in}}%
\pgfpathlineto{\pgfqpoint{3.300207in}{3.004174in}}%
\pgfpathlineto{\pgfqpoint{3.300367in}{2.997913in}}%
\pgfpathlineto{\pgfqpoint{3.300474in}{2.991653in}}%
\pgfpathlineto{\pgfqpoint{3.300530in}{2.985392in}}%
\pgfpathlineto{\pgfqpoint{3.300535in}{2.979132in}}%
\pgfpathlineto{\pgfqpoint{3.300491in}{2.972871in}}%
\pgfpathlineto{\pgfqpoint{3.300397in}{2.966611in}}%
\pgfpathlineto{\pgfqpoint{3.300256in}{2.960351in}}%
\pgfpathlineto{\pgfqpoint{3.300067in}{2.954090in}}%
\pgfpathlineto{\pgfqpoint{3.299831in}{2.947830in}}%
\pgfpathlineto{\pgfqpoint{3.299548in}{2.941569in}}%
\pgfpathlineto{\pgfqpoint{3.299218in}{2.935309in}}%
\pgfpathlineto{\pgfqpoint{3.298843in}{2.929048in}}%
\pgfpathlineto{\pgfqpoint{3.298422in}{2.922788in}}%
\pgfpathlineto{\pgfqpoint{3.298414in}{2.922686in}}%
\pgfpathlineto{\pgfqpoint{3.297976in}{2.916528in}}%
\pgfpathlineto{\pgfqpoint{3.297488in}{2.910267in}}%
\pgfpathlineto{\pgfqpoint{3.296956in}{2.904007in}}%
\pgfpathlineto{\pgfqpoint{3.296382in}{2.897746in}}%
\pgfpathlineto{\pgfqpoint{3.295764in}{2.891486in}}%
\pgfpathlineto{\pgfqpoint{3.295103in}{2.885225in}}%
\pgfpathlineto{\pgfqpoint{3.294399in}{2.878965in}}%
\pgfpathlineto{\pgfqpoint{3.293651in}{2.872705in}}%
\pgfpathlineto{\pgfqpoint{3.292859in}{2.866444in}}%
\pgfpathlineto{\pgfqpoint{3.292154in}{2.861155in}}%
\pgfpathlineto{\pgfqpoint{3.292030in}{2.860184in}}%
\pgfpathlineto{\pgfqpoint{3.291193in}{2.853923in}}%
\pgfpathlineto{\pgfqpoint{3.290313in}{2.847663in}}%
\pgfpathlineto{\pgfqpoint{3.289392in}{2.841402in}}%
\pgfpathlineto{\pgfqpoint{3.288427in}{2.835142in}}%
\pgfpathlineto{\pgfqpoint{3.287420in}{2.828881in}}%
\pgfpathlineto{\pgfqpoint{3.286369in}{2.822621in}}%
\pgfpathlineto{\pgfqpoint{3.285893in}{2.819892in}}%
\pgfpathlineto{\pgfqpoint{3.285304in}{2.816361in}}%
\pgfpathlineto{\pgfqpoint{3.284221in}{2.810100in}}%
\pgfpathlineto{\pgfqpoint{3.283097in}{2.803840in}}%
\pgfpathlineto{\pgfqpoint{3.281930in}{2.797579in}}%
\pgfpathlineto{\pgfqpoint{3.280719in}{2.791319in}}%
\pgfpathlineto{\pgfqpoint{3.279633in}{2.785885in}}%
\pgfpathlineto{\pgfqpoint{3.279474in}{2.785058in}}%
\pgfpathlineto{\pgfqpoint{3.278242in}{2.778798in}}%
\pgfpathlineto{\pgfqpoint{3.276970in}{2.772538in}}%
\pgfpathlineto{\pgfqpoint{3.275655in}{2.766277in}}%
\pgfpathlineto{\pgfqpoint{3.274299in}{2.760017in}}%
\pgfpathlineto{\pgfqpoint{3.273372in}{2.755850in}}%
\pgfpathlineto{\pgfqpoint{3.272925in}{2.753756in}}%
\pgfpathlineto{\pgfqpoint{3.271558in}{2.747496in}}%
\pgfpathlineto{\pgfqpoint{3.270151in}{2.741235in}}%
\pgfpathlineto{\pgfqpoint{3.268706in}{2.734975in}}%
\pgfpathlineto{\pgfqpoint{3.267221in}{2.728715in}}%
\pgfpathlineto{\pgfqpoint{3.267112in}{2.728259in}}%
\pgfpathlineto{\pgfqpoint{3.265769in}{2.722454in}}%
\pgfpathlineto{\pgfqpoint{3.264287in}{2.716194in}}%
\pgfpathlineto{\pgfqpoint{3.262769in}{2.709933in}}%
\pgfpathlineto{\pgfqpoint{3.261217in}{2.703673in}}%
\pgfpathlineto{\pgfqpoint{3.260851in}{2.702212in}}%
\pgfpathlineto{\pgfqpoint{3.259692in}{2.697412in}}%
\pgfpathlineto{\pgfqpoint{3.258155in}{2.691152in}}%
\pgfpathlineto{\pgfqpoint{3.256589in}{2.684891in}}%
\pgfpathlineto{\pgfqpoint{3.254995in}{2.678631in}}%
\pgfpathlineto{\pgfqpoint{3.254591in}{2.677052in}}%
\pgfpathlineto{\pgfqpoint{3.253435in}{2.672371in}}%
\pgfpathlineto{\pgfqpoint{3.251872in}{2.666110in}}%
\pgfpathlineto{\pgfqpoint{3.250290in}{2.659850in}}%
\pgfpathlineto{\pgfqpoint{3.248689in}{2.653589in}}%
\pgfpathlineto{\pgfqpoint{3.248331in}{2.652183in}}%
\pgfpathlineto{\pgfqpoint{3.247135in}{2.647329in}}%
\pgfpathlineto{\pgfqpoint{3.245587in}{2.641068in}}%
\pgfpathlineto{\pgfqpoint{3.244031in}{2.634808in}}%
\pgfpathlineto{\pgfqpoint{3.242471in}{2.628548in}}%
\pgfpathlineto{\pgfqpoint{3.242070in}{2.626916in}}%
\pgfpathlineto{\pgfqpoint{3.240967in}{2.622287in}}%
\pgfpathlineto{\pgfqpoint{3.239487in}{2.616027in}}%
\pgfpathlineto{\pgfqpoint{3.238016in}{2.609766in}}%
\pgfpathlineto{\pgfqpoint{3.236560in}{2.603506in}}%
\pgfpathlineto{\pgfqpoint{3.235810in}{2.600207in}}%
\pgfpathlineto{\pgfqpoint{3.235156in}{2.597245in}}%
\pgfpathlineto{\pgfqpoint{3.233812in}{2.590985in}}%
\pgfpathlineto{\pgfqpoint{3.232500in}{2.584725in}}%
\pgfpathlineto{\pgfqpoint{3.231225in}{2.578464in}}%
\pgfpathlineto{\pgfqpoint{3.229995in}{2.572204in}}%
\pgfpathlineto{\pgfqpoint{3.229549in}{2.569806in}}%
\pgfpathlineto{\pgfqpoint{3.228851in}{2.565943in}}%
\pgfpathlineto{\pgfqpoint{3.227785in}{2.559683in}}%
\pgfpathlineto{\pgfqpoint{3.226784in}{2.553422in}}%
\pgfpathlineto{\pgfqpoint{3.225857in}{2.547162in}}%
\pgfpathlineto{\pgfqpoint{3.225012in}{2.540902in}}%
\pgfpathlineto{\pgfqpoint{3.224257in}{2.534641in}}%
\pgfpathlineto{\pgfqpoint{3.223602in}{2.528381in}}%
\pgfpathlineto{\pgfqpoint{3.223289in}{2.524781in}}%
\pgfpathlineto{\pgfqpoint{3.223064in}{2.522120in}}%
\pgfpathlineto{\pgfqpoint{3.222650in}{2.515860in}}%
\pgfpathlineto{\pgfqpoint{3.222356in}{2.509599in}}%
\pgfpathlineto{\pgfqpoint{3.222187in}{2.503339in}}%
\pgfpathlineto{\pgfqpoint{3.222151in}{2.497078in}}%
\pgfpathlineto{\pgfqpoint{3.222254in}{2.490818in}}%
\pgfpathlineto{\pgfqpoint{3.222500in}{2.484558in}}%
\pgfpathlineto{\pgfqpoint{3.222894in}{2.478297in}}%
\pgfpathlineto{\pgfqpoint{3.223289in}{2.473748in}}%
\pgfpathlineto{\pgfqpoint{3.223444in}{2.472037in}}%
\pgfpathlineto{\pgfqpoint{3.224166in}{2.465776in}}%
\pgfpathlineto{\pgfqpoint{3.225043in}{2.459516in}}%
\pgfpathlineto{\pgfqpoint{3.226072in}{2.453255in}}%
\pgfpathlineto{\pgfqpoint{3.227250in}{2.446995in}}%
\pgfpathlineto{\pgfqpoint{3.228571in}{2.440735in}}%
\pgfpathlineto{\pgfqpoint{3.229549in}{2.436525in}}%
\pgfpathlineto{\pgfqpoint{3.230042in}{2.434474in}}%
\pgfpathlineto{\pgfqpoint{3.231673in}{2.428214in}}%
\pgfpathlineto{\pgfqpoint{3.233420in}{2.421953in}}%
\pgfpathlineto{\pgfqpoint{3.235270in}{2.415693in}}%
\pgfpathlineto{\pgfqpoint{3.235810in}{2.413960in}}%
\pgfpathlineto{\pgfqpoint{3.237253in}{2.409432in}}%
\pgfpathlineto{\pgfqpoint{3.239326in}{2.403172in}}%
\pgfpathlineto{\pgfqpoint{3.241458in}{2.396912in}}%
\pgfpathlineto{\pgfqpoint{3.242070in}{2.395164in}}%
\pgfpathlineto{\pgfqpoint{3.243679in}{2.390651in}}%
\pgfpathlineto{\pgfqpoint{3.245945in}{2.384391in}}%
\pgfpathlineto{\pgfqpoint{3.248224in}{2.378130in}}%
\pgfpathlineto{\pgfqpoint{3.248331in}{2.377840in}}%
\pgfpathlineto{\pgfqpoint{3.250561in}{2.371870in}}%
\pgfpathlineto{\pgfqpoint{3.252882in}{2.365609in}}%
\pgfpathlineto{\pgfqpoint{3.254591in}{2.360961in}}%
\pgfpathlineto{\pgfqpoint{3.255190in}{2.359349in}}%
\pgfpathlineto{\pgfqpoint{3.257503in}{2.353088in}}%
\pgfpathlineto{\pgfqpoint{3.259761in}{2.346828in}}%
\pgfpathlineto{\pgfqpoint{3.260851in}{2.343748in}}%
\pgfpathlineto{\pgfqpoint{3.261988in}{2.340568in}}%
\pgfpathlineto{\pgfqpoint{3.264174in}{2.334307in}}%
\pgfpathlineto{\pgfqpoint{3.266279in}{2.328047in}}%
\pgfpathlineto{\pgfqpoint{3.267112in}{2.325498in}}%
\pgfpathlineto{\pgfqpoint{3.268336in}{2.321786in}}%
\pgfpathlineto{\pgfqpoint{3.270327in}{2.315526in}}%
\pgfpathlineto{\pgfqpoint{3.272223in}{2.309265in}}%
\pgfpathlineto{\pgfqpoint{3.273372in}{2.305293in}}%
\pgfpathlineto{\pgfqpoint{3.274042in}{2.303005in}}%
\pgfpathlineto{\pgfqpoint{3.275798in}{2.296745in}}%
\pgfpathlineto{\pgfqpoint{3.277452in}{2.290484in}}%
\pgfpathlineto{\pgfqpoint{3.279004in}{2.284224in}}%
\pgfpathlineto{\pgfqpoint{3.279633in}{2.281539in}}%
\pgfpathlineto{\pgfqpoint{3.280483in}{2.277963in}}%
\pgfpathlineto{\pgfqpoint{3.281880in}{2.271703in}}%
\pgfpathlineto{\pgfqpoint{3.283175in}{2.265442in}}%
\pgfpathlineto{\pgfqpoint{3.284370in}{2.259182in}}%
\pgfpathlineto{\pgfqpoint{3.285465in}{2.252922in}}%
\pgfpathlineto{\pgfqpoint{3.285893in}{2.250258in}}%
\pgfpathlineto{\pgfqpoint{3.286484in}{2.246661in}}%
\pgfpathlineto{\pgfqpoint{3.287420in}{2.240401in}}%
\pgfpathlineto{\pgfqpoint{3.288260in}{2.234140in}}%
\pgfpathlineto{\pgfqpoint{3.289005in}{2.227880in}}%
\pgfpathlineto{\pgfqpoint{3.289657in}{2.221619in}}%
\pgfpathlineto{\pgfqpoint{3.290218in}{2.215359in}}%
\pgfpathlineto{\pgfqpoint{3.290689in}{2.209098in}}%
\pgfpathlineto{\pgfqpoint{3.291073in}{2.202838in}}%
\pgfpathlineto{\pgfqpoint{3.291370in}{2.196578in}}%
\pgfpathlineto{\pgfqpoint{3.291583in}{2.190317in}}%
\pgfpathlineto{\pgfqpoint{3.291713in}{2.184057in}}%
\pgfpathlineto{\pgfqpoint{3.291760in}{2.177796in}}%
\pgfpathlineto{\pgfqpoint{3.291726in}{2.171536in}}%
\pgfpathlineto{\pgfqpoint{3.291612in}{2.165275in}}%
\pgfpathlineto{\pgfqpoint{3.291419in}{2.159015in}}%
\pgfpathlineto{\pgfqpoint{3.291147in}{2.152755in}}%
\pgfpathlineto{\pgfqpoint{3.290796in}{2.146494in}}%
\pgfpathlineto{\pgfqpoint{3.290367in}{2.140234in}}%
\pgfpathlineto{\pgfqpoint{3.289859in}{2.133973in}}%
\pgfpathlineto{\pgfqpoint{3.289273in}{2.127713in}}%
\pgfpathlineto{\pgfqpoint{3.288609in}{2.121452in}}%
\pgfpathlineto{\pgfqpoint{3.287865in}{2.115192in}}%
\pgfpathlineto{\pgfqpoint{3.287040in}{2.108932in}}%
\pgfpathlineto{\pgfqpoint{3.286135in}{2.102671in}}%
\pgfpathlineto{\pgfqpoint{3.285893in}{2.101133in}}%
\pgfpathlineto{\pgfqpoint{3.285182in}{2.096411in}}%
\pgfpathlineto{\pgfqpoint{3.284162in}{2.090150in}}%
\pgfpathlineto{\pgfqpoint{3.283061in}{2.083890in}}%
\pgfpathlineto{\pgfqpoint{3.281878in}{2.077629in}}%
\pgfpathlineto{\pgfqpoint{3.280611in}{2.071369in}}%
\pgfpathlineto{\pgfqpoint{3.279633in}{2.066834in}}%
\pgfpathlineto{\pgfqpoint{3.279275in}{2.065109in}}%
\pgfpathlineto{\pgfqpoint{3.277903in}{2.058848in}}%
\pgfpathlineto{\pgfqpoint{3.276445in}{2.052588in}}%
\pgfpathlineto{\pgfqpoint{3.274899in}{2.046327in}}%
\pgfpathlineto{\pgfqpoint{3.273372in}{2.040488in}}%
\pgfpathlineto{\pgfqpoint{3.273266in}{2.040067in}}%
\pgfpathlineto{\pgfqpoint{3.271618in}{2.033806in}}%
\pgfpathlineto{\pgfqpoint{3.269878in}{2.027546in}}%
\pgfpathlineto{\pgfqpoint{3.268042in}{2.021285in}}%
\pgfpathlineto{\pgfqpoint{3.267112in}{2.018254in}}%
\pgfpathlineto{\pgfqpoint{3.266155in}{2.015025in}}%
\pgfpathlineto{\pgfqpoint{3.264216in}{2.008765in}}%
\pgfpathlineto{\pgfqpoint{3.262176in}{2.002504in}}%
\pgfpathlineto{\pgfqpoint{3.260851in}{1.998615in}}%
\pgfpathlineto{\pgfqpoint{3.260070in}{1.996244in}}%
\pgfpathlineto{\pgfqpoint{3.257923in}{1.989983in}}%
\pgfpathlineto{\pgfqpoint{3.255668in}{1.983723in}}%
\pgfpathlineto{\pgfqpoint{3.254591in}{1.980846in}}%
\pgfpathlineto{\pgfqpoint{3.253361in}{1.977462in}}%
\pgfpathlineto{\pgfqpoint{3.250995in}{1.971202in}}%
\pgfpathlineto{\pgfqpoint{3.248510in}{1.964942in}}%
\pgfpathlineto{\pgfqpoint{3.248331in}{1.964503in}}%
\pgfpathlineto{\pgfqpoint{3.246018in}{1.958681in}}%
\pgfpathlineto{\pgfqpoint{3.243413in}{1.952421in}}%
\pgfpathlineto{\pgfqpoint{3.242070in}{1.949309in}}%
\pgfpathlineto{\pgfqpoint{3.240746in}{1.946160in}}%
\pgfpathlineto{\pgfqpoint{3.238015in}{1.939900in}}%
\pgfpathlineto{\pgfqpoint{3.235810in}{1.935055in}}%
\pgfpathlineto{\pgfqpoint{3.235180in}{1.933639in}}%
\pgfpathlineto{\pgfqpoint{3.232316in}{1.927379in}}%
\pgfpathlineto{\pgfqpoint{3.229549in}{1.921603in}}%
\pgfpathlineto{\pgfqpoint{3.229322in}{1.921119in}}%
\pgfpathlineto{\pgfqpoint{3.226316in}{1.914858in}}%
\pgfpathlineto{\pgfqpoint{3.223289in}{1.908841in}}%
\pgfpathlineto{\pgfqpoint{3.223169in}{1.908598in}}%
\pgfpathlineto{\pgfqpoint{3.220012in}{1.902337in}}%
\pgfpathlineto{\pgfqpoint{3.217028in}{1.896683in}}%
\pgfpathlineto{\pgfqpoint{3.216714in}{1.896077in}}%
\pgfpathlineto{\pgfqpoint{3.213396in}{1.889816in}}%
\pgfpathlineto{\pgfqpoint{3.210768in}{1.885060in}}%
\pgfpathlineto{\pgfqpoint{3.209950in}{1.883556in}}%
\pgfpathlineto{\pgfqpoint{3.206457in}{1.877295in}}%
\pgfpathlineto{\pgfqpoint{3.204508in}{1.873920in}}%
\pgfpathlineto{\pgfqpoint{3.202865in}{1.871035in}}%
\pgfpathlineto{\pgfqpoint{3.199183in}{1.864775in}}%
\pgfpathlineto{\pgfqpoint{3.198247in}{1.863223in}}%
\pgfpathlineto{\pgfqpoint{3.195443in}{1.858514in}}%
\pgfpathlineto{\pgfqpoint{3.191987in}{1.852927in}}%
\pgfpathlineto{\pgfqpoint{3.191575in}{1.852254in}}%
\pgfpathlineto{\pgfqpoint{3.187665in}{1.845993in}}%
\pgfpathlineto{\pgfqpoint{3.185726in}{1.842986in}}%
\pgfpathlineto{\pgfqpoint{3.183647in}{1.839733in}}%
\pgfpathlineto{\pgfqpoint{3.179509in}{1.833472in}}%
\pgfpathlineto{\pgfqpoint{3.179466in}{1.833408in}}%
\pgfpathlineto{\pgfqpoint{3.175335in}{1.827212in}}%
\pgfpathlineto{\pgfqpoint{3.173205in}{1.824117in}}%
\pgfpathlineto{\pgfqpoint{3.171039in}{1.820952in}}%
\pgfpathlineto{\pgfqpoint{3.166945in}{1.815149in}}%
\pgfpathlineto{\pgfqpoint{3.166623in}{1.814691in}}%
\pgfpathlineto{\pgfqpoint{3.162147in}{1.808431in}}%
\pgfpathlineto{\pgfqpoint{3.160684in}{1.806436in}}%
\pgfpathlineto{\pgfqpoint{3.157562in}{1.802170in}}%
\pgfpathlineto{\pgfqpoint{3.154424in}{1.797996in}}%
\pgfpathlineto{\pgfqpoint{3.152856in}{1.795910in}}%
\pgfpathlineto{\pgfqpoint{3.148164in}{1.789820in}}%
\pgfpathlineto{\pgfqpoint{3.148032in}{1.789649in}}%
\pgfpathlineto{\pgfqpoint{3.143134in}{1.783389in}}%
\pgfpathlineto{\pgfqpoint{3.141903in}{1.781850in}}%
\pgfpathlineto{\pgfqpoint{3.138115in}{1.777129in}}%
\pgfpathlineto{\pgfqpoint{3.135643in}{1.774112in}}%
\pgfpathlineto{\pgfqpoint{3.132971in}{1.770868in}}%
\pgfpathlineto{\pgfqpoint{3.129382in}{1.766597in}}%
\pgfpathlineto{\pgfqpoint{3.127700in}{1.764608in}}%
\pgfpathlineto{\pgfqpoint{3.123122in}{1.759293in}}%
\pgfpathlineto{\pgfqpoint{3.122301in}{1.758347in}}%
\pgfpathlineto{\pgfqpoint{3.116861in}{1.752188in}}%
\pgfpathlineto{\pgfqpoint{3.116771in}{1.752087in}}%
\pgfpathlineto{\pgfqpoint{3.111123in}{1.745826in}}%
\pgfpathlineto{\pgfqpoint{3.110601in}{1.745256in}}%
\pgfpathlineto{\pgfqpoint{3.105333in}{1.739566in}}%
\pgfpathlineto{\pgfqpoint{3.104341in}{1.738508in}}%
\pgfpathlineto{\pgfqpoint{3.099395in}{1.733306in}}%
\pgfpathlineto{\pgfqpoint{3.098080in}{1.731939in}}%
\pgfpathlineto{\pgfqpoint{3.093302in}{1.727045in}}%
\pgfpathlineto{\pgfqpoint{3.091820in}{1.725543in}}%
\pgfpathlineto{\pgfqpoint{3.087047in}{1.720785in}}%
\pgfpathlineto{\pgfqpoint{3.085559in}{1.719315in}}%
\pgfpathlineto{\pgfqpoint{3.080621in}{1.714524in}}%
\pgfpathlineto{\pgfqpoint{3.079299in}{1.713251in}}%
\pgfpathlineto{\pgfqpoint{3.074017in}{1.708264in}}%
\pgfpathlineto{\pgfqpoint{3.073038in}{1.707346in}}%
\pgfpathlineto{\pgfqpoint{3.067222in}{1.702003in}}%
\pgfpathlineto{\pgfqpoint{3.066778in}{1.701598in}}%
\pgfpathlineto{\pgfqpoint{3.060518in}{1.695995in}}%
\pgfpathlineto{\pgfqpoint{3.060230in}{1.695743in}}%
\pgfpathlineto{\pgfqpoint{3.054257in}{1.690525in}}%
\pgfpathlineto{\pgfqpoint{3.053032in}{1.689482in}}%
\pgfpathlineto{\pgfqpoint{3.047997in}{1.685201in}}%
\pgfpathlineto{\pgfqpoint{3.045603in}{1.683222in}}%
\pgfpathlineto{\pgfqpoint{3.041736in}{1.680024in}}%
\pgfpathlineto{\pgfqpoint{3.037923in}{1.676962in}}%
\pgfpathlineto{\pgfqpoint{3.035476in}{1.674993in}}%
\pgfpathlineto{\pgfqpoint{3.029971in}{1.670701in}}%
\pgfpathlineto{\pgfqpoint{3.029215in}{1.670110in}}%
\pgfpathlineto{\pgfqpoint{3.022955in}{1.665341in}}%
\pgfpathlineto{\pgfqpoint{3.021736in}{1.664441in}}%
\pgfpathlineto{\pgfqpoint{3.016694in}{1.660695in}}%
\pgfpathlineto{\pgfqpoint{3.013184in}{1.658180in}}%
\pgfpathlineto{\pgfqpoint{3.010434in}{1.656197in}}%
\pgfpathlineto{\pgfqpoint{3.004273in}{1.651920in}}%
\pgfpathlineto{\pgfqpoint{3.004174in}{1.651850in}}%
\pgfpathlineto{\pgfqpoint{2.997913in}{1.647581in}}%
\pgfpathlineto{\pgfqpoint{2.994976in}{1.645659in}}%
\pgfpathlineto{\pgfqpoint{2.991653in}{1.643462in}}%
\pgfpathlineto{\pgfqpoint{2.985392in}{1.639497in}}%
\pgfpathlineto{\pgfqpoint{2.985233in}{1.639399in}}%
\pgfpathlineto{\pgfqpoint{2.979132in}{1.635601in}}%
\pgfpathlineto{\pgfqpoint{2.974982in}{1.633139in}}%
\pgfpathlineto{\pgfqpoint{2.972871in}{1.631868in}}%
\pgfpathlineto{\pgfqpoint{2.966611in}{1.628246in}}%
\pgfpathlineto{\pgfqpoint{2.964142in}{1.626878in}}%
\pgfpathlineto{\pgfqpoint{2.960351in}{1.624741in}}%
\pgfpathlineto{\pgfqpoint{2.954090in}{1.621381in}}%
\pgfpathlineto{\pgfqpoint{2.952608in}{1.620618in}}%
\pgfpathlineto{\pgfqpoint{2.947830in}{1.618110in}}%
\pgfpathlineto{\pgfqpoint{2.941569in}{1.614996in}}%
\pgfpathlineto{\pgfqpoint{2.940226in}{1.614357in}}%
\pgfpathlineto{\pgfqpoint{2.935309in}{1.611967in}}%
\pgfpathlineto{\pgfqpoint{2.929048in}{1.609091in}}%
\pgfpathlineto{\pgfqpoint{2.926773in}{1.608097in}}%
\pgfpathlineto{\pgfqpoint{2.922788in}{1.606314in}}%
\pgfpathlineto{\pgfqpoint{2.916528in}{1.603668in}}%
\pgfpathlineto{\pgfqpoint{2.911927in}{1.601836in}}%
\pgfpathlineto{\pgfqpoint{2.910267in}{1.601158in}}%
\pgfpathlineto{\pgfqpoint{2.904007in}{1.598738in}}%
\pgfpathlineto{\pgfqpoint{2.897746in}{1.596476in}}%
\pgfpathlineto{\pgfqpoint{2.895091in}{1.595576in}}%
\pgfpathlineto{\pgfqpoint{2.891486in}{1.594318in}}%
\pgfpathlineto{\pgfqpoint{2.885225in}{1.592278in}}%
\pgfpathlineto{\pgfqpoint{2.878965in}{1.590388in}}%
\pgfpathlineto{\pgfqpoint{2.875126in}{1.589316in}}%
\pgfpathlineto{\pgfqpoint{2.872705in}{1.588617in}}%
\pgfpathlineto{\pgfqpoint{2.866444in}{1.586952in}}%
\pgfpathlineto{\pgfqpoint{2.860184in}{1.585434in}}%
\pgfpathlineto{\pgfqpoint{2.853923in}{1.584058in}}%
\pgfpathlineto{\pgfqpoint{2.848836in}{1.583055in}}%
\pgfpathlineto{\pgfqpoint{2.847663in}{1.582815in}}%
\pgfpathlineto{\pgfqpoint{2.841402in}{1.581679in}}%
\pgfpathlineto{\pgfqpoint{2.835142in}{1.580687in}}%
\pgfpathlineto{\pgfqpoint{2.828881in}{1.579840in}}%
\pgfpathlineto{\pgfqpoint{2.822621in}{1.579138in}}%
\pgfpathlineto{\pgfqpoint{2.816361in}{1.578582in}}%
\pgfpathlineto{\pgfqpoint{2.810100in}{1.578172in}}%
\pgfpathlineto{\pgfqpoint{2.803840in}{1.577911in}}%
\pgfpathlineto{\pgfqpoint{2.797579in}{1.577801in}}%
\pgfpathlineto{\pgfqpoint{2.791319in}{1.577847in}}%
\pgfpathlineto{\pgfqpoint{2.785058in}{1.578051in}}%
\pgfpathlineto{\pgfqpoint{2.778798in}{1.578419in}}%
\pgfpathlineto{\pgfqpoint{2.772538in}{1.578957in}}%
\pgfpathlineto{\pgfqpoint{2.766277in}{1.579671in}}%
\pgfpathlineto{\pgfqpoint{2.760017in}{1.580569in}}%
\pgfpathlineto{\pgfqpoint{2.753756in}{1.581658in}}%
\pgfpathlineto{\pgfqpoint{2.747496in}{1.582948in}}%
\pgfpathclose%
\pgfusepath{fill}%
\end{pgfscope}%
\begin{pgfscope}%
\pgfpathrectangle{\pgfqpoint{0.500000in}{0.500000in}}{\pgfqpoint{3.750000in}{3.750000in}}%
\pgfusepath{clip}%
\pgfsetbuttcap%
\pgfsetroundjoin%
\definecolor{currentfill}{rgb}{0.849273,0.845998,0.916355}%
\pgfsetfillcolor{currentfill}%
\pgfsetlinewidth{0.000000pt}%
\definecolor{currentstroke}{rgb}{0.000000,0.000000,0.000000}%
\pgfsetstrokecolor{currentstroke}%
\pgfsetdash{}{0pt}%
\pgfpathmoveto{\pgfqpoint{2.747496in}{1.582948in}}%
\pgfpathlineto{\pgfqpoint{2.753756in}{1.581658in}}%
\pgfpathlineto{\pgfqpoint{2.760017in}{1.580569in}}%
\pgfpathlineto{\pgfqpoint{2.766277in}{1.579671in}}%
\pgfpathlineto{\pgfqpoint{2.772538in}{1.578957in}}%
\pgfpathlineto{\pgfqpoint{2.778798in}{1.578419in}}%
\pgfpathlineto{\pgfqpoint{2.785058in}{1.578051in}}%
\pgfpathlineto{\pgfqpoint{2.791319in}{1.577847in}}%
\pgfpathlineto{\pgfqpoint{2.797579in}{1.577801in}}%
\pgfpathlineto{\pgfqpoint{2.803840in}{1.577911in}}%
\pgfpathlineto{\pgfqpoint{2.810100in}{1.578172in}}%
\pgfpathlineto{\pgfqpoint{2.816361in}{1.578582in}}%
\pgfpathlineto{\pgfqpoint{2.822621in}{1.579138in}}%
\pgfpathlineto{\pgfqpoint{2.828881in}{1.579840in}}%
\pgfpathlineto{\pgfqpoint{2.835142in}{1.580687in}}%
\pgfpathlineto{\pgfqpoint{2.841402in}{1.581679in}}%
\pgfpathlineto{\pgfqpoint{2.847663in}{1.582815in}}%
\pgfpathlineto{\pgfqpoint{2.848836in}{1.583055in}}%
\pgfpathlineto{\pgfqpoint{2.853923in}{1.584058in}}%
\pgfpathlineto{\pgfqpoint{2.860184in}{1.585434in}}%
\pgfpathlineto{\pgfqpoint{2.866444in}{1.586952in}}%
\pgfpathlineto{\pgfqpoint{2.872705in}{1.588617in}}%
\pgfpathlineto{\pgfqpoint{2.875126in}{1.589316in}}%
\pgfpathlineto{\pgfqpoint{2.878965in}{1.590388in}}%
\pgfpathlineto{\pgfqpoint{2.885225in}{1.592278in}}%
\pgfpathlineto{\pgfqpoint{2.891486in}{1.594318in}}%
\pgfpathlineto{\pgfqpoint{2.895091in}{1.595576in}}%
\pgfpathlineto{\pgfqpoint{2.897746in}{1.596476in}}%
\pgfpathlineto{\pgfqpoint{2.904007in}{1.598738in}}%
\pgfpathlineto{\pgfqpoint{2.910267in}{1.601158in}}%
\pgfpathlineto{\pgfqpoint{2.911927in}{1.601836in}}%
\pgfpathlineto{\pgfqpoint{2.916528in}{1.603668in}}%
\pgfpathlineto{\pgfqpoint{2.922788in}{1.606314in}}%
\pgfpathlineto{\pgfqpoint{2.926773in}{1.608097in}}%
\pgfpathlineto{\pgfqpoint{2.929048in}{1.609091in}}%
\pgfpathlineto{\pgfqpoint{2.935309in}{1.611967in}}%
\pgfpathlineto{\pgfqpoint{2.940226in}{1.614357in}}%
\pgfpathlineto{\pgfqpoint{2.941569in}{1.614996in}}%
\pgfpathlineto{\pgfqpoint{2.947830in}{1.618110in}}%
\pgfpathlineto{\pgfqpoint{2.952608in}{1.620618in}}%
\pgfpathlineto{\pgfqpoint{2.954090in}{1.621381in}}%
\pgfpathlineto{\pgfqpoint{2.960351in}{1.624741in}}%
\pgfpathlineto{\pgfqpoint{2.964142in}{1.626878in}}%
\pgfpathlineto{\pgfqpoint{2.966611in}{1.628246in}}%
\pgfpathlineto{\pgfqpoint{2.972871in}{1.631868in}}%
\pgfpathlineto{\pgfqpoint{2.974982in}{1.633139in}}%
\pgfpathlineto{\pgfqpoint{2.979132in}{1.635601in}}%
\pgfpathlineto{\pgfqpoint{2.985233in}{1.639399in}}%
\pgfpathlineto{\pgfqpoint{2.985392in}{1.639497in}}%
\pgfpathlineto{\pgfqpoint{2.991653in}{1.643462in}}%
\pgfpathlineto{\pgfqpoint{2.994976in}{1.645659in}}%
\pgfpathlineto{\pgfqpoint{2.997913in}{1.647581in}}%
\pgfpathlineto{\pgfqpoint{3.004174in}{1.651850in}}%
\pgfpathlineto{\pgfqpoint{3.004273in}{1.651920in}}%
\pgfpathlineto{\pgfqpoint{3.010434in}{1.656197in}}%
\pgfpathlineto{\pgfqpoint{3.013184in}{1.658180in}}%
\pgfpathlineto{\pgfqpoint{3.016694in}{1.660695in}}%
\pgfpathlineto{\pgfqpoint{3.021736in}{1.664441in}}%
\pgfpathlineto{\pgfqpoint{3.022955in}{1.665341in}}%
\pgfpathlineto{\pgfqpoint{3.029215in}{1.670110in}}%
\pgfpathlineto{\pgfqpoint{3.029971in}{1.670701in}}%
\pgfpathlineto{\pgfqpoint{3.035476in}{1.674993in}}%
\pgfpathlineto{\pgfqpoint{3.037923in}{1.676962in}}%
\pgfpathlineto{\pgfqpoint{3.041736in}{1.680024in}}%
\pgfpathlineto{\pgfqpoint{3.045603in}{1.683222in}}%
\pgfpathlineto{\pgfqpoint{3.047997in}{1.685201in}}%
\pgfpathlineto{\pgfqpoint{3.053032in}{1.689482in}}%
\pgfpathlineto{\pgfqpoint{3.054257in}{1.690525in}}%
\pgfpathlineto{\pgfqpoint{3.060230in}{1.695743in}}%
\pgfpathlineto{\pgfqpoint{3.060518in}{1.695995in}}%
\pgfpathlineto{\pgfqpoint{3.066778in}{1.701598in}}%
\pgfpathlineto{\pgfqpoint{3.067222in}{1.702003in}}%
\pgfpathlineto{\pgfqpoint{3.073038in}{1.707346in}}%
\pgfpathlineto{\pgfqpoint{3.074017in}{1.708264in}}%
\pgfpathlineto{\pgfqpoint{3.079299in}{1.713251in}}%
\pgfpathlineto{\pgfqpoint{3.080621in}{1.714524in}}%
\pgfpathlineto{\pgfqpoint{3.085559in}{1.719315in}}%
\pgfpathlineto{\pgfqpoint{3.087047in}{1.720785in}}%
\pgfpathlineto{\pgfqpoint{3.091820in}{1.725543in}}%
\pgfpathlineto{\pgfqpoint{3.093302in}{1.727045in}}%
\pgfpathlineto{\pgfqpoint{3.098080in}{1.731939in}}%
\pgfpathlineto{\pgfqpoint{3.099395in}{1.733306in}}%
\pgfpathlineto{\pgfqpoint{3.104341in}{1.738508in}}%
\pgfpathlineto{\pgfqpoint{3.105333in}{1.739566in}}%
\pgfpathlineto{\pgfqpoint{3.110601in}{1.745256in}}%
\pgfpathlineto{\pgfqpoint{3.111123in}{1.745826in}}%
\pgfpathlineto{\pgfqpoint{3.116771in}{1.752087in}}%
\pgfpathlineto{\pgfqpoint{3.116861in}{1.752188in}}%
\pgfpathlineto{\pgfqpoint{3.122301in}{1.758347in}}%
\pgfpathlineto{\pgfqpoint{3.123122in}{1.759293in}}%
\pgfpathlineto{\pgfqpoint{3.127700in}{1.764608in}}%
\pgfpathlineto{\pgfqpoint{3.129382in}{1.766597in}}%
\pgfpathlineto{\pgfqpoint{3.132971in}{1.770868in}}%
\pgfpathlineto{\pgfqpoint{3.135643in}{1.774112in}}%
\pgfpathlineto{\pgfqpoint{3.138115in}{1.777129in}}%
\pgfpathlineto{\pgfqpoint{3.141903in}{1.781850in}}%
\pgfpathlineto{\pgfqpoint{3.143134in}{1.783389in}}%
\pgfpathlineto{\pgfqpoint{3.148032in}{1.789649in}}%
\pgfpathlineto{\pgfqpoint{3.148164in}{1.789820in}}%
\pgfpathlineto{\pgfqpoint{3.152856in}{1.795910in}}%
\pgfpathlineto{\pgfqpoint{3.154424in}{1.797996in}}%
\pgfpathlineto{\pgfqpoint{3.157562in}{1.802170in}}%
\pgfpathlineto{\pgfqpoint{3.160684in}{1.806436in}}%
\pgfpathlineto{\pgfqpoint{3.162147in}{1.808431in}}%
\pgfpathlineto{\pgfqpoint{3.166623in}{1.814691in}}%
\pgfpathlineto{\pgfqpoint{3.166945in}{1.815149in}}%
\pgfpathlineto{\pgfqpoint{3.171039in}{1.820952in}}%
\pgfpathlineto{\pgfqpoint{3.173205in}{1.824117in}}%
\pgfpathlineto{\pgfqpoint{3.175335in}{1.827212in}}%
\pgfpathlineto{\pgfqpoint{3.179466in}{1.833408in}}%
\pgfpathlineto{\pgfqpoint{3.179509in}{1.833472in}}%
\pgfpathlineto{\pgfqpoint{3.183647in}{1.839733in}}%
\pgfpathlineto{\pgfqpoint{3.185726in}{1.842986in}}%
\pgfpathlineto{\pgfqpoint{3.187665in}{1.845993in}}%
\pgfpathlineto{\pgfqpoint{3.191575in}{1.852254in}}%
\pgfpathlineto{\pgfqpoint{3.191987in}{1.852927in}}%
\pgfpathlineto{\pgfqpoint{3.195443in}{1.858514in}}%
\pgfpathlineto{\pgfqpoint{3.198247in}{1.863223in}}%
\pgfpathlineto{\pgfqpoint{3.199183in}{1.864775in}}%
\pgfpathlineto{\pgfqpoint{3.202865in}{1.871035in}}%
\pgfpathlineto{\pgfqpoint{3.204508in}{1.873920in}}%
\pgfpathlineto{\pgfqpoint{3.206457in}{1.877295in}}%
\pgfpathlineto{\pgfqpoint{3.209950in}{1.883556in}}%
\pgfpathlineto{\pgfqpoint{3.210768in}{1.885060in}}%
\pgfpathlineto{\pgfqpoint{3.213396in}{1.889816in}}%
\pgfpathlineto{\pgfqpoint{3.216714in}{1.896077in}}%
\pgfpathlineto{\pgfqpoint{3.217028in}{1.896683in}}%
\pgfpathlineto{\pgfqpoint{3.220012in}{1.902337in}}%
\pgfpathlineto{\pgfqpoint{3.223169in}{1.908598in}}%
\pgfpathlineto{\pgfqpoint{3.223289in}{1.908841in}}%
\pgfpathlineto{\pgfqpoint{3.226316in}{1.914858in}}%
\pgfpathlineto{\pgfqpoint{3.229322in}{1.921119in}}%
\pgfpathlineto{\pgfqpoint{3.229549in}{1.921603in}}%
\pgfpathlineto{\pgfqpoint{3.232316in}{1.927379in}}%
\pgfpathlineto{\pgfqpoint{3.235180in}{1.933639in}}%
\pgfpathlineto{\pgfqpoint{3.235810in}{1.935055in}}%
\pgfpathlineto{\pgfqpoint{3.238015in}{1.939900in}}%
\pgfpathlineto{\pgfqpoint{3.240746in}{1.946160in}}%
\pgfpathlineto{\pgfqpoint{3.242070in}{1.949309in}}%
\pgfpathlineto{\pgfqpoint{3.243413in}{1.952421in}}%
\pgfpathlineto{\pgfqpoint{3.246018in}{1.958681in}}%
\pgfpathlineto{\pgfqpoint{3.248331in}{1.964503in}}%
\pgfpathlineto{\pgfqpoint{3.248510in}{1.964942in}}%
\pgfpathlineto{\pgfqpoint{3.250995in}{1.971202in}}%
\pgfpathlineto{\pgfqpoint{3.253361in}{1.977462in}}%
\pgfpathlineto{\pgfqpoint{3.254591in}{1.980846in}}%
\pgfpathlineto{\pgfqpoint{3.255668in}{1.983723in}}%
\pgfpathlineto{\pgfqpoint{3.257923in}{1.989983in}}%
\pgfpathlineto{\pgfqpoint{3.260070in}{1.996244in}}%
\pgfpathlineto{\pgfqpoint{3.260851in}{1.998615in}}%
\pgfpathlineto{\pgfqpoint{3.262176in}{2.002504in}}%
\pgfpathlineto{\pgfqpoint{3.264216in}{2.008765in}}%
\pgfpathlineto{\pgfqpoint{3.266155in}{2.015025in}}%
\pgfpathlineto{\pgfqpoint{3.267112in}{2.018254in}}%
\pgfpathlineto{\pgfqpoint{3.268042in}{2.021285in}}%
\pgfpathlineto{\pgfqpoint{3.269878in}{2.027546in}}%
\pgfpathlineto{\pgfqpoint{3.271618in}{2.033806in}}%
\pgfpathlineto{\pgfqpoint{3.273266in}{2.040067in}}%
\pgfpathlineto{\pgfqpoint{3.273372in}{2.040488in}}%
\pgfpathlineto{\pgfqpoint{3.274899in}{2.046327in}}%
\pgfpathlineto{\pgfqpoint{3.276445in}{2.052588in}}%
\pgfpathlineto{\pgfqpoint{3.277903in}{2.058848in}}%
\pgfpathlineto{\pgfqpoint{3.279275in}{2.065109in}}%
\pgfpathlineto{\pgfqpoint{3.279633in}{2.066834in}}%
\pgfpathlineto{\pgfqpoint{3.280611in}{2.071369in}}%
\pgfpathlineto{\pgfqpoint{3.281878in}{2.077629in}}%
\pgfpathlineto{\pgfqpoint{3.283061in}{2.083890in}}%
\pgfpathlineto{\pgfqpoint{3.284162in}{2.090150in}}%
\pgfpathlineto{\pgfqpoint{3.285182in}{2.096411in}}%
\pgfpathlineto{\pgfqpoint{3.285893in}{2.101133in}}%
\pgfpathlineto{\pgfqpoint{3.286135in}{2.102671in}}%
\pgfpathlineto{\pgfqpoint{3.287040in}{2.108932in}}%
\pgfpathlineto{\pgfqpoint{3.287865in}{2.115192in}}%
\pgfpathlineto{\pgfqpoint{3.288609in}{2.121452in}}%
\pgfpathlineto{\pgfqpoint{3.289273in}{2.127713in}}%
\pgfpathlineto{\pgfqpoint{3.289859in}{2.133973in}}%
\pgfpathlineto{\pgfqpoint{3.290367in}{2.140234in}}%
\pgfpathlineto{\pgfqpoint{3.290796in}{2.146494in}}%
\pgfpathlineto{\pgfqpoint{3.291147in}{2.152755in}}%
\pgfpathlineto{\pgfqpoint{3.291419in}{2.159015in}}%
\pgfpathlineto{\pgfqpoint{3.291612in}{2.165275in}}%
\pgfpathlineto{\pgfqpoint{3.291726in}{2.171536in}}%
\pgfpathlineto{\pgfqpoint{3.291760in}{2.177796in}}%
\pgfpathlineto{\pgfqpoint{3.291713in}{2.184057in}}%
\pgfpathlineto{\pgfqpoint{3.291583in}{2.190317in}}%
\pgfpathlineto{\pgfqpoint{3.291370in}{2.196578in}}%
\pgfpathlineto{\pgfqpoint{3.291073in}{2.202838in}}%
\pgfpathlineto{\pgfqpoint{3.290689in}{2.209098in}}%
\pgfpathlineto{\pgfqpoint{3.290218in}{2.215359in}}%
\pgfpathlineto{\pgfqpoint{3.289657in}{2.221619in}}%
\pgfpathlineto{\pgfqpoint{3.289005in}{2.227880in}}%
\pgfpathlineto{\pgfqpoint{3.288260in}{2.234140in}}%
\pgfpathlineto{\pgfqpoint{3.287420in}{2.240401in}}%
\pgfpathlineto{\pgfqpoint{3.286484in}{2.246661in}}%
\pgfpathlineto{\pgfqpoint{3.285893in}{2.250258in}}%
\pgfpathlineto{\pgfqpoint{3.285465in}{2.252922in}}%
\pgfpathlineto{\pgfqpoint{3.284370in}{2.259182in}}%
\pgfpathlineto{\pgfqpoint{3.283175in}{2.265442in}}%
\pgfpathlineto{\pgfqpoint{3.281880in}{2.271703in}}%
\pgfpathlineto{\pgfqpoint{3.280483in}{2.277963in}}%
\pgfpathlineto{\pgfqpoint{3.279633in}{2.281539in}}%
\pgfpathlineto{\pgfqpoint{3.279004in}{2.284224in}}%
\pgfpathlineto{\pgfqpoint{3.277452in}{2.290484in}}%
\pgfpathlineto{\pgfqpoint{3.275798in}{2.296745in}}%
\pgfpathlineto{\pgfqpoint{3.274042in}{2.303005in}}%
\pgfpathlineto{\pgfqpoint{3.273372in}{2.305293in}}%
\pgfpathlineto{\pgfqpoint{3.272223in}{2.309265in}}%
\pgfpathlineto{\pgfqpoint{3.270327in}{2.315526in}}%
\pgfpathlineto{\pgfqpoint{3.268336in}{2.321786in}}%
\pgfpathlineto{\pgfqpoint{3.267112in}{2.325498in}}%
\pgfpathlineto{\pgfqpoint{3.266279in}{2.328047in}}%
\pgfpathlineto{\pgfqpoint{3.264174in}{2.334307in}}%
\pgfpathlineto{\pgfqpoint{3.261988in}{2.340568in}}%
\pgfpathlineto{\pgfqpoint{3.260851in}{2.343748in}}%
\pgfpathlineto{\pgfqpoint{3.259761in}{2.346828in}}%
\pgfpathlineto{\pgfqpoint{3.257503in}{2.353088in}}%
\pgfpathlineto{\pgfqpoint{3.255190in}{2.359349in}}%
\pgfpathlineto{\pgfqpoint{3.254591in}{2.360961in}}%
\pgfpathlineto{\pgfqpoint{3.252882in}{2.365609in}}%
\pgfpathlineto{\pgfqpoint{3.250561in}{2.371870in}}%
\pgfpathlineto{\pgfqpoint{3.248331in}{2.377840in}}%
\pgfpathlineto{\pgfqpoint{3.248224in}{2.378130in}}%
\pgfpathlineto{\pgfqpoint{3.245945in}{2.384391in}}%
\pgfpathlineto{\pgfqpoint{3.243679in}{2.390651in}}%
\pgfpathlineto{\pgfqpoint{3.242070in}{2.395164in}}%
\pgfpathlineto{\pgfqpoint{3.241458in}{2.396912in}}%
\pgfpathlineto{\pgfqpoint{3.239326in}{2.403172in}}%
\pgfpathlineto{\pgfqpoint{3.237253in}{2.409432in}}%
\pgfpathlineto{\pgfqpoint{3.235810in}{2.413960in}}%
\pgfpathlineto{\pgfqpoint{3.235270in}{2.415693in}}%
\pgfpathlineto{\pgfqpoint{3.233420in}{2.421953in}}%
\pgfpathlineto{\pgfqpoint{3.231673in}{2.428214in}}%
\pgfpathlineto{\pgfqpoint{3.230042in}{2.434474in}}%
\pgfpathlineto{\pgfqpoint{3.229549in}{2.436525in}}%
\pgfpathlineto{\pgfqpoint{3.228571in}{2.440735in}}%
\pgfpathlineto{\pgfqpoint{3.227250in}{2.446995in}}%
\pgfpathlineto{\pgfqpoint{3.226072in}{2.453255in}}%
\pgfpathlineto{\pgfqpoint{3.225043in}{2.459516in}}%
\pgfpathlineto{\pgfqpoint{3.224166in}{2.465776in}}%
\pgfpathlineto{\pgfqpoint{3.223444in}{2.472037in}}%
\pgfpathlineto{\pgfqpoint{3.223289in}{2.473748in}}%
\pgfpathlineto{\pgfqpoint{3.222894in}{2.478297in}}%
\pgfpathlineto{\pgfqpoint{3.222500in}{2.484558in}}%
\pgfpathlineto{\pgfqpoint{3.222254in}{2.490818in}}%
\pgfpathlineto{\pgfqpoint{3.222151in}{2.497078in}}%
\pgfpathlineto{\pgfqpoint{3.222187in}{2.503339in}}%
\pgfpathlineto{\pgfqpoint{3.222356in}{2.509599in}}%
\pgfpathlineto{\pgfqpoint{3.222650in}{2.515860in}}%
\pgfpathlineto{\pgfqpoint{3.223064in}{2.522120in}}%
\pgfpathlineto{\pgfqpoint{3.223289in}{2.524781in}}%
\pgfpathlineto{\pgfqpoint{3.223602in}{2.528381in}}%
\pgfpathlineto{\pgfqpoint{3.224257in}{2.534641in}}%
\pgfpathlineto{\pgfqpoint{3.225012in}{2.540902in}}%
\pgfpathlineto{\pgfqpoint{3.225857in}{2.547162in}}%
\pgfpathlineto{\pgfqpoint{3.226784in}{2.553422in}}%
\pgfpathlineto{\pgfqpoint{3.227785in}{2.559683in}}%
\pgfpathlineto{\pgfqpoint{3.228851in}{2.565943in}}%
\pgfpathlineto{\pgfqpoint{3.229549in}{2.569806in}}%
\pgfpathlineto{\pgfqpoint{3.229995in}{2.572204in}}%
\pgfpathlineto{\pgfqpoint{3.231225in}{2.578464in}}%
\pgfpathlineto{\pgfqpoint{3.232500in}{2.584725in}}%
\pgfpathlineto{\pgfqpoint{3.233812in}{2.590985in}}%
\pgfpathlineto{\pgfqpoint{3.235156in}{2.597245in}}%
\pgfpathlineto{\pgfqpoint{3.235810in}{2.600207in}}%
\pgfpathlineto{\pgfqpoint{3.236560in}{2.603506in}}%
\pgfpathlineto{\pgfqpoint{3.238016in}{2.609766in}}%
\pgfpathlineto{\pgfqpoint{3.239487in}{2.616027in}}%
\pgfpathlineto{\pgfqpoint{3.240967in}{2.622287in}}%
\pgfpathlineto{\pgfqpoint{3.242070in}{2.626916in}}%
\pgfpathlineto{\pgfqpoint{3.242471in}{2.628548in}}%
\pgfpathlineto{\pgfqpoint{3.244031in}{2.634808in}}%
\pgfpathlineto{\pgfqpoint{3.245587in}{2.641068in}}%
\pgfpathlineto{\pgfqpoint{3.247135in}{2.647329in}}%
\pgfpathlineto{\pgfqpoint{3.248331in}{2.652183in}}%
\pgfpathlineto{\pgfqpoint{3.248689in}{2.653589in}}%
\pgfpathlineto{\pgfqpoint{3.250290in}{2.659850in}}%
\pgfpathlineto{\pgfqpoint{3.251872in}{2.666110in}}%
\pgfpathlineto{\pgfqpoint{3.253435in}{2.672371in}}%
\pgfpathlineto{\pgfqpoint{3.254591in}{2.677052in}}%
\pgfpathlineto{\pgfqpoint{3.254995in}{2.678631in}}%
\pgfpathlineto{\pgfqpoint{3.256589in}{2.684891in}}%
\pgfpathlineto{\pgfqpoint{3.258155in}{2.691152in}}%
\pgfpathlineto{\pgfqpoint{3.259692in}{2.697412in}}%
\pgfpathlineto{\pgfqpoint{3.260851in}{2.702212in}}%
\pgfpathlineto{\pgfqpoint{3.261217in}{2.703673in}}%
\pgfpathlineto{\pgfqpoint{3.262769in}{2.709933in}}%
\pgfpathlineto{\pgfqpoint{3.264287in}{2.716194in}}%
\pgfpathlineto{\pgfqpoint{3.265769in}{2.722454in}}%
\pgfpathlineto{\pgfqpoint{3.267112in}{2.728259in}}%
\pgfpathlineto{\pgfqpoint{3.267221in}{2.728715in}}%
\pgfpathlineto{\pgfqpoint{3.268706in}{2.734975in}}%
\pgfpathlineto{\pgfqpoint{3.270151in}{2.741235in}}%
\pgfpathlineto{\pgfqpoint{3.271558in}{2.747496in}}%
\pgfpathlineto{\pgfqpoint{3.272925in}{2.753756in}}%
\pgfpathlineto{\pgfqpoint{3.273372in}{2.755850in}}%
\pgfpathlineto{\pgfqpoint{3.274299in}{2.760017in}}%
\pgfpathlineto{\pgfqpoint{3.275655in}{2.766277in}}%
\pgfpathlineto{\pgfqpoint{3.276970in}{2.772538in}}%
\pgfpathlineto{\pgfqpoint{3.278242in}{2.778798in}}%
\pgfpathlineto{\pgfqpoint{3.279474in}{2.785058in}}%
\pgfpathlineto{\pgfqpoint{3.279633in}{2.785885in}}%
\pgfpathlineto{\pgfqpoint{3.280719in}{2.791319in}}%
\pgfpathlineto{\pgfqpoint{3.281930in}{2.797579in}}%
\pgfpathlineto{\pgfqpoint{3.283097in}{2.803840in}}%
\pgfpathlineto{\pgfqpoint{3.284221in}{2.810100in}}%
\pgfpathlineto{\pgfqpoint{3.285304in}{2.816361in}}%
\pgfpathlineto{\pgfqpoint{3.285893in}{2.819892in}}%
\pgfpathlineto{\pgfqpoint{3.286369in}{2.822621in}}%
\pgfpathlineto{\pgfqpoint{3.287420in}{2.828881in}}%
\pgfpathlineto{\pgfqpoint{3.288427in}{2.835142in}}%
\pgfpathlineto{\pgfqpoint{3.289392in}{2.841402in}}%
\pgfpathlineto{\pgfqpoint{3.290313in}{2.847663in}}%
\pgfpathlineto{\pgfqpoint{3.291193in}{2.853923in}}%
\pgfpathlineto{\pgfqpoint{3.292030in}{2.860184in}}%
\pgfpathlineto{\pgfqpoint{3.292154in}{2.861155in}}%
\pgfpathlineto{\pgfqpoint{3.292859in}{2.866444in}}%
\pgfpathlineto{\pgfqpoint{3.293651in}{2.872705in}}%
\pgfpathlineto{\pgfqpoint{3.294399in}{2.878965in}}%
\pgfpathlineto{\pgfqpoint{3.295103in}{2.885225in}}%
\pgfpathlineto{\pgfqpoint{3.295764in}{2.891486in}}%
\pgfpathlineto{\pgfqpoint{3.296382in}{2.897746in}}%
\pgfpathlineto{\pgfqpoint{3.296956in}{2.904007in}}%
\pgfpathlineto{\pgfqpoint{3.297488in}{2.910267in}}%
\pgfpathlineto{\pgfqpoint{3.297976in}{2.916528in}}%
\pgfpathlineto{\pgfqpoint{3.298414in}{2.922686in}}%
\pgfpathlineto{\pgfqpoint{3.298422in}{2.922788in}}%
\pgfpathlineto{\pgfqpoint{3.298843in}{2.929048in}}%
\pgfpathlineto{\pgfqpoint{3.299218in}{2.935309in}}%
\pgfpathlineto{\pgfqpoint{3.299548in}{2.941569in}}%
\pgfpathlineto{\pgfqpoint{3.299831in}{2.947830in}}%
\pgfpathlineto{\pgfqpoint{3.300067in}{2.954090in}}%
\pgfpathlineto{\pgfqpoint{3.300256in}{2.960351in}}%
\pgfpathlineto{\pgfqpoint{3.300397in}{2.966611in}}%
\pgfpathlineto{\pgfqpoint{3.300491in}{2.972871in}}%
\pgfpathlineto{\pgfqpoint{3.300535in}{2.979132in}}%
\pgfpathlineto{\pgfqpoint{3.300530in}{2.985392in}}%
\pgfpathlineto{\pgfqpoint{3.300474in}{2.991653in}}%
\pgfpathlineto{\pgfqpoint{3.300367in}{2.997913in}}%
\pgfpathlineto{\pgfqpoint{3.300207in}{3.004174in}}%
\pgfpathlineto{\pgfqpoint{3.299994in}{3.010434in}}%
\pgfpathlineto{\pgfqpoint{3.299726in}{3.016694in}}%
\pgfpathlineto{\pgfqpoint{3.299403in}{3.022955in}}%
\pgfpathlineto{\pgfqpoint{3.299021in}{3.029215in}}%
\pgfpathlineto{\pgfqpoint{3.298581in}{3.035476in}}%
\pgfpathlineto{\pgfqpoint{3.298414in}{3.037580in}}%
\pgfpathlineto{\pgfqpoint{3.298093in}{3.041736in}}%
\pgfpathlineto{\pgfqpoint{3.297551in}{3.047997in}}%
\pgfpathlineto{\pgfqpoint{3.296947in}{3.054257in}}%
\pgfpathlineto{\pgfqpoint{3.296278in}{3.060518in}}%
\pgfpathlineto{\pgfqpoint{3.295543in}{3.066778in}}%
\pgfpathlineto{\pgfqpoint{3.294739in}{3.073038in}}%
\pgfpathlineto{\pgfqpoint{3.293863in}{3.079299in}}%
\pgfpathlineto{\pgfqpoint{3.292913in}{3.085559in}}%
\pgfpathlineto{\pgfqpoint{3.292154in}{3.090210in}}%
\pgfpathlineto{\pgfqpoint{3.291896in}{3.091820in}}%
\pgfpathlineto{\pgfqpoint{3.290825in}{3.098080in}}%
\pgfpathlineto{\pgfqpoint{3.289673in}{3.104341in}}%
\pgfpathlineto{\pgfqpoint{3.288436in}{3.110601in}}%
\pgfpathlineto{\pgfqpoint{3.287111in}{3.116861in}}%
\pgfpathlineto{\pgfqpoint{3.285893in}{3.122252in}}%
\pgfpathlineto{\pgfqpoint{3.285699in}{3.123122in}}%
\pgfpathlineto{\pgfqpoint{3.284228in}{3.129382in}}%
\pgfpathlineto{\pgfqpoint{3.282658in}{3.135643in}}%
\pgfpathlineto{\pgfqpoint{3.280983in}{3.141903in}}%
\pgfpathlineto{\pgfqpoint{3.279633in}{3.146668in}}%
\pgfpathlineto{\pgfqpoint{3.279211in}{3.148164in}}%
\pgfpathlineto{\pgfqpoint{3.277360in}{3.154424in}}%
\pgfpathlineto{\pgfqpoint{3.275390in}{3.160684in}}%
\pgfpathlineto{\pgfqpoint{3.273372in}{3.166722in}}%
\pgfpathlineto{\pgfqpoint{3.273298in}{3.166945in}}%
\pgfpathlineto{\pgfqpoint{3.271120in}{3.173205in}}%
\pgfpathlineto{\pgfqpoint{3.268804in}{3.179466in}}%
\pgfpathlineto{\pgfqpoint{3.267112in}{3.183807in}}%
\pgfpathlineto{\pgfqpoint{3.266359in}{3.185726in}}%
\pgfpathlineto{\pgfqpoint{3.263793in}{3.191987in}}%
\pgfpathlineto{\pgfqpoint{3.261066in}{3.198247in}}%
\pgfpathlineto{\pgfqpoint{3.260851in}{3.198724in}}%
\pgfpathlineto{\pgfqpoint{3.258210in}{3.204508in}}%
\pgfpathlineto{\pgfqpoint{3.255177in}{3.210768in}}%
\pgfpathlineto{\pgfqpoint{3.254591in}{3.211934in}}%
\pgfpathlineto{\pgfqpoint{3.251984in}{3.217028in}}%
\pgfpathlineto{\pgfqpoint{3.248596in}{3.223289in}}%
\pgfpathlineto{\pgfqpoint{3.248331in}{3.223764in}}%
\pgfpathlineto{\pgfqpoint{3.245020in}{3.229549in}}%
\pgfpathlineto{\pgfqpoint{3.242070in}{3.234438in}}%
\pgfpathlineto{\pgfqpoint{3.241220in}{3.235810in}}%
\pgfpathlineto{\pgfqpoint{3.237188in}{3.242070in}}%
\pgfpathlineto{\pgfqpoint{3.235810in}{3.244127in}}%
\pgfpathlineto{\pgfqpoint{3.232897in}{3.248331in}}%
\pgfpathlineto{\pgfqpoint{3.229549in}{3.252955in}}%
\pgfpathlineto{\pgfqpoint{3.228320in}{3.254591in}}%
\pgfpathlineto{\pgfqpoint{3.223426in}{3.260851in}}%
\pgfpathlineto{\pgfqpoint{3.223289in}{3.261022in}}%
\pgfpathlineto{\pgfqpoint{3.218167in}{3.267112in}}%
\pgfpathlineto{\pgfqpoint{3.217028in}{3.268419in}}%
\pgfpathlineto{\pgfqpoint{3.212496in}{3.273372in}}%
\pgfpathlineto{\pgfqpoint{3.210768in}{3.275202in}}%
\pgfpathlineto{\pgfqpoint{3.206350in}{3.279633in}}%
\pgfpathlineto{\pgfqpoint{3.204508in}{3.281429in}}%
\pgfpathlineto{\pgfqpoint{3.199643in}{3.285893in}}%
\pgfpathlineto{\pgfqpoint{3.198247in}{3.287143in}}%
\pgfpathlineto{\pgfqpoint{3.192266in}{3.292154in}}%
\pgfpathlineto{\pgfqpoint{3.191987in}{3.292382in}}%
\pgfpathlineto{\pgfqpoint{3.185726in}{3.297196in}}%
\pgfpathlineto{\pgfqpoint{3.184016in}{3.298414in}}%
\pgfpathlineto{\pgfqpoint{3.179466in}{3.301601in}}%
\pgfpathlineto{\pgfqpoint{3.174683in}{3.304674in}}%
\pgfpathlineto{\pgfqpoint{3.173205in}{3.305611in}}%
\pgfpathlineto{\pgfqpoint{3.166945in}{3.309272in}}%
\pgfpathlineto{\pgfqpoint{3.163819in}{3.310935in}}%
\pgfpathlineto{\pgfqpoint{3.160684in}{3.312588in}}%
\pgfpathlineto{\pgfqpoint{3.154424in}{3.315582in}}%
\pgfpathlineto{\pgfqpoint{3.150663in}{3.317195in}}%
\pgfpathlineto{\pgfqpoint{3.148164in}{3.318264in}}%
\pgfpathlineto{\pgfqpoint{3.141903in}{3.320664in}}%
\pgfpathlineto{\pgfqpoint{3.135643in}{3.322765in}}%
\pgfpathlineto{\pgfqpoint{3.133296in}{3.323456in}}%
\pgfpathlineto{\pgfqpoint{3.129382in}{3.324611in}}%
\pgfpathlineto{\pgfqpoint{3.123122in}{3.326202in}}%
\pgfpathlineto{\pgfqpoint{3.116861in}{3.327535in}}%
\pgfpathlineto{\pgfqpoint{3.110601in}{3.328621in}}%
\pgfpathlineto{\pgfqpoint{3.104341in}{3.329472in}}%
\pgfpathlineto{\pgfqpoint{3.101909in}{3.329716in}}%
\pgfpathlineto{\pgfqpoint{3.098080in}{3.330107in}}%
\pgfpathlineto{\pgfqpoint{3.091820in}{3.330525in}}%
\pgfpathlineto{\pgfqpoint{3.085559in}{3.330727in}}%
\pgfpathlineto{\pgfqpoint{3.079299in}{3.330719in}}%
\pgfpathlineto{\pgfqpoint{3.073038in}{3.330505in}}%
\pgfpathlineto{\pgfqpoint{3.066778in}{3.330091in}}%
\pgfpathlineto{\pgfqpoint{3.062950in}{3.329716in}}%
\pgfpathlineto{\pgfqpoint{3.060518in}{3.329486in}}%
\pgfpathlineto{\pgfqpoint{3.054257in}{3.328701in}}%
\pgfpathlineto{\pgfqpoint{3.047997in}{3.327728in}}%
\pgfpathlineto{\pgfqpoint{3.041736in}{3.326568in}}%
\pgfpathlineto{\pgfqpoint{3.035476in}{3.325220in}}%
\pgfpathlineto{\pgfqpoint{3.029215in}{3.323683in}}%
\pgfpathlineto{\pgfqpoint{3.028390in}{3.323456in}}%
\pgfpathlineto{\pgfqpoint{3.022955in}{3.322000in}}%
\pgfpathlineto{\pgfqpoint{3.016694in}{3.320139in}}%
\pgfpathlineto{\pgfqpoint{3.010434in}{3.318088in}}%
\pgfpathlineto{\pgfqpoint{3.007934in}{3.317195in}}%
\pgfpathlineto{\pgfqpoint{3.004174in}{3.315884in}}%
\pgfpathlineto{\pgfqpoint{2.997913in}{3.313518in}}%
\pgfpathlineto{\pgfqpoint{2.991653in}{3.310955in}}%
\pgfpathlineto{\pgfqpoint{2.991607in}{3.310935in}}%
\pgfpathlineto{\pgfqpoint{2.985392in}{3.308276in}}%
\pgfpathlineto{\pgfqpoint{2.979132in}{3.305396in}}%
\pgfpathlineto{\pgfqpoint{2.977654in}{3.304674in}}%
\pgfpathlineto{\pgfqpoint{2.972871in}{3.302382in}}%
\pgfpathlineto{\pgfqpoint{2.966611in}{3.299181in}}%
\pgfpathlineto{\pgfqpoint{2.965188in}{3.298414in}}%
\pgfpathlineto{\pgfqpoint{2.960351in}{3.295847in}}%
\pgfpathlineto{\pgfqpoint{2.954090in}{3.292318in}}%
\pgfpathlineto{\pgfqpoint{2.953811in}{3.292154in}}%
\pgfpathlineto{\pgfqpoint{2.947830in}{3.288675in}}%
\pgfpathlineto{\pgfqpoint{2.943306in}{3.285893in}}%
\pgfpathlineto{\pgfqpoint{2.941569in}{3.284839in}}%
\pgfpathlineto{\pgfqpoint{2.935309in}{3.280858in}}%
\pgfpathlineto{\pgfqpoint{2.933466in}{3.279633in}}%
\pgfpathlineto{\pgfqpoint{2.929048in}{3.276726in}}%
\pgfpathlineto{\pgfqpoint{2.924189in}{3.273372in}}%
\pgfpathlineto{\pgfqpoint{2.922788in}{3.272414in}}%
\pgfpathlineto{\pgfqpoint{2.916528in}{3.267955in}}%
\pgfpathlineto{\pgfqpoint{2.915389in}{3.267112in}}%
\pgfpathlineto{\pgfqpoint{2.910267in}{3.263349in}}%
\pgfpathlineto{\pgfqpoint{2.907004in}{3.260851in}}%
\pgfpathlineto{\pgfqpoint{2.904007in}{3.258571in}}%
\pgfpathlineto{\pgfqpoint{2.898975in}{3.254591in}}%
\pgfpathlineto{\pgfqpoint{2.897746in}{3.253623in}}%
\pgfpathlineto{\pgfqpoint{2.891486in}{3.248514in}}%
\pgfpathlineto{\pgfqpoint{2.891267in}{3.248331in}}%
\pgfpathlineto{\pgfqpoint{2.885225in}{3.243266in}}%
\pgfpathlineto{\pgfqpoint{2.883846in}{3.242070in}}%
\pgfpathlineto{\pgfqpoint{2.878965in}{3.237847in}}%
\pgfpathlineto{\pgfqpoint{2.876682in}{3.235810in}}%
\pgfpathlineto{\pgfqpoint{2.872705in}{3.232261in}}%
\pgfpathlineto{\pgfqpoint{2.869754in}{3.229549in}}%
\pgfpathlineto{\pgfqpoint{2.866444in}{3.226506in}}%
\pgfpathlineto{\pgfqpoint{2.863041in}{3.223289in}}%
\pgfpathlineto{\pgfqpoint{2.860184in}{3.220582in}}%
\pgfpathlineto{\pgfqpoint{2.856528in}{3.217028in}}%
\pgfpathlineto{\pgfqpoint{2.853923in}{3.214489in}}%
\pgfpathlineto{\pgfqpoint{2.850198in}{3.210768in}}%
\pgfpathlineto{\pgfqpoint{2.847663in}{3.208224in}}%
\pgfpathlineto{\pgfqpoint{2.844040in}{3.204508in}}%
\pgfpathlineto{\pgfqpoint{2.841402in}{3.201786in}}%
\pgfpathlineto{\pgfqpoint{2.838042in}{3.198247in}}%
\pgfpathlineto{\pgfqpoint{2.835142in}{3.195171in}}%
\pgfpathlineto{\pgfqpoint{2.832195in}{3.191987in}}%
\pgfpathlineto{\pgfqpoint{2.828881in}{3.188377in}}%
\pgfpathlineto{\pgfqpoint{2.826490in}{3.185726in}}%
\pgfpathlineto{\pgfqpoint{2.822621in}{3.181399in}}%
\pgfpathlineto{\pgfqpoint{2.820920in}{3.179466in}}%
\pgfpathlineto{\pgfqpoint{2.816361in}{3.174231in}}%
\pgfpathlineto{\pgfqpoint{2.815480in}{3.173205in}}%
\pgfpathlineto{\pgfqpoint{2.810162in}{3.166945in}}%
\pgfpathlineto{\pgfqpoint{2.810100in}{3.166871in}}%
\pgfpathlineto{\pgfqpoint{2.804942in}{3.160684in}}%
\pgfpathlineto{\pgfqpoint{2.803840in}{3.159345in}}%
\pgfpathlineto{\pgfqpoint{2.799834in}{3.154424in}}%
\pgfpathlineto{\pgfqpoint{2.797579in}{3.151615in}}%
\pgfpathlineto{\pgfqpoint{2.794834in}{3.148164in}}%
\pgfpathlineto{\pgfqpoint{2.791319in}{3.143675in}}%
\pgfpathlineto{\pgfqpoint{2.789942in}{3.141903in}}%
\pgfpathlineto{\pgfqpoint{2.785152in}{3.135643in}}%
\pgfpathlineto{\pgfqpoint{2.785058in}{3.135519in}}%
\pgfpathlineto{\pgfqpoint{2.780426in}{3.129382in}}%
\pgfpathlineto{\pgfqpoint{2.778798in}{3.127187in}}%
\pgfpathlineto{\pgfqpoint{2.775797in}{3.123122in}}%
\pgfpathlineto{\pgfqpoint{2.772538in}{3.118622in}}%
\pgfpathlineto{\pgfqpoint{2.771266in}{3.116861in}}%
\pgfpathlineto{\pgfqpoint{2.766818in}{3.110601in}}%
\pgfpathlineto{\pgfqpoint{2.766277in}{3.109832in}}%
\pgfpathlineto{\pgfqpoint{2.762424in}{3.104341in}}%
\pgfpathlineto{\pgfqpoint{2.760017in}{3.100836in}}%
\pgfpathlineto{\pgfqpoint{2.758124in}{3.098080in}}%
\pgfpathlineto{\pgfqpoint{2.753914in}{3.091820in}}%
\pgfpathlineto{\pgfqpoint{2.753756in}{3.091583in}}%
\pgfpathlineto{\pgfqpoint{2.749734in}{3.085559in}}%
\pgfpathlineto{\pgfqpoint{2.747496in}{3.082126in}}%
\pgfpathlineto{\pgfqpoint{2.745647in}{3.079299in}}%
\pgfpathlineto{\pgfqpoint{2.741640in}{3.073038in}}%
\pgfpathlineto{\pgfqpoint{2.741235in}{3.072400in}}%
\pgfpathlineto{\pgfqpoint{2.737659in}{3.066778in}}%
\pgfpathlineto{\pgfqpoint{2.734975in}{3.062446in}}%
\pgfpathlineto{\pgfqpoint{2.733772in}{3.060518in}}%
\pgfpathlineto{\pgfqpoint{2.729934in}{3.054257in}}%
\pgfpathlineto{\pgfqpoint{2.728715in}{3.052232in}}%
\pgfpathlineto{\pgfqpoint{2.726143in}{3.047997in}}%
\pgfpathlineto{\pgfqpoint{2.722454in}{3.041745in}}%
\pgfpathlineto{\pgfqpoint{2.722449in}{3.041736in}}%
\pgfpathlineto{\pgfqpoint{2.718748in}{3.035476in}}%
\pgfpathlineto{\pgfqpoint{2.716194in}{3.031025in}}%
\pgfpathlineto{\pgfqpoint{2.715143in}{3.029215in}}%
\pgfpathlineto{\pgfqpoint{2.711568in}{3.022955in}}%
\pgfpathlineto{\pgfqpoint{2.709933in}{3.020029in}}%
\pgfpathlineto{\pgfqpoint{2.708045in}{3.016694in}}%
\pgfpathlineto{\pgfqpoint{2.704583in}{3.010434in}}%
\pgfpathlineto{\pgfqpoint{2.703673in}{3.008762in}}%
\pgfpathlineto{\pgfqpoint{2.701137in}{3.004174in}}%
\pgfpathlineto{\pgfqpoint{2.697775in}{2.997913in}}%
\pgfpathlineto{\pgfqpoint{2.697412in}{2.997229in}}%
\pgfpathlineto{\pgfqpoint{2.694401in}{2.991653in}}%
\pgfpathlineto{\pgfqpoint{2.691152in}{2.985440in}}%
\pgfpathlineto{\pgfqpoint{2.691127in}{2.985392in}}%
\pgfpathlineto{\pgfqpoint{2.687818in}{2.979132in}}%
\pgfpathlineto{\pgfqpoint{2.684891in}{2.973411in}}%
\pgfpathlineto{\pgfqpoint{2.684609in}{2.972871in}}%
\pgfpathlineto{\pgfqpoint{2.681372in}{2.966611in}}%
\pgfpathlineto{\pgfqpoint{2.678631in}{2.961148in}}%
\pgfpathlineto{\pgfqpoint{2.678222in}{2.960351in}}%
\pgfpathlineto{\pgfqpoint{2.675042in}{2.954090in}}%
\pgfpathlineto{\pgfqpoint{2.672371in}{2.948674in}}%
\pgfpathlineto{\pgfqpoint{2.671943in}{2.947830in}}%
\pgfpathlineto{\pgfqpoint{2.668809in}{2.941569in}}%
\pgfpathlineto{\pgfqpoint{2.666110in}{2.936022in}}%
\pgfpathlineto{\pgfqpoint{2.665753in}{2.935309in}}%
\pgfpathlineto{\pgfqpoint{2.662651in}{2.929048in}}%
\pgfpathlineto{\pgfqpoint{2.659850in}{2.923234in}}%
\pgfpathlineto{\pgfqpoint{2.659628in}{2.922788in}}%
\pgfpathlineto{\pgfqpoint{2.656543in}{2.916528in}}%
\pgfpathlineto{\pgfqpoint{2.653589in}{2.910364in}}%
\pgfpathlineto{\pgfqpoint{2.653541in}{2.910267in}}%
\pgfpathlineto{\pgfqpoint{2.650458in}{2.904007in}}%
\pgfpathlineto{\pgfqpoint{2.647455in}{2.897746in}}%
\pgfpathlineto{\pgfqpoint{2.647329in}{2.897482in}}%
\pgfpathlineto{\pgfqpoint{2.644363in}{2.891486in}}%
\pgfpathlineto{\pgfqpoint{2.641338in}{2.885225in}}%
\pgfpathlineto{\pgfqpoint{2.641068in}{2.884663in}}%
\pgfpathlineto{\pgfqpoint{2.638220in}{2.878965in}}%
\pgfpathlineto{\pgfqpoint{2.635155in}{2.872705in}}%
\pgfpathlineto{\pgfqpoint{2.634808in}{2.871993in}}%
\pgfpathlineto{\pgfqpoint{2.631985in}{2.866444in}}%
\pgfpathlineto{\pgfqpoint{2.628855in}{2.860184in}}%
\pgfpathlineto{\pgfqpoint{2.628548in}{2.859566in}}%
\pgfpathlineto{\pgfqpoint{2.625600in}{2.853923in}}%
\pgfpathlineto{\pgfqpoint{2.622380in}{2.847663in}}%
\pgfpathlineto{\pgfqpoint{2.622287in}{2.847481in}}%
\pgfpathlineto{\pgfqpoint{2.618997in}{2.841402in}}%
\pgfpathlineto{\pgfqpoint{2.616027in}{2.835846in}}%
\pgfpathlineto{\pgfqpoint{2.615628in}{2.835142in}}%
\pgfpathlineto{\pgfqpoint{2.612086in}{2.828881in}}%
\pgfpathlineto{\pgfqpoint{2.609766in}{2.824751in}}%
\pgfpathlineto{\pgfqpoint{2.608493in}{2.822621in}}%
\pgfpathlineto{\pgfqpoint{2.604754in}{2.816361in}}%
\pgfpathlineto{\pgfqpoint{2.603506in}{2.814269in}}%
\pgfpathlineto{\pgfqpoint{2.600840in}{2.810100in}}%
\pgfpathlineto{\pgfqpoint{2.597245in}{2.804471in}}%
\pgfpathlineto{\pgfqpoint{2.596811in}{2.803840in}}%
\pgfpathlineto{\pgfqpoint{2.592478in}{2.797579in}}%
\pgfpathlineto{\pgfqpoint{2.590985in}{2.795429in}}%
\pgfpathlineto{\pgfqpoint{2.587879in}{2.791319in}}%
\pgfpathlineto{\pgfqpoint{2.584725in}{2.787164in}}%
\pgfpathlineto{\pgfqpoint{2.582969in}{2.785058in}}%
\pgfpathlineto{\pgfqpoint{2.578464in}{2.779700in}}%
\pgfpathlineto{\pgfqpoint{2.577623in}{2.778798in}}%
\pgfpathlineto{\pgfqpoint{2.572204in}{2.773053in}}%
\pgfpathlineto{\pgfqpoint{2.571656in}{2.772538in}}%
\pgfpathlineto{\pgfqpoint{2.565943in}{2.767226in}}%
\pgfpathlineto{\pgfqpoint{2.564775in}{2.766277in}}%
\pgfpathlineto{\pgfqpoint{2.559683in}{2.762204in}}%
\pgfpathlineto{\pgfqpoint{2.556483in}{2.760017in}}%
\pgfpathlineto{\pgfqpoint{2.553422in}{2.757960in}}%
\pgfpathlineto{\pgfqpoint{2.547162in}{2.754470in}}%
\pgfpathlineto{\pgfqpoint{2.545558in}{2.753756in}}%
\pgfpathlineto{\pgfqpoint{2.540902in}{2.751724in}}%
\pgfpathlineto{\pgfqpoint{2.534641in}{2.749674in}}%
\pgfpathlineto{\pgfqpoint{2.528381in}{2.748287in}}%
\pgfpathlineto{\pgfqpoint{2.522120in}{2.747536in}}%
\pgfpathlineto{\pgfqpoint{2.520386in}{2.747496in}}%
\pgfpathlineto{\pgfqpoint{2.515860in}{2.747393in}}%
\pgfpathlineto{\pgfqpoint{2.514336in}{2.747496in}}%
\pgfpathlineto{\pgfqpoint{2.509599in}{2.747818in}}%
\pgfpathlineto{\pgfqpoint{2.503339in}{2.748781in}}%
\pgfpathlineto{\pgfqpoint{2.497078in}{2.750241in}}%
\pgfpathlineto{\pgfqpoint{2.490818in}{2.752160in}}%
\pgfpathlineto{\pgfqpoint{2.486562in}{2.753756in}}%
\pgfpathlineto{\pgfqpoint{2.484558in}{2.754510in}}%
\pgfpathlineto{\pgfqpoint{2.478297in}{2.757270in}}%
\pgfpathlineto{\pgfqpoint{2.472772in}{2.760017in}}%
\pgfpathlineto{\pgfqpoint{2.472037in}{2.760383in}}%
\pgfpathlineto{\pgfqpoint{2.465776in}{2.763849in}}%
\pgfpathlineto{\pgfqpoint{2.461734in}{2.766277in}}%
\pgfpathlineto{\pgfqpoint{2.459516in}{2.767609in}}%
\pgfpathlineto{\pgfqpoint{2.453255in}{2.771646in}}%
\pgfpathlineto{\pgfqpoint{2.451958in}{2.772538in}}%
\pgfpathlineto{\pgfqpoint{2.446995in}{2.775940in}}%
\pgfpathlineto{\pgfqpoint{2.443029in}{2.778798in}}%
\pgfpathlineto{\pgfqpoint{2.440735in}{2.780447in}}%
\pgfpathlineto{\pgfqpoint{2.434585in}{2.785058in}}%
\pgfpathlineto{\pgfqpoint{2.434474in}{2.785141in}}%
\pgfpathlineto{\pgfqpoint{2.428214in}{2.790025in}}%
\pgfpathlineto{\pgfqpoint{2.426607in}{2.791319in}}%
\pgfpathlineto{\pgfqpoint{2.421953in}{2.795052in}}%
\pgfpathlineto{\pgfqpoint{2.418889in}{2.797579in}}%
\pgfpathlineto{\pgfqpoint{2.415693in}{2.800204in}}%
\pgfpathlineto{\pgfqpoint{2.411367in}{2.803840in}}%
\pgfpathlineto{\pgfqpoint{2.409432in}{2.805459in}}%
\pgfpathlineto{\pgfqpoint{2.403997in}{2.810100in}}%
\pgfpathlineto{\pgfqpoint{2.403172in}{2.810802in}}%
\pgfpathlineto{\pgfqpoint{2.396912in}{2.816218in}}%
\pgfpathlineto{\pgfqpoint{2.396749in}{2.816361in}}%
\pgfpathlineto{\pgfqpoint{2.390651in}{2.821702in}}%
\pgfpathlineto{\pgfqpoint{2.389615in}{2.822621in}}%
\pgfpathlineto{\pgfqpoint{2.384391in}{2.827229in}}%
\pgfpathlineto{\pgfqpoint{2.382536in}{2.828881in}}%
\pgfpathlineto{\pgfqpoint{2.378130in}{2.832787in}}%
\pgfpathlineto{\pgfqpoint{2.375495in}{2.835142in}}%
\pgfpathlineto{\pgfqpoint{2.371870in}{2.838365in}}%
\pgfpathlineto{\pgfqpoint{2.368475in}{2.841402in}}%
\pgfpathlineto{\pgfqpoint{2.365609in}{2.843953in}}%
\pgfpathlineto{\pgfqpoint{2.361461in}{2.847663in}}%
\pgfpathlineto{\pgfqpoint{2.359349in}{2.849543in}}%
\pgfpathlineto{\pgfqpoint{2.354443in}{2.853923in}}%
\pgfpathlineto{\pgfqpoint{2.353088in}{2.855127in}}%
\pgfpathlineto{\pgfqpoint{2.347410in}{2.860184in}}%
\pgfpathlineto{\pgfqpoint{2.346828in}{2.860699in}}%
\pgfpathlineto{\pgfqpoint{2.340568in}{2.866257in}}%
\pgfpathlineto{\pgfqpoint{2.340358in}{2.866444in}}%
\pgfpathlineto{\pgfqpoint{2.334307in}{2.871804in}}%
\pgfpathlineto{\pgfqpoint{2.333289in}{2.872705in}}%
\pgfpathlineto{\pgfqpoint{2.328047in}{2.877324in}}%
\pgfpathlineto{\pgfqpoint{2.326180in}{2.878965in}}%
\pgfpathlineto{\pgfqpoint{2.321786in}{2.882814in}}%
\pgfpathlineto{\pgfqpoint{2.319025in}{2.885225in}}%
\pgfpathlineto{\pgfqpoint{2.315526in}{2.888271in}}%
\pgfpathlineto{\pgfqpoint{2.311817in}{2.891486in}}%
\pgfpathlineto{\pgfqpoint{2.309265in}{2.893692in}}%
\pgfpathlineto{\pgfqpoint{2.304551in}{2.897746in}}%
\pgfpathlineto{\pgfqpoint{2.303005in}{2.899073in}}%
\pgfpathlineto{\pgfqpoint{2.297222in}{2.904007in}}%
\pgfpathlineto{\pgfqpoint{2.296745in}{2.904414in}}%
\pgfpathlineto{\pgfqpoint{2.290484in}{2.909725in}}%
\pgfpathlineto{\pgfqpoint{2.289842in}{2.910267in}}%
\pgfpathlineto{\pgfqpoint{2.284224in}{2.915003in}}%
\pgfpathlineto{\pgfqpoint{2.282402in}{2.916528in}}%
\pgfpathlineto{\pgfqpoint{2.277963in}{2.920237in}}%
\pgfpathlineto{\pgfqpoint{2.274887in}{2.922788in}}%
\pgfpathlineto{\pgfqpoint{2.271703in}{2.925426in}}%
\pgfpathlineto{\pgfqpoint{2.267293in}{2.929048in}}%
\pgfpathlineto{\pgfqpoint{2.265442in}{2.930568in}}%
\pgfpathlineto{\pgfqpoint{2.259616in}{2.935309in}}%
\pgfpathlineto{\pgfqpoint{2.259182in}{2.935662in}}%
\pgfpathlineto{\pgfqpoint{2.252922in}{2.940730in}}%
\pgfpathlineto{\pgfqpoint{2.251877in}{2.941569in}}%
\pgfpathlineto{\pgfqpoint{2.246661in}{2.945760in}}%
\pgfpathlineto{\pgfqpoint{2.244058in}{2.947830in}}%
\pgfpathlineto{\pgfqpoint{2.240401in}{2.950741in}}%
\pgfpathlineto{\pgfqpoint{2.236147in}{2.954090in}}%
\pgfpathlineto{\pgfqpoint{2.234140in}{2.955673in}}%
\pgfpathlineto{\pgfqpoint{2.228140in}{2.960351in}}%
\pgfpathlineto{\pgfqpoint{2.227880in}{2.960554in}}%
\pgfpathlineto{\pgfqpoint{2.221619in}{2.965418in}}%
\pgfpathlineto{\pgfqpoint{2.220066in}{2.966611in}}%
\pgfpathlineto{\pgfqpoint{2.215359in}{2.970237in}}%
\pgfpathlineto{\pgfqpoint{2.211896in}{2.972871in}}%
\pgfpathlineto{\pgfqpoint{2.209098in}{2.975006in}}%
\pgfpathlineto{\pgfqpoint{2.203621in}{2.979132in}}%
\pgfpathlineto{\pgfqpoint{2.202838in}{2.979723in}}%
\pgfpathlineto{\pgfqpoint{2.196578in}{2.984416in}}%
\pgfpathlineto{\pgfqpoint{2.195262in}{2.985392in}}%
\pgfpathlineto{\pgfqpoint{2.190317in}{2.989075in}}%
\pgfpathlineto{\pgfqpoint{2.186806in}{2.991653in}}%
\pgfpathlineto{\pgfqpoint{2.184057in}{2.993681in}}%
\pgfpathlineto{\pgfqpoint{2.178235in}{2.997913in}}%
\pgfpathlineto{\pgfqpoint{2.177796in}{2.998234in}}%
\pgfpathlineto{\pgfqpoint{2.171536in}{3.002774in}}%
\pgfpathlineto{\pgfqpoint{2.169580in}{3.004174in}}%
\pgfpathlineto{\pgfqpoint{2.165275in}{3.007271in}}%
\pgfpathlineto{\pgfqpoint{2.160811in}{3.010434in}}%
\pgfpathlineto{\pgfqpoint{2.159015in}{3.011714in}}%
\pgfpathlineto{\pgfqpoint{2.152755in}{3.016118in}}%
\pgfpathlineto{\pgfqpoint{2.151928in}{3.016694in}}%
\pgfpathlineto{\pgfqpoint{2.146494in}{3.020506in}}%
\pgfpathlineto{\pgfqpoint{2.142945in}{3.022955in}}%
\pgfpathlineto{\pgfqpoint{2.140234in}{3.024839in}}%
\pgfpathlineto{\pgfqpoint{2.133973in}{3.029118in}}%
\pgfpathlineto{\pgfqpoint{2.133830in}{3.029215in}}%
\pgfpathlineto{\pgfqpoint{2.127713in}{3.033397in}}%
\pgfpathlineto{\pgfqpoint{2.124619in}{3.035476in}}%
\pgfpathlineto{\pgfqpoint{2.121452in}{3.037620in}}%
\pgfpathlineto{\pgfqpoint{2.115263in}{3.041736in}}%
\pgfpathlineto{\pgfqpoint{2.115192in}{3.041784in}}%
\pgfpathlineto{\pgfqpoint{2.108932in}{3.045952in}}%
\pgfpathlineto{\pgfqpoint{2.105804in}{3.047997in}}%
\pgfpathlineto{\pgfqpoint{2.102671in}{3.050063in}}%
\pgfpathlineto{\pgfqpoint{2.096411in}{3.054118in}}%
\pgfpathlineto{\pgfqpoint{2.096194in}{3.054257in}}%
\pgfpathlineto{\pgfqpoint{2.090150in}{3.058176in}}%
\pgfpathlineto{\pgfqpoint{2.086468in}{3.060518in}}%
\pgfpathlineto{\pgfqpoint{2.083890in}{3.062174in}}%
\pgfpathlineto{\pgfqpoint{2.077629in}{3.066129in}}%
\pgfpathlineto{\pgfqpoint{2.076590in}{3.066778in}}%
\pgfpathlineto{\pgfqpoint{2.071369in}{3.070073in}}%
\pgfpathlineto{\pgfqpoint{2.066573in}{3.073038in}}%
\pgfpathlineto{\pgfqpoint{2.065109in}{3.073954in}}%
\pgfpathlineto{\pgfqpoint{2.058848in}{3.077816in}}%
\pgfpathlineto{\pgfqpoint{2.056405in}{3.079299in}}%
\pgfpathlineto{\pgfqpoint{2.052588in}{3.081643in}}%
\pgfpathlineto{\pgfqpoint{2.046327in}{3.085407in}}%
\pgfpathlineto{\pgfqpoint{2.046072in}{3.085559in}}%
\pgfpathlineto{\pgfqpoint{2.040067in}{3.089179in}}%
\pgfpathlineto{\pgfqpoint{2.035589in}{3.091820in}}%
\pgfpathlineto{\pgfqpoint{2.033806in}{3.092884in}}%
\pgfpathlineto{\pgfqpoint{2.027546in}{3.096567in}}%
\pgfpathlineto{\pgfqpoint{2.024929in}{3.098080in}}%
\pgfpathlineto{\pgfqpoint{2.021285in}{3.100215in}}%
\pgfpathlineto{\pgfqpoint{2.015025in}{3.103808in}}%
\pgfpathlineto{\pgfqpoint{2.014086in}{3.104341in}}%
\pgfpathlineto{\pgfqpoint{2.008765in}{3.107399in}}%
\pgfpathlineto{\pgfqpoint{2.003057in}{3.110601in}}%
\pgfpathlineto{\pgfqpoint{2.002504in}{3.110916in}}%
\pgfpathlineto{\pgfqpoint{1.996244in}{3.114437in}}%
\pgfpathlineto{\pgfqpoint{1.991835in}{3.116861in}}%
\pgfpathlineto{\pgfqpoint{1.989983in}{3.117894in}}%
\pgfpathlineto{\pgfqpoint{1.983723in}{3.121333in}}%
\pgfpathlineto{\pgfqpoint{1.980400in}{3.123122in}}%
\pgfpathlineto{\pgfqpoint{1.977462in}{3.124728in}}%
\pgfpathlineto{\pgfqpoint{1.971202in}{3.128086in}}%
\pgfpathlineto{\pgfqpoint{1.968742in}{3.129382in}}%
\pgfpathlineto{\pgfqpoint{1.964942in}{3.131417in}}%
\pgfpathlineto{\pgfqpoint{1.958681in}{3.134698in}}%
\pgfpathlineto{\pgfqpoint{1.956848in}{3.135643in}}%
\pgfpathlineto{\pgfqpoint{1.952421in}{3.137963in}}%
\pgfpathlineto{\pgfqpoint{1.946160in}{3.141169in}}%
\pgfpathlineto{\pgfqpoint{1.944703in}{3.141903in}}%
\pgfpathlineto{\pgfqpoint{1.939900in}{3.144365in}}%
\pgfpathlineto{\pgfqpoint{1.933639in}{3.147498in}}%
\pgfpathlineto{\pgfqpoint{1.932289in}{3.148164in}}%
\pgfpathlineto{\pgfqpoint{1.927379in}{3.150624in}}%
\pgfpathlineto{\pgfqpoint{1.921119in}{3.153686in}}%
\pgfpathlineto{\pgfqpoint{1.919585in}{3.154424in}}%
\pgfpathlineto{\pgfqpoint{1.914858in}{3.156739in}}%
\pgfpathlineto{\pgfqpoint{1.908598in}{3.159731in}}%
\pgfpathlineto{\pgfqpoint{1.906568in}{3.160684in}}%
\pgfpathlineto{\pgfqpoint{1.902337in}{3.162707in}}%
\pgfpathlineto{\pgfqpoint{1.896077in}{3.165632in}}%
\pgfpathlineto{\pgfqpoint{1.893208in}{3.166945in}}%
\pgfpathlineto{\pgfqpoint{1.889816in}{3.168527in}}%
\pgfpathlineto{\pgfqpoint{1.883556in}{3.171385in}}%
\pgfpathlineto{\pgfqpoint{1.879476in}{3.173205in}}%
\pgfpathlineto{\pgfqpoint{1.877295in}{3.174197in}}%
\pgfpathlineto{\pgfqpoint{1.871035in}{3.176989in}}%
\pgfpathlineto{\pgfqpoint{1.865334in}{3.179466in}}%
\pgfpathlineto{\pgfqpoint{1.864775in}{3.179714in}}%
\pgfpathlineto{\pgfqpoint{1.858514in}{3.182440in}}%
\pgfpathlineto{\pgfqpoint{1.852254in}{3.185093in}}%
\pgfpathlineto{\pgfqpoint{1.850730in}{3.185726in}}%
\pgfpathlineto{\pgfqpoint{1.845993in}{3.187735in}}%
\pgfpathlineto{\pgfqpoint{1.839733in}{3.190323in}}%
\pgfpathlineto{\pgfqpoint{1.835613in}{3.191987in}}%
\pgfpathlineto{\pgfqpoint{1.833472in}{3.192869in}}%
\pgfpathlineto{\pgfqpoint{1.827212in}{3.195393in}}%
\pgfpathlineto{\pgfqpoint{1.820952in}{3.197849in}}%
\pgfpathlineto{\pgfqpoint{1.819915in}{3.198247in}}%
\pgfpathlineto{\pgfqpoint{1.814691in}{3.200297in}}%
\pgfpathlineto{\pgfqpoint{1.808431in}{3.202688in}}%
\pgfpathlineto{\pgfqpoint{1.803540in}{3.204508in}}%
\pgfpathlineto{\pgfqpoint{1.802170in}{3.205028in}}%
\pgfpathlineto{\pgfqpoint{1.795910in}{3.207354in}}%
\pgfpathlineto{\pgfqpoint{1.789649in}{3.209615in}}%
\pgfpathlineto{\pgfqpoint{1.786372in}{3.210768in}}%
\pgfpathlineto{\pgfqpoint{1.783389in}{3.211841in}}%
\pgfpathlineto{\pgfqpoint{1.777129in}{3.214035in}}%
\pgfpathlineto{\pgfqpoint{1.770868in}{3.216165in}}%
\pgfpathlineto{\pgfqpoint{1.768262in}{3.217028in}}%
\pgfpathlineto{\pgfqpoint{1.764608in}{3.218267in}}%
\pgfpathlineto{\pgfqpoint{1.758347in}{3.220328in}}%
\pgfpathlineto{\pgfqpoint{1.752087in}{3.222327in}}%
\pgfpathlineto{\pgfqpoint{1.748986in}{3.223289in}}%
\pgfpathlineto{\pgfqpoint{1.745826in}{3.224291in}}%
\pgfpathlineto{\pgfqpoint{1.739566in}{3.226219in}}%
\pgfpathlineto{\pgfqpoint{1.733306in}{3.228084in}}%
\pgfpathlineto{\pgfqpoint{1.728227in}{3.229549in}}%
\pgfpathlineto{\pgfqpoint{1.727045in}{3.229898in}}%
\pgfpathlineto{\pgfqpoint{1.720785in}{3.231688in}}%
\pgfpathlineto{\pgfqpoint{1.714524in}{3.233416in}}%
\pgfpathlineto{\pgfqpoint{1.708264in}{3.235083in}}%
\pgfpathlineto{\pgfqpoint{1.705436in}{3.235810in}}%
\pgfpathlineto{\pgfqpoint{1.702003in}{3.236712in}}%
\pgfpathlineto{\pgfqpoint{1.695743in}{3.238297in}}%
\pgfpathlineto{\pgfqpoint{1.689482in}{3.239820in}}%
\pgfpathlineto{\pgfqpoint{1.683222in}{3.241281in}}%
\pgfpathlineto{\pgfqpoint{1.679694in}{3.242070in}}%
\pgfpathlineto{\pgfqpoint{1.676962in}{3.242696in}}%
\pgfpathlineto{\pgfqpoint{1.670701in}{3.244066in}}%
\pgfpathlineto{\pgfqpoint{1.664441in}{3.245372in}}%
\pgfpathlineto{\pgfqpoint{1.658180in}{3.246614in}}%
\pgfpathlineto{\pgfqpoint{1.651920in}{3.247794in}}%
\pgfpathlineto{\pgfqpoint{1.648900in}{3.248331in}}%
\pgfpathlineto{\pgfqpoint{1.645659in}{3.248921in}}%
\pgfpathlineto{\pgfqpoint{1.639399in}{3.249993in}}%
\pgfpathlineto{\pgfqpoint{1.633139in}{3.250997in}}%
\pgfpathlineto{\pgfqpoint{1.626878in}{3.251933in}}%
\pgfpathlineto{\pgfqpoint{1.620618in}{3.252800in}}%
\pgfpathlineto{\pgfqpoint{1.614357in}{3.253597in}}%
\pgfpathlineto{\pgfqpoint{1.608097in}{3.254322in}}%
\pgfpathlineto{\pgfqpoint{1.605502in}{3.254591in}}%
\pgfpathlineto{\pgfqpoint{1.601836in}{3.254981in}}%
\pgfpathlineto{\pgfqpoint{1.595576in}{3.255567in}}%
\pgfpathlineto{\pgfqpoint{1.589316in}{3.256075in}}%
\pgfpathlineto{\pgfqpoint{1.583055in}{3.256502in}}%
\pgfpathlineto{\pgfqpoint{1.576795in}{3.256845in}}%
\pgfpathlineto{\pgfqpoint{1.570534in}{3.257103in}}%
\pgfpathlineto{\pgfqpoint{1.564274in}{3.257272in}}%
\pgfpathlineto{\pgfqpoint{1.558013in}{3.257348in}}%
\pgfpathlineto{\pgfqpoint{1.551753in}{3.257329in}}%
\pgfpathlineto{\pgfqpoint{1.545492in}{3.257210in}}%
\pgfpathlineto{\pgfqpoint{1.539232in}{3.256986in}}%
\pgfpathlineto{\pgfqpoint{1.532972in}{3.256654in}}%
\pgfpathlineto{\pgfqpoint{1.526711in}{3.256207in}}%
\pgfpathlineto{\pgfqpoint{1.520451in}{3.255640in}}%
\pgfpathlineto{\pgfqpoint{1.514190in}{3.254947in}}%
\pgfpathlineto{\pgfqpoint{1.511465in}{3.254591in}}%
\pgfpathlineto{\pgfqpoint{1.507930in}{3.254122in}}%
\pgfpathlineto{\pgfqpoint{1.501669in}{3.253158in}}%
\pgfpathlineto{\pgfqpoint{1.495409in}{3.252043in}}%
\pgfpathlineto{\pgfqpoint{1.489149in}{3.250769in}}%
\pgfpathlineto{\pgfqpoint{1.482888in}{3.249325in}}%
\pgfpathlineto{\pgfqpoint{1.479019in}{3.248331in}}%
\pgfpathlineto{\pgfqpoint{1.476628in}{3.247699in}}%
\pgfpathlineto{\pgfqpoint{1.470367in}{3.245875in}}%
\pgfpathlineto{\pgfqpoint{1.464107in}{3.243841in}}%
\pgfpathlineto{\pgfqpoint{1.459172in}{3.242070in}}%
\pgfpathlineto{\pgfqpoint{1.457846in}{3.241578in}}%
\pgfpathlineto{\pgfqpoint{1.451586in}{3.239054in}}%
\pgfpathlineto{\pgfqpoint{1.445326in}{3.236261in}}%
\pgfpathlineto{\pgfqpoint{1.444383in}{3.235810in}}%
\pgfpathlineto{\pgfqpoint{1.439065in}{3.233145in}}%
\pgfpathlineto{\pgfqpoint{1.432805in}{3.229698in}}%
\pgfpathlineto{\pgfqpoint{1.432551in}{3.229549in}}%
\pgfpathlineto{\pgfqpoint{1.426544in}{3.225833in}}%
\pgfpathlineto{\pgfqpoint{1.422758in}{3.223289in}}%
\pgfpathlineto{\pgfqpoint{1.420284in}{3.221526in}}%
\pgfpathlineto{\pgfqpoint{1.414427in}{3.217028in}}%
\pgfpathlineto{\pgfqpoint{1.414023in}{3.216698in}}%
\pgfpathlineto{\pgfqpoint{1.407763in}{3.211222in}}%
\pgfpathlineto{\pgfqpoint{1.407275in}{3.210768in}}%
\pgfpathlineto{\pgfqpoint{1.401503in}{3.204971in}}%
\pgfpathlineto{\pgfqpoint{1.401068in}{3.204508in}}%
\pgfpathlineto{\pgfqpoint{1.395650in}{3.198247in}}%
\pgfpathlineto{\pgfqpoint{1.395242in}{3.197732in}}%
\pgfpathlineto{\pgfqpoint{1.390910in}{3.191987in}}%
\pgfpathlineto{\pgfqpoint{1.388982in}{3.189155in}}%
\pgfpathlineto{\pgfqpoint{1.386750in}{3.185726in}}%
\pgfpathlineto{\pgfqpoint{1.383101in}{3.179466in}}%
\pgfpathlineto{\pgfqpoint{1.382721in}{3.178744in}}%
\pgfpathlineto{\pgfqpoint{1.379915in}{3.173205in}}%
\pgfpathlineto{\pgfqpoint{1.377137in}{3.166945in}}%
\pgfpathlineto{\pgfqpoint{1.376461in}{3.165221in}}%
\pgfpathlineto{\pgfqpoint{1.374734in}{3.160684in}}%
\pgfpathlineto{\pgfqpoint{1.372672in}{3.154424in}}%
\pgfpathlineto{\pgfqpoint{1.370925in}{3.148164in}}%
\pgfpathlineto{\pgfqpoint{1.370200in}{3.145088in}}%
\pgfpathlineto{\pgfqpoint{1.369467in}{3.141903in}}%
\pgfpathlineto{\pgfqpoint{1.368276in}{3.135643in}}%
\pgfpathlineto{\pgfqpoint{1.367339in}{3.129382in}}%
\pgfpathlineto{\pgfqpoint{1.366637in}{3.123122in}}%
\pgfpathlineto{\pgfqpoint{1.366158in}{3.116861in}}%
\pgfpathlineto{\pgfqpoint{1.365888in}{3.110601in}}%
\pgfpathlineto{\pgfqpoint{1.365816in}{3.104341in}}%
\pgfpathlineto{\pgfqpoint{1.365932in}{3.098080in}}%
\pgfpathlineto{\pgfqpoint{1.366226in}{3.091820in}}%
\pgfpathlineto{\pgfqpoint{1.366692in}{3.085559in}}%
\pgfpathlineto{\pgfqpoint{1.367322in}{3.079299in}}%
\pgfpathlineto{\pgfqpoint{1.368109in}{3.073038in}}%
\pgfpathlineto{\pgfqpoint{1.369049in}{3.066778in}}%
\pgfpathlineto{\pgfqpoint{1.370137in}{3.060518in}}%
\pgfpathlineto{\pgfqpoint{1.370200in}{3.060198in}}%
\pgfpathlineto{\pgfqpoint{1.371352in}{3.054257in}}%
\pgfpathlineto{\pgfqpoint{1.372704in}{3.047997in}}%
\pgfpathlineto{\pgfqpoint{1.374190in}{3.041736in}}%
\pgfpathlineto{\pgfqpoint{1.375809in}{3.035476in}}%
\pgfpathlineto{\pgfqpoint{1.376461in}{3.033151in}}%
\pgfpathlineto{\pgfqpoint{1.377541in}{3.029215in}}%
\pgfpathlineto{\pgfqpoint{1.379388in}{3.022955in}}%
\pgfpathlineto{\pgfqpoint{1.381360in}{3.016694in}}%
\pgfpathlineto{\pgfqpoint{1.382721in}{3.012633in}}%
\pgfpathlineto{\pgfqpoint{1.383443in}{3.010434in}}%
\pgfpathlineto{\pgfqpoint{1.385624in}{3.004174in}}%
\pgfpathlineto{\pgfqpoint{1.387926in}{2.997913in}}%
\pgfpathlineto{\pgfqpoint{1.388982in}{2.995186in}}%
\pgfpathlineto{\pgfqpoint{1.390323in}{2.991653in}}%
\pgfpathlineto{\pgfqpoint{1.392820in}{2.985392in}}%
\pgfpathlineto{\pgfqpoint{1.395242in}{2.979596in}}%
\pgfpathlineto{\pgfqpoint{1.395432in}{2.979132in}}%
\pgfpathlineto{\pgfqpoint{1.398116in}{2.972871in}}%
\pgfpathlineto{\pgfqpoint{1.400918in}{2.966611in}}%
\pgfpathlineto{\pgfqpoint{1.401503in}{2.965357in}}%
\pgfpathlineto{\pgfqpoint{1.403794in}{2.960351in}}%
\pgfpathlineto{\pgfqpoint{1.406778in}{2.954090in}}%
\pgfpathlineto{\pgfqpoint{1.407763in}{2.952098in}}%
\pgfpathlineto{\pgfqpoint{1.409839in}{2.947830in}}%
\pgfpathlineto{\pgfqpoint{1.412999in}{2.941569in}}%
\pgfpathlineto{\pgfqpoint{1.414023in}{2.939609in}}%
\pgfpathlineto{\pgfqpoint{1.416235in}{2.935309in}}%
\pgfpathlineto{\pgfqpoint{1.419571in}{2.929048in}}%
\pgfpathlineto{\pgfqpoint{1.420284in}{2.927753in}}%
\pgfpathlineto{\pgfqpoint{1.422973in}{2.922788in}}%
\pgfpathlineto{\pgfqpoint{1.426485in}{2.916528in}}%
\pgfpathlineto{\pgfqpoint{1.426544in}{2.916424in}}%
\pgfpathlineto{\pgfqpoint{1.430044in}{2.910267in}}%
\pgfpathlineto{\pgfqpoint{1.432805in}{2.905569in}}%
\pgfpathlineto{\pgfqpoint{1.433710in}{2.904007in}}%
\pgfpathlineto{\pgfqpoint{1.437442in}{2.897746in}}%
\pgfpathlineto{\pgfqpoint{1.439065in}{2.895102in}}%
\pgfpathlineto{\pgfqpoint{1.441255in}{2.891486in}}%
\pgfpathlineto{\pgfqpoint{1.445164in}{2.885225in}}%
\pgfpathlineto{\pgfqpoint{1.445326in}{2.884973in}}%
\pgfpathlineto{\pgfqpoint{1.449119in}{2.878965in}}%
\pgfpathlineto{\pgfqpoint{1.451586in}{2.875170in}}%
\pgfpathlineto{\pgfqpoint{1.453170in}{2.872705in}}%
\pgfpathlineto{\pgfqpoint{1.457300in}{2.866444in}}%
\pgfpathlineto{\pgfqpoint{1.457846in}{2.865635in}}%
\pgfpathlineto{\pgfqpoint{1.461485in}{2.860184in}}%
\pgfpathlineto{\pgfqpoint{1.464107in}{2.856360in}}%
\pgfpathlineto{\pgfqpoint{1.465760in}{2.853923in}}%
\pgfpathlineto{\pgfqpoint{1.470117in}{2.847663in}}%
\pgfpathlineto{\pgfqpoint{1.470367in}{2.847310in}}%
\pgfpathlineto{\pgfqpoint{1.474521in}{2.841402in}}%
\pgfpathlineto{\pgfqpoint{1.476628in}{2.838478in}}%
\pgfpathlineto{\pgfqpoint{1.479010in}{2.835142in}}%
\pgfpathlineto{\pgfqpoint{1.482888in}{2.829840in}}%
\pgfpathlineto{\pgfqpoint{1.483583in}{2.828881in}}%
\pgfpathlineto{\pgfqpoint{1.488214in}{2.822621in}}%
\pgfpathlineto{\pgfqpoint{1.489149in}{2.821384in}}%
\pgfpathlineto{\pgfqpoint{1.492910in}{2.816361in}}%
\pgfpathlineto{\pgfqpoint{1.495409in}{2.813097in}}%
\pgfpathlineto{\pgfqpoint{1.497687in}{2.810100in}}%
\pgfpathlineto{\pgfqpoint{1.501669in}{2.804971in}}%
\pgfpathlineto{\pgfqpoint{1.502542in}{2.803840in}}%
\pgfpathlineto{\pgfqpoint{1.507462in}{2.797579in}}%
\pgfpathlineto{\pgfqpoint{1.507930in}{2.796994in}}%
\pgfpathlineto{\pgfqpoint{1.512437in}{2.791319in}}%
\pgfpathlineto{\pgfqpoint{1.514190in}{2.789155in}}%
\pgfpathlineto{\pgfqpoint{1.517489in}{2.785058in}}%
\pgfpathlineto{\pgfqpoint{1.520451in}{2.781450in}}%
\pgfpathlineto{\pgfqpoint{1.522616in}{2.778798in}}%
\pgfpathlineto{\pgfqpoint{1.526711in}{2.773873in}}%
\pgfpathlineto{\pgfqpoint{1.527816in}{2.772538in}}%
\pgfpathlineto{\pgfqpoint{1.532972in}{2.766418in}}%
\pgfpathlineto{\pgfqpoint{1.533090in}{2.766277in}}%
\pgfpathlineto{\pgfqpoint{1.538414in}{2.760017in}}%
\pgfpathlineto{\pgfqpoint{1.539232in}{2.759070in}}%
\pgfpathlineto{\pgfqpoint{1.543808in}{2.753756in}}%
\pgfpathlineto{\pgfqpoint{1.545492in}{2.751831in}}%
\pgfpathlineto{\pgfqpoint{1.549274in}{2.747496in}}%
\pgfpathlineto{\pgfqpoint{1.551753in}{2.744698in}}%
\pgfpathlineto{\pgfqpoint{1.554811in}{2.741235in}}%
\pgfpathlineto{\pgfqpoint{1.558013in}{2.737664in}}%
\pgfpathlineto{\pgfqpoint{1.560419in}{2.734975in}}%
\pgfpathlineto{\pgfqpoint{1.564274in}{2.730727in}}%
\pgfpathlineto{\pgfqpoint{1.566097in}{2.728715in}}%
\pgfpathlineto{\pgfqpoint{1.570534in}{2.723882in}}%
\pgfpathlineto{\pgfqpoint{1.571844in}{2.722454in}}%
\pgfpathlineto{\pgfqpoint{1.576795in}{2.717126in}}%
\pgfpathlineto{\pgfqpoint{1.577660in}{2.716194in}}%
\pgfpathlineto{\pgfqpoint{1.583055in}{2.710455in}}%
\pgfpathlineto{\pgfqpoint{1.583545in}{2.709933in}}%
\pgfpathlineto{\pgfqpoint{1.589316in}{2.703865in}}%
\pgfpathlineto{\pgfqpoint{1.589498in}{2.703673in}}%
\pgfpathlineto{\pgfqpoint{1.595518in}{2.697412in}}%
\pgfpathlineto{\pgfqpoint{1.595576in}{2.697353in}}%
\pgfpathlineto{\pgfqpoint{1.601603in}{2.691152in}}%
\pgfpathlineto{\pgfqpoint{1.601836in}{2.690914in}}%
\pgfpathlineto{\pgfqpoint{1.607757in}{2.684891in}}%
\pgfpathlineto{\pgfqpoint{1.608097in}{2.684549in}}%
\pgfpathlineto{\pgfqpoint{1.613981in}{2.678631in}}%
\pgfpathlineto{\pgfqpoint{1.614357in}{2.678256in}}%
\pgfpathlineto{\pgfqpoint{1.620274in}{2.672371in}}%
\pgfpathlineto{\pgfqpoint{1.620618in}{2.672032in}}%
\pgfpathlineto{\pgfqpoint{1.626636in}{2.666110in}}%
\pgfpathlineto{\pgfqpoint{1.626878in}{2.665874in}}%
\pgfpathlineto{\pgfqpoint{1.633068in}{2.659850in}}%
\pgfpathlineto{\pgfqpoint{1.633139in}{2.659782in}}%
\pgfpathlineto{\pgfqpoint{1.639399in}{2.653749in}}%
\pgfpathlineto{\pgfqpoint{1.639565in}{2.653589in}}%
\pgfpathlineto{\pgfqpoint{1.645659in}{2.647774in}}%
\pgfpathlineto{\pgfqpoint{1.646128in}{2.647329in}}%
\pgfpathlineto{\pgfqpoint{1.651920in}{2.641857in}}%
\pgfpathlineto{\pgfqpoint{1.652758in}{2.641068in}}%
\pgfpathlineto{\pgfqpoint{1.658180in}{2.635995in}}%
\pgfpathlineto{\pgfqpoint{1.659455in}{2.634808in}}%
\pgfpathlineto{\pgfqpoint{1.664441in}{2.630188in}}%
\pgfpathlineto{\pgfqpoint{1.666220in}{2.628548in}}%
\pgfpathlineto{\pgfqpoint{1.670701in}{2.624433in}}%
\pgfpathlineto{\pgfqpoint{1.673051in}{2.622287in}}%
\pgfpathlineto{\pgfqpoint{1.676962in}{2.618729in}}%
\pgfpathlineto{\pgfqpoint{1.679948in}{2.616027in}}%
\pgfpathlineto{\pgfqpoint{1.683222in}{2.613074in}}%
\pgfpathlineto{\pgfqpoint{1.686911in}{2.609766in}}%
\pgfpathlineto{\pgfqpoint{1.689482in}{2.607466in}}%
\pgfpathlineto{\pgfqpoint{1.693938in}{2.603506in}}%
\pgfpathlineto{\pgfqpoint{1.695743in}{2.601905in}}%
\pgfpathlineto{\pgfqpoint{1.701029in}{2.597245in}}%
\pgfpathlineto{\pgfqpoint{1.702003in}{2.596388in}}%
\pgfpathlineto{\pgfqpoint{1.708183in}{2.590985in}}%
\pgfpathlineto{\pgfqpoint{1.708264in}{2.590914in}}%
\pgfpathlineto{\pgfqpoint{1.714524in}{2.585461in}}%
\pgfpathlineto{\pgfqpoint{1.715376in}{2.584725in}}%
\pgfpathlineto{\pgfqpoint{1.720785in}{2.580044in}}%
\pgfpathlineto{\pgfqpoint{1.722624in}{2.578464in}}%
\pgfpathlineto{\pgfqpoint{1.727045in}{2.574662in}}%
\pgfpathlineto{\pgfqpoint{1.729927in}{2.572204in}}%
\pgfpathlineto{\pgfqpoint{1.733306in}{2.569314in}}%
\pgfpathlineto{\pgfqpoint{1.737280in}{2.565943in}}%
\pgfpathlineto{\pgfqpoint{1.739566in}{2.563998in}}%
\pgfpathlineto{\pgfqpoint{1.744680in}{2.559683in}}%
\pgfpathlineto{\pgfqpoint{1.745826in}{2.558712in}}%
\pgfpathlineto{\pgfqpoint{1.752087in}{2.553452in}}%
\pgfpathlineto{\pgfqpoint{1.752122in}{2.553422in}}%
\pgfpathlineto{\pgfqpoint{1.758347in}{2.548186in}}%
\pgfpathlineto{\pgfqpoint{1.759575in}{2.547162in}}%
\pgfpathlineto{\pgfqpoint{1.764608in}{2.542939in}}%
\pgfpathlineto{\pgfqpoint{1.767058in}{2.540902in}}%
\pgfpathlineto{\pgfqpoint{1.770868in}{2.537710in}}%
\pgfpathlineto{\pgfqpoint{1.774565in}{2.534641in}}%
\pgfpathlineto{\pgfqpoint{1.777129in}{2.532494in}}%
\pgfpathlineto{\pgfqpoint{1.782086in}{2.528381in}}%
\pgfpathlineto{\pgfqpoint{1.783389in}{2.527289in}}%
\pgfpathlineto{\pgfqpoint{1.789613in}{2.522120in}}%
\pgfpathlineto{\pgfqpoint{1.789649in}{2.522090in}}%
\pgfpathlineto{\pgfqpoint{1.795910in}{2.516855in}}%
\pgfpathlineto{\pgfqpoint{1.797110in}{2.515860in}}%
\pgfpathlineto{\pgfqpoint{1.802170in}{2.511612in}}%
\pgfpathlineto{\pgfqpoint{1.804591in}{2.509599in}}%
\pgfpathlineto{\pgfqpoint{1.808431in}{2.506359in}}%
\pgfpathlineto{\pgfqpoint{1.812042in}{2.503339in}}%
\pgfpathlineto{\pgfqpoint{1.814691in}{2.501088in}}%
\pgfpathlineto{\pgfqpoint{1.819451in}{2.497078in}}%
\pgfpathlineto{\pgfqpoint{1.820952in}{2.495791in}}%
\pgfpathlineto{\pgfqpoint{1.826801in}{2.490818in}}%
\pgfpathlineto{\pgfqpoint{1.827212in}{2.490462in}}%
\pgfpathlineto{\pgfqpoint{1.833472in}{2.485065in}}%
\pgfpathlineto{\pgfqpoint{1.834065in}{2.484558in}}%
\pgfpathlineto{\pgfqpoint{1.839733in}{2.479596in}}%
\pgfpathlineto{\pgfqpoint{1.841230in}{2.478297in}}%
\pgfpathlineto{\pgfqpoint{1.845993in}{2.474058in}}%
\pgfpathlineto{\pgfqpoint{1.848284in}{2.472037in}}%
\pgfpathlineto{\pgfqpoint{1.852254in}{2.468435in}}%
\pgfpathlineto{\pgfqpoint{1.855208in}{2.465776in}}%
\pgfpathlineto{\pgfqpoint{1.858514in}{2.462708in}}%
\pgfpathlineto{\pgfqpoint{1.861980in}{2.459516in}}%
\pgfpathlineto{\pgfqpoint{1.864775in}{2.456854in}}%
\pgfpathlineto{\pgfqpoint{1.868579in}{2.453255in}}%
\pgfpathlineto{\pgfqpoint{1.871035in}{2.450844in}}%
\pgfpathlineto{\pgfqpoint{1.874983in}{2.446995in}}%
\pgfpathlineto{\pgfqpoint{1.877295in}{2.444644in}}%
\pgfpathlineto{\pgfqpoint{1.881167in}{2.440735in}}%
\pgfpathlineto{\pgfqpoint{1.883556in}{2.438209in}}%
\pgfpathlineto{\pgfqpoint{1.887109in}{2.434474in}}%
\pgfpathlineto{\pgfqpoint{1.889816in}{2.431480in}}%
\pgfpathlineto{\pgfqpoint{1.892785in}{2.428214in}}%
\pgfpathlineto{\pgfqpoint{1.896077in}{2.424382in}}%
\pgfpathlineto{\pgfqpoint{1.898173in}{2.421953in}}%
\pgfpathlineto{\pgfqpoint{1.902337in}{2.416815in}}%
\pgfpathlineto{\pgfqpoint{1.903251in}{2.415693in}}%
\pgfpathlineto{\pgfqpoint{1.907986in}{2.409432in}}%
\pgfpathlineto{\pgfqpoint{1.908598in}{2.408551in}}%
\pgfpathlineto{\pgfqpoint{1.912347in}{2.403172in}}%
\pgfpathlineto{\pgfqpoint{1.914858in}{2.399221in}}%
\pgfpathlineto{\pgfqpoint{1.916331in}{2.396912in}}%
\pgfpathlineto{\pgfqpoint{1.919899in}{2.390651in}}%
\pgfpathlineto{\pgfqpoint{1.921119in}{2.388208in}}%
\pgfpathlineto{\pgfqpoint{1.923027in}{2.384391in}}%
\pgfpathlineto{\pgfqpoint{1.925696in}{2.378130in}}%
\pgfpathlineto{\pgfqpoint{1.927379in}{2.373320in}}%
\pgfpathlineto{\pgfqpoint{1.927886in}{2.371870in}}%
\pgfpathlineto{\pgfqpoint{1.929555in}{2.365609in}}%
\pgfpathlineto{\pgfqpoint{1.930699in}{2.359349in}}%
\pgfpathlineto{\pgfqpoint{1.931293in}{2.353088in}}%
\pgfpathlineto{\pgfqpoint{1.931312in}{2.346828in}}%
\pgfpathlineto{\pgfqpoint{1.930737in}{2.340568in}}%
\pgfpathlineto{\pgfqpoint{1.929552in}{2.334307in}}%
\pgfpathlineto{\pgfqpoint{1.927746in}{2.328047in}}%
\pgfpathlineto{\pgfqpoint{1.927379in}{2.327097in}}%
\pgfpathlineto{\pgfqpoint{1.925281in}{2.321786in}}%
\pgfpathlineto{\pgfqpoint{1.922174in}{2.315526in}}%
\pgfpathlineto{\pgfqpoint{1.921119in}{2.313751in}}%
\pgfpathlineto{\pgfqpoint{1.918397in}{2.309265in}}%
\pgfpathlineto{\pgfqpoint{1.914858in}{2.304247in}}%
\pgfpathlineto{\pgfqpoint{1.913966in}{2.303005in}}%
\pgfpathlineto{\pgfqpoint{1.908870in}{2.296745in}}%
\pgfpathlineto{\pgfqpoint{1.908598in}{2.296444in}}%
\pgfpathlineto{\pgfqpoint{1.903115in}{2.290484in}}%
\pgfpathlineto{\pgfqpoint{1.902337in}{2.289714in}}%
\pgfpathlineto{\pgfqpoint{1.896724in}{2.284224in}}%
\pgfpathlineto{\pgfqpoint{1.896077in}{2.283640in}}%
\pgfpathlineto{\pgfqpoint{1.889816in}{2.278048in}}%
\pgfpathlineto{\pgfqpoint{1.889720in}{2.277963in}}%
\pgfpathlineto{\pgfqpoint{1.883556in}{2.272877in}}%
\pgfpathlineto{\pgfqpoint{1.882122in}{2.271703in}}%
\pgfpathlineto{\pgfqpoint{1.877295in}{2.267983in}}%
\pgfpathlineto{\pgfqpoint{1.873982in}{2.265442in}}%
\pgfpathlineto{\pgfqpoint{1.871035in}{2.263301in}}%
\pgfpathlineto{\pgfqpoint{1.865349in}{2.259182in}}%
\pgfpathlineto{\pgfqpoint{1.864775in}{2.258785in}}%
\pgfpathlineto{\pgfqpoint{1.858514in}{2.254475in}}%
\pgfpathlineto{\pgfqpoint{1.856255in}{2.252922in}}%
\pgfpathlineto{\pgfqpoint{1.852254in}{2.250281in}}%
\pgfpathlineto{\pgfqpoint{1.846779in}{2.246661in}}%
\pgfpathlineto{\pgfqpoint{1.845993in}{2.246160in}}%
\pgfpathlineto{\pgfqpoint{1.839733in}{2.242172in}}%
\pgfpathlineto{\pgfqpoint{1.836961in}{2.240401in}}%
\pgfpathlineto{\pgfqpoint{1.833472in}{2.238241in}}%
\pgfpathlineto{\pgfqpoint{1.827212in}{2.234345in}}%
\pgfpathlineto{\pgfqpoint{1.826883in}{2.234140in}}%
\pgfpathlineto{\pgfqpoint{1.820952in}{2.230547in}}%
\pgfpathlineto{\pgfqpoint{1.816583in}{2.227880in}}%
\pgfpathlineto{\pgfqpoint{1.814691in}{2.226752in}}%
\pgfpathlineto{\pgfqpoint{1.808431in}{2.223003in}}%
\pgfpathlineto{\pgfqpoint{1.806135in}{2.221619in}}%
\pgfpathlineto{\pgfqpoint{1.802170in}{2.219279in}}%
\pgfpathlineto{\pgfqpoint{1.795910in}{2.215546in}}%
\pgfpathlineto{\pgfqpoint{1.795596in}{2.215359in}}%
\pgfpathlineto{\pgfqpoint{1.789649in}{2.211865in}}%
\pgfpathlineto{\pgfqpoint{1.784999in}{2.209098in}}%
\pgfpathlineto{\pgfqpoint{1.783389in}{2.208155in}}%
\pgfpathlineto{\pgfqpoint{1.777129in}{2.204462in}}%
\pgfpathlineto{\pgfqpoint{1.774402in}{2.202838in}}%
\pgfpathlineto{\pgfqpoint{1.770868in}{2.200760in}}%
\pgfpathlineto{\pgfqpoint{1.764608in}{2.197033in}}%
\pgfpathlineto{\pgfqpoint{1.763847in}{2.196578in}}%
\pgfpathlineto{\pgfqpoint{1.758347in}{2.193319in}}%
\pgfpathlineto{\pgfqpoint{1.753355in}{2.190317in}}%
\pgfpathlineto{\pgfqpoint{1.752087in}{2.189561in}}%
\pgfpathlineto{\pgfqpoint{1.745826in}{2.185800in}}%
\pgfpathlineto{\pgfqpoint{1.742959in}{2.184057in}}%
\pgfpathlineto{\pgfqpoint{1.739566in}{2.182008in}}%
\pgfpathlineto{\pgfqpoint{1.733306in}{2.178175in}}%
\pgfpathlineto{\pgfqpoint{1.732691in}{2.177796in}}%
\pgfpathlineto{\pgfqpoint{1.727045in}{2.174334in}}%
\pgfpathlineto{\pgfqpoint{1.722552in}{2.171536in}}%
\pgfpathlineto{\pgfqpoint{1.720785in}{2.170439in}}%
\pgfpathlineto{\pgfqpoint{1.714524in}{2.166514in}}%
\pgfpathlineto{\pgfqpoint{1.712570in}{2.165275in}}%
\pgfpathlineto{\pgfqpoint{1.708264in}{2.162551in}}%
\pgfpathlineto{\pgfqpoint{1.702757in}{2.159015in}}%
\pgfpathlineto{\pgfqpoint{1.702003in}{2.158531in}}%
\pgfpathlineto{\pgfqpoint{1.695743in}{2.154481in}}%
\pgfpathlineto{\pgfqpoint{1.693107in}{2.152755in}}%
\pgfpathlineto{\pgfqpoint{1.689482in}{2.150377in}}%
\pgfpathlineto{\pgfqpoint{1.683649in}{2.146494in}}%
\pgfpathlineto{\pgfqpoint{1.683222in}{2.146209in}}%
\pgfpathlineto{\pgfqpoint{1.676962in}{2.142004in}}%
\pgfpathlineto{\pgfqpoint{1.674359in}{2.140234in}}%
\pgfpathlineto{\pgfqpoint{1.670701in}{2.137735in}}%
\pgfpathlineto{\pgfqpoint{1.665268in}{2.133973in}}%
\pgfpathlineto{\pgfqpoint{1.664441in}{2.133397in}}%
\pgfpathlineto{\pgfqpoint{1.658180in}{2.129007in}}%
\pgfpathlineto{\pgfqpoint{1.656355in}{2.127713in}}%
\pgfpathlineto{\pgfqpoint{1.651920in}{2.124549in}}%
\pgfpathlineto{\pgfqpoint{1.647633in}{2.121452in}}%
\pgfpathlineto{\pgfqpoint{1.645659in}{2.120015in}}%
\pgfpathlineto{\pgfqpoint{1.639399in}{2.115408in}}%
\pgfpathlineto{\pgfqpoint{1.639107in}{2.115192in}}%
\pgfpathlineto{\pgfqpoint{1.633139in}{2.110734in}}%
\pgfpathlineto{\pgfqpoint{1.630752in}{2.108932in}}%
\pgfpathlineto{\pgfqpoint{1.626878in}{2.105975in}}%
\pgfpathlineto{\pgfqpoint{1.622592in}{2.102671in}}%
\pgfpathlineto{\pgfqpoint{1.620618in}{2.101131in}}%
\pgfpathlineto{\pgfqpoint{1.614624in}{2.096411in}}%
\pgfpathlineto{\pgfqpoint{1.614357in}{2.096198in}}%
\pgfpathlineto{\pgfqpoint{1.608097in}{2.091177in}}%
\pgfpathlineto{\pgfqpoint{1.606826in}{2.090150in}}%
\pgfpathlineto{\pgfqpoint{1.601836in}{2.086059in}}%
\pgfpathlineto{\pgfqpoint{1.599211in}{2.083890in}}%
\pgfpathlineto{\pgfqpoint{1.595576in}{2.080838in}}%
\pgfpathlineto{\pgfqpoint{1.591781in}{2.077629in}}%
\pgfpathlineto{\pgfqpoint{1.589316in}{2.075510in}}%
\pgfpathlineto{\pgfqpoint{1.584531in}{2.071369in}}%
\pgfpathlineto{\pgfqpoint{1.583055in}{2.070068in}}%
\pgfpathlineto{\pgfqpoint{1.577460in}{2.065109in}}%
\pgfpathlineto{\pgfqpoint{1.576795in}{2.064508in}}%
\pgfpathlineto{\pgfqpoint{1.570564in}{2.058848in}}%
\pgfpathlineto{\pgfqpoint{1.570534in}{2.058820in}}%
\pgfpathlineto{\pgfqpoint{1.564274in}{2.052997in}}%
\pgfpathlineto{\pgfqpoint{1.563835in}{2.052588in}}%
\pgfpathlineto{\pgfqpoint{1.558013in}{2.047030in}}%
\pgfpathlineto{\pgfqpoint{1.557280in}{2.046327in}}%
\pgfpathlineto{\pgfqpoint{1.551753in}{2.040907in}}%
\pgfpathlineto{\pgfqpoint{1.550898in}{2.040067in}}%
\pgfpathlineto{\pgfqpoint{1.545492in}{2.034619in}}%
\pgfpathlineto{\pgfqpoint{1.544688in}{2.033806in}}%
\pgfpathlineto{\pgfqpoint{1.539232in}{2.028152in}}%
\pgfpathlineto{\pgfqpoint{1.538648in}{2.027546in}}%
\pgfpathlineto{\pgfqpoint{1.532972in}{2.021491in}}%
\pgfpathlineto{\pgfqpoint{1.532779in}{2.021285in}}%
\pgfpathlineto{\pgfqpoint{1.527075in}{2.015025in}}%
\pgfpathlineto{\pgfqpoint{1.526711in}{2.014614in}}%
\pgfpathlineto{\pgfqpoint{1.521535in}{2.008765in}}%
\pgfpathlineto{\pgfqpoint{1.520451in}{2.007498in}}%
\pgfpathlineto{\pgfqpoint{1.516165in}{2.002504in}}%
\pgfpathlineto{\pgfqpoint{1.514190in}{2.000123in}}%
\pgfpathlineto{\pgfqpoint{1.510964in}{1.996244in}}%
\pgfpathlineto{\pgfqpoint{1.507930in}{1.992462in}}%
\pgfpathlineto{\pgfqpoint{1.505934in}{1.989983in}}%
\pgfpathlineto{\pgfqpoint{1.501669in}{1.984483in}}%
\pgfpathlineto{\pgfqpoint{1.501077in}{1.983723in}}%
\pgfpathlineto{\pgfqpoint{1.496380in}{1.977462in}}%
\pgfpathlineto{\pgfqpoint{1.495409in}{1.976114in}}%
\pgfpathlineto{\pgfqpoint{1.491849in}{1.971202in}}%
\pgfpathlineto{\pgfqpoint{1.489149in}{1.967309in}}%
\pgfpathlineto{\pgfqpoint{1.487496in}{1.964942in}}%
\pgfpathlineto{\pgfqpoint{1.483317in}{1.958681in}}%
\pgfpathlineto{\pgfqpoint{1.482888in}{1.958008in}}%
\pgfpathlineto{\pgfqpoint{1.479295in}{1.952421in}}%
\pgfpathlineto{\pgfqpoint{1.476628in}{1.948055in}}%
\pgfpathlineto{\pgfqpoint{1.475460in}{1.946160in}}%
\pgfpathlineto{\pgfqpoint{1.471794in}{1.939900in}}%
\pgfpathlineto{\pgfqpoint{1.470367in}{1.937326in}}%
\pgfpathlineto{\pgfqpoint{1.468303in}{1.933639in}}%
\pgfpathlineto{\pgfqpoint{1.464997in}{1.927379in}}%
\pgfpathlineto{\pgfqpoint{1.464107in}{1.925587in}}%
\pgfpathlineto{\pgfqpoint{1.461862in}{1.921119in}}%
\pgfpathlineto{\pgfqpoint{1.458919in}{1.914858in}}%
\pgfpathlineto{\pgfqpoint{1.457846in}{1.912410in}}%
\pgfpathlineto{\pgfqpoint{1.456155in}{1.908598in}}%
\pgfpathlineto{\pgfqpoint{1.453581in}{1.902337in}}%
\pgfpathlineto{\pgfqpoint{1.451586in}{1.897065in}}%
\pgfpathlineto{\pgfqpoint{1.451206in}{1.896077in}}%
\pgfpathlineto{\pgfqpoint{1.449012in}{1.889816in}}%
\pgfpathlineto{\pgfqpoint{1.447028in}{1.883556in}}%
\pgfpathlineto{\pgfqpoint{1.445326in}{1.877551in}}%
\pgfpathlineto{\pgfqpoint{1.445252in}{1.877295in}}%
\pgfpathlineto{\pgfqpoint{1.443669in}{1.871035in}}%
\pgfpathlineto{\pgfqpoint{1.442305in}{1.864775in}}%
\pgfpathlineto{\pgfqpoint{1.441163in}{1.858514in}}%
\pgfpathlineto{\pgfqpoint{1.440248in}{1.852254in}}%
\pgfpathlineto{\pgfqpoint{1.439564in}{1.845993in}}%
\pgfpathlineto{\pgfqpoint{1.439119in}{1.839733in}}%
\pgfpathlineto{\pgfqpoint{1.439065in}{1.838030in}}%
\pgfpathlineto{\pgfqpoint{1.438918in}{1.833472in}}%
\pgfpathlineto{\pgfqpoint{1.438974in}{1.827212in}}%
\pgfpathlineto{\pgfqpoint{1.439065in}{1.825427in}}%
\pgfpathlineto{\pgfqpoint{1.439295in}{1.820952in}}%
\pgfpathlineto{\pgfqpoint{1.439891in}{1.814691in}}%
\pgfpathlineto{\pgfqpoint{1.440776in}{1.808431in}}%
\pgfpathlineto{\pgfqpoint{1.441965in}{1.802170in}}%
\pgfpathlineto{\pgfqpoint{1.443473in}{1.795910in}}%
\pgfpathlineto{\pgfqpoint{1.445319in}{1.789649in}}%
\pgfpathlineto{\pgfqpoint{1.445326in}{1.789630in}}%
\pgfpathlineto{\pgfqpoint{1.447509in}{1.783389in}}%
\pgfpathlineto{\pgfqpoint{1.450080in}{1.777129in}}%
\pgfpathlineto{\pgfqpoint{1.451586in}{1.773920in}}%
\pgfpathlineto{\pgfqpoint{1.453049in}{1.770868in}}%
\pgfpathlineto{\pgfqpoint{1.456446in}{1.764608in}}%
\pgfpathlineto{\pgfqpoint{1.457846in}{1.762297in}}%
\pgfpathlineto{\pgfqpoint{1.460303in}{1.758347in}}%
\pgfpathlineto{\pgfqpoint{1.464107in}{1.752860in}}%
\pgfpathlineto{\pgfqpoint{1.464659in}{1.752087in}}%
\pgfpathlineto{\pgfqpoint{1.469557in}{1.745826in}}%
\pgfpathlineto{\pgfqpoint{1.470367in}{1.744877in}}%
\pgfpathlineto{\pgfqpoint{1.475054in}{1.739566in}}%
\pgfpathlineto{\pgfqpoint{1.476628in}{1.737921in}}%
\pgfpathlineto{\pgfqpoint{1.481212in}{1.733306in}}%
\pgfpathlineto{\pgfqpoint{1.482888in}{1.731736in}}%
\pgfpathlineto{\pgfqpoint{1.488109in}{1.727045in}}%
\pgfpathlineto{\pgfqpoint{1.489149in}{1.726171in}}%
\pgfpathlineto{\pgfqpoint{1.495409in}{1.721123in}}%
\pgfpathlineto{\pgfqpoint{1.495847in}{1.720785in}}%
\pgfpathlineto{\pgfqpoint{1.501669in}{1.716531in}}%
\pgfpathlineto{\pgfqpoint{1.504557in}{1.714524in}}%
\pgfpathlineto{\pgfqpoint{1.507930in}{1.712299in}}%
\pgfpathlineto{\pgfqpoint{1.514190in}{1.708383in}}%
\pgfpathlineto{\pgfqpoint{1.514389in}{1.708264in}}%
\pgfpathlineto{\pgfqpoint{1.520451in}{1.704783in}}%
\pgfpathlineto{\pgfqpoint{1.525592in}{1.702003in}}%
\pgfpathlineto{\pgfqpoint{1.526711in}{1.701422in}}%
\pgfpathlineto{\pgfqpoint{1.532972in}{1.698311in}}%
\pgfpathlineto{\pgfqpoint{1.538470in}{1.695743in}}%
\pgfpathlineto{\pgfqpoint{1.539232in}{1.695399in}}%
\pgfpathlineto{\pgfqpoint{1.545492in}{1.692696in}}%
\pgfpathlineto{\pgfqpoint{1.551753in}{1.690164in}}%
\pgfpathlineto{\pgfqpoint{1.553524in}{1.689482in}}%
\pgfpathlineto{\pgfqpoint{1.558013in}{1.687803in}}%
\pgfpathlineto{\pgfqpoint{1.564274in}{1.685598in}}%
\pgfpathlineto{\pgfqpoint{1.570534in}{1.683535in}}%
\pgfpathlineto{\pgfqpoint{1.571534in}{1.683222in}}%
\pgfpathlineto{\pgfqpoint{1.576795in}{1.681611in}}%
\pgfpathlineto{\pgfqpoint{1.583055in}{1.679814in}}%
\pgfpathlineto{\pgfqpoint{1.589316in}{1.678139in}}%
\pgfpathlineto{\pgfqpoint{1.594023in}{1.676962in}}%
\pgfpathlineto{\pgfqpoint{1.595576in}{1.676579in}}%
\pgfpathlineto{\pgfqpoint{1.601836in}{1.675126in}}%
\pgfpathlineto{\pgfqpoint{1.608097in}{1.673778in}}%
\pgfpathlineto{\pgfqpoint{1.614357in}{1.672530in}}%
\pgfpathlineto{\pgfqpoint{1.620618in}{1.671379in}}%
\pgfpathlineto{\pgfqpoint{1.624599in}{1.670701in}}%
\pgfpathlineto{\pgfqpoint{1.626878in}{1.670316in}}%
\pgfpathlineto{\pgfqpoint{1.633139in}{1.669338in}}%
\pgfpathlineto{\pgfqpoint{1.639399in}{1.668445in}}%
\pgfpathlineto{\pgfqpoint{1.645659in}{1.667633in}}%
\pgfpathlineto{\pgfqpoint{1.651920in}{1.666900in}}%
\pgfpathlineto{\pgfqpoint{1.658180in}{1.666243in}}%
\pgfpathlineto{\pgfqpoint{1.664441in}{1.665659in}}%
\pgfpathlineto{\pgfqpoint{1.670701in}{1.665147in}}%
\pgfpathlineto{\pgfqpoint{1.676962in}{1.664703in}}%
\pgfpathlineto{\pgfqpoint{1.681297in}{1.664441in}}%
\pgfpathlineto{\pgfqpoint{1.683222in}{1.664324in}}%
\pgfpathlineto{\pgfqpoint{1.689482in}{1.664007in}}%
\pgfpathlineto{\pgfqpoint{1.695743in}{1.663753in}}%
\pgfpathlineto{\pgfqpoint{1.702003in}{1.663561in}}%
\pgfpathlineto{\pgfqpoint{1.708264in}{1.663428in}}%
\pgfpathlineto{\pgfqpoint{1.714524in}{1.663355in}}%
\pgfpathlineto{\pgfqpoint{1.720785in}{1.663338in}}%
\pgfpathlineto{\pgfqpoint{1.727045in}{1.663378in}}%
\pgfpathlineto{\pgfqpoint{1.733306in}{1.663472in}}%
\pgfpathlineto{\pgfqpoint{1.739566in}{1.663620in}}%
\pgfpathlineto{\pgfqpoint{1.745826in}{1.663821in}}%
\pgfpathlineto{\pgfqpoint{1.752087in}{1.664073in}}%
\pgfpathlineto{\pgfqpoint{1.758347in}{1.664377in}}%
\pgfpathlineto{\pgfqpoint{1.759471in}{1.664441in}}%
\pgfpathlineto{\pgfqpoint{1.764608in}{1.664725in}}%
\pgfpathlineto{\pgfqpoint{1.770868in}{1.665121in}}%
\pgfpathlineto{\pgfqpoint{1.777129in}{1.665564in}}%
\pgfpathlineto{\pgfqpoint{1.783389in}{1.666054in}}%
\pgfpathlineto{\pgfqpoint{1.789649in}{1.666591in}}%
\pgfpathlineto{\pgfqpoint{1.795910in}{1.667175in}}%
\pgfpathlineto{\pgfqpoint{1.802170in}{1.667805in}}%
\pgfpathlineto{\pgfqpoint{1.808431in}{1.668480in}}%
\pgfpathlineto{\pgfqpoint{1.814691in}{1.669202in}}%
\pgfpathlineto{\pgfqpoint{1.820952in}{1.669969in}}%
\pgfpathlineto{\pgfqpoint{1.826596in}{1.670701in}}%
\pgfpathlineto{\pgfqpoint{1.827212in}{1.670779in}}%
\pgfpathlineto{\pgfqpoint{1.833472in}{1.671617in}}%
\pgfpathlineto{\pgfqpoint{1.839733in}{1.672499in}}%
\pgfpathlineto{\pgfqpoint{1.845993in}{1.673425in}}%
\pgfpathlineto{\pgfqpoint{1.852254in}{1.674395in}}%
\pgfpathlineto{\pgfqpoint{1.858514in}{1.675409in}}%
\pgfpathlineto{\pgfqpoint{1.864775in}{1.676468in}}%
\pgfpathlineto{\pgfqpoint{1.867575in}{1.676962in}}%
\pgfpathlineto{\pgfqpoint{1.871035in}{1.677557in}}%
\pgfpathlineto{\pgfqpoint{1.877295in}{1.678678in}}%
\pgfpathlineto{\pgfqpoint{1.883556in}{1.679842in}}%
\pgfpathlineto{\pgfqpoint{1.889816in}{1.681051in}}%
\pgfpathlineto{\pgfqpoint{1.896077in}{1.682305in}}%
\pgfpathlineto{\pgfqpoint{1.900500in}{1.683222in}}%
\pgfpathlineto{\pgfqpoint{1.902337in}{1.683594in}}%
\pgfpathlineto{\pgfqpoint{1.908598in}{1.684905in}}%
\pgfpathlineto{\pgfqpoint{1.914858in}{1.686260in}}%
\pgfpathlineto{\pgfqpoint{1.921119in}{1.687661in}}%
\pgfpathlineto{\pgfqpoint{1.927379in}{1.689108in}}%
\pgfpathlineto{\pgfqpoint{1.928954in}{1.689482in}}%
\pgfpathlineto{\pgfqpoint{1.933639in}{1.690573in}}%
\pgfpathlineto{\pgfqpoint{1.939900in}{1.692074in}}%
\pgfpathlineto{\pgfqpoint{1.946160in}{1.693621in}}%
\pgfpathlineto{\pgfqpoint{1.952421in}{1.695216in}}%
\pgfpathlineto{\pgfqpoint{1.954438in}{1.695743in}}%
\pgfpathlineto{\pgfqpoint{1.958681in}{1.696829in}}%
\pgfpathlineto{\pgfqpoint{1.964942in}{1.698477in}}%
\pgfpathlineto{\pgfqpoint{1.971202in}{1.700172in}}%
\pgfpathlineto{\pgfqpoint{1.977462in}{1.701917in}}%
\pgfpathlineto{\pgfqpoint{1.977766in}{1.702003in}}%
\pgfpathlineto{\pgfqpoint{1.983723in}{1.703667in}}%
\pgfpathlineto{\pgfqpoint{1.989983in}{1.705463in}}%
\pgfpathlineto{\pgfqpoint{1.996244in}{1.707311in}}%
\pgfpathlineto{\pgfqpoint{1.999399in}{1.708264in}}%
\pgfpathlineto{\pgfqpoint{2.002504in}{1.709184in}}%
\pgfpathlineto{\pgfqpoint{2.008765in}{1.711084in}}%
\pgfpathlineto{\pgfqpoint{2.015025in}{1.713035in}}%
\pgfpathlineto{\pgfqpoint{2.019686in}{1.714524in}}%
\pgfpathlineto{\pgfqpoint{2.021285in}{1.715026in}}%
\pgfpathlineto{\pgfqpoint{2.027546in}{1.717031in}}%
\pgfpathlineto{\pgfqpoint{2.033806in}{1.719088in}}%
\pgfpathlineto{\pgfqpoint{2.038842in}{1.720785in}}%
\pgfpathlineto{\pgfqpoint{2.040067in}{1.721190in}}%
\pgfpathlineto{\pgfqpoint{2.046327in}{1.723302in}}%
\pgfpathlineto{\pgfqpoint{2.052588in}{1.725469in}}%
\pgfpathlineto{\pgfqpoint{2.057036in}{1.727045in}}%
\pgfpathlineto{\pgfqpoint{2.058848in}{1.727677in}}%
\pgfpathlineto{\pgfqpoint{2.065109in}{1.729900in}}%
\pgfpathlineto{\pgfqpoint{2.071369in}{1.732180in}}%
\pgfpathlineto{\pgfqpoint{2.074397in}{1.733306in}}%
\pgfpathlineto{\pgfqpoint{2.077629in}{1.734489in}}%
\pgfpathlineto{\pgfqpoint{2.083890in}{1.736826in}}%
\pgfpathlineto{\pgfqpoint{2.090150in}{1.739224in}}%
\pgfpathlineto{\pgfqpoint{2.091029in}{1.739566in}}%
\pgfpathlineto{\pgfqpoint{2.096411in}{1.741630in}}%
\pgfpathlineto{\pgfqpoint{2.102671in}{1.744087in}}%
\pgfpathlineto{\pgfqpoint{2.107010in}{1.745826in}}%
\pgfpathlineto{\pgfqpoint{2.108932in}{1.746587in}}%
\pgfpathlineto{\pgfqpoint{2.115192in}{1.749105in}}%
\pgfpathlineto{\pgfqpoint{2.121452in}{1.751687in}}%
\pgfpathlineto{\pgfqpoint{2.122409in}{1.752087in}}%
\pgfpathlineto{\pgfqpoint{2.127713in}{1.754278in}}%
\pgfpathlineto{\pgfqpoint{2.133973in}{1.756923in}}%
\pgfpathlineto{\pgfqpoint{2.137284in}{1.758347in}}%
\pgfpathlineto{\pgfqpoint{2.140234in}{1.759602in}}%
\pgfpathlineto{\pgfqpoint{2.146494in}{1.762312in}}%
\pgfpathlineto{\pgfqpoint{2.151682in}{1.764608in}}%
\pgfpathlineto{\pgfqpoint{2.152755in}{1.765077in}}%
\pgfpathlineto{\pgfqpoint{2.159015in}{1.767854in}}%
\pgfpathlineto{\pgfqpoint{2.165275in}{1.770697in}}%
\pgfpathlineto{\pgfqpoint{2.165648in}{1.770868in}}%
\pgfpathlineto{\pgfqpoint{2.171536in}{1.773545in}}%
\pgfpathlineto{\pgfqpoint{2.177796in}{1.776457in}}%
\pgfpathlineto{\pgfqpoint{2.179222in}{1.777129in}}%
\pgfpathlineto{\pgfqpoint{2.184057in}{1.779386in}}%
\pgfpathlineto{\pgfqpoint{2.190317in}{1.782367in}}%
\pgfpathlineto{\pgfqpoint{2.192435in}{1.783389in}}%
\pgfpathlineto{\pgfqpoint{2.196578in}{1.785373in}}%
\pgfpathlineto{\pgfqpoint{2.202838in}{1.788424in}}%
\pgfpathlineto{\pgfqpoint{2.205320in}{1.789649in}}%
\pgfpathlineto{\pgfqpoint{2.209098in}{1.791503in}}%
\pgfpathlineto{\pgfqpoint{2.215359in}{1.794623in}}%
\pgfpathlineto{\pgfqpoint{2.217908in}{1.795910in}}%
\pgfpathlineto{\pgfqpoint{2.221619in}{1.797773in}}%
\pgfpathlineto{\pgfqpoint{2.227880in}{1.800962in}}%
\pgfpathlineto{\pgfqpoint{2.230225in}{1.802170in}}%
\pgfpathlineto{\pgfqpoint{2.234140in}{1.804177in}}%
\pgfpathlineto{\pgfqpoint{2.240401in}{1.807434in}}%
\pgfpathlineto{\pgfqpoint{2.242301in}{1.808431in}}%
\pgfpathlineto{\pgfqpoint{2.246661in}{1.810710in}}%
\pgfpathlineto{\pgfqpoint{2.252922in}{1.814030in}}%
\pgfpathlineto{\pgfqpoint{2.254161in}{1.814691in}}%
\pgfpathlineto{\pgfqpoint{2.259182in}{1.817361in}}%
\pgfpathlineto{\pgfqpoint{2.265442in}{1.820741in}}%
\pgfpathlineto{\pgfqpoint{2.265831in}{1.820952in}}%
\pgfpathlineto{\pgfqpoint{2.271703in}{1.824120in}}%
\pgfpathlineto{\pgfqpoint{2.277351in}{1.827212in}}%
\pgfpathlineto{\pgfqpoint{2.277963in}{1.827547in}}%
\pgfpathlineto{\pgfqpoint{2.284224in}{1.830971in}}%
\pgfpathlineto{\pgfqpoint{2.288749in}{1.833472in}}%
\pgfpathlineto{\pgfqpoint{2.290484in}{1.834431in}}%
\pgfpathlineto{\pgfqpoint{2.296745in}{1.837895in}}%
\pgfpathlineto{\pgfqpoint{2.300050in}{1.839733in}}%
\pgfpathlineto{\pgfqpoint{2.303005in}{1.841376in}}%
\pgfpathlineto{\pgfqpoint{2.309265in}{1.844866in}}%
\pgfpathlineto{\pgfqpoint{2.311289in}{1.845993in}}%
\pgfpathlineto{\pgfqpoint{2.315526in}{1.848354in}}%
\pgfpathlineto{\pgfqpoint{2.321786in}{1.851856in}}%
\pgfpathlineto{\pgfqpoint{2.322504in}{1.852254in}}%
\pgfpathlineto{\pgfqpoint{2.328047in}{1.855333in}}%
\pgfpathlineto{\pgfqpoint{2.333760in}{1.858514in}}%
\pgfpathlineto{\pgfqpoint{2.334307in}{1.858819in}}%
\pgfpathlineto{\pgfqpoint{2.340568in}{1.862269in}}%
\pgfpathlineto{\pgfqpoint{2.345127in}{1.864775in}}%
\pgfpathlineto{\pgfqpoint{2.346828in}{1.865712in}}%
\pgfpathlineto{\pgfqpoint{2.353088in}{1.869113in}}%
\pgfpathlineto{\pgfqpoint{2.356664in}{1.871035in}}%
\pgfpathlineto{\pgfqpoint{2.359349in}{1.872483in}}%
\pgfpathlineto{\pgfqpoint{2.365609in}{1.875804in}}%
\pgfpathlineto{\pgfqpoint{2.368476in}{1.877295in}}%
\pgfpathlineto{\pgfqpoint{2.371870in}{1.879068in}}%
\pgfpathlineto{\pgfqpoint{2.378130in}{1.882268in}}%
\pgfpathlineto{\pgfqpoint{2.380722in}{1.883556in}}%
\pgfpathlineto{\pgfqpoint{2.384391in}{1.885387in}}%
\pgfpathlineto{\pgfqpoint{2.390651in}{1.888419in}}%
\pgfpathlineto{\pgfqpoint{2.393648in}{1.889816in}}%
\pgfpathlineto{\pgfqpoint{2.396912in}{1.891346in}}%
\pgfpathlineto{\pgfqpoint{2.403172in}{1.894154in}}%
\pgfpathlineto{\pgfqpoint{2.407667in}{1.896077in}}%
\pgfpathlineto{\pgfqpoint{2.409432in}{1.896836in}}%
\pgfpathlineto{\pgfqpoint{2.415693in}{1.899355in}}%
\pgfpathlineto{\pgfqpoint{2.421953in}{1.901718in}}%
\pgfpathlineto{\pgfqpoint{2.423757in}{1.902337in}}%
\pgfpathlineto{\pgfqpoint{2.428214in}{1.903878in}}%
\pgfpathlineto{\pgfqpoint{2.434474in}{1.905826in}}%
\pgfpathlineto{\pgfqpoint{2.440735in}{1.907543in}}%
\pgfpathlineto{\pgfqpoint{2.445305in}{1.908598in}}%
\pgfpathlineto{\pgfqpoint{2.446995in}{1.908991in}}%
\pgfpathlineto{\pgfqpoint{2.453255in}{1.910128in}}%
\pgfpathlineto{\pgfqpoint{2.459516in}{1.910933in}}%
\pgfpathlineto{\pgfqpoint{2.465776in}{1.911362in}}%
\pgfpathlineto{\pgfqpoint{2.472037in}{1.911368in}}%
\pgfpathlineto{\pgfqpoint{2.478297in}{1.910894in}}%
\pgfpathlineto{\pgfqpoint{2.484558in}{1.909880in}}%
\pgfpathlineto{\pgfqpoint{2.489530in}{1.908598in}}%
\pgfpathlineto{\pgfqpoint{2.490818in}{1.908249in}}%
\pgfpathlineto{\pgfqpoint{2.497078in}{1.905888in}}%
\pgfpathlineto{\pgfqpoint{2.503339in}{1.902732in}}%
\pgfpathlineto{\pgfqpoint{2.503966in}{1.902337in}}%
\pgfpathlineto{\pgfqpoint{2.509599in}{1.898601in}}%
\pgfpathlineto{\pgfqpoint{2.512707in}{1.896077in}}%
\pgfpathlineto{\pgfqpoint{2.515860in}{1.893374in}}%
\pgfpathlineto{\pgfqpoint{2.519358in}{1.889816in}}%
\pgfpathlineto{\pgfqpoint{2.522120in}{1.886851in}}%
\pgfpathlineto{\pgfqpoint{2.524770in}{1.883556in}}%
\pgfpathlineto{\pgfqpoint{2.528381in}{1.878822in}}%
\pgfpathlineto{\pgfqpoint{2.529404in}{1.877295in}}%
\pgfpathlineto{\pgfqpoint{2.533420in}{1.871035in}}%
\pgfpathlineto{\pgfqpoint{2.534641in}{1.869045in}}%
\pgfpathlineto{\pgfqpoint{2.536990in}{1.864775in}}%
\pgfpathlineto{\pgfqpoint{2.540292in}{1.858514in}}%
\pgfpathlineto{\pgfqpoint{2.540902in}{1.857322in}}%
\pgfpathlineto{\pgfqpoint{2.543258in}{1.852254in}}%
\pgfpathlineto{\pgfqpoint{2.546079in}{1.845993in}}%
\pgfpathlineto{\pgfqpoint{2.547162in}{1.843532in}}%
\pgfpathlineto{\pgfqpoint{2.548703in}{1.839733in}}%
\pgfpathlineto{\pgfqpoint{2.551197in}{1.833472in}}%
\pgfpathlineto{\pgfqpoint{2.553422in}{1.827798in}}%
\pgfpathlineto{\pgfqpoint{2.553637in}{1.827212in}}%
\pgfpathlineto{\pgfqpoint{2.555912in}{1.820952in}}%
\pgfpathlineto{\pgfqpoint{2.558179in}{1.814691in}}%
\pgfpathlineto{\pgfqpoint{2.559683in}{1.810542in}}%
\pgfpathlineto{\pgfqpoint{2.560405in}{1.808431in}}%
\pgfpathlineto{\pgfqpoint{2.562560in}{1.802170in}}%
\pgfpathlineto{\pgfqpoint{2.564741in}{1.795910in}}%
\pgfpathlineto{\pgfqpoint{2.565943in}{1.792505in}}%
\pgfpathlineto{\pgfqpoint{2.566906in}{1.789649in}}%
\pgfpathlineto{\pgfqpoint{2.569055in}{1.783389in}}%
\pgfpathlineto{\pgfqpoint{2.571261in}{1.777129in}}%
\pgfpathlineto{\pgfqpoint{2.572204in}{1.774513in}}%
\pgfpathlineto{\pgfqpoint{2.573471in}{1.770868in}}%
\pgfpathlineto{\pgfqpoint{2.575713in}{1.764608in}}%
\pgfpathlineto{\pgfqpoint{2.578039in}{1.758347in}}%
\pgfpathlineto{\pgfqpoint{2.578464in}{1.757236in}}%
\pgfpathlineto{\pgfqpoint{2.580380in}{1.752087in}}%
\pgfpathlineto{\pgfqpoint{2.582807in}{1.745826in}}%
\pgfpathlineto{\pgfqpoint{2.584725in}{1.741083in}}%
\pgfpathlineto{\pgfqpoint{2.585326in}{1.739566in}}%
\pgfpathlineto{\pgfqpoint{2.587899in}{1.733306in}}%
\pgfpathlineto{\pgfqpoint{2.590609in}{1.727045in}}%
\pgfpathlineto{\pgfqpoint{2.590985in}{1.726205in}}%
\pgfpathlineto{\pgfqpoint{2.593387in}{1.720785in}}%
\pgfpathlineto{\pgfqpoint{2.596315in}{1.714524in}}%
\pgfpathlineto{\pgfqpoint{2.597245in}{1.712613in}}%
\pgfpathlineto{\pgfqpoint{2.599355in}{1.708264in}}%
\pgfpathlineto{\pgfqpoint{2.602556in}{1.702003in}}%
\pgfpathlineto{\pgfqpoint{2.603506in}{1.700222in}}%
\pgfpathlineto{\pgfqpoint{2.605901in}{1.695743in}}%
\pgfpathlineto{\pgfqpoint{2.609442in}{1.689482in}}%
\pgfpathlineto{\pgfqpoint{2.609766in}{1.688931in}}%
\pgfpathlineto{\pgfqpoint{2.613150in}{1.683222in}}%
\pgfpathlineto{\pgfqpoint{2.616027in}{1.678633in}}%
\pgfpathlineto{\pgfqpoint{2.617090in}{1.676962in}}%
\pgfpathlineto{\pgfqpoint{2.621257in}{1.670701in}}%
\pgfpathlineto{\pgfqpoint{2.622287in}{1.669218in}}%
\pgfpathlineto{\pgfqpoint{2.625680in}{1.664441in}}%
\pgfpathlineto{\pgfqpoint{2.628548in}{1.660597in}}%
\pgfpathlineto{\pgfqpoint{2.630400in}{1.658180in}}%
\pgfpathlineto{\pgfqpoint{2.634808in}{1.652690in}}%
\pgfpathlineto{\pgfqpoint{2.635447in}{1.651920in}}%
\pgfpathlineto{\pgfqpoint{2.640858in}{1.645659in}}%
\pgfpathlineto{\pgfqpoint{2.641068in}{1.645425in}}%
\pgfpathlineto{\pgfqpoint{2.646688in}{1.639399in}}%
\pgfpathlineto{\pgfqpoint{2.647329in}{1.638738in}}%
\pgfpathlineto{\pgfqpoint{2.653004in}{1.633139in}}%
\pgfpathlineto{\pgfqpoint{2.653589in}{1.632581in}}%
\pgfpathlineto{\pgfqpoint{2.659850in}{1.626910in}}%
\pgfpathlineto{\pgfqpoint{2.659886in}{1.626878in}}%
\pgfpathlineto{\pgfqpoint{2.666110in}{1.621675in}}%
\pgfpathlineto{\pgfqpoint{2.667455in}{1.620618in}}%
\pgfpathlineto{\pgfqpoint{2.672371in}{1.616850in}}%
\pgfpathlineto{\pgfqpoint{2.675852in}{1.614357in}}%
\pgfpathlineto{\pgfqpoint{2.678631in}{1.612411in}}%
\pgfpathlineto{\pgfqpoint{2.684891in}{1.608335in}}%
\pgfpathlineto{\pgfqpoint{2.685281in}{1.608097in}}%
\pgfpathlineto{\pgfqpoint{2.691152in}{1.604568in}}%
\pgfpathlineto{\pgfqpoint{2.696118in}{1.601836in}}%
\pgfpathlineto{\pgfqpoint{2.697412in}{1.601133in}}%
\pgfpathlineto{\pgfqpoint{2.703673in}{1.597972in}}%
\pgfpathlineto{\pgfqpoint{2.708906in}{1.595576in}}%
\pgfpathlineto{\pgfqpoint{2.709933in}{1.595109in}}%
\pgfpathlineto{\pgfqpoint{2.716194in}{1.592482in}}%
\pgfpathlineto{\pgfqpoint{2.722454in}{1.590128in}}%
\pgfpathlineto{\pgfqpoint{2.724841in}{1.589316in}}%
\pgfpathlineto{\pgfqpoint{2.728715in}{1.587996in}}%
\pgfpathlineto{\pgfqpoint{2.734975in}{1.586087in}}%
\pgfpathlineto{\pgfqpoint{2.741235in}{1.584410in}}%
\pgfpathlineto{\pgfqpoint{2.747042in}{1.583055in}}%
\pgfpathclose%
\pgfpathmoveto{\pgfqpoint{2.779825in}{1.676962in}}%
\pgfpathlineto{\pgfqpoint{2.778798in}{1.677234in}}%
\pgfpathlineto{\pgfqpoint{2.772538in}{1.679149in}}%
\pgfpathlineto{\pgfqpoint{2.766277in}{1.681361in}}%
\pgfpathlineto{\pgfqpoint{2.761638in}{1.683222in}}%
\pgfpathlineto{\pgfqpoint{2.760017in}{1.683883in}}%
\pgfpathlineto{\pgfqpoint{2.753756in}{1.686726in}}%
\pgfpathlineto{\pgfqpoint{2.748347in}{1.689482in}}%
\pgfpathlineto{\pgfqpoint{2.747496in}{1.689926in}}%
\pgfpathlineto{\pgfqpoint{2.741235in}{1.693497in}}%
\pgfpathlineto{\pgfqpoint{2.737660in}{1.695743in}}%
\pgfpathlineto{\pgfqpoint{2.734975in}{1.697480in}}%
\pgfpathlineto{\pgfqpoint{2.728715in}{1.701906in}}%
\pgfpathlineto{\pgfqpoint{2.728586in}{1.702003in}}%
\pgfpathlineto{\pgfqpoint{2.722454in}{1.706818in}}%
\pgfpathlineto{\pgfqpoint{2.720746in}{1.708264in}}%
\pgfpathlineto{\pgfqpoint{2.716194in}{1.712270in}}%
\pgfpathlineto{\pgfqpoint{2.713798in}{1.714524in}}%
\pgfpathlineto{\pgfqpoint{2.709933in}{1.718323in}}%
\pgfpathlineto{\pgfqpoint{2.707572in}{1.720785in}}%
\pgfpathlineto{\pgfqpoint{2.703673in}{1.725051in}}%
\pgfpathlineto{\pgfqpoint{2.701943in}{1.727045in}}%
\pgfpathlineto{\pgfqpoint{2.697412in}{1.732548in}}%
\pgfpathlineto{\pgfqpoint{2.696817in}{1.733306in}}%
\pgfpathlineto{\pgfqpoint{2.692129in}{1.739566in}}%
\pgfpathlineto{\pgfqpoint{2.691152in}{1.740944in}}%
\pgfpathlineto{\pgfqpoint{2.687817in}{1.745826in}}%
\pgfpathlineto{\pgfqpoint{2.684891in}{1.750389in}}%
\pgfpathlineto{\pgfqpoint{2.683836in}{1.752087in}}%
\pgfpathlineto{\pgfqpoint{2.680153in}{1.758347in}}%
\pgfpathlineto{\pgfqpoint{2.678631in}{1.761098in}}%
\pgfpathlineto{\pgfqpoint{2.676734in}{1.764608in}}%
\pgfpathlineto{\pgfqpoint{2.673558in}{1.770868in}}%
\pgfpathlineto{\pgfqpoint{2.672371in}{1.773354in}}%
\pgfpathlineto{\pgfqpoint{2.670597in}{1.777129in}}%
\pgfpathlineto{\pgfqpoint{2.667837in}{1.783389in}}%
\pgfpathlineto{\pgfqpoint{2.666110in}{1.787577in}}%
\pgfpathlineto{\pgfqpoint{2.665263in}{1.789649in}}%
\pgfpathlineto{\pgfqpoint{2.662849in}{1.795910in}}%
\pgfpathlineto{\pgfqpoint{2.660603in}{1.802170in}}%
\pgfpathlineto{\pgfqpoint{2.659850in}{1.804402in}}%
\pgfpathlineto{\pgfqpoint{2.658490in}{1.808431in}}%
\pgfpathlineto{\pgfqpoint{2.656509in}{1.814691in}}%
\pgfpathlineto{\pgfqpoint{2.654663in}{1.820952in}}%
\pgfpathlineto{\pgfqpoint{2.653589in}{1.824839in}}%
\pgfpathlineto{\pgfqpoint{2.652928in}{1.827212in}}%
\pgfpathlineto{\pgfqpoint{2.651286in}{1.833472in}}%
\pgfpathlineto{\pgfqpoint{2.649753in}{1.839733in}}%
\pgfpathlineto{\pgfqpoint{2.648322in}{1.845993in}}%
\pgfpathlineto{\pgfqpoint{2.647329in}{1.850628in}}%
\pgfpathlineto{\pgfqpoint{2.646974in}{1.852254in}}%
\pgfpathlineto{\pgfqpoint{2.645685in}{1.858514in}}%
\pgfpathlineto{\pgfqpoint{2.644477in}{1.864775in}}%
\pgfpathlineto{\pgfqpoint{2.643343in}{1.871035in}}%
\pgfpathlineto{\pgfqpoint{2.642277in}{1.877295in}}%
\pgfpathlineto{\pgfqpoint{2.641273in}{1.883556in}}%
\pgfpathlineto{\pgfqpoint{2.641068in}{1.884905in}}%
\pgfpathlineto{\pgfqpoint{2.640298in}{1.889816in}}%
\pgfpathlineto{\pgfqpoint{2.639368in}{1.896077in}}%
\pgfpathlineto{\pgfqpoint{2.638483in}{1.902337in}}%
\pgfpathlineto{\pgfqpoint{2.637639in}{1.908598in}}%
\pgfpathlineto{\pgfqpoint{2.636830in}{1.914858in}}%
\pgfpathlineto{\pgfqpoint{2.636051in}{1.921119in}}%
\pgfpathlineto{\pgfqpoint{2.635297in}{1.927379in}}%
\pgfpathlineto{\pgfqpoint{2.634808in}{1.931551in}}%
\pgfpathlineto{\pgfqpoint{2.634551in}{1.933639in}}%
\pgfpathlineto{\pgfqpoint{2.633796in}{1.939900in}}%
\pgfpathlineto{\pgfqpoint{2.633049in}{1.946160in}}%
\pgfpathlineto{\pgfqpoint{2.632304in}{1.952421in}}%
\pgfpathlineto{\pgfqpoint{2.631555in}{1.958681in}}%
\pgfpathlineto{\pgfqpoint{2.630797in}{1.964942in}}%
\pgfpathlineto{\pgfqpoint{2.630022in}{1.971202in}}%
\pgfpathlineto{\pgfqpoint{2.629224in}{1.977462in}}%
\pgfpathlineto{\pgfqpoint{2.628548in}{1.982599in}}%
\pgfpathlineto{\pgfqpoint{2.628389in}{1.983723in}}%
\pgfpathlineto{\pgfqpoint{2.627475in}{1.989983in}}%
\pgfpathlineto{\pgfqpoint{2.626514in}{1.996244in}}%
\pgfpathlineto{\pgfqpoint{2.625495in}{2.002504in}}%
\pgfpathlineto{\pgfqpoint{2.624411in}{2.008765in}}%
\pgfpathlineto{\pgfqpoint{2.623251in}{2.015025in}}%
\pgfpathlineto{\pgfqpoint{2.622287in}{2.019900in}}%
\pgfpathlineto{\pgfqpoint{2.621988in}{2.021285in}}%
\pgfpathlineto{\pgfqpoint{2.620558in}{2.027546in}}%
\pgfpathlineto{\pgfqpoint{2.619011in}{2.033806in}}%
\pgfpathlineto{\pgfqpoint{2.617335in}{2.040067in}}%
\pgfpathlineto{\pgfqpoint{2.616027in}{2.044603in}}%
\pgfpathlineto{\pgfqpoint{2.615478in}{2.046327in}}%
\pgfpathlineto{\pgfqpoint{2.613352in}{2.052588in}}%
\pgfpathlineto{\pgfqpoint{2.611037in}{2.058848in}}%
\pgfpathlineto{\pgfqpoint{2.609766in}{2.062051in}}%
\pgfpathlineto{\pgfqpoint{2.608412in}{2.065109in}}%
\pgfpathlineto{\pgfqpoint{2.605436in}{2.071369in}}%
\pgfpathlineto{\pgfqpoint{2.603506in}{2.075143in}}%
\pgfpathlineto{\pgfqpoint{2.602072in}{2.077629in}}%
\pgfpathlineto{\pgfqpoint{2.598212in}{2.083890in}}%
\pgfpathlineto{\pgfqpoint{2.597245in}{2.085369in}}%
\pgfpathlineto{\pgfqpoint{2.593682in}{2.090150in}}%
\pgfpathlineto{\pgfqpoint{2.590985in}{2.093555in}}%
\pgfpathlineto{\pgfqpoint{2.588380in}{2.096411in}}%
\pgfpathlineto{\pgfqpoint{2.584725in}{2.100220in}}%
\pgfpathlineto{\pgfqpoint{2.581983in}{2.102671in}}%
\pgfpathlineto{\pgfqpoint{2.578464in}{2.105690in}}%
\pgfpathlineto{\pgfqpoint{2.573996in}{2.108932in}}%
\pgfpathlineto{\pgfqpoint{2.572204in}{2.110190in}}%
\pgfpathlineto{\pgfqpoint{2.565943in}{2.113875in}}%
\pgfpathlineto{\pgfqpoint{2.563218in}{2.115192in}}%
\pgfpathlineto{\pgfqpoint{2.559683in}{2.116866in}}%
\pgfpathlineto{\pgfqpoint{2.553422in}{2.119248in}}%
\pgfpathlineto{\pgfqpoint{2.547162in}{2.121088in}}%
\pgfpathlineto{\pgfqpoint{2.545484in}{2.121452in}}%
\pgfpathlineto{\pgfqpoint{2.540902in}{2.122443in}}%
\pgfpathlineto{\pgfqpoint{2.534641in}{2.123353in}}%
\pgfpathlineto{\pgfqpoint{2.528381in}{2.123850in}}%
\pgfpathlineto{\pgfqpoint{2.522120in}{2.123964in}}%
\pgfpathlineto{\pgfqpoint{2.515860in}{2.123718in}}%
\pgfpathlineto{\pgfqpoint{2.509599in}{2.123131in}}%
\pgfpathlineto{\pgfqpoint{2.503339in}{2.122217in}}%
\pgfpathlineto{\pgfqpoint{2.499439in}{2.121452in}}%
\pgfpathlineto{\pgfqpoint{2.497078in}{2.120996in}}%
\pgfpathlineto{\pgfqpoint{2.490818in}{2.119487in}}%
\pgfpathlineto{\pgfqpoint{2.484558in}{2.117687in}}%
\pgfpathlineto{\pgfqpoint{2.478297in}{2.115600in}}%
\pgfpathlineto{\pgfqpoint{2.477214in}{2.115192in}}%
\pgfpathlineto{\pgfqpoint{2.472037in}{2.113264in}}%
\pgfpathlineto{\pgfqpoint{2.465776in}{2.110661in}}%
\pgfpathlineto{\pgfqpoint{2.462004in}{2.108932in}}%
\pgfpathlineto{\pgfqpoint{2.459516in}{2.107804in}}%
\pgfpathlineto{\pgfqpoint{2.453255in}{2.104712in}}%
\pgfpathlineto{\pgfqpoint{2.449443in}{2.102671in}}%
\pgfpathlineto{\pgfqpoint{2.446995in}{2.101376in}}%
\pgfpathlineto{\pgfqpoint{2.440735in}{2.097817in}}%
\pgfpathlineto{\pgfqpoint{2.438418in}{2.096411in}}%
\pgfpathlineto{\pgfqpoint{2.434474in}{2.094044in}}%
\pgfpathlineto{\pgfqpoint{2.428388in}{2.090150in}}%
\pgfpathlineto{\pgfqpoint{2.428214in}{2.090040in}}%
\pgfpathlineto{\pgfqpoint{2.421953in}{2.085869in}}%
\pgfpathlineto{\pgfqpoint{2.419140in}{2.083890in}}%
\pgfpathlineto{\pgfqpoint{2.415693in}{2.081497in}}%
\pgfpathlineto{\pgfqpoint{2.410397in}{2.077629in}}%
\pgfpathlineto{\pgfqpoint{2.409432in}{2.076935in}}%
\pgfpathlineto{\pgfqpoint{2.403172in}{2.072224in}}%
\pgfpathlineto{\pgfqpoint{2.402082in}{2.071369in}}%
\pgfpathlineto{\pgfqpoint{2.396912in}{2.067374in}}%
\pgfpathlineto{\pgfqpoint{2.394097in}{2.065109in}}%
\pgfpathlineto{\pgfqpoint{2.390651in}{2.062382in}}%
\pgfpathlineto{\pgfqpoint{2.386351in}{2.058848in}}%
\pgfpathlineto{\pgfqpoint{2.384391in}{2.057266in}}%
\pgfpathlineto{\pgfqpoint{2.378792in}{2.052588in}}%
\pgfpathlineto{\pgfqpoint{2.378130in}{2.052045in}}%
\pgfpathlineto{\pgfqpoint{2.371870in}{2.046748in}}%
\pgfpathlineto{\pgfqpoint{2.371386in}{2.046327in}}%
\pgfpathlineto{\pgfqpoint{2.365609in}{2.041397in}}%
\pgfpathlineto{\pgfqpoint{2.364094in}{2.040067in}}%
\pgfpathlineto{\pgfqpoint{2.359349in}{2.035990in}}%
\pgfpathlineto{\pgfqpoint{2.356871in}{2.033806in}}%
\pgfpathlineto{\pgfqpoint{2.353088in}{2.030545in}}%
\pgfpathlineto{\pgfqpoint{2.349689in}{2.027546in}}%
\pgfpathlineto{\pgfqpoint{2.346828in}{2.025077in}}%
\pgfpathlineto{\pgfqpoint{2.342524in}{2.021285in}}%
\pgfpathlineto{\pgfqpoint{2.340568in}{2.019601in}}%
\pgfpathlineto{\pgfqpoint{2.335352in}{2.015025in}}%
\pgfpathlineto{\pgfqpoint{2.334307in}{2.014129in}}%
\pgfpathlineto{\pgfqpoint{2.328152in}{2.008765in}}%
\pgfpathlineto{\pgfqpoint{2.328047in}{2.008675in}}%
\pgfpathlineto{\pgfqpoint{2.321786in}{2.003269in}}%
\pgfpathlineto{\pgfqpoint{2.320913in}{2.002504in}}%
\pgfpathlineto{\pgfqpoint{2.315526in}{1.997902in}}%
\pgfpathlineto{\pgfqpoint{2.313608in}{1.996244in}}%
\pgfpathlineto{\pgfqpoint{2.309265in}{1.992580in}}%
\pgfpathlineto{\pgfqpoint{2.306220in}{1.989983in}}%
\pgfpathlineto{\pgfqpoint{2.303005in}{1.987308in}}%
\pgfpathlineto{\pgfqpoint{2.298733in}{1.983723in}}%
\pgfpathlineto{\pgfqpoint{2.296745in}{1.982094in}}%
\pgfpathlineto{\pgfqpoint{2.291131in}{1.977462in}}%
\pgfpathlineto{\pgfqpoint{2.290484in}{1.976941in}}%
\pgfpathlineto{\pgfqpoint{2.284224in}{1.971871in}}%
\pgfpathlineto{\pgfqpoint{2.283402in}{1.971202in}}%
\pgfpathlineto{\pgfqpoint{2.277963in}{1.966881in}}%
\pgfpathlineto{\pgfqpoint{2.275530in}{1.964942in}}%
\pgfpathlineto{\pgfqpoint{2.271703in}{1.961962in}}%
\pgfpathlineto{\pgfqpoint{2.267497in}{1.958681in}}%
\pgfpathlineto{\pgfqpoint{2.265442in}{1.957114in}}%
\pgfpathlineto{\pgfqpoint{2.259290in}{1.952421in}}%
\pgfpathlineto{\pgfqpoint{2.259182in}{1.952340in}}%
\pgfpathlineto{\pgfqpoint{2.252922in}{1.947671in}}%
\pgfpathlineto{\pgfqpoint{2.250893in}{1.946160in}}%
\pgfpathlineto{\pgfqpoint{2.246661in}{1.943077in}}%
\pgfpathlineto{\pgfqpoint{2.242291in}{1.939900in}}%
\pgfpathlineto{\pgfqpoint{2.240401in}{1.938555in}}%
\pgfpathlineto{\pgfqpoint{2.234140in}{1.934116in}}%
\pgfpathlineto{\pgfqpoint{2.233466in}{1.933639in}}%
\pgfpathlineto{\pgfqpoint{2.227880in}{1.929774in}}%
\pgfpathlineto{\pgfqpoint{2.224401in}{1.927379in}}%
\pgfpathlineto{\pgfqpoint{2.221619in}{1.925502in}}%
\pgfpathlineto{\pgfqpoint{2.215359in}{1.921304in}}%
\pgfpathlineto{\pgfqpoint{2.215081in}{1.921119in}}%
\pgfpathlineto{\pgfqpoint{2.209098in}{1.917206in}}%
\pgfpathlineto{\pgfqpoint{2.205480in}{1.914858in}}%
\pgfpathlineto{\pgfqpoint{2.202838in}{1.913175in}}%
\pgfpathlineto{\pgfqpoint{2.196578in}{1.909220in}}%
\pgfpathlineto{\pgfqpoint{2.195584in}{1.908598in}}%
\pgfpathlineto{\pgfqpoint{2.190317in}{1.905355in}}%
\pgfpathlineto{\pgfqpoint{2.185364in}{1.902337in}}%
\pgfpathlineto{\pgfqpoint{2.184057in}{1.901553in}}%
\pgfpathlineto{\pgfqpoint{2.177796in}{1.897838in}}%
\pgfpathlineto{\pgfqpoint{2.174796in}{1.896077in}}%
\pgfpathlineto{\pgfqpoint{2.171536in}{1.894192in}}%
\pgfpathlineto{\pgfqpoint{2.165275in}{1.890617in}}%
\pgfpathlineto{\pgfqpoint{2.163857in}{1.889816in}}%
\pgfpathlineto{\pgfqpoint{2.159015in}{1.887123in}}%
\pgfpathlineto{\pgfqpoint{2.152755in}{1.883687in}}%
\pgfpathlineto{\pgfqpoint{2.152512in}{1.883556in}}%
\pgfpathlineto{\pgfqpoint{2.146494in}{1.880338in}}%
\pgfpathlineto{\pgfqpoint{2.140718in}{1.877295in}}%
\pgfpathlineto{\pgfqpoint{2.140234in}{1.877044in}}%
\pgfpathlineto{\pgfqpoint{2.133973in}{1.873831in}}%
\pgfpathlineto{\pgfqpoint{2.128436in}{1.871035in}}%
\pgfpathlineto{\pgfqpoint{2.127713in}{1.870674in}}%
\pgfpathlineto{\pgfqpoint{2.121452in}{1.867595in}}%
\pgfpathlineto{\pgfqpoint{2.115618in}{1.864775in}}%
\pgfpathlineto{\pgfqpoint{2.115192in}{1.864571in}}%
\pgfpathlineto{\pgfqpoint{2.108932in}{1.861623in}}%
\pgfpathlineto{\pgfqpoint{2.102671in}{1.858729in}}%
\pgfpathlineto{\pgfqpoint{2.102200in}{1.858514in}}%
\pgfpathlineto{\pgfqpoint{2.096411in}{1.855908in}}%
\pgfpathlineto{\pgfqpoint{2.090150in}{1.853143in}}%
\pgfpathlineto{\pgfqpoint{2.088099in}{1.852254in}}%
\pgfpathlineto{\pgfqpoint{2.083890in}{1.850444in}}%
\pgfpathlineto{\pgfqpoint{2.077629in}{1.847806in}}%
\pgfpathlineto{\pgfqpoint{2.073239in}{1.845993in}}%
\pgfpathlineto{\pgfqpoint{2.071369in}{1.845227in}}%
\pgfpathlineto{\pgfqpoint{2.065109in}{1.842713in}}%
\pgfpathlineto{\pgfqpoint{2.058848in}{1.840254in}}%
\pgfpathlineto{\pgfqpoint{2.057492in}{1.839733in}}%
\pgfpathlineto{\pgfqpoint{2.052588in}{1.837860in}}%
\pgfpathlineto{\pgfqpoint{2.046327in}{1.835524in}}%
\pgfpathlineto{\pgfqpoint{2.040700in}{1.833472in}}%
\pgfpathlineto{\pgfqpoint{2.040067in}{1.833243in}}%
\pgfpathlineto{\pgfqpoint{2.033806in}{1.831028in}}%
\pgfpathlineto{\pgfqpoint{2.027546in}{1.828867in}}%
\pgfpathlineto{\pgfqpoint{2.022624in}{1.827212in}}%
\pgfpathlineto{\pgfqpoint{2.021285in}{1.826764in}}%
\pgfpathlineto{\pgfqpoint{2.015025in}{1.824724in}}%
\pgfpathlineto{\pgfqpoint{2.008765in}{1.822739in}}%
\pgfpathlineto{\pgfqpoint{2.002968in}{1.820952in}}%
\pgfpathlineto{\pgfqpoint{2.002504in}{1.820809in}}%
\pgfpathlineto{\pgfqpoint{1.996244in}{1.818944in}}%
\pgfpathlineto{\pgfqpoint{1.989983in}{1.817135in}}%
\pgfpathlineto{\pgfqpoint{1.983723in}{1.815382in}}%
\pgfpathlineto{\pgfqpoint{1.981172in}{1.814691in}}%
\pgfpathlineto{\pgfqpoint{1.977462in}{1.813689in}}%
\pgfpathlineto{\pgfqpoint{1.971202in}{1.812058in}}%
\pgfpathlineto{\pgfqpoint{1.964942in}{1.810483in}}%
\pgfpathlineto{\pgfqpoint{1.958681in}{1.808966in}}%
\pgfpathlineto{\pgfqpoint{1.956381in}{1.808431in}}%
\pgfpathlineto{\pgfqpoint{1.952421in}{1.807511in}}%
\pgfpathlineto{\pgfqpoint{1.946160in}{1.806118in}}%
\pgfpathlineto{\pgfqpoint{1.939900in}{1.804785in}}%
\pgfpathlineto{\pgfqpoint{1.933639in}{1.803511in}}%
\pgfpathlineto{\pgfqpoint{1.927379in}{1.802298in}}%
\pgfpathlineto{\pgfqpoint{1.926683in}{1.802170in}}%
\pgfpathlineto{\pgfqpoint{1.921119in}{1.801152in}}%
\pgfpathlineto{\pgfqpoint{1.914858in}{1.800070in}}%
\pgfpathlineto{\pgfqpoint{1.908598in}{1.799052in}}%
\pgfpathlineto{\pgfqpoint{1.902337in}{1.798098in}}%
\pgfpathlineto{\pgfqpoint{1.896077in}{1.797209in}}%
\pgfpathlineto{\pgfqpoint{1.889816in}{1.796387in}}%
\pgfpathlineto{\pgfqpoint{1.885847in}{1.795910in}}%
\pgfpathlineto{\pgfqpoint{1.883556in}{1.795635in}}%
\pgfpathlineto{\pgfqpoint{1.877295in}{1.794957in}}%
\pgfpathlineto{\pgfqpoint{1.871035in}{1.794350in}}%
\pgfpathlineto{\pgfqpoint{1.864775in}{1.793817in}}%
\pgfpathlineto{\pgfqpoint{1.858514in}{1.793358in}}%
\pgfpathlineto{\pgfqpoint{1.852254in}{1.792977in}}%
\pgfpathlineto{\pgfqpoint{1.845993in}{1.792675in}}%
\pgfpathlineto{\pgfqpoint{1.839733in}{1.792454in}}%
\pgfpathlineto{\pgfqpoint{1.833472in}{1.792318in}}%
\pgfpathlineto{\pgfqpoint{1.827212in}{1.792270in}}%
\pgfpathlineto{\pgfqpoint{1.820952in}{1.792311in}}%
\pgfpathlineto{\pgfqpoint{1.814691in}{1.792447in}}%
\pgfpathlineto{\pgfqpoint{1.808431in}{1.792681in}}%
\pgfpathlineto{\pgfqpoint{1.802170in}{1.793016in}}%
\pgfpathlineto{\pgfqpoint{1.795910in}{1.793457in}}%
\pgfpathlineto{\pgfqpoint{1.789649in}{1.794008in}}%
\pgfpathlineto{\pgfqpoint{1.783389in}{1.794675in}}%
\pgfpathlineto{\pgfqpoint{1.777129in}{1.795463in}}%
\pgfpathlineto{\pgfqpoint{1.774047in}{1.795910in}}%
\pgfpathlineto{\pgfqpoint{1.770868in}{1.796390in}}%
\pgfpathlineto{\pgfqpoint{1.764608in}{1.797464in}}%
\pgfpathlineto{\pgfqpoint{1.758347in}{1.798684in}}%
\pgfpathlineto{\pgfqpoint{1.752087in}{1.800058in}}%
\pgfpathlineto{\pgfqpoint{1.745826in}{1.801593in}}%
\pgfpathlineto{\pgfqpoint{1.743679in}{1.802170in}}%
\pgfpathlineto{\pgfqpoint{1.739566in}{1.803337in}}%
\pgfpathlineto{\pgfqpoint{1.733306in}{1.805290in}}%
\pgfpathlineto{\pgfqpoint{1.727045in}{1.807446in}}%
\pgfpathlineto{\pgfqpoint{1.724411in}{1.808431in}}%
\pgfpathlineto{\pgfqpoint{1.720785in}{1.809876in}}%
\pgfpathlineto{\pgfqpoint{1.714524in}{1.812590in}}%
\pgfpathlineto{\pgfqpoint{1.710073in}{1.814691in}}%
\pgfpathlineto{\pgfqpoint{1.708264in}{1.815611in}}%
\pgfpathlineto{\pgfqpoint{1.702003in}{1.819037in}}%
\pgfpathlineto{\pgfqpoint{1.698766in}{1.820952in}}%
\pgfpathlineto{\pgfqpoint{1.695743in}{1.822900in}}%
\pgfpathlineto{\pgfqpoint{1.689552in}{1.827212in}}%
\pgfpathlineto{\pgfqpoint{1.689482in}{1.827265in}}%
\pgfpathlineto{\pgfqpoint{1.683222in}{1.832385in}}%
\pgfpathlineto{\pgfqpoint{1.681979in}{1.833472in}}%
\pgfpathlineto{\pgfqpoint{1.676962in}{1.838393in}}%
\pgfpathlineto{\pgfqpoint{1.675676in}{1.839733in}}%
\pgfpathlineto{\pgfqpoint{1.670701in}{1.845651in}}%
\pgfpathlineto{\pgfqpoint{1.670429in}{1.845993in}}%
\pgfpathlineto{\pgfqpoint{1.666102in}{1.852254in}}%
\pgfpathlineto{\pgfqpoint{1.664441in}{1.855109in}}%
\pgfpathlineto{\pgfqpoint{1.662552in}{1.858514in}}%
\pgfpathlineto{\pgfqpoint{1.659696in}{1.864775in}}%
\pgfpathlineto{\pgfqpoint{1.658180in}{1.868966in}}%
\pgfpathlineto{\pgfqpoint{1.657462in}{1.871035in}}%
\pgfpathlineto{\pgfqpoint{1.655802in}{1.877295in}}%
\pgfpathlineto{\pgfqpoint{1.654656in}{1.883556in}}%
\pgfpathlineto{\pgfqpoint{1.653989in}{1.889816in}}%
\pgfpathlineto{\pgfqpoint{1.653770in}{1.896077in}}%
\pgfpathlineto{\pgfqpoint{1.653972in}{1.902337in}}%
\pgfpathlineto{\pgfqpoint{1.654570in}{1.908598in}}%
\pgfpathlineto{\pgfqpoint{1.655543in}{1.914858in}}%
\pgfpathlineto{\pgfqpoint{1.656874in}{1.921119in}}%
\pgfpathlineto{\pgfqpoint{1.658180in}{1.925992in}}%
\pgfpathlineto{\pgfqpoint{1.658549in}{1.927379in}}%
\pgfpathlineto{\pgfqpoint{1.660560in}{1.933639in}}%
\pgfpathlineto{\pgfqpoint{1.662886in}{1.939900in}}%
\pgfpathlineto{\pgfqpoint{1.664441in}{1.943580in}}%
\pgfpathlineto{\pgfqpoint{1.665524in}{1.946160in}}%
\pgfpathlineto{\pgfqpoint{1.668465in}{1.952421in}}%
\pgfpathlineto{\pgfqpoint{1.670701in}{1.956744in}}%
\pgfpathlineto{\pgfqpoint{1.671698in}{1.958681in}}%
\pgfpathlineto{\pgfqpoint{1.675223in}{1.964942in}}%
\pgfpathlineto{\pgfqpoint{1.676962in}{1.967795in}}%
\pgfpathlineto{\pgfqpoint{1.679030in}{1.971202in}}%
\pgfpathlineto{\pgfqpoint{1.683115in}{1.977462in}}%
\pgfpathlineto{\pgfqpoint{1.683222in}{1.977615in}}%
\pgfpathlineto{\pgfqpoint{1.687481in}{1.983723in}}%
\pgfpathlineto{\pgfqpoint{1.689482in}{1.986424in}}%
\pgfpathlineto{\pgfqpoint{1.692118in}{1.989983in}}%
\pgfpathlineto{\pgfqpoint{1.695743in}{1.994607in}}%
\pgfpathlineto{\pgfqpoint{1.697026in}{1.996244in}}%
\pgfpathlineto{\pgfqpoint{1.702003in}{2.002262in}}%
\pgfpathlineto{\pgfqpoint{1.702204in}{2.002504in}}%
\pgfpathlineto{\pgfqpoint{1.707654in}{2.008765in}}%
\pgfpathlineto{\pgfqpoint{1.708264in}{2.009434in}}%
\pgfpathlineto{\pgfqpoint{1.713373in}{2.015025in}}%
\pgfpathlineto{\pgfqpoint{1.714524in}{2.016230in}}%
\pgfpathlineto{\pgfqpoint{1.719364in}{2.021285in}}%
\pgfpathlineto{\pgfqpoint{1.720785in}{2.022709in}}%
\pgfpathlineto{\pgfqpoint{1.725626in}{2.027546in}}%
\pgfpathlineto{\pgfqpoint{1.727045in}{2.028909in}}%
\pgfpathlineto{\pgfqpoint{1.732162in}{2.033806in}}%
\pgfpathlineto{\pgfqpoint{1.733306in}{2.034861in}}%
\pgfpathlineto{\pgfqpoint{1.738974in}{2.040067in}}%
\pgfpathlineto{\pgfqpoint{1.739566in}{2.040592in}}%
\pgfpathlineto{\pgfqpoint{1.745826in}{2.046118in}}%
\pgfpathlineto{\pgfqpoint{1.746065in}{2.046327in}}%
\pgfpathlineto{\pgfqpoint{1.752087in}{2.051444in}}%
\pgfpathlineto{\pgfqpoint{1.753441in}{2.052588in}}%
\pgfpathlineto{\pgfqpoint{1.758347in}{2.056606in}}%
\pgfpathlineto{\pgfqpoint{1.761103in}{2.058848in}}%
\pgfpathlineto{\pgfqpoint{1.764608in}{2.061619in}}%
\pgfpathlineto{\pgfqpoint{1.769054in}{2.065109in}}%
\pgfpathlineto{\pgfqpoint{1.770868in}{2.066495in}}%
\pgfpathlineto{\pgfqpoint{1.777129in}{2.071241in}}%
\pgfpathlineto{\pgfqpoint{1.777299in}{2.071369in}}%
\pgfpathlineto{\pgfqpoint{1.783389in}{2.075839in}}%
\pgfpathlineto{\pgfqpoint{1.785851in}{2.077629in}}%
\pgfpathlineto{\pgfqpoint{1.789649in}{2.080331in}}%
\pgfpathlineto{\pgfqpoint{1.794703in}{2.083890in}}%
\pgfpathlineto{\pgfqpoint{1.795910in}{2.084722in}}%
\pgfpathlineto{\pgfqpoint{1.802170in}{2.089001in}}%
\pgfpathlineto{\pgfqpoint{1.803868in}{2.090150in}}%
\pgfpathlineto{\pgfqpoint{1.808431in}{2.093180in}}%
\pgfpathlineto{\pgfqpoint{1.813348in}{2.096411in}}%
\pgfpathlineto{\pgfqpoint{1.814691in}{2.097278in}}%
\pgfpathlineto{\pgfqpoint{1.820952in}{2.101280in}}%
\pgfpathlineto{\pgfqpoint{1.823150in}{2.102671in}}%
\pgfpathlineto{\pgfqpoint{1.827212in}{2.105201in}}%
\pgfpathlineto{\pgfqpoint{1.833270in}{2.108932in}}%
\pgfpathlineto{\pgfqpoint{1.833472in}{2.109055in}}%
\pgfpathlineto{\pgfqpoint{1.839733in}{2.112819in}}%
\pgfpathlineto{\pgfqpoint{1.843725in}{2.115192in}}%
\pgfpathlineto{\pgfqpoint{1.845993in}{2.116524in}}%
\pgfpathlineto{\pgfqpoint{1.852254in}{2.120160in}}%
\pgfpathlineto{\pgfqpoint{1.854502in}{2.121452in}}%
\pgfpathlineto{\pgfqpoint{1.858514in}{2.123734in}}%
\pgfpathlineto{\pgfqpoint{1.864775in}{2.127254in}}%
\pgfpathlineto{\pgfqpoint{1.865600in}{2.127713in}}%
\pgfpathlineto{\pgfqpoint{1.871035in}{2.130711in}}%
\pgfpathlineto{\pgfqpoint{1.877017in}{2.133973in}}%
\pgfpathlineto{\pgfqpoint{1.877295in}{2.134124in}}%
\pgfpathlineto{\pgfqpoint{1.883556in}{2.137481in}}%
\pgfpathlineto{\pgfqpoint{1.888748in}{2.140234in}}%
\pgfpathlineto{\pgfqpoint{1.889816in}{2.140798in}}%
\pgfpathlineto{\pgfqpoint{1.896077in}{2.144069in}}%
\pgfpathlineto{\pgfqpoint{1.900769in}{2.146494in}}%
\pgfpathlineto{\pgfqpoint{1.902337in}{2.147304in}}%
\pgfpathlineto{\pgfqpoint{1.908598in}{2.150500in}}%
\pgfpathlineto{\pgfqpoint{1.913057in}{2.152755in}}%
\pgfpathlineto{\pgfqpoint{1.914858in}{2.153666in}}%
\pgfpathlineto{\pgfqpoint{1.921119in}{2.156801in}}%
\pgfpathlineto{\pgfqpoint{1.925582in}{2.159015in}}%
\pgfpathlineto{\pgfqpoint{1.927379in}{2.159910in}}%
\pgfpathlineto{\pgfqpoint{1.933639in}{2.162996in}}%
\pgfpathlineto{\pgfqpoint{1.938301in}{2.165275in}}%
\pgfpathlineto{\pgfqpoint{1.939900in}{2.166062in}}%
\pgfpathlineto{\pgfqpoint{1.946160in}{2.169114in}}%
\pgfpathlineto{\pgfqpoint{1.951163in}{2.171536in}}%
\pgfpathlineto{\pgfqpoint{1.952421in}{2.172150in}}%
\pgfpathlineto{\pgfqpoint{1.958681in}{2.175183in}}%
\pgfpathlineto{\pgfqpoint{1.964103in}{2.177796in}}%
\pgfpathlineto{\pgfqpoint{1.964942in}{2.178206in}}%
\pgfpathlineto{\pgfqpoint{1.971202in}{2.181236in}}%
\pgfpathlineto{\pgfqpoint{1.977046in}{2.184057in}}%
\pgfpathlineto{\pgfqpoint{1.977462in}{2.184261in}}%
\pgfpathlineto{\pgfqpoint{1.983723in}{2.187309in}}%
\pgfpathlineto{\pgfqpoint{1.989908in}{2.190317in}}%
\pgfpathlineto{\pgfqpoint{1.989983in}{2.190355in}}%
\pgfpathlineto{\pgfqpoint{1.996244in}{2.193441in}}%
\pgfpathlineto{\pgfqpoint{2.002504in}{2.196532in}}%
\pgfpathlineto{\pgfqpoint{2.002597in}{2.196578in}}%
\pgfpathlineto{\pgfqpoint{2.008765in}{2.199680in}}%
\pgfpathlineto{\pgfqpoint{2.015014in}{2.202838in}}%
\pgfpathlineto{\pgfqpoint{2.015025in}{2.202844in}}%
\pgfpathlineto{\pgfqpoint{2.021285in}{2.206086in}}%
\pgfpathlineto{\pgfqpoint{2.027065in}{2.209098in}}%
\pgfpathlineto{\pgfqpoint{2.027546in}{2.209357in}}%
\pgfpathlineto{\pgfqpoint{2.033806in}{2.212728in}}%
\pgfpathlineto{\pgfqpoint{2.038655in}{2.215359in}}%
\pgfpathlineto{\pgfqpoint{2.040067in}{2.216156in}}%
\pgfpathlineto{\pgfqpoint{2.046327in}{2.219698in}}%
\pgfpathlineto{\pgfqpoint{2.049696in}{2.221619in}}%
\pgfpathlineto{\pgfqpoint{2.052588in}{2.223346in}}%
\pgfpathlineto{\pgfqpoint{2.058848in}{2.227117in}}%
\pgfpathlineto{\pgfqpoint{2.060108in}{2.227880in}}%
\pgfpathlineto{\pgfqpoint{2.065109in}{2.231073in}}%
\pgfpathlineto{\pgfqpoint{2.069846in}{2.234140in}}%
\pgfpathlineto{\pgfqpoint{2.071369in}{2.235188in}}%
\pgfpathlineto{\pgfqpoint{2.077629in}{2.239538in}}%
\pgfpathlineto{\pgfqpoint{2.078859in}{2.240401in}}%
\pgfpathlineto{\pgfqpoint{2.083890in}{2.244188in}}%
\pgfpathlineto{\pgfqpoint{2.087125in}{2.246661in}}%
\pgfpathlineto{\pgfqpoint{2.090150in}{2.249165in}}%
\pgfpathlineto{\pgfqpoint{2.094619in}{2.252922in}}%
\pgfpathlineto{\pgfqpoint{2.096411in}{2.254570in}}%
\pgfpathlineto{\pgfqpoint{2.101345in}{2.259182in}}%
\pgfpathlineto{\pgfqpoint{2.102671in}{2.260558in}}%
\pgfpathlineto{\pgfqpoint{2.107310in}{2.265442in}}%
\pgfpathlineto{\pgfqpoint{2.108932in}{2.267369in}}%
\pgfpathlineto{\pgfqpoint{2.112527in}{2.271703in}}%
\pgfpathlineto{\pgfqpoint{2.115192in}{2.275406in}}%
\pgfpathlineto{\pgfqpoint{2.117009in}{2.277963in}}%
\pgfpathlineto{\pgfqpoint{2.120778in}{2.284224in}}%
\pgfpathlineto{\pgfqpoint{2.121452in}{2.285576in}}%
\pgfpathlineto{\pgfqpoint{2.123872in}{2.290484in}}%
\pgfpathlineto{\pgfqpoint{2.126296in}{2.296745in}}%
\pgfpathlineto{\pgfqpoint{2.127713in}{2.301708in}}%
\pgfpathlineto{\pgfqpoint{2.128080in}{2.303005in}}%
\pgfpathlineto{\pgfqpoint{2.129261in}{2.309265in}}%
\pgfpathlineto{\pgfqpoint{2.129846in}{2.315526in}}%
\pgfpathlineto{\pgfqpoint{2.129860in}{2.321786in}}%
\pgfpathlineto{\pgfqpoint{2.129323in}{2.328047in}}%
\pgfpathlineto{\pgfqpoint{2.128253in}{2.334307in}}%
\pgfpathlineto{\pgfqpoint{2.127713in}{2.336418in}}%
\pgfpathlineto{\pgfqpoint{2.126678in}{2.340568in}}%
\pgfpathlineto{\pgfqpoint{2.124610in}{2.346828in}}%
\pgfpathlineto{\pgfqpoint{2.122053in}{2.353088in}}%
\pgfpathlineto{\pgfqpoint{2.121452in}{2.354318in}}%
\pgfpathlineto{\pgfqpoint{2.119047in}{2.359349in}}%
\pgfpathlineto{\pgfqpoint{2.115582in}{2.365609in}}%
\pgfpathlineto{\pgfqpoint{2.115192in}{2.366228in}}%
\pgfpathlineto{\pgfqpoint{2.111701in}{2.371870in}}%
\pgfpathlineto{\pgfqpoint{2.108932in}{2.375872in}}%
\pgfpathlineto{\pgfqpoint{2.107397in}{2.378130in}}%
\pgfpathlineto{\pgfqpoint{2.102691in}{2.384391in}}%
\pgfpathlineto{\pgfqpoint{2.102671in}{2.384415in}}%
\pgfpathlineto{\pgfqpoint{2.097616in}{2.390651in}}%
\pgfpathlineto{\pgfqpoint{2.096411in}{2.392020in}}%
\pgfpathlineto{\pgfqpoint{2.092173in}{2.396912in}}%
\pgfpathlineto{\pgfqpoint{2.090150in}{2.399077in}}%
\pgfpathlineto{\pgfqpoint{2.086382in}{2.403172in}}%
\pgfpathlineto{\pgfqpoint{2.083890in}{2.405699in}}%
\pgfpathlineto{\pgfqpoint{2.080262in}{2.409432in}}%
\pgfpathlineto{\pgfqpoint{2.077629in}{2.411975in}}%
\pgfpathlineto{\pgfqpoint{2.073833in}{2.415693in}}%
\pgfpathlineto{\pgfqpoint{2.071369in}{2.417969in}}%
\pgfpathlineto{\pgfqpoint{2.067113in}{2.421953in}}%
\pgfpathlineto{\pgfqpoint{2.065109in}{2.423732in}}%
\pgfpathlineto{\pgfqpoint{2.060123in}{2.428214in}}%
\pgfpathlineto{\pgfqpoint{2.058848in}{2.429304in}}%
\pgfpathlineto{\pgfqpoint{2.052882in}{2.434474in}}%
\pgfpathlineto{\pgfqpoint{2.052588in}{2.434717in}}%
\pgfpathlineto{\pgfqpoint{2.046327in}{2.439962in}}%
\pgfpathlineto{\pgfqpoint{2.045417in}{2.440735in}}%
\pgfpathlineto{\pgfqpoint{2.040067in}{2.445084in}}%
\pgfpathlineto{\pgfqpoint{2.037745in}{2.446995in}}%
\pgfpathlineto{\pgfqpoint{2.033806in}{2.450113in}}%
\pgfpathlineto{\pgfqpoint{2.029884in}{2.453255in}}%
\pgfpathlineto{\pgfqpoint{2.027546in}{2.455063in}}%
\pgfpathlineto{\pgfqpoint{2.021856in}{2.459516in}}%
\pgfpathlineto{\pgfqpoint{2.021285in}{2.459948in}}%
\pgfpathlineto{\pgfqpoint{2.015025in}{2.464740in}}%
\pgfpathlineto{\pgfqpoint{2.013686in}{2.465776in}}%
\pgfpathlineto{\pgfqpoint{2.008765in}{2.469474in}}%
\pgfpathlineto{\pgfqpoint{2.005393in}{2.472037in}}%
\pgfpathlineto{\pgfqpoint{2.002504in}{2.474173in}}%
\pgfpathlineto{\pgfqpoint{1.996992in}{2.478297in}}%
\pgfpathlineto{\pgfqpoint{1.996244in}{2.478843in}}%
\pgfpathlineto{\pgfqpoint{1.989983in}{2.483459in}}%
\pgfpathlineto{\pgfqpoint{1.988509in}{2.484558in}}%
\pgfpathlineto{\pgfqpoint{1.983723in}{2.488044in}}%
\pgfpathlineto{\pgfqpoint{1.979959in}{2.490818in}}%
\pgfpathlineto{\pgfqpoint{1.977462in}{2.492620in}}%
\pgfpathlineto{\pgfqpoint{1.971358in}{2.497078in}}%
\pgfpathlineto{\pgfqpoint{1.971202in}{2.497190in}}%
\pgfpathlineto{\pgfqpoint{1.964942in}{2.501718in}}%
\pgfpathlineto{\pgfqpoint{1.962726in}{2.503339in}}%
\pgfpathlineto{\pgfqpoint{1.958681in}{2.506248in}}%
\pgfpathlineto{\pgfqpoint{1.954075in}{2.509599in}}%
\pgfpathlineto{\pgfqpoint{1.952421in}{2.510785in}}%
\pgfpathlineto{\pgfqpoint{1.946160in}{2.515319in}}%
\pgfpathlineto{\pgfqpoint{1.945421in}{2.515860in}}%
\pgfpathlineto{\pgfqpoint{1.939900in}{2.519842in}}%
\pgfpathlineto{\pgfqpoint{1.936777in}{2.522120in}}%
\pgfpathlineto{\pgfqpoint{1.933639in}{2.524382in}}%
\pgfpathlineto{\pgfqpoint{1.928155in}{2.528381in}}%
\pgfpathlineto{\pgfqpoint{1.927379in}{2.528941in}}%
\pgfpathlineto{\pgfqpoint{1.921119in}{2.533499in}}%
\pgfpathlineto{\pgfqpoint{1.919566in}{2.534641in}}%
\pgfpathlineto{\pgfqpoint{1.914858in}{2.538072in}}%
\pgfpathlineto{\pgfqpoint{1.911018in}{2.540902in}}%
\pgfpathlineto{\pgfqpoint{1.908598in}{2.542672in}}%
\pgfpathlineto{\pgfqpoint{1.902524in}{2.547162in}}%
\pgfpathlineto{\pgfqpoint{1.902337in}{2.547299in}}%
\pgfpathlineto{\pgfqpoint{1.896077in}{2.551933in}}%
\pgfpathlineto{\pgfqpoint{1.894086in}{2.553422in}}%
\pgfpathlineto{\pgfqpoint{1.889816in}{2.556599in}}%
\pgfpathlineto{\pgfqpoint{1.885714in}{2.559683in}}%
\pgfpathlineto{\pgfqpoint{1.883556in}{2.561299in}}%
\pgfpathlineto{\pgfqpoint{1.877417in}{2.565943in}}%
\pgfpathlineto{\pgfqpoint{1.877295in}{2.566035in}}%
\pgfpathlineto{\pgfqpoint{1.871035in}{2.570790in}}%
\pgfpathlineto{\pgfqpoint{1.869192in}{2.572204in}}%
\pgfpathlineto{\pgfqpoint{1.864775in}{2.575585in}}%
\pgfpathlineto{\pgfqpoint{1.861050in}{2.578464in}}%
\pgfpathlineto{\pgfqpoint{1.858514in}{2.580422in}}%
\pgfpathlineto{\pgfqpoint{1.852995in}{2.584725in}}%
\pgfpathlineto{\pgfqpoint{1.852254in}{2.585303in}}%
\pgfpathlineto{\pgfqpoint{1.845993in}{2.590219in}}%
\pgfpathlineto{\pgfqpoint{1.845026in}{2.590985in}}%
\pgfpathlineto{\pgfqpoint{1.839733in}{2.595178in}}%
\pgfpathlineto{\pgfqpoint{1.837147in}{2.597245in}}%
\pgfpathlineto{\pgfqpoint{1.833472in}{2.600187in}}%
\pgfpathlineto{\pgfqpoint{1.829362in}{2.603506in}}%
\pgfpathlineto{\pgfqpoint{1.827212in}{2.605246in}}%
\pgfpathlineto{\pgfqpoint{1.821673in}{2.609766in}}%
\pgfpathlineto{\pgfqpoint{1.820952in}{2.610357in}}%
\pgfpathlineto{\pgfqpoint{1.814691in}{2.615518in}}%
\pgfpathlineto{\pgfqpoint{1.814078in}{2.616027in}}%
\pgfpathlineto{\pgfqpoint{1.808431in}{2.620731in}}%
\pgfpathlineto{\pgfqpoint{1.806577in}{2.622287in}}%
\pgfpathlineto{\pgfqpoint{1.802170in}{2.626003in}}%
\pgfpathlineto{\pgfqpoint{1.799175in}{2.628548in}}%
\pgfpathlineto{\pgfqpoint{1.795910in}{2.631336in}}%
\pgfpathlineto{\pgfqpoint{1.791872in}{2.634808in}}%
\pgfpathlineto{\pgfqpoint{1.789649in}{2.636731in}}%
\pgfpathlineto{\pgfqpoint{1.784668in}{2.641068in}}%
\pgfpathlineto{\pgfqpoint{1.783389in}{2.642190in}}%
\pgfpathlineto{\pgfqpoint{1.777563in}{2.647329in}}%
\pgfpathlineto{\pgfqpoint{1.777129in}{2.647715in}}%
\pgfpathlineto{\pgfqpoint{1.770868in}{2.653308in}}%
\pgfpathlineto{\pgfqpoint{1.770555in}{2.653589in}}%
\pgfpathlineto{\pgfqpoint{1.764608in}{2.658972in}}%
\pgfpathlineto{\pgfqpoint{1.763642in}{2.659850in}}%
\pgfpathlineto{\pgfqpoint{1.758347in}{2.664710in}}%
\pgfpathlineto{\pgfqpoint{1.756829in}{2.666110in}}%
\pgfpathlineto{\pgfqpoint{1.752087in}{2.670525in}}%
\pgfpathlineto{\pgfqpoint{1.750114in}{2.672371in}}%
\pgfpathlineto{\pgfqpoint{1.745826in}{2.676421in}}%
\pgfpathlineto{\pgfqpoint{1.743496in}{2.678631in}}%
\pgfpathlineto{\pgfqpoint{1.739566in}{2.682400in}}%
\pgfpathlineto{\pgfqpoint{1.736976in}{2.684891in}}%
\pgfpathlineto{\pgfqpoint{1.733306in}{2.688465in}}%
\pgfpathlineto{\pgfqpoint{1.730553in}{2.691152in}}%
\pgfpathlineto{\pgfqpoint{1.727045in}{2.694620in}}%
\pgfpathlineto{\pgfqpoint{1.724227in}{2.697412in}}%
\pgfpathlineto{\pgfqpoint{1.720785in}{2.700869in}}%
\pgfpathlineto{\pgfqpoint{1.717998in}{2.703673in}}%
\pgfpathlineto{\pgfqpoint{1.714524in}{2.707215in}}%
\pgfpathlineto{\pgfqpoint{1.711864in}{2.709933in}}%
\pgfpathlineto{\pgfqpoint{1.708264in}{2.713665in}}%
\pgfpathlineto{\pgfqpoint{1.705827in}{2.716194in}}%
\pgfpathlineto{\pgfqpoint{1.702003in}{2.720221in}}%
\pgfpathlineto{\pgfqpoint{1.699885in}{2.722454in}}%
\pgfpathlineto{\pgfqpoint{1.695743in}{2.726890in}}%
\pgfpathlineto{\pgfqpoint{1.694039in}{2.728715in}}%
\pgfpathlineto{\pgfqpoint{1.689482in}{2.733676in}}%
\pgfpathlineto{\pgfqpoint{1.688290in}{2.734975in}}%
\pgfpathlineto{\pgfqpoint{1.683222in}{2.740587in}}%
\pgfpathlineto{\pgfqpoint{1.682636in}{2.741235in}}%
\pgfpathlineto{\pgfqpoint{1.677078in}{2.747496in}}%
\pgfpathlineto{\pgfqpoint{1.676962in}{2.747629in}}%
\pgfpathlineto{\pgfqpoint{1.671610in}{2.753756in}}%
\pgfpathlineto{\pgfqpoint{1.670701in}{2.754817in}}%
\pgfpathlineto{\pgfqpoint{1.666238in}{2.760017in}}%
\pgfpathlineto{\pgfqpoint{1.664441in}{2.762153in}}%
\pgfpathlineto{\pgfqpoint{1.660963in}{2.766277in}}%
\pgfpathlineto{\pgfqpoint{1.658180in}{2.769646in}}%
\pgfpathlineto{\pgfqpoint{1.655786in}{2.772538in}}%
\pgfpathlineto{\pgfqpoint{1.651920in}{2.777306in}}%
\pgfpathlineto{\pgfqpoint{1.650707in}{2.778798in}}%
\pgfpathlineto{\pgfqpoint{1.645726in}{2.785058in}}%
\pgfpathlineto{\pgfqpoint{1.645659in}{2.785144in}}%
\pgfpathlineto{\pgfqpoint{1.640834in}{2.791319in}}%
\pgfpathlineto{\pgfqpoint{1.639399in}{2.793199in}}%
\pgfpathlineto{\pgfqpoint{1.636042in}{2.797579in}}%
\pgfpathlineto{\pgfqpoint{1.633139in}{2.801462in}}%
\pgfpathlineto{\pgfqpoint{1.631352in}{2.803840in}}%
\pgfpathlineto{\pgfqpoint{1.626878in}{2.809947in}}%
\pgfpathlineto{\pgfqpoint{1.626765in}{2.810100in}}%
\pgfpathlineto{\pgfqpoint{1.622269in}{2.816361in}}%
\pgfpathlineto{\pgfqpoint{1.620618in}{2.818722in}}%
\pgfpathlineto{\pgfqpoint{1.617876in}{2.822621in}}%
\pgfpathlineto{\pgfqpoint{1.614357in}{2.827767in}}%
\pgfpathlineto{\pgfqpoint{1.613591in}{2.828881in}}%
\pgfpathlineto{\pgfqpoint{1.609404in}{2.835142in}}%
\pgfpathlineto{\pgfqpoint{1.608097in}{2.837155in}}%
\pgfpathlineto{\pgfqpoint{1.605320in}{2.841402in}}%
\pgfpathlineto{\pgfqpoint{1.601836in}{2.846900in}}%
\pgfpathlineto{\pgfqpoint{1.601349in}{2.847663in}}%
\pgfpathlineto{\pgfqpoint{1.597478in}{2.853923in}}%
\pgfpathlineto{\pgfqpoint{1.595576in}{2.857104in}}%
\pgfpathlineto{\pgfqpoint{1.593719in}{2.860184in}}%
\pgfpathlineto{\pgfqpoint{1.590074in}{2.866444in}}%
\pgfpathlineto{\pgfqpoint{1.589316in}{2.867795in}}%
\pgfpathlineto{\pgfqpoint{1.586536in}{2.872705in}}%
\pgfpathlineto{\pgfqpoint{1.583120in}{2.878965in}}%
\pgfpathlineto{\pgfqpoint{1.583055in}{2.879090in}}%
\pgfpathlineto{\pgfqpoint{1.579814in}{2.885225in}}%
\pgfpathlineto{\pgfqpoint{1.576795in}{2.891175in}}%
\pgfpathlineto{\pgfqpoint{1.576635in}{2.891486in}}%
\pgfpathlineto{\pgfqpoint{1.573572in}{2.897746in}}%
\pgfpathlineto{\pgfqpoint{1.570642in}{2.904007in}}%
\pgfpathlineto{\pgfqpoint{1.570534in}{2.904249in}}%
\pgfpathlineto{\pgfqpoint{1.567834in}{2.910267in}}%
\pgfpathlineto{\pgfqpoint{1.565164in}{2.916528in}}%
\pgfpathlineto{\pgfqpoint{1.564274in}{2.918742in}}%
\pgfpathlineto{\pgfqpoint{1.562628in}{2.922788in}}%
\pgfpathlineto{\pgfqpoint{1.560233in}{2.929048in}}%
\pgfpathlineto{\pgfqpoint{1.558013in}{2.935233in}}%
\pgfpathlineto{\pgfqpoint{1.557986in}{2.935309in}}%
\pgfpathlineto{\pgfqpoint{1.555886in}{2.941569in}}%
\pgfpathlineto{\pgfqpoint{1.553945in}{2.947830in}}%
\pgfpathlineto{\pgfqpoint{1.552165in}{2.954090in}}%
\pgfpathlineto{\pgfqpoint{1.551753in}{2.955710in}}%
\pgfpathlineto{\pgfqpoint{1.550555in}{2.960351in}}%
\pgfpathlineto{\pgfqpoint{1.549121in}{2.966611in}}%
\pgfpathlineto{\pgfqpoint{1.547873in}{2.972871in}}%
\pgfpathlineto{\pgfqpoint{1.546818in}{2.979132in}}%
\pgfpathlineto{\pgfqpoint{1.545968in}{2.985392in}}%
\pgfpathlineto{\pgfqpoint{1.545492in}{2.990100in}}%
\pgfpathlineto{\pgfqpoint{1.545333in}{2.991653in}}%
\pgfpathlineto{\pgfqpoint{1.544930in}{2.997913in}}%
\pgfpathlineto{\pgfqpoint{1.544771in}{3.004174in}}%
\pgfpathlineto{\pgfqpoint{1.544874in}{3.010434in}}%
\pgfpathlineto{\pgfqpoint{1.545256in}{3.016694in}}%
\pgfpathlineto{\pgfqpoint{1.545492in}{3.018892in}}%
\pgfpathlineto{\pgfqpoint{1.545946in}{3.022955in}}%
\pgfpathlineto{\pgfqpoint{1.546970in}{3.029215in}}%
\pgfpathlineto{\pgfqpoint{1.548357in}{3.035476in}}%
\pgfpathlineto{\pgfqpoint{1.550141in}{3.041736in}}%
\pgfpathlineto{\pgfqpoint{1.551753in}{3.046325in}}%
\pgfpathlineto{\pgfqpoint{1.552371in}{3.047997in}}%
\pgfpathlineto{\pgfqpoint{1.555122in}{3.054257in}}%
\pgfpathlineto{\pgfqpoint{1.558013in}{3.059764in}}%
\pgfpathlineto{\pgfqpoint{1.558434in}{3.060518in}}%
\pgfpathlineto{\pgfqpoint{1.562443in}{3.066778in}}%
\pgfpathlineto{\pgfqpoint{1.564274in}{3.069269in}}%
\pgfpathlineto{\pgfqpoint{1.567251in}{3.073038in}}%
\pgfpathlineto{\pgfqpoint{1.570534in}{3.076713in}}%
\pgfpathlineto{\pgfqpoint{1.573038in}{3.079299in}}%
\pgfpathlineto{\pgfqpoint{1.576795in}{3.082782in}}%
\pgfpathlineto{\pgfqpoint{1.580073in}{3.085559in}}%
\pgfpathlineto{\pgfqpoint{1.583055in}{3.087858in}}%
\pgfpathlineto{\pgfqpoint{1.588744in}{3.091820in}}%
\pgfpathlineto{\pgfqpoint{1.589316in}{3.092186in}}%
\pgfpathlineto{\pgfqpoint{1.595576in}{3.095852in}}%
\pgfpathlineto{\pgfqpoint{1.599857in}{3.098080in}}%
\pgfpathlineto{\pgfqpoint{1.601836in}{3.099040in}}%
\pgfpathlineto{\pgfqpoint{1.608097in}{3.101772in}}%
\pgfpathlineto{\pgfqpoint{1.614357in}{3.104159in}}%
\pgfpathlineto{\pgfqpoint{1.614889in}{3.104341in}}%
\pgfpathlineto{\pgfqpoint{1.620618in}{3.106185in}}%
\pgfpathlineto{\pgfqpoint{1.626878in}{3.107931in}}%
\pgfpathlineto{\pgfqpoint{1.633139in}{3.109422in}}%
\pgfpathlineto{\pgfqpoint{1.638995in}{3.110601in}}%
\pgfpathlineto{\pgfqpoint{1.639399in}{3.110679in}}%
\pgfpathlineto{\pgfqpoint{1.645659in}{3.111703in}}%
\pgfpathlineto{\pgfqpoint{1.651920in}{3.112533in}}%
\pgfpathlineto{\pgfqpoint{1.658180in}{3.113180in}}%
\pgfpathlineto{\pgfqpoint{1.664441in}{3.113658in}}%
\pgfpathlineto{\pgfqpoint{1.670701in}{3.113977in}}%
\pgfpathlineto{\pgfqpoint{1.676962in}{3.114146in}}%
\pgfpathlineto{\pgfqpoint{1.683222in}{3.114174in}}%
\pgfpathlineto{\pgfqpoint{1.689482in}{3.114069in}}%
\pgfpathlineto{\pgfqpoint{1.695743in}{3.113838in}}%
\pgfpathlineto{\pgfqpoint{1.702003in}{3.113488in}}%
\pgfpathlineto{\pgfqpoint{1.708264in}{3.113024in}}%
\pgfpathlineto{\pgfqpoint{1.714524in}{3.112452in}}%
\pgfpathlineto{\pgfqpoint{1.720785in}{3.111775in}}%
\pgfpathlineto{\pgfqpoint{1.727045in}{3.110998in}}%
\pgfpathlineto{\pgfqpoint{1.729875in}{3.110601in}}%
\pgfpathlineto{\pgfqpoint{1.733306in}{3.110125in}}%
\pgfpathlineto{\pgfqpoint{1.739566in}{3.109159in}}%
\pgfpathlineto{\pgfqpoint{1.745826in}{3.108104in}}%
\pgfpathlineto{\pgfqpoint{1.752087in}{3.106963in}}%
\pgfpathlineto{\pgfqpoint{1.758347in}{3.105739in}}%
\pgfpathlineto{\pgfqpoint{1.764608in}{3.104432in}}%
\pgfpathlineto{\pgfqpoint{1.765019in}{3.104341in}}%
\pgfpathlineto{\pgfqpoint{1.770868in}{3.103050in}}%
\pgfpathlineto{\pgfqpoint{1.777129in}{3.101591in}}%
\pgfpathlineto{\pgfqpoint{1.783389in}{3.100056in}}%
\pgfpathlineto{\pgfqpoint{1.789649in}{3.098448in}}%
\pgfpathlineto{\pgfqpoint{1.791011in}{3.098080in}}%
\pgfpathlineto{\pgfqpoint{1.795910in}{3.096772in}}%
\pgfpathlineto{\pgfqpoint{1.802170in}{3.095028in}}%
\pgfpathlineto{\pgfqpoint{1.808431in}{3.093213in}}%
\pgfpathlineto{\pgfqpoint{1.813055in}{3.091820in}}%
\pgfpathlineto{\pgfqpoint{1.814691in}{3.091332in}}%
\pgfpathlineto{\pgfqpoint{1.820952in}{3.089392in}}%
\pgfpathlineto{\pgfqpoint{1.827212in}{3.087384in}}%
\pgfpathlineto{\pgfqpoint{1.832716in}{3.085559in}}%
\pgfpathlineto{\pgfqpoint{1.833472in}{3.085311in}}%
\pgfpathlineto{\pgfqpoint{1.839733in}{3.083185in}}%
\pgfpathlineto{\pgfqpoint{1.845993in}{3.080995in}}%
\pgfpathlineto{\pgfqpoint{1.850695in}{3.079299in}}%
\pgfpathlineto{\pgfqpoint{1.852254in}{3.078742in}}%
\pgfpathlineto{\pgfqpoint{1.858514in}{3.076439in}}%
\pgfpathlineto{\pgfqpoint{1.864775in}{3.074072in}}%
\pgfpathlineto{\pgfqpoint{1.867432in}{3.073038in}}%
\pgfpathlineto{\pgfqpoint{1.871035in}{3.071651in}}%
\pgfpathlineto{\pgfqpoint{1.877295in}{3.069176in}}%
\pgfpathlineto{\pgfqpoint{1.883208in}{3.066778in}}%
\pgfpathlineto{\pgfqpoint{1.883556in}{3.066638in}}%
\pgfpathlineto{\pgfqpoint{1.889816in}{3.064058in}}%
\pgfpathlineto{\pgfqpoint{1.896077in}{3.061414in}}%
\pgfpathlineto{\pgfqpoint{1.898150in}{3.060518in}}%
\pgfpathlineto{\pgfqpoint{1.902337in}{3.058723in}}%
\pgfpathlineto{\pgfqpoint{1.908598in}{3.055976in}}%
\pgfpathlineto{\pgfqpoint{1.912428in}{3.054257in}}%
\pgfpathlineto{\pgfqpoint{1.914858in}{3.053176in}}%
\pgfpathlineto{\pgfqpoint{1.921119in}{3.050328in}}%
\pgfpathlineto{\pgfqpoint{1.926131in}{3.047997in}}%
\pgfpathlineto{\pgfqpoint{1.927379in}{3.047421in}}%
\pgfpathlineto{\pgfqpoint{1.933639in}{3.044473in}}%
\pgfpathlineto{\pgfqpoint{1.939326in}{3.041736in}}%
\pgfpathlineto{\pgfqpoint{1.939900in}{3.041462in}}%
\pgfpathlineto{\pgfqpoint{1.946160in}{3.038415in}}%
\pgfpathlineto{\pgfqpoint{1.952069in}{3.035476in}}%
\pgfpathlineto{\pgfqpoint{1.952421in}{3.035302in}}%
\pgfpathlineto{\pgfqpoint{1.958681in}{3.032155in}}%
\pgfpathlineto{\pgfqpoint{1.964407in}{3.029215in}}%
\pgfpathlineto{\pgfqpoint{1.964942in}{3.028943in}}%
\pgfpathlineto{\pgfqpoint{1.971202in}{3.025696in}}%
\pgfpathlineto{\pgfqpoint{1.976381in}{3.022955in}}%
\pgfpathlineto{\pgfqpoint{1.977462in}{3.022386in}}%
\pgfpathlineto{\pgfqpoint{1.983723in}{3.019038in}}%
\pgfpathlineto{\pgfqpoint{1.988024in}{3.016694in}}%
\pgfpathlineto{\pgfqpoint{1.989983in}{3.015633in}}%
\pgfpathlineto{\pgfqpoint{1.996244in}{3.012183in}}%
\pgfpathlineto{\pgfqpoint{1.999364in}{3.010434in}}%
\pgfpathlineto{\pgfqpoint{2.002504in}{3.008682in}}%
\pgfpathlineto{\pgfqpoint{2.008765in}{3.005130in}}%
\pgfpathlineto{\pgfqpoint{2.010424in}{3.004174in}}%
\pgfpathlineto{\pgfqpoint{2.015025in}{3.001534in}}%
\pgfpathlineto{\pgfqpoint{2.021224in}{2.997913in}}%
\pgfpathlineto{\pgfqpoint{2.021285in}{2.997877in}}%
\pgfpathlineto{\pgfqpoint{2.027546in}{2.994188in}}%
\pgfpathlineto{\pgfqpoint{2.031773in}{2.991653in}}%
\pgfpathlineto{\pgfqpoint{2.033806in}{2.990438in}}%
\pgfpathlineto{\pgfqpoint{2.040067in}{2.986641in}}%
\pgfpathlineto{\pgfqpoint{2.042096in}{2.985392in}}%
\pgfpathlineto{\pgfqpoint{2.046327in}{2.982797in}}%
\pgfpathlineto{\pgfqpoint{2.052204in}{2.979132in}}%
\pgfpathlineto{\pgfqpoint{2.052588in}{2.978893in}}%
\pgfpathlineto{\pgfqpoint{2.058848in}{2.974953in}}%
\pgfpathlineto{\pgfqpoint{2.062105in}{2.972871in}}%
\pgfpathlineto{\pgfqpoint{2.065109in}{2.970955in}}%
\pgfpathlineto{\pgfqpoint{2.071369in}{2.966902in}}%
\pgfpathlineto{\pgfqpoint{2.071814in}{2.966611in}}%
\pgfpathlineto{\pgfqpoint{2.077629in}{2.962810in}}%
\pgfpathlineto{\pgfqpoint{2.081336in}{2.960351in}}%
\pgfpathlineto{\pgfqpoint{2.083890in}{2.958658in}}%
\pgfpathlineto{\pgfqpoint{2.090150in}{2.954452in}}%
\pgfpathlineto{\pgfqpoint{2.090683in}{2.954090in}}%
\pgfpathlineto{\pgfqpoint{2.096411in}{2.950203in}}%
\pgfpathlineto{\pgfqpoint{2.099860in}{2.947830in}}%
\pgfpathlineto{\pgfqpoint{2.102671in}{2.945896in}}%
\pgfpathlineto{\pgfqpoint{2.108875in}{2.941569in}}%
\pgfpathlineto{\pgfqpoint{2.108932in}{2.941530in}}%
\pgfpathlineto{\pgfqpoint{2.115192in}{2.937123in}}%
\pgfpathlineto{\pgfqpoint{2.117736in}{2.935309in}}%
\pgfpathlineto{\pgfqpoint{2.121452in}{2.932656in}}%
\pgfpathlineto{\pgfqpoint{2.126445in}{2.929048in}}%
\pgfpathlineto{\pgfqpoint{2.127713in}{2.928131in}}%
\pgfpathlineto{\pgfqpoint{2.133973in}{2.923553in}}%
\pgfpathlineto{\pgfqpoint{2.135009in}{2.922788in}}%
\pgfpathlineto{\pgfqpoint{2.140234in}{2.918923in}}%
\pgfpathlineto{\pgfqpoint{2.143435in}{2.916528in}}%
\pgfpathlineto{\pgfqpoint{2.146494in}{2.914232in}}%
\pgfpathlineto{\pgfqpoint{2.151722in}{2.910267in}}%
\pgfpathlineto{\pgfqpoint{2.152755in}{2.909481in}}%
\pgfpathlineto{\pgfqpoint{2.159015in}{2.904675in}}%
\pgfpathlineto{\pgfqpoint{2.159879in}{2.904007in}}%
\pgfpathlineto{\pgfqpoint{2.165275in}{2.899813in}}%
\pgfpathlineto{\pgfqpoint{2.167910in}{2.897746in}}%
\pgfpathlineto{\pgfqpoint{2.171536in}{2.894889in}}%
\pgfpathlineto{\pgfqpoint{2.175814in}{2.891486in}}%
\pgfpathlineto{\pgfqpoint{2.177796in}{2.889901in}}%
\pgfpathlineto{\pgfqpoint{2.183595in}{2.885225in}}%
\pgfpathlineto{\pgfqpoint{2.184057in}{2.884851in}}%
\pgfpathlineto{\pgfqpoint{2.190317in}{2.879741in}}%
\pgfpathlineto{\pgfqpoint{2.191261in}{2.878965in}}%
\pgfpathlineto{\pgfqpoint{2.196578in}{2.874568in}}%
\pgfpathlineto{\pgfqpoint{2.198814in}{2.872705in}}%
\pgfpathlineto{\pgfqpoint{2.202838in}{2.869329in}}%
\pgfpathlineto{\pgfqpoint{2.206252in}{2.866444in}}%
\pgfpathlineto{\pgfqpoint{2.209098in}{2.864022in}}%
\pgfpathlineto{\pgfqpoint{2.213580in}{2.860184in}}%
\pgfpathlineto{\pgfqpoint{2.215359in}{2.858648in}}%
\pgfpathlineto{\pgfqpoint{2.220799in}{2.853923in}}%
\pgfpathlineto{\pgfqpoint{2.221619in}{2.853205in}}%
\pgfpathlineto{\pgfqpoint{2.227880in}{2.847693in}}%
\pgfpathlineto{\pgfqpoint{2.227914in}{2.847663in}}%
\pgfpathlineto{\pgfqpoint{2.234140in}{2.842112in}}%
\pgfpathlineto{\pgfqpoint{2.234932in}{2.841402in}}%
\pgfpathlineto{\pgfqpoint{2.240401in}{2.836457in}}%
\pgfpathlineto{\pgfqpoint{2.241849in}{2.835142in}}%
\pgfpathlineto{\pgfqpoint{2.246661in}{2.830729in}}%
\pgfpathlineto{\pgfqpoint{2.248668in}{2.828881in}}%
\pgfpathlineto{\pgfqpoint{2.252922in}{2.824926in}}%
\pgfpathlineto{\pgfqpoint{2.255391in}{2.822621in}}%
\pgfpathlineto{\pgfqpoint{2.259182in}{2.819047in}}%
\pgfpathlineto{\pgfqpoint{2.262022in}{2.816361in}}%
\pgfpathlineto{\pgfqpoint{2.265442in}{2.813090in}}%
\pgfpathlineto{\pgfqpoint{2.268562in}{2.810100in}}%
\pgfpathlineto{\pgfqpoint{2.271703in}{2.807056in}}%
\pgfpathlineto{\pgfqpoint{2.275015in}{2.803840in}}%
\pgfpathlineto{\pgfqpoint{2.277963in}{2.800943in}}%
\pgfpathlineto{\pgfqpoint{2.281383in}{2.797579in}}%
\pgfpathlineto{\pgfqpoint{2.284224in}{2.794750in}}%
\pgfpathlineto{\pgfqpoint{2.287667in}{2.791319in}}%
\pgfpathlineto{\pgfqpoint{2.290484in}{2.788476in}}%
\pgfpathlineto{\pgfqpoint{2.293872in}{2.785058in}}%
\pgfpathlineto{\pgfqpoint{2.296745in}{2.782122in}}%
\pgfpathlineto{\pgfqpoint{2.300001in}{2.778798in}}%
\pgfpathlineto{\pgfqpoint{2.303005in}{2.775688in}}%
\pgfpathlineto{\pgfqpoint{2.306054in}{2.772538in}}%
\pgfpathlineto{\pgfqpoint{2.309265in}{2.769173in}}%
\pgfpathlineto{\pgfqpoint{2.312037in}{2.766277in}}%
\pgfpathlineto{\pgfqpoint{2.315526in}{2.762579in}}%
\pgfpathlineto{\pgfqpoint{2.317952in}{2.760017in}}%
\pgfpathlineto{\pgfqpoint{2.321786in}{2.755908in}}%
\pgfpathlineto{\pgfqpoint{2.323803in}{2.753756in}}%
\pgfpathlineto{\pgfqpoint{2.328047in}{2.749160in}}%
\pgfpathlineto{\pgfqpoint{2.329592in}{2.747496in}}%
\pgfpathlineto{\pgfqpoint{2.334307in}{2.742340in}}%
\pgfpathlineto{\pgfqpoint{2.335325in}{2.741235in}}%
\pgfpathlineto{\pgfqpoint{2.340568in}{2.735452in}}%
\pgfpathlineto{\pgfqpoint{2.341003in}{2.734975in}}%
\pgfpathlineto{\pgfqpoint{2.346636in}{2.728715in}}%
\pgfpathlineto{\pgfqpoint{2.346828in}{2.728498in}}%
\pgfpathlineto{\pgfqpoint{2.352232in}{2.722454in}}%
\pgfpathlineto{\pgfqpoint{2.353088in}{2.721481in}}%
\pgfpathlineto{\pgfqpoint{2.357791in}{2.716194in}}%
\pgfpathlineto{\pgfqpoint{2.359349in}{2.714412in}}%
\pgfpathlineto{\pgfqpoint{2.363317in}{2.709933in}}%
\pgfpathlineto{\pgfqpoint{2.365609in}{2.707302in}}%
\pgfpathlineto{\pgfqpoint{2.368817in}{2.703673in}}%
\pgfpathlineto{\pgfqpoint{2.371870in}{2.700161in}}%
\pgfpathlineto{\pgfqpoint{2.374298in}{2.697412in}}%
\pgfpathlineto{\pgfqpoint{2.378130in}{2.693002in}}%
\pgfpathlineto{\pgfqpoint{2.379766in}{2.691152in}}%
\pgfpathlineto{\pgfqpoint{2.384391in}{2.685838in}}%
\pgfpathlineto{\pgfqpoint{2.385231in}{2.684891in}}%
\pgfpathlineto{\pgfqpoint{2.390651in}{2.678687in}}%
\pgfpathlineto{\pgfqpoint{2.390701in}{2.678631in}}%
\pgfpathlineto{\pgfqpoint{2.396206in}{2.672371in}}%
\pgfpathlineto{\pgfqpoint{2.396912in}{2.671554in}}%
\pgfpathlineto{\pgfqpoint{2.401742in}{2.666110in}}%
\pgfpathlineto{\pgfqpoint{2.403172in}{2.664472in}}%
\pgfpathlineto{\pgfqpoint{2.407324in}{2.659850in}}%
\pgfpathlineto{\pgfqpoint{2.409432in}{2.657465in}}%
\pgfpathlineto{\pgfqpoint{2.412968in}{2.653589in}}%
\pgfpathlineto{\pgfqpoint{2.415693in}{2.650556in}}%
\pgfpathlineto{\pgfqpoint{2.418695in}{2.647329in}}%
\pgfpathlineto{\pgfqpoint{2.421953in}{2.643773in}}%
\pgfpathlineto{\pgfqpoint{2.424530in}{2.641068in}}%
\pgfpathlineto{\pgfqpoint{2.428214in}{2.637144in}}%
\pgfpathlineto{\pgfqpoint{2.430504in}{2.634808in}}%
\pgfpathlineto{\pgfqpoint{2.434474in}{2.630699in}}%
\pgfpathlineto{\pgfqpoint{2.436656in}{2.628548in}}%
\pgfpathlineto{\pgfqpoint{2.440735in}{2.624468in}}%
\pgfpathlineto{\pgfqpoint{2.443037in}{2.622287in}}%
\pgfpathlineto{\pgfqpoint{2.446995in}{2.618484in}}%
\pgfpathlineto{\pgfqpoint{2.449713in}{2.616027in}}%
\pgfpathlineto{\pgfqpoint{2.453255in}{2.612780in}}%
\pgfpathlineto{\pgfqpoint{2.456776in}{2.609766in}}%
\pgfpathlineto{\pgfqpoint{2.459516in}{2.607388in}}%
\pgfpathlineto{\pgfqpoint{2.464349in}{2.603506in}}%
\pgfpathlineto{\pgfqpoint{2.465776in}{2.602344in}}%
\pgfpathlineto{\pgfqpoint{2.472037in}{2.597674in}}%
\pgfpathlineto{\pgfqpoint{2.472670in}{2.597245in}}%
\pgfpathlineto{\pgfqpoint{2.478297in}{2.593390in}}%
\pgfpathlineto{\pgfqpoint{2.482240in}{2.590985in}}%
\pgfpathlineto{\pgfqpoint{2.484558in}{2.589551in}}%
\pgfpathlineto{\pgfqpoint{2.490818in}{2.586167in}}%
\pgfpathlineto{\pgfqpoint{2.493951in}{2.584725in}}%
\pgfpathlineto{\pgfqpoint{2.497078in}{2.583264in}}%
\pgfpathlineto{\pgfqpoint{2.503339in}{2.580866in}}%
\pgfpathlineto{\pgfqpoint{2.509599in}{2.579009in}}%
\pgfpathlineto{\pgfqpoint{2.512215in}{2.578464in}}%
\pgfpathlineto{\pgfqpoint{2.515860in}{2.577693in}}%
\pgfpathlineto{\pgfqpoint{2.522120in}{2.576947in}}%
\pgfpathlineto{\pgfqpoint{2.528381in}{2.576800in}}%
\pgfpathlineto{\pgfqpoint{2.534641in}{2.577271in}}%
\pgfpathlineto{\pgfqpoint{2.540902in}{2.578385in}}%
\pgfpathlineto{\pgfqpoint{2.541181in}{2.578464in}}%
\pgfpathlineto{\pgfqpoint{2.547162in}{2.580142in}}%
\pgfpathlineto{\pgfqpoint{2.553422in}{2.582591in}}%
\pgfpathlineto{\pgfqpoint{2.557636in}{2.584725in}}%
\pgfpathlineto{\pgfqpoint{2.559683in}{2.585752in}}%
\pgfpathlineto{\pgfqpoint{2.565943in}{2.589641in}}%
\pgfpathlineto{\pgfqpoint{2.567748in}{2.590985in}}%
\pgfpathlineto{\pgfqpoint{2.572204in}{2.594288in}}%
\pgfpathlineto{\pgfqpoint{2.575605in}{2.597245in}}%
\pgfpathlineto{\pgfqpoint{2.578464in}{2.599725in}}%
\pgfpathlineto{\pgfqpoint{2.582250in}{2.603506in}}%
\pgfpathlineto{\pgfqpoint{2.584725in}{2.605975in}}%
\pgfpathlineto{\pgfqpoint{2.588073in}{2.609766in}}%
\pgfpathlineto{\pgfqpoint{2.590985in}{2.613065in}}%
\pgfpathlineto{\pgfqpoint{2.593317in}{2.616027in}}%
\pgfpathlineto{\pgfqpoint{2.597245in}{2.621021in}}%
\pgfpathlineto{\pgfqpoint{2.598143in}{2.622287in}}%
\pgfpathlineto{\pgfqpoint{2.602578in}{2.628548in}}%
\pgfpathlineto{\pgfqpoint{2.603506in}{2.629859in}}%
\pgfpathlineto{\pgfqpoint{2.606693in}{2.634808in}}%
\pgfpathlineto{\pgfqpoint{2.609766in}{2.639592in}}%
\pgfpathlineto{\pgfqpoint{2.610636in}{2.641068in}}%
\pgfpathlineto{\pgfqpoint{2.614324in}{2.647329in}}%
\pgfpathlineto{\pgfqpoint{2.616027in}{2.650223in}}%
\pgfpathlineto{\pgfqpoint{2.617857in}{2.653589in}}%
\pgfpathlineto{\pgfqpoint{2.621257in}{2.659850in}}%
\pgfpathlineto{\pgfqpoint{2.622287in}{2.661744in}}%
\pgfpathlineto{\pgfqpoint{2.624495in}{2.666110in}}%
\pgfpathlineto{\pgfqpoint{2.627660in}{2.672371in}}%
\pgfpathlineto{\pgfqpoint{2.628548in}{2.674123in}}%
\pgfpathlineto{\pgfqpoint{2.630685in}{2.678631in}}%
\pgfpathlineto{\pgfqpoint{2.633657in}{2.684891in}}%
\pgfpathlineto{\pgfqpoint{2.634808in}{2.687308in}}%
\pgfpathlineto{\pgfqpoint{2.636530in}{2.691152in}}%
\pgfpathlineto{\pgfqpoint{2.639345in}{2.697412in}}%
\pgfpathlineto{\pgfqpoint{2.641068in}{2.701229in}}%
\pgfpathlineto{\pgfqpoint{2.642112in}{2.703673in}}%
\pgfpathlineto{\pgfqpoint{2.644800in}{2.709933in}}%
\pgfpathlineto{\pgfqpoint{2.647329in}{2.715789in}}%
\pgfpathlineto{\pgfqpoint{2.647495in}{2.716194in}}%
\pgfpathlineto{\pgfqpoint{2.650080in}{2.722454in}}%
\pgfpathlineto{\pgfqpoint{2.652687in}{2.728715in}}%
\pgfpathlineto{\pgfqpoint{2.653589in}{2.730865in}}%
\pgfpathlineto{\pgfqpoint{2.655235in}{2.734975in}}%
\pgfpathlineto{\pgfqpoint{2.657767in}{2.741235in}}%
\pgfpathlineto{\pgfqpoint{2.659850in}{2.746332in}}%
\pgfpathlineto{\pgfqpoint{2.660305in}{2.747496in}}%
\pgfpathlineto{\pgfqpoint{2.662780in}{2.753756in}}%
\pgfpathlineto{\pgfqpoint{2.665288in}{2.760017in}}%
\pgfpathlineto{\pgfqpoint{2.666110in}{2.762047in}}%
\pgfpathlineto{\pgfqpoint{2.667758in}{2.766277in}}%
\pgfpathlineto{\pgfqpoint{2.670230in}{2.772538in}}%
\pgfpathlineto{\pgfqpoint{2.672371in}{2.777875in}}%
\pgfpathlineto{\pgfqpoint{2.672728in}{2.778798in}}%
\pgfpathlineto{\pgfqpoint{2.675179in}{2.785058in}}%
\pgfpathlineto{\pgfqpoint{2.677675in}{2.791319in}}%
\pgfpathlineto{\pgfqpoint{2.678631in}{2.793685in}}%
\pgfpathlineto{\pgfqpoint{2.680156in}{2.797579in}}%
\pgfpathlineto{\pgfqpoint{2.682648in}{2.803840in}}%
\pgfpathlineto{\pgfqpoint{2.684891in}{2.809367in}}%
\pgfpathlineto{\pgfqpoint{2.685180in}{2.810100in}}%
\pgfpathlineto{\pgfqpoint{2.687681in}{2.816361in}}%
\pgfpathlineto{\pgfqpoint{2.690237in}{2.822621in}}%
\pgfpathlineto{\pgfqpoint{2.691152in}{2.824828in}}%
\pgfpathlineto{\pgfqpoint{2.692791in}{2.828881in}}%
\pgfpathlineto{\pgfqpoint{2.695370in}{2.835142in}}%
\pgfpathlineto{\pgfqpoint{2.697412in}{2.839993in}}%
\pgfpathlineto{\pgfqpoint{2.697992in}{2.841402in}}%
\pgfpathlineto{\pgfqpoint{2.700606in}{2.847663in}}%
\pgfpathlineto{\pgfqpoint{2.703288in}{2.853923in}}%
\pgfpathlineto{\pgfqpoint{2.703673in}{2.854810in}}%
\pgfpathlineto{\pgfqpoint{2.705960in}{2.860184in}}%
\pgfpathlineto{\pgfqpoint{2.708690in}{2.866444in}}%
\pgfpathlineto{\pgfqpoint{2.709933in}{2.869243in}}%
\pgfpathlineto{\pgfqpoint{2.711444in}{2.872705in}}%
\pgfpathlineto{\pgfqpoint{2.714233in}{2.878965in}}%
\pgfpathlineto{\pgfqpoint{2.716194in}{2.883267in}}%
\pgfpathlineto{\pgfqpoint{2.717072in}{2.885225in}}%
\pgfpathlineto{\pgfqpoint{2.719930in}{2.891486in}}%
\pgfpathlineto{\pgfqpoint{2.722454in}{2.896872in}}%
\pgfpathlineto{\pgfqpoint{2.722858in}{2.897746in}}%
\pgfpathlineto{\pgfqpoint{2.725795in}{2.904007in}}%
\pgfpathlineto{\pgfqpoint{2.728715in}{2.910055in}}%
\pgfpathlineto{\pgfqpoint{2.728816in}{2.910267in}}%
\pgfpathlineto{\pgfqpoint{2.731842in}{2.916528in}}%
\pgfpathlineto{\pgfqpoint{2.734959in}{2.922788in}}%
\pgfpathlineto{\pgfqpoint{2.734975in}{2.922819in}}%
\pgfpathlineto{\pgfqpoint{2.738084in}{2.929048in}}%
\pgfpathlineto{\pgfqpoint{2.741235in}{2.935175in}}%
\pgfpathlineto{\pgfqpoint{2.741304in}{2.935309in}}%
\pgfpathlineto{\pgfqpoint{2.744538in}{2.941569in}}%
\pgfpathlineto{\pgfqpoint{2.747496in}{2.947131in}}%
\pgfpathlineto{\pgfqpoint{2.747865in}{2.947830in}}%
\pgfpathlineto{\pgfqpoint{2.751220in}{2.954090in}}%
\pgfpathlineto{\pgfqpoint{2.753756in}{2.958701in}}%
\pgfpathlineto{\pgfqpoint{2.754660in}{2.960351in}}%
\pgfpathlineto{\pgfqpoint{2.758147in}{2.966611in}}%
\pgfpathlineto{\pgfqpoint{2.760017in}{2.969896in}}%
\pgfpathlineto{\pgfqpoint{2.761707in}{2.972871in}}%
\pgfpathlineto{\pgfqpoint{2.765337in}{2.979132in}}%
\pgfpathlineto{\pgfqpoint{2.766277in}{2.980726in}}%
\pgfpathlineto{\pgfqpoint{2.769026in}{2.985392in}}%
\pgfpathlineto{\pgfqpoint{2.772538in}{2.991206in}}%
\pgfpathlineto{\pgfqpoint{2.772808in}{2.991653in}}%
\pgfpathlineto{\pgfqpoint{2.776639in}{2.997913in}}%
\pgfpathlineto{\pgfqpoint{2.778798in}{3.001362in}}%
\pgfpathlineto{\pgfqpoint{2.780563in}{3.004174in}}%
\pgfpathlineto{\pgfqpoint{2.784572in}{3.010434in}}%
\pgfpathlineto{\pgfqpoint{2.785058in}{3.011184in}}%
\pgfpathlineto{\pgfqpoint{2.788649in}{3.016694in}}%
\pgfpathlineto{\pgfqpoint{2.791319in}{3.020705in}}%
\pgfpathlineto{\pgfqpoint{2.792826in}{3.022955in}}%
\pgfpathlineto{\pgfqpoint{2.797095in}{3.029215in}}%
\pgfpathlineto{\pgfqpoint{2.797579in}{3.029917in}}%
\pgfpathlineto{\pgfqpoint{2.801447in}{3.035476in}}%
\pgfpathlineto{\pgfqpoint{2.803840in}{3.038851in}}%
\pgfpathlineto{\pgfqpoint{2.805906in}{3.041736in}}%
\pgfpathlineto{\pgfqpoint{2.810100in}{3.047490in}}%
\pgfpathlineto{\pgfqpoint{2.810473in}{3.047997in}}%
\pgfpathlineto{\pgfqpoint{2.815139in}{3.054257in}}%
\pgfpathlineto{\pgfqpoint{2.816361in}{3.055873in}}%
\pgfpathlineto{\pgfqpoint{2.819920in}{3.060518in}}%
\pgfpathlineto{\pgfqpoint{2.822621in}{3.063989in}}%
\pgfpathlineto{\pgfqpoint{2.824824in}{3.066778in}}%
\pgfpathlineto{\pgfqpoint{2.828881in}{3.071843in}}%
\pgfpathlineto{\pgfqpoint{2.829855in}{3.073038in}}%
\pgfpathlineto{\pgfqpoint{2.835019in}{3.079299in}}%
\pgfpathlineto{\pgfqpoint{2.835142in}{3.079446in}}%
\pgfpathlineto{\pgfqpoint{2.840321in}{3.085559in}}%
\pgfpathlineto{\pgfqpoint{2.841402in}{3.086821in}}%
\pgfpathlineto{\pgfqpoint{2.845775in}{3.091820in}}%
\pgfpathlineto{\pgfqpoint{2.847663in}{3.093956in}}%
\pgfpathlineto{\pgfqpoint{2.851391in}{3.098080in}}%
\pgfpathlineto{\pgfqpoint{2.853923in}{3.100857in}}%
\pgfpathlineto{\pgfqpoint{2.857179in}{3.104341in}}%
\pgfpathlineto{\pgfqpoint{2.860184in}{3.107530in}}%
\pgfpathlineto{\pgfqpoint{2.863154in}{3.110601in}}%
\pgfpathlineto{\pgfqpoint{2.866444in}{3.113981in}}%
\pgfpathlineto{\pgfqpoint{2.869329in}{3.116861in}}%
\pgfpathlineto{\pgfqpoint{2.872705in}{3.120213in}}%
\pgfpathlineto{\pgfqpoint{2.875725in}{3.123122in}}%
\pgfpathlineto{\pgfqpoint{2.878965in}{3.126230in}}%
\pgfpathlineto{\pgfqpoint{2.882361in}{3.129382in}}%
\pgfpathlineto{\pgfqpoint{2.885225in}{3.132034in}}%
\pgfpathlineto{\pgfqpoint{2.889262in}{3.135643in}}%
\pgfpathlineto{\pgfqpoint{2.891486in}{3.137627in}}%
\pgfpathlineto{\pgfqpoint{2.896457in}{3.141903in}}%
\pgfpathlineto{\pgfqpoint{2.897746in}{3.143011in}}%
\pgfpathlineto{\pgfqpoint{2.903981in}{3.148164in}}%
\pgfpathlineto{\pgfqpoint{2.904007in}{3.148185in}}%
\pgfpathlineto{\pgfqpoint{2.910267in}{3.153174in}}%
\pgfpathlineto{\pgfqpoint{2.911908in}{3.154424in}}%
\pgfpathlineto{\pgfqpoint{2.916528in}{3.157956in}}%
\pgfpathlineto{\pgfqpoint{2.920268in}{3.160684in}}%
\pgfpathlineto{\pgfqpoint{2.922788in}{3.162530in}}%
\pgfpathlineto{\pgfqpoint{2.929048in}{3.166893in}}%
\pgfpathlineto{\pgfqpoint{2.929127in}{3.166945in}}%
\pgfpathlineto{\pgfqpoint{2.935309in}{3.171082in}}%
\pgfpathlineto{\pgfqpoint{2.938663in}{3.173205in}}%
\pgfpathlineto{\pgfqpoint{2.941569in}{3.175058in}}%
\pgfpathlineto{\pgfqpoint{2.947830in}{3.178828in}}%
\pgfpathlineto{\pgfqpoint{2.948952in}{3.179466in}}%
\pgfpathlineto{\pgfqpoint{2.954090in}{3.182415in}}%
\pgfpathlineto{\pgfqpoint{2.960259in}{3.185726in}}%
\pgfpathlineto{\pgfqpoint{2.960351in}{3.185776in}}%
\pgfpathlineto{\pgfqpoint{2.966611in}{3.188966in}}%
\pgfpathlineto{\pgfqpoint{2.972871in}{3.191924in}}%
\pgfpathlineto{\pgfqpoint{2.973014in}{3.191987in}}%
\pgfpathlineto{\pgfqpoint{2.979132in}{3.194707in}}%
\pgfpathlineto{\pgfqpoint{2.985392in}{3.197263in}}%
\pgfpathlineto{\pgfqpoint{2.988031in}{3.198247in}}%
\pgfpathlineto{\pgfqpoint{2.991653in}{3.199620in}}%
\pgfpathlineto{\pgfqpoint{2.997913in}{3.201768in}}%
\pgfpathlineto{\pgfqpoint{3.004174in}{3.203691in}}%
\pgfpathlineto{\pgfqpoint{3.007187in}{3.204508in}}%
\pgfpathlineto{\pgfqpoint{3.010434in}{3.205405in}}%
\pgfpathlineto{\pgfqpoint{3.016694in}{3.206905in}}%
\pgfpathlineto{\pgfqpoint{3.022955in}{3.208174in}}%
\pgfpathlineto{\pgfqpoint{3.029215in}{3.209211in}}%
\pgfpathlineto{\pgfqpoint{3.035476in}{3.210012in}}%
\pgfpathlineto{\pgfqpoint{3.041736in}{3.210574in}}%
\pgfpathlineto{\pgfqpoint{3.045589in}{3.210768in}}%
\pgfpathlineto{\pgfqpoint{3.047997in}{3.210893in}}%
\pgfpathlineto{\pgfqpoint{3.054257in}{3.210959in}}%
\pgfpathlineto{\pgfqpoint{3.060354in}{3.210768in}}%
\pgfpathlineto{\pgfqpoint{3.060518in}{3.210763in}}%
\pgfpathlineto{\pgfqpoint{3.066778in}{3.210301in}}%
\pgfpathlineto{\pgfqpoint{3.073038in}{3.209560in}}%
\pgfpathlineto{\pgfqpoint{3.079299in}{3.208528in}}%
\pgfpathlineto{\pgfqpoint{3.085559in}{3.207191in}}%
\pgfpathlineto{\pgfqpoint{3.091820in}{3.205533in}}%
\pgfpathlineto{\pgfqpoint{3.095067in}{3.204508in}}%
\pgfpathlineto{\pgfqpoint{3.098080in}{3.203547in}}%
\pgfpathlineto{\pgfqpoint{3.104341in}{3.201217in}}%
\pgfpathlineto{\pgfqpoint{3.110601in}{3.198508in}}%
\pgfpathlineto{\pgfqpoint{3.111139in}{3.198247in}}%
\pgfpathlineto{\pgfqpoint{3.116861in}{3.195421in}}%
\pgfpathlineto{\pgfqpoint{3.122979in}{3.191987in}}%
\pgfpathlineto{\pgfqpoint{3.123122in}{3.191905in}}%
\pgfpathlineto{\pgfqpoint{3.129382in}{3.187953in}}%
\pgfpathlineto{\pgfqpoint{3.132564in}{3.185726in}}%
\pgfpathlineto{\pgfqpoint{3.135643in}{3.183515in}}%
\pgfpathlineto{\pgfqpoint{3.140780in}{3.179466in}}%
\pgfpathlineto{\pgfqpoint{3.141903in}{3.178554in}}%
\pgfpathlineto{\pgfqpoint{3.147964in}{3.173205in}}%
\pgfpathlineto{\pgfqpoint{3.148164in}{3.173023in}}%
\pgfpathlineto{\pgfqpoint{3.154344in}{3.166945in}}%
\pgfpathlineto{\pgfqpoint{3.154424in}{3.166864in}}%
\pgfpathlineto{\pgfqpoint{3.160085in}{3.160684in}}%
\pgfpathlineto{\pgfqpoint{3.160684in}{3.160003in}}%
\pgfpathlineto{\pgfqpoint{3.165302in}{3.154424in}}%
\pgfpathlineto{\pgfqpoint{3.166945in}{3.152349in}}%
\pgfpathlineto{\pgfqpoint{3.170084in}{3.148164in}}%
\pgfpathlineto{\pgfqpoint{3.173205in}{3.143792in}}%
\pgfpathlineto{\pgfqpoint{3.174491in}{3.141903in}}%
\pgfpathlineto{\pgfqpoint{3.178559in}{3.135643in}}%
\pgfpathlineto{\pgfqpoint{3.179466in}{3.134182in}}%
\pgfpathlineto{\pgfqpoint{3.182325in}{3.129382in}}%
\pgfpathlineto{\pgfqpoint{3.185726in}{3.123323in}}%
\pgfpathlineto{\pgfqpoint{3.185835in}{3.123122in}}%
\pgfpathlineto{\pgfqpoint{3.189093in}{3.116861in}}%
\pgfpathlineto{\pgfqpoint{3.191987in}{3.110924in}}%
\pgfpathlineto{\pgfqpoint{3.192140in}{3.110601in}}%
\pgfpathlineto{\pgfqpoint{3.194979in}{3.104341in}}%
\pgfpathlineto{\pgfqpoint{3.197635in}{3.098080in}}%
\pgfpathlineto{\pgfqpoint{3.198247in}{3.096562in}}%
\pgfpathlineto{\pgfqpoint{3.200118in}{3.091820in}}%
\pgfpathlineto{\pgfqpoint{3.202439in}{3.085559in}}%
\pgfpathlineto{\pgfqpoint{3.204508in}{3.079587in}}%
\pgfpathlineto{\pgfqpoint{3.204606in}{3.079299in}}%
\pgfpathlineto{\pgfqpoint{3.206637in}{3.073038in}}%
\pgfpathlineto{\pgfqpoint{3.208530in}{3.066778in}}%
\pgfpathlineto{\pgfqpoint{3.210292in}{3.060518in}}%
\pgfpathlineto{\pgfqpoint{3.210768in}{3.058723in}}%
\pgfpathlineto{\pgfqpoint{3.211941in}{3.054257in}}%
\pgfpathlineto{\pgfqpoint{3.213475in}{3.047997in}}%
\pgfpathlineto{\pgfqpoint{3.214895in}{3.041736in}}%
\pgfpathlineto{\pgfqpoint{3.216207in}{3.035476in}}%
\pgfpathlineto{\pgfqpoint{3.217028in}{3.031241in}}%
\pgfpathlineto{\pgfqpoint{3.217421in}{3.029215in}}%
\pgfpathlineto{\pgfqpoint{3.218546in}{3.022955in}}%
\pgfpathlineto{\pgfqpoint{3.219575in}{3.016694in}}%
\pgfpathlineto{\pgfqpoint{3.220512in}{3.010434in}}%
\pgfpathlineto{\pgfqpoint{3.221361in}{3.004174in}}%
\pgfpathlineto{\pgfqpoint{3.222124in}{2.997913in}}%
\pgfpathlineto{\pgfqpoint{3.222805in}{2.991653in}}%
\pgfpathlineto{\pgfqpoint{3.223289in}{2.986626in}}%
\pgfpathlineto{\pgfqpoint{3.223409in}{2.985392in}}%
\pgfpathlineto{\pgfqpoint{3.223944in}{2.979132in}}%
\pgfpathlineto{\pgfqpoint{3.224404in}{2.972871in}}%
\pgfpathlineto{\pgfqpoint{3.224791in}{2.966611in}}%
\pgfpathlineto{\pgfqpoint{3.225107in}{2.960351in}}%
\pgfpathlineto{\pgfqpoint{3.225354in}{2.954090in}}%
\pgfpathlineto{\pgfqpoint{3.225533in}{2.947830in}}%
\pgfpathlineto{\pgfqpoint{3.225647in}{2.941569in}}%
\pgfpathlineto{\pgfqpoint{3.225697in}{2.935309in}}%
\pgfpathlineto{\pgfqpoint{3.225683in}{2.929048in}}%
\pgfpathlineto{\pgfqpoint{3.225608in}{2.922788in}}%
\pgfpathlineto{\pgfqpoint{3.225473in}{2.916528in}}%
\pgfpathlineto{\pgfqpoint{3.225278in}{2.910267in}}%
\pgfpathlineto{\pgfqpoint{3.225025in}{2.904007in}}%
\pgfpathlineto{\pgfqpoint{3.224714in}{2.897746in}}%
\pgfpathlineto{\pgfqpoint{3.224346in}{2.891486in}}%
\pgfpathlineto{\pgfqpoint{3.223922in}{2.885225in}}%
\pgfpathlineto{\pgfqpoint{3.223442in}{2.878965in}}%
\pgfpathlineto{\pgfqpoint{3.223289in}{2.877174in}}%
\pgfpathlineto{\pgfqpoint{3.222918in}{2.872705in}}%
\pgfpathlineto{\pgfqpoint{3.222345in}{2.866444in}}%
\pgfpathlineto{\pgfqpoint{3.221720in}{2.860184in}}%
\pgfpathlineto{\pgfqpoint{3.221043in}{2.853923in}}%
\pgfpathlineto{\pgfqpoint{3.220314in}{2.847663in}}%
\pgfpathlineto{\pgfqpoint{3.219532in}{2.841402in}}%
\pgfpathlineto{\pgfqpoint{3.218699in}{2.835142in}}%
\pgfpathlineto{\pgfqpoint{3.217814in}{2.828881in}}%
\pgfpathlineto{\pgfqpoint{3.217028in}{2.823634in}}%
\pgfpathlineto{\pgfqpoint{3.216881in}{2.822621in}}%
\pgfpathlineto{\pgfqpoint{3.215923in}{2.816361in}}%
\pgfpathlineto{\pgfqpoint{3.214914in}{2.810100in}}%
\pgfpathlineto{\pgfqpoint{3.213856in}{2.803840in}}%
\pgfpathlineto{\pgfqpoint{3.212747in}{2.797579in}}%
\pgfpathlineto{\pgfqpoint{3.211587in}{2.791319in}}%
\pgfpathlineto{\pgfqpoint{3.210768in}{2.787073in}}%
\pgfpathlineto{\pgfqpoint{3.210389in}{2.785058in}}%
\pgfpathlineto{\pgfqpoint{3.209168in}{2.778798in}}%
\pgfpathlineto{\pgfqpoint{3.207898in}{2.772538in}}%
\pgfpathlineto{\pgfqpoint{3.206579in}{2.766277in}}%
\pgfpathlineto{\pgfqpoint{3.205210in}{2.760017in}}%
\pgfpathlineto{\pgfqpoint{3.204508in}{2.756906in}}%
\pgfpathlineto{\pgfqpoint{3.203814in}{2.753756in}}%
\pgfpathlineto{\pgfqpoint{3.202392in}{2.747496in}}%
\pgfpathlineto{\pgfqpoint{3.200923in}{2.741235in}}%
\pgfpathlineto{\pgfqpoint{3.199405in}{2.734975in}}%
\pgfpathlineto{\pgfqpoint{3.198247in}{2.730337in}}%
\pgfpathlineto{\pgfqpoint{3.197851in}{2.728715in}}%
\pgfpathlineto{\pgfqpoint{3.196288in}{2.722454in}}%
\pgfpathlineto{\pgfqpoint{3.194679in}{2.716194in}}%
\pgfpathlineto{\pgfqpoint{3.193023in}{2.709933in}}%
\pgfpathlineto{\pgfqpoint{3.191987in}{2.706106in}}%
\pgfpathlineto{\pgfqpoint{3.191342in}{2.703673in}}%
\pgfpathlineto{\pgfqpoint{3.189649in}{2.697412in}}%
\pgfpathlineto{\pgfqpoint{3.187913in}{2.691152in}}%
\pgfpathlineto{\pgfqpoint{3.186133in}{2.684891in}}%
\pgfpathlineto{\pgfqpoint{3.185726in}{2.683481in}}%
\pgfpathlineto{\pgfqpoint{3.184355in}{2.678631in}}%
\pgfpathlineto{\pgfqpoint{3.182549in}{2.672371in}}%
\pgfpathlineto{\pgfqpoint{3.180703in}{2.666110in}}%
\pgfpathlineto{\pgfqpoint{3.179466in}{2.661981in}}%
\pgfpathlineto{\pgfqpoint{3.178838in}{2.659850in}}%
\pgfpathlineto{\pgfqpoint{3.176976in}{2.653589in}}%
\pgfpathlineto{\pgfqpoint{3.175079in}{2.647329in}}%
\pgfpathlineto{\pgfqpoint{3.173205in}{2.641252in}}%
\pgfpathlineto{\pgfqpoint{3.173150in}{2.641068in}}%
\pgfpathlineto{\pgfqpoint{3.171247in}{2.634808in}}%
\pgfpathlineto{\pgfqpoint{3.169316in}{2.628548in}}%
\pgfpathlineto{\pgfqpoint{3.167359in}{2.622287in}}%
\pgfpathlineto{\pgfqpoint{3.166945in}{2.620962in}}%
\pgfpathlineto{\pgfqpoint{3.165425in}{2.616027in}}%
\pgfpathlineto{\pgfqpoint{3.163482in}{2.609766in}}%
\pgfpathlineto{\pgfqpoint{3.161523in}{2.603506in}}%
\pgfpathlineto{\pgfqpoint{3.160684in}{2.600818in}}%
\pgfpathlineto{\pgfqpoint{3.159583in}{2.597245in}}%
\pgfpathlineto{\pgfqpoint{3.157658in}{2.590985in}}%
\pgfpathlineto{\pgfqpoint{3.155727in}{2.584725in}}%
\pgfpathlineto{\pgfqpoint{3.154424in}{2.580476in}}%
\pgfpathlineto{\pgfqpoint{3.153813in}{2.578464in}}%
\pgfpathlineto{\pgfqpoint{3.151939in}{2.572204in}}%
\pgfpathlineto{\pgfqpoint{3.150075in}{2.565943in}}%
\pgfpathlineto{\pgfqpoint{3.148225in}{2.559683in}}%
\pgfpathlineto{\pgfqpoint{3.148164in}{2.559469in}}%
\pgfpathlineto{\pgfqpoint{3.146443in}{2.553422in}}%
\pgfpathlineto{\pgfqpoint{3.144690in}{2.547162in}}%
\pgfpathlineto{\pgfqpoint{3.142971in}{2.540902in}}%
\pgfpathlineto{\pgfqpoint{3.141903in}{2.536892in}}%
\pgfpathlineto{\pgfqpoint{3.141308in}{2.534641in}}%
\pgfpathlineto{\pgfqpoint{3.139721in}{2.528381in}}%
\pgfpathlineto{\pgfqpoint{3.138190in}{2.522120in}}%
\pgfpathlineto{\pgfqpoint{3.136724in}{2.515860in}}%
\pgfpathlineto{\pgfqpoint{3.135643in}{2.510969in}}%
\pgfpathlineto{\pgfqpoint{3.135341in}{2.509599in}}%
\pgfpathlineto{\pgfqpoint{3.134066in}{2.503339in}}%
\pgfpathlineto{\pgfqpoint{3.132883in}{2.497078in}}%
\pgfpathlineto{\pgfqpoint{3.131804in}{2.490818in}}%
\pgfpathlineto{\pgfqpoint{3.130839in}{2.484558in}}%
\pgfpathlineto{\pgfqpoint{3.130000in}{2.478297in}}%
\pgfpathlineto{\pgfqpoint{3.129382in}{2.472779in}}%
\pgfpathlineto{\pgfqpoint{3.129299in}{2.472037in}}%
\pgfpathlineto{\pgfqpoint{3.128757in}{2.465776in}}%
\pgfpathlineto{\pgfqpoint{3.128370in}{2.459516in}}%
\pgfpathlineto{\pgfqpoint{3.128151in}{2.453255in}}%
\pgfpathlineto{\pgfqpoint{3.128110in}{2.446995in}}%
\pgfpathlineto{\pgfqpoint{3.128256in}{2.440735in}}%
\pgfpathlineto{\pgfqpoint{3.128600in}{2.434474in}}%
\pgfpathlineto{\pgfqpoint{3.129148in}{2.428214in}}%
\pgfpathlineto{\pgfqpoint{3.129382in}{2.426263in}}%
\pgfpathlineto{\pgfqpoint{3.129915in}{2.421953in}}%
\pgfpathlineto{\pgfqpoint{3.130903in}{2.415693in}}%
\pgfpathlineto{\pgfqpoint{3.132110in}{2.409432in}}%
\pgfpathlineto{\pgfqpoint{3.133537in}{2.403172in}}%
\pgfpathlineto{\pgfqpoint{3.135180in}{2.396912in}}%
\pgfpathlineto{\pgfqpoint{3.135643in}{2.395340in}}%
\pgfpathlineto{\pgfqpoint{3.137048in}{2.390651in}}%
\pgfpathlineto{\pgfqpoint{3.139122in}{2.384391in}}%
\pgfpathlineto{\pgfqpoint{3.141386in}{2.378130in}}%
\pgfpathlineto{\pgfqpoint{3.141903in}{2.376801in}}%
\pgfpathlineto{\pgfqpoint{3.143840in}{2.371870in}}%
\pgfpathlineto{\pgfqpoint{3.146453in}{2.365609in}}%
\pgfpathlineto{\pgfqpoint{3.148164in}{2.361717in}}%
\pgfpathlineto{\pgfqpoint{3.149208in}{2.359349in}}%
\pgfpathlineto{\pgfqpoint{3.152088in}{2.353088in}}%
\pgfpathlineto{\pgfqpoint{3.154424in}{2.348167in}}%
\pgfpathlineto{\pgfqpoint{3.155059in}{2.346828in}}%
\pgfpathlineto{\pgfqpoint{3.158108in}{2.340568in}}%
\pgfpathlineto{\pgfqpoint{3.160684in}{2.335352in}}%
\pgfpathlineto{\pgfqpoint{3.161198in}{2.334307in}}%
\pgfpathlineto{\pgfqpoint{3.164317in}{2.328047in}}%
\pgfpathlineto{\pgfqpoint{3.166945in}{2.322768in}}%
\pgfpathlineto{\pgfqpoint{3.167430in}{2.321786in}}%
\pgfpathlineto{\pgfqpoint{3.170527in}{2.315526in}}%
\pgfpathlineto{\pgfqpoint{3.173205in}{2.310036in}}%
\pgfpathlineto{\pgfqpoint{3.173578in}{2.309265in}}%
\pgfpathlineto{\pgfqpoint{3.176577in}{2.303005in}}%
\pgfpathlineto{\pgfqpoint{3.179466in}{2.296809in}}%
\pgfpathlineto{\pgfqpoint{3.179495in}{2.296745in}}%
\pgfpathlineto{\pgfqpoint{3.182339in}{2.290484in}}%
\pgfpathlineto{\pgfqpoint{3.185076in}{2.284224in}}%
\pgfpathlineto{\pgfqpoint{3.185726in}{2.282695in}}%
\pgfpathlineto{\pgfqpoint{3.187716in}{2.277963in}}%
\pgfpathlineto{\pgfqpoint{3.190240in}{2.271703in}}%
\pgfpathlineto{\pgfqpoint{3.191987in}{2.267167in}}%
\pgfpathlineto{\pgfqpoint{3.192644in}{2.265442in}}%
\pgfpathlineto{\pgfqpoint{3.194935in}{2.259182in}}%
\pgfpathlineto{\pgfqpoint{3.197094in}{2.252922in}}%
\pgfpathlineto{\pgfqpoint{3.198247in}{2.249383in}}%
\pgfpathlineto{\pgfqpoint{3.199129in}{2.246661in}}%
\pgfpathlineto{\pgfqpoint{3.201042in}{2.240401in}}%
\pgfpathlineto{\pgfqpoint{3.202822in}{2.234140in}}%
\pgfpathlineto{\pgfqpoint{3.204467in}{2.227880in}}%
\pgfpathlineto{\pgfqpoint{3.204508in}{2.227716in}}%
\pgfpathlineto{\pgfqpoint{3.206003in}{2.221619in}}%
\pgfpathlineto{\pgfqpoint{3.207409in}{2.215359in}}%
\pgfpathlineto{\pgfqpoint{3.208685in}{2.209098in}}%
\pgfpathlineto{\pgfqpoint{3.209835in}{2.202838in}}%
\pgfpathlineto{\pgfqpoint{3.210768in}{2.197154in}}%
\pgfpathlineto{\pgfqpoint{3.210863in}{2.196578in}}%
\pgfpathlineto{\pgfqpoint{3.211785in}{2.190317in}}%
\pgfpathlineto{\pgfqpoint{3.212586in}{2.184057in}}%
\pgfpathlineto{\pgfqpoint{3.213269in}{2.177796in}}%
\pgfpathlineto{\pgfqpoint{3.213837in}{2.171536in}}%
\pgfpathlineto{\pgfqpoint{3.214292in}{2.165275in}}%
\pgfpathlineto{\pgfqpoint{3.214636in}{2.159015in}}%
\pgfpathlineto{\pgfqpoint{3.214872in}{2.152755in}}%
\pgfpathlineto{\pgfqpoint{3.215000in}{2.146494in}}%
\pgfpathlineto{\pgfqpoint{3.215024in}{2.140234in}}%
\pgfpathlineto{\pgfqpoint{3.214944in}{2.133973in}}%
\pgfpathlineto{\pgfqpoint{3.214763in}{2.127713in}}%
\pgfpathlineto{\pgfqpoint{3.214480in}{2.121452in}}%
\pgfpathlineto{\pgfqpoint{3.214098in}{2.115192in}}%
\pgfpathlineto{\pgfqpoint{3.213617in}{2.108932in}}%
\pgfpathlineto{\pgfqpoint{3.213038in}{2.102671in}}%
\pgfpathlineto{\pgfqpoint{3.212360in}{2.096411in}}%
\pgfpathlineto{\pgfqpoint{3.211585in}{2.090150in}}%
\pgfpathlineto{\pgfqpoint{3.210768in}{2.084300in}}%
\pgfpathlineto{\pgfqpoint{3.210712in}{2.083890in}}%
\pgfpathlineto{\pgfqpoint{3.209767in}{2.077629in}}%
\pgfpathlineto{\pgfqpoint{3.208725in}{2.071369in}}%
\pgfpathlineto{\pgfqpoint{3.207588in}{2.065109in}}%
\pgfpathlineto{\pgfqpoint{3.206352in}{2.058848in}}%
\pgfpathlineto{\pgfqpoint{3.205018in}{2.052588in}}%
\pgfpathlineto{\pgfqpoint{3.204508in}{2.050352in}}%
\pgfpathlineto{\pgfqpoint{3.203610in}{2.046327in}}%
\pgfpathlineto{\pgfqpoint{3.202119in}{2.040067in}}%
\pgfpathlineto{\pgfqpoint{3.200527in}{2.033806in}}%
\pgfpathlineto{\pgfqpoint{3.198833in}{2.027546in}}%
\pgfpathlineto{\pgfqpoint{3.198247in}{2.025495in}}%
\pgfpathlineto{\pgfqpoint{3.197070in}{2.021285in}}%
\pgfpathlineto{\pgfqpoint{3.195221in}{2.015025in}}%
\pgfpathlineto{\pgfqpoint{3.193266in}{2.008765in}}%
\pgfpathlineto{\pgfqpoint{3.191987in}{2.004869in}}%
\pgfpathlineto{\pgfqpoint{3.191225in}{2.002504in}}%
\pgfpathlineto{\pgfqpoint{3.189113in}{1.996244in}}%
\pgfpathlineto{\pgfqpoint{3.186890in}{1.989983in}}%
\pgfpathlineto{\pgfqpoint{3.185726in}{1.986849in}}%
\pgfpathlineto{\pgfqpoint{3.184586in}{1.983723in}}%
\pgfpathlineto{\pgfqpoint{3.182201in}{1.977462in}}%
\pgfpathlineto{\pgfqpoint{3.179697in}{1.971202in}}%
\pgfpathlineto{\pgfqpoint{3.179466in}{1.970644in}}%
\pgfpathlineto{\pgfqpoint{3.177139in}{1.964942in}}%
\pgfpathlineto{\pgfqpoint{3.174465in}{1.958681in}}%
\pgfpathlineto{\pgfqpoint{3.173205in}{1.955841in}}%
\pgfpathlineto{\pgfqpoint{3.171707in}{1.952421in}}%
\pgfpathlineto{\pgfqpoint{3.168857in}{1.946160in}}%
\pgfpathlineto{\pgfqpoint{3.166945in}{1.942124in}}%
\pgfpathlineto{\pgfqpoint{3.165903in}{1.939900in}}%
\pgfpathlineto{\pgfqpoint{3.162867in}{1.933639in}}%
\pgfpathlineto{\pgfqpoint{3.160684in}{1.929310in}}%
\pgfpathlineto{\pgfqpoint{3.159719in}{1.927379in}}%
\pgfpathlineto{\pgfqpoint{3.156489in}{1.921119in}}%
\pgfpathlineto{\pgfqpoint{3.154424in}{1.917261in}}%
\pgfpathlineto{\pgfqpoint{3.153147in}{1.914858in}}%
\pgfpathlineto{\pgfqpoint{3.149711in}{1.908598in}}%
\pgfpathlineto{\pgfqpoint{3.148164in}{1.905868in}}%
\pgfpathlineto{\pgfqpoint{3.146172in}{1.902337in}}%
\pgfpathlineto{\pgfqpoint{3.142516in}{1.896077in}}%
\pgfpathlineto{\pgfqpoint{3.141903in}{1.895054in}}%
\pgfpathlineto{\pgfqpoint{3.138776in}{1.889816in}}%
\pgfpathlineto{\pgfqpoint{3.135643in}{1.884754in}}%
\pgfpathlineto{\pgfqpoint{3.134903in}{1.883556in}}%
\pgfpathlineto{\pgfqpoint{3.130935in}{1.877295in}}%
\pgfpathlineto{\pgfqpoint{3.129382in}{1.874914in}}%
\pgfpathlineto{\pgfqpoint{3.126851in}{1.871035in}}%
\pgfpathlineto{\pgfqpoint{3.123122in}{1.865498in}}%
\pgfpathlineto{\pgfqpoint{3.122633in}{1.864775in}}%
\pgfpathlineto{\pgfqpoint{3.118315in}{1.858514in}}%
\pgfpathlineto{\pgfqpoint{3.116861in}{1.856462in}}%
\pgfpathlineto{\pgfqpoint{3.113868in}{1.852254in}}%
\pgfpathlineto{\pgfqpoint{3.110601in}{1.847784in}}%
\pgfpathlineto{\pgfqpoint{3.109284in}{1.845993in}}%
\pgfpathlineto{\pgfqpoint{3.104565in}{1.839733in}}%
\pgfpathlineto{\pgfqpoint{3.104341in}{1.839440in}}%
\pgfpathlineto{\pgfqpoint{3.099725in}{1.833472in}}%
\pgfpathlineto{\pgfqpoint{3.098080in}{1.831394in}}%
\pgfpathlineto{\pgfqpoint{3.094737in}{1.827212in}}%
\pgfpathlineto{\pgfqpoint{3.091820in}{1.823643in}}%
\pgfpathlineto{\pgfqpoint{3.089595in}{1.820952in}}%
\pgfpathlineto{\pgfqpoint{3.085559in}{1.816168in}}%
\pgfpathlineto{\pgfqpoint{3.084297in}{1.814691in}}%
\pgfpathlineto{\pgfqpoint{3.079299in}{1.808954in}}%
\pgfpathlineto{\pgfqpoint{3.078836in}{1.808431in}}%
\pgfpathlineto{\pgfqpoint{3.073208in}{1.802170in}}%
\pgfpathlineto{\pgfqpoint{3.073038in}{1.801984in}}%
\pgfpathlineto{\pgfqpoint{3.067404in}{1.795910in}}%
\pgfpathlineto{\pgfqpoint{3.066778in}{1.795245in}}%
\pgfpathlineto{\pgfqpoint{3.061410in}{1.789649in}}%
\pgfpathlineto{\pgfqpoint{3.060518in}{1.788732in}}%
\pgfpathlineto{\pgfqpoint{3.055213in}{1.783389in}}%
\pgfpathlineto{\pgfqpoint{3.054257in}{1.782438in}}%
\pgfpathlineto{\pgfqpoint{3.048802in}{1.777129in}}%
\pgfpathlineto{\pgfqpoint{3.047997in}{1.776353in}}%
\pgfpathlineto{\pgfqpoint{3.042160in}{1.770868in}}%
\pgfpathlineto{\pgfqpoint{3.041736in}{1.770473in}}%
\pgfpathlineto{\pgfqpoint{3.035476in}{1.764788in}}%
\pgfpathlineto{\pgfqpoint{3.035272in}{1.764608in}}%
\pgfpathlineto{\pgfqpoint{3.029215in}{1.759284in}}%
\pgfpathlineto{\pgfqpoint{3.028117in}{1.758347in}}%
\pgfpathlineto{\pgfqpoint{3.022955in}{1.753966in}}%
\pgfpathlineto{\pgfqpoint{3.020669in}{1.752087in}}%
\pgfpathlineto{\pgfqpoint{3.016694in}{1.748832in}}%
\pgfpathlineto{\pgfqpoint{3.012899in}{1.745826in}}%
\pgfpathlineto{\pgfqpoint{3.010434in}{1.743879in}}%
\pgfpathlineto{\pgfqpoint{3.004776in}{1.739566in}}%
\pgfpathlineto{\pgfqpoint{3.004174in}{1.739107in}}%
\pgfpathlineto{\pgfqpoint{2.997913in}{1.734490in}}%
\pgfpathlineto{\pgfqpoint{2.996246in}{1.733306in}}%
\pgfpathlineto{\pgfqpoint{2.991653in}{1.730038in}}%
\pgfpathlineto{\pgfqpoint{2.987264in}{1.727045in}}%
\pgfpathlineto{\pgfqpoint{2.985392in}{1.725764in}}%
\pgfpathlineto{\pgfqpoint{2.979132in}{1.721650in}}%
\pgfpathlineto{\pgfqpoint{2.977760in}{1.720785in}}%
\pgfpathlineto{\pgfqpoint{2.972871in}{1.717684in}}%
\pgfpathlineto{\pgfqpoint{2.967639in}{1.714524in}}%
\pgfpathlineto{\pgfqpoint{2.966611in}{1.713899in}}%
\pgfpathlineto{\pgfqpoint{2.960351in}{1.710251in}}%
\pgfpathlineto{\pgfqpoint{2.956762in}{1.708264in}}%
\pgfpathlineto{\pgfqpoint{2.954090in}{1.706771in}}%
\pgfpathlineto{\pgfqpoint{2.947830in}{1.703446in}}%
\pgfpathlineto{\pgfqpoint{2.944965in}{1.702003in}}%
\pgfpathlineto{\pgfqpoint{2.941569in}{1.700274in}}%
\pgfpathlineto{\pgfqpoint{2.935309in}{1.697261in}}%
\pgfpathlineto{\pgfqpoint{2.931963in}{1.695743in}}%
\pgfpathlineto{\pgfqpoint{2.929048in}{1.694403in}}%
\pgfpathlineto{\pgfqpoint{2.922788in}{1.691695in}}%
\pgfpathlineto{\pgfqpoint{2.917311in}{1.689482in}}%
\pgfpathlineto{\pgfqpoint{2.916528in}{1.689161in}}%
\pgfpathlineto{\pgfqpoint{2.910267in}{1.686756in}}%
\pgfpathlineto{\pgfqpoint{2.904007in}{1.684529in}}%
\pgfpathlineto{\pgfqpoint{2.900025in}{1.683222in}}%
\pgfpathlineto{\pgfqpoint{2.897746in}{1.682460in}}%
\pgfpathlineto{\pgfqpoint{2.891486in}{1.680536in}}%
\pgfpathlineto{\pgfqpoint{2.885225in}{1.678788in}}%
\pgfpathlineto{\pgfqpoint{2.878965in}{1.677212in}}%
\pgfpathlineto{\pgfqpoint{2.877845in}{1.676962in}}%
\pgfpathlineto{\pgfqpoint{2.872705in}{1.675783in}}%
\pgfpathlineto{\pgfqpoint{2.866444in}{1.674525in}}%
\pgfpathlineto{\pgfqpoint{2.860184in}{1.673445in}}%
\pgfpathlineto{\pgfqpoint{2.853923in}{1.672543in}}%
\pgfpathlineto{\pgfqpoint{2.847663in}{1.671821in}}%
\pgfpathlineto{\pgfqpoint{2.841402in}{1.671282in}}%
\pgfpathlineto{\pgfqpoint{2.835142in}{1.670929in}}%
\pgfpathlineto{\pgfqpoint{2.828881in}{1.670768in}}%
\pgfpathlineto{\pgfqpoint{2.822621in}{1.670803in}}%
\pgfpathlineto{\pgfqpoint{2.816361in}{1.671041in}}%
\pgfpathlineto{\pgfqpoint{2.810100in}{1.671490in}}%
\pgfpathlineto{\pgfqpoint{2.803840in}{1.672158in}}%
\pgfpathlineto{\pgfqpoint{2.797579in}{1.673055in}}%
\pgfpathlineto{\pgfqpoint{2.791319in}{1.674193in}}%
\pgfpathlineto{\pgfqpoint{2.785058in}{1.675582in}}%
\pgfpathclose%
\pgfusepath{fill}%
\end{pgfscope}%
\begin{pgfscope}%
\pgfpathrectangle{\pgfqpoint{0.500000in}{0.500000in}}{\pgfqpoint{3.750000in}{3.750000in}}%
\pgfusepath{clip}%
\pgfsetbuttcap%
\pgfsetroundjoin%
\definecolor{currentfill}{rgb}{0.674879,0.752557,0.865083}%
\pgfsetfillcolor{currentfill}%
\pgfsetlinewidth{0.000000pt}%
\definecolor{currentstroke}{rgb}{0.000000,0.000000,0.000000}%
\pgfsetstrokecolor{currentstroke}%
\pgfsetdash{}{0pt}%
\pgfpathmoveto{\pgfqpoint{2.785058in}{1.675582in}}%
\pgfpathlineto{\pgfqpoint{2.791319in}{1.674193in}}%
\pgfpathlineto{\pgfqpoint{2.797579in}{1.673055in}}%
\pgfpathlineto{\pgfqpoint{2.803840in}{1.672158in}}%
\pgfpathlineto{\pgfqpoint{2.810100in}{1.671490in}}%
\pgfpathlineto{\pgfqpoint{2.816361in}{1.671041in}}%
\pgfpathlineto{\pgfqpoint{2.822621in}{1.670803in}}%
\pgfpathlineto{\pgfqpoint{2.828881in}{1.670768in}}%
\pgfpathlineto{\pgfqpoint{2.835142in}{1.670929in}}%
\pgfpathlineto{\pgfqpoint{2.841402in}{1.671282in}}%
\pgfpathlineto{\pgfqpoint{2.847663in}{1.671821in}}%
\pgfpathlineto{\pgfqpoint{2.853923in}{1.672543in}}%
\pgfpathlineto{\pgfqpoint{2.860184in}{1.673445in}}%
\pgfpathlineto{\pgfqpoint{2.866444in}{1.674525in}}%
\pgfpathlineto{\pgfqpoint{2.872705in}{1.675783in}}%
\pgfpathlineto{\pgfqpoint{2.877845in}{1.676962in}}%
\pgfpathlineto{\pgfqpoint{2.878965in}{1.677212in}}%
\pgfpathlineto{\pgfqpoint{2.885225in}{1.678788in}}%
\pgfpathlineto{\pgfqpoint{2.891486in}{1.680536in}}%
\pgfpathlineto{\pgfqpoint{2.897746in}{1.682460in}}%
\pgfpathlineto{\pgfqpoint{2.900025in}{1.683222in}}%
\pgfpathlineto{\pgfqpoint{2.904007in}{1.684529in}}%
\pgfpathlineto{\pgfqpoint{2.910267in}{1.686756in}}%
\pgfpathlineto{\pgfqpoint{2.916528in}{1.689161in}}%
\pgfpathlineto{\pgfqpoint{2.917311in}{1.689482in}}%
\pgfpathlineto{\pgfqpoint{2.922788in}{1.691695in}}%
\pgfpathlineto{\pgfqpoint{2.929048in}{1.694403in}}%
\pgfpathlineto{\pgfqpoint{2.931963in}{1.695743in}}%
\pgfpathlineto{\pgfqpoint{2.935309in}{1.697261in}}%
\pgfpathlineto{\pgfqpoint{2.941569in}{1.700274in}}%
\pgfpathlineto{\pgfqpoint{2.944965in}{1.702003in}}%
\pgfpathlineto{\pgfqpoint{2.947830in}{1.703446in}}%
\pgfpathlineto{\pgfqpoint{2.954090in}{1.706771in}}%
\pgfpathlineto{\pgfqpoint{2.956762in}{1.708264in}}%
\pgfpathlineto{\pgfqpoint{2.960351in}{1.710251in}}%
\pgfpathlineto{\pgfqpoint{2.966611in}{1.713899in}}%
\pgfpathlineto{\pgfqpoint{2.967639in}{1.714524in}}%
\pgfpathlineto{\pgfqpoint{2.972871in}{1.717684in}}%
\pgfpathlineto{\pgfqpoint{2.977760in}{1.720785in}}%
\pgfpathlineto{\pgfqpoint{2.979132in}{1.721650in}}%
\pgfpathlineto{\pgfqpoint{2.985392in}{1.725764in}}%
\pgfpathlineto{\pgfqpoint{2.987264in}{1.727045in}}%
\pgfpathlineto{\pgfqpoint{2.991653in}{1.730038in}}%
\pgfpathlineto{\pgfqpoint{2.996246in}{1.733306in}}%
\pgfpathlineto{\pgfqpoint{2.997913in}{1.734490in}}%
\pgfpathlineto{\pgfqpoint{3.004174in}{1.739107in}}%
\pgfpathlineto{\pgfqpoint{3.004776in}{1.739566in}}%
\pgfpathlineto{\pgfqpoint{3.010434in}{1.743879in}}%
\pgfpathlineto{\pgfqpoint{3.012899in}{1.745826in}}%
\pgfpathlineto{\pgfqpoint{3.016694in}{1.748832in}}%
\pgfpathlineto{\pgfqpoint{3.020669in}{1.752087in}}%
\pgfpathlineto{\pgfqpoint{3.022955in}{1.753966in}}%
\pgfpathlineto{\pgfqpoint{3.028117in}{1.758347in}}%
\pgfpathlineto{\pgfqpoint{3.029215in}{1.759284in}}%
\pgfpathlineto{\pgfqpoint{3.035272in}{1.764608in}}%
\pgfpathlineto{\pgfqpoint{3.035476in}{1.764788in}}%
\pgfpathlineto{\pgfqpoint{3.041736in}{1.770473in}}%
\pgfpathlineto{\pgfqpoint{3.042160in}{1.770868in}}%
\pgfpathlineto{\pgfqpoint{3.047997in}{1.776353in}}%
\pgfpathlineto{\pgfqpoint{3.048802in}{1.777129in}}%
\pgfpathlineto{\pgfqpoint{3.054257in}{1.782438in}}%
\pgfpathlineto{\pgfqpoint{3.055213in}{1.783389in}}%
\pgfpathlineto{\pgfqpoint{3.060518in}{1.788732in}}%
\pgfpathlineto{\pgfqpoint{3.061410in}{1.789649in}}%
\pgfpathlineto{\pgfqpoint{3.066778in}{1.795245in}}%
\pgfpathlineto{\pgfqpoint{3.067404in}{1.795910in}}%
\pgfpathlineto{\pgfqpoint{3.073038in}{1.801984in}}%
\pgfpathlineto{\pgfqpoint{3.073208in}{1.802170in}}%
\pgfpathlineto{\pgfqpoint{3.078836in}{1.808431in}}%
\pgfpathlineto{\pgfqpoint{3.079299in}{1.808954in}}%
\pgfpathlineto{\pgfqpoint{3.084297in}{1.814691in}}%
\pgfpathlineto{\pgfqpoint{3.085559in}{1.816168in}}%
\pgfpathlineto{\pgfqpoint{3.089595in}{1.820952in}}%
\pgfpathlineto{\pgfqpoint{3.091820in}{1.823643in}}%
\pgfpathlineto{\pgfqpoint{3.094737in}{1.827212in}}%
\pgfpathlineto{\pgfqpoint{3.098080in}{1.831394in}}%
\pgfpathlineto{\pgfqpoint{3.099725in}{1.833472in}}%
\pgfpathlineto{\pgfqpoint{3.104341in}{1.839440in}}%
\pgfpathlineto{\pgfqpoint{3.104565in}{1.839733in}}%
\pgfpathlineto{\pgfqpoint{3.109284in}{1.845993in}}%
\pgfpathlineto{\pgfqpoint{3.110601in}{1.847784in}}%
\pgfpathlineto{\pgfqpoint{3.113868in}{1.852254in}}%
\pgfpathlineto{\pgfqpoint{3.116861in}{1.856462in}}%
\pgfpathlineto{\pgfqpoint{3.118315in}{1.858514in}}%
\pgfpathlineto{\pgfqpoint{3.122633in}{1.864775in}}%
\pgfpathlineto{\pgfqpoint{3.123122in}{1.865498in}}%
\pgfpathlineto{\pgfqpoint{3.126851in}{1.871035in}}%
\pgfpathlineto{\pgfqpoint{3.129382in}{1.874914in}}%
\pgfpathlineto{\pgfqpoint{3.130935in}{1.877295in}}%
\pgfpathlineto{\pgfqpoint{3.134903in}{1.883556in}}%
\pgfpathlineto{\pgfqpoint{3.135643in}{1.884754in}}%
\pgfpathlineto{\pgfqpoint{3.138776in}{1.889816in}}%
\pgfpathlineto{\pgfqpoint{3.141903in}{1.895054in}}%
\pgfpathlineto{\pgfqpoint{3.142516in}{1.896077in}}%
\pgfpathlineto{\pgfqpoint{3.146172in}{1.902337in}}%
\pgfpathlineto{\pgfqpoint{3.148164in}{1.905868in}}%
\pgfpathlineto{\pgfqpoint{3.149711in}{1.908598in}}%
\pgfpathlineto{\pgfqpoint{3.153147in}{1.914858in}}%
\pgfpathlineto{\pgfqpoint{3.154424in}{1.917261in}}%
\pgfpathlineto{\pgfqpoint{3.156489in}{1.921119in}}%
\pgfpathlineto{\pgfqpoint{3.159719in}{1.927379in}}%
\pgfpathlineto{\pgfqpoint{3.160684in}{1.929310in}}%
\pgfpathlineto{\pgfqpoint{3.162867in}{1.933639in}}%
\pgfpathlineto{\pgfqpoint{3.165903in}{1.939900in}}%
\pgfpathlineto{\pgfqpoint{3.166945in}{1.942124in}}%
\pgfpathlineto{\pgfqpoint{3.168857in}{1.946160in}}%
\pgfpathlineto{\pgfqpoint{3.171707in}{1.952421in}}%
\pgfpathlineto{\pgfqpoint{3.173205in}{1.955841in}}%
\pgfpathlineto{\pgfqpoint{3.174465in}{1.958681in}}%
\pgfpathlineto{\pgfqpoint{3.177139in}{1.964942in}}%
\pgfpathlineto{\pgfqpoint{3.179466in}{1.970644in}}%
\pgfpathlineto{\pgfqpoint{3.179697in}{1.971202in}}%
\pgfpathlineto{\pgfqpoint{3.182201in}{1.977462in}}%
\pgfpathlineto{\pgfqpoint{3.184586in}{1.983723in}}%
\pgfpathlineto{\pgfqpoint{3.185726in}{1.986849in}}%
\pgfpathlineto{\pgfqpoint{3.186890in}{1.989983in}}%
\pgfpathlineto{\pgfqpoint{3.189113in}{1.996244in}}%
\pgfpathlineto{\pgfqpoint{3.191225in}{2.002504in}}%
\pgfpathlineto{\pgfqpoint{3.191987in}{2.004869in}}%
\pgfpathlineto{\pgfqpoint{3.193266in}{2.008765in}}%
\pgfpathlineto{\pgfqpoint{3.195221in}{2.015025in}}%
\pgfpathlineto{\pgfqpoint{3.197070in}{2.021285in}}%
\pgfpathlineto{\pgfqpoint{3.198247in}{2.025495in}}%
\pgfpathlineto{\pgfqpoint{3.198833in}{2.027546in}}%
\pgfpathlineto{\pgfqpoint{3.200527in}{2.033806in}}%
\pgfpathlineto{\pgfqpoint{3.202119in}{2.040067in}}%
\pgfpathlineto{\pgfqpoint{3.203610in}{2.046327in}}%
\pgfpathlineto{\pgfqpoint{3.204508in}{2.050352in}}%
\pgfpathlineto{\pgfqpoint{3.205018in}{2.052588in}}%
\pgfpathlineto{\pgfqpoint{3.206352in}{2.058848in}}%
\pgfpathlineto{\pgfqpoint{3.207588in}{2.065109in}}%
\pgfpathlineto{\pgfqpoint{3.208725in}{2.071369in}}%
\pgfpathlineto{\pgfqpoint{3.209767in}{2.077629in}}%
\pgfpathlineto{\pgfqpoint{3.210712in}{2.083890in}}%
\pgfpathlineto{\pgfqpoint{3.210768in}{2.084300in}}%
\pgfpathlineto{\pgfqpoint{3.211585in}{2.090150in}}%
\pgfpathlineto{\pgfqpoint{3.212360in}{2.096411in}}%
\pgfpathlineto{\pgfqpoint{3.213038in}{2.102671in}}%
\pgfpathlineto{\pgfqpoint{3.213617in}{2.108932in}}%
\pgfpathlineto{\pgfqpoint{3.214098in}{2.115192in}}%
\pgfpathlineto{\pgfqpoint{3.214480in}{2.121452in}}%
\pgfpathlineto{\pgfqpoint{3.214763in}{2.127713in}}%
\pgfpathlineto{\pgfqpoint{3.214944in}{2.133973in}}%
\pgfpathlineto{\pgfqpoint{3.215024in}{2.140234in}}%
\pgfpathlineto{\pgfqpoint{3.215000in}{2.146494in}}%
\pgfpathlineto{\pgfqpoint{3.214872in}{2.152755in}}%
\pgfpathlineto{\pgfqpoint{3.214636in}{2.159015in}}%
\pgfpathlineto{\pgfqpoint{3.214292in}{2.165275in}}%
\pgfpathlineto{\pgfqpoint{3.213837in}{2.171536in}}%
\pgfpathlineto{\pgfqpoint{3.213269in}{2.177796in}}%
\pgfpathlineto{\pgfqpoint{3.212586in}{2.184057in}}%
\pgfpathlineto{\pgfqpoint{3.211785in}{2.190317in}}%
\pgfpathlineto{\pgfqpoint{3.210863in}{2.196578in}}%
\pgfpathlineto{\pgfqpoint{3.210768in}{2.197154in}}%
\pgfpathlineto{\pgfqpoint{3.209835in}{2.202838in}}%
\pgfpathlineto{\pgfqpoint{3.208685in}{2.209098in}}%
\pgfpathlineto{\pgfqpoint{3.207409in}{2.215359in}}%
\pgfpathlineto{\pgfqpoint{3.206003in}{2.221619in}}%
\pgfpathlineto{\pgfqpoint{3.204508in}{2.227716in}}%
\pgfpathlineto{\pgfqpoint{3.204467in}{2.227880in}}%
\pgfpathlineto{\pgfqpoint{3.202822in}{2.234140in}}%
\pgfpathlineto{\pgfqpoint{3.201042in}{2.240401in}}%
\pgfpathlineto{\pgfqpoint{3.199129in}{2.246661in}}%
\pgfpathlineto{\pgfqpoint{3.198247in}{2.249383in}}%
\pgfpathlineto{\pgfqpoint{3.197094in}{2.252922in}}%
\pgfpathlineto{\pgfqpoint{3.194935in}{2.259182in}}%
\pgfpathlineto{\pgfqpoint{3.192644in}{2.265442in}}%
\pgfpathlineto{\pgfqpoint{3.191987in}{2.267167in}}%
\pgfpathlineto{\pgfqpoint{3.190240in}{2.271703in}}%
\pgfpathlineto{\pgfqpoint{3.187716in}{2.277963in}}%
\pgfpathlineto{\pgfqpoint{3.185726in}{2.282695in}}%
\pgfpathlineto{\pgfqpoint{3.185076in}{2.284224in}}%
\pgfpathlineto{\pgfqpoint{3.182339in}{2.290484in}}%
\pgfpathlineto{\pgfqpoint{3.179495in}{2.296745in}}%
\pgfpathlineto{\pgfqpoint{3.179466in}{2.296809in}}%
\pgfpathlineto{\pgfqpoint{3.176577in}{2.303005in}}%
\pgfpathlineto{\pgfqpoint{3.173578in}{2.309265in}}%
\pgfpathlineto{\pgfqpoint{3.173205in}{2.310036in}}%
\pgfpathlineto{\pgfqpoint{3.170527in}{2.315526in}}%
\pgfpathlineto{\pgfqpoint{3.167430in}{2.321786in}}%
\pgfpathlineto{\pgfqpoint{3.166945in}{2.322768in}}%
\pgfpathlineto{\pgfqpoint{3.164317in}{2.328047in}}%
\pgfpathlineto{\pgfqpoint{3.161198in}{2.334307in}}%
\pgfpathlineto{\pgfqpoint{3.160684in}{2.335352in}}%
\pgfpathlineto{\pgfqpoint{3.158108in}{2.340568in}}%
\pgfpathlineto{\pgfqpoint{3.155059in}{2.346828in}}%
\pgfpathlineto{\pgfqpoint{3.154424in}{2.348167in}}%
\pgfpathlineto{\pgfqpoint{3.152088in}{2.353088in}}%
\pgfpathlineto{\pgfqpoint{3.149208in}{2.359349in}}%
\pgfpathlineto{\pgfqpoint{3.148164in}{2.361717in}}%
\pgfpathlineto{\pgfqpoint{3.146453in}{2.365609in}}%
\pgfpathlineto{\pgfqpoint{3.143840in}{2.371870in}}%
\pgfpathlineto{\pgfqpoint{3.141903in}{2.376801in}}%
\pgfpathlineto{\pgfqpoint{3.141386in}{2.378130in}}%
\pgfpathlineto{\pgfqpoint{3.139122in}{2.384391in}}%
\pgfpathlineto{\pgfqpoint{3.137048in}{2.390651in}}%
\pgfpathlineto{\pgfqpoint{3.135643in}{2.395340in}}%
\pgfpathlineto{\pgfqpoint{3.135180in}{2.396912in}}%
\pgfpathlineto{\pgfqpoint{3.133537in}{2.403172in}}%
\pgfpathlineto{\pgfqpoint{3.132110in}{2.409432in}}%
\pgfpathlineto{\pgfqpoint{3.130903in}{2.415693in}}%
\pgfpathlineto{\pgfqpoint{3.129915in}{2.421953in}}%
\pgfpathlineto{\pgfqpoint{3.129382in}{2.426263in}}%
\pgfpathlineto{\pgfqpoint{3.129148in}{2.428214in}}%
\pgfpathlineto{\pgfqpoint{3.128600in}{2.434474in}}%
\pgfpathlineto{\pgfqpoint{3.128256in}{2.440735in}}%
\pgfpathlineto{\pgfqpoint{3.128110in}{2.446995in}}%
\pgfpathlineto{\pgfqpoint{3.128151in}{2.453255in}}%
\pgfpathlineto{\pgfqpoint{3.128370in}{2.459516in}}%
\pgfpathlineto{\pgfqpoint{3.128757in}{2.465776in}}%
\pgfpathlineto{\pgfqpoint{3.129299in}{2.472037in}}%
\pgfpathlineto{\pgfqpoint{3.129382in}{2.472779in}}%
\pgfpathlineto{\pgfqpoint{3.130000in}{2.478297in}}%
\pgfpathlineto{\pgfqpoint{3.130839in}{2.484558in}}%
\pgfpathlineto{\pgfqpoint{3.131804in}{2.490818in}}%
\pgfpathlineto{\pgfqpoint{3.132883in}{2.497078in}}%
\pgfpathlineto{\pgfqpoint{3.134066in}{2.503339in}}%
\pgfpathlineto{\pgfqpoint{3.135341in}{2.509599in}}%
\pgfpathlineto{\pgfqpoint{3.135643in}{2.510969in}}%
\pgfpathlineto{\pgfqpoint{3.136724in}{2.515860in}}%
\pgfpathlineto{\pgfqpoint{3.138190in}{2.522120in}}%
\pgfpathlineto{\pgfqpoint{3.139721in}{2.528381in}}%
\pgfpathlineto{\pgfqpoint{3.141308in}{2.534641in}}%
\pgfpathlineto{\pgfqpoint{3.141903in}{2.536892in}}%
\pgfpathlineto{\pgfqpoint{3.142971in}{2.540902in}}%
\pgfpathlineto{\pgfqpoint{3.144690in}{2.547162in}}%
\pgfpathlineto{\pgfqpoint{3.146443in}{2.553422in}}%
\pgfpathlineto{\pgfqpoint{3.148164in}{2.559469in}}%
\pgfpathlineto{\pgfqpoint{3.148225in}{2.559683in}}%
\pgfpathlineto{\pgfqpoint{3.150075in}{2.565943in}}%
\pgfpathlineto{\pgfqpoint{3.151939in}{2.572204in}}%
\pgfpathlineto{\pgfqpoint{3.153813in}{2.578464in}}%
\pgfpathlineto{\pgfqpoint{3.154424in}{2.580476in}}%
\pgfpathlineto{\pgfqpoint{3.155727in}{2.584725in}}%
\pgfpathlineto{\pgfqpoint{3.157658in}{2.590985in}}%
\pgfpathlineto{\pgfqpoint{3.159583in}{2.597245in}}%
\pgfpathlineto{\pgfqpoint{3.160684in}{2.600818in}}%
\pgfpathlineto{\pgfqpoint{3.161523in}{2.603506in}}%
\pgfpathlineto{\pgfqpoint{3.163482in}{2.609766in}}%
\pgfpathlineto{\pgfqpoint{3.165425in}{2.616027in}}%
\pgfpathlineto{\pgfqpoint{3.166945in}{2.620962in}}%
\pgfpathlineto{\pgfqpoint{3.167359in}{2.622287in}}%
\pgfpathlineto{\pgfqpoint{3.169316in}{2.628548in}}%
\pgfpathlineto{\pgfqpoint{3.171247in}{2.634808in}}%
\pgfpathlineto{\pgfqpoint{3.173150in}{2.641068in}}%
\pgfpathlineto{\pgfqpoint{3.173205in}{2.641252in}}%
\pgfpathlineto{\pgfqpoint{3.175079in}{2.647329in}}%
\pgfpathlineto{\pgfqpoint{3.176976in}{2.653589in}}%
\pgfpathlineto{\pgfqpoint{3.178838in}{2.659850in}}%
\pgfpathlineto{\pgfqpoint{3.179466in}{2.661981in}}%
\pgfpathlineto{\pgfqpoint{3.180703in}{2.666110in}}%
\pgfpathlineto{\pgfqpoint{3.182549in}{2.672371in}}%
\pgfpathlineto{\pgfqpoint{3.184355in}{2.678631in}}%
\pgfpathlineto{\pgfqpoint{3.185726in}{2.683481in}}%
\pgfpathlineto{\pgfqpoint{3.186133in}{2.684891in}}%
\pgfpathlineto{\pgfqpoint{3.187913in}{2.691152in}}%
\pgfpathlineto{\pgfqpoint{3.189649in}{2.697412in}}%
\pgfpathlineto{\pgfqpoint{3.191342in}{2.703673in}}%
\pgfpathlineto{\pgfqpoint{3.191987in}{2.706106in}}%
\pgfpathlineto{\pgfqpoint{3.193023in}{2.709933in}}%
\pgfpathlineto{\pgfqpoint{3.194679in}{2.716194in}}%
\pgfpathlineto{\pgfqpoint{3.196288in}{2.722454in}}%
\pgfpathlineto{\pgfqpoint{3.197851in}{2.728715in}}%
\pgfpathlineto{\pgfqpoint{3.198247in}{2.730337in}}%
\pgfpathlineto{\pgfqpoint{3.199405in}{2.734975in}}%
\pgfpathlineto{\pgfqpoint{3.200923in}{2.741235in}}%
\pgfpathlineto{\pgfqpoint{3.202392in}{2.747496in}}%
\pgfpathlineto{\pgfqpoint{3.203814in}{2.753756in}}%
\pgfpathlineto{\pgfqpoint{3.204508in}{2.756906in}}%
\pgfpathlineto{\pgfqpoint{3.205210in}{2.760017in}}%
\pgfpathlineto{\pgfqpoint{3.206579in}{2.766277in}}%
\pgfpathlineto{\pgfqpoint{3.207898in}{2.772538in}}%
\pgfpathlineto{\pgfqpoint{3.209168in}{2.778798in}}%
\pgfpathlineto{\pgfqpoint{3.210389in}{2.785058in}}%
\pgfpathlineto{\pgfqpoint{3.210768in}{2.787073in}}%
\pgfpathlineto{\pgfqpoint{3.211587in}{2.791319in}}%
\pgfpathlineto{\pgfqpoint{3.212747in}{2.797579in}}%
\pgfpathlineto{\pgfqpoint{3.213856in}{2.803840in}}%
\pgfpathlineto{\pgfqpoint{3.214914in}{2.810100in}}%
\pgfpathlineto{\pgfqpoint{3.215923in}{2.816361in}}%
\pgfpathlineto{\pgfqpoint{3.216881in}{2.822621in}}%
\pgfpathlineto{\pgfqpoint{3.217028in}{2.823634in}}%
\pgfpathlineto{\pgfqpoint{3.217814in}{2.828881in}}%
\pgfpathlineto{\pgfqpoint{3.218699in}{2.835142in}}%
\pgfpathlineto{\pgfqpoint{3.219532in}{2.841402in}}%
\pgfpathlineto{\pgfqpoint{3.220314in}{2.847663in}}%
\pgfpathlineto{\pgfqpoint{3.221043in}{2.853923in}}%
\pgfpathlineto{\pgfqpoint{3.221720in}{2.860184in}}%
\pgfpathlineto{\pgfqpoint{3.222345in}{2.866444in}}%
\pgfpathlineto{\pgfqpoint{3.222918in}{2.872705in}}%
\pgfpathlineto{\pgfqpoint{3.223289in}{2.877174in}}%
\pgfpathlineto{\pgfqpoint{3.223442in}{2.878965in}}%
\pgfpathlineto{\pgfqpoint{3.223922in}{2.885225in}}%
\pgfpathlineto{\pgfqpoint{3.224346in}{2.891486in}}%
\pgfpathlineto{\pgfqpoint{3.224714in}{2.897746in}}%
\pgfpathlineto{\pgfqpoint{3.225025in}{2.904007in}}%
\pgfpathlineto{\pgfqpoint{3.225278in}{2.910267in}}%
\pgfpathlineto{\pgfqpoint{3.225473in}{2.916528in}}%
\pgfpathlineto{\pgfqpoint{3.225608in}{2.922788in}}%
\pgfpathlineto{\pgfqpoint{3.225683in}{2.929048in}}%
\pgfpathlineto{\pgfqpoint{3.225697in}{2.935309in}}%
\pgfpathlineto{\pgfqpoint{3.225647in}{2.941569in}}%
\pgfpathlineto{\pgfqpoint{3.225533in}{2.947830in}}%
\pgfpathlineto{\pgfqpoint{3.225354in}{2.954090in}}%
\pgfpathlineto{\pgfqpoint{3.225107in}{2.960351in}}%
\pgfpathlineto{\pgfqpoint{3.224791in}{2.966611in}}%
\pgfpathlineto{\pgfqpoint{3.224404in}{2.972871in}}%
\pgfpathlineto{\pgfqpoint{3.223944in}{2.979132in}}%
\pgfpathlineto{\pgfqpoint{3.223409in}{2.985392in}}%
\pgfpathlineto{\pgfqpoint{3.223289in}{2.986626in}}%
\pgfpathlineto{\pgfqpoint{3.222805in}{2.991653in}}%
\pgfpathlineto{\pgfqpoint{3.222124in}{2.997913in}}%
\pgfpathlineto{\pgfqpoint{3.221361in}{3.004174in}}%
\pgfpathlineto{\pgfqpoint{3.220512in}{3.010434in}}%
\pgfpathlineto{\pgfqpoint{3.219575in}{3.016694in}}%
\pgfpathlineto{\pgfqpoint{3.218546in}{3.022955in}}%
\pgfpathlineto{\pgfqpoint{3.217421in}{3.029215in}}%
\pgfpathlineto{\pgfqpoint{3.217028in}{3.031241in}}%
\pgfpathlineto{\pgfqpoint{3.216207in}{3.035476in}}%
\pgfpathlineto{\pgfqpoint{3.214895in}{3.041736in}}%
\pgfpathlineto{\pgfqpoint{3.213475in}{3.047997in}}%
\pgfpathlineto{\pgfqpoint{3.211941in}{3.054257in}}%
\pgfpathlineto{\pgfqpoint{3.210768in}{3.058723in}}%
\pgfpathlineto{\pgfqpoint{3.210292in}{3.060518in}}%
\pgfpathlineto{\pgfqpoint{3.208530in}{3.066778in}}%
\pgfpathlineto{\pgfqpoint{3.206637in}{3.073038in}}%
\pgfpathlineto{\pgfqpoint{3.204606in}{3.079299in}}%
\pgfpathlineto{\pgfqpoint{3.204508in}{3.079587in}}%
\pgfpathlineto{\pgfqpoint{3.202439in}{3.085559in}}%
\pgfpathlineto{\pgfqpoint{3.200118in}{3.091820in}}%
\pgfpathlineto{\pgfqpoint{3.198247in}{3.096562in}}%
\pgfpathlineto{\pgfqpoint{3.197635in}{3.098080in}}%
\pgfpathlineto{\pgfqpoint{3.194979in}{3.104341in}}%
\pgfpathlineto{\pgfqpoint{3.192140in}{3.110601in}}%
\pgfpathlineto{\pgfqpoint{3.191987in}{3.110924in}}%
\pgfpathlineto{\pgfqpoint{3.189093in}{3.116861in}}%
\pgfpathlineto{\pgfqpoint{3.185835in}{3.123122in}}%
\pgfpathlineto{\pgfqpoint{3.185726in}{3.123323in}}%
\pgfpathlineto{\pgfqpoint{3.182325in}{3.129382in}}%
\pgfpathlineto{\pgfqpoint{3.179466in}{3.134182in}}%
\pgfpathlineto{\pgfqpoint{3.178559in}{3.135643in}}%
\pgfpathlineto{\pgfqpoint{3.174491in}{3.141903in}}%
\pgfpathlineto{\pgfqpoint{3.173205in}{3.143792in}}%
\pgfpathlineto{\pgfqpoint{3.170084in}{3.148164in}}%
\pgfpathlineto{\pgfqpoint{3.166945in}{3.152349in}}%
\pgfpathlineto{\pgfqpoint{3.165302in}{3.154424in}}%
\pgfpathlineto{\pgfqpoint{3.160684in}{3.160003in}}%
\pgfpathlineto{\pgfqpoint{3.160085in}{3.160684in}}%
\pgfpathlineto{\pgfqpoint{3.154424in}{3.166864in}}%
\pgfpathlineto{\pgfqpoint{3.154344in}{3.166945in}}%
\pgfpathlineto{\pgfqpoint{3.148164in}{3.173023in}}%
\pgfpathlineto{\pgfqpoint{3.147964in}{3.173205in}}%
\pgfpathlineto{\pgfqpoint{3.141903in}{3.178554in}}%
\pgfpathlineto{\pgfqpoint{3.140780in}{3.179466in}}%
\pgfpathlineto{\pgfqpoint{3.135643in}{3.183515in}}%
\pgfpathlineto{\pgfqpoint{3.132564in}{3.185726in}}%
\pgfpathlineto{\pgfqpoint{3.129382in}{3.187953in}}%
\pgfpathlineto{\pgfqpoint{3.123122in}{3.191905in}}%
\pgfpathlineto{\pgfqpoint{3.122979in}{3.191987in}}%
\pgfpathlineto{\pgfqpoint{3.116861in}{3.195421in}}%
\pgfpathlineto{\pgfqpoint{3.111139in}{3.198247in}}%
\pgfpathlineto{\pgfqpoint{3.110601in}{3.198508in}}%
\pgfpathlineto{\pgfqpoint{3.104341in}{3.201217in}}%
\pgfpathlineto{\pgfqpoint{3.098080in}{3.203547in}}%
\pgfpathlineto{\pgfqpoint{3.095067in}{3.204508in}}%
\pgfpathlineto{\pgfqpoint{3.091820in}{3.205533in}}%
\pgfpathlineto{\pgfqpoint{3.085559in}{3.207191in}}%
\pgfpathlineto{\pgfqpoint{3.079299in}{3.208528in}}%
\pgfpathlineto{\pgfqpoint{3.073038in}{3.209560in}}%
\pgfpathlineto{\pgfqpoint{3.066778in}{3.210301in}}%
\pgfpathlineto{\pgfqpoint{3.060518in}{3.210763in}}%
\pgfpathlineto{\pgfqpoint{3.060354in}{3.210768in}}%
\pgfpathlineto{\pgfqpoint{3.054257in}{3.210959in}}%
\pgfpathlineto{\pgfqpoint{3.047997in}{3.210893in}}%
\pgfpathlineto{\pgfqpoint{3.045589in}{3.210768in}}%
\pgfpathlineto{\pgfqpoint{3.041736in}{3.210574in}}%
\pgfpathlineto{\pgfqpoint{3.035476in}{3.210012in}}%
\pgfpathlineto{\pgfqpoint{3.029215in}{3.209211in}}%
\pgfpathlineto{\pgfqpoint{3.022955in}{3.208174in}}%
\pgfpathlineto{\pgfqpoint{3.016694in}{3.206905in}}%
\pgfpathlineto{\pgfqpoint{3.010434in}{3.205405in}}%
\pgfpathlineto{\pgfqpoint{3.007187in}{3.204508in}}%
\pgfpathlineto{\pgfqpoint{3.004174in}{3.203691in}}%
\pgfpathlineto{\pgfqpoint{2.997913in}{3.201768in}}%
\pgfpathlineto{\pgfqpoint{2.991653in}{3.199620in}}%
\pgfpathlineto{\pgfqpoint{2.988031in}{3.198247in}}%
\pgfpathlineto{\pgfqpoint{2.985392in}{3.197263in}}%
\pgfpathlineto{\pgfqpoint{2.979132in}{3.194707in}}%
\pgfpathlineto{\pgfqpoint{2.973014in}{3.191987in}}%
\pgfpathlineto{\pgfqpoint{2.972871in}{3.191924in}}%
\pgfpathlineto{\pgfqpoint{2.966611in}{3.188966in}}%
\pgfpathlineto{\pgfqpoint{2.960351in}{3.185776in}}%
\pgfpathlineto{\pgfqpoint{2.960259in}{3.185726in}}%
\pgfpathlineto{\pgfqpoint{2.954090in}{3.182415in}}%
\pgfpathlineto{\pgfqpoint{2.948952in}{3.179466in}}%
\pgfpathlineto{\pgfqpoint{2.947830in}{3.178828in}}%
\pgfpathlineto{\pgfqpoint{2.941569in}{3.175058in}}%
\pgfpathlineto{\pgfqpoint{2.938663in}{3.173205in}}%
\pgfpathlineto{\pgfqpoint{2.935309in}{3.171082in}}%
\pgfpathlineto{\pgfqpoint{2.929127in}{3.166945in}}%
\pgfpathlineto{\pgfqpoint{2.929048in}{3.166893in}}%
\pgfpathlineto{\pgfqpoint{2.922788in}{3.162530in}}%
\pgfpathlineto{\pgfqpoint{2.920268in}{3.160684in}}%
\pgfpathlineto{\pgfqpoint{2.916528in}{3.157956in}}%
\pgfpathlineto{\pgfqpoint{2.911908in}{3.154424in}}%
\pgfpathlineto{\pgfqpoint{2.910267in}{3.153174in}}%
\pgfpathlineto{\pgfqpoint{2.904007in}{3.148185in}}%
\pgfpathlineto{\pgfqpoint{2.903981in}{3.148164in}}%
\pgfpathlineto{\pgfqpoint{2.897746in}{3.143011in}}%
\pgfpathlineto{\pgfqpoint{2.896457in}{3.141903in}}%
\pgfpathlineto{\pgfqpoint{2.891486in}{3.137627in}}%
\pgfpathlineto{\pgfqpoint{2.889262in}{3.135643in}}%
\pgfpathlineto{\pgfqpoint{2.885225in}{3.132034in}}%
\pgfpathlineto{\pgfqpoint{2.882361in}{3.129382in}}%
\pgfpathlineto{\pgfqpoint{2.878965in}{3.126230in}}%
\pgfpathlineto{\pgfqpoint{2.875725in}{3.123122in}}%
\pgfpathlineto{\pgfqpoint{2.872705in}{3.120213in}}%
\pgfpathlineto{\pgfqpoint{2.869329in}{3.116861in}}%
\pgfpathlineto{\pgfqpoint{2.866444in}{3.113981in}}%
\pgfpathlineto{\pgfqpoint{2.863154in}{3.110601in}}%
\pgfpathlineto{\pgfqpoint{2.860184in}{3.107530in}}%
\pgfpathlineto{\pgfqpoint{2.857179in}{3.104341in}}%
\pgfpathlineto{\pgfqpoint{2.853923in}{3.100857in}}%
\pgfpathlineto{\pgfqpoint{2.851391in}{3.098080in}}%
\pgfpathlineto{\pgfqpoint{2.847663in}{3.093956in}}%
\pgfpathlineto{\pgfqpoint{2.845775in}{3.091820in}}%
\pgfpathlineto{\pgfqpoint{2.841402in}{3.086821in}}%
\pgfpathlineto{\pgfqpoint{2.840321in}{3.085559in}}%
\pgfpathlineto{\pgfqpoint{2.835142in}{3.079446in}}%
\pgfpathlineto{\pgfqpoint{2.835019in}{3.079299in}}%
\pgfpathlineto{\pgfqpoint{2.829855in}{3.073038in}}%
\pgfpathlineto{\pgfqpoint{2.828881in}{3.071843in}}%
\pgfpathlineto{\pgfqpoint{2.824824in}{3.066778in}}%
\pgfpathlineto{\pgfqpoint{2.822621in}{3.063989in}}%
\pgfpathlineto{\pgfqpoint{2.819920in}{3.060518in}}%
\pgfpathlineto{\pgfqpoint{2.816361in}{3.055873in}}%
\pgfpathlineto{\pgfqpoint{2.815139in}{3.054257in}}%
\pgfpathlineto{\pgfqpoint{2.810473in}{3.047997in}}%
\pgfpathlineto{\pgfqpoint{2.810100in}{3.047490in}}%
\pgfpathlineto{\pgfqpoint{2.805906in}{3.041736in}}%
\pgfpathlineto{\pgfqpoint{2.803840in}{3.038851in}}%
\pgfpathlineto{\pgfqpoint{2.801447in}{3.035476in}}%
\pgfpathlineto{\pgfqpoint{2.797579in}{3.029917in}}%
\pgfpathlineto{\pgfqpoint{2.797095in}{3.029215in}}%
\pgfpathlineto{\pgfqpoint{2.792826in}{3.022955in}}%
\pgfpathlineto{\pgfqpoint{2.791319in}{3.020705in}}%
\pgfpathlineto{\pgfqpoint{2.788649in}{3.016694in}}%
\pgfpathlineto{\pgfqpoint{2.785058in}{3.011184in}}%
\pgfpathlineto{\pgfqpoint{2.784572in}{3.010434in}}%
\pgfpathlineto{\pgfqpoint{2.780563in}{3.004174in}}%
\pgfpathlineto{\pgfqpoint{2.778798in}{3.001362in}}%
\pgfpathlineto{\pgfqpoint{2.776639in}{2.997913in}}%
\pgfpathlineto{\pgfqpoint{2.772808in}{2.991653in}}%
\pgfpathlineto{\pgfqpoint{2.772538in}{2.991206in}}%
\pgfpathlineto{\pgfqpoint{2.769026in}{2.985392in}}%
\pgfpathlineto{\pgfqpoint{2.766277in}{2.980726in}}%
\pgfpathlineto{\pgfqpoint{2.765337in}{2.979132in}}%
\pgfpathlineto{\pgfqpoint{2.761707in}{2.972871in}}%
\pgfpathlineto{\pgfqpoint{2.760017in}{2.969896in}}%
\pgfpathlineto{\pgfqpoint{2.758147in}{2.966611in}}%
\pgfpathlineto{\pgfqpoint{2.754660in}{2.960351in}}%
\pgfpathlineto{\pgfqpoint{2.753756in}{2.958701in}}%
\pgfpathlineto{\pgfqpoint{2.751220in}{2.954090in}}%
\pgfpathlineto{\pgfqpoint{2.747865in}{2.947830in}}%
\pgfpathlineto{\pgfqpoint{2.747496in}{2.947131in}}%
\pgfpathlineto{\pgfqpoint{2.744538in}{2.941569in}}%
\pgfpathlineto{\pgfqpoint{2.741304in}{2.935309in}}%
\pgfpathlineto{\pgfqpoint{2.741235in}{2.935175in}}%
\pgfpathlineto{\pgfqpoint{2.738084in}{2.929048in}}%
\pgfpathlineto{\pgfqpoint{2.734975in}{2.922819in}}%
\pgfpathlineto{\pgfqpoint{2.734959in}{2.922788in}}%
\pgfpathlineto{\pgfqpoint{2.731842in}{2.916528in}}%
\pgfpathlineto{\pgfqpoint{2.728816in}{2.910267in}}%
\pgfpathlineto{\pgfqpoint{2.728715in}{2.910055in}}%
\pgfpathlineto{\pgfqpoint{2.725795in}{2.904007in}}%
\pgfpathlineto{\pgfqpoint{2.722858in}{2.897746in}}%
\pgfpathlineto{\pgfqpoint{2.722454in}{2.896872in}}%
\pgfpathlineto{\pgfqpoint{2.719930in}{2.891486in}}%
\pgfpathlineto{\pgfqpoint{2.717072in}{2.885225in}}%
\pgfpathlineto{\pgfqpoint{2.716194in}{2.883267in}}%
\pgfpathlineto{\pgfqpoint{2.714233in}{2.878965in}}%
\pgfpathlineto{\pgfqpoint{2.711444in}{2.872705in}}%
\pgfpathlineto{\pgfqpoint{2.709933in}{2.869243in}}%
\pgfpathlineto{\pgfqpoint{2.708690in}{2.866444in}}%
\pgfpathlineto{\pgfqpoint{2.705960in}{2.860184in}}%
\pgfpathlineto{\pgfqpoint{2.703673in}{2.854810in}}%
\pgfpathlineto{\pgfqpoint{2.703288in}{2.853923in}}%
\pgfpathlineto{\pgfqpoint{2.700606in}{2.847663in}}%
\pgfpathlineto{\pgfqpoint{2.697992in}{2.841402in}}%
\pgfpathlineto{\pgfqpoint{2.697412in}{2.839993in}}%
\pgfpathlineto{\pgfqpoint{2.695370in}{2.835142in}}%
\pgfpathlineto{\pgfqpoint{2.692791in}{2.828881in}}%
\pgfpathlineto{\pgfqpoint{2.691152in}{2.824828in}}%
\pgfpathlineto{\pgfqpoint{2.690237in}{2.822621in}}%
\pgfpathlineto{\pgfqpoint{2.687681in}{2.816361in}}%
\pgfpathlineto{\pgfqpoint{2.685180in}{2.810100in}}%
\pgfpathlineto{\pgfqpoint{2.684891in}{2.809367in}}%
\pgfpathlineto{\pgfqpoint{2.682648in}{2.803840in}}%
\pgfpathlineto{\pgfqpoint{2.680156in}{2.797579in}}%
\pgfpathlineto{\pgfqpoint{2.678631in}{2.793685in}}%
\pgfpathlineto{\pgfqpoint{2.677675in}{2.791319in}}%
\pgfpathlineto{\pgfqpoint{2.675179in}{2.785058in}}%
\pgfpathlineto{\pgfqpoint{2.672728in}{2.778798in}}%
\pgfpathlineto{\pgfqpoint{2.672371in}{2.777875in}}%
\pgfpathlineto{\pgfqpoint{2.670230in}{2.772538in}}%
\pgfpathlineto{\pgfqpoint{2.667758in}{2.766277in}}%
\pgfpathlineto{\pgfqpoint{2.666110in}{2.762047in}}%
\pgfpathlineto{\pgfqpoint{2.665288in}{2.760017in}}%
\pgfpathlineto{\pgfqpoint{2.662780in}{2.753756in}}%
\pgfpathlineto{\pgfqpoint{2.660305in}{2.747496in}}%
\pgfpathlineto{\pgfqpoint{2.659850in}{2.746332in}}%
\pgfpathlineto{\pgfqpoint{2.657767in}{2.741235in}}%
\pgfpathlineto{\pgfqpoint{2.655235in}{2.734975in}}%
\pgfpathlineto{\pgfqpoint{2.653589in}{2.730865in}}%
\pgfpathlineto{\pgfqpoint{2.652687in}{2.728715in}}%
\pgfpathlineto{\pgfqpoint{2.650080in}{2.722454in}}%
\pgfpathlineto{\pgfqpoint{2.647495in}{2.716194in}}%
\pgfpathlineto{\pgfqpoint{2.647329in}{2.715789in}}%
\pgfpathlineto{\pgfqpoint{2.644800in}{2.709933in}}%
\pgfpathlineto{\pgfqpoint{2.642112in}{2.703673in}}%
\pgfpathlineto{\pgfqpoint{2.641068in}{2.701229in}}%
\pgfpathlineto{\pgfqpoint{2.639345in}{2.697412in}}%
\pgfpathlineto{\pgfqpoint{2.636530in}{2.691152in}}%
\pgfpathlineto{\pgfqpoint{2.634808in}{2.687308in}}%
\pgfpathlineto{\pgfqpoint{2.633657in}{2.684891in}}%
\pgfpathlineto{\pgfqpoint{2.630685in}{2.678631in}}%
\pgfpathlineto{\pgfqpoint{2.628548in}{2.674123in}}%
\pgfpathlineto{\pgfqpoint{2.627660in}{2.672371in}}%
\pgfpathlineto{\pgfqpoint{2.624495in}{2.666110in}}%
\pgfpathlineto{\pgfqpoint{2.622287in}{2.661744in}}%
\pgfpathlineto{\pgfqpoint{2.621257in}{2.659850in}}%
\pgfpathlineto{\pgfqpoint{2.617857in}{2.653589in}}%
\pgfpathlineto{\pgfqpoint{2.616027in}{2.650223in}}%
\pgfpathlineto{\pgfqpoint{2.614324in}{2.647329in}}%
\pgfpathlineto{\pgfqpoint{2.610636in}{2.641068in}}%
\pgfpathlineto{\pgfqpoint{2.609766in}{2.639592in}}%
\pgfpathlineto{\pgfqpoint{2.606693in}{2.634808in}}%
\pgfpathlineto{\pgfqpoint{2.603506in}{2.629859in}}%
\pgfpathlineto{\pgfqpoint{2.602578in}{2.628548in}}%
\pgfpathlineto{\pgfqpoint{2.598143in}{2.622287in}}%
\pgfpathlineto{\pgfqpoint{2.597245in}{2.621021in}}%
\pgfpathlineto{\pgfqpoint{2.593317in}{2.616027in}}%
\pgfpathlineto{\pgfqpoint{2.590985in}{2.613065in}}%
\pgfpathlineto{\pgfqpoint{2.588073in}{2.609766in}}%
\pgfpathlineto{\pgfqpoint{2.584725in}{2.605975in}}%
\pgfpathlineto{\pgfqpoint{2.582250in}{2.603506in}}%
\pgfpathlineto{\pgfqpoint{2.578464in}{2.599725in}}%
\pgfpathlineto{\pgfqpoint{2.575605in}{2.597245in}}%
\pgfpathlineto{\pgfqpoint{2.572204in}{2.594288in}}%
\pgfpathlineto{\pgfqpoint{2.567748in}{2.590985in}}%
\pgfpathlineto{\pgfqpoint{2.565943in}{2.589641in}}%
\pgfpathlineto{\pgfqpoint{2.559683in}{2.585752in}}%
\pgfpathlineto{\pgfqpoint{2.557636in}{2.584725in}}%
\pgfpathlineto{\pgfqpoint{2.553422in}{2.582591in}}%
\pgfpathlineto{\pgfqpoint{2.547162in}{2.580142in}}%
\pgfpathlineto{\pgfqpoint{2.541181in}{2.578464in}}%
\pgfpathlineto{\pgfqpoint{2.540902in}{2.578385in}}%
\pgfpathlineto{\pgfqpoint{2.534641in}{2.577271in}}%
\pgfpathlineto{\pgfqpoint{2.528381in}{2.576800in}}%
\pgfpathlineto{\pgfqpoint{2.522120in}{2.576947in}}%
\pgfpathlineto{\pgfqpoint{2.515860in}{2.577693in}}%
\pgfpathlineto{\pgfqpoint{2.512215in}{2.578464in}}%
\pgfpathlineto{\pgfqpoint{2.509599in}{2.579009in}}%
\pgfpathlineto{\pgfqpoint{2.503339in}{2.580866in}}%
\pgfpathlineto{\pgfqpoint{2.497078in}{2.583264in}}%
\pgfpathlineto{\pgfqpoint{2.493951in}{2.584725in}}%
\pgfpathlineto{\pgfqpoint{2.490818in}{2.586167in}}%
\pgfpathlineto{\pgfqpoint{2.484558in}{2.589551in}}%
\pgfpathlineto{\pgfqpoint{2.482240in}{2.590985in}}%
\pgfpathlineto{\pgfqpoint{2.478297in}{2.593390in}}%
\pgfpathlineto{\pgfqpoint{2.472670in}{2.597245in}}%
\pgfpathlineto{\pgfqpoint{2.472037in}{2.597674in}}%
\pgfpathlineto{\pgfqpoint{2.465776in}{2.602344in}}%
\pgfpathlineto{\pgfqpoint{2.464349in}{2.603506in}}%
\pgfpathlineto{\pgfqpoint{2.459516in}{2.607388in}}%
\pgfpathlineto{\pgfqpoint{2.456776in}{2.609766in}}%
\pgfpathlineto{\pgfqpoint{2.453255in}{2.612780in}}%
\pgfpathlineto{\pgfqpoint{2.449713in}{2.616027in}}%
\pgfpathlineto{\pgfqpoint{2.446995in}{2.618484in}}%
\pgfpathlineto{\pgfqpoint{2.443037in}{2.622287in}}%
\pgfpathlineto{\pgfqpoint{2.440735in}{2.624468in}}%
\pgfpathlineto{\pgfqpoint{2.436656in}{2.628548in}}%
\pgfpathlineto{\pgfqpoint{2.434474in}{2.630699in}}%
\pgfpathlineto{\pgfqpoint{2.430504in}{2.634808in}}%
\pgfpathlineto{\pgfqpoint{2.428214in}{2.637144in}}%
\pgfpathlineto{\pgfqpoint{2.424530in}{2.641068in}}%
\pgfpathlineto{\pgfqpoint{2.421953in}{2.643773in}}%
\pgfpathlineto{\pgfqpoint{2.418695in}{2.647329in}}%
\pgfpathlineto{\pgfqpoint{2.415693in}{2.650556in}}%
\pgfpathlineto{\pgfqpoint{2.412968in}{2.653589in}}%
\pgfpathlineto{\pgfqpoint{2.409432in}{2.657465in}}%
\pgfpathlineto{\pgfqpoint{2.407324in}{2.659850in}}%
\pgfpathlineto{\pgfqpoint{2.403172in}{2.664472in}}%
\pgfpathlineto{\pgfqpoint{2.401742in}{2.666110in}}%
\pgfpathlineto{\pgfqpoint{2.396912in}{2.671554in}}%
\pgfpathlineto{\pgfqpoint{2.396206in}{2.672371in}}%
\pgfpathlineto{\pgfqpoint{2.390701in}{2.678631in}}%
\pgfpathlineto{\pgfqpoint{2.390651in}{2.678687in}}%
\pgfpathlineto{\pgfqpoint{2.385231in}{2.684891in}}%
\pgfpathlineto{\pgfqpoint{2.384391in}{2.685838in}}%
\pgfpathlineto{\pgfqpoint{2.379766in}{2.691152in}}%
\pgfpathlineto{\pgfqpoint{2.378130in}{2.693002in}}%
\pgfpathlineto{\pgfqpoint{2.374298in}{2.697412in}}%
\pgfpathlineto{\pgfqpoint{2.371870in}{2.700161in}}%
\pgfpathlineto{\pgfqpoint{2.368817in}{2.703673in}}%
\pgfpathlineto{\pgfqpoint{2.365609in}{2.707302in}}%
\pgfpathlineto{\pgfqpoint{2.363317in}{2.709933in}}%
\pgfpathlineto{\pgfqpoint{2.359349in}{2.714412in}}%
\pgfpathlineto{\pgfqpoint{2.357791in}{2.716194in}}%
\pgfpathlineto{\pgfqpoint{2.353088in}{2.721481in}}%
\pgfpathlineto{\pgfqpoint{2.352232in}{2.722454in}}%
\pgfpathlineto{\pgfqpoint{2.346828in}{2.728498in}}%
\pgfpathlineto{\pgfqpoint{2.346636in}{2.728715in}}%
\pgfpathlineto{\pgfqpoint{2.341003in}{2.734975in}}%
\pgfpathlineto{\pgfqpoint{2.340568in}{2.735452in}}%
\pgfpathlineto{\pgfqpoint{2.335325in}{2.741235in}}%
\pgfpathlineto{\pgfqpoint{2.334307in}{2.742340in}}%
\pgfpathlineto{\pgfqpoint{2.329592in}{2.747496in}}%
\pgfpathlineto{\pgfqpoint{2.328047in}{2.749160in}}%
\pgfpathlineto{\pgfqpoint{2.323803in}{2.753756in}}%
\pgfpathlineto{\pgfqpoint{2.321786in}{2.755908in}}%
\pgfpathlineto{\pgfqpoint{2.317952in}{2.760017in}}%
\pgfpathlineto{\pgfqpoint{2.315526in}{2.762579in}}%
\pgfpathlineto{\pgfqpoint{2.312037in}{2.766277in}}%
\pgfpathlineto{\pgfqpoint{2.309265in}{2.769173in}}%
\pgfpathlineto{\pgfqpoint{2.306054in}{2.772538in}}%
\pgfpathlineto{\pgfqpoint{2.303005in}{2.775688in}}%
\pgfpathlineto{\pgfqpoint{2.300001in}{2.778798in}}%
\pgfpathlineto{\pgfqpoint{2.296745in}{2.782122in}}%
\pgfpathlineto{\pgfqpoint{2.293872in}{2.785058in}}%
\pgfpathlineto{\pgfqpoint{2.290484in}{2.788476in}}%
\pgfpathlineto{\pgfqpoint{2.287667in}{2.791319in}}%
\pgfpathlineto{\pgfqpoint{2.284224in}{2.794750in}}%
\pgfpathlineto{\pgfqpoint{2.281383in}{2.797579in}}%
\pgfpathlineto{\pgfqpoint{2.277963in}{2.800943in}}%
\pgfpathlineto{\pgfqpoint{2.275015in}{2.803840in}}%
\pgfpathlineto{\pgfqpoint{2.271703in}{2.807056in}}%
\pgfpathlineto{\pgfqpoint{2.268562in}{2.810100in}}%
\pgfpathlineto{\pgfqpoint{2.265442in}{2.813090in}}%
\pgfpathlineto{\pgfqpoint{2.262022in}{2.816361in}}%
\pgfpathlineto{\pgfqpoint{2.259182in}{2.819047in}}%
\pgfpathlineto{\pgfqpoint{2.255391in}{2.822621in}}%
\pgfpathlineto{\pgfqpoint{2.252922in}{2.824926in}}%
\pgfpathlineto{\pgfqpoint{2.248668in}{2.828881in}}%
\pgfpathlineto{\pgfqpoint{2.246661in}{2.830729in}}%
\pgfpathlineto{\pgfqpoint{2.241849in}{2.835142in}}%
\pgfpathlineto{\pgfqpoint{2.240401in}{2.836457in}}%
\pgfpathlineto{\pgfqpoint{2.234932in}{2.841402in}}%
\pgfpathlineto{\pgfqpoint{2.234140in}{2.842112in}}%
\pgfpathlineto{\pgfqpoint{2.227914in}{2.847663in}}%
\pgfpathlineto{\pgfqpoint{2.227880in}{2.847693in}}%
\pgfpathlineto{\pgfqpoint{2.221619in}{2.853205in}}%
\pgfpathlineto{\pgfqpoint{2.220799in}{2.853923in}}%
\pgfpathlineto{\pgfqpoint{2.215359in}{2.858648in}}%
\pgfpathlineto{\pgfqpoint{2.213580in}{2.860184in}}%
\pgfpathlineto{\pgfqpoint{2.209098in}{2.864022in}}%
\pgfpathlineto{\pgfqpoint{2.206252in}{2.866444in}}%
\pgfpathlineto{\pgfqpoint{2.202838in}{2.869329in}}%
\pgfpathlineto{\pgfqpoint{2.198814in}{2.872705in}}%
\pgfpathlineto{\pgfqpoint{2.196578in}{2.874568in}}%
\pgfpathlineto{\pgfqpoint{2.191261in}{2.878965in}}%
\pgfpathlineto{\pgfqpoint{2.190317in}{2.879741in}}%
\pgfpathlineto{\pgfqpoint{2.184057in}{2.884851in}}%
\pgfpathlineto{\pgfqpoint{2.183595in}{2.885225in}}%
\pgfpathlineto{\pgfqpoint{2.177796in}{2.889901in}}%
\pgfpathlineto{\pgfqpoint{2.175814in}{2.891486in}}%
\pgfpathlineto{\pgfqpoint{2.171536in}{2.894889in}}%
\pgfpathlineto{\pgfqpoint{2.167910in}{2.897746in}}%
\pgfpathlineto{\pgfqpoint{2.165275in}{2.899813in}}%
\pgfpathlineto{\pgfqpoint{2.159879in}{2.904007in}}%
\pgfpathlineto{\pgfqpoint{2.159015in}{2.904675in}}%
\pgfpathlineto{\pgfqpoint{2.152755in}{2.909481in}}%
\pgfpathlineto{\pgfqpoint{2.151722in}{2.910267in}}%
\pgfpathlineto{\pgfqpoint{2.146494in}{2.914232in}}%
\pgfpathlineto{\pgfqpoint{2.143435in}{2.916528in}}%
\pgfpathlineto{\pgfqpoint{2.140234in}{2.918923in}}%
\pgfpathlineto{\pgfqpoint{2.135009in}{2.922788in}}%
\pgfpathlineto{\pgfqpoint{2.133973in}{2.923553in}}%
\pgfpathlineto{\pgfqpoint{2.127713in}{2.928131in}}%
\pgfpathlineto{\pgfqpoint{2.126445in}{2.929048in}}%
\pgfpathlineto{\pgfqpoint{2.121452in}{2.932656in}}%
\pgfpathlineto{\pgfqpoint{2.117736in}{2.935309in}}%
\pgfpathlineto{\pgfqpoint{2.115192in}{2.937123in}}%
\pgfpathlineto{\pgfqpoint{2.108932in}{2.941530in}}%
\pgfpathlineto{\pgfqpoint{2.108875in}{2.941569in}}%
\pgfpathlineto{\pgfqpoint{2.102671in}{2.945896in}}%
\pgfpathlineto{\pgfqpoint{2.099860in}{2.947830in}}%
\pgfpathlineto{\pgfqpoint{2.096411in}{2.950203in}}%
\pgfpathlineto{\pgfqpoint{2.090683in}{2.954090in}}%
\pgfpathlineto{\pgfqpoint{2.090150in}{2.954452in}}%
\pgfpathlineto{\pgfqpoint{2.083890in}{2.958658in}}%
\pgfpathlineto{\pgfqpoint{2.081336in}{2.960351in}}%
\pgfpathlineto{\pgfqpoint{2.077629in}{2.962810in}}%
\pgfpathlineto{\pgfqpoint{2.071814in}{2.966611in}}%
\pgfpathlineto{\pgfqpoint{2.071369in}{2.966902in}}%
\pgfpathlineto{\pgfqpoint{2.065109in}{2.970955in}}%
\pgfpathlineto{\pgfqpoint{2.062105in}{2.972871in}}%
\pgfpathlineto{\pgfqpoint{2.058848in}{2.974953in}}%
\pgfpathlineto{\pgfqpoint{2.052588in}{2.978893in}}%
\pgfpathlineto{\pgfqpoint{2.052204in}{2.979132in}}%
\pgfpathlineto{\pgfqpoint{2.046327in}{2.982797in}}%
\pgfpathlineto{\pgfqpoint{2.042096in}{2.985392in}}%
\pgfpathlineto{\pgfqpoint{2.040067in}{2.986641in}}%
\pgfpathlineto{\pgfqpoint{2.033806in}{2.990438in}}%
\pgfpathlineto{\pgfqpoint{2.031773in}{2.991653in}}%
\pgfpathlineto{\pgfqpoint{2.027546in}{2.994188in}}%
\pgfpathlineto{\pgfqpoint{2.021285in}{2.997877in}}%
\pgfpathlineto{\pgfqpoint{2.021224in}{2.997913in}}%
\pgfpathlineto{\pgfqpoint{2.015025in}{3.001534in}}%
\pgfpathlineto{\pgfqpoint{2.010424in}{3.004174in}}%
\pgfpathlineto{\pgfqpoint{2.008765in}{3.005130in}}%
\pgfpathlineto{\pgfqpoint{2.002504in}{3.008682in}}%
\pgfpathlineto{\pgfqpoint{1.999364in}{3.010434in}}%
\pgfpathlineto{\pgfqpoint{1.996244in}{3.012183in}}%
\pgfpathlineto{\pgfqpoint{1.989983in}{3.015633in}}%
\pgfpathlineto{\pgfqpoint{1.988024in}{3.016694in}}%
\pgfpathlineto{\pgfqpoint{1.983723in}{3.019038in}}%
\pgfpathlineto{\pgfqpoint{1.977462in}{3.022386in}}%
\pgfpathlineto{\pgfqpoint{1.976381in}{3.022955in}}%
\pgfpathlineto{\pgfqpoint{1.971202in}{3.025696in}}%
\pgfpathlineto{\pgfqpoint{1.964942in}{3.028943in}}%
\pgfpathlineto{\pgfqpoint{1.964407in}{3.029215in}}%
\pgfpathlineto{\pgfqpoint{1.958681in}{3.032155in}}%
\pgfpathlineto{\pgfqpoint{1.952421in}{3.035302in}}%
\pgfpathlineto{\pgfqpoint{1.952069in}{3.035476in}}%
\pgfpathlineto{\pgfqpoint{1.946160in}{3.038415in}}%
\pgfpathlineto{\pgfqpoint{1.939900in}{3.041462in}}%
\pgfpathlineto{\pgfqpoint{1.939326in}{3.041736in}}%
\pgfpathlineto{\pgfqpoint{1.933639in}{3.044473in}}%
\pgfpathlineto{\pgfqpoint{1.927379in}{3.047421in}}%
\pgfpathlineto{\pgfqpoint{1.926131in}{3.047997in}}%
\pgfpathlineto{\pgfqpoint{1.921119in}{3.050328in}}%
\pgfpathlineto{\pgfqpoint{1.914858in}{3.053176in}}%
\pgfpathlineto{\pgfqpoint{1.912428in}{3.054257in}}%
\pgfpathlineto{\pgfqpoint{1.908598in}{3.055976in}}%
\pgfpathlineto{\pgfqpoint{1.902337in}{3.058723in}}%
\pgfpathlineto{\pgfqpoint{1.898150in}{3.060518in}}%
\pgfpathlineto{\pgfqpoint{1.896077in}{3.061414in}}%
\pgfpathlineto{\pgfqpoint{1.889816in}{3.064058in}}%
\pgfpathlineto{\pgfqpoint{1.883556in}{3.066638in}}%
\pgfpathlineto{\pgfqpoint{1.883208in}{3.066778in}}%
\pgfpathlineto{\pgfqpoint{1.877295in}{3.069176in}}%
\pgfpathlineto{\pgfqpoint{1.871035in}{3.071651in}}%
\pgfpathlineto{\pgfqpoint{1.867432in}{3.073038in}}%
\pgfpathlineto{\pgfqpoint{1.864775in}{3.074072in}}%
\pgfpathlineto{\pgfqpoint{1.858514in}{3.076439in}}%
\pgfpathlineto{\pgfqpoint{1.852254in}{3.078742in}}%
\pgfpathlineto{\pgfqpoint{1.850695in}{3.079299in}}%
\pgfpathlineto{\pgfqpoint{1.845993in}{3.080995in}}%
\pgfpathlineto{\pgfqpoint{1.839733in}{3.083185in}}%
\pgfpathlineto{\pgfqpoint{1.833472in}{3.085311in}}%
\pgfpathlineto{\pgfqpoint{1.832716in}{3.085559in}}%
\pgfpathlineto{\pgfqpoint{1.827212in}{3.087384in}}%
\pgfpathlineto{\pgfqpoint{1.820952in}{3.089392in}}%
\pgfpathlineto{\pgfqpoint{1.814691in}{3.091332in}}%
\pgfpathlineto{\pgfqpoint{1.813055in}{3.091820in}}%
\pgfpathlineto{\pgfqpoint{1.808431in}{3.093213in}}%
\pgfpathlineto{\pgfqpoint{1.802170in}{3.095028in}}%
\pgfpathlineto{\pgfqpoint{1.795910in}{3.096772in}}%
\pgfpathlineto{\pgfqpoint{1.791011in}{3.098080in}}%
\pgfpathlineto{\pgfqpoint{1.789649in}{3.098448in}}%
\pgfpathlineto{\pgfqpoint{1.783389in}{3.100056in}}%
\pgfpathlineto{\pgfqpoint{1.777129in}{3.101591in}}%
\pgfpathlineto{\pgfqpoint{1.770868in}{3.103050in}}%
\pgfpathlineto{\pgfqpoint{1.765019in}{3.104341in}}%
\pgfpathlineto{\pgfqpoint{1.764608in}{3.104432in}}%
\pgfpathlineto{\pgfqpoint{1.758347in}{3.105739in}}%
\pgfpathlineto{\pgfqpoint{1.752087in}{3.106963in}}%
\pgfpathlineto{\pgfqpoint{1.745826in}{3.108104in}}%
\pgfpathlineto{\pgfqpoint{1.739566in}{3.109159in}}%
\pgfpathlineto{\pgfqpoint{1.733306in}{3.110125in}}%
\pgfpathlineto{\pgfqpoint{1.729875in}{3.110601in}}%
\pgfpathlineto{\pgfqpoint{1.727045in}{3.110998in}}%
\pgfpathlineto{\pgfqpoint{1.720785in}{3.111775in}}%
\pgfpathlineto{\pgfqpoint{1.714524in}{3.112452in}}%
\pgfpathlineto{\pgfqpoint{1.708264in}{3.113024in}}%
\pgfpathlineto{\pgfqpoint{1.702003in}{3.113488in}}%
\pgfpathlineto{\pgfqpoint{1.695743in}{3.113838in}}%
\pgfpathlineto{\pgfqpoint{1.689482in}{3.114069in}}%
\pgfpathlineto{\pgfqpoint{1.683222in}{3.114174in}}%
\pgfpathlineto{\pgfqpoint{1.676962in}{3.114146in}}%
\pgfpathlineto{\pgfqpoint{1.670701in}{3.113977in}}%
\pgfpathlineto{\pgfqpoint{1.664441in}{3.113658in}}%
\pgfpathlineto{\pgfqpoint{1.658180in}{3.113180in}}%
\pgfpathlineto{\pgfqpoint{1.651920in}{3.112533in}}%
\pgfpathlineto{\pgfqpoint{1.645659in}{3.111703in}}%
\pgfpathlineto{\pgfqpoint{1.639399in}{3.110679in}}%
\pgfpathlineto{\pgfqpoint{1.638995in}{3.110601in}}%
\pgfpathlineto{\pgfqpoint{1.633139in}{3.109422in}}%
\pgfpathlineto{\pgfqpoint{1.626878in}{3.107931in}}%
\pgfpathlineto{\pgfqpoint{1.620618in}{3.106185in}}%
\pgfpathlineto{\pgfqpoint{1.614889in}{3.104341in}}%
\pgfpathlineto{\pgfqpoint{1.614357in}{3.104159in}}%
\pgfpathlineto{\pgfqpoint{1.608097in}{3.101772in}}%
\pgfpathlineto{\pgfqpoint{1.601836in}{3.099040in}}%
\pgfpathlineto{\pgfqpoint{1.599857in}{3.098080in}}%
\pgfpathlineto{\pgfqpoint{1.595576in}{3.095852in}}%
\pgfpathlineto{\pgfqpoint{1.589316in}{3.092186in}}%
\pgfpathlineto{\pgfqpoint{1.588744in}{3.091820in}}%
\pgfpathlineto{\pgfqpoint{1.583055in}{3.087858in}}%
\pgfpathlineto{\pgfqpoint{1.580073in}{3.085559in}}%
\pgfpathlineto{\pgfqpoint{1.576795in}{3.082782in}}%
\pgfpathlineto{\pgfqpoint{1.573038in}{3.079299in}}%
\pgfpathlineto{\pgfqpoint{1.570534in}{3.076713in}}%
\pgfpathlineto{\pgfqpoint{1.567251in}{3.073038in}}%
\pgfpathlineto{\pgfqpoint{1.564274in}{3.069269in}}%
\pgfpathlineto{\pgfqpoint{1.562443in}{3.066778in}}%
\pgfpathlineto{\pgfqpoint{1.558434in}{3.060518in}}%
\pgfpathlineto{\pgfqpoint{1.558013in}{3.059764in}}%
\pgfpathlineto{\pgfqpoint{1.555122in}{3.054257in}}%
\pgfpathlineto{\pgfqpoint{1.552371in}{3.047997in}}%
\pgfpathlineto{\pgfqpoint{1.551753in}{3.046325in}}%
\pgfpathlineto{\pgfqpoint{1.550141in}{3.041736in}}%
\pgfpathlineto{\pgfqpoint{1.548357in}{3.035476in}}%
\pgfpathlineto{\pgfqpoint{1.546970in}{3.029215in}}%
\pgfpathlineto{\pgfqpoint{1.545946in}{3.022955in}}%
\pgfpathlineto{\pgfqpoint{1.545492in}{3.018892in}}%
\pgfpathlineto{\pgfqpoint{1.545256in}{3.016694in}}%
\pgfpathlineto{\pgfqpoint{1.544874in}{3.010434in}}%
\pgfpathlineto{\pgfqpoint{1.544771in}{3.004174in}}%
\pgfpathlineto{\pgfqpoint{1.544930in}{2.997913in}}%
\pgfpathlineto{\pgfqpoint{1.545333in}{2.991653in}}%
\pgfpathlineto{\pgfqpoint{1.545492in}{2.990100in}}%
\pgfpathlineto{\pgfqpoint{1.545968in}{2.985392in}}%
\pgfpathlineto{\pgfqpoint{1.546818in}{2.979132in}}%
\pgfpathlineto{\pgfqpoint{1.547873in}{2.972871in}}%
\pgfpathlineto{\pgfqpoint{1.549121in}{2.966611in}}%
\pgfpathlineto{\pgfqpoint{1.550555in}{2.960351in}}%
\pgfpathlineto{\pgfqpoint{1.551753in}{2.955710in}}%
\pgfpathlineto{\pgfqpoint{1.552165in}{2.954090in}}%
\pgfpathlineto{\pgfqpoint{1.553945in}{2.947830in}}%
\pgfpathlineto{\pgfqpoint{1.555886in}{2.941569in}}%
\pgfpathlineto{\pgfqpoint{1.557986in}{2.935309in}}%
\pgfpathlineto{\pgfqpoint{1.558013in}{2.935233in}}%
\pgfpathlineto{\pgfqpoint{1.560233in}{2.929048in}}%
\pgfpathlineto{\pgfqpoint{1.562628in}{2.922788in}}%
\pgfpathlineto{\pgfqpoint{1.564274in}{2.918742in}}%
\pgfpathlineto{\pgfqpoint{1.565164in}{2.916528in}}%
\pgfpathlineto{\pgfqpoint{1.567834in}{2.910267in}}%
\pgfpathlineto{\pgfqpoint{1.570534in}{2.904249in}}%
\pgfpathlineto{\pgfqpoint{1.570642in}{2.904007in}}%
\pgfpathlineto{\pgfqpoint{1.573572in}{2.897746in}}%
\pgfpathlineto{\pgfqpoint{1.576635in}{2.891486in}}%
\pgfpathlineto{\pgfqpoint{1.576795in}{2.891175in}}%
\pgfpathlineto{\pgfqpoint{1.579814in}{2.885225in}}%
\pgfpathlineto{\pgfqpoint{1.583055in}{2.879090in}}%
\pgfpathlineto{\pgfqpoint{1.583120in}{2.878965in}}%
\pgfpathlineto{\pgfqpoint{1.586536in}{2.872705in}}%
\pgfpathlineto{\pgfqpoint{1.589316in}{2.867795in}}%
\pgfpathlineto{\pgfqpoint{1.590074in}{2.866444in}}%
\pgfpathlineto{\pgfqpoint{1.593719in}{2.860184in}}%
\pgfpathlineto{\pgfqpoint{1.595576in}{2.857104in}}%
\pgfpathlineto{\pgfqpoint{1.597478in}{2.853923in}}%
\pgfpathlineto{\pgfqpoint{1.601349in}{2.847663in}}%
\pgfpathlineto{\pgfqpoint{1.601836in}{2.846900in}}%
\pgfpathlineto{\pgfqpoint{1.605320in}{2.841402in}}%
\pgfpathlineto{\pgfqpoint{1.608097in}{2.837155in}}%
\pgfpathlineto{\pgfqpoint{1.609404in}{2.835142in}}%
\pgfpathlineto{\pgfqpoint{1.613591in}{2.828881in}}%
\pgfpathlineto{\pgfqpoint{1.614357in}{2.827767in}}%
\pgfpathlineto{\pgfqpoint{1.617876in}{2.822621in}}%
\pgfpathlineto{\pgfqpoint{1.620618in}{2.818722in}}%
\pgfpathlineto{\pgfqpoint{1.622269in}{2.816361in}}%
\pgfpathlineto{\pgfqpoint{1.626765in}{2.810100in}}%
\pgfpathlineto{\pgfqpoint{1.626878in}{2.809947in}}%
\pgfpathlineto{\pgfqpoint{1.631352in}{2.803840in}}%
\pgfpathlineto{\pgfqpoint{1.633139in}{2.801462in}}%
\pgfpathlineto{\pgfqpoint{1.636042in}{2.797579in}}%
\pgfpathlineto{\pgfqpoint{1.639399in}{2.793199in}}%
\pgfpathlineto{\pgfqpoint{1.640834in}{2.791319in}}%
\pgfpathlineto{\pgfqpoint{1.645659in}{2.785144in}}%
\pgfpathlineto{\pgfqpoint{1.645726in}{2.785058in}}%
\pgfpathlineto{\pgfqpoint{1.650707in}{2.778798in}}%
\pgfpathlineto{\pgfqpoint{1.651920in}{2.777306in}}%
\pgfpathlineto{\pgfqpoint{1.655786in}{2.772538in}}%
\pgfpathlineto{\pgfqpoint{1.658180in}{2.769646in}}%
\pgfpathlineto{\pgfqpoint{1.660963in}{2.766277in}}%
\pgfpathlineto{\pgfqpoint{1.664441in}{2.762153in}}%
\pgfpathlineto{\pgfqpoint{1.666238in}{2.760017in}}%
\pgfpathlineto{\pgfqpoint{1.670701in}{2.754817in}}%
\pgfpathlineto{\pgfqpoint{1.671610in}{2.753756in}}%
\pgfpathlineto{\pgfqpoint{1.676962in}{2.747629in}}%
\pgfpathlineto{\pgfqpoint{1.677078in}{2.747496in}}%
\pgfpathlineto{\pgfqpoint{1.682636in}{2.741235in}}%
\pgfpathlineto{\pgfqpoint{1.683222in}{2.740587in}}%
\pgfpathlineto{\pgfqpoint{1.688290in}{2.734975in}}%
\pgfpathlineto{\pgfqpoint{1.689482in}{2.733676in}}%
\pgfpathlineto{\pgfqpoint{1.694039in}{2.728715in}}%
\pgfpathlineto{\pgfqpoint{1.695743in}{2.726890in}}%
\pgfpathlineto{\pgfqpoint{1.699885in}{2.722454in}}%
\pgfpathlineto{\pgfqpoint{1.702003in}{2.720221in}}%
\pgfpathlineto{\pgfqpoint{1.705827in}{2.716194in}}%
\pgfpathlineto{\pgfqpoint{1.708264in}{2.713665in}}%
\pgfpathlineto{\pgfqpoint{1.711864in}{2.709933in}}%
\pgfpathlineto{\pgfqpoint{1.714524in}{2.707215in}}%
\pgfpathlineto{\pgfqpoint{1.717998in}{2.703673in}}%
\pgfpathlineto{\pgfqpoint{1.720785in}{2.700869in}}%
\pgfpathlineto{\pgfqpoint{1.724227in}{2.697412in}}%
\pgfpathlineto{\pgfqpoint{1.727045in}{2.694620in}}%
\pgfpathlineto{\pgfqpoint{1.730553in}{2.691152in}}%
\pgfpathlineto{\pgfqpoint{1.733306in}{2.688465in}}%
\pgfpathlineto{\pgfqpoint{1.736976in}{2.684891in}}%
\pgfpathlineto{\pgfqpoint{1.739566in}{2.682400in}}%
\pgfpathlineto{\pgfqpoint{1.743496in}{2.678631in}}%
\pgfpathlineto{\pgfqpoint{1.745826in}{2.676421in}}%
\pgfpathlineto{\pgfqpoint{1.750114in}{2.672371in}}%
\pgfpathlineto{\pgfqpoint{1.752087in}{2.670525in}}%
\pgfpathlineto{\pgfqpoint{1.756829in}{2.666110in}}%
\pgfpathlineto{\pgfqpoint{1.758347in}{2.664710in}}%
\pgfpathlineto{\pgfqpoint{1.763642in}{2.659850in}}%
\pgfpathlineto{\pgfqpoint{1.764608in}{2.658972in}}%
\pgfpathlineto{\pgfqpoint{1.770555in}{2.653589in}}%
\pgfpathlineto{\pgfqpoint{1.770868in}{2.653308in}}%
\pgfpathlineto{\pgfqpoint{1.777129in}{2.647715in}}%
\pgfpathlineto{\pgfqpoint{1.777563in}{2.647329in}}%
\pgfpathlineto{\pgfqpoint{1.783389in}{2.642190in}}%
\pgfpathlineto{\pgfqpoint{1.784668in}{2.641068in}}%
\pgfpathlineto{\pgfqpoint{1.789649in}{2.636731in}}%
\pgfpathlineto{\pgfqpoint{1.791872in}{2.634808in}}%
\pgfpathlineto{\pgfqpoint{1.795910in}{2.631336in}}%
\pgfpathlineto{\pgfqpoint{1.799175in}{2.628548in}}%
\pgfpathlineto{\pgfqpoint{1.802170in}{2.626003in}}%
\pgfpathlineto{\pgfqpoint{1.806577in}{2.622287in}}%
\pgfpathlineto{\pgfqpoint{1.808431in}{2.620731in}}%
\pgfpathlineto{\pgfqpoint{1.814078in}{2.616027in}}%
\pgfpathlineto{\pgfqpoint{1.814691in}{2.615518in}}%
\pgfpathlineto{\pgfqpoint{1.820952in}{2.610357in}}%
\pgfpathlineto{\pgfqpoint{1.821673in}{2.609766in}}%
\pgfpathlineto{\pgfqpoint{1.827212in}{2.605246in}}%
\pgfpathlineto{\pgfqpoint{1.829362in}{2.603506in}}%
\pgfpathlineto{\pgfqpoint{1.833472in}{2.600187in}}%
\pgfpathlineto{\pgfqpoint{1.837147in}{2.597245in}}%
\pgfpathlineto{\pgfqpoint{1.839733in}{2.595178in}}%
\pgfpathlineto{\pgfqpoint{1.845026in}{2.590985in}}%
\pgfpathlineto{\pgfqpoint{1.845993in}{2.590219in}}%
\pgfpathlineto{\pgfqpoint{1.852254in}{2.585303in}}%
\pgfpathlineto{\pgfqpoint{1.852995in}{2.584725in}}%
\pgfpathlineto{\pgfqpoint{1.858514in}{2.580422in}}%
\pgfpathlineto{\pgfqpoint{1.861050in}{2.578464in}}%
\pgfpathlineto{\pgfqpoint{1.864775in}{2.575585in}}%
\pgfpathlineto{\pgfqpoint{1.869192in}{2.572204in}}%
\pgfpathlineto{\pgfqpoint{1.871035in}{2.570790in}}%
\pgfpathlineto{\pgfqpoint{1.877295in}{2.566035in}}%
\pgfpathlineto{\pgfqpoint{1.877417in}{2.565943in}}%
\pgfpathlineto{\pgfqpoint{1.883556in}{2.561299in}}%
\pgfpathlineto{\pgfqpoint{1.885714in}{2.559683in}}%
\pgfpathlineto{\pgfqpoint{1.889816in}{2.556599in}}%
\pgfpathlineto{\pgfqpoint{1.894086in}{2.553422in}}%
\pgfpathlineto{\pgfqpoint{1.896077in}{2.551933in}}%
\pgfpathlineto{\pgfqpoint{1.902337in}{2.547299in}}%
\pgfpathlineto{\pgfqpoint{1.902524in}{2.547162in}}%
\pgfpathlineto{\pgfqpoint{1.908598in}{2.542672in}}%
\pgfpathlineto{\pgfqpoint{1.911018in}{2.540902in}}%
\pgfpathlineto{\pgfqpoint{1.914858in}{2.538072in}}%
\pgfpathlineto{\pgfqpoint{1.919566in}{2.534641in}}%
\pgfpathlineto{\pgfqpoint{1.921119in}{2.533499in}}%
\pgfpathlineto{\pgfqpoint{1.927379in}{2.528941in}}%
\pgfpathlineto{\pgfqpoint{1.928155in}{2.528381in}}%
\pgfpathlineto{\pgfqpoint{1.933639in}{2.524382in}}%
\pgfpathlineto{\pgfqpoint{1.936777in}{2.522120in}}%
\pgfpathlineto{\pgfqpoint{1.939900in}{2.519842in}}%
\pgfpathlineto{\pgfqpoint{1.945421in}{2.515860in}}%
\pgfpathlineto{\pgfqpoint{1.946160in}{2.515319in}}%
\pgfpathlineto{\pgfqpoint{1.952421in}{2.510785in}}%
\pgfpathlineto{\pgfqpoint{1.954075in}{2.509599in}}%
\pgfpathlineto{\pgfqpoint{1.958681in}{2.506248in}}%
\pgfpathlineto{\pgfqpoint{1.962726in}{2.503339in}}%
\pgfpathlineto{\pgfqpoint{1.964942in}{2.501718in}}%
\pgfpathlineto{\pgfqpoint{1.971202in}{2.497190in}}%
\pgfpathlineto{\pgfqpoint{1.971358in}{2.497078in}}%
\pgfpathlineto{\pgfqpoint{1.977462in}{2.492620in}}%
\pgfpathlineto{\pgfqpoint{1.979959in}{2.490818in}}%
\pgfpathlineto{\pgfqpoint{1.983723in}{2.488044in}}%
\pgfpathlineto{\pgfqpoint{1.988509in}{2.484558in}}%
\pgfpathlineto{\pgfqpoint{1.989983in}{2.483459in}}%
\pgfpathlineto{\pgfqpoint{1.996244in}{2.478843in}}%
\pgfpathlineto{\pgfqpoint{1.996992in}{2.478297in}}%
\pgfpathlineto{\pgfqpoint{2.002504in}{2.474173in}}%
\pgfpathlineto{\pgfqpoint{2.005393in}{2.472037in}}%
\pgfpathlineto{\pgfqpoint{2.008765in}{2.469474in}}%
\pgfpathlineto{\pgfqpoint{2.013686in}{2.465776in}}%
\pgfpathlineto{\pgfqpoint{2.015025in}{2.464740in}}%
\pgfpathlineto{\pgfqpoint{2.021285in}{2.459948in}}%
\pgfpathlineto{\pgfqpoint{2.021856in}{2.459516in}}%
\pgfpathlineto{\pgfqpoint{2.027546in}{2.455063in}}%
\pgfpathlineto{\pgfqpoint{2.029884in}{2.453255in}}%
\pgfpathlineto{\pgfqpoint{2.033806in}{2.450113in}}%
\pgfpathlineto{\pgfqpoint{2.037745in}{2.446995in}}%
\pgfpathlineto{\pgfqpoint{2.040067in}{2.445084in}}%
\pgfpathlineto{\pgfqpoint{2.045417in}{2.440735in}}%
\pgfpathlineto{\pgfqpoint{2.046327in}{2.439962in}}%
\pgfpathlineto{\pgfqpoint{2.052588in}{2.434717in}}%
\pgfpathlineto{\pgfqpoint{2.052882in}{2.434474in}}%
\pgfpathlineto{\pgfqpoint{2.058848in}{2.429304in}}%
\pgfpathlineto{\pgfqpoint{2.060123in}{2.428214in}}%
\pgfpathlineto{\pgfqpoint{2.065109in}{2.423732in}}%
\pgfpathlineto{\pgfqpoint{2.067113in}{2.421953in}}%
\pgfpathlineto{\pgfqpoint{2.071369in}{2.417969in}}%
\pgfpathlineto{\pgfqpoint{2.073833in}{2.415693in}}%
\pgfpathlineto{\pgfqpoint{2.077629in}{2.411975in}}%
\pgfpathlineto{\pgfqpoint{2.080262in}{2.409432in}}%
\pgfpathlineto{\pgfqpoint{2.083890in}{2.405699in}}%
\pgfpathlineto{\pgfqpoint{2.086382in}{2.403172in}}%
\pgfpathlineto{\pgfqpoint{2.090150in}{2.399077in}}%
\pgfpathlineto{\pgfqpoint{2.092173in}{2.396912in}}%
\pgfpathlineto{\pgfqpoint{2.096411in}{2.392020in}}%
\pgfpathlineto{\pgfqpoint{2.097616in}{2.390651in}}%
\pgfpathlineto{\pgfqpoint{2.102671in}{2.384415in}}%
\pgfpathlineto{\pgfqpoint{2.102691in}{2.384391in}}%
\pgfpathlineto{\pgfqpoint{2.107397in}{2.378130in}}%
\pgfpathlineto{\pgfqpoint{2.108932in}{2.375872in}}%
\pgfpathlineto{\pgfqpoint{2.111701in}{2.371870in}}%
\pgfpathlineto{\pgfqpoint{2.115192in}{2.366228in}}%
\pgfpathlineto{\pgfqpoint{2.115582in}{2.365609in}}%
\pgfpathlineto{\pgfqpoint{2.119047in}{2.359349in}}%
\pgfpathlineto{\pgfqpoint{2.121452in}{2.354318in}}%
\pgfpathlineto{\pgfqpoint{2.122053in}{2.353088in}}%
\pgfpathlineto{\pgfqpoint{2.124610in}{2.346828in}}%
\pgfpathlineto{\pgfqpoint{2.126678in}{2.340568in}}%
\pgfpathlineto{\pgfqpoint{2.127713in}{2.336418in}}%
\pgfpathlineto{\pgfqpoint{2.128253in}{2.334307in}}%
\pgfpathlineto{\pgfqpoint{2.129323in}{2.328047in}}%
\pgfpathlineto{\pgfqpoint{2.129860in}{2.321786in}}%
\pgfpathlineto{\pgfqpoint{2.129846in}{2.315526in}}%
\pgfpathlineto{\pgfqpoint{2.129261in}{2.309265in}}%
\pgfpathlineto{\pgfqpoint{2.128080in}{2.303005in}}%
\pgfpathlineto{\pgfqpoint{2.127713in}{2.301708in}}%
\pgfpathlineto{\pgfqpoint{2.126296in}{2.296745in}}%
\pgfpathlineto{\pgfqpoint{2.123872in}{2.290484in}}%
\pgfpathlineto{\pgfqpoint{2.121452in}{2.285576in}}%
\pgfpathlineto{\pgfqpoint{2.120778in}{2.284224in}}%
\pgfpathlineto{\pgfqpoint{2.117009in}{2.277963in}}%
\pgfpathlineto{\pgfqpoint{2.115192in}{2.275406in}}%
\pgfpathlineto{\pgfqpoint{2.112527in}{2.271703in}}%
\pgfpathlineto{\pgfqpoint{2.108932in}{2.267369in}}%
\pgfpathlineto{\pgfqpoint{2.107310in}{2.265442in}}%
\pgfpathlineto{\pgfqpoint{2.102671in}{2.260558in}}%
\pgfpathlineto{\pgfqpoint{2.101345in}{2.259182in}}%
\pgfpathlineto{\pgfqpoint{2.096411in}{2.254570in}}%
\pgfpathlineto{\pgfqpoint{2.094619in}{2.252922in}}%
\pgfpathlineto{\pgfqpoint{2.090150in}{2.249165in}}%
\pgfpathlineto{\pgfqpoint{2.087125in}{2.246661in}}%
\pgfpathlineto{\pgfqpoint{2.083890in}{2.244188in}}%
\pgfpathlineto{\pgfqpoint{2.078859in}{2.240401in}}%
\pgfpathlineto{\pgfqpoint{2.077629in}{2.239538in}}%
\pgfpathlineto{\pgfqpoint{2.071369in}{2.235188in}}%
\pgfpathlineto{\pgfqpoint{2.069846in}{2.234140in}}%
\pgfpathlineto{\pgfqpoint{2.065109in}{2.231073in}}%
\pgfpathlineto{\pgfqpoint{2.060108in}{2.227880in}}%
\pgfpathlineto{\pgfqpoint{2.058848in}{2.227117in}}%
\pgfpathlineto{\pgfqpoint{2.052588in}{2.223346in}}%
\pgfpathlineto{\pgfqpoint{2.049696in}{2.221619in}}%
\pgfpathlineto{\pgfqpoint{2.046327in}{2.219698in}}%
\pgfpathlineto{\pgfqpoint{2.040067in}{2.216156in}}%
\pgfpathlineto{\pgfqpoint{2.038655in}{2.215359in}}%
\pgfpathlineto{\pgfqpoint{2.033806in}{2.212728in}}%
\pgfpathlineto{\pgfqpoint{2.027546in}{2.209357in}}%
\pgfpathlineto{\pgfqpoint{2.027065in}{2.209098in}}%
\pgfpathlineto{\pgfqpoint{2.021285in}{2.206086in}}%
\pgfpathlineto{\pgfqpoint{2.015025in}{2.202844in}}%
\pgfpathlineto{\pgfqpoint{2.015014in}{2.202838in}}%
\pgfpathlineto{\pgfqpoint{2.008765in}{2.199680in}}%
\pgfpathlineto{\pgfqpoint{2.002597in}{2.196578in}}%
\pgfpathlineto{\pgfqpoint{2.002504in}{2.196532in}}%
\pgfpathlineto{\pgfqpoint{1.996244in}{2.193441in}}%
\pgfpathlineto{\pgfqpoint{1.989983in}{2.190355in}}%
\pgfpathlineto{\pgfqpoint{1.989908in}{2.190317in}}%
\pgfpathlineto{\pgfqpoint{1.983723in}{2.187309in}}%
\pgfpathlineto{\pgfqpoint{1.977462in}{2.184261in}}%
\pgfpathlineto{\pgfqpoint{1.977046in}{2.184057in}}%
\pgfpathlineto{\pgfqpoint{1.971202in}{2.181236in}}%
\pgfpathlineto{\pgfqpoint{1.964942in}{2.178206in}}%
\pgfpathlineto{\pgfqpoint{1.964103in}{2.177796in}}%
\pgfpathlineto{\pgfqpoint{1.958681in}{2.175183in}}%
\pgfpathlineto{\pgfqpoint{1.952421in}{2.172150in}}%
\pgfpathlineto{\pgfqpoint{1.951163in}{2.171536in}}%
\pgfpathlineto{\pgfqpoint{1.946160in}{2.169114in}}%
\pgfpathlineto{\pgfqpoint{1.939900in}{2.166062in}}%
\pgfpathlineto{\pgfqpoint{1.938301in}{2.165275in}}%
\pgfpathlineto{\pgfqpoint{1.933639in}{2.162996in}}%
\pgfpathlineto{\pgfqpoint{1.927379in}{2.159910in}}%
\pgfpathlineto{\pgfqpoint{1.925582in}{2.159015in}}%
\pgfpathlineto{\pgfqpoint{1.921119in}{2.156801in}}%
\pgfpathlineto{\pgfqpoint{1.914858in}{2.153666in}}%
\pgfpathlineto{\pgfqpoint{1.913057in}{2.152755in}}%
\pgfpathlineto{\pgfqpoint{1.908598in}{2.150500in}}%
\pgfpathlineto{\pgfqpoint{1.902337in}{2.147304in}}%
\pgfpathlineto{\pgfqpoint{1.900769in}{2.146494in}}%
\pgfpathlineto{\pgfqpoint{1.896077in}{2.144069in}}%
\pgfpathlineto{\pgfqpoint{1.889816in}{2.140798in}}%
\pgfpathlineto{\pgfqpoint{1.888748in}{2.140234in}}%
\pgfpathlineto{\pgfqpoint{1.883556in}{2.137481in}}%
\pgfpathlineto{\pgfqpoint{1.877295in}{2.134124in}}%
\pgfpathlineto{\pgfqpoint{1.877017in}{2.133973in}}%
\pgfpathlineto{\pgfqpoint{1.871035in}{2.130711in}}%
\pgfpathlineto{\pgfqpoint{1.865600in}{2.127713in}}%
\pgfpathlineto{\pgfqpoint{1.864775in}{2.127254in}}%
\pgfpathlineto{\pgfqpoint{1.858514in}{2.123734in}}%
\pgfpathlineto{\pgfqpoint{1.854502in}{2.121452in}}%
\pgfpathlineto{\pgfqpoint{1.852254in}{2.120160in}}%
\pgfpathlineto{\pgfqpoint{1.845993in}{2.116524in}}%
\pgfpathlineto{\pgfqpoint{1.843725in}{2.115192in}}%
\pgfpathlineto{\pgfqpoint{1.839733in}{2.112819in}}%
\pgfpathlineto{\pgfqpoint{1.833472in}{2.109055in}}%
\pgfpathlineto{\pgfqpoint{1.833270in}{2.108932in}}%
\pgfpathlineto{\pgfqpoint{1.827212in}{2.105201in}}%
\pgfpathlineto{\pgfqpoint{1.823150in}{2.102671in}}%
\pgfpathlineto{\pgfqpoint{1.820952in}{2.101280in}}%
\pgfpathlineto{\pgfqpoint{1.814691in}{2.097278in}}%
\pgfpathlineto{\pgfqpoint{1.813348in}{2.096411in}}%
\pgfpathlineto{\pgfqpoint{1.808431in}{2.093180in}}%
\pgfpathlineto{\pgfqpoint{1.803868in}{2.090150in}}%
\pgfpathlineto{\pgfqpoint{1.802170in}{2.089001in}}%
\pgfpathlineto{\pgfqpoint{1.795910in}{2.084722in}}%
\pgfpathlineto{\pgfqpoint{1.794703in}{2.083890in}}%
\pgfpathlineto{\pgfqpoint{1.789649in}{2.080331in}}%
\pgfpathlineto{\pgfqpoint{1.785851in}{2.077629in}}%
\pgfpathlineto{\pgfqpoint{1.783389in}{2.075839in}}%
\pgfpathlineto{\pgfqpoint{1.777299in}{2.071369in}}%
\pgfpathlineto{\pgfqpoint{1.777129in}{2.071241in}}%
\pgfpathlineto{\pgfqpoint{1.770868in}{2.066495in}}%
\pgfpathlineto{\pgfqpoint{1.769054in}{2.065109in}}%
\pgfpathlineto{\pgfqpoint{1.764608in}{2.061619in}}%
\pgfpathlineto{\pgfqpoint{1.761103in}{2.058848in}}%
\pgfpathlineto{\pgfqpoint{1.758347in}{2.056606in}}%
\pgfpathlineto{\pgfqpoint{1.753441in}{2.052588in}}%
\pgfpathlineto{\pgfqpoint{1.752087in}{2.051444in}}%
\pgfpathlineto{\pgfqpoint{1.746065in}{2.046327in}}%
\pgfpathlineto{\pgfqpoint{1.745826in}{2.046118in}}%
\pgfpathlineto{\pgfqpoint{1.739566in}{2.040592in}}%
\pgfpathlineto{\pgfqpoint{1.738974in}{2.040067in}}%
\pgfpathlineto{\pgfqpoint{1.733306in}{2.034861in}}%
\pgfpathlineto{\pgfqpoint{1.732162in}{2.033806in}}%
\pgfpathlineto{\pgfqpoint{1.727045in}{2.028909in}}%
\pgfpathlineto{\pgfqpoint{1.725626in}{2.027546in}}%
\pgfpathlineto{\pgfqpoint{1.720785in}{2.022709in}}%
\pgfpathlineto{\pgfqpoint{1.719364in}{2.021285in}}%
\pgfpathlineto{\pgfqpoint{1.714524in}{2.016230in}}%
\pgfpathlineto{\pgfqpoint{1.713373in}{2.015025in}}%
\pgfpathlineto{\pgfqpoint{1.708264in}{2.009434in}}%
\pgfpathlineto{\pgfqpoint{1.707654in}{2.008765in}}%
\pgfpathlineto{\pgfqpoint{1.702204in}{2.002504in}}%
\pgfpathlineto{\pgfqpoint{1.702003in}{2.002262in}}%
\pgfpathlineto{\pgfqpoint{1.697026in}{1.996244in}}%
\pgfpathlineto{\pgfqpoint{1.695743in}{1.994607in}}%
\pgfpathlineto{\pgfqpoint{1.692118in}{1.989983in}}%
\pgfpathlineto{\pgfqpoint{1.689482in}{1.986424in}}%
\pgfpathlineto{\pgfqpoint{1.687481in}{1.983723in}}%
\pgfpathlineto{\pgfqpoint{1.683222in}{1.977615in}}%
\pgfpathlineto{\pgfqpoint{1.683115in}{1.977462in}}%
\pgfpathlineto{\pgfqpoint{1.679030in}{1.971202in}}%
\pgfpathlineto{\pgfqpoint{1.676962in}{1.967795in}}%
\pgfpathlineto{\pgfqpoint{1.675223in}{1.964942in}}%
\pgfpathlineto{\pgfqpoint{1.671698in}{1.958681in}}%
\pgfpathlineto{\pgfqpoint{1.670701in}{1.956744in}}%
\pgfpathlineto{\pgfqpoint{1.668465in}{1.952421in}}%
\pgfpathlineto{\pgfqpoint{1.665524in}{1.946160in}}%
\pgfpathlineto{\pgfqpoint{1.664441in}{1.943580in}}%
\pgfpathlineto{\pgfqpoint{1.662886in}{1.939900in}}%
\pgfpathlineto{\pgfqpoint{1.660560in}{1.933639in}}%
\pgfpathlineto{\pgfqpoint{1.658549in}{1.927379in}}%
\pgfpathlineto{\pgfqpoint{1.658180in}{1.925992in}}%
\pgfpathlineto{\pgfqpoint{1.656874in}{1.921119in}}%
\pgfpathlineto{\pgfqpoint{1.655543in}{1.914858in}}%
\pgfpathlineto{\pgfqpoint{1.654570in}{1.908598in}}%
\pgfpathlineto{\pgfqpoint{1.653972in}{1.902337in}}%
\pgfpathlineto{\pgfqpoint{1.653770in}{1.896077in}}%
\pgfpathlineto{\pgfqpoint{1.653989in}{1.889816in}}%
\pgfpathlineto{\pgfqpoint{1.654656in}{1.883556in}}%
\pgfpathlineto{\pgfqpoint{1.655802in}{1.877295in}}%
\pgfpathlineto{\pgfqpoint{1.657462in}{1.871035in}}%
\pgfpathlineto{\pgfqpoint{1.658180in}{1.868966in}}%
\pgfpathlineto{\pgfqpoint{1.659696in}{1.864775in}}%
\pgfpathlineto{\pgfqpoint{1.662552in}{1.858514in}}%
\pgfpathlineto{\pgfqpoint{1.664441in}{1.855109in}}%
\pgfpathlineto{\pgfqpoint{1.666102in}{1.852254in}}%
\pgfpathlineto{\pgfqpoint{1.670429in}{1.845993in}}%
\pgfpathlineto{\pgfqpoint{1.670701in}{1.845651in}}%
\pgfpathlineto{\pgfqpoint{1.675676in}{1.839733in}}%
\pgfpathlineto{\pgfqpoint{1.676962in}{1.838393in}}%
\pgfpathlineto{\pgfqpoint{1.681979in}{1.833472in}}%
\pgfpathlineto{\pgfqpoint{1.683222in}{1.832385in}}%
\pgfpathlineto{\pgfqpoint{1.689482in}{1.827265in}}%
\pgfpathlineto{\pgfqpoint{1.689552in}{1.827212in}}%
\pgfpathlineto{\pgfqpoint{1.695743in}{1.822900in}}%
\pgfpathlineto{\pgfqpoint{1.698766in}{1.820952in}}%
\pgfpathlineto{\pgfqpoint{1.702003in}{1.819037in}}%
\pgfpathlineto{\pgfqpoint{1.708264in}{1.815611in}}%
\pgfpathlineto{\pgfqpoint{1.710073in}{1.814691in}}%
\pgfpathlineto{\pgfqpoint{1.714524in}{1.812590in}}%
\pgfpathlineto{\pgfqpoint{1.720785in}{1.809876in}}%
\pgfpathlineto{\pgfqpoint{1.724411in}{1.808431in}}%
\pgfpathlineto{\pgfqpoint{1.727045in}{1.807446in}}%
\pgfpathlineto{\pgfqpoint{1.733306in}{1.805290in}}%
\pgfpathlineto{\pgfqpoint{1.739566in}{1.803337in}}%
\pgfpathlineto{\pgfqpoint{1.743679in}{1.802170in}}%
\pgfpathlineto{\pgfqpoint{1.745826in}{1.801593in}}%
\pgfpathlineto{\pgfqpoint{1.752087in}{1.800058in}}%
\pgfpathlineto{\pgfqpoint{1.758347in}{1.798684in}}%
\pgfpathlineto{\pgfqpoint{1.764608in}{1.797464in}}%
\pgfpathlineto{\pgfqpoint{1.770868in}{1.796390in}}%
\pgfpathlineto{\pgfqpoint{1.774047in}{1.795910in}}%
\pgfpathlineto{\pgfqpoint{1.777129in}{1.795463in}}%
\pgfpathlineto{\pgfqpoint{1.783389in}{1.794675in}}%
\pgfpathlineto{\pgfqpoint{1.789649in}{1.794008in}}%
\pgfpathlineto{\pgfqpoint{1.795910in}{1.793457in}}%
\pgfpathlineto{\pgfqpoint{1.802170in}{1.793016in}}%
\pgfpathlineto{\pgfqpoint{1.808431in}{1.792681in}}%
\pgfpathlineto{\pgfqpoint{1.814691in}{1.792447in}}%
\pgfpathlineto{\pgfqpoint{1.820952in}{1.792311in}}%
\pgfpathlineto{\pgfqpoint{1.827212in}{1.792270in}}%
\pgfpathlineto{\pgfqpoint{1.833472in}{1.792318in}}%
\pgfpathlineto{\pgfqpoint{1.839733in}{1.792454in}}%
\pgfpathlineto{\pgfqpoint{1.845993in}{1.792675in}}%
\pgfpathlineto{\pgfqpoint{1.852254in}{1.792977in}}%
\pgfpathlineto{\pgfqpoint{1.858514in}{1.793358in}}%
\pgfpathlineto{\pgfqpoint{1.864775in}{1.793817in}}%
\pgfpathlineto{\pgfqpoint{1.871035in}{1.794350in}}%
\pgfpathlineto{\pgfqpoint{1.877295in}{1.794957in}}%
\pgfpathlineto{\pgfqpoint{1.883556in}{1.795635in}}%
\pgfpathlineto{\pgfqpoint{1.885847in}{1.795910in}}%
\pgfpathlineto{\pgfqpoint{1.889816in}{1.796387in}}%
\pgfpathlineto{\pgfqpoint{1.896077in}{1.797209in}}%
\pgfpathlineto{\pgfqpoint{1.902337in}{1.798098in}}%
\pgfpathlineto{\pgfqpoint{1.908598in}{1.799052in}}%
\pgfpathlineto{\pgfqpoint{1.914858in}{1.800070in}}%
\pgfpathlineto{\pgfqpoint{1.921119in}{1.801152in}}%
\pgfpathlineto{\pgfqpoint{1.926683in}{1.802170in}}%
\pgfpathlineto{\pgfqpoint{1.927379in}{1.802298in}}%
\pgfpathlineto{\pgfqpoint{1.933639in}{1.803511in}}%
\pgfpathlineto{\pgfqpoint{1.939900in}{1.804785in}}%
\pgfpathlineto{\pgfqpoint{1.946160in}{1.806118in}}%
\pgfpathlineto{\pgfqpoint{1.952421in}{1.807511in}}%
\pgfpathlineto{\pgfqpoint{1.956381in}{1.808431in}}%
\pgfpathlineto{\pgfqpoint{1.958681in}{1.808966in}}%
\pgfpathlineto{\pgfqpoint{1.964942in}{1.810483in}}%
\pgfpathlineto{\pgfqpoint{1.971202in}{1.812058in}}%
\pgfpathlineto{\pgfqpoint{1.977462in}{1.813689in}}%
\pgfpathlineto{\pgfqpoint{1.981172in}{1.814691in}}%
\pgfpathlineto{\pgfqpoint{1.983723in}{1.815382in}}%
\pgfpathlineto{\pgfqpoint{1.989983in}{1.817135in}}%
\pgfpathlineto{\pgfqpoint{1.996244in}{1.818944in}}%
\pgfpathlineto{\pgfqpoint{2.002504in}{1.820809in}}%
\pgfpathlineto{\pgfqpoint{2.002968in}{1.820952in}}%
\pgfpathlineto{\pgfqpoint{2.008765in}{1.822739in}}%
\pgfpathlineto{\pgfqpoint{2.015025in}{1.824724in}}%
\pgfpathlineto{\pgfqpoint{2.021285in}{1.826764in}}%
\pgfpathlineto{\pgfqpoint{2.022624in}{1.827212in}}%
\pgfpathlineto{\pgfqpoint{2.027546in}{1.828867in}}%
\pgfpathlineto{\pgfqpoint{2.033806in}{1.831028in}}%
\pgfpathlineto{\pgfqpoint{2.040067in}{1.833243in}}%
\pgfpathlineto{\pgfqpoint{2.040700in}{1.833472in}}%
\pgfpathlineto{\pgfqpoint{2.046327in}{1.835524in}}%
\pgfpathlineto{\pgfqpoint{2.052588in}{1.837860in}}%
\pgfpathlineto{\pgfqpoint{2.057492in}{1.839733in}}%
\pgfpathlineto{\pgfqpoint{2.058848in}{1.840254in}}%
\pgfpathlineto{\pgfqpoint{2.065109in}{1.842713in}}%
\pgfpathlineto{\pgfqpoint{2.071369in}{1.845227in}}%
\pgfpathlineto{\pgfqpoint{2.073239in}{1.845993in}}%
\pgfpathlineto{\pgfqpoint{2.077629in}{1.847806in}}%
\pgfpathlineto{\pgfqpoint{2.083890in}{1.850444in}}%
\pgfpathlineto{\pgfqpoint{2.088099in}{1.852254in}}%
\pgfpathlineto{\pgfqpoint{2.090150in}{1.853143in}}%
\pgfpathlineto{\pgfqpoint{2.096411in}{1.855908in}}%
\pgfpathlineto{\pgfqpoint{2.102200in}{1.858514in}}%
\pgfpathlineto{\pgfqpoint{2.102671in}{1.858729in}}%
\pgfpathlineto{\pgfqpoint{2.108932in}{1.861623in}}%
\pgfpathlineto{\pgfqpoint{2.115192in}{1.864571in}}%
\pgfpathlineto{\pgfqpoint{2.115618in}{1.864775in}}%
\pgfpathlineto{\pgfqpoint{2.121452in}{1.867595in}}%
\pgfpathlineto{\pgfqpoint{2.127713in}{1.870674in}}%
\pgfpathlineto{\pgfqpoint{2.128436in}{1.871035in}}%
\pgfpathlineto{\pgfqpoint{2.133973in}{1.873831in}}%
\pgfpathlineto{\pgfqpoint{2.140234in}{1.877044in}}%
\pgfpathlineto{\pgfqpoint{2.140718in}{1.877295in}}%
\pgfpathlineto{\pgfqpoint{2.146494in}{1.880338in}}%
\pgfpathlineto{\pgfqpoint{2.152512in}{1.883556in}}%
\pgfpathlineto{\pgfqpoint{2.152755in}{1.883687in}}%
\pgfpathlineto{\pgfqpoint{2.159015in}{1.887123in}}%
\pgfpathlineto{\pgfqpoint{2.163857in}{1.889816in}}%
\pgfpathlineto{\pgfqpoint{2.165275in}{1.890617in}}%
\pgfpathlineto{\pgfqpoint{2.171536in}{1.894192in}}%
\pgfpathlineto{\pgfqpoint{2.174796in}{1.896077in}}%
\pgfpathlineto{\pgfqpoint{2.177796in}{1.897838in}}%
\pgfpathlineto{\pgfqpoint{2.184057in}{1.901553in}}%
\pgfpathlineto{\pgfqpoint{2.185364in}{1.902337in}}%
\pgfpathlineto{\pgfqpoint{2.190317in}{1.905355in}}%
\pgfpathlineto{\pgfqpoint{2.195584in}{1.908598in}}%
\pgfpathlineto{\pgfqpoint{2.196578in}{1.909220in}}%
\pgfpathlineto{\pgfqpoint{2.202838in}{1.913175in}}%
\pgfpathlineto{\pgfqpoint{2.205480in}{1.914858in}}%
\pgfpathlineto{\pgfqpoint{2.209098in}{1.917206in}}%
\pgfpathlineto{\pgfqpoint{2.215081in}{1.921119in}}%
\pgfpathlineto{\pgfqpoint{2.215359in}{1.921304in}}%
\pgfpathlineto{\pgfqpoint{2.221619in}{1.925502in}}%
\pgfpathlineto{\pgfqpoint{2.224401in}{1.927379in}}%
\pgfpathlineto{\pgfqpoint{2.227880in}{1.929774in}}%
\pgfpathlineto{\pgfqpoint{2.233466in}{1.933639in}}%
\pgfpathlineto{\pgfqpoint{2.234140in}{1.934116in}}%
\pgfpathlineto{\pgfqpoint{2.240401in}{1.938555in}}%
\pgfpathlineto{\pgfqpoint{2.242291in}{1.939900in}}%
\pgfpathlineto{\pgfqpoint{2.246661in}{1.943077in}}%
\pgfpathlineto{\pgfqpoint{2.250893in}{1.946160in}}%
\pgfpathlineto{\pgfqpoint{2.252922in}{1.947671in}}%
\pgfpathlineto{\pgfqpoint{2.259182in}{1.952340in}}%
\pgfpathlineto{\pgfqpoint{2.259290in}{1.952421in}}%
\pgfpathlineto{\pgfqpoint{2.265442in}{1.957114in}}%
\pgfpathlineto{\pgfqpoint{2.267497in}{1.958681in}}%
\pgfpathlineto{\pgfqpoint{2.271703in}{1.961962in}}%
\pgfpathlineto{\pgfqpoint{2.275530in}{1.964942in}}%
\pgfpathlineto{\pgfqpoint{2.277963in}{1.966881in}}%
\pgfpathlineto{\pgfqpoint{2.283402in}{1.971202in}}%
\pgfpathlineto{\pgfqpoint{2.284224in}{1.971871in}}%
\pgfpathlineto{\pgfqpoint{2.290484in}{1.976941in}}%
\pgfpathlineto{\pgfqpoint{2.291131in}{1.977462in}}%
\pgfpathlineto{\pgfqpoint{2.296745in}{1.982094in}}%
\pgfpathlineto{\pgfqpoint{2.298733in}{1.983723in}}%
\pgfpathlineto{\pgfqpoint{2.303005in}{1.987308in}}%
\pgfpathlineto{\pgfqpoint{2.306220in}{1.989983in}}%
\pgfpathlineto{\pgfqpoint{2.309265in}{1.992580in}}%
\pgfpathlineto{\pgfqpoint{2.313608in}{1.996244in}}%
\pgfpathlineto{\pgfqpoint{2.315526in}{1.997902in}}%
\pgfpathlineto{\pgfqpoint{2.320913in}{2.002504in}}%
\pgfpathlineto{\pgfqpoint{2.321786in}{2.003269in}}%
\pgfpathlineto{\pgfqpoint{2.328047in}{2.008675in}}%
\pgfpathlineto{\pgfqpoint{2.328152in}{2.008765in}}%
\pgfpathlineto{\pgfqpoint{2.334307in}{2.014129in}}%
\pgfpathlineto{\pgfqpoint{2.335352in}{2.015025in}}%
\pgfpathlineto{\pgfqpoint{2.340568in}{2.019601in}}%
\pgfpathlineto{\pgfqpoint{2.342524in}{2.021285in}}%
\pgfpathlineto{\pgfqpoint{2.346828in}{2.025077in}}%
\pgfpathlineto{\pgfqpoint{2.349689in}{2.027546in}}%
\pgfpathlineto{\pgfqpoint{2.353088in}{2.030545in}}%
\pgfpathlineto{\pgfqpoint{2.356871in}{2.033806in}}%
\pgfpathlineto{\pgfqpoint{2.359349in}{2.035990in}}%
\pgfpathlineto{\pgfqpoint{2.364094in}{2.040067in}}%
\pgfpathlineto{\pgfqpoint{2.365609in}{2.041397in}}%
\pgfpathlineto{\pgfqpoint{2.371386in}{2.046327in}}%
\pgfpathlineto{\pgfqpoint{2.371870in}{2.046748in}}%
\pgfpathlineto{\pgfqpoint{2.378130in}{2.052045in}}%
\pgfpathlineto{\pgfqpoint{2.378792in}{2.052588in}}%
\pgfpathlineto{\pgfqpoint{2.384391in}{2.057266in}}%
\pgfpathlineto{\pgfqpoint{2.386351in}{2.058848in}}%
\pgfpathlineto{\pgfqpoint{2.390651in}{2.062382in}}%
\pgfpathlineto{\pgfqpoint{2.394097in}{2.065109in}}%
\pgfpathlineto{\pgfqpoint{2.396912in}{2.067374in}}%
\pgfpathlineto{\pgfqpoint{2.402082in}{2.071369in}}%
\pgfpathlineto{\pgfqpoint{2.403172in}{2.072224in}}%
\pgfpathlineto{\pgfqpoint{2.409432in}{2.076935in}}%
\pgfpathlineto{\pgfqpoint{2.410397in}{2.077629in}}%
\pgfpathlineto{\pgfqpoint{2.415693in}{2.081497in}}%
\pgfpathlineto{\pgfqpoint{2.419140in}{2.083890in}}%
\pgfpathlineto{\pgfqpoint{2.421953in}{2.085869in}}%
\pgfpathlineto{\pgfqpoint{2.428214in}{2.090040in}}%
\pgfpathlineto{\pgfqpoint{2.428388in}{2.090150in}}%
\pgfpathlineto{\pgfqpoint{2.434474in}{2.094044in}}%
\pgfpathlineto{\pgfqpoint{2.438418in}{2.096411in}}%
\pgfpathlineto{\pgfqpoint{2.440735in}{2.097817in}}%
\pgfpathlineto{\pgfqpoint{2.446995in}{2.101376in}}%
\pgfpathlineto{\pgfqpoint{2.449443in}{2.102671in}}%
\pgfpathlineto{\pgfqpoint{2.453255in}{2.104712in}}%
\pgfpathlineto{\pgfqpoint{2.459516in}{2.107804in}}%
\pgfpathlineto{\pgfqpoint{2.462004in}{2.108932in}}%
\pgfpathlineto{\pgfqpoint{2.465776in}{2.110661in}}%
\pgfpathlineto{\pgfqpoint{2.472037in}{2.113264in}}%
\pgfpathlineto{\pgfqpoint{2.477214in}{2.115192in}}%
\pgfpathlineto{\pgfqpoint{2.478297in}{2.115600in}}%
\pgfpathlineto{\pgfqpoint{2.484558in}{2.117687in}}%
\pgfpathlineto{\pgfqpoint{2.490818in}{2.119487in}}%
\pgfpathlineto{\pgfqpoint{2.497078in}{2.120996in}}%
\pgfpathlineto{\pgfqpoint{2.499439in}{2.121452in}}%
\pgfpathlineto{\pgfqpoint{2.503339in}{2.122217in}}%
\pgfpathlineto{\pgfqpoint{2.509599in}{2.123131in}}%
\pgfpathlineto{\pgfqpoint{2.515860in}{2.123718in}}%
\pgfpathlineto{\pgfqpoint{2.522120in}{2.123964in}}%
\pgfpathlineto{\pgfqpoint{2.528381in}{2.123850in}}%
\pgfpathlineto{\pgfqpoint{2.534641in}{2.123353in}}%
\pgfpathlineto{\pgfqpoint{2.540902in}{2.122443in}}%
\pgfpathlineto{\pgfqpoint{2.545484in}{2.121452in}}%
\pgfpathlineto{\pgfqpoint{2.547162in}{2.121088in}}%
\pgfpathlineto{\pgfqpoint{2.553422in}{2.119248in}}%
\pgfpathlineto{\pgfqpoint{2.559683in}{2.116866in}}%
\pgfpathlineto{\pgfqpoint{2.563218in}{2.115192in}}%
\pgfpathlineto{\pgfqpoint{2.565943in}{2.113875in}}%
\pgfpathlineto{\pgfqpoint{2.572204in}{2.110190in}}%
\pgfpathlineto{\pgfqpoint{2.573996in}{2.108932in}}%
\pgfpathlineto{\pgfqpoint{2.578464in}{2.105690in}}%
\pgfpathlineto{\pgfqpoint{2.581983in}{2.102671in}}%
\pgfpathlineto{\pgfqpoint{2.584725in}{2.100220in}}%
\pgfpathlineto{\pgfqpoint{2.588380in}{2.096411in}}%
\pgfpathlineto{\pgfqpoint{2.590985in}{2.093555in}}%
\pgfpathlineto{\pgfqpoint{2.593682in}{2.090150in}}%
\pgfpathlineto{\pgfqpoint{2.597245in}{2.085369in}}%
\pgfpathlineto{\pgfqpoint{2.598212in}{2.083890in}}%
\pgfpathlineto{\pgfqpoint{2.602072in}{2.077629in}}%
\pgfpathlineto{\pgfqpoint{2.603506in}{2.075143in}}%
\pgfpathlineto{\pgfqpoint{2.605436in}{2.071369in}}%
\pgfpathlineto{\pgfqpoint{2.608412in}{2.065109in}}%
\pgfpathlineto{\pgfqpoint{2.609766in}{2.062051in}}%
\pgfpathlineto{\pgfqpoint{2.611037in}{2.058848in}}%
\pgfpathlineto{\pgfqpoint{2.613352in}{2.052588in}}%
\pgfpathlineto{\pgfqpoint{2.615478in}{2.046327in}}%
\pgfpathlineto{\pgfqpoint{2.616027in}{2.044603in}}%
\pgfpathlineto{\pgfqpoint{2.617335in}{2.040067in}}%
\pgfpathlineto{\pgfqpoint{2.619011in}{2.033806in}}%
\pgfpathlineto{\pgfqpoint{2.620558in}{2.027546in}}%
\pgfpathlineto{\pgfqpoint{2.621988in}{2.021285in}}%
\pgfpathlineto{\pgfqpoint{2.622287in}{2.019900in}}%
\pgfpathlineto{\pgfqpoint{2.623251in}{2.015025in}}%
\pgfpathlineto{\pgfqpoint{2.624411in}{2.008765in}}%
\pgfpathlineto{\pgfqpoint{2.625495in}{2.002504in}}%
\pgfpathlineto{\pgfqpoint{2.626514in}{1.996244in}}%
\pgfpathlineto{\pgfqpoint{2.627475in}{1.989983in}}%
\pgfpathlineto{\pgfqpoint{2.628389in}{1.983723in}}%
\pgfpathlineto{\pgfqpoint{2.628548in}{1.982599in}}%
\pgfpathlineto{\pgfqpoint{2.629224in}{1.977462in}}%
\pgfpathlineto{\pgfqpoint{2.630022in}{1.971202in}}%
\pgfpathlineto{\pgfqpoint{2.630797in}{1.964942in}}%
\pgfpathlineto{\pgfqpoint{2.631555in}{1.958681in}}%
\pgfpathlineto{\pgfqpoint{2.632304in}{1.952421in}}%
\pgfpathlineto{\pgfqpoint{2.633049in}{1.946160in}}%
\pgfpathlineto{\pgfqpoint{2.633796in}{1.939900in}}%
\pgfpathlineto{\pgfqpoint{2.634551in}{1.933639in}}%
\pgfpathlineto{\pgfqpoint{2.634808in}{1.931551in}}%
\pgfpathlineto{\pgfqpoint{2.635297in}{1.927379in}}%
\pgfpathlineto{\pgfqpoint{2.636051in}{1.921119in}}%
\pgfpathlineto{\pgfqpoint{2.636830in}{1.914858in}}%
\pgfpathlineto{\pgfqpoint{2.637639in}{1.908598in}}%
\pgfpathlineto{\pgfqpoint{2.638483in}{1.902337in}}%
\pgfpathlineto{\pgfqpoint{2.639368in}{1.896077in}}%
\pgfpathlineto{\pgfqpoint{2.640298in}{1.889816in}}%
\pgfpathlineto{\pgfqpoint{2.641068in}{1.884905in}}%
\pgfpathlineto{\pgfqpoint{2.641273in}{1.883556in}}%
\pgfpathlineto{\pgfqpoint{2.642277in}{1.877295in}}%
\pgfpathlineto{\pgfqpoint{2.643343in}{1.871035in}}%
\pgfpathlineto{\pgfqpoint{2.644477in}{1.864775in}}%
\pgfpathlineto{\pgfqpoint{2.645685in}{1.858514in}}%
\pgfpathlineto{\pgfqpoint{2.646974in}{1.852254in}}%
\pgfpathlineto{\pgfqpoint{2.647329in}{1.850628in}}%
\pgfpathlineto{\pgfqpoint{2.648322in}{1.845993in}}%
\pgfpathlineto{\pgfqpoint{2.649753in}{1.839733in}}%
\pgfpathlineto{\pgfqpoint{2.651286in}{1.833472in}}%
\pgfpathlineto{\pgfqpoint{2.652928in}{1.827212in}}%
\pgfpathlineto{\pgfqpoint{2.653589in}{1.824839in}}%
\pgfpathlineto{\pgfqpoint{2.654663in}{1.820952in}}%
\pgfpathlineto{\pgfqpoint{2.656509in}{1.814691in}}%
\pgfpathlineto{\pgfqpoint{2.658490in}{1.808431in}}%
\pgfpathlineto{\pgfqpoint{2.659850in}{1.804402in}}%
\pgfpathlineto{\pgfqpoint{2.660603in}{1.802170in}}%
\pgfpathlineto{\pgfqpoint{2.662849in}{1.795910in}}%
\pgfpathlineto{\pgfqpoint{2.665263in}{1.789649in}}%
\pgfpathlineto{\pgfqpoint{2.666110in}{1.787577in}}%
\pgfpathlineto{\pgfqpoint{2.667837in}{1.783389in}}%
\pgfpathlineto{\pgfqpoint{2.670597in}{1.777129in}}%
\pgfpathlineto{\pgfqpoint{2.672371in}{1.773354in}}%
\pgfpathlineto{\pgfqpoint{2.673558in}{1.770868in}}%
\pgfpathlineto{\pgfqpoint{2.676734in}{1.764608in}}%
\pgfpathlineto{\pgfqpoint{2.678631in}{1.761098in}}%
\pgfpathlineto{\pgfqpoint{2.680153in}{1.758347in}}%
\pgfpathlineto{\pgfqpoint{2.683836in}{1.752087in}}%
\pgfpathlineto{\pgfqpoint{2.684891in}{1.750389in}}%
\pgfpathlineto{\pgfqpoint{2.687817in}{1.745826in}}%
\pgfpathlineto{\pgfqpoint{2.691152in}{1.740944in}}%
\pgfpathlineto{\pgfqpoint{2.692129in}{1.739566in}}%
\pgfpathlineto{\pgfqpoint{2.696817in}{1.733306in}}%
\pgfpathlineto{\pgfqpoint{2.697412in}{1.732548in}}%
\pgfpathlineto{\pgfqpoint{2.701943in}{1.727045in}}%
\pgfpathlineto{\pgfqpoint{2.703673in}{1.725051in}}%
\pgfpathlineto{\pgfqpoint{2.707572in}{1.720785in}}%
\pgfpathlineto{\pgfqpoint{2.709933in}{1.718323in}}%
\pgfpathlineto{\pgfqpoint{2.713798in}{1.714524in}}%
\pgfpathlineto{\pgfqpoint{2.716194in}{1.712270in}}%
\pgfpathlineto{\pgfqpoint{2.720746in}{1.708264in}}%
\pgfpathlineto{\pgfqpoint{2.722454in}{1.706818in}}%
\pgfpathlineto{\pgfqpoint{2.728586in}{1.702003in}}%
\pgfpathlineto{\pgfqpoint{2.728715in}{1.701906in}}%
\pgfpathlineto{\pgfqpoint{2.734975in}{1.697480in}}%
\pgfpathlineto{\pgfqpoint{2.737660in}{1.695743in}}%
\pgfpathlineto{\pgfqpoint{2.741235in}{1.693497in}}%
\pgfpathlineto{\pgfqpoint{2.747496in}{1.689926in}}%
\pgfpathlineto{\pgfqpoint{2.748347in}{1.689482in}}%
\pgfpathlineto{\pgfqpoint{2.753756in}{1.686726in}}%
\pgfpathlineto{\pgfqpoint{2.760017in}{1.683883in}}%
\pgfpathlineto{\pgfqpoint{2.761638in}{1.683222in}}%
\pgfpathlineto{\pgfqpoint{2.766277in}{1.681361in}}%
\pgfpathlineto{\pgfqpoint{2.772538in}{1.679149in}}%
\pgfpathlineto{\pgfqpoint{2.778798in}{1.677234in}}%
\pgfpathlineto{\pgfqpoint{2.779825in}{1.676962in}}%
\pgfpathclose%
\pgfpathmoveto{\pgfqpoint{2.807791in}{1.745826in}}%
\pgfpathlineto{\pgfqpoint{2.803840in}{1.747098in}}%
\pgfpathlineto{\pgfqpoint{2.797579in}{1.749464in}}%
\pgfpathlineto{\pgfqpoint{2.791622in}{1.752087in}}%
\pgfpathlineto{\pgfqpoint{2.791319in}{1.752225in}}%
\pgfpathlineto{\pgfqpoint{2.785058in}{1.755425in}}%
\pgfpathlineto{\pgfqpoint{2.780028in}{1.758347in}}%
\pgfpathlineto{\pgfqpoint{2.778798in}{1.759091in}}%
\pgfpathlineto{\pgfqpoint{2.772538in}{1.763284in}}%
\pgfpathlineto{\pgfqpoint{2.770746in}{1.764608in}}%
\pgfpathlineto{\pgfqpoint{2.766277in}{1.768065in}}%
\pgfpathlineto{\pgfqpoint{2.762970in}{1.770868in}}%
\pgfpathlineto{\pgfqpoint{2.760017in}{1.773503in}}%
\pgfpathlineto{\pgfqpoint{2.756269in}{1.777129in}}%
\pgfpathlineto{\pgfqpoint{2.753756in}{1.779700in}}%
\pgfpathlineto{\pgfqpoint{2.750400in}{1.783389in}}%
\pgfpathlineto{\pgfqpoint{2.747496in}{1.786785in}}%
\pgfpathlineto{\pgfqpoint{2.745194in}{1.789649in}}%
\pgfpathlineto{\pgfqpoint{2.741235in}{1.794920in}}%
\pgfpathlineto{\pgfqpoint{2.740531in}{1.795910in}}%
\pgfpathlineto{\pgfqpoint{2.736348in}{1.802170in}}%
\pgfpathlineto{\pgfqpoint{2.734975in}{1.804377in}}%
\pgfpathlineto{\pgfqpoint{2.732563in}{1.808431in}}%
\pgfpathlineto{\pgfqpoint{2.729119in}{1.814691in}}%
\pgfpathlineto{\pgfqpoint{2.728715in}{1.815478in}}%
\pgfpathlineto{\pgfqpoint{2.725999in}{1.820952in}}%
\pgfpathlineto{\pgfqpoint{2.723147in}{1.827212in}}%
\pgfpathlineto{\pgfqpoint{2.722454in}{1.828849in}}%
\pgfpathlineto{\pgfqpoint{2.720550in}{1.833472in}}%
\pgfpathlineto{\pgfqpoint{2.718181in}{1.839733in}}%
\pgfpathlineto{\pgfqpoint{2.716194in}{1.845476in}}%
\pgfpathlineto{\pgfqpoint{2.716018in}{1.845993in}}%
\pgfpathlineto{\pgfqpoint{2.714047in}{1.852254in}}%
\pgfpathlineto{\pgfqpoint{2.712254in}{1.858514in}}%
\pgfpathlineto{\pgfqpoint{2.710626in}{1.864775in}}%
\pgfpathlineto{\pgfqpoint{2.709933in}{1.867700in}}%
\pgfpathlineto{\pgfqpoint{2.709149in}{1.871035in}}%
\pgfpathlineto{\pgfqpoint{2.707812in}{1.877295in}}%
\pgfpathlineto{\pgfqpoint{2.706610in}{1.883556in}}%
\pgfpathlineto{\pgfqpoint{2.705535in}{1.889816in}}%
\pgfpathlineto{\pgfqpoint{2.704579in}{1.896077in}}%
\pgfpathlineto{\pgfqpoint{2.703736in}{1.902337in}}%
\pgfpathlineto{\pgfqpoint{2.703673in}{1.902871in}}%
\pgfpathlineto{\pgfqpoint{2.702989in}{1.908598in}}%
\pgfpathlineto{\pgfqpoint{2.702341in}{1.914858in}}%
\pgfpathlineto{\pgfqpoint{2.701788in}{1.921119in}}%
\pgfpathlineto{\pgfqpoint{2.701323in}{1.927379in}}%
\pgfpathlineto{\pgfqpoint{2.700943in}{1.933639in}}%
\pgfpathlineto{\pgfqpoint{2.700642in}{1.939900in}}%
\pgfpathlineto{\pgfqpoint{2.700416in}{1.946160in}}%
\pgfpathlineto{\pgfqpoint{2.700261in}{1.952421in}}%
\pgfpathlineto{\pgfqpoint{2.700174in}{1.958681in}}%
\pgfpathlineto{\pgfqpoint{2.700149in}{1.964942in}}%
\pgfpathlineto{\pgfqpoint{2.700185in}{1.971202in}}%
\pgfpathlineto{\pgfqpoint{2.700277in}{1.977462in}}%
\pgfpathlineto{\pgfqpoint{2.700423in}{1.983723in}}%
\pgfpathlineto{\pgfqpoint{2.700618in}{1.989983in}}%
\pgfpathlineto{\pgfqpoint{2.700860in}{1.996244in}}%
\pgfpathlineto{\pgfqpoint{2.701147in}{2.002504in}}%
\pgfpathlineto{\pgfqpoint{2.701474in}{2.008765in}}%
\pgfpathlineto{\pgfqpoint{2.701839in}{2.015025in}}%
\pgfpathlineto{\pgfqpoint{2.702240in}{2.021285in}}%
\pgfpathlineto{\pgfqpoint{2.702672in}{2.027546in}}%
\pgfpathlineto{\pgfqpoint{2.703134in}{2.033806in}}%
\pgfpathlineto{\pgfqpoint{2.703621in}{2.040067in}}%
\pgfpathlineto{\pgfqpoint{2.703673in}{2.040713in}}%
\pgfpathlineto{\pgfqpoint{2.704118in}{2.046327in}}%
\pgfpathlineto{\pgfqpoint{2.704631in}{2.052588in}}%
\pgfpathlineto{\pgfqpoint{2.705159in}{2.058848in}}%
\pgfpathlineto{\pgfqpoint{2.705698in}{2.065109in}}%
\pgfpathlineto{\pgfqpoint{2.706245in}{2.071369in}}%
\pgfpathlineto{\pgfqpoint{2.706794in}{2.077629in}}%
\pgfpathlineto{\pgfqpoint{2.707343in}{2.083890in}}%
\pgfpathlineto{\pgfqpoint{2.707885in}{2.090150in}}%
\pgfpathlineto{\pgfqpoint{2.708417in}{2.096411in}}%
\pgfpathlineto{\pgfqpoint{2.708932in}{2.102671in}}%
\pgfpathlineto{\pgfqpoint{2.709425in}{2.108932in}}%
\pgfpathlineto{\pgfqpoint{2.709889in}{2.115192in}}%
\pgfpathlineto{\pgfqpoint{2.709933in}{2.115857in}}%
\pgfpathlineto{\pgfqpoint{2.710305in}{2.121452in}}%
\pgfpathlineto{\pgfqpoint{2.710678in}{2.127713in}}%
\pgfpathlineto{\pgfqpoint{2.711002in}{2.133973in}}%
\pgfpathlineto{\pgfqpoint{2.711268in}{2.140234in}}%
\pgfpathlineto{\pgfqpoint{2.711468in}{2.146494in}}%
\pgfpathlineto{\pgfqpoint{2.711592in}{2.152755in}}%
\pgfpathlineto{\pgfqpoint{2.711630in}{2.159015in}}%
\pgfpathlineto{\pgfqpoint{2.711572in}{2.165275in}}%
\pgfpathlineto{\pgfqpoint{2.711408in}{2.171536in}}%
\pgfpathlineto{\pgfqpoint{2.711125in}{2.177796in}}%
\pgfpathlineto{\pgfqpoint{2.710713in}{2.184057in}}%
\pgfpathlineto{\pgfqpoint{2.710159in}{2.190317in}}%
\pgfpathlineto{\pgfqpoint{2.709933in}{2.192368in}}%
\pgfpathlineto{\pgfqpoint{2.709439in}{2.196578in}}%
\pgfpathlineto{\pgfqpoint{2.708544in}{2.202838in}}%
\pgfpathlineto{\pgfqpoint{2.707473in}{2.209098in}}%
\pgfpathlineto{\pgfqpoint{2.706218in}{2.215359in}}%
\pgfpathlineto{\pgfqpoint{2.704773in}{2.221619in}}%
\pgfpathlineto{\pgfqpoint{2.703673in}{2.225860in}}%
\pgfpathlineto{\pgfqpoint{2.703118in}{2.227880in}}%
\pgfpathlineto{\pgfqpoint{2.701232in}{2.234140in}}%
\pgfpathlineto{\pgfqpoint{2.699154in}{2.240401in}}%
\pgfpathlineto{\pgfqpoint{2.697412in}{2.245241in}}%
\pgfpathlineto{\pgfqpoint{2.696876in}{2.246661in}}%
\pgfpathlineto{\pgfqpoint{2.694373in}{2.252922in}}%
\pgfpathlineto{\pgfqpoint{2.691716in}{2.259182in}}%
\pgfpathlineto{\pgfqpoint{2.691152in}{2.260461in}}%
\pgfpathlineto{\pgfqpoint{2.688866in}{2.265442in}}%
\pgfpathlineto{\pgfqpoint{2.685897in}{2.271703in}}%
\pgfpathlineto{\pgfqpoint{2.684891in}{2.273781in}}%
\pgfpathlineto{\pgfqpoint{2.682801in}{2.277963in}}%
\pgfpathlineto{\pgfqpoint{2.679642in}{2.284224in}}%
\pgfpathlineto{\pgfqpoint{2.678631in}{2.286226in}}%
\pgfpathlineto{\pgfqpoint{2.676424in}{2.290484in}}%
\pgfpathlineto{\pgfqpoint{2.673219in}{2.296745in}}%
\pgfpathlineto{\pgfqpoint{2.672371in}{2.298432in}}%
\pgfpathlineto{\pgfqpoint{2.670023in}{2.303005in}}%
\pgfpathlineto{\pgfqpoint{2.666916in}{2.309265in}}%
\pgfpathlineto{\pgfqpoint{2.666110in}{2.310951in}}%
\pgfpathlineto{\pgfqpoint{2.663885in}{2.315526in}}%
\pgfpathlineto{\pgfqpoint{2.661001in}{2.321786in}}%
\pgfpathlineto{\pgfqpoint{2.659850in}{2.324437in}}%
\pgfpathlineto{\pgfqpoint{2.658259in}{2.328047in}}%
\pgfpathlineto{\pgfqpoint{2.655692in}{2.334307in}}%
\pgfpathlineto{\pgfqpoint{2.653589in}{2.339912in}}%
\pgfpathlineto{\pgfqpoint{2.653340in}{2.340568in}}%
\pgfpathlineto{\pgfqpoint{2.651151in}{2.346828in}}%
\pgfpathlineto{\pgfqpoint{2.649201in}{2.353088in}}%
\pgfpathlineto{\pgfqpoint{2.647486in}{2.359349in}}%
\pgfpathlineto{\pgfqpoint{2.647329in}{2.359997in}}%
\pgfpathlineto{\pgfqpoint{2.645953in}{2.365609in}}%
\pgfpathlineto{\pgfqpoint{2.644647in}{2.371870in}}%
\pgfpathlineto{\pgfqpoint{2.643566in}{2.378130in}}%
\pgfpathlineto{\pgfqpoint{2.642701in}{2.384391in}}%
\pgfpathlineto{\pgfqpoint{2.642040in}{2.390651in}}%
\pgfpathlineto{\pgfqpoint{2.641572in}{2.396912in}}%
\pgfpathlineto{\pgfqpoint{2.641287in}{2.403172in}}%
\pgfpathlineto{\pgfqpoint{2.641172in}{2.409432in}}%
\pgfpathlineto{\pgfqpoint{2.641218in}{2.415693in}}%
\pgfpathlineto{\pgfqpoint{2.641412in}{2.421953in}}%
\pgfpathlineto{\pgfqpoint{2.641745in}{2.428214in}}%
\pgfpathlineto{\pgfqpoint{2.642206in}{2.434474in}}%
\pgfpathlineto{\pgfqpoint{2.642785in}{2.440735in}}%
\pgfpathlineto{\pgfqpoint{2.643475in}{2.446995in}}%
\pgfpathlineto{\pgfqpoint{2.644265in}{2.453255in}}%
\pgfpathlineto{\pgfqpoint{2.645150in}{2.459516in}}%
\pgfpathlineto{\pgfqpoint{2.646120in}{2.465776in}}%
\pgfpathlineto{\pgfqpoint{2.647171in}{2.472037in}}%
\pgfpathlineto{\pgfqpoint{2.647329in}{2.472900in}}%
\pgfpathlineto{\pgfqpoint{2.648254in}{2.478297in}}%
\pgfpathlineto{\pgfqpoint{2.649394in}{2.484558in}}%
\pgfpathlineto{\pgfqpoint{2.650594in}{2.490818in}}%
\pgfpathlineto{\pgfqpoint{2.651849in}{2.497078in}}%
\pgfpathlineto{\pgfqpoint{2.653154in}{2.503339in}}%
\pgfpathlineto{\pgfqpoint{2.653589in}{2.505325in}}%
\pgfpathlineto{\pgfqpoint{2.654470in}{2.509599in}}%
\pgfpathlineto{\pgfqpoint{2.655809in}{2.515860in}}%
\pgfpathlineto{\pgfqpoint{2.657188in}{2.522120in}}%
\pgfpathlineto{\pgfqpoint{2.658602in}{2.528381in}}%
\pgfpathlineto{\pgfqpoint{2.659850in}{2.533757in}}%
\pgfpathlineto{\pgfqpoint{2.660044in}{2.534641in}}%
\pgfpathlineto{\pgfqpoint{2.661468in}{2.540902in}}%
\pgfpathlineto{\pgfqpoint{2.662922in}{2.547162in}}%
\pgfpathlineto{\pgfqpoint{2.664404in}{2.553422in}}%
\pgfpathlineto{\pgfqpoint{2.665913in}{2.559683in}}%
\pgfpathlineto{\pgfqpoint{2.666110in}{2.560474in}}%
\pgfpathlineto{\pgfqpoint{2.667401in}{2.565943in}}%
\pgfpathlineto{\pgfqpoint{2.668905in}{2.572204in}}%
\pgfpathlineto{\pgfqpoint{2.670434in}{2.578464in}}%
\pgfpathlineto{\pgfqpoint{2.671986in}{2.584725in}}%
\pgfpathlineto{\pgfqpoint{2.672371in}{2.586235in}}%
\pgfpathlineto{\pgfqpoint{2.673523in}{2.590985in}}%
\pgfpathlineto{\pgfqpoint{2.675068in}{2.597245in}}%
\pgfpathlineto{\pgfqpoint{2.676637in}{2.603506in}}%
\pgfpathlineto{\pgfqpoint{2.678230in}{2.609766in}}%
\pgfpathlineto{\pgfqpoint{2.678631in}{2.611308in}}%
\pgfpathlineto{\pgfqpoint{2.679808in}{2.616027in}}%
\pgfpathlineto{\pgfqpoint{2.681396in}{2.622287in}}%
\pgfpathlineto{\pgfqpoint{2.683009in}{2.628548in}}%
\pgfpathlineto{\pgfqpoint{2.684647in}{2.634808in}}%
\pgfpathlineto{\pgfqpoint{2.684891in}{2.635722in}}%
\pgfpathlineto{\pgfqpoint{2.686269in}{2.641068in}}%
\pgfpathlineto{\pgfqpoint{2.687908in}{2.647329in}}%
\pgfpathlineto{\pgfqpoint{2.689575in}{2.653589in}}%
\pgfpathlineto{\pgfqpoint{2.691152in}{2.659411in}}%
\pgfpathlineto{\pgfqpoint{2.691267in}{2.659850in}}%
\pgfpathlineto{\pgfqpoint{2.692942in}{2.666110in}}%
\pgfpathlineto{\pgfqpoint{2.694647in}{2.672371in}}%
\pgfpathlineto{\pgfqpoint{2.696382in}{2.678631in}}%
\pgfpathlineto{\pgfqpoint{2.697412in}{2.682273in}}%
\pgfpathlineto{\pgfqpoint{2.698130in}{2.684891in}}%
\pgfpathlineto{\pgfqpoint{2.699883in}{2.691152in}}%
\pgfpathlineto{\pgfqpoint{2.701670in}{2.697412in}}%
\pgfpathlineto{\pgfqpoint{2.703492in}{2.703673in}}%
\pgfpathlineto{\pgfqpoint{2.703673in}{2.704280in}}%
\pgfpathlineto{\pgfqpoint{2.705309in}{2.709933in}}%
\pgfpathlineto{\pgfqpoint{2.707158in}{2.716194in}}%
\pgfpathlineto{\pgfqpoint{2.709047in}{2.722454in}}%
\pgfpathlineto{\pgfqpoint{2.709933in}{2.725331in}}%
\pgfpathlineto{\pgfqpoint{2.710951in}{2.728715in}}%
\pgfpathlineto{\pgfqpoint{2.712876in}{2.734975in}}%
\pgfpathlineto{\pgfqpoint{2.714843in}{2.741235in}}%
\pgfpathlineto{\pgfqpoint{2.716194in}{2.745440in}}%
\pgfpathlineto{\pgfqpoint{2.716840in}{2.747496in}}%
\pgfpathlineto{\pgfqpoint{2.718853in}{2.753756in}}%
\pgfpathlineto{\pgfqpoint{2.720912in}{2.760017in}}%
\pgfpathlineto{\pgfqpoint{2.722454in}{2.764598in}}%
\pgfpathlineto{\pgfqpoint{2.723009in}{2.766277in}}%
\pgfpathlineto{\pgfqpoint{2.725123in}{2.772538in}}%
\pgfpathlineto{\pgfqpoint{2.727288in}{2.778798in}}%
\pgfpathlineto{\pgfqpoint{2.728715in}{2.782828in}}%
\pgfpathlineto{\pgfqpoint{2.729492in}{2.785058in}}%
\pgfpathlineto{\pgfqpoint{2.731722in}{2.791319in}}%
\pgfpathlineto{\pgfqpoint{2.734008in}{2.797579in}}%
\pgfpathlineto{\pgfqpoint{2.734975in}{2.800170in}}%
\pgfpathlineto{\pgfqpoint{2.736327in}{2.803840in}}%
\pgfpathlineto{\pgfqpoint{2.738688in}{2.810100in}}%
\pgfpathlineto{\pgfqpoint{2.741110in}{2.816361in}}%
\pgfpathlineto{\pgfqpoint{2.741235in}{2.816678in}}%
\pgfpathlineto{\pgfqpoint{2.743555in}{2.822621in}}%
\pgfpathlineto{\pgfqpoint{2.746062in}{2.828881in}}%
\pgfpathlineto{\pgfqpoint{2.747496in}{2.832379in}}%
\pgfpathlineto{\pgfqpoint{2.748619in}{2.835142in}}%
\pgfpathlineto{\pgfqpoint{2.751219in}{2.841402in}}%
\pgfpathlineto{\pgfqpoint{2.753756in}{2.847351in}}%
\pgfpathlineto{\pgfqpoint{2.753889in}{2.847663in}}%
\pgfpathlineto{\pgfqpoint{2.756591in}{2.853923in}}%
\pgfpathlineto{\pgfqpoint{2.759367in}{2.860184in}}%
\pgfpathlineto{\pgfqpoint{2.760017in}{2.861620in}}%
\pgfpathlineto{\pgfqpoint{2.762190in}{2.866444in}}%
\pgfpathlineto{\pgfqpoint{2.765080in}{2.872705in}}%
\pgfpathlineto{\pgfqpoint{2.766277in}{2.875244in}}%
\pgfpathlineto{\pgfqpoint{2.768028in}{2.878965in}}%
\pgfpathlineto{\pgfqpoint{2.771043in}{2.885225in}}%
\pgfpathlineto{\pgfqpoint{2.772538in}{2.888264in}}%
\pgfpathlineto{\pgfqpoint{2.774123in}{2.891486in}}%
\pgfpathlineto{\pgfqpoint{2.777271in}{2.897746in}}%
\pgfpathlineto{\pgfqpoint{2.778798in}{2.900720in}}%
\pgfpathlineto{\pgfqpoint{2.780490in}{2.904007in}}%
\pgfpathlineto{\pgfqpoint{2.783782in}{2.910267in}}%
\pgfpathlineto{\pgfqpoint{2.785058in}{2.912646in}}%
\pgfpathlineto{\pgfqpoint{2.787149in}{2.916528in}}%
\pgfpathlineto{\pgfqpoint{2.790597in}{2.922788in}}%
\pgfpathlineto{\pgfqpoint{2.791319in}{2.924076in}}%
\pgfpathlineto{\pgfqpoint{2.794121in}{2.929048in}}%
\pgfpathlineto{\pgfqpoint{2.797579in}{2.935037in}}%
\pgfpathlineto{\pgfqpoint{2.797737in}{2.935309in}}%
\pgfpathlineto{\pgfqpoint{2.801433in}{2.941569in}}%
\pgfpathlineto{\pgfqpoint{2.803840in}{2.945558in}}%
\pgfpathlineto{\pgfqpoint{2.805224in}{2.947830in}}%
\pgfpathlineto{\pgfqpoint{2.809109in}{2.954090in}}%
\pgfpathlineto{\pgfqpoint{2.810100in}{2.955661in}}%
\pgfpathlineto{\pgfqpoint{2.813093in}{2.960351in}}%
\pgfpathlineto{\pgfqpoint{2.816361in}{2.965367in}}%
\pgfpathlineto{\pgfqpoint{2.817182in}{2.966611in}}%
\pgfpathlineto{\pgfqpoint{2.821378in}{2.972871in}}%
\pgfpathlineto{\pgfqpoint{2.822621in}{2.974696in}}%
\pgfpathlineto{\pgfqpoint{2.825689in}{2.979132in}}%
\pgfpathlineto{\pgfqpoint{2.828881in}{2.983664in}}%
\pgfpathlineto{\pgfqpoint{2.830120in}{2.985392in}}%
\pgfpathlineto{\pgfqpoint{2.834676in}{2.991653in}}%
\pgfpathlineto{\pgfqpoint{2.835142in}{2.992286in}}%
\pgfpathlineto{\pgfqpoint{2.839368in}{2.997913in}}%
\pgfpathlineto{\pgfqpoint{2.841402in}{3.000580in}}%
\pgfpathlineto{\pgfqpoint{2.844203in}{3.004174in}}%
\pgfpathlineto{\pgfqpoint{2.847663in}{3.008551in}}%
\pgfpathlineto{\pgfqpoint{2.849187in}{3.010434in}}%
\pgfpathlineto{\pgfqpoint{2.853923in}{3.016211in}}%
\pgfpathlineto{\pgfqpoint{2.854330in}{3.016694in}}%
\pgfpathlineto{\pgfqpoint{2.859650in}{3.022955in}}%
\pgfpathlineto{\pgfqpoint{2.860184in}{3.023576in}}%
\pgfpathlineto{\pgfqpoint{2.865163in}{3.029215in}}%
\pgfpathlineto{\pgfqpoint{2.866444in}{3.030652in}}%
\pgfpathlineto{\pgfqpoint{2.870879in}{3.035476in}}%
\pgfpathlineto{\pgfqpoint{2.872705in}{3.037444in}}%
\pgfpathlineto{\pgfqpoint{2.876818in}{3.041736in}}%
\pgfpathlineto{\pgfqpoint{2.878965in}{3.043959in}}%
\pgfpathlineto{\pgfqpoint{2.883003in}{3.047997in}}%
\pgfpathlineto{\pgfqpoint{2.885225in}{3.050204in}}%
\pgfpathlineto{\pgfqpoint{2.889462in}{3.054257in}}%
\pgfpathlineto{\pgfqpoint{2.891486in}{3.056183in}}%
\pgfpathlineto{\pgfqpoint{2.896227in}{3.060518in}}%
\pgfpathlineto{\pgfqpoint{2.897746in}{3.061901in}}%
\pgfpathlineto{\pgfqpoint{2.903337in}{3.066778in}}%
\pgfpathlineto{\pgfqpoint{2.904007in}{3.067361in}}%
\pgfpathlineto{\pgfqpoint{2.910267in}{3.072569in}}%
\pgfpathlineto{\pgfqpoint{2.910859in}{3.073038in}}%
\pgfpathlineto{\pgfqpoint{2.916528in}{3.077531in}}%
\pgfpathlineto{\pgfqpoint{2.918877in}{3.079299in}}%
\pgfpathlineto{\pgfqpoint{2.922788in}{3.082242in}}%
\pgfpathlineto{\pgfqpoint{2.927448in}{3.085559in}}%
\pgfpathlineto{\pgfqpoint{2.929048in}{3.086700in}}%
\pgfpathlineto{\pgfqpoint{2.935309in}{3.090914in}}%
\pgfpathlineto{\pgfqpoint{2.936740in}{3.091820in}}%
\pgfpathlineto{\pgfqpoint{2.941569in}{3.094887in}}%
\pgfpathlineto{\pgfqpoint{2.946950in}{3.098080in}}%
\pgfpathlineto{\pgfqpoint{2.947830in}{3.098605in}}%
\pgfpathlineto{\pgfqpoint{2.954090in}{3.102085in}}%
\pgfpathlineto{\pgfqpoint{2.958478in}{3.104341in}}%
\pgfpathlineto{\pgfqpoint{2.960351in}{3.105310in}}%
\pgfpathlineto{\pgfqpoint{2.966611in}{3.108291in}}%
\pgfpathlineto{\pgfqpoint{2.971934in}{3.110601in}}%
\pgfpathlineto{\pgfqpoint{2.972871in}{3.111012in}}%
\pgfpathlineto{\pgfqpoint{2.979132in}{3.113488in}}%
\pgfpathlineto{\pgfqpoint{2.985392in}{3.115698in}}%
\pgfpathlineto{\pgfqpoint{2.989148in}{3.116861in}}%
\pgfpathlineto{\pgfqpoint{2.991653in}{3.117647in}}%
\pgfpathlineto{\pgfqpoint{2.997913in}{3.119333in}}%
\pgfpathlineto{\pgfqpoint{3.004174in}{3.120741in}}%
\pgfpathlineto{\pgfqpoint{3.010434in}{3.121869in}}%
\pgfpathlineto{\pgfqpoint{3.016694in}{3.122709in}}%
\pgfpathlineto{\pgfqpoint{3.021458in}{3.123122in}}%
\pgfpathlineto{\pgfqpoint{3.022955in}{3.123254in}}%
\pgfpathlineto{\pgfqpoint{3.029215in}{3.123496in}}%
\pgfpathlineto{\pgfqpoint{3.035476in}{3.123419in}}%
\pgfpathlineto{\pgfqpoint{3.040074in}{3.123122in}}%
\pgfpathlineto{\pgfqpoint{3.041736in}{3.123013in}}%
\pgfpathlineto{\pgfqpoint{3.047997in}{3.122265in}}%
\pgfpathlineto{\pgfqpoint{3.054257in}{3.121157in}}%
\pgfpathlineto{\pgfqpoint{3.060518in}{3.119668in}}%
\pgfpathlineto{\pgfqpoint{3.066778in}{3.117776in}}%
\pgfpathlineto{\pgfqpoint{3.069281in}{3.116861in}}%
\pgfpathlineto{\pgfqpoint{3.073038in}{3.115458in}}%
\pgfpathlineto{\pgfqpoint{3.079299in}{3.112684in}}%
\pgfpathlineto{\pgfqpoint{3.083332in}{3.110601in}}%
\pgfpathlineto{\pgfqpoint{3.085559in}{3.109418in}}%
\pgfpathlineto{\pgfqpoint{3.091820in}{3.105619in}}%
\pgfpathlineto{\pgfqpoint{3.093690in}{3.104341in}}%
\pgfpathlineto{\pgfqpoint{3.098080in}{3.101235in}}%
\pgfpathlineto{\pgfqpoint{3.102064in}{3.098080in}}%
\pgfpathlineto{\pgfqpoint{3.104341in}{3.096207in}}%
\pgfpathlineto{\pgfqpoint{3.109166in}{3.091820in}}%
\pgfpathlineto{\pgfqpoint{3.110601in}{3.090459in}}%
\pgfpathlineto{\pgfqpoint{3.115327in}{3.085559in}}%
\pgfpathlineto{\pgfqpoint{3.116861in}{3.083891in}}%
\pgfpathlineto{\pgfqpoint{3.120766in}{3.079299in}}%
\pgfpathlineto{\pgfqpoint{3.123122in}{3.076380in}}%
\pgfpathlineto{\pgfqpoint{3.125636in}{3.073038in}}%
\pgfpathlineto{\pgfqpoint{3.129382in}{3.067768in}}%
\pgfpathlineto{\pgfqpoint{3.130043in}{3.066778in}}%
\pgfpathlineto{\pgfqpoint{3.134019in}{3.060518in}}%
\pgfpathlineto{\pgfqpoint{3.135643in}{3.057803in}}%
\pgfpathlineto{\pgfqpoint{3.137653in}{3.054257in}}%
\pgfpathlineto{\pgfqpoint{3.140984in}{3.047997in}}%
\pgfpathlineto{\pgfqpoint{3.141903in}{3.046167in}}%
\pgfpathlineto{\pgfqpoint{3.144030in}{3.041736in}}%
\pgfpathlineto{\pgfqpoint{3.146836in}{3.035476in}}%
\pgfpathlineto{\pgfqpoint{3.148164in}{3.032314in}}%
\pgfpathlineto{\pgfqpoint{3.149416in}{3.029215in}}%
\pgfpathlineto{\pgfqpoint{3.151786in}{3.022955in}}%
\pgfpathlineto{\pgfqpoint{3.153977in}{3.016694in}}%
\pgfpathlineto{\pgfqpoint{3.154424in}{3.015331in}}%
\pgfpathlineto{\pgfqpoint{3.155983in}{3.010434in}}%
\pgfpathlineto{\pgfqpoint{3.157828in}{3.004174in}}%
\pgfpathlineto{\pgfqpoint{3.159523in}{2.997913in}}%
\pgfpathlineto{\pgfqpoint{3.160684in}{2.993254in}}%
\pgfpathlineto{\pgfqpoint{3.161075in}{2.991653in}}%
\pgfpathlineto{\pgfqpoint{3.162488in}{2.985392in}}%
\pgfpathlineto{\pgfqpoint{3.163774in}{2.979132in}}%
\pgfpathlineto{\pgfqpoint{3.164939in}{2.972871in}}%
\pgfpathlineto{\pgfqpoint{3.165987in}{2.966611in}}%
\pgfpathlineto{\pgfqpoint{3.166924in}{2.960351in}}%
\pgfpathlineto{\pgfqpoint{3.166945in}{2.960196in}}%
\pgfpathlineto{\pgfqpoint{3.167757in}{2.954090in}}%
\pgfpathlineto{\pgfqpoint{3.168487in}{2.947830in}}%
\pgfpathlineto{\pgfqpoint{3.169119in}{2.941569in}}%
\pgfpathlineto{\pgfqpoint{3.169657in}{2.935309in}}%
\pgfpathlineto{\pgfqpoint{3.170103in}{2.929048in}}%
\pgfpathlineto{\pgfqpoint{3.170461in}{2.922788in}}%
\pgfpathlineto{\pgfqpoint{3.170734in}{2.916528in}}%
\pgfpathlineto{\pgfqpoint{3.170923in}{2.910267in}}%
\pgfpathlineto{\pgfqpoint{3.171032in}{2.904007in}}%
\pgfpathlineto{\pgfqpoint{3.171062in}{2.897746in}}%
\pgfpathlineto{\pgfqpoint{3.171017in}{2.891486in}}%
\pgfpathlineto{\pgfqpoint{3.170896in}{2.885225in}}%
\pgfpathlineto{\pgfqpoint{3.170703in}{2.878965in}}%
\pgfpathlineto{\pgfqpoint{3.170439in}{2.872705in}}%
\pgfpathlineto{\pgfqpoint{3.170106in}{2.866444in}}%
\pgfpathlineto{\pgfqpoint{3.169703in}{2.860184in}}%
\pgfpathlineto{\pgfqpoint{3.169234in}{2.853923in}}%
\pgfpathlineto{\pgfqpoint{3.168698in}{2.847663in}}%
\pgfpathlineto{\pgfqpoint{3.168097in}{2.841402in}}%
\pgfpathlineto{\pgfqpoint{3.167431in}{2.835142in}}%
\pgfpathlineto{\pgfqpoint{3.166945in}{2.830979in}}%
\pgfpathlineto{\pgfqpoint{3.166704in}{2.828881in}}%
\pgfpathlineto{\pgfqpoint{3.165923in}{2.822621in}}%
\pgfpathlineto{\pgfqpoint{3.165080in}{2.816361in}}%
\pgfpathlineto{\pgfqpoint{3.164177in}{2.810100in}}%
\pgfpathlineto{\pgfqpoint{3.163214in}{2.803840in}}%
\pgfpathlineto{\pgfqpoint{3.162190in}{2.797579in}}%
\pgfpathlineto{\pgfqpoint{3.161106in}{2.791319in}}%
\pgfpathlineto{\pgfqpoint{3.160684in}{2.789008in}}%
\pgfpathlineto{\pgfqpoint{3.159974in}{2.785058in}}%
\pgfpathlineto{\pgfqpoint{3.158791in}{2.778798in}}%
\pgfpathlineto{\pgfqpoint{3.157550in}{2.772538in}}%
\pgfpathlineto{\pgfqpoint{3.156252in}{2.766277in}}%
\pgfpathlineto{\pgfqpoint{3.154896in}{2.760017in}}%
\pgfpathlineto{\pgfqpoint{3.154424in}{2.757923in}}%
\pgfpathlineto{\pgfqpoint{3.153497in}{2.753756in}}%
\pgfpathlineto{\pgfqpoint{3.152050in}{2.747496in}}%
\pgfpathlineto{\pgfqpoint{3.150547in}{2.741235in}}%
\pgfpathlineto{\pgfqpoint{3.148988in}{2.734975in}}%
\pgfpathlineto{\pgfqpoint{3.148164in}{2.731771in}}%
\pgfpathlineto{\pgfqpoint{3.147386in}{2.728715in}}%
\pgfpathlineto{\pgfqpoint{3.145742in}{2.722454in}}%
\pgfpathlineto{\pgfqpoint{3.144044in}{2.716194in}}%
\pgfpathlineto{\pgfqpoint{3.142291in}{2.709933in}}%
\pgfpathlineto{\pgfqpoint{3.141903in}{2.708581in}}%
\pgfpathlineto{\pgfqpoint{3.140507in}{2.703673in}}%
\pgfpathlineto{\pgfqpoint{3.138676in}{2.697412in}}%
\pgfpathlineto{\pgfqpoint{3.136793in}{2.691152in}}%
\pgfpathlineto{\pgfqpoint{3.135643in}{2.687422in}}%
\pgfpathlineto{\pgfqpoint{3.134868in}{2.684891in}}%
\pgfpathlineto{\pgfqpoint{3.132912in}{2.678631in}}%
\pgfpathlineto{\pgfqpoint{3.130904in}{2.672371in}}%
\pgfpathlineto{\pgfqpoint{3.129382in}{2.667729in}}%
\pgfpathlineto{\pgfqpoint{3.128854in}{2.666110in}}%
\pgfpathlineto{\pgfqpoint{3.126779in}{2.659850in}}%
\pgfpathlineto{\pgfqpoint{3.124657in}{2.653589in}}%
\pgfpathlineto{\pgfqpoint{3.123122in}{2.649147in}}%
\pgfpathlineto{\pgfqpoint{3.122496in}{2.647329in}}%
\pgfpathlineto{\pgfqpoint{3.120313in}{2.641068in}}%
\pgfpathlineto{\pgfqpoint{3.118085in}{2.634808in}}%
\pgfpathlineto{\pgfqpoint{3.116861in}{2.631415in}}%
\pgfpathlineto{\pgfqpoint{3.115829in}{2.628548in}}%
\pgfpathlineto{\pgfqpoint{3.113550in}{2.622287in}}%
\pgfpathlineto{\pgfqpoint{3.111231in}{2.616027in}}%
\pgfpathlineto{\pgfqpoint{3.110601in}{2.614336in}}%
\pgfpathlineto{\pgfqpoint{3.108897in}{2.609766in}}%
\pgfpathlineto{\pgfqpoint{3.106537in}{2.603506in}}%
\pgfpathlineto{\pgfqpoint{3.104341in}{2.597751in}}%
\pgfpathlineto{\pgfqpoint{3.104147in}{2.597245in}}%
\pgfpathlineto{\pgfqpoint{3.101754in}{2.590985in}}%
\pgfpathlineto{\pgfqpoint{3.099334in}{2.584725in}}%
\pgfpathlineto{\pgfqpoint{3.098080in}{2.581491in}}%
\pgfpathlineto{\pgfqpoint{3.096902in}{2.578464in}}%
\pgfpathlineto{\pgfqpoint{3.094464in}{2.572204in}}%
\pgfpathlineto{\pgfqpoint{3.092010in}{2.565943in}}%
\pgfpathlineto{\pgfqpoint{3.091820in}{2.565453in}}%
\pgfpathlineto{\pgfqpoint{3.089564in}{2.559683in}}%
\pgfpathlineto{\pgfqpoint{3.087113in}{2.553422in}}%
\pgfpathlineto{\pgfqpoint{3.085559in}{2.549438in}}%
\pgfpathlineto{\pgfqpoint{3.084664in}{2.547162in}}%
\pgfpathlineto{\pgfqpoint{3.082230in}{2.540902in}}%
\pgfpathlineto{\pgfqpoint{3.079803in}{2.534641in}}%
\pgfpathlineto{\pgfqpoint{3.079299in}{2.533316in}}%
\pgfpathlineto{\pgfqpoint{3.077401in}{2.528381in}}%
\pgfpathlineto{\pgfqpoint{3.075021in}{2.522120in}}%
\pgfpathlineto{\pgfqpoint{3.073038in}{2.516833in}}%
\pgfpathlineto{\pgfqpoint{3.072669in}{2.515860in}}%
\pgfpathlineto{\pgfqpoint{3.070359in}{2.509599in}}%
\pgfpathlineto{\pgfqpoint{3.068089in}{2.503339in}}%
\pgfpathlineto{\pgfqpoint{3.066778in}{2.499614in}}%
\pgfpathlineto{\pgfqpoint{3.065871in}{2.497078in}}%
\pgfpathlineto{\pgfqpoint{3.063715in}{2.490818in}}%
\pgfpathlineto{\pgfqpoint{3.061625in}{2.484558in}}%
\pgfpathlineto{\pgfqpoint{3.060518in}{2.481082in}}%
\pgfpathlineto{\pgfqpoint{3.059614in}{2.478297in}}%
\pgfpathlineto{\pgfqpoint{3.057691in}{2.472037in}}%
\pgfpathlineto{\pgfqpoint{3.055868in}{2.465776in}}%
\pgfpathlineto{\pgfqpoint{3.054257in}{2.459877in}}%
\pgfpathlineto{\pgfqpoint{3.054156in}{2.459516in}}%
\pgfpathlineto{\pgfqpoint{3.052568in}{2.453255in}}%
\pgfpathlineto{\pgfqpoint{3.051118in}{2.446995in}}%
\pgfpathlineto{\pgfqpoint{3.049822in}{2.440735in}}%
\pgfpathlineto{\pgfqpoint{3.048696in}{2.434474in}}%
\pgfpathlineto{\pgfqpoint{3.047997in}{2.429791in}}%
\pgfpathlineto{\pgfqpoint{3.047755in}{2.428214in}}%
\pgfpathlineto{\pgfqpoint{3.047015in}{2.421953in}}%
\pgfpathlineto{\pgfqpoint{3.046498in}{2.415693in}}%
\pgfpathlineto{\pgfqpoint{3.046224in}{2.409432in}}%
\pgfpathlineto{\pgfqpoint{3.046210in}{2.403172in}}%
\pgfpathlineto{\pgfqpoint{3.046479in}{2.396912in}}%
\pgfpathlineto{\pgfqpoint{3.047048in}{2.390651in}}%
\pgfpathlineto{\pgfqpoint{3.047938in}{2.384391in}}%
\pgfpathlineto{\pgfqpoint{3.047997in}{2.384085in}}%
\pgfpathlineto{\pgfqpoint{3.049153in}{2.378130in}}%
\pgfpathlineto{\pgfqpoint{3.050715in}{2.371870in}}%
\pgfpathlineto{\pgfqpoint{3.052638in}{2.365609in}}%
\pgfpathlineto{\pgfqpoint{3.054257in}{2.361166in}}%
\pgfpathlineto{\pgfqpoint{3.054922in}{2.359349in}}%
\pgfpathlineto{\pgfqpoint{3.057547in}{2.353088in}}%
\pgfpathlineto{\pgfqpoint{3.060518in}{2.346866in}}%
\pgfpathlineto{\pgfqpoint{3.060536in}{2.346828in}}%
\pgfpathlineto{\pgfqpoint{3.063822in}{2.340568in}}%
\pgfpathlineto{\pgfqpoint{3.066778in}{2.335439in}}%
\pgfpathlineto{\pgfqpoint{3.067425in}{2.334307in}}%
\pgfpathlineto{\pgfqpoint{3.071271in}{2.328047in}}%
\pgfpathlineto{\pgfqpoint{3.073038in}{2.325351in}}%
\pgfpathlineto{\pgfqpoint{3.075340in}{2.321786in}}%
\pgfpathlineto{\pgfqpoint{3.079299in}{2.315964in}}%
\pgfpathlineto{\pgfqpoint{3.079592in}{2.315526in}}%
\pgfpathlineto{\pgfqpoint{3.083943in}{2.309265in}}%
\pgfpathlineto{\pgfqpoint{3.085559in}{2.307017in}}%
\pgfpathlineto{\pgfqpoint{3.088373in}{2.303005in}}%
\pgfpathlineto{\pgfqpoint{3.091820in}{2.298200in}}%
\pgfpathlineto{\pgfqpoint{3.092836in}{2.296745in}}%
\pgfpathlineto{\pgfqpoint{3.097274in}{2.290484in}}%
\pgfpathlineto{\pgfqpoint{3.098080in}{2.289358in}}%
\pgfpathlineto{\pgfqpoint{3.101644in}{2.284224in}}%
\pgfpathlineto{\pgfqpoint{3.104341in}{2.280335in}}%
\pgfpathlineto{\pgfqpoint{3.105934in}{2.277963in}}%
\pgfpathlineto{\pgfqpoint{3.110106in}{2.271703in}}%
\pgfpathlineto{\pgfqpoint{3.110601in}{2.270953in}}%
\pgfpathlineto{\pgfqpoint{3.114121in}{2.265442in}}%
\pgfpathlineto{\pgfqpoint{3.116861in}{2.261052in}}%
\pgfpathlineto{\pgfqpoint{3.117991in}{2.259182in}}%
\pgfpathlineto{\pgfqpoint{3.121682in}{2.252922in}}%
\pgfpathlineto{\pgfqpoint{3.123122in}{2.250403in}}%
\pgfpathlineto{\pgfqpoint{3.125193in}{2.246661in}}%
\pgfpathlineto{\pgfqpoint{3.128522in}{2.240401in}}%
\pgfpathlineto{\pgfqpoint{3.129382in}{2.238719in}}%
\pgfpathlineto{\pgfqpoint{3.131654in}{2.234140in}}%
\pgfpathlineto{\pgfqpoint{3.134601in}{2.227880in}}%
\pgfpathlineto{\pgfqpoint{3.135643in}{2.225553in}}%
\pgfpathlineto{\pgfqpoint{3.137355in}{2.221619in}}%
\pgfpathlineto{\pgfqpoint{3.139920in}{2.215359in}}%
\pgfpathlineto{\pgfqpoint{3.141903in}{2.210172in}}%
\pgfpathlineto{\pgfqpoint{3.142303in}{2.209098in}}%
\pgfpathlineto{\pgfqpoint{3.144498in}{2.202838in}}%
\pgfpathlineto{\pgfqpoint{3.146516in}{2.196578in}}%
\pgfpathlineto{\pgfqpoint{3.148164in}{2.190999in}}%
\pgfpathlineto{\pgfqpoint{3.148361in}{2.190317in}}%
\pgfpathlineto{\pgfqpoint{3.150032in}{2.184057in}}%
\pgfpathlineto{\pgfqpoint{3.151537in}{2.177796in}}%
\pgfpathlineto{\pgfqpoint{3.152880in}{2.171536in}}%
\pgfpathlineto{\pgfqpoint{3.154064in}{2.165275in}}%
\pgfpathlineto{\pgfqpoint{3.154424in}{2.163102in}}%
\pgfpathlineto{\pgfqpoint{3.155093in}{2.159015in}}%
\pgfpathlineto{\pgfqpoint{3.155973in}{2.152755in}}%
\pgfpathlineto{\pgfqpoint{3.156705in}{2.146494in}}%
\pgfpathlineto{\pgfqpoint{3.157294in}{2.140234in}}%
\pgfpathlineto{\pgfqpoint{3.157741in}{2.133973in}}%
\pgfpathlineto{\pgfqpoint{3.158052in}{2.127713in}}%
\pgfpathlineto{\pgfqpoint{3.158228in}{2.121452in}}%
\pgfpathlineto{\pgfqpoint{3.158273in}{2.115192in}}%
\pgfpathlineto{\pgfqpoint{3.158189in}{2.108932in}}%
\pgfpathlineto{\pgfqpoint{3.157978in}{2.102671in}}%
\pgfpathlineto{\pgfqpoint{3.157643in}{2.096411in}}%
\pgfpathlineto{\pgfqpoint{3.157185in}{2.090150in}}%
\pgfpathlineto{\pgfqpoint{3.156606in}{2.083890in}}%
\pgfpathlineto{\pgfqpoint{3.155908in}{2.077629in}}%
\pgfpathlineto{\pgfqpoint{3.155090in}{2.071369in}}%
\pgfpathlineto{\pgfqpoint{3.154424in}{2.066917in}}%
\pgfpathlineto{\pgfqpoint{3.154157in}{2.065109in}}%
\pgfpathlineto{\pgfqpoint{3.153116in}{2.058848in}}%
\pgfpathlineto{\pgfqpoint{3.151960in}{2.052588in}}%
\pgfpathlineto{\pgfqpoint{3.150688in}{2.046327in}}%
\pgfpathlineto{\pgfqpoint{3.149300in}{2.040067in}}%
\pgfpathlineto{\pgfqpoint{3.148164in}{2.035336in}}%
\pgfpathlineto{\pgfqpoint{3.147800in}{2.033806in}}%
\pgfpathlineto{\pgfqpoint{3.146200in}{2.027546in}}%
\pgfpathlineto{\pgfqpoint{3.144483in}{2.021285in}}%
\pgfpathlineto{\pgfqpoint{3.142649in}{2.015025in}}%
\pgfpathlineto{\pgfqpoint{3.141903in}{2.012627in}}%
\pgfpathlineto{\pgfqpoint{3.140712in}{2.008765in}}%
\pgfpathlineto{\pgfqpoint{3.138666in}{2.002504in}}%
\pgfpathlineto{\pgfqpoint{3.136500in}{1.996244in}}%
\pgfpathlineto{\pgfqpoint{3.135643in}{1.993889in}}%
\pgfpathlineto{\pgfqpoint{3.134230in}{1.989983in}}%
\pgfpathlineto{\pgfqpoint{3.131848in}{1.983723in}}%
\pgfpathlineto{\pgfqpoint{3.129382in}{1.977564in}}%
\pgfpathlineto{\pgfqpoint{3.129342in}{1.977462in}}%
\pgfpathlineto{\pgfqpoint{3.126739in}{1.971202in}}%
\pgfpathlineto{\pgfqpoint{3.124008in}{1.964942in}}%
\pgfpathlineto{\pgfqpoint{3.123122in}{1.962990in}}%
\pgfpathlineto{\pgfqpoint{3.121169in}{1.958681in}}%
\pgfpathlineto{\pgfqpoint{3.118207in}{1.952421in}}%
\pgfpathlineto{\pgfqpoint{3.116861in}{1.949686in}}%
\pgfpathlineto{\pgfqpoint{3.115127in}{1.946160in}}%
\pgfpathlineto{\pgfqpoint{3.111924in}{1.939900in}}%
\pgfpathlineto{\pgfqpoint{3.110601in}{1.937405in}}%
\pgfpathlineto{\pgfqpoint{3.108600in}{1.933639in}}%
\pgfpathlineto{\pgfqpoint{3.105144in}{1.927379in}}%
\pgfpathlineto{\pgfqpoint{3.104341in}{1.925968in}}%
\pgfpathlineto{\pgfqpoint{3.101566in}{1.921119in}}%
\pgfpathlineto{\pgfqpoint{3.098080in}{1.915249in}}%
\pgfpathlineto{\pgfqpoint{3.097846in}{1.914858in}}%
\pgfpathlineto{\pgfqpoint{3.094000in}{1.908598in}}%
\pgfpathlineto{\pgfqpoint{3.091820in}{1.905165in}}%
\pgfpathlineto{\pgfqpoint{3.090009in}{1.902337in}}%
\pgfpathlineto{\pgfqpoint{3.085871in}{1.896077in}}%
\pgfpathlineto{\pgfqpoint{3.085559in}{1.895616in}}%
\pgfpathlineto{\pgfqpoint{3.081590in}{1.889816in}}%
\pgfpathlineto{\pgfqpoint{3.079299in}{1.886567in}}%
\pgfpathlineto{\pgfqpoint{3.077149in}{1.883556in}}%
\pgfpathlineto{\pgfqpoint{3.073038in}{1.877956in}}%
\pgfpathlineto{\pgfqpoint{3.072547in}{1.877295in}}%
\pgfpathlineto{\pgfqpoint{3.067779in}{1.871035in}}%
\pgfpathlineto{\pgfqpoint{3.066778in}{1.869752in}}%
\pgfpathlineto{\pgfqpoint{3.062833in}{1.864775in}}%
\pgfpathlineto{\pgfqpoint{3.060518in}{1.861922in}}%
\pgfpathlineto{\pgfqpoint{3.057701in}{1.858514in}}%
\pgfpathlineto{\pgfqpoint{3.054257in}{1.854438in}}%
\pgfpathlineto{\pgfqpoint{3.052374in}{1.852254in}}%
\pgfpathlineto{\pgfqpoint{3.047997in}{1.847279in}}%
\pgfpathlineto{\pgfqpoint{3.046840in}{1.845993in}}%
\pgfpathlineto{\pgfqpoint{3.041736in}{1.840425in}}%
\pgfpathlineto{\pgfqpoint{3.041086in}{1.839733in}}%
\pgfpathlineto{\pgfqpoint{3.035476in}{1.833861in}}%
\pgfpathlineto{\pgfqpoint{3.035095in}{1.833472in}}%
\pgfpathlineto{\pgfqpoint{3.029215in}{1.827571in}}%
\pgfpathlineto{\pgfqpoint{3.028848in}{1.827212in}}%
\pgfpathlineto{\pgfqpoint{3.022955in}{1.821542in}}%
\pgfpathlineto{\pgfqpoint{3.022322in}{1.820952in}}%
\pgfpathlineto{\pgfqpoint{3.016694in}{1.815764in}}%
\pgfpathlineto{\pgfqpoint{3.015491in}{1.814691in}}%
\pgfpathlineto{\pgfqpoint{3.010434in}{1.810229in}}%
\pgfpathlineto{\pgfqpoint{3.008324in}{1.808431in}}%
\pgfpathlineto{\pgfqpoint{3.004174in}{1.804927in}}%
\pgfpathlineto{\pgfqpoint{3.000783in}{1.802170in}}%
\pgfpathlineto{\pgfqpoint{2.997913in}{1.799854in}}%
\pgfpathlineto{\pgfqpoint{2.992828in}{1.795910in}}%
\pgfpathlineto{\pgfqpoint{2.991653in}{1.795004in}}%
\pgfpathlineto{\pgfqpoint{2.985392in}{1.790369in}}%
\pgfpathlineto{\pgfqpoint{2.984379in}{1.789649in}}%
\pgfpathlineto{\pgfqpoint{2.979132in}{1.785939in}}%
\pgfpathlineto{\pgfqpoint{2.975349in}{1.783389in}}%
\pgfpathlineto{\pgfqpoint{2.972871in}{1.781722in}}%
\pgfpathlineto{\pgfqpoint{2.966611in}{1.777711in}}%
\pgfpathlineto{\pgfqpoint{2.965655in}{1.777129in}}%
\pgfpathlineto{\pgfqpoint{2.960351in}{1.773893in}}%
\pgfpathlineto{\pgfqpoint{2.955101in}{1.770868in}}%
\pgfpathlineto{\pgfqpoint{2.954090in}{1.770285in}}%
\pgfpathlineto{\pgfqpoint{2.947830in}{1.766865in}}%
\pgfpathlineto{\pgfqpoint{2.943431in}{1.764608in}}%
\pgfpathlineto{\pgfqpoint{2.941569in}{1.763648in}}%
\pgfpathlineto{\pgfqpoint{2.935309in}{1.760622in}}%
\pgfpathlineto{\pgfqpoint{2.930257in}{1.758347in}}%
\pgfpathlineto{\pgfqpoint{2.929048in}{1.757800in}}%
\pgfpathlineto{\pgfqpoint{2.922788in}{1.755163in}}%
\pgfpathlineto{\pgfqpoint{2.916528in}{1.752734in}}%
\pgfpathlineto{\pgfqpoint{2.914706in}{1.752087in}}%
\pgfpathlineto{\pgfqpoint{2.910267in}{1.750495in}}%
\pgfpathlineto{\pgfqpoint{2.904007in}{1.748458in}}%
\pgfpathlineto{\pgfqpoint{2.897746in}{1.746628in}}%
\pgfpathlineto{\pgfqpoint{2.894646in}{1.745826in}}%
\pgfpathlineto{\pgfqpoint{2.891486in}{1.744999in}}%
\pgfpathlineto{\pgfqpoint{2.885225in}{1.743576in}}%
\pgfpathlineto{\pgfqpoint{2.878965in}{1.742369in}}%
\pgfpathlineto{\pgfqpoint{2.872705in}{1.741380in}}%
\pgfpathlineto{\pgfqpoint{2.866444in}{1.740614in}}%
\pgfpathlineto{\pgfqpoint{2.860184in}{1.740076in}}%
\pgfpathlineto{\pgfqpoint{2.853923in}{1.739774in}}%
\pgfpathlineto{\pgfqpoint{2.847663in}{1.739715in}}%
\pgfpathlineto{\pgfqpoint{2.841402in}{1.739908in}}%
\pgfpathlineto{\pgfqpoint{2.835142in}{1.740365in}}%
\pgfpathlineto{\pgfqpoint{2.828881in}{1.741098in}}%
\pgfpathlineto{\pgfqpoint{2.822621in}{1.742121in}}%
\pgfpathlineto{\pgfqpoint{2.816361in}{1.743450in}}%
\pgfpathlineto{\pgfqpoint{2.810100in}{1.745101in}}%
\pgfpathclose%
\pgfpathmoveto{\pgfqpoint{1.934187in}{1.958681in}}%
\pgfpathlineto{\pgfqpoint{1.933639in}{1.958992in}}%
\pgfpathlineto{\pgfqpoint{1.927379in}{1.963806in}}%
\pgfpathlineto{\pgfqpoint{1.926253in}{1.964942in}}%
\pgfpathlineto{\pgfqpoint{1.922828in}{1.971202in}}%
\pgfpathlineto{\pgfqpoint{1.921888in}{1.977462in}}%
\pgfpathlineto{\pgfqpoint{1.922804in}{1.983723in}}%
\pgfpathlineto{\pgfqpoint{1.925143in}{1.989983in}}%
\pgfpathlineto{\pgfqpoint{1.927379in}{1.993929in}}%
\pgfpathlineto{\pgfqpoint{1.928704in}{1.996244in}}%
\pgfpathlineto{\pgfqpoint{1.933396in}{2.002504in}}%
\pgfpathlineto{\pgfqpoint{1.933639in}{2.002768in}}%
\pgfpathlineto{\pgfqpoint{1.939240in}{2.008765in}}%
\pgfpathlineto{\pgfqpoint{1.939900in}{2.009368in}}%
\pgfpathlineto{\pgfqpoint{1.946154in}{2.015025in}}%
\pgfpathlineto{\pgfqpoint{1.946160in}{2.015030in}}%
\pgfpathlineto{\pgfqpoint{1.952421in}{2.019845in}}%
\pgfpathlineto{\pgfqpoint{1.954316in}{2.021285in}}%
\pgfpathlineto{\pgfqpoint{1.958681in}{2.024219in}}%
\pgfpathlineto{\pgfqpoint{1.963690in}{2.027546in}}%
\pgfpathlineto{\pgfqpoint{1.964942in}{2.028292in}}%
\pgfpathlineto{\pgfqpoint{1.971202in}{2.031893in}}%
\pgfpathlineto{\pgfqpoint{1.974606in}{2.033806in}}%
\pgfpathlineto{\pgfqpoint{1.977462in}{2.035261in}}%
\pgfpathlineto{\pgfqpoint{1.983723in}{2.038331in}}%
\pgfpathlineto{\pgfqpoint{1.987389in}{2.040067in}}%
\pgfpathlineto{\pgfqpoint{1.989983in}{2.041187in}}%
\pgfpathlineto{\pgfqpoint{1.996244in}{2.043744in}}%
\pgfpathlineto{\pgfqpoint{2.002504in}{2.046213in}}%
\pgfpathlineto{\pgfqpoint{2.002820in}{2.046327in}}%
\pgfpathlineto{\pgfqpoint{2.008765in}{2.048288in}}%
\pgfpathlineto{\pgfqpoint{2.015025in}{2.050221in}}%
\pgfpathlineto{\pgfqpoint{2.021285in}{2.052009in}}%
\pgfpathlineto{\pgfqpoint{2.023597in}{2.052588in}}%
\pgfpathlineto{\pgfqpoint{2.027546in}{2.053495in}}%
\pgfpathlineto{\pgfqpoint{2.033806in}{2.054720in}}%
\pgfpathlineto{\pgfqpoint{2.040067in}{2.055728in}}%
\pgfpathlineto{\pgfqpoint{2.046327in}{2.056483in}}%
\pgfpathlineto{\pgfqpoint{2.052588in}{2.056939in}}%
\pgfpathlineto{\pgfqpoint{2.058848in}{2.057037in}}%
\pgfpathlineto{\pgfqpoint{2.065109in}{2.056698in}}%
\pgfpathlineto{\pgfqpoint{2.071369in}{2.055819in}}%
\pgfpathlineto{\pgfqpoint{2.077629in}{2.054257in}}%
\pgfpathlineto{\pgfqpoint{2.082029in}{2.052588in}}%
\pgfpathlineto{\pgfqpoint{2.083890in}{2.051562in}}%
\pgfpathlineto{\pgfqpoint{2.090150in}{2.046577in}}%
\pgfpathlineto{\pgfqpoint{2.090386in}{2.046327in}}%
\pgfpathlineto{\pgfqpoint{2.093595in}{2.040067in}}%
\pgfpathlineto{\pgfqpoint{2.094345in}{2.033806in}}%
\pgfpathlineto{\pgfqpoint{2.093341in}{2.027546in}}%
\pgfpathlineto{\pgfqpoint{2.091025in}{2.021285in}}%
\pgfpathlineto{\pgfqpoint{2.090150in}{2.019738in}}%
\pgfpathlineto{\pgfqpoint{2.087488in}{2.015025in}}%
\pgfpathlineto{\pgfqpoint{2.083890in}{2.010039in}}%
\pgfpathlineto{\pgfqpoint{2.082969in}{2.008765in}}%
\pgfpathlineto{\pgfqpoint{2.077629in}{2.002715in}}%
\pgfpathlineto{\pgfqpoint{2.077443in}{2.002504in}}%
\pgfpathlineto{\pgfqpoint{2.071369in}{1.996711in}}%
\pgfpathlineto{\pgfqpoint{2.070877in}{1.996244in}}%
\pgfpathlineto{\pgfqpoint{2.065109in}{1.991513in}}%
\pgfpathlineto{\pgfqpoint{2.063230in}{1.989983in}}%
\pgfpathlineto{\pgfqpoint{2.058848in}{1.986844in}}%
\pgfpathlineto{\pgfqpoint{2.054453in}{1.983723in}}%
\pgfpathlineto{\pgfqpoint{2.052588in}{1.982541in}}%
\pgfpathlineto{\pgfqpoint{2.046327in}{1.978676in}}%
\pgfpathlineto{\pgfqpoint{2.044305in}{1.977462in}}%
\pgfpathlineto{\pgfqpoint{2.040067in}{1.975178in}}%
\pgfpathlineto{\pgfqpoint{2.033806in}{1.971883in}}%
\pgfpathlineto{\pgfqpoint{2.032446in}{1.971202in}}%
\pgfpathlineto{\pgfqpoint{2.027546in}{1.968986in}}%
\pgfpathlineto{\pgfqpoint{2.021285in}{1.966263in}}%
\pgfpathlineto{\pgfqpoint{2.018067in}{1.964942in}}%
\pgfpathlineto{\pgfqpoint{2.015025in}{1.963808in}}%
\pgfpathlineto{\pgfqpoint{2.008765in}{1.961647in}}%
\pgfpathlineto{\pgfqpoint{2.002504in}{1.959630in}}%
\pgfpathlineto{\pgfqpoint{1.999228in}{1.958681in}}%
\pgfpathlineto{\pgfqpoint{1.996244in}{1.957896in}}%
\pgfpathlineto{\pgfqpoint{1.989983in}{1.956470in}}%
\pgfpathlineto{\pgfqpoint{1.983723in}{1.955254in}}%
\pgfpathlineto{\pgfqpoint{1.977462in}{1.954282in}}%
\pgfpathlineto{\pgfqpoint{1.971202in}{1.953589in}}%
\pgfpathlineto{\pgfqpoint{1.964942in}{1.953222in}}%
\pgfpathlineto{\pgfqpoint{1.958681in}{1.953238in}}%
\pgfpathlineto{\pgfqpoint{1.952421in}{1.953710in}}%
\pgfpathlineto{\pgfqpoint{1.946160in}{1.954729in}}%
\pgfpathlineto{\pgfqpoint{1.939900in}{1.956413in}}%
\pgfpathclose%
\pgfpathmoveto{\pgfqpoint{2.245802in}{2.465776in}}%
\pgfpathlineto{\pgfqpoint{2.240401in}{2.466552in}}%
\pgfpathlineto{\pgfqpoint{2.234140in}{2.467650in}}%
\pgfpathlineto{\pgfqpoint{2.227880in}{2.468924in}}%
\pgfpathlineto{\pgfqpoint{2.221619in}{2.470358in}}%
\pgfpathlineto{\pgfqpoint{2.215359in}{2.471937in}}%
\pgfpathlineto{\pgfqpoint{2.215006in}{2.472037in}}%
\pgfpathlineto{\pgfqpoint{2.209098in}{2.473651in}}%
\pgfpathlineto{\pgfqpoint{2.202838in}{2.475485in}}%
\pgfpathlineto{\pgfqpoint{2.196578in}{2.477431in}}%
\pgfpathlineto{\pgfqpoint{2.193978in}{2.478297in}}%
\pgfpathlineto{\pgfqpoint{2.190317in}{2.479482in}}%
\pgfpathlineto{\pgfqpoint{2.184057in}{2.481627in}}%
\pgfpathlineto{\pgfqpoint{2.177796in}{2.483862in}}%
\pgfpathlineto{\pgfqpoint{2.175952in}{2.484558in}}%
\pgfpathlineto{\pgfqpoint{2.171536in}{2.486181in}}%
\pgfpathlineto{\pgfqpoint{2.165275in}{2.488576in}}%
\pgfpathlineto{\pgfqpoint{2.159605in}{2.490818in}}%
\pgfpathlineto{\pgfqpoint{2.159015in}{2.491046in}}%
\pgfpathlineto{\pgfqpoint{2.152755in}{2.493584in}}%
\pgfpathlineto{\pgfqpoint{2.146494in}{2.496188in}}%
\pgfpathlineto{\pgfqpoint{2.144432in}{2.497078in}}%
\pgfpathlineto{\pgfqpoint{2.140234in}{2.498855in}}%
\pgfpathlineto{\pgfqpoint{2.133973in}{2.501580in}}%
\pgfpathlineto{\pgfqpoint{2.130045in}{2.503339in}}%
\pgfpathlineto{\pgfqpoint{2.127713in}{2.504364in}}%
\pgfpathlineto{\pgfqpoint{2.121452in}{2.507202in}}%
\pgfpathlineto{\pgfqpoint{2.116277in}{2.509599in}}%
\pgfpathlineto{\pgfqpoint{2.115192in}{2.510094in}}%
\pgfpathlineto{\pgfqpoint{2.108932in}{2.513037in}}%
\pgfpathlineto{\pgfqpoint{2.103034in}{2.515860in}}%
\pgfpathlineto{\pgfqpoint{2.102671in}{2.516031in}}%
\pgfpathlineto{\pgfqpoint{2.096411in}{2.519076in}}%
\pgfpathlineto{\pgfqpoint{2.090246in}{2.522120in}}%
\pgfpathlineto{\pgfqpoint{2.090150in}{2.522167in}}%
\pgfpathlineto{\pgfqpoint{2.083890in}{2.525309in}}%
\pgfpathlineto{\pgfqpoint{2.077856in}{2.528381in}}%
\pgfpathlineto{\pgfqpoint{2.077629in}{2.528495in}}%
\pgfpathlineto{\pgfqpoint{2.071369in}{2.531732in}}%
\pgfpathlineto{\pgfqpoint{2.065823in}{2.534641in}}%
\pgfpathlineto{\pgfqpoint{2.065109in}{2.535012in}}%
\pgfpathlineto{\pgfqpoint{2.058848in}{2.538343in}}%
\pgfpathlineto{\pgfqpoint{2.054111in}{2.540902in}}%
\pgfpathlineto{\pgfqpoint{2.052588in}{2.541718in}}%
\pgfpathlineto{\pgfqpoint{2.046327in}{2.545142in}}%
\pgfpathlineto{\pgfqpoint{2.042694in}{2.547162in}}%
\pgfpathlineto{\pgfqpoint{2.040067in}{2.548613in}}%
\pgfpathlineto{\pgfqpoint{2.033806in}{2.552131in}}%
\pgfpathlineto{\pgfqpoint{2.031548in}{2.553422in}}%
\pgfpathlineto{\pgfqpoint{2.027546in}{2.555700in}}%
\pgfpathlineto{\pgfqpoint{2.021285in}{2.559312in}}%
\pgfpathlineto{\pgfqpoint{2.020656in}{2.559683in}}%
\pgfpathlineto{\pgfqpoint{2.015025in}{2.562982in}}%
\pgfpathlineto{\pgfqpoint{2.010035in}{2.565943in}}%
\pgfpathlineto{\pgfqpoint{2.008765in}{2.566695in}}%
\pgfpathlineto{\pgfqpoint{2.002504in}{2.570464in}}%
\pgfpathlineto{\pgfqpoint{1.999655in}{2.572204in}}%
\pgfpathlineto{\pgfqpoint{1.996244in}{2.574285in}}%
\pgfpathlineto{\pgfqpoint{1.989983in}{2.578153in}}%
\pgfpathlineto{\pgfqpoint{1.989488in}{2.578464in}}%
\pgfpathlineto{\pgfqpoint{1.983723in}{2.582084in}}%
\pgfpathlineto{\pgfqpoint{1.979569in}{2.584725in}}%
\pgfpathlineto{\pgfqpoint{1.977462in}{2.586065in}}%
\pgfpathlineto{\pgfqpoint{1.971202in}{2.590101in}}%
\pgfpathlineto{\pgfqpoint{1.969851in}{2.590985in}}%
\pgfpathlineto{\pgfqpoint{1.964942in}{2.594201in}}%
\pgfpathlineto{\pgfqpoint{1.960349in}{2.597245in}}%
\pgfpathlineto{\pgfqpoint{1.958681in}{2.598354in}}%
\pgfpathlineto{\pgfqpoint{1.952421in}{2.602569in}}%
\pgfpathlineto{\pgfqpoint{1.951047in}{2.603506in}}%
\pgfpathlineto{\pgfqpoint{1.946160in}{2.606851in}}%
\pgfpathlineto{\pgfqpoint{1.941948in}{2.609766in}}%
\pgfpathlineto{\pgfqpoint{1.939900in}{2.611191in}}%
\pgfpathlineto{\pgfqpoint{1.933639in}{2.615594in}}%
\pgfpathlineto{\pgfqpoint{1.933031in}{2.616027in}}%
\pgfpathlineto{\pgfqpoint{1.927379in}{2.620074in}}%
\pgfpathlineto{\pgfqpoint{1.924319in}{2.622287in}}%
\pgfpathlineto{\pgfqpoint{1.921119in}{2.624618in}}%
\pgfpathlineto{\pgfqpoint{1.915775in}{2.628548in}}%
\pgfpathlineto{\pgfqpoint{1.914858in}{2.629227in}}%
\pgfpathlineto{\pgfqpoint{1.908598in}{2.633916in}}%
\pgfpathlineto{\pgfqpoint{1.907418in}{2.634808in}}%
\pgfpathlineto{\pgfqpoint{1.902337in}{2.638684in}}%
\pgfpathlineto{\pgfqpoint{1.899239in}{2.641068in}}%
\pgfpathlineto{\pgfqpoint{1.896077in}{2.643525in}}%
\pgfpathlineto{\pgfqpoint{1.891223in}{2.647329in}}%
\pgfpathlineto{\pgfqpoint{1.889816in}{2.648443in}}%
\pgfpathlineto{\pgfqpoint{1.883556in}{2.653442in}}%
\pgfpathlineto{\pgfqpoint{1.883373in}{2.653589in}}%
\pgfpathlineto{\pgfqpoint{1.877295in}{2.658540in}}%
\pgfpathlineto{\pgfqpoint{1.875699in}{2.659850in}}%
\pgfpathlineto{\pgfqpoint{1.871035in}{2.663724in}}%
\pgfpathlineto{\pgfqpoint{1.868183in}{2.666110in}}%
\pgfpathlineto{\pgfqpoint{1.864775in}{2.668999in}}%
\pgfpathlineto{\pgfqpoint{1.860823in}{2.672371in}}%
\pgfpathlineto{\pgfqpoint{1.858514in}{2.674367in}}%
\pgfpathlineto{\pgfqpoint{1.853616in}{2.678631in}}%
\pgfpathlineto{\pgfqpoint{1.852254in}{2.679835in}}%
\pgfpathlineto{\pgfqpoint{1.846563in}{2.684891in}}%
\pgfpathlineto{\pgfqpoint{1.845993in}{2.685406in}}%
\pgfpathlineto{\pgfqpoint{1.839733in}{2.691087in}}%
\pgfpathlineto{\pgfqpoint{1.839662in}{2.691152in}}%
\pgfpathlineto{\pgfqpoint{1.833472in}{2.696892in}}%
\pgfpathlineto{\pgfqpoint{1.832914in}{2.697412in}}%
\pgfpathlineto{\pgfqpoint{1.827212in}{2.702818in}}%
\pgfpathlineto{\pgfqpoint{1.826314in}{2.703673in}}%
\pgfpathlineto{\pgfqpoint{1.820952in}{2.708870in}}%
\pgfpathlineto{\pgfqpoint{1.819860in}{2.709933in}}%
\pgfpathlineto{\pgfqpoint{1.814691in}{2.715057in}}%
\pgfpathlineto{\pgfqpoint{1.813549in}{2.716194in}}%
\pgfpathlineto{\pgfqpoint{1.808431in}{2.721387in}}%
\pgfpathlineto{\pgfqpoint{1.807382in}{2.722454in}}%
\pgfpathlineto{\pgfqpoint{1.802170in}{2.727868in}}%
\pgfpathlineto{\pgfqpoint{1.801358in}{2.728715in}}%
\pgfpathlineto{\pgfqpoint{1.795910in}{2.734511in}}%
\pgfpathlineto{\pgfqpoint{1.795475in}{2.734975in}}%
\pgfpathlineto{\pgfqpoint{1.789735in}{2.741235in}}%
\pgfpathlineto{\pgfqpoint{1.789649in}{2.741331in}}%
\pgfpathlineto{\pgfqpoint{1.784140in}{2.747496in}}%
\pgfpathlineto{\pgfqpoint{1.783389in}{2.748357in}}%
\pgfpathlineto{\pgfqpoint{1.778687in}{2.753756in}}%
\pgfpathlineto{\pgfqpoint{1.777129in}{2.755592in}}%
\pgfpathlineto{\pgfqpoint{1.773375in}{2.760017in}}%
\pgfpathlineto{\pgfqpoint{1.770868in}{2.763051in}}%
\pgfpathlineto{\pgfqpoint{1.768204in}{2.766277in}}%
\pgfpathlineto{\pgfqpoint{1.764608in}{2.770753in}}%
\pgfpathlineto{\pgfqpoint{1.763174in}{2.772538in}}%
\pgfpathlineto{\pgfqpoint{1.758347in}{2.778719in}}%
\pgfpathlineto{\pgfqpoint{1.758285in}{2.778798in}}%
\pgfpathlineto{\pgfqpoint{1.753553in}{2.785058in}}%
\pgfpathlineto{\pgfqpoint{1.752087in}{2.787061in}}%
\pgfpathlineto{\pgfqpoint{1.748965in}{2.791319in}}%
\pgfpathlineto{\pgfqpoint{1.745826in}{2.795743in}}%
\pgfpathlineto{\pgfqpoint{1.744521in}{2.797579in}}%
\pgfpathlineto{\pgfqpoint{1.740229in}{2.803840in}}%
\pgfpathlineto{\pgfqpoint{1.739566in}{2.804848in}}%
\pgfpathlineto{\pgfqpoint{1.736099in}{2.810100in}}%
\pgfpathlineto{\pgfqpoint{1.733306in}{2.814497in}}%
\pgfpathlineto{\pgfqpoint{1.732117in}{2.816361in}}%
\pgfpathlineto{\pgfqpoint{1.728300in}{2.822621in}}%
\pgfpathlineto{\pgfqpoint{1.727045in}{2.824779in}}%
\pgfpathlineto{\pgfqpoint{1.724650in}{2.828881in}}%
\pgfpathlineto{\pgfqpoint{1.721158in}{2.835142in}}%
\pgfpathlineto{\pgfqpoint{1.720785in}{2.835853in}}%
\pgfpathlineto{\pgfqpoint{1.717856in}{2.841402in}}%
\pgfpathlineto{\pgfqpoint{1.714716in}{2.847663in}}%
\pgfpathlineto{\pgfqpoint{1.714524in}{2.848073in}}%
\pgfpathlineto{\pgfqpoint{1.711780in}{2.853923in}}%
\pgfpathlineto{\pgfqpoint{1.709023in}{2.860184in}}%
\pgfpathlineto{\pgfqpoint{1.708264in}{2.862058in}}%
\pgfpathlineto{\pgfqpoint{1.706477in}{2.866444in}}%
\pgfpathlineto{\pgfqpoint{1.704140in}{2.872705in}}%
\pgfpathlineto{\pgfqpoint{1.702011in}{2.878965in}}%
\pgfpathlineto{\pgfqpoint{1.702003in}{2.878989in}}%
\pgfpathlineto{\pgfqpoint{1.700139in}{2.885225in}}%
\pgfpathlineto{\pgfqpoint{1.698507in}{2.891486in}}%
\pgfpathlineto{\pgfqpoint{1.697133in}{2.897746in}}%
\pgfpathlineto{\pgfqpoint{1.696038in}{2.904007in}}%
\pgfpathlineto{\pgfqpoint{1.695743in}{2.906395in}}%
\pgfpathlineto{\pgfqpoint{1.695260in}{2.910267in}}%
\pgfpathlineto{\pgfqpoint{1.694819in}{2.916528in}}%
\pgfpathlineto{\pgfqpoint{1.694742in}{2.922788in}}%
\pgfpathlineto{\pgfqpoint{1.695066in}{2.929048in}}%
\pgfpathlineto{\pgfqpoint{1.695743in}{2.934549in}}%
\pgfpathlineto{\pgfqpoint{1.695843in}{2.935309in}}%
\pgfpathlineto{\pgfqpoint{1.697164in}{2.941569in}}%
\pgfpathlineto{\pgfqpoint{1.699081in}{2.947830in}}%
\pgfpathlineto{\pgfqpoint{1.701678in}{2.954090in}}%
\pgfpathlineto{\pgfqpoint{1.702003in}{2.954720in}}%
\pgfpathlineto{\pgfqpoint{1.705200in}{2.960351in}}%
\pgfpathlineto{\pgfqpoint{1.708264in}{2.964681in}}%
\pgfpathlineto{\pgfqpoint{1.709788in}{2.966611in}}%
\pgfpathlineto{\pgfqpoint{1.714524in}{2.971626in}}%
\pgfpathlineto{\pgfqpoint{1.715859in}{2.972871in}}%
\pgfpathlineto{\pgfqpoint{1.720785in}{2.976828in}}%
\pgfpathlineto{\pgfqpoint{1.724105in}{2.979132in}}%
\pgfpathlineto{\pgfqpoint{1.727045in}{2.980927in}}%
\pgfpathlineto{\pgfqpoint{1.733306in}{2.984200in}}%
\pgfpathlineto{\pgfqpoint{1.735995in}{2.985392in}}%
\pgfpathlineto{\pgfqpoint{1.739566in}{2.986818in}}%
\pgfpathlineto{\pgfqpoint{1.745826in}{2.988908in}}%
\pgfpathlineto{\pgfqpoint{1.752087in}{2.990592in}}%
\pgfpathlineto{\pgfqpoint{1.757070in}{2.991653in}}%
\pgfpathlineto{\pgfqpoint{1.758347in}{2.991903in}}%
\pgfpathlineto{\pgfqpoint{1.764608in}{2.992862in}}%
\pgfpathlineto{\pgfqpoint{1.770868in}{2.993545in}}%
\pgfpathlineto{\pgfqpoint{1.777129in}{2.993977in}}%
\pgfpathlineto{\pgfqpoint{1.783389in}{2.994181in}}%
\pgfpathlineto{\pgfqpoint{1.789649in}{2.994177in}}%
\pgfpathlineto{\pgfqpoint{1.795910in}{2.993979in}}%
\pgfpathlineto{\pgfqpoint{1.802170in}{2.993604in}}%
\pgfpathlineto{\pgfqpoint{1.808431in}{2.993065in}}%
\pgfpathlineto{\pgfqpoint{1.814691in}{2.992372in}}%
\pgfpathlineto{\pgfqpoint{1.820063in}{2.991653in}}%
\pgfpathlineto{\pgfqpoint{1.820952in}{2.991533in}}%
\pgfpathlineto{\pgfqpoint{1.827212in}{2.990540in}}%
\pgfpathlineto{\pgfqpoint{1.833472in}{2.989420in}}%
\pgfpathlineto{\pgfqpoint{1.839733in}{2.988180in}}%
\pgfpathlineto{\pgfqpoint{1.845993in}{2.986827in}}%
\pgfpathlineto{\pgfqpoint{1.852142in}{2.985392in}}%
\pgfpathlineto{\pgfqpoint{1.852254in}{2.985366in}}%
\pgfpathlineto{\pgfqpoint{1.858514in}{2.983773in}}%
\pgfpathlineto{\pgfqpoint{1.864775in}{2.982083in}}%
\pgfpathlineto{\pgfqpoint{1.871035in}{2.980300in}}%
\pgfpathlineto{\pgfqpoint{1.874918in}{2.979132in}}%
\pgfpathlineto{\pgfqpoint{1.877295in}{2.978415in}}%
\pgfpathlineto{\pgfqpoint{1.883556in}{2.976425in}}%
\pgfpathlineto{\pgfqpoint{1.889816in}{2.974353in}}%
\pgfpathlineto{\pgfqpoint{1.894108in}{2.972871in}}%
\pgfpathlineto{\pgfqpoint{1.896077in}{2.972190in}}%
\pgfpathlineto{\pgfqpoint{1.902337in}{2.969928in}}%
\pgfpathlineto{\pgfqpoint{1.908598in}{2.967594in}}%
\pgfpathlineto{\pgfqpoint{1.911135in}{2.966611in}}%
\pgfpathlineto{\pgfqpoint{1.914858in}{2.965166in}}%
\pgfpathlineto{\pgfqpoint{1.921119in}{2.962656in}}%
\pgfpathlineto{\pgfqpoint{1.926711in}{2.960351in}}%
\pgfpathlineto{\pgfqpoint{1.927379in}{2.960075in}}%
\pgfpathlineto{\pgfqpoint{1.933639in}{2.957396in}}%
\pgfpathlineto{\pgfqpoint{1.939900in}{2.954655in}}%
\pgfpathlineto{\pgfqpoint{1.941152in}{2.954090in}}%
\pgfpathlineto{\pgfqpoint{1.946160in}{2.951822in}}%
\pgfpathlineto{\pgfqpoint{1.952421in}{2.948923in}}%
\pgfpathlineto{\pgfqpoint{1.954722in}{2.947830in}}%
\pgfpathlineto{\pgfqpoint{1.958681in}{2.945941in}}%
\pgfpathlineto{\pgfqpoint{1.964942in}{2.942889in}}%
\pgfpathlineto{\pgfqpoint{1.967585in}{2.941569in}}%
\pgfpathlineto{\pgfqpoint{1.971202in}{2.939757in}}%
\pgfpathlineto{\pgfqpoint{1.977462in}{2.936556in}}%
\pgfpathlineto{\pgfqpoint{1.979849in}{2.935309in}}%
\pgfpathlineto{\pgfqpoint{1.983723in}{2.933275in}}%
\pgfpathlineto{\pgfqpoint{1.989983in}{2.929928in}}%
\pgfpathlineto{\pgfqpoint{1.991595in}{2.929048in}}%
\pgfpathlineto{\pgfqpoint{1.996244in}{2.926499in}}%
\pgfpathlineto{\pgfqpoint{2.002504in}{2.923009in}}%
\pgfpathlineto{\pgfqpoint{2.002892in}{2.922788in}}%
\pgfpathlineto{\pgfqpoint{2.008765in}{2.919429in}}%
\pgfpathlineto{\pgfqpoint{2.013762in}{2.916528in}}%
\pgfpathlineto{\pgfqpoint{2.015025in}{2.915790in}}%
\pgfpathlineto{\pgfqpoint{2.021285in}{2.912068in}}%
\pgfpathlineto{\pgfqpoint{2.024268in}{2.910267in}}%
\pgfpathlineto{\pgfqpoint{2.027546in}{2.908275in}}%
\pgfpathlineto{\pgfqpoint{2.033806in}{2.904415in}}%
\pgfpathlineto{\pgfqpoint{2.034458in}{2.904007in}}%
\pgfpathlineto{\pgfqpoint{2.040067in}{2.900467in}}%
\pgfpathlineto{\pgfqpoint{2.044322in}{2.897746in}}%
\pgfpathlineto{\pgfqpoint{2.046327in}{2.896454in}}%
\pgfpathlineto{\pgfqpoint{2.052588in}{2.892365in}}%
\pgfpathlineto{\pgfqpoint{2.053914in}{2.891486in}}%
\pgfpathlineto{\pgfqpoint{2.058848in}{2.888191in}}%
\pgfpathlineto{\pgfqpoint{2.063236in}{2.885225in}}%
\pgfpathlineto{\pgfqpoint{2.065109in}{2.883949in}}%
\pgfpathlineto{\pgfqpoint{2.071369in}{2.879628in}}%
\pgfpathlineto{\pgfqpoint{2.072318in}{2.878965in}}%
\pgfpathlineto{\pgfqpoint{2.077629in}{2.875218in}}%
\pgfpathlineto{\pgfqpoint{2.081155in}{2.872705in}}%
\pgfpathlineto{\pgfqpoint{2.083890in}{2.870735in}}%
\pgfpathlineto{\pgfqpoint{2.089787in}{2.866444in}}%
\pgfpathlineto{\pgfqpoint{2.090150in}{2.866177in}}%
\pgfpathlineto{\pgfqpoint{2.096411in}{2.861522in}}%
\pgfpathlineto{\pgfqpoint{2.098193in}{2.860184in}}%
\pgfpathlineto{\pgfqpoint{2.102671in}{2.856783in}}%
\pgfpathlineto{\pgfqpoint{2.106404in}{2.853923in}}%
\pgfpathlineto{\pgfqpoint{2.108932in}{2.851963in}}%
\pgfpathlineto{\pgfqpoint{2.114430in}{2.847663in}}%
\pgfpathlineto{\pgfqpoint{2.115192in}{2.847059in}}%
\pgfpathlineto{\pgfqpoint{2.121452in}{2.842058in}}%
\pgfpathlineto{\pgfqpoint{2.122266in}{2.841402in}}%
\pgfpathlineto{\pgfqpoint{2.127713in}{2.836956in}}%
\pgfpathlineto{\pgfqpoint{2.129919in}{2.835142in}}%
\pgfpathlineto{\pgfqpoint{2.133973in}{2.831762in}}%
\pgfpathlineto{\pgfqpoint{2.137405in}{2.828881in}}%
\pgfpathlineto{\pgfqpoint{2.140234in}{2.826472in}}%
\pgfpathlineto{\pgfqpoint{2.144726in}{2.822621in}}%
\pgfpathlineto{\pgfqpoint{2.146494in}{2.821083in}}%
\pgfpathlineto{\pgfqpoint{2.151888in}{2.816361in}}%
\pgfpathlineto{\pgfqpoint{2.152755in}{2.815590in}}%
\pgfpathlineto{\pgfqpoint{2.158892in}{2.810100in}}%
\pgfpathlineto{\pgfqpoint{2.159015in}{2.809989in}}%
\pgfpathlineto{\pgfqpoint{2.165275in}{2.804265in}}%
\pgfpathlineto{\pgfqpoint{2.165738in}{2.803840in}}%
\pgfpathlineto{\pgfqpoint{2.171536in}{2.798422in}}%
\pgfpathlineto{\pgfqpoint{2.172434in}{2.797579in}}%
\pgfpathlineto{\pgfqpoint{2.177796in}{2.792458in}}%
\pgfpathlineto{\pgfqpoint{2.178984in}{2.791319in}}%
\pgfpathlineto{\pgfqpoint{2.184057in}{2.786365in}}%
\pgfpathlineto{\pgfqpoint{2.185390in}{2.785058in}}%
\pgfpathlineto{\pgfqpoint{2.190317in}{2.780139in}}%
\pgfpathlineto{\pgfqpoint{2.191656in}{2.778798in}}%
\pgfpathlineto{\pgfqpoint{2.196578in}{2.773772in}}%
\pgfpathlineto{\pgfqpoint{2.197783in}{2.772538in}}%
\pgfpathlineto{\pgfqpoint{2.202838in}{2.767257in}}%
\pgfpathlineto{\pgfqpoint{2.203774in}{2.766277in}}%
\pgfpathlineto{\pgfqpoint{2.209098in}{2.760586in}}%
\pgfpathlineto{\pgfqpoint{2.209631in}{2.760017in}}%
\pgfpathlineto{\pgfqpoint{2.215355in}{2.753756in}}%
\pgfpathlineto{\pgfqpoint{2.215359in}{2.753752in}}%
\pgfpathlineto{\pgfqpoint{2.220942in}{2.747496in}}%
\pgfpathlineto{\pgfqpoint{2.221619in}{2.746719in}}%
\pgfpathlineto{\pgfqpoint{2.226400in}{2.741235in}}%
\pgfpathlineto{\pgfqpoint{2.227880in}{2.739497in}}%
\pgfpathlineto{\pgfqpoint{2.231730in}{2.734975in}}%
\pgfpathlineto{\pgfqpoint{2.234140in}{2.732073in}}%
\pgfpathlineto{\pgfqpoint{2.236933in}{2.728715in}}%
\pgfpathlineto{\pgfqpoint{2.240401in}{2.724437in}}%
\pgfpathlineto{\pgfqpoint{2.242011in}{2.722454in}}%
\pgfpathlineto{\pgfqpoint{2.246661in}{2.716578in}}%
\pgfpathlineto{\pgfqpoint{2.246966in}{2.716194in}}%
\pgfpathlineto{\pgfqpoint{2.251788in}{2.709933in}}%
\pgfpathlineto{\pgfqpoint{2.252922in}{2.708418in}}%
\pgfpathlineto{\pgfqpoint{2.256485in}{2.703673in}}%
\pgfpathlineto{\pgfqpoint{2.259182in}{2.699978in}}%
\pgfpathlineto{\pgfqpoint{2.261062in}{2.697412in}}%
\pgfpathlineto{\pgfqpoint{2.265442in}{2.691259in}}%
\pgfpathlineto{\pgfqpoint{2.265519in}{2.691152in}}%
\pgfpathlineto{\pgfqpoint{2.269841in}{2.684891in}}%
\pgfpathlineto{\pgfqpoint{2.271703in}{2.682108in}}%
\pgfpathlineto{\pgfqpoint{2.274043in}{2.678631in}}%
\pgfpathlineto{\pgfqpoint{2.277963in}{2.672621in}}%
\pgfpathlineto{\pgfqpoint{2.278128in}{2.672371in}}%
\pgfpathlineto{\pgfqpoint{2.282075in}{2.666110in}}%
\pgfpathlineto{\pgfqpoint{2.284224in}{2.662585in}}%
\pgfpathlineto{\pgfqpoint{2.285904in}{2.659850in}}%
\pgfpathlineto{\pgfqpoint{2.289608in}{2.653589in}}%
\pgfpathlineto{\pgfqpoint{2.290484in}{2.652042in}}%
\pgfpathlineto{\pgfqpoint{2.293178in}{2.647329in}}%
\pgfpathlineto{\pgfqpoint{2.296631in}{2.641068in}}%
\pgfpathlineto{\pgfqpoint{2.296745in}{2.640851in}}%
\pgfpathlineto{\pgfqpoint{2.299937in}{2.634808in}}%
\pgfpathlineto{\pgfqpoint{2.303005in}{2.628784in}}%
\pgfpathlineto{\pgfqpoint{2.303127in}{2.628548in}}%
\pgfpathlineto{\pgfqpoint{2.306164in}{2.622287in}}%
\pgfpathlineto{\pgfqpoint{2.309081in}{2.616027in}}%
\pgfpathlineto{\pgfqpoint{2.309265in}{2.615602in}}%
\pgfpathlineto{\pgfqpoint{2.311839in}{2.609766in}}%
\pgfpathlineto{\pgfqpoint{2.314463in}{2.603506in}}%
\pgfpathlineto{\pgfqpoint{2.315526in}{2.600786in}}%
\pgfpathlineto{\pgfqpoint{2.316932in}{2.597245in}}%
\pgfpathlineto{\pgfqpoint{2.319242in}{2.590985in}}%
\pgfpathlineto{\pgfqpoint{2.321402in}{2.584725in}}%
\pgfpathlineto{\pgfqpoint{2.321786in}{2.583489in}}%
\pgfpathlineto{\pgfqpoint{2.323377in}{2.578464in}}%
\pgfpathlineto{\pgfqpoint{2.325176in}{2.572204in}}%
\pgfpathlineto{\pgfqpoint{2.326793in}{2.565943in}}%
\pgfpathlineto{\pgfqpoint{2.328047in}{2.560402in}}%
\pgfpathlineto{\pgfqpoint{2.328213in}{2.559683in}}%
\pgfpathlineto{\pgfqpoint{2.329399in}{2.553422in}}%
\pgfpathlineto{\pgfqpoint{2.330356in}{2.547162in}}%
\pgfpathlineto{\pgfqpoint{2.331060in}{2.540902in}}%
\pgfpathlineto{\pgfqpoint{2.331485in}{2.534641in}}%
\pgfpathlineto{\pgfqpoint{2.331598in}{2.528381in}}%
\pgfpathlineto{\pgfqpoint{2.331358in}{2.522120in}}%
\pgfpathlineto{\pgfqpoint{2.330715in}{2.515860in}}%
\pgfpathlineto{\pgfqpoint{2.329609in}{2.509599in}}%
\pgfpathlineto{\pgfqpoint{2.328047in}{2.503644in}}%
\pgfpathlineto{\pgfqpoint{2.327957in}{2.503339in}}%
\pgfpathlineto{\pgfqpoint{2.325559in}{2.497078in}}%
\pgfpathlineto{\pgfqpoint{2.322335in}{2.490818in}}%
\pgfpathlineto{\pgfqpoint{2.321786in}{2.489944in}}%
\pgfpathlineto{\pgfqpoint{2.317845in}{2.484558in}}%
\pgfpathlineto{\pgfqpoint{2.315526in}{2.481965in}}%
\pgfpathlineto{\pgfqpoint{2.311577in}{2.478297in}}%
\pgfpathlineto{\pgfqpoint{2.309265in}{2.476465in}}%
\pgfpathlineto{\pgfqpoint{2.303005in}{2.472485in}}%
\pgfpathlineto{\pgfqpoint{2.302122in}{2.472037in}}%
\pgfpathlineto{\pgfqpoint{2.296745in}{2.469605in}}%
\pgfpathlineto{\pgfqpoint{2.290484in}{2.467499in}}%
\pgfpathlineto{\pgfqpoint{2.284224in}{2.466015in}}%
\pgfpathlineto{\pgfqpoint{2.282749in}{2.465776in}}%
\pgfpathlineto{\pgfqpoint{2.277963in}{2.465061in}}%
\pgfpathlineto{\pgfqpoint{2.271703in}{2.464536in}}%
\pgfpathlineto{\pgfqpoint{2.265442in}{2.464376in}}%
\pgfpathlineto{\pgfqpoint{2.259182in}{2.464533in}}%
\pgfpathlineto{\pgfqpoint{2.252922in}{2.464968in}}%
\pgfpathlineto{\pgfqpoint{2.246661in}{2.465649in}}%
\pgfpathclose%
\pgfusepath{fill}%
\end{pgfscope}%
\begin{pgfscope}%
\pgfpathrectangle{\pgfqpoint{0.500000in}{0.500000in}}{\pgfqpoint{3.750000in}{3.750000in}}%
\pgfusepath{clip}%
\pgfsetbuttcap%
\pgfsetroundjoin%
\definecolor{currentfill}{rgb}{0.451088,0.661207,0.810842}%
\pgfsetfillcolor{currentfill}%
\pgfsetlinewidth{0.000000pt}%
\definecolor{currentstroke}{rgb}{0.000000,0.000000,0.000000}%
\pgfsetstrokecolor{currentstroke}%
\pgfsetdash{}{0pt}%
\pgfpathmoveto{\pgfqpoint{2.810100in}{1.745101in}}%
\pgfpathlineto{\pgfqpoint{2.816361in}{1.743450in}}%
\pgfpathlineto{\pgfqpoint{2.822621in}{1.742121in}}%
\pgfpathlineto{\pgfqpoint{2.828881in}{1.741098in}}%
\pgfpathlineto{\pgfqpoint{2.835142in}{1.740365in}}%
\pgfpathlineto{\pgfqpoint{2.841402in}{1.739908in}}%
\pgfpathlineto{\pgfqpoint{2.847663in}{1.739715in}}%
\pgfpathlineto{\pgfqpoint{2.853923in}{1.739774in}}%
\pgfpathlineto{\pgfqpoint{2.860184in}{1.740076in}}%
\pgfpathlineto{\pgfqpoint{2.866444in}{1.740614in}}%
\pgfpathlineto{\pgfqpoint{2.872705in}{1.741380in}}%
\pgfpathlineto{\pgfqpoint{2.878965in}{1.742369in}}%
\pgfpathlineto{\pgfqpoint{2.885225in}{1.743576in}}%
\pgfpathlineto{\pgfqpoint{2.891486in}{1.744999in}}%
\pgfpathlineto{\pgfqpoint{2.894646in}{1.745826in}}%
\pgfpathlineto{\pgfqpoint{2.897746in}{1.746628in}}%
\pgfpathlineto{\pgfqpoint{2.904007in}{1.748458in}}%
\pgfpathlineto{\pgfqpoint{2.910267in}{1.750495in}}%
\pgfpathlineto{\pgfqpoint{2.914706in}{1.752087in}}%
\pgfpathlineto{\pgfqpoint{2.916528in}{1.752734in}}%
\pgfpathlineto{\pgfqpoint{2.922788in}{1.755163in}}%
\pgfpathlineto{\pgfqpoint{2.929048in}{1.757800in}}%
\pgfpathlineto{\pgfqpoint{2.930257in}{1.758347in}}%
\pgfpathlineto{\pgfqpoint{2.935309in}{1.760622in}}%
\pgfpathlineto{\pgfqpoint{2.941569in}{1.763648in}}%
\pgfpathlineto{\pgfqpoint{2.943431in}{1.764608in}}%
\pgfpathlineto{\pgfqpoint{2.947830in}{1.766865in}}%
\pgfpathlineto{\pgfqpoint{2.954090in}{1.770285in}}%
\pgfpathlineto{\pgfqpoint{2.955101in}{1.770868in}}%
\pgfpathlineto{\pgfqpoint{2.960351in}{1.773893in}}%
\pgfpathlineto{\pgfqpoint{2.965655in}{1.777129in}}%
\pgfpathlineto{\pgfqpoint{2.966611in}{1.777711in}}%
\pgfpathlineto{\pgfqpoint{2.972871in}{1.781722in}}%
\pgfpathlineto{\pgfqpoint{2.975349in}{1.783389in}}%
\pgfpathlineto{\pgfqpoint{2.979132in}{1.785939in}}%
\pgfpathlineto{\pgfqpoint{2.984379in}{1.789649in}}%
\pgfpathlineto{\pgfqpoint{2.985392in}{1.790369in}}%
\pgfpathlineto{\pgfqpoint{2.991653in}{1.795004in}}%
\pgfpathlineto{\pgfqpoint{2.992828in}{1.795910in}}%
\pgfpathlineto{\pgfqpoint{2.997913in}{1.799854in}}%
\pgfpathlineto{\pgfqpoint{3.000783in}{1.802170in}}%
\pgfpathlineto{\pgfqpoint{3.004174in}{1.804927in}}%
\pgfpathlineto{\pgfqpoint{3.008324in}{1.808431in}}%
\pgfpathlineto{\pgfqpoint{3.010434in}{1.810229in}}%
\pgfpathlineto{\pgfqpoint{3.015491in}{1.814691in}}%
\pgfpathlineto{\pgfqpoint{3.016694in}{1.815764in}}%
\pgfpathlineto{\pgfqpoint{3.022322in}{1.820952in}}%
\pgfpathlineto{\pgfqpoint{3.022955in}{1.821542in}}%
\pgfpathlineto{\pgfqpoint{3.028848in}{1.827212in}}%
\pgfpathlineto{\pgfqpoint{3.029215in}{1.827571in}}%
\pgfpathlineto{\pgfqpoint{3.035095in}{1.833472in}}%
\pgfpathlineto{\pgfqpoint{3.035476in}{1.833861in}}%
\pgfpathlineto{\pgfqpoint{3.041086in}{1.839733in}}%
\pgfpathlineto{\pgfqpoint{3.041736in}{1.840425in}}%
\pgfpathlineto{\pgfqpoint{3.046840in}{1.845993in}}%
\pgfpathlineto{\pgfqpoint{3.047997in}{1.847279in}}%
\pgfpathlineto{\pgfqpoint{3.052374in}{1.852254in}}%
\pgfpathlineto{\pgfqpoint{3.054257in}{1.854438in}}%
\pgfpathlineto{\pgfqpoint{3.057701in}{1.858514in}}%
\pgfpathlineto{\pgfqpoint{3.060518in}{1.861922in}}%
\pgfpathlineto{\pgfqpoint{3.062833in}{1.864775in}}%
\pgfpathlineto{\pgfqpoint{3.066778in}{1.869752in}}%
\pgfpathlineto{\pgfqpoint{3.067779in}{1.871035in}}%
\pgfpathlineto{\pgfqpoint{3.072547in}{1.877295in}}%
\pgfpathlineto{\pgfqpoint{3.073038in}{1.877956in}}%
\pgfpathlineto{\pgfqpoint{3.077149in}{1.883556in}}%
\pgfpathlineto{\pgfqpoint{3.079299in}{1.886567in}}%
\pgfpathlineto{\pgfqpoint{3.081590in}{1.889816in}}%
\pgfpathlineto{\pgfqpoint{3.085559in}{1.895616in}}%
\pgfpathlineto{\pgfqpoint{3.085871in}{1.896077in}}%
\pgfpathlineto{\pgfqpoint{3.090009in}{1.902337in}}%
\pgfpathlineto{\pgfqpoint{3.091820in}{1.905165in}}%
\pgfpathlineto{\pgfqpoint{3.094000in}{1.908598in}}%
\pgfpathlineto{\pgfqpoint{3.097846in}{1.914858in}}%
\pgfpathlineto{\pgfqpoint{3.098080in}{1.915249in}}%
\pgfpathlineto{\pgfqpoint{3.101566in}{1.921119in}}%
\pgfpathlineto{\pgfqpoint{3.104341in}{1.925968in}}%
\pgfpathlineto{\pgfqpoint{3.105144in}{1.927379in}}%
\pgfpathlineto{\pgfqpoint{3.108600in}{1.933639in}}%
\pgfpathlineto{\pgfqpoint{3.110601in}{1.937405in}}%
\pgfpathlineto{\pgfqpoint{3.111924in}{1.939900in}}%
\pgfpathlineto{\pgfqpoint{3.115127in}{1.946160in}}%
\pgfpathlineto{\pgfqpoint{3.116861in}{1.949686in}}%
\pgfpathlineto{\pgfqpoint{3.118207in}{1.952421in}}%
\pgfpathlineto{\pgfqpoint{3.121169in}{1.958681in}}%
\pgfpathlineto{\pgfqpoint{3.123122in}{1.962990in}}%
\pgfpathlineto{\pgfqpoint{3.124008in}{1.964942in}}%
\pgfpathlineto{\pgfqpoint{3.126739in}{1.971202in}}%
\pgfpathlineto{\pgfqpoint{3.129342in}{1.977462in}}%
\pgfpathlineto{\pgfqpoint{3.129382in}{1.977564in}}%
\pgfpathlineto{\pgfqpoint{3.131848in}{1.983723in}}%
\pgfpathlineto{\pgfqpoint{3.134230in}{1.989983in}}%
\pgfpathlineto{\pgfqpoint{3.135643in}{1.993889in}}%
\pgfpathlineto{\pgfqpoint{3.136500in}{1.996244in}}%
\pgfpathlineto{\pgfqpoint{3.138666in}{2.002504in}}%
\pgfpathlineto{\pgfqpoint{3.140712in}{2.008765in}}%
\pgfpathlineto{\pgfqpoint{3.141903in}{2.012627in}}%
\pgfpathlineto{\pgfqpoint{3.142649in}{2.015025in}}%
\pgfpathlineto{\pgfqpoint{3.144483in}{2.021285in}}%
\pgfpathlineto{\pgfqpoint{3.146200in}{2.027546in}}%
\pgfpathlineto{\pgfqpoint{3.147800in}{2.033806in}}%
\pgfpathlineto{\pgfqpoint{3.148164in}{2.035336in}}%
\pgfpathlineto{\pgfqpoint{3.149300in}{2.040067in}}%
\pgfpathlineto{\pgfqpoint{3.150688in}{2.046327in}}%
\pgfpathlineto{\pgfqpoint{3.151960in}{2.052588in}}%
\pgfpathlineto{\pgfqpoint{3.153116in}{2.058848in}}%
\pgfpathlineto{\pgfqpoint{3.154157in}{2.065109in}}%
\pgfpathlineto{\pgfqpoint{3.154424in}{2.066917in}}%
\pgfpathlineto{\pgfqpoint{3.155090in}{2.071369in}}%
\pgfpathlineto{\pgfqpoint{3.155908in}{2.077629in}}%
\pgfpathlineto{\pgfqpoint{3.156606in}{2.083890in}}%
\pgfpathlineto{\pgfqpoint{3.157185in}{2.090150in}}%
\pgfpathlineto{\pgfqpoint{3.157643in}{2.096411in}}%
\pgfpathlineto{\pgfqpoint{3.157978in}{2.102671in}}%
\pgfpathlineto{\pgfqpoint{3.158189in}{2.108932in}}%
\pgfpathlineto{\pgfqpoint{3.158273in}{2.115192in}}%
\pgfpathlineto{\pgfqpoint{3.158228in}{2.121452in}}%
\pgfpathlineto{\pgfqpoint{3.158052in}{2.127713in}}%
\pgfpathlineto{\pgfqpoint{3.157741in}{2.133973in}}%
\pgfpathlineto{\pgfqpoint{3.157294in}{2.140234in}}%
\pgfpathlineto{\pgfqpoint{3.156705in}{2.146494in}}%
\pgfpathlineto{\pgfqpoint{3.155973in}{2.152755in}}%
\pgfpathlineto{\pgfqpoint{3.155093in}{2.159015in}}%
\pgfpathlineto{\pgfqpoint{3.154424in}{2.163102in}}%
\pgfpathlineto{\pgfqpoint{3.154064in}{2.165275in}}%
\pgfpathlineto{\pgfqpoint{3.152880in}{2.171536in}}%
\pgfpathlineto{\pgfqpoint{3.151537in}{2.177796in}}%
\pgfpathlineto{\pgfqpoint{3.150032in}{2.184057in}}%
\pgfpathlineto{\pgfqpoint{3.148361in}{2.190317in}}%
\pgfpathlineto{\pgfqpoint{3.148164in}{2.190999in}}%
\pgfpathlineto{\pgfqpoint{3.146516in}{2.196578in}}%
\pgfpathlineto{\pgfqpoint{3.144498in}{2.202838in}}%
\pgfpathlineto{\pgfqpoint{3.142303in}{2.209098in}}%
\pgfpathlineto{\pgfqpoint{3.141903in}{2.210172in}}%
\pgfpathlineto{\pgfqpoint{3.139920in}{2.215359in}}%
\pgfpathlineto{\pgfqpoint{3.137355in}{2.221619in}}%
\pgfpathlineto{\pgfqpoint{3.135643in}{2.225553in}}%
\pgfpathlineto{\pgfqpoint{3.134601in}{2.227880in}}%
\pgfpathlineto{\pgfqpoint{3.131654in}{2.234140in}}%
\pgfpathlineto{\pgfqpoint{3.129382in}{2.238719in}}%
\pgfpathlineto{\pgfqpoint{3.128522in}{2.240401in}}%
\pgfpathlineto{\pgfqpoint{3.125193in}{2.246661in}}%
\pgfpathlineto{\pgfqpoint{3.123122in}{2.250403in}}%
\pgfpathlineto{\pgfqpoint{3.121682in}{2.252922in}}%
\pgfpathlineto{\pgfqpoint{3.117991in}{2.259182in}}%
\pgfpathlineto{\pgfqpoint{3.116861in}{2.261052in}}%
\pgfpathlineto{\pgfqpoint{3.114121in}{2.265442in}}%
\pgfpathlineto{\pgfqpoint{3.110601in}{2.270953in}}%
\pgfpathlineto{\pgfqpoint{3.110106in}{2.271703in}}%
\pgfpathlineto{\pgfqpoint{3.105934in}{2.277963in}}%
\pgfpathlineto{\pgfqpoint{3.104341in}{2.280335in}}%
\pgfpathlineto{\pgfqpoint{3.101644in}{2.284224in}}%
\pgfpathlineto{\pgfqpoint{3.098080in}{2.289358in}}%
\pgfpathlineto{\pgfqpoint{3.097274in}{2.290484in}}%
\pgfpathlineto{\pgfqpoint{3.092836in}{2.296745in}}%
\pgfpathlineto{\pgfqpoint{3.091820in}{2.298200in}}%
\pgfpathlineto{\pgfqpoint{3.088373in}{2.303005in}}%
\pgfpathlineto{\pgfqpoint{3.085559in}{2.307017in}}%
\pgfpathlineto{\pgfqpoint{3.083943in}{2.309265in}}%
\pgfpathlineto{\pgfqpoint{3.079592in}{2.315526in}}%
\pgfpathlineto{\pgfqpoint{3.079299in}{2.315964in}}%
\pgfpathlineto{\pgfqpoint{3.075340in}{2.321786in}}%
\pgfpathlineto{\pgfqpoint{3.073038in}{2.325351in}}%
\pgfpathlineto{\pgfqpoint{3.071271in}{2.328047in}}%
\pgfpathlineto{\pgfqpoint{3.067425in}{2.334307in}}%
\pgfpathlineto{\pgfqpoint{3.066778in}{2.335439in}}%
\pgfpathlineto{\pgfqpoint{3.063822in}{2.340568in}}%
\pgfpathlineto{\pgfqpoint{3.060536in}{2.346828in}}%
\pgfpathlineto{\pgfqpoint{3.060518in}{2.346866in}}%
\pgfpathlineto{\pgfqpoint{3.057547in}{2.353088in}}%
\pgfpathlineto{\pgfqpoint{3.054922in}{2.359349in}}%
\pgfpathlineto{\pgfqpoint{3.054257in}{2.361166in}}%
\pgfpathlineto{\pgfqpoint{3.052638in}{2.365609in}}%
\pgfpathlineto{\pgfqpoint{3.050715in}{2.371870in}}%
\pgfpathlineto{\pgfqpoint{3.049153in}{2.378130in}}%
\pgfpathlineto{\pgfqpoint{3.047997in}{2.384085in}}%
\pgfpathlineto{\pgfqpoint{3.047938in}{2.384391in}}%
\pgfpathlineto{\pgfqpoint{3.047048in}{2.390651in}}%
\pgfpathlineto{\pgfqpoint{3.046479in}{2.396912in}}%
\pgfpathlineto{\pgfqpoint{3.046210in}{2.403172in}}%
\pgfpathlineto{\pgfqpoint{3.046224in}{2.409432in}}%
\pgfpathlineto{\pgfqpoint{3.046498in}{2.415693in}}%
\pgfpathlineto{\pgfqpoint{3.047015in}{2.421953in}}%
\pgfpathlineto{\pgfqpoint{3.047755in}{2.428214in}}%
\pgfpathlineto{\pgfqpoint{3.047997in}{2.429791in}}%
\pgfpathlineto{\pgfqpoint{3.048696in}{2.434474in}}%
\pgfpathlineto{\pgfqpoint{3.049822in}{2.440735in}}%
\pgfpathlineto{\pgfqpoint{3.051118in}{2.446995in}}%
\pgfpathlineto{\pgfqpoint{3.052568in}{2.453255in}}%
\pgfpathlineto{\pgfqpoint{3.054156in}{2.459516in}}%
\pgfpathlineto{\pgfqpoint{3.054257in}{2.459877in}}%
\pgfpathlineto{\pgfqpoint{3.055868in}{2.465776in}}%
\pgfpathlineto{\pgfqpoint{3.057691in}{2.472037in}}%
\pgfpathlineto{\pgfqpoint{3.059614in}{2.478297in}}%
\pgfpathlineto{\pgfqpoint{3.060518in}{2.481082in}}%
\pgfpathlineto{\pgfqpoint{3.061625in}{2.484558in}}%
\pgfpathlineto{\pgfqpoint{3.063715in}{2.490818in}}%
\pgfpathlineto{\pgfqpoint{3.065871in}{2.497078in}}%
\pgfpathlineto{\pgfqpoint{3.066778in}{2.499614in}}%
\pgfpathlineto{\pgfqpoint{3.068089in}{2.503339in}}%
\pgfpathlineto{\pgfqpoint{3.070359in}{2.509599in}}%
\pgfpathlineto{\pgfqpoint{3.072669in}{2.515860in}}%
\pgfpathlineto{\pgfqpoint{3.073038in}{2.516833in}}%
\pgfpathlineto{\pgfqpoint{3.075021in}{2.522120in}}%
\pgfpathlineto{\pgfqpoint{3.077401in}{2.528381in}}%
\pgfpathlineto{\pgfqpoint{3.079299in}{2.533316in}}%
\pgfpathlineto{\pgfqpoint{3.079803in}{2.534641in}}%
\pgfpathlineto{\pgfqpoint{3.082230in}{2.540902in}}%
\pgfpathlineto{\pgfqpoint{3.084664in}{2.547162in}}%
\pgfpathlineto{\pgfqpoint{3.085559in}{2.549438in}}%
\pgfpathlineto{\pgfqpoint{3.087113in}{2.553422in}}%
\pgfpathlineto{\pgfqpoint{3.089564in}{2.559683in}}%
\pgfpathlineto{\pgfqpoint{3.091820in}{2.565453in}}%
\pgfpathlineto{\pgfqpoint{3.092010in}{2.565943in}}%
\pgfpathlineto{\pgfqpoint{3.094464in}{2.572204in}}%
\pgfpathlineto{\pgfqpoint{3.096902in}{2.578464in}}%
\pgfpathlineto{\pgfqpoint{3.098080in}{2.581491in}}%
\pgfpathlineto{\pgfqpoint{3.099334in}{2.584725in}}%
\pgfpathlineto{\pgfqpoint{3.101754in}{2.590985in}}%
\pgfpathlineto{\pgfqpoint{3.104147in}{2.597245in}}%
\pgfpathlineto{\pgfqpoint{3.104341in}{2.597751in}}%
\pgfpathlineto{\pgfqpoint{3.106537in}{2.603506in}}%
\pgfpathlineto{\pgfqpoint{3.108897in}{2.609766in}}%
\pgfpathlineto{\pgfqpoint{3.110601in}{2.614336in}}%
\pgfpathlineto{\pgfqpoint{3.111231in}{2.616027in}}%
\pgfpathlineto{\pgfqpoint{3.113550in}{2.622287in}}%
\pgfpathlineto{\pgfqpoint{3.115829in}{2.628548in}}%
\pgfpathlineto{\pgfqpoint{3.116861in}{2.631415in}}%
\pgfpathlineto{\pgfqpoint{3.118085in}{2.634808in}}%
\pgfpathlineto{\pgfqpoint{3.120313in}{2.641068in}}%
\pgfpathlineto{\pgfqpoint{3.122496in}{2.647329in}}%
\pgfpathlineto{\pgfqpoint{3.123122in}{2.649147in}}%
\pgfpathlineto{\pgfqpoint{3.124657in}{2.653589in}}%
\pgfpathlineto{\pgfqpoint{3.126779in}{2.659850in}}%
\pgfpathlineto{\pgfqpoint{3.128854in}{2.666110in}}%
\pgfpathlineto{\pgfqpoint{3.129382in}{2.667729in}}%
\pgfpathlineto{\pgfqpoint{3.130904in}{2.672371in}}%
\pgfpathlineto{\pgfqpoint{3.132912in}{2.678631in}}%
\pgfpathlineto{\pgfqpoint{3.134868in}{2.684891in}}%
\pgfpathlineto{\pgfqpoint{3.135643in}{2.687422in}}%
\pgfpathlineto{\pgfqpoint{3.136793in}{2.691152in}}%
\pgfpathlineto{\pgfqpoint{3.138676in}{2.697412in}}%
\pgfpathlineto{\pgfqpoint{3.140507in}{2.703673in}}%
\pgfpathlineto{\pgfqpoint{3.141903in}{2.708581in}}%
\pgfpathlineto{\pgfqpoint{3.142291in}{2.709933in}}%
\pgfpathlineto{\pgfqpoint{3.144044in}{2.716194in}}%
\pgfpathlineto{\pgfqpoint{3.145742in}{2.722454in}}%
\pgfpathlineto{\pgfqpoint{3.147386in}{2.728715in}}%
\pgfpathlineto{\pgfqpoint{3.148164in}{2.731771in}}%
\pgfpathlineto{\pgfqpoint{3.148988in}{2.734975in}}%
\pgfpathlineto{\pgfqpoint{3.150547in}{2.741235in}}%
\pgfpathlineto{\pgfqpoint{3.152050in}{2.747496in}}%
\pgfpathlineto{\pgfqpoint{3.153497in}{2.753756in}}%
\pgfpathlineto{\pgfqpoint{3.154424in}{2.757923in}}%
\pgfpathlineto{\pgfqpoint{3.154896in}{2.760017in}}%
\pgfpathlineto{\pgfqpoint{3.156252in}{2.766277in}}%
\pgfpathlineto{\pgfqpoint{3.157550in}{2.772538in}}%
\pgfpathlineto{\pgfqpoint{3.158791in}{2.778798in}}%
\pgfpathlineto{\pgfqpoint{3.159974in}{2.785058in}}%
\pgfpathlineto{\pgfqpoint{3.160684in}{2.789008in}}%
\pgfpathlineto{\pgfqpoint{3.161106in}{2.791319in}}%
\pgfpathlineto{\pgfqpoint{3.162190in}{2.797579in}}%
\pgfpathlineto{\pgfqpoint{3.163214in}{2.803840in}}%
\pgfpathlineto{\pgfqpoint{3.164177in}{2.810100in}}%
\pgfpathlineto{\pgfqpoint{3.165080in}{2.816361in}}%
\pgfpathlineto{\pgfqpoint{3.165923in}{2.822621in}}%
\pgfpathlineto{\pgfqpoint{3.166704in}{2.828881in}}%
\pgfpathlineto{\pgfqpoint{3.166945in}{2.830979in}}%
\pgfpathlineto{\pgfqpoint{3.167431in}{2.835142in}}%
\pgfpathlineto{\pgfqpoint{3.168097in}{2.841402in}}%
\pgfpathlineto{\pgfqpoint{3.168698in}{2.847663in}}%
\pgfpathlineto{\pgfqpoint{3.169234in}{2.853923in}}%
\pgfpathlineto{\pgfqpoint{3.169703in}{2.860184in}}%
\pgfpathlineto{\pgfqpoint{3.170106in}{2.866444in}}%
\pgfpathlineto{\pgfqpoint{3.170439in}{2.872705in}}%
\pgfpathlineto{\pgfqpoint{3.170703in}{2.878965in}}%
\pgfpathlineto{\pgfqpoint{3.170896in}{2.885225in}}%
\pgfpathlineto{\pgfqpoint{3.171017in}{2.891486in}}%
\pgfpathlineto{\pgfqpoint{3.171062in}{2.897746in}}%
\pgfpathlineto{\pgfqpoint{3.171032in}{2.904007in}}%
\pgfpathlineto{\pgfqpoint{3.170923in}{2.910267in}}%
\pgfpathlineto{\pgfqpoint{3.170734in}{2.916528in}}%
\pgfpathlineto{\pgfqpoint{3.170461in}{2.922788in}}%
\pgfpathlineto{\pgfqpoint{3.170103in}{2.929048in}}%
\pgfpathlineto{\pgfqpoint{3.169657in}{2.935309in}}%
\pgfpathlineto{\pgfqpoint{3.169119in}{2.941569in}}%
\pgfpathlineto{\pgfqpoint{3.168487in}{2.947830in}}%
\pgfpathlineto{\pgfqpoint{3.167757in}{2.954090in}}%
\pgfpathlineto{\pgfqpoint{3.166945in}{2.960196in}}%
\pgfpathlineto{\pgfqpoint{3.166924in}{2.960351in}}%
\pgfpathlineto{\pgfqpoint{3.165987in}{2.966611in}}%
\pgfpathlineto{\pgfqpoint{3.164939in}{2.972871in}}%
\pgfpathlineto{\pgfqpoint{3.163774in}{2.979132in}}%
\pgfpathlineto{\pgfqpoint{3.162488in}{2.985392in}}%
\pgfpathlineto{\pgfqpoint{3.161075in}{2.991653in}}%
\pgfpathlineto{\pgfqpoint{3.160684in}{2.993254in}}%
\pgfpathlineto{\pgfqpoint{3.159523in}{2.997913in}}%
\pgfpathlineto{\pgfqpoint{3.157828in}{3.004174in}}%
\pgfpathlineto{\pgfqpoint{3.155983in}{3.010434in}}%
\pgfpathlineto{\pgfqpoint{3.154424in}{3.015331in}}%
\pgfpathlineto{\pgfqpoint{3.153977in}{3.016694in}}%
\pgfpathlineto{\pgfqpoint{3.151786in}{3.022955in}}%
\pgfpathlineto{\pgfqpoint{3.149416in}{3.029215in}}%
\pgfpathlineto{\pgfqpoint{3.148164in}{3.032314in}}%
\pgfpathlineto{\pgfqpoint{3.146836in}{3.035476in}}%
\pgfpathlineto{\pgfqpoint{3.144030in}{3.041736in}}%
\pgfpathlineto{\pgfqpoint{3.141903in}{3.046167in}}%
\pgfpathlineto{\pgfqpoint{3.140984in}{3.047997in}}%
\pgfpathlineto{\pgfqpoint{3.137653in}{3.054257in}}%
\pgfpathlineto{\pgfqpoint{3.135643in}{3.057803in}}%
\pgfpathlineto{\pgfqpoint{3.134019in}{3.060518in}}%
\pgfpathlineto{\pgfqpoint{3.130043in}{3.066778in}}%
\pgfpathlineto{\pgfqpoint{3.129382in}{3.067768in}}%
\pgfpathlineto{\pgfqpoint{3.125636in}{3.073038in}}%
\pgfpathlineto{\pgfqpoint{3.123122in}{3.076380in}}%
\pgfpathlineto{\pgfqpoint{3.120766in}{3.079299in}}%
\pgfpathlineto{\pgfqpoint{3.116861in}{3.083891in}}%
\pgfpathlineto{\pgfqpoint{3.115327in}{3.085559in}}%
\pgfpathlineto{\pgfqpoint{3.110601in}{3.090459in}}%
\pgfpathlineto{\pgfqpoint{3.109166in}{3.091820in}}%
\pgfpathlineto{\pgfqpoint{3.104341in}{3.096207in}}%
\pgfpathlineto{\pgfqpoint{3.102064in}{3.098080in}}%
\pgfpathlineto{\pgfqpoint{3.098080in}{3.101235in}}%
\pgfpathlineto{\pgfqpoint{3.093690in}{3.104341in}}%
\pgfpathlineto{\pgfqpoint{3.091820in}{3.105619in}}%
\pgfpathlineto{\pgfqpoint{3.085559in}{3.109418in}}%
\pgfpathlineto{\pgfqpoint{3.083332in}{3.110601in}}%
\pgfpathlineto{\pgfqpoint{3.079299in}{3.112684in}}%
\pgfpathlineto{\pgfqpoint{3.073038in}{3.115458in}}%
\pgfpathlineto{\pgfqpoint{3.069281in}{3.116861in}}%
\pgfpathlineto{\pgfqpoint{3.066778in}{3.117776in}}%
\pgfpathlineto{\pgfqpoint{3.060518in}{3.119668in}}%
\pgfpathlineto{\pgfqpoint{3.054257in}{3.121157in}}%
\pgfpathlineto{\pgfqpoint{3.047997in}{3.122265in}}%
\pgfpathlineto{\pgfqpoint{3.041736in}{3.123013in}}%
\pgfpathlineto{\pgfqpoint{3.040074in}{3.123122in}}%
\pgfpathlineto{\pgfqpoint{3.035476in}{3.123419in}}%
\pgfpathlineto{\pgfqpoint{3.029215in}{3.123496in}}%
\pgfpathlineto{\pgfqpoint{3.022955in}{3.123254in}}%
\pgfpathlineto{\pgfqpoint{3.021458in}{3.123122in}}%
\pgfpathlineto{\pgfqpoint{3.016694in}{3.122709in}}%
\pgfpathlineto{\pgfqpoint{3.010434in}{3.121869in}}%
\pgfpathlineto{\pgfqpoint{3.004174in}{3.120741in}}%
\pgfpathlineto{\pgfqpoint{2.997913in}{3.119333in}}%
\pgfpathlineto{\pgfqpoint{2.991653in}{3.117647in}}%
\pgfpathlineto{\pgfqpoint{2.989148in}{3.116861in}}%
\pgfpathlineto{\pgfqpoint{2.985392in}{3.115698in}}%
\pgfpathlineto{\pgfqpoint{2.979132in}{3.113488in}}%
\pgfpathlineto{\pgfqpoint{2.972871in}{3.111012in}}%
\pgfpathlineto{\pgfqpoint{2.971934in}{3.110601in}}%
\pgfpathlineto{\pgfqpoint{2.966611in}{3.108291in}}%
\pgfpathlineto{\pgfqpoint{2.960351in}{3.105310in}}%
\pgfpathlineto{\pgfqpoint{2.958478in}{3.104341in}}%
\pgfpathlineto{\pgfqpoint{2.954090in}{3.102085in}}%
\pgfpathlineto{\pgfqpoint{2.947830in}{3.098605in}}%
\pgfpathlineto{\pgfqpoint{2.946950in}{3.098080in}}%
\pgfpathlineto{\pgfqpoint{2.941569in}{3.094887in}}%
\pgfpathlineto{\pgfqpoint{2.936740in}{3.091820in}}%
\pgfpathlineto{\pgfqpoint{2.935309in}{3.090914in}}%
\pgfpathlineto{\pgfqpoint{2.929048in}{3.086700in}}%
\pgfpathlineto{\pgfqpoint{2.927448in}{3.085559in}}%
\pgfpathlineto{\pgfqpoint{2.922788in}{3.082242in}}%
\pgfpathlineto{\pgfqpoint{2.918877in}{3.079299in}}%
\pgfpathlineto{\pgfqpoint{2.916528in}{3.077531in}}%
\pgfpathlineto{\pgfqpoint{2.910859in}{3.073038in}}%
\pgfpathlineto{\pgfqpoint{2.910267in}{3.072569in}}%
\pgfpathlineto{\pgfqpoint{2.904007in}{3.067361in}}%
\pgfpathlineto{\pgfqpoint{2.903337in}{3.066778in}}%
\pgfpathlineto{\pgfqpoint{2.897746in}{3.061901in}}%
\pgfpathlineto{\pgfqpoint{2.896227in}{3.060518in}}%
\pgfpathlineto{\pgfqpoint{2.891486in}{3.056183in}}%
\pgfpathlineto{\pgfqpoint{2.889462in}{3.054257in}}%
\pgfpathlineto{\pgfqpoint{2.885225in}{3.050204in}}%
\pgfpathlineto{\pgfqpoint{2.883003in}{3.047997in}}%
\pgfpathlineto{\pgfqpoint{2.878965in}{3.043959in}}%
\pgfpathlineto{\pgfqpoint{2.876818in}{3.041736in}}%
\pgfpathlineto{\pgfqpoint{2.872705in}{3.037444in}}%
\pgfpathlineto{\pgfqpoint{2.870879in}{3.035476in}}%
\pgfpathlineto{\pgfqpoint{2.866444in}{3.030652in}}%
\pgfpathlineto{\pgfqpoint{2.865163in}{3.029215in}}%
\pgfpathlineto{\pgfqpoint{2.860184in}{3.023576in}}%
\pgfpathlineto{\pgfqpoint{2.859650in}{3.022955in}}%
\pgfpathlineto{\pgfqpoint{2.854330in}{3.016694in}}%
\pgfpathlineto{\pgfqpoint{2.853923in}{3.016211in}}%
\pgfpathlineto{\pgfqpoint{2.849187in}{3.010434in}}%
\pgfpathlineto{\pgfqpoint{2.847663in}{3.008551in}}%
\pgfpathlineto{\pgfqpoint{2.844203in}{3.004174in}}%
\pgfpathlineto{\pgfqpoint{2.841402in}{3.000580in}}%
\pgfpathlineto{\pgfqpoint{2.839368in}{2.997913in}}%
\pgfpathlineto{\pgfqpoint{2.835142in}{2.992286in}}%
\pgfpathlineto{\pgfqpoint{2.834676in}{2.991653in}}%
\pgfpathlineto{\pgfqpoint{2.830120in}{2.985392in}}%
\pgfpathlineto{\pgfqpoint{2.828881in}{2.983664in}}%
\pgfpathlineto{\pgfqpoint{2.825689in}{2.979132in}}%
\pgfpathlineto{\pgfqpoint{2.822621in}{2.974696in}}%
\pgfpathlineto{\pgfqpoint{2.821378in}{2.972871in}}%
\pgfpathlineto{\pgfqpoint{2.817182in}{2.966611in}}%
\pgfpathlineto{\pgfqpoint{2.816361in}{2.965367in}}%
\pgfpathlineto{\pgfqpoint{2.813093in}{2.960351in}}%
\pgfpathlineto{\pgfqpoint{2.810100in}{2.955661in}}%
\pgfpathlineto{\pgfqpoint{2.809109in}{2.954090in}}%
\pgfpathlineto{\pgfqpoint{2.805224in}{2.947830in}}%
\pgfpathlineto{\pgfqpoint{2.803840in}{2.945558in}}%
\pgfpathlineto{\pgfqpoint{2.801433in}{2.941569in}}%
\pgfpathlineto{\pgfqpoint{2.797737in}{2.935309in}}%
\pgfpathlineto{\pgfqpoint{2.797579in}{2.935037in}}%
\pgfpathlineto{\pgfqpoint{2.794121in}{2.929048in}}%
\pgfpathlineto{\pgfqpoint{2.791319in}{2.924076in}}%
\pgfpathlineto{\pgfqpoint{2.790597in}{2.922788in}}%
\pgfpathlineto{\pgfqpoint{2.787149in}{2.916528in}}%
\pgfpathlineto{\pgfqpoint{2.785058in}{2.912646in}}%
\pgfpathlineto{\pgfqpoint{2.783782in}{2.910267in}}%
\pgfpathlineto{\pgfqpoint{2.780490in}{2.904007in}}%
\pgfpathlineto{\pgfqpoint{2.778798in}{2.900720in}}%
\pgfpathlineto{\pgfqpoint{2.777271in}{2.897746in}}%
\pgfpathlineto{\pgfqpoint{2.774123in}{2.891486in}}%
\pgfpathlineto{\pgfqpoint{2.772538in}{2.888264in}}%
\pgfpathlineto{\pgfqpoint{2.771043in}{2.885225in}}%
\pgfpathlineto{\pgfqpoint{2.768028in}{2.878965in}}%
\pgfpathlineto{\pgfqpoint{2.766277in}{2.875244in}}%
\pgfpathlineto{\pgfqpoint{2.765080in}{2.872705in}}%
\pgfpathlineto{\pgfqpoint{2.762190in}{2.866444in}}%
\pgfpathlineto{\pgfqpoint{2.760017in}{2.861620in}}%
\pgfpathlineto{\pgfqpoint{2.759367in}{2.860184in}}%
\pgfpathlineto{\pgfqpoint{2.756591in}{2.853923in}}%
\pgfpathlineto{\pgfqpoint{2.753889in}{2.847663in}}%
\pgfpathlineto{\pgfqpoint{2.753756in}{2.847351in}}%
\pgfpathlineto{\pgfqpoint{2.751219in}{2.841402in}}%
\pgfpathlineto{\pgfqpoint{2.748619in}{2.835142in}}%
\pgfpathlineto{\pgfqpoint{2.747496in}{2.832379in}}%
\pgfpathlineto{\pgfqpoint{2.746062in}{2.828881in}}%
\pgfpathlineto{\pgfqpoint{2.743555in}{2.822621in}}%
\pgfpathlineto{\pgfqpoint{2.741235in}{2.816678in}}%
\pgfpathlineto{\pgfqpoint{2.741110in}{2.816361in}}%
\pgfpathlineto{\pgfqpoint{2.738688in}{2.810100in}}%
\pgfpathlineto{\pgfqpoint{2.736327in}{2.803840in}}%
\pgfpathlineto{\pgfqpoint{2.734975in}{2.800170in}}%
\pgfpathlineto{\pgfqpoint{2.734008in}{2.797579in}}%
\pgfpathlineto{\pgfqpoint{2.731722in}{2.791319in}}%
\pgfpathlineto{\pgfqpoint{2.729492in}{2.785058in}}%
\pgfpathlineto{\pgfqpoint{2.728715in}{2.782828in}}%
\pgfpathlineto{\pgfqpoint{2.727288in}{2.778798in}}%
\pgfpathlineto{\pgfqpoint{2.725123in}{2.772538in}}%
\pgfpathlineto{\pgfqpoint{2.723009in}{2.766277in}}%
\pgfpathlineto{\pgfqpoint{2.722454in}{2.764598in}}%
\pgfpathlineto{\pgfqpoint{2.720912in}{2.760017in}}%
\pgfpathlineto{\pgfqpoint{2.718853in}{2.753756in}}%
\pgfpathlineto{\pgfqpoint{2.716840in}{2.747496in}}%
\pgfpathlineto{\pgfqpoint{2.716194in}{2.745440in}}%
\pgfpathlineto{\pgfqpoint{2.714843in}{2.741235in}}%
\pgfpathlineto{\pgfqpoint{2.712876in}{2.734975in}}%
\pgfpathlineto{\pgfqpoint{2.710951in}{2.728715in}}%
\pgfpathlineto{\pgfqpoint{2.709933in}{2.725331in}}%
\pgfpathlineto{\pgfqpoint{2.709047in}{2.722454in}}%
\pgfpathlineto{\pgfqpoint{2.707158in}{2.716194in}}%
\pgfpathlineto{\pgfqpoint{2.705309in}{2.709933in}}%
\pgfpathlineto{\pgfqpoint{2.703673in}{2.704280in}}%
\pgfpathlineto{\pgfqpoint{2.703492in}{2.703673in}}%
\pgfpathlineto{\pgfqpoint{2.701670in}{2.697412in}}%
\pgfpathlineto{\pgfqpoint{2.699883in}{2.691152in}}%
\pgfpathlineto{\pgfqpoint{2.698130in}{2.684891in}}%
\pgfpathlineto{\pgfqpoint{2.697412in}{2.682273in}}%
\pgfpathlineto{\pgfqpoint{2.696382in}{2.678631in}}%
\pgfpathlineto{\pgfqpoint{2.694647in}{2.672371in}}%
\pgfpathlineto{\pgfqpoint{2.692942in}{2.666110in}}%
\pgfpathlineto{\pgfqpoint{2.691267in}{2.659850in}}%
\pgfpathlineto{\pgfqpoint{2.691152in}{2.659411in}}%
\pgfpathlineto{\pgfqpoint{2.689575in}{2.653589in}}%
\pgfpathlineto{\pgfqpoint{2.687908in}{2.647329in}}%
\pgfpathlineto{\pgfqpoint{2.686269in}{2.641068in}}%
\pgfpathlineto{\pgfqpoint{2.684891in}{2.635722in}}%
\pgfpathlineto{\pgfqpoint{2.684647in}{2.634808in}}%
\pgfpathlineto{\pgfqpoint{2.683009in}{2.628548in}}%
\pgfpathlineto{\pgfqpoint{2.681396in}{2.622287in}}%
\pgfpathlineto{\pgfqpoint{2.679808in}{2.616027in}}%
\pgfpathlineto{\pgfqpoint{2.678631in}{2.611308in}}%
\pgfpathlineto{\pgfqpoint{2.678230in}{2.609766in}}%
\pgfpathlineto{\pgfqpoint{2.676637in}{2.603506in}}%
\pgfpathlineto{\pgfqpoint{2.675068in}{2.597245in}}%
\pgfpathlineto{\pgfqpoint{2.673523in}{2.590985in}}%
\pgfpathlineto{\pgfqpoint{2.672371in}{2.586235in}}%
\pgfpathlineto{\pgfqpoint{2.671986in}{2.584725in}}%
\pgfpathlineto{\pgfqpoint{2.670434in}{2.578464in}}%
\pgfpathlineto{\pgfqpoint{2.668905in}{2.572204in}}%
\pgfpathlineto{\pgfqpoint{2.667401in}{2.565943in}}%
\pgfpathlineto{\pgfqpoint{2.666110in}{2.560474in}}%
\pgfpathlineto{\pgfqpoint{2.665913in}{2.559683in}}%
\pgfpathlineto{\pgfqpoint{2.664404in}{2.553422in}}%
\pgfpathlineto{\pgfqpoint{2.662922in}{2.547162in}}%
\pgfpathlineto{\pgfqpoint{2.661468in}{2.540902in}}%
\pgfpathlineto{\pgfqpoint{2.660044in}{2.534641in}}%
\pgfpathlineto{\pgfqpoint{2.659850in}{2.533757in}}%
\pgfpathlineto{\pgfqpoint{2.658602in}{2.528381in}}%
\pgfpathlineto{\pgfqpoint{2.657188in}{2.522120in}}%
\pgfpathlineto{\pgfqpoint{2.655809in}{2.515860in}}%
\pgfpathlineto{\pgfqpoint{2.654470in}{2.509599in}}%
\pgfpathlineto{\pgfqpoint{2.653589in}{2.505325in}}%
\pgfpathlineto{\pgfqpoint{2.653154in}{2.503339in}}%
\pgfpathlineto{\pgfqpoint{2.651849in}{2.497078in}}%
\pgfpathlineto{\pgfqpoint{2.650594in}{2.490818in}}%
\pgfpathlineto{\pgfqpoint{2.649394in}{2.484558in}}%
\pgfpathlineto{\pgfqpoint{2.648254in}{2.478297in}}%
\pgfpathlineto{\pgfqpoint{2.647329in}{2.472900in}}%
\pgfpathlineto{\pgfqpoint{2.647171in}{2.472037in}}%
\pgfpathlineto{\pgfqpoint{2.646120in}{2.465776in}}%
\pgfpathlineto{\pgfqpoint{2.645150in}{2.459516in}}%
\pgfpathlineto{\pgfqpoint{2.644265in}{2.453255in}}%
\pgfpathlineto{\pgfqpoint{2.643475in}{2.446995in}}%
\pgfpathlineto{\pgfqpoint{2.642785in}{2.440735in}}%
\pgfpathlineto{\pgfqpoint{2.642206in}{2.434474in}}%
\pgfpathlineto{\pgfqpoint{2.641745in}{2.428214in}}%
\pgfpathlineto{\pgfqpoint{2.641412in}{2.421953in}}%
\pgfpathlineto{\pgfqpoint{2.641218in}{2.415693in}}%
\pgfpathlineto{\pgfqpoint{2.641172in}{2.409432in}}%
\pgfpathlineto{\pgfqpoint{2.641287in}{2.403172in}}%
\pgfpathlineto{\pgfqpoint{2.641572in}{2.396912in}}%
\pgfpathlineto{\pgfqpoint{2.642040in}{2.390651in}}%
\pgfpathlineto{\pgfqpoint{2.642701in}{2.384391in}}%
\pgfpathlineto{\pgfqpoint{2.643566in}{2.378130in}}%
\pgfpathlineto{\pgfqpoint{2.644647in}{2.371870in}}%
\pgfpathlineto{\pgfqpoint{2.645953in}{2.365609in}}%
\pgfpathlineto{\pgfqpoint{2.647329in}{2.359997in}}%
\pgfpathlineto{\pgfqpoint{2.647486in}{2.359349in}}%
\pgfpathlineto{\pgfqpoint{2.649201in}{2.353088in}}%
\pgfpathlineto{\pgfqpoint{2.651151in}{2.346828in}}%
\pgfpathlineto{\pgfqpoint{2.653340in}{2.340568in}}%
\pgfpathlineto{\pgfqpoint{2.653589in}{2.339912in}}%
\pgfpathlineto{\pgfqpoint{2.655692in}{2.334307in}}%
\pgfpathlineto{\pgfqpoint{2.658259in}{2.328047in}}%
\pgfpathlineto{\pgfqpoint{2.659850in}{2.324437in}}%
\pgfpathlineto{\pgfqpoint{2.661001in}{2.321786in}}%
\pgfpathlineto{\pgfqpoint{2.663885in}{2.315526in}}%
\pgfpathlineto{\pgfqpoint{2.666110in}{2.310951in}}%
\pgfpathlineto{\pgfqpoint{2.666916in}{2.309265in}}%
\pgfpathlineto{\pgfqpoint{2.670023in}{2.303005in}}%
\pgfpathlineto{\pgfqpoint{2.672371in}{2.298432in}}%
\pgfpathlineto{\pgfqpoint{2.673219in}{2.296745in}}%
\pgfpathlineto{\pgfqpoint{2.676424in}{2.290484in}}%
\pgfpathlineto{\pgfqpoint{2.678631in}{2.286226in}}%
\pgfpathlineto{\pgfqpoint{2.679642in}{2.284224in}}%
\pgfpathlineto{\pgfqpoint{2.682801in}{2.277963in}}%
\pgfpathlineto{\pgfqpoint{2.684891in}{2.273781in}}%
\pgfpathlineto{\pgfqpoint{2.685897in}{2.271703in}}%
\pgfpathlineto{\pgfqpoint{2.688866in}{2.265442in}}%
\pgfpathlineto{\pgfqpoint{2.691152in}{2.260461in}}%
\pgfpathlineto{\pgfqpoint{2.691716in}{2.259182in}}%
\pgfpathlineto{\pgfqpoint{2.694373in}{2.252922in}}%
\pgfpathlineto{\pgfqpoint{2.696876in}{2.246661in}}%
\pgfpathlineto{\pgfqpoint{2.697412in}{2.245241in}}%
\pgfpathlineto{\pgfqpoint{2.699154in}{2.240401in}}%
\pgfpathlineto{\pgfqpoint{2.701232in}{2.234140in}}%
\pgfpathlineto{\pgfqpoint{2.703118in}{2.227880in}}%
\pgfpathlineto{\pgfqpoint{2.703673in}{2.225860in}}%
\pgfpathlineto{\pgfqpoint{2.704773in}{2.221619in}}%
\pgfpathlineto{\pgfqpoint{2.706218in}{2.215359in}}%
\pgfpathlineto{\pgfqpoint{2.707473in}{2.209098in}}%
\pgfpathlineto{\pgfqpoint{2.708544in}{2.202838in}}%
\pgfpathlineto{\pgfqpoint{2.709439in}{2.196578in}}%
\pgfpathlineto{\pgfqpoint{2.709933in}{2.192368in}}%
\pgfpathlineto{\pgfqpoint{2.710159in}{2.190317in}}%
\pgfpathlineto{\pgfqpoint{2.710713in}{2.184057in}}%
\pgfpathlineto{\pgfqpoint{2.711125in}{2.177796in}}%
\pgfpathlineto{\pgfqpoint{2.711408in}{2.171536in}}%
\pgfpathlineto{\pgfqpoint{2.711572in}{2.165275in}}%
\pgfpathlineto{\pgfqpoint{2.711630in}{2.159015in}}%
\pgfpathlineto{\pgfqpoint{2.711592in}{2.152755in}}%
\pgfpathlineto{\pgfqpoint{2.711468in}{2.146494in}}%
\pgfpathlineto{\pgfqpoint{2.711268in}{2.140234in}}%
\pgfpathlineto{\pgfqpoint{2.711002in}{2.133973in}}%
\pgfpathlineto{\pgfqpoint{2.710678in}{2.127713in}}%
\pgfpathlineto{\pgfqpoint{2.710305in}{2.121452in}}%
\pgfpathlineto{\pgfqpoint{2.709933in}{2.115857in}}%
\pgfpathlineto{\pgfqpoint{2.709889in}{2.115192in}}%
\pgfpathlineto{\pgfqpoint{2.709425in}{2.108932in}}%
\pgfpathlineto{\pgfqpoint{2.708932in}{2.102671in}}%
\pgfpathlineto{\pgfqpoint{2.708417in}{2.096411in}}%
\pgfpathlineto{\pgfqpoint{2.707885in}{2.090150in}}%
\pgfpathlineto{\pgfqpoint{2.707343in}{2.083890in}}%
\pgfpathlineto{\pgfqpoint{2.706794in}{2.077629in}}%
\pgfpathlineto{\pgfqpoint{2.706245in}{2.071369in}}%
\pgfpathlineto{\pgfqpoint{2.705698in}{2.065109in}}%
\pgfpathlineto{\pgfqpoint{2.705159in}{2.058848in}}%
\pgfpathlineto{\pgfqpoint{2.704631in}{2.052588in}}%
\pgfpathlineto{\pgfqpoint{2.704118in}{2.046327in}}%
\pgfpathlineto{\pgfqpoint{2.703673in}{2.040713in}}%
\pgfpathlineto{\pgfqpoint{2.703621in}{2.040067in}}%
\pgfpathlineto{\pgfqpoint{2.703134in}{2.033806in}}%
\pgfpathlineto{\pgfqpoint{2.702672in}{2.027546in}}%
\pgfpathlineto{\pgfqpoint{2.702240in}{2.021285in}}%
\pgfpathlineto{\pgfqpoint{2.701839in}{2.015025in}}%
\pgfpathlineto{\pgfqpoint{2.701474in}{2.008765in}}%
\pgfpathlineto{\pgfqpoint{2.701147in}{2.002504in}}%
\pgfpathlineto{\pgfqpoint{2.700860in}{1.996244in}}%
\pgfpathlineto{\pgfqpoint{2.700618in}{1.989983in}}%
\pgfpathlineto{\pgfqpoint{2.700423in}{1.983723in}}%
\pgfpathlineto{\pgfqpoint{2.700277in}{1.977462in}}%
\pgfpathlineto{\pgfqpoint{2.700185in}{1.971202in}}%
\pgfpathlineto{\pgfqpoint{2.700149in}{1.964942in}}%
\pgfpathlineto{\pgfqpoint{2.700174in}{1.958681in}}%
\pgfpathlineto{\pgfqpoint{2.700261in}{1.952421in}}%
\pgfpathlineto{\pgfqpoint{2.700416in}{1.946160in}}%
\pgfpathlineto{\pgfqpoint{2.700642in}{1.939900in}}%
\pgfpathlineto{\pgfqpoint{2.700943in}{1.933639in}}%
\pgfpathlineto{\pgfqpoint{2.701323in}{1.927379in}}%
\pgfpathlineto{\pgfqpoint{2.701788in}{1.921119in}}%
\pgfpathlineto{\pgfqpoint{2.702341in}{1.914858in}}%
\pgfpathlineto{\pgfqpoint{2.702989in}{1.908598in}}%
\pgfpathlineto{\pgfqpoint{2.703673in}{1.902871in}}%
\pgfpathlineto{\pgfqpoint{2.703736in}{1.902337in}}%
\pgfpathlineto{\pgfqpoint{2.704579in}{1.896077in}}%
\pgfpathlineto{\pgfqpoint{2.705535in}{1.889816in}}%
\pgfpathlineto{\pgfqpoint{2.706610in}{1.883556in}}%
\pgfpathlineto{\pgfqpoint{2.707812in}{1.877295in}}%
\pgfpathlineto{\pgfqpoint{2.709149in}{1.871035in}}%
\pgfpathlineto{\pgfqpoint{2.709933in}{1.867700in}}%
\pgfpathlineto{\pgfqpoint{2.710626in}{1.864775in}}%
\pgfpathlineto{\pgfqpoint{2.712254in}{1.858514in}}%
\pgfpathlineto{\pgfqpoint{2.714047in}{1.852254in}}%
\pgfpathlineto{\pgfqpoint{2.716018in}{1.845993in}}%
\pgfpathlineto{\pgfqpoint{2.716194in}{1.845476in}}%
\pgfpathlineto{\pgfqpoint{2.718181in}{1.839733in}}%
\pgfpathlineto{\pgfqpoint{2.720550in}{1.833472in}}%
\pgfpathlineto{\pgfqpoint{2.722454in}{1.828849in}}%
\pgfpathlineto{\pgfqpoint{2.723147in}{1.827212in}}%
\pgfpathlineto{\pgfqpoint{2.725999in}{1.820952in}}%
\pgfpathlineto{\pgfqpoint{2.728715in}{1.815478in}}%
\pgfpathlineto{\pgfqpoint{2.729119in}{1.814691in}}%
\pgfpathlineto{\pgfqpoint{2.732563in}{1.808431in}}%
\pgfpathlineto{\pgfqpoint{2.734975in}{1.804377in}}%
\pgfpathlineto{\pgfqpoint{2.736348in}{1.802170in}}%
\pgfpathlineto{\pgfqpoint{2.740531in}{1.795910in}}%
\pgfpathlineto{\pgfqpoint{2.741235in}{1.794920in}}%
\pgfpathlineto{\pgfqpoint{2.745194in}{1.789649in}}%
\pgfpathlineto{\pgfqpoint{2.747496in}{1.786785in}}%
\pgfpathlineto{\pgfqpoint{2.750400in}{1.783389in}}%
\pgfpathlineto{\pgfqpoint{2.753756in}{1.779700in}}%
\pgfpathlineto{\pgfqpoint{2.756269in}{1.777129in}}%
\pgfpathlineto{\pgfqpoint{2.760017in}{1.773503in}}%
\pgfpathlineto{\pgfqpoint{2.762970in}{1.770868in}}%
\pgfpathlineto{\pgfqpoint{2.766277in}{1.768065in}}%
\pgfpathlineto{\pgfqpoint{2.770746in}{1.764608in}}%
\pgfpathlineto{\pgfqpoint{2.772538in}{1.763284in}}%
\pgfpathlineto{\pgfqpoint{2.778798in}{1.759091in}}%
\pgfpathlineto{\pgfqpoint{2.780028in}{1.758347in}}%
\pgfpathlineto{\pgfqpoint{2.785058in}{1.755425in}}%
\pgfpathlineto{\pgfqpoint{2.791319in}{1.752225in}}%
\pgfpathlineto{\pgfqpoint{2.791622in}{1.752087in}}%
\pgfpathlineto{\pgfqpoint{2.797579in}{1.749464in}}%
\pgfpathlineto{\pgfqpoint{2.803840in}{1.747098in}}%
\pgfpathlineto{\pgfqpoint{2.807791in}{1.745826in}}%
\pgfpathclose%
\pgfpathmoveto{\pgfqpoint{2.866051in}{1.802170in}}%
\pgfpathlineto{\pgfqpoint{2.860184in}{1.802469in}}%
\pgfpathlineto{\pgfqpoint{2.853923in}{1.803139in}}%
\pgfpathlineto{\pgfqpoint{2.847663in}{1.804181in}}%
\pgfpathlineto{\pgfqpoint{2.841402in}{1.805618in}}%
\pgfpathlineto{\pgfqpoint{2.835142in}{1.807476in}}%
\pgfpathlineto{\pgfqpoint{2.832513in}{1.808431in}}%
\pgfpathlineto{\pgfqpoint{2.828881in}{1.809809in}}%
\pgfpathlineto{\pgfqpoint{2.822621in}{1.812655in}}%
\pgfpathlineto{\pgfqpoint{2.818805in}{1.814691in}}%
\pgfpathlineto{\pgfqpoint{2.816361in}{1.816066in}}%
\pgfpathlineto{\pgfqpoint{2.810100in}{1.820121in}}%
\pgfpathlineto{\pgfqpoint{2.808962in}{1.820952in}}%
\pgfpathlineto{\pgfqpoint{2.803840in}{1.824931in}}%
\pgfpathlineto{\pgfqpoint{2.801218in}{1.827212in}}%
\pgfpathlineto{\pgfqpoint{2.797579in}{1.830602in}}%
\pgfpathlineto{\pgfqpoint{2.794787in}{1.833472in}}%
\pgfpathlineto{\pgfqpoint{2.791319in}{1.837316in}}%
\pgfpathlineto{\pgfqpoint{2.789318in}{1.839733in}}%
\pgfpathlineto{\pgfqpoint{2.785058in}{1.845318in}}%
\pgfpathlineto{\pgfqpoint{2.784581in}{1.845993in}}%
\pgfpathlineto{\pgfqpoint{2.780480in}{1.852254in}}%
\pgfpathlineto{\pgfqpoint{2.778798in}{1.855070in}}%
\pgfpathlineto{\pgfqpoint{2.776864in}{1.858514in}}%
\pgfpathlineto{\pgfqpoint{2.773666in}{1.864775in}}%
\pgfpathlineto{\pgfqpoint{2.772538in}{1.867204in}}%
\pgfpathlineto{\pgfqpoint{2.770846in}{1.871035in}}%
\pgfpathlineto{\pgfqpoint{2.768350in}{1.877295in}}%
\pgfpathlineto{\pgfqpoint{2.766277in}{1.883119in}}%
\pgfpathlineto{\pgfqpoint{2.766128in}{1.883556in}}%
\pgfpathlineto{\pgfqpoint{2.764192in}{1.889816in}}%
\pgfpathlineto{\pgfqpoint{2.762482in}{1.896077in}}%
\pgfpathlineto{\pgfqpoint{2.760984in}{1.902337in}}%
\pgfpathlineto{\pgfqpoint{2.760017in}{1.906976in}}%
\pgfpathlineto{\pgfqpoint{2.759687in}{1.908598in}}%
\pgfpathlineto{\pgfqpoint{2.758585in}{1.914858in}}%
\pgfpathlineto{\pgfqpoint{2.757652in}{1.921119in}}%
\pgfpathlineto{\pgfqpoint{2.756880in}{1.927379in}}%
\pgfpathlineto{\pgfqpoint{2.756261in}{1.933639in}}%
\pgfpathlineto{\pgfqpoint{2.755785in}{1.939900in}}%
\pgfpathlineto{\pgfqpoint{2.755446in}{1.946160in}}%
\pgfpathlineto{\pgfqpoint{2.755236in}{1.952421in}}%
\pgfpathlineto{\pgfqpoint{2.755151in}{1.958681in}}%
\pgfpathlineto{\pgfqpoint{2.755183in}{1.964942in}}%
\pgfpathlineto{\pgfqpoint{2.755329in}{1.971202in}}%
\pgfpathlineto{\pgfqpoint{2.755584in}{1.977462in}}%
\pgfpathlineto{\pgfqpoint{2.755943in}{1.983723in}}%
\pgfpathlineto{\pgfqpoint{2.756404in}{1.989983in}}%
\pgfpathlineto{\pgfqpoint{2.756962in}{1.996244in}}%
\pgfpathlineto{\pgfqpoint{2.757615in}{2.002504in}}%
\pgfpathlineto{\pgfqpoint{2.758361in}{2.008765in}}%
\pgfpathlineto{\pgfqpoint{2.759197in}{2.015025in}}%
\pgfpathlineto{\pgfqpoint{2.760017in}{2.020586in}}%
\pgfpathlineto{\pgfqpoint{2.760120in}{2.021285in}}%
\pgfpathlineto{\pgfqpoint{2.761126in}{2.027546in}}%
\pgfpathlineto{\pgfqpoint{2.762215in}{2.033806in}}%
\pgfpathlineto{\pgfqpoint{2.763386in}{2.040067in}}%
\pgfpathlineto{\pgfqpoint{2.764637in}{2.046327in}}%
\pgfpathlineto{\pgfqpoint{2.765969in}{2.052588in}}%
\pgfpathlineto{\pgfqpoint{2.766277in}{2.053967in}}%
\pgfpathlineto{\pgfqpoint{2.767375in}{2.058848in}}%
\pgfpathlineto{\pgfqpoint{2.768855in}{2.065109in}}%
\pgfpathlineto{\pgfqpoint{2.770412in}{2.071369in}}%
\pgfpathlineto{\pgfqpoint{2.772044in}{2.077629in}}%
\pgfpathlineto{\pgfqpoint{2.772538in}{2.079460in}}%
\pgfpathlineto{\pgfqpoint{2.773746in}{2.083890in}}%
\pgfpathlineto{\pgfqpoint{2.775519in}{2.090150in}}%
\pgfpathlineto{\pgfqpoint{2.777365in}{2.096411in}}%
\pgfpathlineto{\pgfqpoint{2.778798in}{2.101107in}}%
\pgfpathlineto{\pgfqpoint{2.779283in}{2.102671in}}%
\pgfpathlineto{\pgfqpoint{2.781266in}{2.108932in}}%
\pgfpathlineto{\pgfqpoint{2.783320in}{2.115192in}}%
\pgfpathlineto{\pgfqpoint{2.785058in}{2.120335in}}%
\pgfpathlineto{\pgfqpoint{2.785444in}{2.121452in}}%
\pgfpathlineto{\pgfqpoint{2.787632in}{2.127713in}}%
\pgfpathlineto{\pgfqpoint{2.789889in}{2.133973in}}%
\pgfpathlineto{\pgfqpoint{2.791319in}{2.137862in}}%
\pgfpathlineto{\pgfqpoint{2.792213in}{2.140234in}}%
\pgfpathlineto{\pgfqpoint{2.794602in}{2.146494in}}%
\pgfpathlineto{\pgfqpoint{2.797056in}{2.152755in}}%
\pgfpathlineto{\pgfqpoint{2.797579in}{2.154087in}}%
\pgfpathlineto{\pgfqpoint{2.799576in}{2.159015in}}%
\pgfpathlineto{\pgfqpoint{2.802158in}{2.165275in}}%
\pgfpathlineto{\pgfqpoint{2.803840in}{2.169308in}}%
\pgfpathlineto{\pgfqpoint{2.804803in}{2.171536in}}%
\pgfpathlineto{\pgfqpoint{2.807511in}{2.177796in}}%
\pgfpathlineto{\pgfqpoint{2.810100in}{2.183674in}}%
\pgfpathlineto{\pgfqpoint{2.810276in}{2.184057in}}%
\pgfpathlineto{\pgfqpoint{2.813109in}{2.190317in}}%
\pgfpathlineto{\pgfqpoint{2.815989in}{2.196578in}}%
\pgfpathlineto{\pgfqpoint{2.816361in}{2.197401in}}%
\pgfpathlineto{\pgfqpoint{2.818939in}{2.202838in}}%
\pgfpathlineto{\pgfqpoint{2.821930in}{2.209098in}}%
\pgfpathlineto{\pgfqpoint{2.822621in}{2.210576in}}%
\pgfpathlineto{\pgfqpoint{2.824988in}{2.215359in}}%
\pgfpathlineto{\pgfqpoint{2.828081in}{2.221619in}}%
\pgfpathlineto{\pgfqpoint{2.828881in}{2.223286in}}%
\pgfpathlineto{\pgfqpoint{2.831238in}{2.227880in}}%
\pgfpathlineto{\pgfqpoint{2.834418in}{2.234140in}}%
\pgfpathlineto{\pgfqpoint{2.835142in}{2.235624in}}%
\pgfpathlineto{\pgfqpoint{2.837661in}{2.240401in}}%
\pgfpathlineto{\pgfqpoint{2.840907in}{2.246661in}}%
\pgfpathlineto{\pgfqpoint{2.841402in}{2.247675in}}%
\pgfpathlineto{\pgfqpoint{2.844216in}{2.252922in}}%
\pgfpathlineto{\pgfqpoint{2.847492in}{2.259182in}}%
\pgfpathlineto{\pgfqpoint{2.847663in}{2.259538in}}%
\pgfpathlineto{\pgfqpoint{2.850836in}{2.265442in}}%
\pgfpathlineto{\pgfqpoint{2.853923in}{2.271416in}}%
\pgfpathlineto{\pgfqpoint{2.854092in}{2.271703in}}%
\pgfpathlineto{\pgfqpoint{2.857396in}{2.277963in}}%
\pgfpathlineto{\pgfqpoint{2.860184in}{2.283627in}}%
\pgfpathlineto{\pgfqpoint{2.860531in}{2.284224in}}%
\pgfpathlineto{\pgfqpoint{2.863637in}{2.290484in}}%
\pgfpathlineto{\pgfqpoint{2.866365in}{2.296745in}}%
\pgfpathlineto{\pgfqpoint{2.866444in}{2.296997in}}%
\pgfpathlineto{\pgfqpoint{2.868867in}{2.303005in}}%
\pgfpathlineto{\pgfqpoint{2.870542in}{2.309265in}}%
\pgfpathlineto{\pgfqpoint{2.870882in}{2.315526in}}%
\pgfpathlineto{\pgfqpoint{2.869037in}{2.321786in}}%
\pgfpathlineto{\pgfqpoint{2.866444in}{2.325025in}}%
\pgfpathlineto{\pgfqpoint{2.864234in}{2.328047in}}%
\pgfpathlineto{\pgfqpoint{2.860184in}{2.331376in}}%
\pgfpathlineto{\pgfqpoint{2.856965in}{2.334307in}}%
\pgfpathlineto{\pgfqpoint{2.853923in}{2.336330in}}%
\pgfpathlineto{\pgfqpoint{2.848231in}{2.340568in}}%
\pgfpathlineto{\pgfqpoint{2.847663in}{2.340907in}}%
\pgfpathlineto{\pgfqpoint{2.841402in}{2.345207in}}%
\pgfpathlineto{\pgfqpoint{2.839281in}{2.346828in}}%
\pgfpathlineto{\pgfqpoint{2.835142in}{2.349560in}}%
\pgfpathlineto{\pgfqpoint{2.830342in}{2.353088in}}%
\pgfpathlineto{\pgfqpoint{2.828881in}{2.354049in}}%
\pgfpathlineto{\pgfqpoint{2.822621in}{2.358660in}}%
\pgfpathlineto{\pgfqpoint{2.821777in}{2.359349in}}%
\pgfpathlineto{\pgfqpoint{2.816361in}{2.363444in}}%
\pgfpathlineto{\pgfqpoint{2.813770in}{2.365609in}}%
\pgfpathlineto{\pgfqpoint{2.810100in}{2.368510in}}%
\pgfpathlineto{\pgfqpoint{2.806253in}{2.371870in}}%
\pgfpathlineto{\pgfqpoint{2.803840in}{2.373898in}}%
\pgfpathlineto{\pgfqpoint{2.799272in}{2.378130in}}%
\pgfpathlineto{\pgfqpoint{2.797579in}{2.379662in}}%
\pgfpathlineto{\pgfqpoint{2.792827in}{2.384391in}}%
\pgfpathlineto{\pgfqpoint{2.791319in}{2.385877in}}%
\pgfpathlineto{\pgfqpoint{2.786896in}{2.390651in}}%
\pgfpathlineto{\pgfqpoint{2.785058in}{2.392640in}}%
\pgfpathlineto{\pgfqpoint{2.781441in}{2.396912in}}%
\pgfpathlineto{\pgfqpoint{2.778798in}{2.400081in}}%
\pgfpathlineto{\pgfqpoint{2.776424in}{2.403172in}}%
\pgfpathlineto{\pgfqpoint{2.772538in}{2.408373in}}%
\pgfpathlineto{\pgfqpoint{2.771805in}{2.409432in}}%
\pgfpathlineto{\pgfqpoint{2.767601in}{2.415693in}}%
\pgfpathlineto{\pgfqpoint{2.766277in}{2.417746in}}%
\pgfpathlineto{\pgfqpoint{2.763754in}{2.421953in}}%
\pgfpathlineto{\pgfqpoint{2.760208in}{2.428214in}}%
\pgfpathlineto{\pgfqpoint{2.760017in}{2.428566in}}%
\pgfpathlineto{\pgfqpoint{2.757015in}{2.434474in}}%
\pgfpathlineto{\pgfqpoint{2.754069in}{2.440735in}}%
\pgfpathlineto{\pgfqpoint{2.753756in}{2.441435in}}%
\pgfpathlineto{\pgfqpoint{2.751421in}{2.446995in}}%
\pgfpathlineto{\pgfqpoint{2.749001in}{2.453255in}}%
\pgfpathlineto{\pgfqpoint{2.747496in}{2.457483in}}%
\pgfpathlineto{\pgfqpoint{2.746809in}{2.459516in}}%
\pgfpathlineto{\pgfqpoint{2.744846in}{2.465776in}}%
\pgfpathlineto{\pgfqpoint{2.743072in}{2.472037in}}%
\pgfpathlineto{\pgfqpoint{2.741478in}{2.478297in}}%
\pgfpathlineto{\pgfqpoint{2.741235in}{2.479340in}}%
\pgfpathlineto{\pgfqpoint{2.740071in}{2.484558in}}%
\pgfpathlineto{\pgfqpoint{2.738826in}{2.490818in}}%
\pgfpathlineto{\pgfqpoint{2.737730in}{2.497078in}}%
\pgfpathlineto{\pgfqpoint{2.736776in}{2.503339in}}%
\pgfpathlineto{\pgfqpoint{2.735957in}{2.509599in}}%
\pgfpathlineto{\pgfqpoint{2.735264in}{2.515860in}}%
\pgfpathlineto{\pgfqpoint{2.734975in}{2.518975in}}%
\pgfpathlineto{\pgfqpoint{2.734692in}{2.522120in}}%
\pgfpathlineto{\pgfqpoint{2.734234in}{2.528381in}}%
\pgfpathlineto{\pgfqpoint{2.733882in}{2.534641in}}%
\pgfpathlineto{\pgfqpoint{2.733630in}{2.540902in}}%
\pgfpathlineto{\pgfqpoint{2.733474in}{2.547162in}}%
\pgfpathlineto{\pgfqpoint{2.733408in}{2.553422in}}%
\pgfpathlineto{\pgfqpoint{2.733429in}{2.559683in}}%
\pgfpathlineto{\pgfqpoint{2.733533in}{2.565943in}}%
\pgfpathlineto{\pgfqpoint{2.733716in}{2.572204in}}%
\pgfpathlineto{\pgfqpoint{2.733974in}{2.578464in}}%
\pgfpathlineto{\pgfqpoint{2.734305in}{2.584725in}}%
\pgfpathlineto{\pgfqpoint{2.734706in}{2.590985in}}%
\pgfpathlineto{\pgfqpoint{2.734975in}{2.594544in}}%
\pgfpathlineto{\pgfqpoint{2.735175in}{2.597245in}}%
\pgfpathlineto{\pgfqpoint{2.735708in}{2.603506in}}%
\pgfpathlineto{\pgfqpoint{2.736304in}{2.609766in}}%
\pgfpathlineto{\pgfqpoint{2.736961in}{2.616027in}}%
\pgfpathlineto{\pgfqpoint{2.737678in}{2.622287in}}%
\pgfpathlineto{\pgfqpoint{2.738454in}{2.628548in}}%
\pgfpathlineto{\pgfqpoint{2.739288in}{2.634808in}}%
\pgfpathlineto{\pgfqpoint{2.740178in}{2.641068in}}%
\pgfpathlineto{\pgfqpoint{2.741125in}{2.647329in}}%
\pgfpathlineto{\pgfqpoint{2.741235in}{2.648013in}}%
\pgfpathlineto{\pgfqpoint{2.742123in}{2.653589in}}%
\pgfpathlineto{\pgfqpoint{2.743175in}{2.659850in}}%
\pgfpathlineto{\pgfqpoint{2.744281in}{2.666110in}}%
\pgfpathlineto{\pgfqpoint{2.745440in}{2.672371in}}%
\pgfpathlineto{\pgfqpoint{2.746653in}{2.678631in}}%
\pgfpathlineto{\pgfqpoint{2.747496in}{2.682782in}}%
\pgfpathlineto{\pgfqpoint{2.747919in}{2.684891in}}%
\pgfpathlineto{\pgfqpoint{2.749234in}{2.691152in}}%
\pgfpathlineto{\pgfqpoint{2.750603in}{2.697412in}}%
\pgfpathlineto{\pgfqpoint{2.752026in}{2.703673in}}%
\pgfpathlineto{\pgfqpoint{2.753503in}{2.709933in}}%
\pgfpathlineto{\pgfqpoint{2.753756in}{2.710964in}}%
\pgfpathlineto{\pgfqpoint{2.755030in}{2.716194in}}%
\pgfpathlineto{\pgfqpoint{2.756611in}{2.722454in}}%
\pgfpathlineto{\pgfqpoint{2.758247in}{2.728715in}}%
\pgfpathlineto{\pgfqpoint{2.759940in}{2.734975in}}%
\pgfpathlineto{\pgfqpoint{2.760017in}{2.735247in}}%
\pgfpathlineto{\pgfqpoint{2.761685in}{2.741235in}}%
\pgfpathlineto{\pgfqpoint{2.763487in}{2.747496in}}%
\pgfpathlineto{\pgfqpoint{2.765347in}{2.753756in}}%
\pgfpathlineto{\pgfqpoint{2.766277in}{2.756790in}}%
\pgfpathlineto{\pgfqpoint{2.767264in}{2.760017in}}%
\pgfpathlineto{\pgfqpoint{2.769239in}{2.766277in}}%
\pgfpathlineto{\pgfqpoint{2.771274in}{2.772538in}}%
\pgfpathlineto{\pgfqpoint{2.772538in}{2.776308in}}%
\pgfpathlineto{\pgfqpoint{2.773371in}{2.778798in}}%
\pgfpathlineto{\pgfqpoint{2.775529in}{2.785058in}}%
\pgfpathlineto{\pgfqpoint{2.777751in}{2.791319in}}%
\pgfpathlineto{\pgfqpoint{2.778798in}{2.794188in}}%
\pgfpathlineto{\pgfqpoint{2.780038in}{2.797579in}}%
\pgfpathlineto{\pgfqpoint{2.782390in}{2.803840in}}%
\pgfpathlineto{\pgfqpoint{2.784810in}{2.810100in}}%
\pgfpathlineto{\pgfqpoint{2.785058in}{2.810726in}}%
\pgfpathlineto{\pgfqpoint{2.787301in}{2.816361in}}%
\pgfpathlineto{\pgfqpoint{2.789862in}{2.822621in}}%
\pgfpathlineto{\pgfqpoint{2.791319in}{2.826091in}}%
\pgfpathlineto{\pgfqpoint{2.792497in}{2.828881in}}%
\pgfpathlineto{\pgfqpoint{2.795207in}{2.835142in}}%
\pgfpathlineto{\pgfqpoint{2.797579in}{2.840479in}}%
\pgfpathlineto{\pgfqpoint{2.797993in}{2.841402in}}%
\pgfpathlineto{\pgfqpoint{2.800862in}{2.847663in}}%
\pgfpathlineto{\pgfqpoint{2.803808in}{2.853923in}}%
\pgfpathlineto{\pgfqpoint{2.803840in}{2.853988in}}%
\pgfpathlineto{\pgfqpoint{2.806847in}{2.860184in}}%
\pgfpathlineto{\pgfqpoint{2.809966in}{2.866444in}}%
\pgfpathlineto{\pgfqpoint{2.810100in}{2.866709in}}%
\pgfpathlineto{\pgfqpoint{2.813184in}{2.872705in}}%
\pgfpathlineto{\pgfqpoint{2.816361in}{2.878727in}}%
\pgfpathlineto{\pgfqpoint{2.816488in}{2.878965in}}%
\pgfpathlineto{\pgfqpoint{2.819901in}{2.885225in}}%
\pgfpathlineto{\pgfqpoint{2.822621in}{2.890100in}}%
\pgfpathlineto{\pgfqpoint{2.823407in}{2.891486in}}%
\pgfpathlineto{\pgfqpoint{2.827027in}{2.897746in}}%
\pgfpathlineto{\pgfqpoint{2.828881in}{2.900887in}}%
\pgfpathlineto{\pgfqpoint{2.830758in}{2.904007in}}%
\pgfpathlineto{\pgfqpoint{2.834599in}{2.910267in}}%
\pgfpathlineto{\pgfqpoint{2.835142in}{2.911136in}}%
\pgfpathlineto{\pgfqpoint{2.838578in}{2.916528in}}%
\pgfpathlineto{\pgfqpoint{2.841402in}{2.920873in}}%
\pgfpathlineto{\pgfqpoint{2.842677in}{2.922788in}}%
\pgfpathlineto{\pgfqpoint{2.846913in}{2.929048in}}%
\pgfpathlineto{\pgfqpoint{2.847663in}{2.930138in}}%
\pgfpathlineto{\pgfqpoint{2.851310in}{2.935309in}}%
\pgfpathlineto{\pgfqpoint{2.853923in}{2.938952in}}%
\pgfpathlineto{\pgfqpoint{2.855854in}{2.941569in}}%
\pgfpathlineto{\pgfqpoint{2.860184in}{2.947347in}}%
\pgfpathlineto{\pgfqpoint{2.860557in}{2.947830in}}%
\pgfpathlineto{\pgfqpoint{2.865458in}{2.954090in}}%
\pgfpathlineto{\pgfqpoint{2.866444in}{2.955333in}}%
\pgfpathlineto{\pgfqpoint{2.870559in}{2.960351in}}%
\pgfpathlineto{\pgfqpoint{2.872705in}{2.962933in}}%
\pgfpathlineto{\pgfqpoint{2.875872in}{2.966611in}}%
\pgfpathlineto{\pgfqpoint{2.878965in}{2.970161in}}%
\pgfpathlineto{\pgfqpoint{2.881419in}{2.972871in}}%
\pgfpathlineto{\pgfqpoint{2.885225in}{2.977031in}}%
\pgfpathlineto{\pgfqpoint{2.887230in}{2.979132in}}%
\pgfpathlineto{\pgfqpoint{2.891486in}{2.983552in}}%
\pgfpathlineto{\pgfqpoint{2.893339in}{2.985392in}}%
\pgfpathlineto{\pgfqpoint{2.897746in}{2.989734in}}%
\pgfpathlineto{\pgfqpoint{2.899790in}{2.991653in}}%
\pgfpathlineto{\pgfqpoint{2.904007in}{2.995585in}}%
\pgfpathlineto{\pgfqpoint{2.906636in}{2.997913in}}%
\pgfpathlineto{\pgfqpoint{2.910267in}{3.001111in}}%
\pgfpathlineto{\pgfqpoint{2.913943in}{3.004174in}}%
\pgfpathlineto{\pgfqpoint{2.916528in}{3.006317in}}%
\pgfpathlineto{\pgfqpoint{2.921794in}{3.010434in}}%
\pgfpathlineto{\pgfqpoint{2.922788in}{3.011208in}}%
\pgfpathlineto{\pgfqpoint{2.929048in}{3.015786in}}%
\pgfpathlineto{\pgfqpoint{2.930379in}{3.016694in}}%
\pgfpathlineto{\pgfqpoint{2.935309in}{3.020054in}}%
\pgfpathlineto{\pgfqpoint{2.939898in}{3.022955in}}%
\pgfpathlineto{\pgfqpoint{2.941569in}{3.024011in}}%
\pgfpathlineto{\pgfqpoint{2.947830in}{3.027658in}}%
\pgfpathlineto{\pgfqpoint{2.950756in}{3.029215in}}%
\pgfpathlineto{\pgfqpoint{2.954090in}{3.030994in}}%
\pgfpathlineto{\pgfqpoint{2.960351in}{3.034015in}}%
\pgfpathlineto{\pgfqpoint{2.963740in}{3.035476in}}%
\pgfpathlineto{\pgfqpoint{2.966611in}{3.036719in}}%
\pgfpathlineto{\pgfqpoint{2.972871in}{3.039100in}}%
\pgfpathlineto{\pgfqpoint{2.979132in}{3.041154in}}%
\pgfpathlineto{\pgfqpoint{2.981265in}{3.041736in}}%
\pgfpathlineto{\pgfqpoint{2.985392in}{3.042871in}}%
\pgfpathlineto{\pgfqpoint{2.991653in}{3.044244in}}%
\pgfpathlineto{\pgfqpoint{2.997913in}{3.045261in}}%
\pgfpathlineto{\pgfqpoint{3.004174in}{3.045912in}}%
\pgfpathlineto{\pgfqpoint{3.010434in}{3.046180in}}%
\pgfpathlineto{\pgfqpoint{3.016694in}{3.046047in}}%
\pgfpathlineto{\pgfqpoint{3.022955in}{3.045493in}}%
\pgfpathlineto{\pgfqpoint{3.029215in}{3.044495in}}%
\pgfpathlineto{\pgfqpoint{3.035476in}{3.043025in}}%
\pgfpathlineto{\pgfqpoint{3.039597in}{3.041736in}}%
\pgfpathlineto{\pgfqpoint{3.041736in}{3.041045in}}%
\pgfpathlineto{\pgfqpoint{3.047997in}{3.038509in}}%
\pgfpathlineto{\pgfqpoint{3.054079in}{3.035476in}}%
\pgfpathlineto{\pgfqpoint{3.054257in}{3.035384in}}%
\pgfpathlineto{\pgfqpoint{3.060518in}{3.031576in}}%
\pgfpathlineto{\pgfqpoint{3.063842in}{3.029215in}}%
\pgfpathlineto{\pgfqpoint{3.066778in}{3.027030in}}%
\pgfpathlineto{\pgfqpoint{3.071571in}{3.022955in}}%
\pgfpathlineto{\pgfqpoint{3.073038in}{3.021640in}}%
\pgfpathlineto{\pgfqpoint{3.077953in}{3.016694in}}%
\pgfpathlineto{\pgfqpoint{3.079299in}{3.015259in}}%
\pgfpathlineto{\pgfqpoint{3.083387in}{3.010434in}}%
\pgfpathlineto{\pgfqpoint{3.085559in}{3.007700in}}%
\pgfpathlineto{\pgfqpoint{3.088122in}{3.004174in}}%
\pgfpathlineto{\pgfqpoint{3.091820in}{2.998714in}}%
\pgfpathlineto{\pgfqpoint{3.092321in}{2.997913in}}%
\pgfpathlineto{\pgfqpoint{3.096006in}{2.991653in}}%
\pgfpathlineto{\pgfqpoint{3.098080in}{2.987843in}}%
\pgfpathlineto{\pgfqpoint{3.099328in}{2.985392in}}%
\pgfpathlineto{\pgfqpoint{3.102293in}{2.979132in}}%
\pgfpathlineto{\pgfqpoint{3.104341in}{2.974431in}}%
\pgfpathlineto{\pgfqpoint{3.104982in}{2.972871in}}%
\pgfpathlineto{\pgfqpoint{3.107375in}{2.966611in}}%
\pgfpathlineto{\pgfqpoint{3.109555in}{2.960351in}}%
\pgfpathlineto{\pgfqpoint{3.110601in}{2.957073in}}%
\pgfpathlineto{\pgfqpoint{3.111509in}{2.954090in}}%
\pgfpathlineto{\pgfqpoint{3.113254in}{2.947830in}}%
\pgfpathlineto{\pgfqpoint{3.114825in}{2.941569in}}%
\pgfpathlineto{\pgfqpoint{3.116233in}{2.935309in}}%
\pgfpathlineto{\pgfqpoint{3.116861in}{2.932192in}}%
\pgfpathlineto{\pgfqpoint{3.117473in}{2.929048in}}%
\pgfpathlineto{\pgfqpoint{3.118558in}{2.922788in}}%
\pgfpathlineto{\pgfqpoint{3.119506in}{2.916528in}}%
\pgfpathlineto{\pgfqpoint{3.120322in}{2.910267in}}%
\pgfpathlineto{\pgfqpoint{3.121013in}{2.904007in}}%
\pgfpathlineto{\pgfqpoint{3.121584in}{2.897746in}}%
\pgfpathlineto{\pgfqpoint{3.122039in}{2.891486in}}%
\pgfpathlineto{\pgfqpoint{3.122383in}{2.885225in}}%
\pgfpathlineto{\pgfqpoint{3.122620in}{2.878965in}}%
\pgfpathlineto{\pgfqpoint{3.122755in}{2.872705in}}%
\pgfpathlineto{\pgfqpoint{3.122790in}{2.866444in}}%
\pgfpathlineto{\pgfqpoint{3.122728in}{2.860184in}}%
\pgfpathlineto{\pgfqpoint{3.122574in}{2.853923in}}%
\pgfpathlineto{\pgfqpoint{3.122329in}{2.847663in}}%
\pgfpathlineto{\pgfqpoint{3.121997in}{2.841402in}}%
\pgfpathlineto{\pgfqpoint{3.121579in}{2.835142in}}%
\pgfpathlineto{\pgfqpoint{3.121078in}{2.828881in}}%
\pgfpathlineto{\pgfqpoint{3.120495in}{2.822621in}}%
\pgfpathlineto{\pgfqpoint{3.119832in}{2.816361in}}%
\pgfpathlineto{\pgfqpoint{3.119092in}{2.810100in}}%
\pgfpathlineto{\pgfqpoint{3.118274in}{2.803840in}}%
\pgfpathlineto{\pgfqpoint{3.117381in}{2.797579in}}%
\pgfpathlineto{\pgfqpoint{3.116861in}{2.794223in}}%
\pgfpathlineto{\pgfqpoint{3.116413in}{2.791319in}}%
\pgfpathlineto{\pgfqpoint{3.115371in}{2.785058in}}%
\pgfpathlineto{\pgfqpoint{3.114258in}{2.778798in}}%
\pgfpathlineto{\pgfqpoint{3.113074in}{2.772538in}}%
\pgfpathlineto{\pgfqpoint{3.111821in}{2.766277in}}%
\pgfpathlineto{\pgfqpoint{3.110601in}{2.760506in}}%
\pgfpathlineto{\pgfqpoint{3.110498in}{2.760017in}}%
\pgfpathlineto{\pgfqpoint{3.109105in}{2.753756in}}%
\pgfpathlineto{\pgfqpoint{3.107646in}{2.747496in}}%
\pgfpathlineto{\pgfqpoint{3.106121in}{2.741235in}}%
\pgfpathlineto{\pgfqpoint{3.104529in}{2.734975in}}%
\pgfpathlineto{\pgfqpoint{3.104341in}{2.734263in}}%
\pgfpathlineto{\pgfqpoint{3.102871in}{2.728715in}}%
\pgfpathlineto{\pgfqpoint{3.101149in}{2.722454in}}%
\pgfpathlineto{\pgfqpoint{3.099362in}{2.716194in}}%
\pgfpathlineto{\pgfqpoint{3.098080in}{2.711852in}}%
\pgfpathlineto{\pgfqpoint{3.097512in}{2.709933in}}%
\pgfpathlineto{\pgfqpoint{3.095598in}{2.703673in}}%
\pgfpathlineto{\pgfqpoint{3.093622in}{2.697412in}}%
\pgfpathlineto{\pgfqpoint{3.091820in}{2.691873in}}%
\pgfpathlineto{\pgfqpoint{3.091584in}{2.691152in}}%
\pgfpathlineto{\pgfqpoint{3.089483in}{2.684891in}}%
\pgfpathlineto{\pgfqpoint{3.087322in}{2.678631in}}%
\pgfpathlineto{\pgfqpoint{3.085559in}{2.673656in}}%
\pgfpathlineto{\pgfqpoint{3.085101in}{2.672371in}}%
\pgfpathlineto{\pgfqpoint{3.082817in}{2.666110in}}%
\pgfpathlineto{\pgfqpoint{3.080476in}{2.659850in}}%
\pgfpathlineto{\pgfqpoint{3.079299in}{2.656766in}}%
\pgfpathlineto{\pgfqpoint{3.078075in}{2.653589in}}%
\pgfpathlineto{\pgfqpoint{3.075614in}{2.647329in}}%
\pgfpathlineto{\pgfqpoint{3.073099in}{2.641068in}}%
\pgfpathlineto{\pgfqpoint{3.073038in}{2.640920in}}%
\pgfpathlineto{\pgfqpoint{3.070520in}{2.634808in}}%
\pgfpathlineto{\pgfqpoint{3.067888in}{2.628548in}}%
\pgfpathlineto{\pgfqpoint{3.066778in}{2.625945in}}%
\pgfpathlineto{\pgfqpoint{3.065197in}{2.622287in}}%
\pgfpathlineto{\pgfqpoint{3.062451in}{2.616027in}}%
\pgfpathlineto{\pgfqpoint{3.060518in}{2.611683in}}%
\pgfpathlineto{\pgfqpoint{3.059651in}{2.609766in}}%
\pgfpathlineto{\pgfqpoint{3.056792in}{2.603506in}}%
\pgfpathlineto{\pgfqpoint{3.054257in}{2.598034in}}%
\pgfpathlineto{\pgfqpoint{3.053885in}{2.597245in}}%
\pgfpathlineto{\pgfqpoint{3.050916in}{2.590985in}}%
\pgfpathlineto{\pgfqpoint{3.047997in}{2.584911in}}%
\pgfpathlineto{\pgfqpoint{3.047905in}{2.584725in}}%
\pgfpathlineto{\pgfqpoint{3.044828in}{2.578464in}}%
\pgfpathlineto{\pgfqpoint{3.041736in}{2.572244in}}%
\pgfpathlineto{\pgfqpoint{3.041716in}{2.572204in}}%
\pgfpathlineto{\pgfqpoint{3.038535in}{2.565943in}}%
\pgfpathlineto{\pgfqpoint{3.035476in}{2.559974in}}%
\pgfpathlineto{\pgfqpoint{3.035323in}{2.559683in}}%
\pgfpathlineto{\pgfqpoint{3.032043in}{2.553422in}}%
\pgfpathlineto{\pgfqpoint{3.029215in}{2.548053in}}%
\pgfpathlineto{\pgfqpoint{3.028733in}{2.547162in}}%
\pgfpathlineto{\pgfqpoint{3.025361in}{2.540902in}}%
\pgfpathlineto{\pgfqpoint{3.022955in}{2.536439in}}%
\pgfpathlineto{\pgfqpoint{3.021956in}{2.534641in}}%
\pgfpathlineto{\pgfqpoint{3.018498in}{2.528381in}}%
\pgfpathlineto{\pgfqpoint{3.016694in}{2.525102in}}%
\pgfpathlineto{\pgfqpoint{3.015000in}{2.522120in}}%
\pgfpathlineto{\pgfqpoint{3.011464in}{2.515860in}}%
\pgfpathlineto{\pgfqpoint{3.010434in}{2.514016in}}%
\pgfpathlineto{\pgfqpoint{3.007877in}{2.509599in}}%
\pgfpathlineto{\pgfqpoint{3.004274in}{2.503339in}}%
\pgfpathlineto{\pgfqpoint{3.004174in}{2.503161in}}%
\pgfpathlineto{\pgfqpoint{3.000600in}{2.497078in}}%
\pgfpathlineto{\pgfqpoint{2.997913in}{2.492459in}}%
\pgfpathlineto{\pgfqpoint{2.996917in}{2.490818in}}%
\pgfpathlineto{\pgfqpoint{2.993187in}{2.484558in}}%
\pgfpathlineto{\pgfqpoint{2.991653in}{2.481934in}}%
\pgfpathlineto{\pgfqpoint{2.989424in}{2.478297in}}%
\pgfpathlineto{\pgfqpoint{2.985656in}{2.472037in}}%
\pgfpathlineto{\pgfqpoint{2.985392in}{2.471586in}}%
\pgfpathlineto{\pgfqpoint{2.981819in}{2.465776in}}%
\pgfpathlineto{\pgfqpoint{2.979132in}{2.461314in}}%
\pgfpathlineto{\pgfqpoint{2.977988in}{2.459516in}}%
\pgfpathlineto{\pgfqpoint{2.974124in}{2.453255in}}%
\pgfpathlineto{\pgfqpoint{2.972871in}{2.451159in}}%
\pgfpathlineto{\pgfqpoint{2.970229in}{2.446995in}}%
\pgfpathlineto{\pgfqpoint{2.966611in}{2.441123in}}%
\pgfpathlineto{\pgfqpoint{2.966355in}{2.440735in}}%
\pgfpathlineto{\pgfqpoint{2.962416in}{2.434474in}}%
\pgfpathlineto{\pgfqpoint{2.960351in}{2.431066in}}%
\pgfpathlineto{\pgfqpoint{2.958492in}{2.428214in}}%
\pgfpathlineto{\pgfqpoint{2.954583in}{2.421953in}}%
\pgfpathlineto{\pgfqpoint{2.954090in}{2.421116in}}%
\pgfpathlineto{\pgfqpoint{2.950621in}{2.415693in}}%
\pgfpathlineto{\pgfqpoint{2.947830in}{2.411112in}}%
\pgfpathlineto{\pgfqpoint{2.946710in}{2.409432in}}%
\pgfpathlineto{\pgfqpoint{2.942790in}{2.403172in}}%
\pgfpathlineto{\pgfqpoint{2.941569in}{2.401081in}}%
\pgfpathlineto{\pgfqpoint{2.938870in}{2.396912in}}%
\pgfpathlineto{\pgfqpoint{2.935309in}{2.391040in}}%
\pgfpathlineto{\pgfqpoint{2.935044in}{2.390651in}}%
\pgfpathlineto{\pgfqpoint{2.931165in}{2.384391in}}%
\pgfpathlineto{\pgfqpoint{2.929048in}{2.380669in}}%
\pgfpathlineto{\pgfqpoint{2.927393in}{2.378130in}}%
\pgfpathlineto{\pgfqpoint{2.923714in}{2.371870in}}%
\pgfpathlineto{\pgfqpoint{2.922788in}{2.370065in}}%
\pgfpathlineto{\pgfqpoint{2.920084in}{2.365609in}}%
\pgfpathlineto{\pgfqpoint{2.916740in}{2.359349in}}%
\pgfpathlineto{\pgfqpoint{2.916528in}{2.358850in}}%
\pgfpathlineto{\pgfqpoint{2.913467in}{2.353088in}}%
\pgfpathlineto{\pgfqpoint{2.910769in}{2.346828in}}%
\pgfpathlineto{\pgfqpoint{2.910267in}{2.345024in}}%
\pgfpathlineto{\pgfqpoint{2.908568in}{2.340568in}}%
\pgfpathlineto{\pgfqpoint{2.907561in}{2.334307in}}%
\pgfpathlineto{\pgfqpoint{2.908676in}{2.328047in}}%
\pgfpathlineto{\pgfqpoint{2.910267in}{2.325732in}}%
\pgfpathlineto{\pgfqpoint{2.912785in}{2.321786in}}%
\pgfpathlineto{\pgfqpoint{2.916528in}{2.318767in}}%
\pgfpathlineto{\pgfqpoint{2.920300in}{2.315526in}}%
\pgfpathlineto{\pgfqpoint{2.922788in}{2.314091in}}%
\pgfpathlineto{\pgfqpoint{2.929048in}{2.310156in}}%
\pgfpathlineto{\pgfqpoint{2.930348in}{2.309265in}}%
\pgfpathlineto{\pgfqpoint{2.935309in}{2.306634in}}%
\pgfpathlineto{\pgfqpoint{2.941569in}{2.303131in}}%
\pgfpathlineto{\pgfqpoint{2.941773in}{2.303005in}}%
\pgfpathlineto{\pgfqpoint{2.947830in}{2.299892in}}%
\pgfpathlineto{\pgfqpoint{2.953652in}{2.296745in}}%
\pgfpathlineto{\pgfqpoint{2.954090in}{2.296541in}}%
\pgfpathlineto{\pgfqpoint{2.960351in}{2.293322in}}%
\pgfpathlineto{\pgfqpoint{2.965563in}{2.290484in}}%
\pgfpathlineto{\pgfqpoint{2.966611in}{2.289976in}}%
\pgfpathlineto{\pgfqpoint{2.972871in}{2.286664in}}%
\pgfpathlineto{\pgfqpoint{2.977200in}{2.284224in}}%
\pgfpathlineto{\pgfqpoint{2.979132in}{2.283229in}}%
\pgfpathlineto{\pgfqpoint{2.985392in}{2.279742in}}%
\pgfpathlineto{\pgfqpoint{2.988371in}{2.277963in}}%
\pgfpathlineto{\pgfqpoint{2.991653in}{2.276137in}}%
\pgfpathlineto{\pgfqpoint{2.997913in}{2.272393in}}%
\pgfpathlineto{\pgfqpoint{2.998986in}{2.271703in}}%
\pgfpathlineto{\pgfqpoint{3.004174in}{2.268536in}}%
\pgfpathlineto{\pgfqpoint{3.008898in}{2.265442in}}%
\pgfpathlineto{\pgfqpoint{3.010434in}{2.264477in}}%
\pgfpathlineto{\pgfqpoint{3.016694in}{2.260240in}}%
\pgfpathlineto{\pgfqpoint{3.018150in}{2.259182in}}%
\pgfpathlineto{\pgfqpoint{3.022955in}{2.255783in}}%
\pgfpathlineto{\pgfqpoint{3.026720in}{2.252922in}}%
\pgfpathlineto{\pgfqpoint{3.029215in}{2.251058in}}%
\pgfpathlineto{\pgfqpoint{3.034688in}{2.246661in}}%
\pgfpathlineto{\pgfqpoint{3.035476in}{2.246034in}}%
\pgfpathlineto{\pgfqpoint{3.041736in}{2.240671in}}%
\pgfpathlineto{\pgfqpoint{3.042031in}{2.240401in}}%
\pgfpathlineto{\pgfqpoint{3.047997in}{2.234909in}}%
\pgfpathlineto{\pgfqpoint{3.048777in}{2.234140in}}%
\pgfpathlineto{\pgfqpoint{3.054257in}{2.228671in}}%
\pgfpathlineto{\pgfqpoint{3.054998in}{2.227880in}}%
\pgfpathlineto{\pgfqpoint{3.060518in}{2.221866in}}%
\pgfpathlineto{\pgfqpoint{3.060730in}{2.221619in}}%
\pgfpathlineto{\pgfqpoint{3.065974in}{2.215359in}}%
\pgfpathlineto{\pgfqpoint{3.066778in}{2.214366in}}%
\pgfpathlineto{\pgfqpoint{3.070781in}{2.209098in}}%
\pgfpathlineto{\pgfqpoint{3.073038in}{2.205996in}}%
\pgfpathlineto{\pgfqpoint{3.075202in}{2.202838in}}%
\pgfpathlineto{\pgfqpoint{3.079269in}{2.196578in}}%
\pgfpathlineto{\pgfqpoint{3.079299in}{2.196529in}}%
\pgfpathlineto{\pgfqpoint{3.082933in}{2.190317in}}%
\pgfpathlineto{\pgfqpoint{3.085559in}{2.185502in}}%
\pgfpathlineto{\pgfqpoint{3.086307in}{2.184057in}}%
\pgfpathlineto{\pgfqpoint{3.089344in}{2.177796in}}%
\pgfpathlineto{\pgfqpoint{3.091820in}{2.172240in}}%
\pgfpathlineto{\pgfqpoint{3.092119in}{2.171536in}}%
\pgfpathlineto{\pgfqpoint{3.094585in}{2.165275in}}%
\pgfpathlineto{\pgfqpoint{3.096812in}{2.159015in}}%
\pgfpathlineto{\pgfqpoint{3.098080in}{2.155062in}}%
\pgfpathlineto{\pgfqpoint{3.098791in}{2.152755in}}%
\pgfpathlineto{\pgfqpoint{3.100520in}{2.146494in}}%
\pgfpathlineto{\pgfqpoint{3.102034in}{2.140234in}}%
\pgfpathlineto{\pgfqpoint{3.103338in}{2.133973in}}%
\pgfpathlineto{\pgfqpoint{3.104341in}{2.128279in}}%
\pgfpathlineto{\pgfqpoint{3.104437in}{2.127713in}}%
\pgfpathlineto{\pgfqpoint{3.105326in}{2.121452in}}%
\pgfpathlineto{\pgfqpoint{3.106028in}{2.115192in}}%
\pgfpathlineto{\pgfqpoint{3.106547in}{2.108932in}}%
\pgfpathlineto{\pgfqpoint{3.106889in}{2.102671in}}%
\pgfpathlineto{\pgfqpoint{3.107059in}{2.096411in}}%
\pgfpathlineto{\pgfqpoint{3.107060in}{2.090150in}}%
\pgfpathlineto{\pgfqpoint{3.106899in}{2.083890in}}%
\pgfpathlineto{\pgfqpoint{3.106577in}{2.077629in}}%
\pgfpathlineto{\pgfqpoint{3.106100in}{2.071369in}}%
\pgfpathlineto{\pgfqpoint{3.105469in}{2.065109in}}%
\pgfpathlineto{\pgfqpoint{3.104689in}{2.058848in}}%
\pgfpathlineto{\pgfqpoint{3.104341in}{2.056510in}}%
\pgfpathlineto{\pgfqpoint{3.103756in}{2.052588in}}%
\pgfpathlineto{\pgfqpoint{3.102676in}{2.046327in}}%
\pgfpathlineto{\pgfqpoint{3.101452in}{2.040067in}}%
\pgfpathlineto{\pgfqpoint{3.100086in}{2.033806in}}%
\pgfpathlineto{\pgfqpoint{3.098579in}{2.027546in}}%
\pgfpathlineto{\pgfqpoint{3.098080in}{2.025654in}}%
\pgfpathlineto{\pgfqpoint{3.096925in}{2.021285in}}%
\pgfpathlineto{\pgfqpoint{3.095129in}{2.015025in}}%
\pgfpathlineto{\pgfqpoint{3.093195in}{2.008765in}}%
\pgfpathlineto{\pgfqpoint{3.091820in}{2.004614in}}%
\pgfpathlineto{\pgfqpoint{3.091117in}{2.002504in}}%
\pgfpathlineto{\pgfqpoint{3.088893in}{1.996244in}}%
\pgfpathlineto{\pgfqpoint{3.086530in}{1.989983in}}%
\pgfpathlineto{\pgfqpoint{3.085559in}{1.987553in}}%
\pgfpathlineto{\pgfqpoint{3.084018in}{1.983723in}}%
\pgfpathlineto{\pgfqpoint{3.081359in}{1.977462in}}%
\pgfpathlineto{\pgfqpoint{3.079299in}{1.972852in}}%
\pgfpathlineto{\pgfqpoint{3.078554in}{1.971202in}}%
\pgfpathlineto{\pgfqpoint{3.075592in}{1.964942in}}%
\pgfpathlineto{\pgfqpoint{3.073038in}{1.959790in}}%
\pgfpathlineto{\pgfqpoint{3.072482in}{1.958681in}}%
\pgfpathlineto{\pgfqpoint{3.069206in}{1.952421in}}%
\pgfpathlineto{\pgfqpoint{3.066778in}{1.947975in}}%
\pgfpathlineto{\pgfqpoint{3.065773in}{1.946160in}}%
\pgfpathlineto{\pgfqpoint{3.062169in}{1.939900in}}%
\pgfpathlineto{\pgfqpoint{3.060518in}{1.937138in}}%
\pgfpathlineto{\pgfqpoint{3.058391in}{1.933639in}}%
\pgfpathlineto{\pgfqpoint{3.054444in}{1.927379in}}%
\pgfpathlineto{\pgfqpoint{3.054257in}{1.927092in}}%
\pgfpathlineto{\pgfqpoint{3.050293in}{1.921119in}}%
\pgfpathlineto{\pgfqpoint{3.047997in}{1.917772in}}%
\pgfpathlineto{\pgfqpoint{3.045955in}{1.914858in}}%
\pgfpathlineto{\pgfqpoint{3.041736in}{1.909025in}}%
\pgfpathlineto{\pgfqpoint{3.041420in}{1.908598in}}%
\pgfpathlineto{\pgfqpoint{3.036655in}{1.902337in}}%
\pgfpathlineto{\pgfqpoint{3.035476in}{1.900831in}}%
\pgfpathlineto{\pgfqpoint{3.031658in}{1.896077in}}%
\pgfpathlineto{\pgfqpoint{3.029215in}{1.893115in}}%
\pgfpathlineto{\pgfqpoint{3.026418in}{1.889816in}}%
\pgfpathlineto{\pgfqpoint{3.022955in}{1.885831in}}%
\pgfpathlineto{\pgfqpoint{3.020916in}{1.883556in}}%
\pgfpathlineto{\pgfqpoint{3.016694in}{1.878951in}}%
\pgfpathlineto{\pgfqpoint{3.015125in}{1.877295in}}%
\pgfpathlineto{\pgfqpoint{3.010434in}{1.872448in}}%
\pgfpathlineto{\pgfqpoint{3.009017in}{1.871035in}}%
\pgfpathlineto{\pgfqpoint{3.004174in}{1.866297in}}%
\pgfpathlineto{\pgfqpoint{3.002556in}{1.864775in}}%
\pgfpathlineto{\pgfqpoint{2.997913in}{1.860481in}}%
\pgfpathlineto{\pgfqpoint{2.995698in}{1.858514in}}%
\pgfpathlineto{\pgfqpoint{2.991653in}{1.854980in}}%
\pgfpathlineto{\pgfqpoint{2.988391in}{1.852254in}}%
\pgfpathlineto{\pgfqpoint{2.985392in}{1.849782in}}%
\pgfpathlineto{\pgfqpoint{2.980574in}{1.845993in}}%
\pgfpathlineto{\pgfqpoint{2.979132in}{1.844873in}}%
\pgfpathlineto{\pgfqpoint{2.972871in}{1.840248in}}%
\pgfpathlineto{\pgfqpoint{2.972135in}{1.839733in}}%
\pgfpathlineto{\pgfqpoint{2.966611in}{1.835903in}}%
\pgfpathlineto{\pgfqpoint{2.962895in}{1.833472in}}%
\pgfpathlineto{\pgfqpoint{2.960351in}{1.831822in}}%
\pgfpathlineto{\pgfqpoint{2.954090in}{1.828004in}}%
\pgfpathlineto{\pgfqpoint{2.952702in}{1.827212in}}%
\pgfpathlineto{\pgfqpoint{2.947830in}{1.824449in}}%
\pgfpathlineto{\pgfqpoint{2.941569in}{1.821146in}}%
\pgfpathlineto{\pgfqpoint{2.941170in}{1.820952in}}%
\pgfpathlineto{\pgfqpoint{2.935309in}{1.818106in}}%
\pgfpathlineto{\pgfqpoint{2.929048in}{1.815316in}}%
\pgfpathlineto{\pgfqpoint{2.927505in}{1.814691in}}%
\pgfpathlineto{\pgfqpoint{2.922788in}{1.812785in}}%
\pgfpathlineto{\pgfqpoint{2.916528in}{1.810511in}}%
\pgfpathlineto{\pgfqpoint{2.910267in}{1.808492in}}%
\pgfpathlineto{\pgfqpoint{2.910047in}{1.808431in}}%
\pgfpathlineto{\pgfqpoint{2.904007in}{1.806745in}}%
\pgfpathlineto{\pgfqpoint{2.897746in}{1.805265in}}%
\pgfpathlineto{\pgfqpoint{2.891486in}{1.804057in}}%
\pgfpathlineto{\pgfqpoint{2.885225in}{1.803131in}}%
\pgfpathlineto{\pgfqpoint{2.878965in}{1.802496in}}%
\pgfpathlineto{\pgfqpoint{2.872805in}{1.802170in}}%
\pgfpathlineto{\pgfqpoint{2.872705in}{1.802165in}}%
\pgfpathlineto{\pgfqpoint{2.866444in}{1.802151in}}%
\pgfpathclose%
\pgfusepath{fill}%
\end{pgfscope}%
\begin{pgfscope}%
\pgfpathrectangle{\pgfqpoint{0.500000in}{0.500000in}}{\pgfqpoint{3.750000in}{3.750000in}}%
\pgfusepath{clip}%
\pgfsetbuttcap%
\pgfsetroundjoin%
\definecolor{currentfill}{rgb}{0.451088,0.661207,0.810842}%
\pgfsetfillcolor{currentfill}%
\pgfsetlinewidth{0.000000pt}%
\definecolor{currentstroke}{rgb}{0.000000,0.000000,0.000000}%
\pgfsetstrokecolor{currentstroke}%
\pgfsetdash{}{0pt}%
\pgfpathmoveto{\pgfqpoint{1.939900in}{1.956413in}}%
\pgfpathlineto{\pgfqpoint{1.946160in}{1.954729in}}%
\pgfpathlineto{\pgfqpoint{1.952421in}{1.953710in}}%
\pgfpathlineto{\pgfqpoint{1.958681in}{1.953238in}}%
\pgfpathlineto{\pgfqpoint{1.964942in}{1.953222in}}%
\pgfpathlineto{\pgfqpoint{1.971202in}{1.953589in}}%
\pgfpathlineto{\pgfqpoint{1.977462in}{1.954282in}}%
\pgfpathlineto{\pgfqpoint{1.983723in}{1.955254in}}%
\pgfpathlineto{\pgfqpoint{1.989983in}{1.956470in}}%
\pgfpathlineto{\pgfqpoint{1.996244in}{1.957896in}}%
\pgfpathlineto{\pgfqpoint{1.999228in}{1.958681in}}%
\pgfpathlineto{\pgfqpoint{2.002504in}{1.959630in}}%
\pgfpathlineto{\pgfqpoint{2.008765in}{1.961647in}}%
\pgfpathlineto{\pgfqpoint{2.015025in}{1.963808in}}%
\pgfpathlineto{\pgfqpoint{2.018067in}{1.964942in}}%
\pgfpathlineto{\pgfqpoint{2.021285in}{1.966263in}}%
\pgfpathlineto{\pgfqpoint{2.027546in}{1.968986in}}%
\pgfpathlineto{\pgfqpoint{2.032446in}{1.971202in}}%
\pgfpathlineto{\pgfqpoint{2.033806in}{1.971883in}}%
\pgfpathlineto{\pgfqpoint{2.040067in}{1.975178in}}%
\pgfpathlineto{\pgfqpoint{2.044305in}{1.977462in}}%
\pgfpathlineto{\pgfqpoint{2.046327in}{1.978676in}}%
\pgfpathlineto{\pgfqpoint{2.052588in}{1.982541in}}%
\pgfpathlineto{\pgfqpoint{2.054453in}{1.983723in}}%
\pgfpathlineto{\pgfqpoint{2.058848in}{1.986844in}}%
\pgfpathlineto{\pgfqpoint{2.063230in}{1.989983in}}%
\pgfpathlineto{\pgfqpoint{2.065109in}{1.991513in}}%
\pgfpathlineto{\pgfqpoint{2.070877in}{1.996244in}}%
\pgfpathlineto{\pgfqpoint{2.071369in}{1.996711in}}%
\pgfpathlineto{\pgfqpoint{2.077443in}{2.002504in}}%
\pgfpathlineto{\pgfqpoint{2.077629in}{2.002715in}}%
\pgfpathlineto{\pgfqpoint{2.082969in}{2.008765in}}%
\pgfpathlineto{\pgfqpoint{2.083890in}{2.010039in}}%
\pgfpathlineto{\pgfqpoint{2.087488in}{2.015025in}}%
\pgfpathlineto{\pgfqpoint{2.090150in}{2.019738in}}%
\pgfpathlineto{\pgfqpoint{2.091025in}{2.021285in}}%
\pgfpathlineto{\pgfqpoint{2.093341in}{2.027546in}}%
\pgfpathlineto{\pgfqpoint{2.094345in}{2.033806in}}%
\pgfpathlineto{\pgfqpoint{2.093595in}{2.040067in}}%
\pgfpathlineto{\pgfqpoint{2.090386in}{2.046327in}}%
\pgfpathlineto{\pgfqpoint{2.090150in}{2.046577in}}%
\pgfpathlineto{\pgfqpoint{2.083890in}{2.051562in}}%
\pgfpathlineto{\pgfqpoint{2.082029in}{2.052588in}}%
\pgfpathlineto{\pgfqpoint{2.077629in}{2.054257in}}%
\pgfpathlineto{\pgfqpoint{2.071369in}{2.055819in}}%
\pgfpathlineto{\pgfqpoint{2.065109in}{2.056698in}}%
\pgfpathlineto{\pgfqpoint{2.058848in}{2.057037in}}%
\pgfpathlineto{\pgfqpoint{2.052588in}{2.056939in}}%
\pgfpathlineto{\pgfqpoint{2.046327in}{2.056483in}}%
\pgfpathlineto{\pgfqpoint{2.040067in}{2.055728in}}%
\pgfpathlineto{\pgfqpoint{2.033806in}{2.054720in}}%
\pgfpathlineto{\pgfqpoint{2.027546in}{2.053495in}}%
\pgfpathlineto{\pgfqpoint{2.023597in}{2.052588in}}%
\pgfpathlineto{\pgfqpoint{2.021285in}{2.052009in}}%
\pgfpathlineto{\pgfqpoint{2.015025in}{2.050221in}}%
\pgfpathlineto{\pgfqpoint{2.008765in}{2.048288in}}%
\pgfpathlineto{\pgfqpoint{2.002820in}{2.046327in}}%
\pgfpathlineto{\pgfqpoint{2.002504in}{2.046213in}}%
\pgfpathlineto{\pgfqpoint{1.996244in}{2.043744in}}%
\pgfpathlineto{\pgfqpoint{1.989983in}{2.041187in}}%
\pgfpathlineto{\pgfqpoint{1.987389in}{2.040067in}}%
\pgfpathlineto{\pgfqpoint{1.983723in}{2.038331in}}%
\pgfpathlineto{\pgfqpoint{1.977462in}{2.035261in}}%
\pgfpathlineto{\pgfqpoint{1.974606in}{2.033806in}}%
\pgfpathlineto{\pgfqpoint{1.971202in}{2.031893in}}%
\pgfpathlineto{\pgfqpoint{1.964942in}{2.028292in}}%
\pgfpathlineto{\pgfqpoint{1.963690in}{2.027546in}}%
\pgfpathlineto{\pgfqpoint{1.958681in}{2.024219in}}%
\pgfpathlineto{\pgfqpoint{1.954316in}{2.021285in}}%
\pgfpathlineto{\pgfqpoint{1.952421in}{2.019845in}}%
\pgfpathlineto{\pgfqpoint{1.946160in}{2.015030in}}%
\pgfpathlineto{\pgfqpoint{1.946154in}{2.015025in}}%
\pgfpathlineto{\pgfqpoint{1.939900in}{2.009368in}}%
\pgfpathlineto{\pgfqpoint{1.939240in}{2.008765in}}%
\pgfpathlineto{\pgfqpoint{1.933639in}{2.002768in}}%
\pgfpathlineto{\pgfqpoint{1.933396in}{2.002504in}}%
\pgfpathlineto{\pgfqpoint{1.928704in}{1.996244in}}%
\pgfpathlineto{\pgfqpoint{1.927379in}{1.993929in}}%
\pgfpathlineto{\pgfqpoint{1.925143in}{1.989983in}}%
\pgfpathlineto{\pgfqpoint{1.922804in}{1.983723in}}%
\pgfpathlineto{\pgfqpoint{1.921888in}{1.977462in}}%
\pgfpathlineto{\pgfqpoint{1.922828in}{1.971202in}}%
\pgfpathlineto{\pgfqpoint{1.926253in}{1.964942in}}%
\pgfpathlineto{\pgfqpoint{1.927379in}{1.963806in}}%
\pgfpathlineto{\pgfqpoint{1.933639in}{1.958992in}}%
\pgfpathlineto{\pgfqpoint{1.934187in}{1.958681in}}%
\pgfpathclose%
\pgfusepath{fill}%
\end{pgfscope}%
\begin{pgfscope}%
\pgfpathrectangle{\pgfqpoint{0.500000in}{0.500000in}}{\pgfqpoint{3.750000in}{3.750000in}}%
\pgfusepath{clip}%
\pgfsetbuttcap%
\pgfsetroundjoin%
\definecolor{currentfill}{rgb}{0.451088,0.661207,0.810842}%
\pgfsetfillcolor{currentfill}%
\pgfsetlinewidth{0.000000pt}%
\definecolor{currentstroke}{rgb}{0.000000,0.000000,0.000000}%
\pgfsetstrokecolor{currentstroke}%
\pgfsetdash{}{0pt}%
\pgfpathmoveto{\pgfqpoint{2.246661in}{2.465649in}}%
\pgfpathlineto{\pgfqpoint{2.252922in}{2.464968in}}%
\pgfpathlineto{\pgfqpoint{2.259182in}{2.464533in}}%
\pgfpathlineto{\pgfqpoint{2.265442in}{2.464376in}}%
\pgfpathlineto{\pgfqpoint{2.271703in}{2.464536in}}%
\pgfpathlineto{\pgfqpoint{2.277963in}{2.465061in}}%
\pgfpathlineto{\pgfqpoint{2.282749in}{2.465776in}}%
\pgfpathlineto{\pgfqpoint{2.284224in}{2.466015in}}%
\pgfpathlineto{\pgfqpoint{2.290484in}{2.467499in}}%
\pgfpathlineto{\pgfqpoint{2.296745in}{2.469605in}}%
\pgfpathlineto{\pgfqpoint{2.302122in}{2.472037in}}%
\pgfpathlineto{\pgfqpoint{2.303005in}{2.472485in}}%
\pgfpathlineto{\pgfqpoint{2.309265in}{2.476465in}}%
\pgfpathlineto{\pgfqpoint{2.311577in}{2.478297in}}%
\pgfpathlineto{\pgfqpoint{2.315526in}{2.481965in}}%
\pgfpathlineto{\pgfqpoint{2.317845in}{2.484558in}}%
\pgfpathlineto{\pgfqpoint{2.321786in}{2.489944in}}%
\pgfpathlineto{\pgfqpoint{2.322335in}{2.490818in}}%
\pgfpathlineto{\pgfqpoint{2.325559in}{2.497078in}}%
\pgfpathlineto{\pgfqpoint{2.327957in}{2.503339in}}%
\pgfpathlineto{\pgfqpoint{2.328047in}{2.503644in}}%
\pgfpathlineto{\pgfqpoint{2.329609in}{2.509599in}}%
\pgfpathlineto{\pgfqpoint{2.330715in}{2.515860in}}%
\pgfpathlineto{\pgfqpoint{2.331358in}{2.522120in}}%
\pgfpathlineto{\pgfqpoint{2.331598in}{2.528381in}}%
\pgfpathlineto{\pgfqpoint{2.331485in}{2.534641in}}%
\pgfpathlineto{\pgfqpoint{2.331060in}{2.540902in}}%
\pgfpathlineto{\pgfqpoint{2.330356in}{2.547162in}}%
\pgfpathlineto{\pgfqpoint{2.329399in}{2.553422in}}%
\pgfpathlineto{\pgfqpoint{2.328213in}{2.559683in}}%
\pgfpathlineto{\pgfqpoint{2.328047in}{2.560402in}}%
\pgfpathlineto{\pgfqpoint{2.326793in}{2.565943in}}%
\pgfpathlineto{\pgfqpoint{2.325176in}{2.572204in}}%
\pgfpathlineto{\pgfqpoint{2.323377in}{2.578464in}}%
\pgfpathlineto{\pgfqpoint{2.321786in}{2.583489in}}%
\pgfpathlineto{\pgfqpoint{2.321402in}{2.584725in}}%
\pgfpathlineto{\pgfqpoint{2.319242in}{2.590985in}}%
\pgfpathlineto{\pgfqpoint{2.316932in}{2.597245in}}%
\pgfpathlineto{\pgfqpoint{2.315526in}{2.600786in}}%
\pgfpathlineto{\pgfqpoint{2.314463in}{2.603506in}}%
\pgfpathlineto{\pgfqpoint{2.311839in}{2.609766in}}%
\pgfpathlineto{\pgfqpoint{2.309265in}{2.615602in}}%
\pgfpathlineto{\pgfqpoint{2.309081in}{2.616027in}}%
\pgfpathlineto{\pgfqpoint{2.306164in}{2.622287in}}%
\pgfpathlineto{\pgfqpoint{2.303127in}{2.628548in}}%
\pgfpathlineto{\pgfqpoint{2.303005in}{2.628784in}}%
\pgfpathlineto{\pgfqpoint{2.299937in}{2.634808in}}%
\pgfpathlineto{\pgfqpoint{2.296745in}{2.640851in}}%
\pgfpathlineto{\pgfqpoint{2.296631in}{2.641068in}}%
\pgfpathlineto{\pgfqpoint{2.293178in}{2.647329in}}%
\pgfpathlineto{\pgfqpoint{2.290484in}{2.652042in}}%
\pgfpathlineto{\pgfqpoint{2.289608in}{2.653589in}}%
\pgfpathlineto{\pgfqpoint{2.285904in}{2.659850in}}%
\pgfpathlineto{\pgfqpoint{2.284224in}{2.662585in}}%
\pgfpathlineto{\pgfqpoint{2.282075in}{2.666110in}}%
\pgfpathlineto{\pgfqpoint{2.278128in}{2.672371in}}%
\pgfpathlineto{\pgfqpoint{2.277963in}{2.672621in}}%
\pgfpathlineto{\pgfqpoint{2.274043in}{2.678631in}}%
\pgfpathlineto{\pgfqpoint{2.271703in}{2.682108in}}%
\pgfpathlineto{\pgfqpoint{2.269841in}{2.684891in}}%
\pgfpathlineto{\pgfqpoint{2.265519in}{2.691152in}}%
\pgfpathlineto{\pgfqpoint{2.265442in}{2.691259in}}%
\pgfpathlineto{\pgfqpoint{2.261062in}{2.697412in}}%
\pgfpathlineto{\pgfqpoint{2.259182in}{2.699978in}}%
\pgfpathlineto{\pgfqpoint{2.256485in}{2.703673in}}%
\pgfpathlineto{\pgfqpoint{2.252922in}{2.708418in}}%
\pgfpathlineto{\pgfqpoint{2.251788in}{2.709933in}}%
\pgfpathlineto{\pgfqpoint{2.246966in}{2.716194in}}%
\pgfpathlineto{\pgfqpoint{2.246661in}{2.716578in}}%
\pgfpathlineto{\pgfqpoint{2.242011in}{2.722454in}}%
\pgfpathlineto{\pgfqpoint{2.240401in}{2.724437in}}%
\pgfpathlineto{\pgfqpoint{2.236933in}{2.728715in}}%
\pgfpathlineto{\pgfqpoint{2.234140in}{2.732073in}}%
\pgfpathlineto{\pgfqpoint{2.231730in}{2.734975in}}%
\pgfpathlineto{\pgfqpoint{2.227880in}{2.739497in}}%
\pgfpathlineto{\pgfqpoint{2.226400in}{2.741235in}}%
\pgfpathlineto{\pgfqpoint{2.221619in}{2.746719in}}%
\pgfpathlineto{\pgfqpoint{2.220942in}{2.747496in}}%
\pgfpathlineto{\pgfqpoint{2.215359in}{2.753752in}}%
\pgfpathlineto{\pgfqpoint{2.215355in}{2.753756in}}%
\pgfpathlineto{\pgfqpoint{2.209631in}{2.760017in}}%
\pgfpathlineto{\pgfqpoint{2.209098in}{2.760586in}}%
\pgfpathlineto{\pgfqpoint{2.203774in}{2.766277in}}%
\pgfpathlineto{\pgfqpoint{2.202838in}{2.767257in}}%
\pgfpathlineto{\pgfqpoint{2.197783in}{2.772538in}}%
\pgfpathlineto{\pgfqpoint{2.196578in}{2.773772in}}%
\pgfpathlineto{\pgfqpoint{2.191656in}{2.778798in}}%
\pgfpathlineto{\pgfqpoint{2.190317in}{2.780139in}}%
\pgfpathlineto{\pgfqpoint{2.185390in}{2.785058in}}%
\pgfpathlineto{\pgfqpoint{2.184057in}{2.786365in}}%
\pgfpathlineto{\pgfqpoint{2.178984in}{2.791319in}}%
\pgfpathlineto{\pgfqpoint{2.177796in}{2.792458in}}%
\pgfpathlineto{\pgfqpoint{2.172434in}{2.797579in}}%
\pgfpathlineto{\pgfqpoint{2.171536in}{2.798422in}}%
\pgfpathlineto{\pgfqpoint{2.165738in}{2.803840in}}%
\pgfpathlineto{\pgfqpoint{2.165275in}{2.804265in}}%
\pgfpathlineto{\pgfqpoint{2.159015in}{2.809989in}}%
\pgfpathlineto{\pgfqpoint{2.158892in}{2.810100in}}%
\pgfpathlineto{\pgfqpoint{2.152755in}{2.815590in}}%
\pgfpathlineto{\pgfqpoint{2.151888in}{2.816361in}}%
\pgfpathlineto{\pgfqpoint{2.146494in}{2.821083in}}%
\pgfpathlineto{\pgfqpoint{2.144726in}{2.822621in}}%
\pgfpathlineto{\pgfqpoint{2.140234in}{2.826472in}}%
\pgfpathlineto{\pgfqpoint{2.137405in}{2.828881in}}%
\pgfpathlineto{\pgfqpoint{2.133973in}{2.831762in}}%
\pgfpathlineto{\pgfqpoint{2.129919in}{2.835142in}}%
\pgfpathlineto{\pgfqpoint{2.127713in}{2.836956in}}%
\pgfpathlineto{\pgfqpoint{2.122266in}{2.841402in}}%
\pgfpathlineto{\pgfqpoint{2.121452in}{2.842058in}}%
\pgfpathlineto{\pgfqpoint{2.115192in}{2.847059in}}%
\pgfpathlineto{\pgfqpoint{2.114430in}{2.847663in}}%
\pgfpathlineto{\pgfqpoint{2.108932in}{2.851963in}}%
\pgfpathlineto{\pgfqpoint{2.106404in}{2.853923in}}%
\pgfpathlineto{\pgfqpoint{2.102671in}{2.856783in}}%
\pgfpathlineto{\pgfqpoint{2.098193in}{2.860184in}}%
\pgfpathlineto{\pgfqpoint{2.096411in}{2.861522in}}%
\pgfpathlineto{\pgfqpoint{2.090150in}{2.866177in}}%
\pgfpathlineto{\pgfqpoint{2.089787in}{2.866444in}}%
\pgfpathlineto{\pgfqpoint{2.083890in}{2.870735in}}%
\pgfpathlineto{\pgfqpoint{2.081155in}{2.872705in}}%
\pgfpathlineto{\pgfqpoint{2.077629in}{2.875218in}}%
\pgfpathlineto{\pgfqpoint{2.072318in}{2.878965in}}%
\pgfpathlineto{\pgfqpoint{2.071369in}{2.879628in}}%
\pgfpathlineto{\pgfqpoint{2.065109in}{2.883949in}}%
\pgfpathlineto{\pgfqpoint{2.063236in}{2.885225in}}%
\pgfpathlineto{\pgfqpoint{2.058848in}{2.888191in}}%
\pgfpathlineto{\pgfqpoint{2.053914in}{2.891486in}}%
\pgfpathlineto{\pgfqpoint{2.052588in}{2.892365in}}%
\pgfpathlineto{\pgfqpoint{2.046327in}{2.896454in}}%
\pgfpathlineto{\pgfqpoint{2.044322in}{2.897746in}}%
\pgfpathlineto{\pgfqpoint{2.040067in}{2.900467in}}%
\pgfpathlineto{\pgfqpoint{2.034458in}{2.904007in}}%
\pgfpathlineto{\pgfqpoint{2.033806in}{2.904415in}}%
\pgfpathlineto{\pgfqpoint{2.027546in}{2.908275in}}%
\pgfpathlineto{\pgfqpoint{2.024268in}{2.910267in}}%
\pgfpathlineto{\pgfqpoint{2.021285in}{2.912068in}}%
\pgfpathlineto{\pgfqpoint{2.015025in}{2.915790in}}%
\pgfpathlineto{\pgfqpoint{2.013762in}{2.916528in}}%
\pgfpathlineto{\pgfqpoint{2.008765in}{2.919429in}}%
\pgfpathlineto{\pgfqpoint{2.002892in}{2.922788in}}%
\pgfpathlineto{\pgfqpoint{2.002504in}{2.923009in}}%
\pgfpathlineto{\pgfqpoint{1.996244in}{2.926499in}}%
\pgfpathlineto{\pgfqpoint{1.991595in}{2.929048in}}%
\pgfpathlineto{\pgfqpoint{1.989983in}{2.929928in}}%
\pgfpathlineto{\pgfqpoint{1.983723in}{2.933275in}}%
\pgfpathlineto{\pgfqpoint{1.979849in}{2.935309in}}%
\pgfpathlineto{\pgfqpoint{1.977462in}{2.936556in}}%
\pgfpathlineto{\pgfqpoint{1.971202in}{2.939757in}}%
\pgfpathlineto{\pgfqpoint{1.967585in}{2.941569in}}%
\pgfpathlineto{\pgfqpoint{1.964942in}{2.942889in}}%
\pgfpathlineto{\pgfqpoint{1.958681in}{2.945941in}}%
\pgfpathlineto{\pgfqpoint{1.954722in}{2.947830in}}%
\pgfpathlineto{\pgfqpoint{1.952421in}{2.948923in}}%
\pgfpathlineto{\pgfqpoint{1.946160in}{2.951822in}}%
\pgfpathlineto{\pgfqpoint{1.941152in}{2.954090in}}%
\pgfpathlineto{\pgfqpoint{1.939900in}{2.954655in}}%
\pgfpathlineto{\pgfqpoint{1.933639in}{2.957396in}}%
\pgfpathlineto{\pgfqpoint{1.927379in}{2.960075in}}%
\pgfpathlineto{\pgfqpoint{1.926711in}{2.960351in}}%
\pgfpathlineto{\pgfqpoint{1.921119in}{2.962656in}}%
\pgfpathlineto{\pgfqpoint{1.914858in}{2.965166in}}%
\pgfpathlineto{\pgfqpoint{1.911135in}{2.966611in}}%
\pgfpathlineto{\pgfqpoint{1.908598in}{2.967594in}}%
\pgfpathlineto{\pgfqpoint{1.902337in}{2.969928in}}%
\pgfpathlineto{\pgfqpoint{1.896077in}{2.972190in}}%
\pgfpathlineto{\pgfqpoint{1.894108in}{2.972871in}}%
\pgfpathlineto{\pgfqpoint{1.889816in}{2.974353in}}%
\pgfpathlineto{\pgfqpoint{1.883556in}{2.976425in}}%
\pgfpathlineto{\pgfqpoint{1.877295in}{2.978415in}}%
\pgfpathlineto{\pgfqpoint{1.874918in}{2.979132in}}%
\pgfpathlineto{\pgfqpoint{1.871035in}{2.980300in}}%
\pgfpathlineto{\pgfqpoint{1.864775in}{2.982083in}}%
\pgfpathlineto{\pgfqpoint{1.858514in}{2.983773in}}%
\pgfpathlineto{\pgfqpoint{1.852254in}{2.985366in}}%
\pgfpathlineto{\pgfqpoint{1.852142in}{2.985392in}}%
\pgfpathlineto{\pgfqpoint{1.845993in}{2.986827in}}%
\pgfpathlineto{\pgfqpoint{1.839733in}{2.988180in}}%
\pgfpathlineto{\pgfqpoint{1.833472in}{2.989420in}}%
\pgfpathlineto{\pgfqpoint{1.827212in}{2.990540in}}%
\pgfpathlineto{\pgfqpoint{1.820952in}{2.991533in}}%
\pgfpathlineto{\pgfqpoint{1.820063in}{2.991653in}}%
\pgfpathlineto{\pgfqpoint{1.814691in}{2.992372in}}%
\pgfpathlineto{\pgfqpoint{1.808431in}{2.993065in}}%
\pgfpathlineto{\pgfqpoint{1.802170in}{2.993604in}}%
\pgfpathlineto{\pgfqpoint{1.795910in}{2.993979in}}%
\pgfpathlineto{\pgfqpoint{1.789649in}{2.994177in}}%
\pgfpathlineto{\pgfqpoint{1.783389in}{2.994181in}}%
\pgfpathlineto{\pgfqpoint{1.777129in}{2.993977in}}%
\pgfpathlineto{\pgfqpoint{1.770868in}{2.993545in}}%
\pgfpathlineto{\pgfqpoint{1.764608in}{2.992862in}}%
\pgfpathlineto{\pgfqpoint{1.758347in}{2.991903in}}%
\pgfpathlineto{\pgfqpoint{1.757070in}{2.991653in}}%
\pgfpathlineto{\pgfqpoint{1.752087in}{2.990592in}}%
\pgfpathlineto{\pgfqpoint{1.745826in}{2.988908in}}%
\pgfpathlineto{\pgfqpoint{1.739566in}{2.986818in}}%
\pgfpathlineto{\pgfqpoint{1.735995in}{2.985392in}}%
\pgfpathlineto{\pgfqpoint{1.733306in}{2.984200in}}%
\pgfpathlineto{\pgfqpoint{1.727045in}{2.980927in}}%
\pgfpathlineto{\pgfqpoint{1.724105in}{2.979132in}}%
\pgfpathlineto{\pgfqpoint{1.720785in}{2.976828in}}%
\pgfpathlineto{\pgfqpoint{1.715859in}{2.972871in}}%
\pgfpathlineto{\pgfqpoint{1.714524in}{2.971626in}}%
\pgfpathlineto{\pgfqpoint{1.709788in}{2.966611in}}%
\pgfpathlineto{\pgfqpoint{1.708264in}{2.964681in}}%
\pgfpathlineto{\pgfqpoint{1.705200in}{2.960351in}}%
\pgfpathlineto{\pgfqpoint{1.702003in}{2.954720in}}%
\pgfpathlineto{\pgfqpoint{1.701678in}{2.954090in}}%
\pgfpathlineto{\pgfqpoint{1.699081in}{2.947830in}}%
\pgfpathlineto{\pgfqpoint{1.697164in}{2.941569in}}%
\pgfpathlineto{\pgfqpoint{1.695843in}{2.935309in}}%
\pgfpathlineto{\pgfqpoint{1.695743in}{2.934549in}}%
\pgfpathlineto{\pgfqpoint{1.695066in}{2.929048in}}%
\pgfpathlineto{\pgfqpoint{1.694742in}{2.922788in}}%
\pgfpathlineto{\pgfqpoint{1.694819in}{2.916528in}}%
\pgfpathlineto{\pgfqpoint{1.695260in}{2.910267in}}%
\pgfpathlineto{\pgfqpoint{1.695743in}{2.906395in}}%
\pgfpathlineto{\pgfqpoint{1.696038in}{2.904007in}}%
\pgfpathlineto{\pgfqpoint{1.697133in}{2.897746in}}%
\pgfpathlineto{\pgfqpoint{1.698507in}{2.891486in}}%
\pgfpathlineto{\pgfqpoint{1.700139in}{2.885225in}}%
\pgfpathlineto{\pgfqpoint{1.702003in}{2.878989in}}%
\pgfpathlineto{\pgfqpoint{1.702011in}{2.878965in}}%
\pgfpathlineto{\pgfqpoint{1.704140in}{2.872705in}}%
\pgfpathlineto{\pgfqpoint{1.706477in}{2.866444in}}%
\pgfpathlineto{\pgfqpoint{1.708264in}{2.862058in}}%
\pgfpathlineto{\pgfqpoint{1.709023in}{2.860184in}}%
\pgfpathlineto{\pgfqpoint{1.711780in}{2.853923in}}%
\pgfpathlineto{\pgfqpoint{1.714524in}{2.848073in}}%
\pgfpathlineto{\pgfqpoint{1.714716in}{2.847663in}}%
\pgfpathlineto{\pgfqpoint{1.717856in}{2.841402in}}%
\pgfpathlineto{\pgfqpoint{1.720785in}{2.835853in}}%
\pgfpathlineto{\pgfqpoint{1.721158in}{2.835142in}}%
\pgfpathlineto{\pgfqpoint{1.724650in}{2.828881in}}%
\pgfpathlineto{\pgfqpoint{1.727045in}{2.824779in}}%
\pgfpathlineto{\pgfqpoint{1.728300in}{2.822621in}}%
\pgfpathlineto{\pgfqpoint{1.732117in}{2.816361in}}%
\pgfpathlineto{\pgfqpoint{1.733306in}{2.814497in}}%
\pgfpathlineto{\pgfqpoint{1.736099in}{2.810100in}}%
\pgfpathlineto{\pgfqpoint{1.739566in}{2.804848in}}%
\pgfpathlineto{\pgfqpoint{1.740229in}{2.803840in}}%
\pgfpathlineto{\pgfqpoint{1.744521in}{2.797579in}}%
\pgfpathlineto{\pgfqpoint{1.745826in}{2.795743in}}%
\pgfpathlineto{\pgfqpoint{1.748965in}{2.791319in}}%
\pgfpathlineto{\pgfqpoint{1.752087in}{2.787061in}}%
\pgfpathlineto{\pgfqpoint{1.753553in}{2.785058in}}%
\pgfpathlineto{\pgfqpoint{1.758285in}{2.778798in}}%
\pgfpathlineto{\pgfqpoint{1.758347in}{2.778719in}}%
\pgfpathlineto{\pgfqpoint{1.763174in}{2.772538in}}%
\pgfpathlineto{\pgfqpoint{1.764608in}{2.770753in}}%
\pgfpathlineto{\pgfqpoint{1.768204in}{2.766277in}}%
\pgfpathlineto{\pgfqpoint{1.770868in}{2.763051in}}%
\pgfpathlineto{\pgfqpoint{1.773375in}{2.760017in}}%
\pgfpathlineto{\pgfqpoint{1.777129in}{2.755592in}}%
\pgfpathlineto{\pgfqpoint{1.778687in}{2.753756in}}%
\pgfpathlineto{\pgfqpoint{1.783389in}{2.748357in}}%
\pgfpathlineto{\pgfqpoint{1.784140in}{2.747496in}}%
\pgfpathlineto{\pgfqpoint{1.789649in}{2.741331in}}%
\pgfpathlineto{\pgfqpoint{1.789735in}{2.741235in}}%
\pgfpathlineto{\pgfqpoint{1.795475in}{2.734975in}}%
\pgfpathlineto{\pgfqpoint{1.795910in}{2.734511in}}%
\pgfpathlineto{\pgfqpoint{1.801358in}{2.728715in}}%
\pgfpathlineto{\pgfqpoint{1.802170in}{2.727868in}}%
\pgfpathlineto{\pgfqpoint{1.807382in}{2.722454in}}%
\pgfpathlineto{\pgfqpoint{1.808431in}{2.721387in}}%
\pgfpathlineto{\pgfqpoint{1.813549in}{2.716194in}}%
\pgfpathlineto{\pgfqpoint{1.814691in}{2.715057in}}%
\pgfpathlineto{\pgfqpoint{1.819860in}{2.709933in}}%
\pgfpathlineto{\pgfqpoint{1.820952in}{2.708870in}}%
\pgfpathlineto{\pgfqpoint{1.826314in}{2.703673in}}%
\pgfpathlineto{\pgfqpoint{1.827212in}{2.702818in}}%
\pgfpathlineto{\pgfqpoint{1.832914in}{2.697412in}}%
\pgfpathlineto{\pgfqpoint{1.833472in}{2.696892in}}%
\pgfpathlineto{\pgfqpoint{1.839662in}{2.691152in}}%
\pgfpathlineto{\pgfqpoint{1.839733in}{2.691087in}}%
\pgfpathlineto{\pgfqpoint{1.845993in}{2.685406in}}%
\pgfpathlineto{\pgfqpoint{1.846563in}{2.684891in}}%
\pgfpathlineto{\pgfqpoint{1.852254in}{2.679835in}}%
\pgfpathlineto{\pgfqpoint{1.853616in}{2.678631in}}%
\pgfpathlineto{\pgfqpoint{1.858514in}{2.674367in}}%
\pgfpathlineto{\pgfqpoint{1.860823in}{2.672371in}}%
\pgfpathlineto{\pgfqpoint{1.864775in}{2.668999in}}%
\pgfpathlineto{\pgfqpoint{1.868183in}{2.666110in}}%
\pgfpathlineto{\pgfqpoint{1.871035in}{2.663724in}}%
\pgfpathlineto{\pgfqpoint{1.875699in}{2.659850in}}%
\pgfpathlineto{\pgfqpoint{1.877295in}{2.658540in}}%
\pgfpathlineto{\pgfqpoint{1.883373in}{2.653589in}}%
\pgfpathlineto{\pgfqpoint{1.883556in}{2.653442in}}%
\pgfpathlineto{\pgfqpoint{1.889816in}{2.648443in}}%
\pgfpathlineto{\pgfqpoint{1.891223in}{2.647329in}}%
\pgfpathlineto{\pgfqpoint{1.896077in}{2.643525in}}%
\pgfpathlineto{\pgfqpoint{1.899239in}{2.641068in}}%
\pgfpathlineto{\pgfqpoint{1.902337in}{2.638684in}}%
\pgfpathlineto{\pgfqpoint{1.907418in}{2.634808in}}%
\pgfpathlineto{\pgfqpoint{1.908598in}{2.633916in}}%
\pgfpathlineto{\pgfqpoint{1.914858in}{2.629227in}}%
\pgfpathlineto{\pgfqpoint{1.915775in}{2.628548in}}%
\pgfpathlineto{\pgfqpoint{1.921119in}{2.624618in}}%
\pgfpathlineto{\pgfqpoint{1.924319in}{2.622287in}}%
\pgfpathlineto{\pgfqpoint{1.927379in}{2.620074in}}%
\pgfpathlineto{\pgfqpoint{1.933031in}{2.616027in}}%
\pgfpathlineto{\pgfqpoint{1.933639in}{2.615594in}}%
\pgfpathlineto{\pgfqpoint{1.939900in}{2.611191in}}%
\pgfpathlineto{\pgfqpoint{1.941948in}{2.609766in}}%
\pgfpathlineto{\pgfqpoint{1.946160in}{2.606851in}}%
\pgfpathlineto{\pgfqpoint{1.951047in}{2.603506in}}%
\pgfpathlineto{\pgfqpoint{1.952421in}{2.602569in}}%
\pgfpathlineto{\pgfqpoint{1.958681in}{2.598354in}}%
\pgfpathlineto{\pgfqpoint{1.960349in}{2.597245in}}%
\pgfpathlineto{\pgfqpoint{1.964942in}{2.594201in}}%
\pgfpathlineto{\pgfqpoint{1.969851in}{2.590985in}}%
\pgfpathlineto{\pgfqpoint{1.971202in}{2.590101in}}%
\pgfpathlineto{\pgfqpoint{1.977462in}{2.586065in}}%
\pgfpathlineto{\pgfqpoint{1.979569in}{2.584725in}}%
\pgfpathlineto{\pgfqpoint{1.983723in}{2.582084in}}%
\pgfpathlineto{\pgfqpoint{1.989488in}{2.578464in}}%
\pgfpathlineto{\pgfqpoint{1.989983in}{2.578153in}}%
\pgfpathlineto{\pgfqpoint{1.996244in}{2.574285in}}%
\pgfpathlineto{\pgfqpoint{1.999655in}{2.572204in}}%
\pgfpathlineto{\pgfqpoint{2.002504in}{2.570464in}}%
\pgfpathlineto{\pgfqpoint{2.008765in}{2.566695in}}%
\pgfpathlineto{\pgfqpoint{2.010035in}{2.565943in}}%
\pgfpathlineto{\pgfqpoint{2.015025in}{2.562982in}}%
\pgfpathlineto{\pgfqpoint{2.020656in}{2.559683in}}%
\pgfpathlineto{\pgfqpoint{2.021285in}{2.559312in}}%
\pgfpathlineto{\pgfqpoint{2.027546in}{2.555700in}}%
\pgfpathlineto{\pgfqpoint{2.031548in}{2.553422in}}%
\pgfpathlineto{\pgfqpoint{2.033806in}{2.552131in}}%
\pgfpathlineto{\pgfqpoint{2.040067in}{2.548613in}}%
\pgfpathlineto{\pgfqpoint{2.042694in}{2.547162in}}%
\pgfpathlineto{\pgfqpoint{2.046327in}{2.545142in}}%
\pgfpathlineto{\pgfqpoint{2.052588in}{2.541718in}}%
\pgfpathlineto{\pgfqpoint{2.054111in}{2.540902in}}%
\pgfpathlineto{\pgfqpoint{2.058848in}{2.538343in}}%
\pgfpathlineto{\pgfqpoint{2.065109in}{2.535012in}}%
\pgfpathlineto{\pgfqpoint{2.065823in}{2.534641in}}%
\pgfpathlineto{\pgfqpoint{2.071369in}{2.531732in}}%
\pgfpathlineto{\pgfqpoint{2.077629in}{2.528495in}}%
\pgfpathlineto{\pgfqpoint{2.077856in}{2.528381in}}%
\pgfpathlineto{\pgfqpoint{2.083890in}{2.525309in}}%
\pgfpathlineto{\pgfqpoint{2.090150in}{2.522167in}}%
\pgfpathlineto{\pgfqpoint{2.090246in}{2.522120in}}%
\pgfpathlineto{\pgfqpoint{2.096411in}{2.519076in}}%
\pgfpathlineto{\pgfqpoint{2.102671in}{2.516031in}}%
\pgfpathlineto{\pgfqpoint{2.103034in}{2.515860in}}%
\pgfpathlineto{\pgfqpoint{2.108932in}{2.513037in}}%
\pgfpathlineto{\pgfqpoint{2.115192in}{2.510094in}}%
\pgfpathlineto{\pgfqpoint{2.116277in}{2.509599in}}%
\pgfpathlineto{\pgfqpoint{2.121452in}{2.507202in}}%
\pgfpathlineto{\pgfqpoint{2.127713in}{2.504364in}}%
\pgfpathlineto{\pgfqpoint{2.130045in}{2.503339in}}%
\pgfpathlineto{\pgfqpoint{2.133973in}{2.501580in}}%
\pgfpathlineto{\pgfqpoint{2.140234in}{2.498855in}}%
\pgfpathlineto{\pgfqpoint{2.144432in}{2.497078in}}%
\pgfpathlineto{\pgfqpoint{2.146494in}{2.496188in}}%
\pgfpathlineto{\pgfqpoint{2.152755in}{2.493584in}}%
\pgfpathlineto{\pgfqpoint{2.159015in}{2.491046in}}%
\pgfpathlineto{\pgfqpoint{2.159605in}{2.490818in}}%
\pgfpathlineto{\pgfqpoint{2.165275in}{2.488576in}}%
\pgfpathlineto{\pgfqpoint{2.171536in}{2.486181in}}%
\pgfpathlineto{\pgfqpoint{2.175952in}{2.484558in}}%
\pgfpathlineto{\pgfqpoint{2.177796in}{2.483862in}}%
\pgfpathlineto{\pgfqpoint{2.184057in}{2.481627in}}%
\pgfpathlineto{\pgfqpoint{2.190317in}{2.479482in}}%
\pgfpathlineto{\pgfqpoint{2.193978in}{2.478297in}}%
\pgfpathlineto{\pgfqpoint{2.196578in}{2.477431in}}%
\pgfpathlineto{\pgfqpoint{2.202838in}{2.475485in}}%
\pgfpathlineto{\pgfqpoint{2.209098in}{2.473651in}}%
\pgfpathlineto{\pgfqpoint{2.215006in}{2.472037in}}%
\pgfpathlineto{\pgfqpoint{2.215359in}{2.471937in}}%
\pgfpathlineto{\pgfqpoint{2.221619in}{2.470358in}}%
\pgfpathlineto{\pgfqpoint{2.227880in}{2.468924in}}%
\pgfpathlineto{\pgfqpoint{2.234140in}{2.467650in}}%
\pgfpathlineto{\pgfqpoint{2.240401in}{2.466552in}}%
\pgfpathlineto{\pgfqpoint{2.245802in}{2.465776in}}%
\pgfpathclose%
\pgfpathmoveto{\pgfqpoint{2.062545in}{2.659850in}}%
\pgfpathlineto{\pgfqpoint{2.058848in}{2.660674in}}%
\pgfpathlineto{\pgfqpoint{2.052588in}{2.662380in}}%
\pgfpathlineto{\pgfqpoint{2.046327in}{2.664308in}}%
\pgfpathlineto{\pgfqpoint{2.041037in}{2.666110in}}%
\pgfpathlineto{\pgfqpoint{2.040067in}{2.666453in}}%
\pgfpathlineto{\pgfqpoint{2.033806in}{2.668933in}}%
\pgfpathlineto{\pgfqpoint{2.027546in}{2.671537in}}%
\pgfpathlineto{\pgfqpoint{2.025689in}{2.672371in}}%
\pgfpathlineto{\pgfqpoint{2.021285in}{2.674419in}}%
\pgfpathlineto{\pgfqpoint{2.015025in}{2.677452in}}%
\pgfpathlineto{\pgfqpoint{2.012716in}{2.678631in}}%
\pgfpathlineto{\pgfqpoint{2.008765in}{2.680722in}}%
\pgfpathlineto{\pgfqpoint{2.002504in}{2.684133in}}%
\pgfpathlineto{\pgfqpoint{2.001176in}{2.684891in}}%
\pgfpathlineto{\pgfqpoint{1.996244in}{2.687810in}}%
\pgfpathlineto{\pgfqpoint{1.990683in}{2.691152in}}%
\pgfpathlineto{\pgfqpoint{1.989983in}{2.691589in}}%
\pgfpathlineto{\pgfqpoint{1.983723in}{2.695654in}}%
\pgfpathlineto{\pgfqpoint{1.981067in}{2.697412in}}%
\pgfpathlineto{\pgfqpoint{1.977462in}{2.699893in}}%
\pgfpathlineto{\pgfqpoint{1.972046in}{2.703673in}}%
\pgfpathlineto{\pgfqpoint{1.971202in}{2.704286in}}%
\pgfpathlineto{\pgfqpoint{1.964942in}{2.708953in}}%
\pgfpathlineto{\pgfqpoint{1.963651in}{2.709933in}}%
\pgfpathlineto{\pgfqpoint{1.958681in}{2.713869in}}%
\pgfpathlineto{\pgfqpoint{1.955780in}{2.716194in}}%
\pgfpathlineto{\pgfqpoint{1.952421in}{2.719007in}}%
\pgfpathlineto{\pgfqpoint{1.948352in}{2.722454in}}%
\pgfpathlineto{\pgfqpoint{1.946160in}{2.724400in}}%
\pgfpathlineto{\pgfqpoint{1.941354in}{2.728715in}}%
\pgfpathlineto{\pgfqpoint{1.939900in}{2.730087in}}%
\pgfpathlineto{\pgfqpoint{1.934775in}{2.734975in}}%
\pgfpathlineto{\pgfqpoint{1.933639in}{2.736117in}}%
\pgfpathlineto{\pgfqpoint{1.928603in}{2.741235in}}%
\pgfpathlineto{\pgfqpoint{1.927379in}{2.742552in}}%
\pgfpathlineto{\pgfqpoint{1.922828in}{2.747496in}}%
\pgfpathlineto{\pgfqpoint{1.921119in}{2.749470in}}%
\pgfpathlineto{\pgfqpoint{1.917440in}{2.753756in}}%
\pgfpathlineto{\pgfqpoint{1.914858in}{2.756970in}}%
\pgfpathlineto{\pgfqpoint{1.912431in}{2.760017in}}%
\pgfpathlineto{\pgfqpoint{1.908598in}{2.765182in}}%
\pgfpathlineto{\pgfqpoint{1.907791in}{2.766277in}}%
\pgfpathlineto{\pgfqpoint{1.903602in}{2.772538in}}%
\pgfpathlineto{\pgfqpoint{1.902337in}{2.774640in}}%
\pgfpathlineto{\pgfqpoint{1.899854in}{2.778798in}}%
\pgfpathlineto{\pgfqpoint{1.896508in}{2.785058in}}%
\pgfpathlineto{\pgfqpoint{1.896077in}{2.786034in}}%
\pgfpathlineto{\pgfqpoint{1.893756in}{2.791319in}}%
\pgfpathlineto{\pgfqpoint{1.891523in}{2.797579in}}%
\pgfpathlineto{\pgfqpoint{1.889903in}{2.803840in}}%
\pgfpathlineto{\pgfqpoint{1.889816in}{2.804553in}}%
\pgfpathlineto{\pgfqpoint{1.889149in}{2.810100in}}%
\pgfpathlineto{\pgfqpoint{1.889365in}{2.816361in}}%
\pgfpathlineto{\pgfqpoint{1.889816in}{2.818393in}}%
\pgfpathlineto{\pgfqpoint{1.891004in}{2.822621in}}%
\pgfpathlineto{\pgfqpoint{1.894799in}{2.828881in}}%
\pgfpathlineto{\pgfqpoint{1.896077in}{2.830186in}}%
\pgfpathlineto{\pgfqpoint{1.902337in}{2.834674in}}%
\pgfpathlineto{\pgfqpoint{1.903337in}{2.835142in}}%
\pgfpathlineto{\pgfqpoint{1.908598in}{2.836964in}}%
\pgfpathlineto{\pgfqpoint{1.914858in}{2.838168in}}%
\pgfpathlineto{\pgfqpoint{1.921119in}{2.838612in}}%
\pgfpathlineto{\pgfqpoint{1.927379in}{2.838465in}}%
\pgfpathlineto{\pgfqpoint{1.933639in}{2.837847in}}%
\pgfpathlineto{\pgfqpoint{1.939900in}{2.836849in}}%
\pgfpathlineto{\pgfqpoint{1.946160in}{2.835538in}}%
\pgfpathlineto{\pgfqpoint{1.947672in}{2.835142in}}%
\pgfpathlineto{\pgfqpoint{1.952421in}{2.833843in}}%
\pgfpathlineto{\pgfqpoint{1.958681in}{2.831882in}}%
\pgfpathlineto{\pgfqpoint{1.964942in}{2.829734in}}%
\pgfpathlineto{\pgfqpoint{1.967181in}{2.828881in}}%
\pgfpathlineto{\pgfqpoint{1.971202in}{2.827288in}}%
\pgfpathlineto{\pgfqpoint{1.977462in}{2.824632in}}%
\pgfpathlineto{\pgfqpoint{1.981957in}{2.822621in}}%
\pgfpathlineto{\pgfqpoint{1.983723in}{2.821799in}}%
\pgfpathlineto{\pgfqpoint{1.989983in}{2.818695in}}%
\pgfpathlineto{\pgfqpoint{1.994538in}{2.816361in}}%
\pgfpathlineto{\pgfqpoint{1.996244in}{2.815451in}}%
\pgfpathlineto{\pgfqpoint{2.002504in}{2.811953in}}%
\pgfpathlineto{\pgfqpoint{2.005721in}{2.810100in}}%
\pgfpathlineto{\pgfqpoint{2.008765in}{2.808277in}}%
\pgfpathlineto{\pgfqpoint{2.015025in}{2.804434in}}%
\pgfpathlineto{\pgfqpoint{2.015957in}{2.803840in}}%
\pgfpathlineto{\pgfqpoint{2.021285in}{2.800306in}}%
\pgfpathlineto{\pgfqpoint{2.025337in}{2.797579in}}%
\pgfpathlineto{\pgfqpoint{2.027546in}{2.796029in}}%
\pgfpathlineto{\pgfqpoint{2.033806in}{2.791562in}}%
\pgfpathlineto{\pgfqpoint{2.034137in}{2.791319in}}%
\pgfpathlineto{\pgfqpoint{2.040067in}{2.786774in}}%
\pgfpathlineto{\pgfqpoint{2.042278in}{2.785058in}}%
\pgfpathlineto{\pgfqpoint{2.046327in}{2.781775in}}%
\pgfpathlineto{\pgfqpoint{2.049957in}{2.778798in}}%
\pgfpathlineto{\pgfqpoint{2.052588in}{2.776537in}}%
\pgfpathlineto{\pgfqpoint{2.057190in}{2.772538in}}%
\pgfpathlineto{\pgfqpoint{2.058848in}{2.771025in}}%
\pgfpathlineto{\pgfqpoint{2.063995in}{2.766277in}}%
\pgfpathlineto{\pgfqpoint{2.065109in}{2.765196in}}%
\pgfpathlineto{\pgfqpoint{2.070386in}{2.760017in}}%
\pgfpathlineto{\pgfqpoint{2.071369in}{2.758997in}}%
\pgfpathlineto{\pgfqpoint{2.076376in}{2.753756in}}%
\pgfpathlineto{\pgfqpoint{2.077629in}{2.752365in}}%
\pgfpathlineto{\pgfqpoint{2.081978in}{2.747496in}}%
\pgfpathlineto{\pgfqpoint{2.083890in}{2.745218in}}%
\pgfpathlineto{\pgfqpoint{2.087204in}{2.741235in}}%
\pgfpathlineto{\pgfqpoint{2.090150in}{2.737452in}}%
\pgfpathlineto{\pgfqpoint{2.092065in}{2.734975in}}%
\pgfpathlineto{\pgfqpoint{2.096411in}{2.728939in}}%
\pgfpathlineto{\pgfqpoint{2.096571in}{2.728715in}}%
\pgfpathlineto{\pgfqpoint{2.100584in}{2.722454in}}%
\pgfpathlineto{\pgfqpoint{2.102671in}{2.718875in}}%
\pgfpathlineto{\pgfqpoint{2.104225in}{2.716194in}}%
\pgfpathlineto{\pgfqpoint{2.107400in}{2.709933in}}%
\pgfpathlineto{\pgfqpoint{2.108932in}{2.706368in}}%
\pgfpathlineto{\pgfqpoint{2.110082in}{2.703673in}}%
\pgfpathlineto{\pgfqpoint{2.112162in}{2.697412in}}%
\pgfpathlineto{\pgfqpoint{2.113629in}{2.691152in}}%
\pgfpathlineto{\pgfqpoint{2.114302in}{2.684891in}}%
\pgfpathlineto{\pgfqpoint{2.113916in}{2.678631in}}%
\pgfpathlineto{\pgfqpoint{2.112059in}{2.672371in}}%
\pgfpathlineto{\pgfqpoint{2.108932in}{2.667312in}}%
\pgfpathlineto{\pgfqpoint{2.107908in}{2.666110in}}%
\pgfpathlineto{\pgfqpoint{2.102671in}{2.662171in}}%
\pgfpathlineto{\pgfqpoint{2.097694in}{2.659850in}}%
\pgfpathlineto{\pgfqpoint{2.096411in}{2.659408in}}%
\pgfpathlineto{\pgfqpoint{2.090150in}{2.658118in}}%
\pgfpathlineto{\pgfqpoint{2.083890in}{2.657619in}}%
\pgfpathlineto{\pgfqpoint{2.077629in}{2.657731in}}%
\pgfpathlineto{\pgfqpoint{2.071369in}{2.658322in}}%
\pgfpathlineto{\pgfqpoint{2.065109in}{2.659300in}}%
\pgfpathclose%
\pgfusepath{fill}%
\end{pgfscope}%
\begin{pgfscope}%
\pgfpathrectangle{\pgfqpoint{0.500000in}{0.500000in}}{\pgfqpoint{3.750000in}{3.750000in}}%
\pgfusepath{clip}%
\pgfsetbuttcap%
\pgfsetroundjoin%
\definecolor{currentfill}{rgb}{0.183883,0.546498,0.743837}%
\pgfsetfillcolor{currentfill}%
\pgfsetlinewidth{0.000000pt}%
\definecolor{currentstroke}{rgb}{0.000000,0.000000,0.000000}%
\pgfsetstrokecolor{currentstroke}%
\pgfsetdash{}{0pt}%
\pgfpathmoveto{\pgfqpoint{2.866444in}{1.802151in}}%
\pgfpathlineto{\pgfqpoint{2.872705in}{1.802165in}}%
\pgfpathlineto{\pgfqpoint{2.872805in}{1.802170in}}%
\pgfpathlineto{\pgfqpoint{2.878965in}{1.802496in}}%
\pgfpathlineto{\pgfqpoint{2.885225in}{1.803131in}}%
\pgfpathlineto{\pgfqpoint{2.891486in}{1.804057in}}%
\pgfpathlineto{\pgfqpoint{2.897746in}{1.805265in}}%
\pgfpathlineto{\pgfqpoint{2.904007in}{1.806745in}}%
\pgfpathlineto{\pgfqpoint{2.910047in}{1.808431in}}%
\pgfpathlineto{\pgfqpoint{2.910267in}{1.808492in}}%
\pgfpathlineto{\pgfqpoint{2.916528in}{1.810511in}}%
\pgfpathlineto{\pgfqpoint{2.922788in}{1.812785in}}%
\pgfpathlineto{\pgfqpoint{2.927505in}{1.814691in}}%
\pgfpathlineto{\pgfqpoint{2.929048in}{1.815316in}}%
\pgfpathlineto{\pgfqpoint{2.935309in}{1.818106in}}%
\pgfpathlineto{\pgfqpoint{2.941170in}{1.820952in}}%
\pgfpathlineto{\pgfqpoint{2.941569in}{1.821146in}}%
\pgfpathlineto{\pgfqpoint{2.947830in}{1.824449in}}%
\pgfpathlineto{\pgfqpoint{2.952702in}{1.827212in}}%
\pgfpathlineto{\pgfqpoint{2.954090in}{1.828004in}}%
\pgfpathlineto{\pgfqpoint{2.960351in}{1.831822in}}%
\pgfpathlineto{\pgfqpoint{2.962895in}{1.833472in}}%
\pgfpathlineto{\pgfqpoint{2.966611in}{1.835903in}}%
\pgfpathlineto{\pgfqpoint{2.972135in}{1.839733in}}%
\pgfpathlineto{\pgfqpoint{2.972871in}{1.840248in}}%
\pgfpathlineto{\pgfqpoint{2.979132in}{1.844873in}}%
\pgfpathlineto{\pgfqpoint{2.980574in}{1.845993in}}%
\pgfpathlineto{\pgfqpoint{2.985392in}{1.849782in}}%
\pgfpathlineto{\pgfqpoint{2.988391in}{1.852254in}}%
\pgfpathlineto{\pgfqpoint{2.991653in}{1.854980in}}%
\pgfpathlineto{\pgfqpoint{2.995698in}{1.858514in}}%
\pgfpathlineto{\pgfqpoint{2.997913in}{1.860481in}}%
\pgfpathlineto{\pgfqpoint{3.002556in}{1.864775in}}%
\pgfpathlineto{\pgfqpoint{3.004174in}{1.866297in}}%
\pgfpathlineto{\pgfqpoint{3.009017in}{1.871035in}}%
\pgfpathlineto{\pgfqpoint{3.010434in}{1.872448in}}%
\pgfpathlineto{\pgfqpoint{3.015125in}{1.877295in}}%
\pgfpathlineto{\pgfqpoint{3.016694in}{1.878951in}}%
\pgfpathlineto{\pgfqpoint{3.020916in}{1.883556in}}%
\pgfpathlineto{\pgfqpoint{3.022955in}{1.885831in}}%
\pgfpathlineto{\pgfqpoint{3.026418in}{1.889816in}}%
\pgfpathlineto{\pgfqpoint{3.029215in}{1.893115in}}%
\pgfpathlineto{\pgfqpoint{3.031658in}{1.896077in}}%
\pgfpathlineto{\pgfqpoint{3.035476in}{1.900831in}}%
\pgfpathlineto{\pgfqpoint{3.036655in}{1.902337in}}%
\pgfpathlineto{\pgfqpoint{3.041420in}{1.908598in}}%
\pgfpathlineto{\pgfqpoint{3.041736in}{1.909025in}}%
\pgfpathlineto{\pgfqpoint{3.045955in}{1.914858in}}%
\pgfpathlineto{\pgfqpoint{3.047997in}{1.917772in}}%
\pgfpathlineto{\pgfqpoint{3.050293in}{1.921119in}}%
\pgfpathlineto{\pgfqpoint{3.054257in}{1.927092in}}%
\pgfpathlineto{\pgfqpoint{3.054444in}{1.927379in}}%
\pgfpathlineto{\pgfqpoint{3.058391in}{1.933639in}}%
\pgfpathlineto{\pgfqpoint{3.060518in}{1.937138in}}%
\pgfpathlineto{\pgfqpoint{3.062169in}{1.939900in}}%
\pgfpathlineto{\pgfqpoint{3.065773in}{1.946160in}}%
\pgfpathlineto{\pgfqpoint{3.066778in}{1.947975in}}%
\pgfpathlineto{\pgfqpoint{3.069206in}{1.952421in}}%
\pgfpathlineto{\pgfqpoint{3.072482in}{1.958681in}}%
\pgfpathlineto{\pgfqpoint{3.073038in}{1.959790in}}%
\pgfpathlineto{\pgfqpoint{3.075592in}{1.964942in}}%
\pgfpathlineto{\pgfqpoint{3.078554in}{1.971202in}}%
\pgfpathlineto{\pgfqpoint{3.079299in}{1.972852in}}%
\pgfpathlineto{\pgfqpoint{3.081359in}{1.977462in}}%
\pgfpathlineto{\pgfqpoint{3.084018in}{1.983723in}}%
\pgfpathlineto{\pgfqpoint{3.085559in}{1.987553in}}%
\pgfpathlineto{\pgfqpoint{3.086530in}{1.989983in}}%
\pgfpathlineto{\pgfqpoint{3.088893in}{1.996244in}}%
\pgfpathlineto{\pgfqpoint{3.091117in}{2.002504in}}%
\pgfpathlineto{\pgfqpoint{3.091820in}{2.004614in}}%
\pgfpathlineto{\pgfqpoint{3.093195in}{2.008765in}}%
\pgfpathlineto{\pgfqpoint{3.095129in}{2.015025in}}%
\pgfpathlineto{\pgfqpoint{3.096925in}{2.021285in}}%
\pgfpathlineto{\pgfqpoint{3.098080in}{2.025654in}}%
\pgfpathlineto{\pgfqpoint{3.098579in}{2.027546in}}%
\pgfpathlineto{\pgfqpoint{3.100086in}{2.033806in}}%
\pgfpathlineto{\pgfqpoint{3.101452in}{2.040067in}}%
\pgfpathlineto{\pgfqpoint{3.102676in}{2.046327in}}%
\pgfpathlineto{\pgfqpoint{3.103756in}{2.052588in}}%
\pgfpathlineto{\pgfqpoint{3.104341in}{2.056510in}}%
\pgfpathlineto{\pgfqpoint{3.104689in}{2.058848in}}%
\pgfpathlineto{\pgfqpoint{3.105469in}{2.065109in}}%
\pgfpathlineto{\pgfqpoint{3.106100in}{2.071369in}}%
\pgfpathlineto{\pgfqpoint{3.106577in}{2.077629in}}%
\pgfpathlineto{\pgfqpoint{3.106899in}{2.083890in}}%
\pgfpathlineto{\pgfqpoint{3.107060in}{2.090150in}}%
\pgfpathlineto{\pgfqpoint{3.107059in}{2.096411in}}%
\pgfpathlineto{\pgfqpoint{3.106889in}{2.102671in}}%
\pgfpathlineto{\pgfqpoint{3.106547in}{2.108932in}}%
\pgfpathlineto{\pgfqpoint{3.106028in}{2.115192in}}%
\pgfpathlineto{\pgfqpoint{3.105326in}{2.121452in}}%
\pgfpathlineto{\pgfqpoint{3.104437in}{2.127713in}}%
\pgfpathlineto{\pgfqpoint{3.104341in}{2.128279in}}%
\pgfpathlineto{\pgfqpoint{3.103338in}{2.133973in}}%
\pgfpathlineto{\pgfqpoint{3.102034in}{2.140234in}}%
\pgfpathlineto{\pgfqpoint{3.100520in}{2.146494in}}%
\pgfpathlineto{\pgfqpoint{3.098791in}{2.152755in}}%
\pgfpathlineto{\pgfqpoint{3.098080in}{2.155062in}}%
\pgfpathlineto{\pgfqpoint{3.096812in}{2.159015in}}%
\pgfpathlineto{\pgfqpoint{3.094585in}{2.165275in}}%
\pgfpathlineto{\pgfqpoint{3.092119in}{2.171536in}}%
\pgfpathlineto{\pgfqpoint{3.091820in}{2.172240in}}%
\pgfpathlineto{\pgfqpoint{3.089344in}{2.177796in}}%
\pgfpathlineto{\pgfqpoint{3.086307in}{2.184057in}}%
\pgfpathlineto{\pgfqpoint{3.085559in}{2.185502in}}%
\pgfpathlineto{\pgfqpoint{3.082933in}{2.190317in}}%
\pgfpathlineto{\pgfqpoint{3.079299in}{2.196529in}}%
\pgfpathlineto{\pgfqpoint{3.079269in}{2.196578in}}%
\pgfpathlineto{\pgfqpoint{3.075202in}{2.202838in}}%
\pgfpathlineto{\pgfqpoint{3.073038in}{2.205996in}}%
\pgfpathlineto{\pgfqpoint{3.070781in}{2.209098in}}%
\pgfpathlineto{\pgfqpoint{3.066778in}{2.214366in}}%
\pgfpathlineto{\pgfqpoint{3.065974in}{2.215359in}}%
\pgfpathlineto{\pgfqpoint{3.060730in}{2.221619in}}%
\pgfpathlineto{\pgfqpoint{3.060518in}{2.221866in}}%
\pgfpathlineto{\pgfqpoint{3.054998in}{2.227880in}}%
\pgfpathlineto{\pgfqpoint{3.054257in}{2.228671in}}%
\pgfpathlineto{\pgfqpoint{3.048777in}{2.234140in}}%
\pgfpathlineto{\pgfqpoint{3.047997in}{2.234909in}}%
\pgfpathlineto{\pgfqpoint{3.042031in}{2.240401in}}%
\pgfpathlineto{\pgfqpoint{3.041736in}{2.240671in}}%
\pgfpathlineto{\pgfqpoint{3.035476in}{2.246034in}}%
\pgfpathlineto{\pgfqpoint{3.034688in}{2.246661in}}%
\pgfpathlineto{\pgfqpoint{3.029215in}{2.251058in}}%
\pgfpathlineto{\pgfqpoint{3.026720in}{2.252922in}}%
\pgfpathlineto{\pgfqpoint{3.022955in}{2.255783in}}%
\pgfpathlineto{\pgfqpoint{3.018150in}{2.259182in}}%
\pgfpathlineto{\pgfqpoint{3.016694in}{2.260240in}}%
\pgfpathlineto{\pgfqpoint{3.010434in}{2.264477in}}%
\pgfpathlineto{\pgfqpoint{3.008898in}{2.265442in}}%
\pgfpathlineto{\pgfqpoint{3.004174in}{2.268536in}}%
\pgfpathlineto{\pgfqpoint{2.998986in}{2.271703in}}%
\pgfpathlineto{\pgfqpoint{2.997913in}{2.272393in}}%
\pgfpathlineto{\pgfqpoint{2.991653in}{2.276137in}}%
\pgfpathlineto{\pgfqpoint{2.988371in}{2.277963in}}%
\pgfpathlineto{\pgfqpoint{2.985392in}{2.279742in}}%
\pgfpathlineto{\pgfqpoint{2.979132in}{2.283229in}}%
\pgfpathlineto{\pgfqpoint{2.977200in}{2.284224in}}%
\pgfpathlineto{\pgfqpoint{2.972871in}{2.286664in}}%
\pgfpathlineto{\pgfqpoint{2.966611in}{2.289976in}}%
\pgfpathlineto{\pgfqpoint{2.965563in}{2.290484in}}%
\pgfpathlineto{\pgfqpoint{2.960351in}{2.293322in}}%
\pgfpathlineto{\pgfqpoint{2.954090in}{2.296541in}}%
\pgfpathlineto{\pgfqpoint{2.953652in}{2.296745in}}%
\pgfpathlineto{\pgfqpoint{2.947830in}{2.299892in}}%
\pgfpathlineto{\pgfqpoint{2.941773in}{2.303005in}}%
\pgfpathlineto{\pgfqpoint{2.941569in}{2.303131in}}%
\pgfpathlineto{\pgfqpoint{2.935309in}{2.306634in}}%
\pgfpathlineto{\pgfqpoint{2.930348in}{2.309265in}}%
\pgfpathlineto{\pgfqpoint{2.929048in}{2.310156in}}%
\pgfpathlineto{\pgfqpoint{2.922788in}{2.314091in}}%
\pgfpathlineto{\pgfqpoint{2.920300in}{2.315526in}}%
\pgfpathlineto{\pgfqpoint{2.916528in}{2.318767in}}%
\pgfpathlineto{\pgfqpoint{2.912785in}{2.321786in}}%
\pgfpathlineto{\pgfqpoint{2.910267in}{2.325732in}}%
\pgfpathlineto{\pgfqpoint{2.908676in}{2.328047in}}%
\pgfpathlineto{\pgfqpoint{2.907561in}{2.334307in}}%
\pgfpathlineto{\pgfqpoint{2.908568in}{2.340568in}}%
\pgfpathlineto{\pgfqpoint{2.910267in}{2.345024in}}%
\pgfpathlineto{\pgfqpoint{2.910769in}{2.346828in}}%
\pgfpathlineto{\pgfqpoint{2.913467in}{2.353088in}}%
\pgfpathlineto{\pgfqpoint{2.916528in}{2.358850in}}%
\pgfpathlineto{\pgfqpoint{2.916740in}{2.359349in}}%
\pgfpathlineto{\pgfqpoint{2.920084in}{2.365609in}}%
\pgfpathlineto{\pgfqpoint{2.922788in}{2.370065in}}%
\pgfpathlineto{\pgfqpoint{2.923714in}{2.371870in}}%
\pgfpathlineto{\pgfqpoint{2.927393in}{2.378130in}}%
\pgfpathlineto{\pgfqpoint{2.929048in}{2.380669in}}%
\pgfpathlineto{\pgfqpoint{2.931165in}{2.384391in}}%
\pgfpathlineto{\pgfqpoint{2.935044in}{2.390651in}}%
\pgfpathlineto{\pgfqpoint{2.935309in}{2.391040in}}%
\pgfpathlineto{\pgfqpoint{2.938870in}{2.396912in}}%
\pgfpathlineto{\pgfqpoint{2.941569in}{2.401081in}}%
\pgfpathlineto{\pgfqpoint{2.942790in}{2.403172in}}%
\pgfpathlineto{\pgfqpoint{2.946710in}{2.409432in}}%
\pgfpathlineto{\pgfqpoint{2.947830in}{2.411112in}}%
\pgfpathlineto{\pgfqpoint{2.950621in}{2.415693in}}%
\pgfpathlineto{\pgfqpoint{2.954090in}{2.421116in}}%
\pgfpathlineto{\pgfqpoint{2.954583in}{2.421953in}}%
\pgfpathlineto{\pgfqpoint{2.958492in}{2.428214in}}%
\pgfpathlineto{\pgfqpoint{2.960351in}{2.431066in}}%
\pgfpathlineto{\pgfqpoint{2.962416in}{2.434474in}}%
\pgfpathlineto{\pgfqpoint{2.966355in}{2.440735in}}%
\pgfpathlineto{\pgfqpoint{2.966611in}{2.441123in}}%
\pgfpathlineto{\pgfqpoint{2.970229in}{2.446995in}}%
\pgfpathlineto{\pgfqpoint{2.972871in}{2.451159in}}%
\pgfpathlineto{\pgfqpoint{2.974124in}{2.453255in}}%
\pgfpathlineto{\pgfqpoint{2.977988in}{2.459516in}}%
\pgfpathlineto{\pgfqpoint{2.979132in}{2.461314in}}%
\pgfpathlineto{\pgfqpoint{2.981819in}{2.465776in}}%
\pgfpathlineto{\pgfqpoint{2.985392in}{2.471586in}}%
\pgfpathlineto{\pgfqpoint{2.985656in}{2.472037in}}%
\pgfpathlineto{\pgfqpoint{2.989424in}{2.478297in}}%
\pgfpathlineto{\pgfqpoint{2.991653in}{2.481934in}}%
\pgfpathlineto{\pgfqpoint{2.993187in}{2.484558in}}%
\pgfpathlineto{\pgfqpoint{2.996917in}{2.490818in}}%
\pgfpathlineto{\pgfqpoint{2.997913in}{2.492459in}}%
\pgfpathlineto{\pgfqpoint{3.000600in}{2.497078in}}%
\pgfpathlineto{\pgfqpoint{3.004174in}{2.503161in}}%
\pgfpathlineto{\pgfqpoint{3.004274in}{2.503339in}}%
\pgfpathlineto{\pgfqpoint{3.007877in}{2.509599in}}%
\pgfpathlineto{\pgfqpoint{3.010434in}{2.514016in}}%
\pgfpathlineto{\pgfqpoint{3.011464in}{2.515860in}}%
\pgfpathlineto{\pgfqpoint{3.015000in}{2.522120in}}%
\pgfpathlineto{\pgfqpoint{3.016694in}{2.525102in}}%
\pgfpathlineto{\pgfqpoint{3.018498in}{2.528381in}}%
\pgfpathlineto{\pgfqpoint{3.021956in}{2.534641in}}%
\pgfpathlineto{\pgfqpoint{3.022955in}{2.536439in}}%
\pgfpathlineto{\pgfqpoint{3.025361in}{2.540902in}}%
\pgfpathlineto{\pgfqpoint{3.028733in}{2.547162in}}%
\pgfpathlineto{\pgfqpoint{3.029215in}{2.548053in}}%
\pgfpathlineto{\pgfqpoint{3.032043in}{2.553422in}}%
\pgfpathlineto{\pgfqpoint{3.035323in}{2.559683in}}%
\pgfpathlineto{\pgfqpoint{3.035476in}{2.559974in}}%
\pgfpathlineto{\pgfqpoint{3.038535in}{2.565943in}}%
\pgfpathlineto{\pgfqpoint{3.041716in}{2.572204in}}%
\pgfpathlineto{\pgfqpoint{3.041736in}{2.572244in}}%
\pgfpathlineto{\pgfqpoint{3.044828in}{2.578464in}}%
\pgfpathlineto{\pgfqpoint{3.047905in}{2.584725in}}%
\pgfpathlineto{\pgfqpoint{3.047997in}{2.584911in}}%
\pgfpathlineto{\pgfqpoint{3.050916in}{2.590985in}}%
\pgfpathlineto{\pgfqpoint{3.053885in}{2.597245in}}%
\pgfpathlineto{\pgfqpoint{3.054257in}{2.598034in}}%
\pgfpathlineto{\pgfqpoint{3.056792in}{2.603506in}}%
\pgfpathlineto{\pgfqpoint{3.059651in}{2.609766in}}%
\pgfpathlineto{\pgfqpoint{3.060518in}{2.611683in}}%
\pgfpathlineto{\pgfqpoint{3.062451in}{2.616027in}}%
\pgfpathlineto{\pgfqpoint{3.065197in}{2.622287in}}%
\pgfpathlineto{\pgfqpoint{3.066778in}{2.625945in}}%
\pgfpathlineto{\pgfqpoint{3.067888in}{2.628548in}}%
\pgfpathlineto{\pgfqpoint{3.070520in}{2.634808in}}%
\pgfpathlineto{\pgfqpoint{3.073038in}{2.640920in}}%
\pgfpathlineto{\pgfqpoint{3.073099in}{2.641068in}}%
\pgfpathlineto{\pgfqpoint{3.075614in}{2.647329in}}%
\pgfpathlineto{\pgfqpoint{3.078075in}{2.653589in}}%
\pgfpathlineto{\pgfqpoint{3.079299in}{2.656766in}}%
\pgfpathlineto{\pgfqpoint{3.080476in}{2.659850in}}%
\pgfpathlineto{\pgfqpoint{3.082817in}{2.666110in}}%
\pgfpathlineto{\pgfqpoint{3.085101in}{2.672371in}}%
\pgfpathlineto{\pgfqpoint{3.085559in}{2.673656in}}%
\pgfpathlineto{\pgfqpoint{3.087322in}{2.678631in}}%
\pgfpathlineto{\pgfqpoint{3.089483in}{2.684891in}}%
\pgfpathlineto{\pgfqpoint{3.091584in}{2.691152in}}%
\pgfpathlineto{\pgfqpoint{3.091820in}{2.691873in}}%
\pgfpathlineto{\pgfqpoint{3.093622in}{2.697412in}}%
\pgfpathlineto{\pgfqpoint{3.095598in}{2.703673in}}%
\pgfpathlineto{\pgfqpoint{3.097512in}{2.709933in}}%
\pgfpathlineto{\pgfqpoint{3.098080in}{2.711852in}}%
\pgfpathlineto{\pgfqpoint{3.099362in}{2.716194in}}%
\pgfpathlineto{\pgfqpoint{3.101149in}{2.722454in}}%
\pgfpathlineto{\pgfqpoint{3.102871in}{2.728715in}}%
\pgfpathlineto{\pgfqpoint{3.104341in}{2.734263in}}%
\pgfpathlineto{\pgfqpoint{3.104529in}{2.734975in}}%
\pgfpathlineto{\pgfqpoint{3.106121in}{2.741235in}}%
\pgfpathlineto{\pgfqpoint{3.107646in}{2.747496in}}%
\pgfpathlineto{\pgfqpoint{3.109105in}{2.753756in}}%
\pgfpathlineto{\pgfqpoint{3.110498in}{2.760017in}}%
\pgfpathlineto{\pgfqpoint{3.110601in}{2.760506in}}%
\pgfpathlineto{\pgfqpoint{3.111821in}{2.766277in}}%
\pgfpathlineto{\pgfqpoint{3.113074in}{2.772538in}}%
\pgfpathlineto{\pgfqpoint{3.114258in}{2.778798in}}%
\pgfpathlineto{\pgfqpoint{3.115371in}{2.785058in}}%
\pgfpathlineto{\pgfqpoint{3.116413in}{2.791319in}}%
\pgfpathlineto{\pgfqpoint{3.116861in}{2.794223in}}%
\pgfpathlineto{\pgfqpoint{3.117381in}{2.797579in}}%
\pgfpathlineto{\pgfqpoint{3.118274in}{2.803840in}}%
\pgfpathlineto{\pgfqpoint{3.119092in}{2.810100in}}%
\pgfpathlineto{\pgfqpoint{3.119832in}{2.816361in}}%
\pgfpathlineto{\pgfqpoint{3.120495in}{2.822621in}}%
\pgfpathlineto{\pgfqpoint{3.121078in}{2.828881in}}%
\pgfpathlineto{\pgfqpoint{3.121579in}{2.835142in}}%
\pgfpathlineto{\pgfqpoint{3.121997in}{2.841402in}}%
\pgfpathlineto{\pgfqpoint{3.122329in}{2.847663in}}%
\pgfpathlineto{\pgfqpoint{3.122574in}{2.853923in}}%
\pgfpathlineto{\pgfqpoint{3.122728in}{2.860184in}}%
\pgfpathlineto{\pgfqpoint{3.122790in}{2.866444in}}%
\pgfpathlineto{\pgfqpoint{3.122755in}{2.872705in}}%
\pgfpathlineto{\pgfqpoint{3.122620in}{2.878965in}}%
\pgfpathlineto{\pgfqpoint{3.122383in}{2.885225in}}%
\pgfpathlineto{\pgfqpoint{3.122039in}{2.891486in}}%
\pgfpathlineto{\pgfqpoint{3.121584in}{2.897746in}}%
\pgfpathlineto{\pgfqpoint{3.121013in}{2.904007in}}%
\pgfpathlineto{\pgfqpoint{3.120322in}{2.910267in}}%
\pgfpathlineto{\pgfqpoint{3.119506in}{2.916528in}}%
\pgfpathlineto{\pgfqpoint{3.118558in}{2.922788in}}%
\pgfpathlineto{\pgfqpoint{3.117473in}{2.929048in}}%
\pgfpathlineto{\pgfqpoint{3.116861in}{2.932192in}}%
\pgfpathlineto{\pgfqpoint{3.116233in}{2.935309in}}%
\pgfpathlineto{\pgfqpoint{3.114825in}{2.941569in}}%
\pgfpathlineto{\pgfqpoint{3.113254in}{2.947830in}}%
\pgfpathlineto{\pgfqpoint{3.111509in}{2.954090in}}%
\pgfpathlineto{\pgfqpoint{3.110601in}{2.957073in}}%
\pgfpathlineto{\pgfqpoint{3.109555in}{2.960351in}}%
\pgfpathlineto{\pgfqpoint{3.107375in}{2.966611in}}%
\pgfpathlineto{\pgfqpoint{3.104982in}{2.972871in}}%
\pgfpathlineto{\pgfqpoint{3.104341in}{2.974431in}}%
\pgfpathlineto{\pgfqpoint{3.102293in}{2.979132in}}%
\pgfpathlineto{\pgfqpoint{3.099328in}{2.985392in}}%
\pgfpathlineto{\pgfqpoint{3.098080in}{2.987843in}}%
\pgfpathlineto{\pgfqpoint{3.096006in}{2.991653in}}%
\pgfpathlineto{\pgfqpoint{3.092321in}{2.997913in}}%
\pgfpathlineto{\pgfqpoint{3.091820in}{2.998714in}}%
\pgfpathlineto{\pgfqpoint{3.088122in}{3.004174in}}%
\pgfpathlineto{\pgfqpoint{3.085559in}{3.007700in}}%
\pgfpathlineto{\pgfqpoint{3.083387in}{3.010434in}}%
\pgfpathlineto{\pgfqpoint{3.079299in}{3.015259in}}%
\pgfpathlineto{\pgfqpoint{3.077953in}{3.016694in}}%
\pgfpathlineto{\pgfqpoint{3.073038in}{3.021640in}}%
\pgfpathlineto{\pgfqpoint{3.071571in}{3.022955in}}%
\pgfpathlineto{\pgfqpoint{3.066778in}{3.027030in}}%
\pgfpathlineto{\pgfqpoint{3.063842in}{3.029215in}}%
\pgfpathlineto{\pgfqpoint{3.060518in}{3.031576in}}%
\pgfpathlineto{\pgfqpoint{3.054257in}{3.035384in}}%
\pgfpathlineto{\pgfqpoint{3.054079in}{3.035476in}}%
\pgfpathlineto{\pgfqpoint{3.047997in}{3.038509in}}%
\pgfpathlineto{\pgfqpoint{3.041736in}{3.041045in}}%
\pgfpathlineto{\pgfqpoint{3.039597in}{3.041736in}}%
\pgfpathlineto{\pgfqpoint{3.035476in}{3.043025in}}%
\pgfpathlineto{\pgfqpoint{3.029215in}{3.044495in}}%
\pgfpathlineto{\pgfqpoint{3.022955in}{3.045493in}}%
\pgfpathlineto{\pgfqpoint{3.016694in}{3.046047in}}%
\pgfpathlineto{\pgfqpoint{3.010434in}{3.046180in}}%
\pgfpathlineto{\pgfqpoint{3.004174in}{3.045912in}}%
\pgfpathlineto{\pgfqpoint{2.997913in}{3.045261in}}%
\pgfpathlineto{\pgfqpoint{2.991653in}{3.044244in}}%
\pgfpathlineto{\pgfqpoint{2.985392in}{3.042871in}}%
\pgfpathlineto{\pgfqpoint{2.981265in}{3.041736in}}%
\pgfpathlineto{\pgfqpoint{2.979132in}{3.041154in}}%
\pgfpathlineto{\pgfqpoint{2.972871in}{3.039100in}}%
\pgfpathlineto{\pgfqpoint{2.966611in}{3.036719in}}%
\pgfpathlineto{\pgfqpoint{2.963740in}{3.035476in}}%
\pgfpathlineto{\pgfqpoint{2.960351in}{3.034015in}}%
\pgfpathlineto{\pgfqpoint{2.954090in}{3.030994in}}%
\pgfpathlineto{\pgfqpoint{2.950756in}{3.029215in}}%
\pgfpathlineto{\pgfqpoint{2.947830in}{3.027658in}}%
\pgfpathlineto{\pgfqpoint{2.941569in}{3.024011in}}%
\pgfpathlineto{\pgfqpoint{2.939898in}{3.022955in}}%
\pgfpathlineto{\pgfqpoint{2.935309in}{3.020054in}}%
\pgfpathlineto{\pgfqpoint{2.930379in}{3.016694in}}%
\pgfpathlineto{\pgfqpoint{2.929048in}{3.015786in}}%
\pgfpathlineto{\pgfqpoint{2.922788in}{3.011208in}}%
\pgfpathlineto{\pgfqpoint{2.921794in}{3.010434in}}%
\pgfpathlineto{\pgfqpoint{2.916528in}{3.006317in}}%
\pgfpathlineto{\pgfqpoint{2.913943in}{3.004174in}}%
\pgfpathlineto{\pgfqpoint{2.910267in}{3.001111in}}%
\pgfpathlineto{\pgfqpoint{2.906636in}{2.997913in}}%
\pgfpathlineto{\pgfqpoint{2.904007in}{2.995585in}}%
\pgfpathlineto{\pgfqpoint{2.899790in}{2.991653in}}%
\pgfpathlineto{\pgfqpoint{2.897746in}{2.989734in}}%
\pgfpathlineto{\pgfqpoint{2.893339in}{2.985392in}}%
\pgfpathlineto{\pgfqpoint{2.891486in}{2.983552in}}%
\pgfpathlineto{\pgfqpoint{2.887230in}{2.979132in}}%
\pgfpathlineto{\pgfqpoint{2.885225in}{2.977031in}}%
\pgfpathlineto{\pgfqpoint{2.881419in}{2.972871in}}%
\pgfpathlineto{\pgfqpoint{2.878965in}{2.970161in}}%
\pgfpathlineto{\pgfqpoint{2.875872in}{2.966611in}}%
\pgfpathlineto{\pgfqpoint{2.872705in}{2.962933in}}%
\pgfpathlineto{\pgfqpoint{2.870559in}{2.960351in}}%
\pgfpathlineto{\pgfqpoint{2.866444in}{2.955333in}}%
\pgfpathlineto{\pgfqpoint{2.865458in}{2.954090in}}%
\pgfpathlineto{\pgfqpoint{2.860557in}{2.947830in}}%
\pgfpathlineto{\pgfqpoint{2.860184in}{2.947347in}}%
\pgfpathlineto{\pgfqpoint{2.855854in}{2.941569in}}%
\pgfpathlineto{\pgfqpoint{2.853923in}{2.938952in}}%
\pgfpathlineto{\pgfqpoint{2.851310in}{2.935309in}}%
\pgfpathlineto{\pgfqpoint{2.847663in}{2.930138in}}%
\pgfpathlineto{\pgfqpoint{2.846913in}{2.929048in}}%
\pgfpathlineto{\pgfqpoint{2.842677in}{2.922788in}}%
\pgfpathlineto{\pgfqpoint{2.841402in}{2.920873in}}%
\pgfpathlineto{\pgfqpoint{2.838578in}{2.916528in}}%
\pgfpathlineto{\pgfqpoint{2.835142in}{2.911136in}}%
\pgfpathlineto{\pgfqpoint{2.834599in}{2.910267in}}%
\pgfpathlineto{\pgfqpoint{2.830758in}{2.904007in}}%
\pgfpathlineto{\pgfqpoint{2.828881in}{2.900887in}}%
\pgfpathlineto{\pgfqpoint{2.827027in}{2.897746in}}%
\pgfpathlineto{\pgfqpoint{2.823407in}{2.891486in}}%
\pgfpathlineto{\pgfqpoint{2.822621in}{2.890100in}}%
\pgfpathlineto{\pgfqpoint{2.819901in}{2.885225in}}%
\pgfpathlineto{\pgfqpoint{2.816488in}{2.878965in}}%
\pgfpathlineto{\pgfqpoint{2.816361in}{2.878727in}}%
\pgfpathlineto{\pgfqpoint{2.813184in}{2.872705in}}%
\pgfpathlineto{\pgfqpoint{2.810100in}{2.866709in}}%
\pgfpathlineto{\pgfqpoint{2.809966in}{2.866444in}}%
\pgfpathlineto{\pgfqpoint{2.806847in}{2.860184in}}%
\pgfpathlineto{\pgfqpoint{2.803840in}{2.853988in}}%
\pgfpathlineto{\pgfqpoint{2.803808in}{2.853923in}}%
\pgfpathlineto{\pgfqpoint{2.800862in}{2.847663in}}%
\pgfpathlineto{\pgfqpoint{2.797993in}{2.841402in}}%
\pgfpathlineto{\pgfqpoint{2.797579in}{2.840479in}}%
\pgfpathlineto{\pgfqpoint{2.795207in}{2.835142in}}%
\pgfpathlineto{\pgfqpoint{2.792497in}{2.828881in}}%
\pgfpathlineto{\pgfqpoint{2.791319in}{2.826091in}}%
\pgfpathlineto{\pgfqpoint{2.789862in}{2.822621in}}%
\pgfpathlineto{\pgfqpoint{2.787301in}{2.816361in}}%
\pgfpathlineto{\pgfqpoint{2.785058in}{2.810726in}}%
\pgfpathlineto{\pgfqpoint{2.784810in}{2.810100in}}%
\pgfpathlineto{\pgfqpoint{2.782390in}{2.803840in}}%
\pgfpathlineto{\pgfqpoint{2.780038in}{2.797579in}}%
\pgfpathlineto{\pgfqpoint{2.778798in}{2.794188in}}%
\pgfpathlineto{\pgfqpoint{2.777751in}{2.791319in}}%
\pgfpathlineto{\pgfqpoint{2.775529in}{2.785058in}}%
\pgfpathlineto{\pgfqpoint{2.773371in}{2.778798in}}%
\pgfpathlineto{\pgfqpoint{2.772538in}{2.776308in}}%
\pgfpathlineto{\pgfqpoint{2.771274in}{2.772538in}}%
\pgfpathlineto{\pgfqpoint{2.769239in}{2.766277in}}%
\pgfpathlineto{\pgfqpoint{2.767264in}{2.760017in}}%
\pgfpathlineto{\pgfqpoint{2.766277in}{2.756790in}}%
\pgfpathlineto{\pgfqpoint{2.765347in}{2.753756in}}%
\pgfpathlineto{\pgfqpoint{2.763487in}{2.747496in}}%
\pgfpathlineto{\pgfqpoint{2.761685in}{2.741235in}}%
\pgfpathlineto{\pgfqpoint{2.760017in}{2.735247in}}%
\pgfpathlineto{\pgfqpoint{2.759940in}{2.734975in}}%
\pgfpathlineto{\pgfqpoint{2.758247in}{2.728715in}}%
\pgfpathlineto{\pgfqpoint{2.756611in}{2.722454in}}%
\pgfpathlineto{\pgfqpoint{2.755030in}{2.716194in}}%
\pgfpathlineto{\pgfqpoint{2.753756in}{2.710964in}}%
\pgfpathlineto{\pgfqpoint{2.753503in}{2.709933in}}%
\pgfpathlineto{\pgfqpoint{2.752026in}{2.703673in}}%
\pgfpathlineto{\pgfqpoint{2.750603in}{2.697412in}}%
\pgfpathlineto{\pgfqpoint{2.749234in}{2.691152in}}%
\pgfpathlineto{\pgfqpoint{2.747919in}{2.684891in}}%
\pgfpathlineto{\pgfqpoint{2.747496in}{2.682782in}}%
\pgfpathlineto{\pgfqpoint{2.746653in}{2.678631in}}%
\pgfpathlineto{\pgfqpoint{2.745440in}{2.672371in}}%
\pgfpathlineto{\pgfqpoint{2.744281in}{2.666110in}}%
\pgfpathlineto{\pgfqpoint{2.743175in}{2.659850in}}%
\pgfpathlineto{\pgfqpoint{2.742123in}{2.653589in}}%
\pgfpathlineto{\pgfqpoint{2.741235in}{2.648013in}}%
\pgfpathlineto{\pgfqpoint{2.741125in}{2.647329in}}%
\pgfpathlineto{\pgfqpoint{2.740178in}{2.641068in}}%
\pgfpathlineto{\pgfqpoint{2.739288in}{2.634808in}}%
\pgfpathlineto{\pgfqpoint{2.738454in}{2.628548in}}%
\pgfpathlineto{\pgfqpoint{2.737678in}{2.622287in}}%
\pgfpathlineto{\pgfqpoint{2.736961in}{2.616027in}}%
\pgfpathlineto{\pgfqpoint{2.736304in}{2.609766in}}%
\pgfpathlineto{\pgfqpoint{2.735708in}{2.603506in}}%
\pgfpathlineto{\pgfqpoint{2.735175in}{2.597245in}}%
\pgfpathlineto{\pgfqpoint{2.734975in}{2.594544in}}%
\pgfpathlineto{\pgfqpoint{2.734706in}{2.590985in}}%
\pgfpathlineto{\pgfqpoint{2.734305in}{2.584725in}}%
\pgfpathlineto{\pgfqpoint{2.733974in}{2.578464in}}%
\pgfpathlineto{\pgfqpoint{2.733716in}{2.572204in}}%
\pgfpathlineto{\pgfqpoint{2.733533in}{2.565943in}}%
\pgfpathlineto{\pgfqpoint{2.733429in}{2.559683in}}%
\pgfpathlineto{\pgfqpoint{2.733408in}{2.553422in}}%
\pgfpathlineto{\pgfqpoint{2.733474in}{2.547162in}}%
\pgfpathlineto{\pgfqpoint{2.733630in}{2.540902in}}%
\pgfpathlineto{\pgfqpoint{2.733882in}{2.534641in}}%
\pgfpathlineto{\pgfqpoint{2.734234in}{2.528381in}}%
\pgfpathlineto{\pgfqpoint{2.734692in}{2.522120in}}%
\pgfpathlineto{\pgfqpoint{2.734975in}{2.518975in}}%
\pgfpathlineto{\pgfqpoint{2.735264in}{2.515860in}}%
\pgfpathlineto{\pgfqpoint{2.735957in}{2.509599in}}%
\pgfpathlineto{\pgfqpoint{2.736776in}{2.503339in}}%
\pgfpathlineto{\pgfqpoint{2.737730in}{2.497078in}}%
\pgfpathlineto{\pgfqpoint{2.738826in}{2.490818in}}%
\pgfpathlineto{\pgfqpoint{2.740071in}{2.484558in}}%
\pgfpathlineto{\pgfqpoint{2.741235in}{2.479340in}}%
\pgfpathlineto{\pgfqpoint{2.741478in}{2.478297in}}%
\pgfpathlineto{\pgfqpoint{2.743072in}{2.472037in}}%
\pgfpathlineto{\pgfqpoint{2.744846in}{2.465776in}}%
\pgfpathlineto{\pgfqpoint{2.746809in}{2.459516in}}%
\pgfpathlineto{\pgfqpoint{2.747496in}{2.457483in}}%
\pgfpathlineto{\pgfqpoint{2.749001in}{2.453255in}}%
\pgfpathlineto{\pgfqpoint{2.751421in}{2.446995in}}%
\pgfpathlineto{\pgfqpoint{2.753756in}{2.441435in}}%
\pgfpathlineto{\pgfqpoint{2.754069in}{2.440735in}}%
\pgfpathlineto{\pgfqpoint{2.757015in}{2.434474in}}%
\pgfpathlineto{\pgfqpoint{2.760017in}{2.428566in}}%
\pgfpathlineto{\pgfqpoint{2.760208in}{2.428214in}}%
\pgfpathlineto{\pgfqpoint{2.763754in}{2.421953in}}%
\pgfpathlineto{\pgfqpoint{2.766277in}{2.417746in}}%
\pgfpathlineto{\pgfqpoint{2.767601in}{2.415693in}}%
\pgfpathlineto{\pgfqpoint{2.771805in}{2.409432in}}%
\pgfpathlineto{\pgfqpoint{2.772538in}{2.408373in}}%
\pgfpathlineto{\pgfqpoint{2.776424in}{2.403172in}}%
\pgfpathlineto{\pgfqpoint{2.778798in}{2.400081in}}%
\pgfpathlineto{\pgfqpoint{2.781441in}{2.396912in}}%
\pgfpathlineto{\pgfqpoint{2.785058in}{2.392640in}}%
\pgfpathlineto{\pgfqpoint{2.786896in}{2.390651in}}%
\pgfpathlineto{\pgfqpoint{2.791319in}{2.385877in}}%
\pgfpathlineto{\pgfqpoint{2.792827in}{2.384391in}}%
\pgfpathlineto{\pgfqpoint{2.797579in}{2.379662in}}%
\pgfpathlineto{\pgfqpoint{2.799272in}{2.378130in}}%
\pgfpathlineto{\pgfqpoint{2.803840in}{2.373898in}}%
\pgfpathlineto{\pgfqpoint{2.806253in}{2.371870in}}%
\pgfpathlineto{\pgfqpoint{2.810100in}{2.368510in}}%
\pgfpathlineto{\pgfqpoint{2.813770in}{2.365609in}}%
\pgfpathlineto{\pgfqpoint{2.816361in}{2.363444in}}%
\pgfpathlineto{\pgfqpoint{2.821777in}{2.359349in}}%
\pgfpathlineto{\pgfqpoint{2.822621in}{2.358660in}}%
\pgfpathlineto{\pgfqpoint{2.828881in}{2.354049in}}%
\pgfpathlineto{\pgfqpoint{2.830342in}{2.353088in}}%
\pgfpathlineto{\pgfqpoint{2.835142in}{2.349560in}}%
\pgfpathlineto{\pgfqpoint{2.839281in}{2.346828in}}%
\pgfpathlineto{\pgfqpoint{2.841402in}{2.345207in}}%
\pgfpathlineto{\pgfqpoint{2.847663in}{2.340907in}}%
\pgfpathlineto{\pgfqpoint{2.848231in}{2.340568in}}%
\pgfpathlineto{\pgfqpoint{2.853923in}{2.336330in}}%
\pgfpathlineto{\pgfqpoint{2.856965in}{2.334307in}}%
\pgfpathlineto{\pgfqpoint{2.860184in}{2.331376in}}%
\pgfpathlineto{\pgfqpoint{2.864234in}{2.328047in}}%
\pgfpathlineto{\pgfqpoint{2.866444in}{2.325025in}}%
\pgfpathlineto{\pgfqpoint{2.869037in}{2.321786in}}%
\pgfpathlineto{\pgfqpoint{2.870882in}{2.315526in}}%
\pgfpathlineto{\pgfqpoint{2.870542in}{2.309265in}}%
\pgfpathlineto{\pgfqpoint{2.868867in}{2.303005in}}%
\pgfpathlineto{\pgfqpoint{2.866444in}{2.296997in}}%
\pgfpathlineto{\pgfqpoint{2.866365in}{2.296745in}}%
\pgfpathlineto{\pgfqpoint{2.863637in}{2.290484in}}%
\pgfpathlineto{\pgfqpoint{2.860531in}{2.284224in}}%
\pgfpathlineto{\pgfqpoint{2.860184in}{2.283627in}}%
\pgfpathlineto{\pgfqpoint{2.857396in}{2.277963in}}%
\pgfpathlineto{\pgfqpoint{2.854092in}{2.271703in}}%
\pgfpathlineto{\pgfqpoint{2.853923in}{2.271416in}}%
\pgfpathlineto{\pgfqpoint{2.850836in}{2.265442in}}%
\pgfpathlineto{\pgfqpoint{2.847663in}{2.259538in}}%
\pgfpathlineto{\pgfqpoint{2.847492in}{2.259182in}}%
\pgfpathlineto{\pgfqpoint{2.844216in}{2.252922in}}%
\pgfpathlineto{\pgfqpoint{2.841402in}{2.247675in}}%
\pgfpathlineto{\pgfqpoint{2.840907in}{2.246661in}}%
\pgfpathlineto{\pgfqpoint{2.837661in}{2.240401in}}%
\pgfpathlineto{\pgfqpoint{2.835142in}{2.235624in}}%
\pgfpathlineto{\pgfqpoint{2.834418in}{2.234140in}}%
\pgfpathlineto{\pgfqpoint{2.831238in}{2.227880in}}%
\pgfpathlineto{\pgfqpoint{2.828881in}{2.223286in}}%
\pgfpathlineto{\pgfqpoint{2.828081in}{2.221619in}}%
\pgfpathlineto{\pgfqpoint{2.824988in}{2.215359in}}%
\pgfpathlineto{\pgfqpoint{2.822621in}{2.210576in}}%
\pgfpathlineto{\pgfqpoint{2.821930in}{2.209098in}}%
\pgfpathlineto{\pgfqpoint{2.818939in}{2.202838in}}%
\pgfpathlineto{\pgfqpoint{2.816361in}{2.197401in}}%
\pgfpathlineto{\pgfqpoint{2.815989in}{2.196578in}}%
\pgfpathlineto{\pgfqpoint{2.813109in}{2.190317in}}%
\pgfpathlineto{\pgfqpoint{2.810276in}{2.184057in}}%
\pgfpathlineto{\pgfqpoint{2.810100in}{2.183674in}}%
\pgfpathlineto{\pgfqpoint{2.807511in}{2.177796in}}%
\pgfpathlineto{\pgfqpoint{2.804803in}{2.171536in}}%
\pgfpathlineto{\pgfqpoint{2.803840in}{2.169308in}}%
\pgfpathlineto{\pgfqpoint{2.802158in}{2.165275in}}%
\pgfpathlineto{\pgfqpoint{2.799576in}{2.159015in}}%
\pgfpathlineto{\pgfqpoint{2.797579in}{2.154087in}}%
\pgfpathlineto{\pgfqpoint{2.797056in}{2.152755in}}%
\pgfpathlineto{\pgfqpoint{2.794602in}{2.146494in}}%
\pgfpathlineto{\pgfqpoint{2.792213in}{2.140234in}}%
\pgfpathlineto{\pgfqpoint{2.791319in}{2.137862in}}%
\pgfpathlineto{\pgfqpoint{2.789889in}{2.133973in}}%
\pgfpathlineto{\pgfqpoint{2.787632in}{2.127713in}}%
\pgfpathlineto{\pgfqpoint{2.785444in}{2.121452in}}%
\pgfpathlineto{\pgfqpoint{2.785058in}{2.120335in}}%
\pgfpathlineto{\pgfqpoint{2.783320in}{2.115192in}}%
\pgfpathlineto{\pgfqpoint{2.781266in}{2.108932in}}%
\pgfpathlineto{\pgfqpoint{2.779283in}{2.102671in}}%
\pgfpathlineto{\pgfqpoint{2.778798in}{2.101107in}}%
\pgfpathlineto{\pgfqpoint{2.777365in}{2.096411in}}%
\pgfpathlineto{\pgfqpoint{2.775519in}{2.090150in}}%
\pgfpathlineto{\pgfqpoint{2.773746in}{2.083890in}}%
\pgfpathlineto{\pgfqpoint{2.772538in}{2.079460in}}%
\pgfpathlineto{\pgfqpoint{2.772044in}{2.077629in}}%
\pgfpathlineto{\pgfqpoint{2.770412in}{2.071369in}}%
\pgfpathlineto{\pgfqpoint{2.768855in}{2.065109in}}%
\pgfpathlineto{\pgfqpoint{2.767375in}{2.058848in}}%
\pgfpathlineto{\pgfqpoint{2.766277in}{2.053967in}}%
\pgfpathlineto{\pgfqpoint{2.765969in}{2.052588in}}%
\pgfpathlineto{\pgfqpoint{2.764637in}{2.046327in}}%
\pgfpathlineto{\pgfqpoint{2.763386in}{2.040067in}}%
\pgfpathlineto{\pgfqpoint{2.762215in}{2.033806in}}%
\pgfpathlineto{\pgfqpoint{2.761126in}{2.027546in}}%
\pgfpathlineto{\pgfqpoint{2.760120in}{2.021285in}}%
\pgfpathlineto{\pgfqpoint{2.760017in}{2.020586in}}%
\pgfpathlineto{\pgfqpoint{2.759197in}{2.015025in}}%
\pgfpathlineto{\pgfqpoint{2.758361in}{2.008765in}}%
\pgfpathlineto{\pgfqpoint{2.757615in}{2.002504in}}%
\pgfpathlineto{\pgfqpoint{2.756962in}{1.996244in}}%
\pgfpathlineto{\pgfqpoint{2.756404in}{1.989983in}}%
\pgfpathlineto{\pgfqpoint{2.755943in}{1.983723in}}%
\pgfpathlineto{\pgfqpoint{2.755584in}{1.977462in}}%
\pgfpathlineto{\pgfqpoint{2.755329in}{1.971202in}}%
\pgfpathlineto{\pgfqpoint{2.755183in}{1.964942in}}%
\pgfpathlineto{\pgfqpoint{2.755151in}{1.958681in}}%
\pgfpathlineto{\pgfqpoint{2.755236in}{1.952421in}}%
\pgfpathlineto{\pgfqpoint{2.755446in}{1.946160in}}%
\pgfpathlineto{\pgfqpoint{2.755785in}{1.939900in}}%
\pgfpathlineto{\pgfqpoint{2.756261in}{1.933639in}}%
\pgfpathlineto{\pgfqpoint{2.756880in}{1.927379in}}%
\pgfpathlineto{\pgfqpoint{2.757652in}{1.921119in}}%
\pgfpathlineto{\pgfqpoint{2.758585in}{1.914858in}}%
\pgfpathlineto{\pgfqpoint{2.759687in}{1.908598in}}%
\pgfpathlineto{\pgfqpoint{2.760017in}{1.906976in}}%
\pgfpathlineto{\pgfqpoint{2.760984in}{1.902337in}}%
\pgfpathlineto{\pgfqpoint{2.762482in}{1.896077in}}%
\pgfpathlineto{\pgfqpoint{2.764192in}{1.889816in}}%
\pgfpathlineto{\pgfqpoint{2.766128in}{1.883556in}}%
\pgfpathlineto{\pgfqpoint{2.766277in}{1.883119in}}%
\pgfpathlineto{\pgfqpoint{2.768350in}{1.877295in}}%
\pgfpathlineto{\pgfqpoint{2.770846in}{1.871035in}}%
\pgfpathlineto{\pgfqpoint{2.772538in}{1.867204in}}%
\pgfpathlineto{\pgfqpoint{2.773666in}{1.864775in}}%
\pgfpathlineto{\pgfqpoint{2.776864in}{1.858514in}}%
\pgfpathlineto{\pgfqpoint{2.778798in}{1.855070in}}%
\pgfpathlineto{\pgfqpoint{2.780480in}{1.852254in}}%
\pgfpathlineto{\pgfqpoint{2.784581in}{1.845993in}}%
\pgfpathlineto{\pgfqpoint{2.785058in}{1.845318in}}%
\pgfpathlineto{\pgfqpoint{2.789318in}{1.839733in}}%
\pgfpathlineto{\pgfqpoint{2.791319in}{1.837316in}}%
\pgfpathlineto{\pgfqpoint{2.794787in}{1.833472in}}%
\pgfpathlineto{\pgfqpoint{2.797579in}{1.830602in}}%
\pgfpathlineto{\pgfqpoint{2.801218in}{1.827212in}}%
\pgfpathlineto{\pgfqpoint{2.803840in}{1.824931in}}%
\pgfpathlineto{\pgfqpoint{2.808962in}{1.820952in}}%
\pgfpathlineto{\pgfqpoint{2.810100in}{1.820121in}}%
\pgfpathlineto{\pgfqpoint{2.816361in}{1.816066in}}%
\pgfpathlineto{\pgfqpoint{2.818805in}{1.814691in}}%
\pgfpathlineto{\pgfqpoint{2.822621in}{1.812655in}}%
\pgfpathlineto{\pgfqpoint{2.828881in}{1.809809in}}%
\pgfpathlineto{\pgfqpoint{2.832513in}{1.808431in}}%
\pgfpathlineto{\pgfqpoint{2.835142in}{1.807476in}}%
\pgfpathlineto{\pgfqpoint{2.841402in}{1.805618in}}%
\pgfpathlineto{\pgfqpoint{2.847663in}{1.804181in}}%
\pgfpathlineto{\pgfqpoint{2.853923in}{1.803139in}}%
\pgfpathlineto{\pgfqpoint{2.860184in}{1.802469in}}%
\pgfpathlineto{\pgfqpoint{2.866051in}{1.802170in}}%
\pgfpathclose%
\pgfpathmoveto{\pgfqpoint{2.869279in}{1.871035in}}%
\pgfpathlineto{\pgfqpoint{2.866444in}{1.871853in}}%
\pgfpathlineto{\pgfqpoint{2.860184in}{1.874277in}}%
\pgfpathlineto{\pgfqpoint{2.854139in}{1.877295in}}%
\pgfpathlineto{\pgfqpoint{2.853923in}{1.877412in}}%
\pgfpathlineto{\pgfqpoint{2.847663in}{1.881510in}}%
\pgfpathlineto{\pgfqpoint{2.845043in}{1.883556in}}%
\pgfpathlineto{\pgfqpoint{2.841402in}{1.886679in}}%
\pgfpathlineto{\pgfqpoint{2.838247in}{1.889816in}}%
\pgfpathlineto{\pgfqpoint{2.835142in}{1.893245in}}%
\pgfpathlineto{\pgfqpoint{2.832877in}{1.896077in}}%
\pgfpathlineto{\pgfqpoint{2.828881in}{1.901694in}}%
\pgfpathlineto{\pgfqpoint{2.828470in}{1.902337in}}%
\pgfpathlineto{\pgfqpoint{2.824889in}{1.908598in}}%
\pgfpathlineto{\pgfqpoint{2.822621in}{1.913158in}}%
\pgfpathlineto{\pgfqpoint{2.821845in}{1.914858in}}%
\pgfpathlineto{\pgfqpoint{2.819344in}{1.921119in}}%
\pgfpathlineto{\pgfqpoint{2.817219in}{1.927379in}}%
\pgfpathlineto{\pgfqpoint{2.816361in}{1.930364in}}%
\pgfpathlineto{\pgfqpoint{2.815480in}{1.933639in}}%
\pgfpathlineto{\pgfqpoint{2.814081in}{1.939900in}}%
\pgfpathlineto{\pgfqpoint{2.812959in}{1.946160in}}%
\pgfpathlineto{\pgfqpoint{2.812098in}{1.952421in}}%
\pgfpathlineto{\pgfqpoint{2.811478in}{1.958681in}}%
\pgfpathlineto{\pgfqpoint{2.811086in}{1.964942in}}%
\pgfpathlineto{\pgfqpoint{2.810909in}{1.971202in}}%
\pgfpathlineto{\pgfqpoint{2.810933in}{1.977462in}}%
\pgfpathlineto{\pgfqpoint{2.811150in}{1.983723in}}%
\pgfpathlineto{\pgfqpoint{2.811549in}{1.989983in}}%
\pgfpathlineto{\pgfqpoint{2.812123in}{1.996244in}}%
\pgfpathlineto{\pgfqpoint{2.812865in}{2.002504in}}%
\pgfpathlineto{\pgfqpoint{2.813769in}{2.008765in}}%
\pgfpathlineto{\pgfqpoint{2.814829in}{2.015025in}}%
\pgfpathlineto{\pgfqpoint{2.816043in}{2.021285in}}%
\pgfpathlineto{\pgfqpoint{2.816361in}{2.022748in}}%
\pgfpathlineto{\pgfqpoint{2.817431in}{2.027546in}}%
\pgfpathlineto{\pgfqpoint{2.818975in}{2.033806in}}%
\pgfpathlineto{\pgfqpoint{2.820664in}{2.040067in}}%
\pgfpathlineto{\pgfqpoint{2.822497in}{2.046327in}}%
\pgfpathlineto{\pgfqpoint{2.822621in}{2.046723in}}%
\pgfpathlineto{\pgfqpoint{2.824519in}{2.052588in}}%
\pgfpathlineto{\pgfqpoint{2.826686in}{2.058848in}}%
\pgfpathlineto{\pgfqpoint{2.828881in}{2.064799in}}%
\pgfpathlineto{\pgfqpoint{2.829000in}{2.065109in}}%
\pgfpathlineto{\pgfqpoint{2.831518in}{2.071369in}}%
\pgfpathlineto{\pgfqpoint{2.834182in}{2.077629in}}%
\pgfpathlineto{\pgfqpoint{2.835142in}{2.079780in}}%
\pgfpathlineto{\pgfqpoint{2.837047in}{2.083890in}}%
\pgfpathlineto{\pgfqpoint{2.840091in}{2.090150in}}%
\pgfpathlineto{\pgfqpoint{2.841402in}{2.092737in}}%
\pgfpathlineto{\pgfqpoint{2.843348in}{2.096411in}}%
\pgfpathlineto{\pgfqpoint{2.846806in}{2.102671in}}%
\pgfpathlineto{\pgfqpoint{2.847663in}{2.104168in}}%
\pgfpathlineto{\pgfqpoint{2.850525in}{2.108932in}}%
\pgfpathlineto{\pgfqpoint{2.853923in}{2.114369in}}%
\pgfpathlineto{\pgfqpoint{2.854466in}{2.115192in}}%
\pgfpathlineto{\pgfqpoint{2.858722in}{2.121452in}}%
\pgfpathlineto{\pgfqpoint{2.860184in}{2.123534in}}%
\pgfpathlineto{\pgfqpoint{2.863298in}{2.127713in}}%
\pgfpathlineto{\pgfqpoint{2.866444in}{2.131806in}}%
\pgfpathlineto{\pgfqpoint{2.868224in}{2.133973in}}%
\pgfpathlineto{\pgfqpoint{2.872705in}{2.139279in}}%
\pgfpathlineto{\pgfqpoint{2.873571in}{2.140234in}}%
\pgfpathlineto{\pgfqpoint{2.878965in}{2.146025in}}%
\pgfpathlineto{\pgfqpoint{2.879438in}{2.146494in}}%
\pgfpathlineto{\pgfqpoint{2.885225in}{2.152100in}}%
\pgfpathlineto{\pgfqpoint{2.885964in}{2.152755in}}%
\pgfpathlineto{\pgfqpoint{2.891486in}{2.157550in}}%
\pgfpathlineto{\pgfqpoint{2.893351in}{2.159015in}}%
\pgfpathlineto{\pgfqpoint{2.897746in}{2.162410in}}%
\pgfpathlineto{\pgfqpoint{2.901899in}{2.165275in}}%
\pgfpathlineto{\pgfqpoint{2.904007in}{2.166709in}}%
\pgfpathlineto{\pgfqpoint{2.910267in}{2.170472in}}%
\pgfpathlineto{\pgfqpoint{2.912300in}{2.171536in}}%
\pgfpathlineto{\pgfqpoint{2.916528in}{2.173724in}}%
\pgfpathlineto{\pgfqpoint{2.922788in}{2.176476in}}%
\pgfpathlineto{\pgfqpoint{2.926411in}{2.177796in}}%
\pgfpathlineto{\pgfqpoint{2.929048in}{2.178750in}}%
\pgfpathlineto{\pgfqpoint{2.935309in}{2.180556in}}%
\pgfpathlineto{\pgfqpoint{2.941569in}{2.181901in}}%
\pgfpathlineto{\pgfqpoint{2.947830in}{2.182796in}}%
\pgfpathlineto{\pgfqpoint{2.954090in}{2.183244in}}%
\pgfpathlineto{\pgfqpoint{2.960351in}{2.183247in}}%
\pgfpathlineto{\pgfqpoint{2.966611in}{2.182799in}}%
\pgfpathlineto{\pgfqpoint{2.972871in}{2.181894in}}%
\pgfpathlineto{\pgfqpoint{2.979132in}{2.180517in}}%
\pgfpathlineto{\pgfqpoint{2.985392in}{2.178653in}}%
\pgfpathlineto{\pgfqpoint{2.987664in}{2.177796in}}%
\pgfpathlineto{\pgfqpoint{2.991653in}{2.176260in}}%
\pgfpathlineto{\pgfqpoint{2.997913in}{2.173305in}}%
\pgfpathlineto{\pgfqpoint{3.001063in}{2.171536in}}%
\pgfpathlineto{\pgfqpoint{3.004174in}{2.169727in}}%
\pgfpathlineto{\pgfqpoint{3.010434in}{2.165459in}}%
\pgfpathlineto{\pgfqpoint{3.010670in}{2.165275in}}%
\pgfpathlineto{\pgfqpoint{3.016694in}{2.160343in}}%
\pgfpathlineto{\pgfqpoint{3.018123in}{2.159015in}}%
\pgfpathlineto{\pgfqpoint{3.022955in}{2.154239in}}%
\pgfpathlineto{\pgfqpoint{3.024293in}{2.152755in}}%
\pgfpathlineto{\pgfqpoint{3.029215in}{2.146879in}}%
\pgfpathlineto{\pgfqpoint{3.029506in}{2.146494in}}%
\pgfpathlineto{\pgfqpoint{3.033878in}{2.140234in}}%
\pgfpathlineto{\pgfqpoint{3.035476in}{2.137716in}}%
\pgfpathlineto{\pgfqpoint{3.037639in}{2.133973in}}%
\pgfpathlineto{\pgfqpoint{3.040883in}{2.127713in}}%
\pgfpathlineto{\pgfqpoint{3.041736in}{2.125868in}}%
\pgfpathlineto{\pgfqpoint{3.043616in}{2.121452in}}%
\pgfpathlineto{\pgfqpoint{3.045943in}{2.115192in}}%
\pgfpathlineto{\pgfqpoint{3.047926in}{2.108932in}}%
\pgfpathlineto{\pgfqpoint{3.047997in}{2.108668in}}%
\pgfpathlineto{\pgfqpoint{3.049503in}{2.102671in}}%
\pgfpathlineto{\pgfqpoint{3.050778in}{2.096411in}}%
\pgfpathlineto{\pgfqpoint{3.051767in}{2.090150in}}%
\pgfpathlineto{\pgfqpoint{3.052481in}{2.083890in}}%
\pgfpathlineto{\pgfqpoint{3.052931in}{2.077629in}}%
\pgfpathlineto{\pgfqpoint{3.053129in}{2.071369in}}%
\pgfpathlineto{\pgfqpoint{3.053084in}{2.065109in}}%
\pgfpathlineto{\pgfqpoint{3.052806in}{2.058848in}}%
\pgfpathlineto{\pgfqpoint{3.052304in}{2.052588in}}%
\pgfpathlineto{\pgfqpoint{3.051586in}{2.046327in}}%
\pgfpathlineto{\pgfqpoint{3.050658in}{2.040067in}}%
\pgfpathlineto{\pgfqpoint{3.049529in}{2.033806in}}%
\pgfpathlineto{\pgfqpoint{3.048204in}{2.027546in}}%
\pgfpathlineto{\pgfqpoint{3.047997in}{2.026700in}}%
\pgfpathlineto{\pgfqpoint{3.046643in}{2.021285in}}%
\pgfpathlineto{\pgfqpoint{3.044886in}{2.015025in}}%
\pgfpathlineto{\pgfqpoint{3.042945in}{2.008765in}}%
\pgfpathlineto{\pgfqpoint{3.041736in}{2.005212in}}%
\pgfpathlineto{\pgfqpoint{3.040792in}{2.002504in}}%
\pgfpathlineto{\pgfqpoint{3.038417in}{1.996244in}}%
\pgfpathlineto{\pgfqpoint{3.035867in}{1.989983in}}%
\pgfpathlineto{\pgfqpoint{3.035476in}{1.989090in}}%
\pgfpathlineto{\pgfqpoint{3.033062in}{1.983723in}}%
\pgfpathlineto{\pgfqpoint{3.030071in}{1.977462in}}%
\pgfpathlineto{\pgfqpoint{3.029215in}{1.975775in}}%
\pgfpathlineto{\pgfqpoint{3.026826in}{1.971202in}}%
\pgfpathlineto{\pgfqpoint{3.023381in}{1.964942in}}%
\pgfpathlineto{\pgfqpoint{3.022955in}{1.964207in}}%
\pgfpathlineto{\pgfqpoint{3.019645in}{1.958681in}}%
\pgfpathlineto{\pgfqpoint{3.016694in}{1.953972in}}%
\pgfpathlineto{\pgfqpoint{3.015687in}{1.952421in}}%
\pgfpathlineto{\pgfqpoint{3.011440in}{1.946160in}}%
\pgfpathlineto{\pgfqpoint{3.010434in}{1.944737in}}%
\pgfpathlineto{\pgfqpoint{3.006879in}{1.939900in}}%
\pgfpathlineto{\pgfqpoint{3.004174in}{1.936353in}}%
\pgfpathlineto{\pgfqpoint{3.002015in}{1.933639in}}%
\pgfpathlineto{\pgfqpoint{2.997913in}{1.928661in}}%
\pgfpathlineto{\pgfqpoint{2.996808in}{1.927379in}}%
\pgfpathlineto{\pgfqpoint{2.991653in}{1.921594in}}%
\pgfpathlineto{\pgfqpoint{2.991208in}{1.921119in}}%
\pgfpathlineto{\pgfqpoint{2.985392in}{1.915094in}}%
\pgfpathlineto{\pgfqpoint{2.985152in}{1.914858in}}%
\pgfpathlineto{\pgfqpoint{2.979132in}{1.909112in}}%
\pgfpathlineto{\pgfqpoint{2.978560in}{1.908598in}}%
\pgfpathlineto{\pgfqpoint{2.972871in}{1.903606in}}%
\pgfpathlineto{\pgfqpoint{2.971331in}{1.902337in}}%
\pgfpathlineto{\pgfqpoint{2.966611in}{1.898541in}}%
\pgfpathlineto{\pgfqpoint{2.963331in}{1.896077in}}%
\pgfpathlineto{\pgfqpoint{2.960351in}{1.893886in}}%
\pgfpathlineto{\pgfqpoint{2.954388in}{1.889816in}}%
\pgfpathlineto{\pgfqpoint{2.954090in}{1.889617in}}%
\pgfpathlineto{\pgfqpoint{2.947830in}{1.885775in}}%
\pgfpathlineto{\pgfqpoint{2.943869in}{1.883556in}}%
\pgfpathlineto{\pgfqpoint{2.941569in}{1.882289in}}%
\pgfpathlineto{\pgfqpoint{2.935309in}{1.879190in}}%
\pgfpathlineto{\pgfqpoint{2.930997in}{1.877295in}}%
\pgfpathlineto{\pgfqpoint{2.929048in}{1.876452in}}%
\pgfpathlineto{\pgfqpoint{2.922788in}{1.874110in}}%
\pgfpathlineto{\pgfqpoint{2.916528in}{1.872126in}}%
\pgfpathlineto{\pgfqpoint{2.912261in}{1.871035in}}%
\pgfpathlineto{\pgfqpoint{2.910267in}{1.870531in}}%
\pgfpathlineto{\pgfqpoint{2.904007in}{1.869354in}}%
\pgfpathlineto{\pgfqpoint{2.897746in}{1.868583in}}%
\pgfpathlineto{\pgfqpoint{2.891486in}{1.868243in}}%
\pgfpathlineto{\pgfqpoint{2.885225in}{1.868363in}}%
\pgfpathlineto{\pgfqpoint{2.878965in}{1.868974in}}%
\pgfpathlineto{\pgfqpoint{2.872705in}{1.870112in}}%
\pgfpathclose%
\pgfpathmoveto{\pgfqpoint{2.856188in}{2.497078in}}%
\pgfpathlineto{\pgfqpoint{2.853923in}{2.497678in}}%
\pgfpathlineto{\pgfqpoint{2.847663in}{2.500067in}}%
\pgfpathlineto{\pgfqpoint{2.841402in}{2.503294in}}%
\pgfpathlineto{\pgfqpoint{2.841332in}{2.503339in}}%
\pgfpathlineto{\pgfqpoint{2.835142in}{2.507496in}}%
\pgfpathlineto{\pgfqpoint{2.832582in}{2.509599in}}%
\pgfpathlineto{\pgfqpoint{2.828881in}{2.512823in}}%
\pgfpathlineto{\pgfqpoint{2.825936in}{2.515860in}}%
\pgfpathlineto{\pgfqpoint{2.822621in}{2.519524in}}%
\pgfpathlineto{\pgfqpoint{2.820586in}{2.522120in}}%
\pgfpathlineto{\pgfqpoint{2.816361in}{2.527970in}}%
\pgfpathlineto{\pgfqpoint{2.816098in}{2.528381in}}%
\pgfpathlineto{\pgfqpoint{2.812376in}{2.534641in}}%
\pgfpathlineto{\pgfqpoint{2.810100in}{2.538870in}}%
\pgfpathlineto{\pgfqpoint{2.809116in}{2.540902in}}%
\pgfpathlineto{\pgfqpoint{2.806335in}{2.547162in}}%
\pgfpathlineto{\pgfqpoint{2.803861in}{2.553422in}}%
\pgfpathlineto{\pgfqpoint{2.803840in}{2.553482in}}%
\pgfpathlineto{\pgfqpoint{2.801783in}{2.559683in}}%
\pgfpathlineto{\pgfqpoint{2.799950in}{2.565943in}}%
\pgfpathlineto{\pgfqpoint{2.798344in}{2.572204in}}%
\pgfpathlineto{\pgfqpoint{2.797579in}{2.575599in}}%
\pgfpathlineto{\pgfqpoint{2.796978in}{2.578464in}}%
\pgfpathlineto{\pgfqpoint{2.795831in}{2.584725in}}%
\pgfpathlineto{\pgfqpoint{2.794861in}{2.590985in}}%
\pgfpathlineto{\pgfqpoint{2.794056in}{2.597245in}}%
\pgfpathlineto{\pgfqpoint{2.793409in}{2.603506in}}%
\pgfpathlineto{\pgfqpoint{2.792909in}{2.609766in}}%
\pgfpathlineto{\pgfqpoint{2.792549in}{2.616027in}}%
\pgfpathlineto{\pgfqpoint{2.792321in}{2.622287in}}%
\pgfpathlineto{\pgfqpoint{2.792220in}{2.628548in}}%
\pgfpathlineto{\pgfqpoint{2.792239in}{2.634808in}}%
\pgfpathlineto{\pgfqpoint{2.792373in}{2.641068in}}%
\pgfpathlineto{\pgfqpoint{2.792616in}{2.647329in}}%
\pgfpathlineto{\pgfqpoint{2.792965in}{2.653589in}}%
\pgfpathlineto{\pgfqpoint{2.793416in}{2.659850in}}%
\pgfpathlineto{\pgfqpoint{2.793964in}{2.666110in}}%
\pgfpathlineto{\pgfqpoint{2.794607in}{2.672371in}}%
\pgfpathlineto{\pgfqpoint{2.795343in}{2.678631in}}%
\pgfpathlineto{\pgfqpoint{2.796168in}{2.684891in}}%
\pgfpathlineto{\pgfqpoint{2.797081in}{2.691152in}}%
\pgfpathlineto{\pgfqpoint{2.797579in}{2.694258in}}%
\pgfpathlineto{\pgfqpoint{2.798090in}{2.697412in}}%
\pgfpathlineto{\pgfqpoint{2.799195in}{2.703673in}}%
\pgfpathlineto{\pgfqpoint{2.800382in}{2.709933in}}%
\pgfpathlineto{\pgfqpoint{2.801650in}{2.716194in}}%
\pgfpathlineto{\pgfqpoint{2.802999in}{2.722454in}}%
\pgfpathlineto{\pgfqpoint{2.803840in}{2.726122in}}%
\pgfpathlineto{\pgfqpoint{2.804441in}{2.728715in}}%
\pgfpathlineto{\pgfqpoint{2.805978in}{2.734975in}}%
\pgfpathlineto{\pgfqpoint{2.807594in}{2.741235in}}%
\pgfpathlineto{\pgfqpoint{2.809287in}{2.747496in}}%
\pgfpathlineto{\pgfqpoint{2.810100in}{2.750358in}}%
\pgfpathlineto{\pgfqpoint{2.811079in}{2.753756in}}%
\pgfpathlineto{\pgfqpoint{2.812964in}{2.760017in}}%
\pgfpathlineto{\pgfqpoint{2.814927in}{2.766277in}}%
\pgfpathlineto{\pgfqpoint{2.816361in}{2.770672in}}%
\pgfpathlineto{\pgfqpoint{2.816980in}{2.772538in}}%
\pgfpathlineto{\pgfqpoint{2.819141in}{2.778798in}}%
\pgfpathlineto{\pgfqpoint{2.821380in}{2.785058in}}%
\pgfpathlineto{\pgfqpoint{2.822621in}{2.788408in}}%
\pgfpathlineto{\pgfqpoint{2.823720in}{2.791319in}}%
\pgfpathlineto{\pgfqpoint{2.826167in}{2.797579in}}%
\pgfpathlineto{\pgfqpoint{2.828692in}{2.803840in}}%
\pgfpathlineto{\pgfqpoint{2.828881in}{2.804295in}}%
\pgfpathlineto{\pgfqpoint{2.831353in}{2.810100in}}%
\pgfpathlineto{\pgfqpoint{2.834097in}{2.816361in}}%
\pgfpathlineto{\pgfqpoint{2.835142in}{2.818674in}}%
\pgfpathlineto{\pgfqpoint{2.836968in}{2.822621in}}%
\pgfpathlineto{\pgfqpoint{2.839944in}{2.828881in}}%
\pgfpathlineto{\pgfqpoint{2.841402in}{2.831867in}}%
\pgfpathlineto{\pgfqpoint{2.843045in}{2.835142in}}%
\pgfpathlineto{\pgfqpoint{2.846268in}{2.841402in}}%
\pgfpathlineto{\pgfqpoint{2.847663in}{2.844045in}}%
\pgfpathlineto{\pgfqpoint{2.849629in}{2.847663in}}%
\pgfpathlineto{\pgfqpoint{2.853113in}{2.853923in}}%
\pgfpathlineto{\pgfqpoint{2.853923in}{2.855345in}}%
\pgfpathlineto{\pgfqpoint{2.856769in}{2.860184in}}%
\pgfpathlineto{\pgfqpoint{2.860184in}{2.865864in}}%
\pgfpathlineto{\pgfqpoint{2.860545in}{2.866444in}}%
\pgfpathlineto{\pgfqpoint{2.864525in}{2.872705in}}%
\pgfpathlineto{\pgfqpoint{2.866444in}{2.875662in}}%
\pgfpathlineto{\pgfqpoint{2.868670in}{2.878965in}}%
\pgfpathlineto{\pgfqpoint{2.872705in}{2.884842in}}%
\pgfpathlineto{\pgfqpoint{2.872979in}{2.885225in}}%
\pgfpathlineto{\pgfqpoint{2.877540in}{2.891486in}}%
\pgfpathlineto{\pgfqpoint{2.878965in}{2.893408in}}%
\pgfpathlineto{\pgfqpoint{2.882326in}{2.897746in}}%
\pgfpathlineto{\pgfqpoint{2.885225in}{2.901429in}}%
\pgfpathlineto{\pgfqpoint{2.887355in}{2.904007in}}%
\pgfpathlineto{\pgfqpoint{2.891486in}{2.908935in}}%
\pgfpathlineto{\pgfqpoint{2.892662in}{2.910267in}}%
\pgfpathlineto{\pgfqpoint{2.897746in}{2.915947in}}%
\pgfpathlineto{\pgfqpoint{2.898296in}{2.916528in}}%
\pgfpathlineto{\pgfqpoint{2.904007in}{2.922483in}}%
\pgfpathlineto{\pgfqpoint{2.904318in}{2.922788in}}%
\pgfpathlineto{\pgfqpoint{2.910267in}{2.928559in}}%
\pgfpathlineto{\pgfqpoint{2.910806in}{2.929048in}}%
\pgfpathlineto{\pgfqpoint{2.916528in}{2.934190in}}%
\pgfpathlineto{\pgfqpoint{2.917866in}{2.935309in}}%
\pgfpathlineto{\pgfqpoint{2.922788in}{2.939388in}}%
\pgfpathlineto{\pgfqpoint{2.925636in}{2.941569in}}%
\pgfpathlineto{\pgfqpoint{2.929048in}{2.944163in}}%
\pgfpathlineto{\pgfqpoint{2.934307in}{2.947830in}}%
\pgfpathlineto{\pgfqpoint{2.935309in}{2.948524in}}%
\pgfpathlineto{\pgfqpoint{2.941569in}{2.952453in}}%
\pgfpathlineto{\pgfqpoint{2.944481in}{2.954090in}}%
\pgfpathlineto{\pgfqpoint{2.947830in}{2.955966in}}%
\pgfpathlineto{\pgfqpoint{2.954090in}{2.959057in}}%
\pgfpathlineto{\pgfqpoint{2.957133in}{2.960351in}}%
\pgfpathlineto{\pgfqpoint{2.960351in}{2.961716in}}%
\pgfpathlineto{\pgfqpoint{2.966611in}{2.963931in}}%
\pgfpathlineto{\pgfqpoint{2.972871in}{2.965705in}}%
\pgfpathlineto{\pgfqpoint{2.977223in}{2.966611in}}%
\pgfpathlineto{\pgfqpoint{2.979132in}{2.967009in}}%
\pgfpathlineto{\pgfqpoint{2.985392in}{2.967816in}}%
\pgfpathlineto{\pgfqpoint{2.991653in}{2.968112in}}%
\pgfpathlineto{\pgfqpoint{2.997913in}{2.967866in}}%
\pgfpathlineto{\pgfqpoint{3.004174in}{2.967040in}}%
\pgfpathlineto{\pgfqpoint{3.006059in}{2.966611in}}%
\pgfpathlineto{\pgfqpoint{3.010434in}{2.965566in}}%
\pgfpathlineto{\pgfqpoint{3.016694in}{2.963390in}}%
\pgfpathlineto{\pgfqpoint{3.022955in}{2.960457in}}%
\pgfpathlineto{\pgfqpoint{3.023139in}{2.960351in}}%
\pgfpathlineto{\pgfqpoint{3.029215in}{2.956594in}}%
\pgfpathlineto{\pgfqpoint{3.032529in}{2.954090in}}%
\pgfpathlineto{\pgfqpoint{3.035476in}{2.951705in}}%
\pgfpathlineto{\pgfqpoint{3.039533in}{2.947830in}}%
\pgfpathlineto{\pgfqpoint{3.041736in}{2.945558in}}%
\pgfpathlineto{\pgfqpoint{3.045099in}{2.941569in}}%
\pgfpathlineto{\pgfqpoint{3.047997in}{2.937825in}}%
\pgfpathlineto{\pgfqpoint{3.049724in}{2.935309in}}%
\pgfpathlineto{\pgfqpoint{3.053649in}{2.929048in}}%
\pgfpathlineto{\pgfqpoint{3.054257in}{2.927990in}}%
\pgfpathlineto{\pgfqpoint{3.056958in}{2.922788in}}%
\pgfpathlineto{\pgfqpoint{3.059876in}{2.916528in}}%
\pgfpathlineto{\pgfqpoint{3.060518in}{2.915009in}}%
\pgfpathlineto{\pgfqpoint{3.062359in}{2.910267in}}%
\pgfpathlineto{\pgfqpoint{3.064529in}{2.904007in}}%
\pgfpathlineto{\pgfqpoint{3.066441in}{2.897746in}}%
\pgfpathlineto{\pgfqpoint{3.066778in}{2.896507in}}%
\pgfpathlineto{\pgfqpoint{3.068051in}{2.891486in}}%
\pgfpathlineto{\pgfqpoint{3.069432in}{2.885225in}}%
\pgfpathlineto{\pgfqpoint{3.070614in}{2.878965in}}%
\pgfpathlineto{\pgfqpoint{3.071607in}{2.872705in}}%
\pgfpathlineto{\pgfqpoint{3.072422in}{2.866444in}}%
\pgfpathlineto{\pgfqpoint{3.073038in}{2.860475in}}%
\pgfpathlineto{\pgfqpoint{3.073067in}{2.860184in}}%
\pgfpathlineto{\pgfqpoint{3.073537in}{2.853923in}}%
\pgfpathlineto{\pgfqpoint{3.073860in}{2.847663in}}%
\pgfpathlineto{\pgfqpoint{3.074044in}{2.841402in}}%
\pgfpathlineto{\pgfqpoint{3.074094in}{2.835142in}}%
\pgfpathlineto{\pgfqpoint{3.074016in}{2.828881in}}%
\pgfpathlineto{\pgfqpoint{3.073816in}{2.822621in}}%
\pgfpathlineto{\pgfqpoint{3.073497in}{2.816361in}}%
\pgfpathlineto{\pgfqpoint{3.073064in}{2.810100in}}%
\pgfpathlineto{\pgfqpoint{3.073038in}{2.809810in}}%
\pgfpathlineto{\pgfqpoint{3.072508in}{2.803840in}}%
\pgfpathlineto{\pgfqpoint{3.071843in}{2.797579in}}%
\pgfpathlineto{\pgfqpoint{3.071073in}{2.791319in}}%
\pgfpathlineto{\pgfqpoint{3.070203in}{2.785058in}}%
\pgfpathlineto{\pgfqpoint{3.069234in}{2.778798in}}%
\pgfpathlineto{\pgfqpoint{3.068169in}{2.772538in}}%
\pgfpathlineto{\pgfqpoint{3.067012in}{2.766277in}}%
\pgfpathlineto{\pgfqpoint{3.066778in}{2.765110in}}%
\pgfpathlineto{\pgfqpoint{3.065740in}{2.760017in}}%
\pgfpathlineto{\pgfqpoint{3.064374in}{2.753756in}}%
\pgfpathlineto{\pgfqpoint{3.062921in}{2.747496in}}%
\pgfpathlineto{\pgfqpoint{3.061383in}{2.741235in}}%
\pgfpathlineto{\pgfqpoint{3.060518in}{2.737902in}}%
\pgfpathlineto{\pgfqpoint{3.059744in}{2.734975in}}%
\pgfpathlineto{\pgfqpoint{3.058003in}{2.728715in}}%
\pgfpathlineto{\pgfqpoint{3.056182in}{2.722454in}}%
\pgfpathlineto{\pgfqpoint{3.054282in}{2.716194in}}%
\pgfpathlineto{\pgfqpoint{3.054257in}{2.716114in}}%
\pgfpathlineto{\pgfqpoint{3.052260in}{2.709933in}}%
\pgfpathlineto{\pgfqpoint{3.050161in}{2.703673in}}%
\pgfpathlineto{\pgfqpoint{3.047997in}{2.697436in}}%
\pgfpathlineto{\pgfqpoint{3.047988in}{2.697412in}}%
\pgfpathlineto{\pgfqpoint{3.045687in}{2.691152in}}%
\pgfpathlineto{\pgfqpoint{3.043316in}{2.684891in}}%
\pgfpathlineto{\pgfqpoint{3.041736in}{2.680845in}}%
\pgfpathlineto{\pgfqpoint{3.040851in}{2.678631in}}%
\pgfpathlineto{\pgfqpoint{3.038276in}{2.672371in}}%
\pgfpathlineto{\pgfqpoint{3.035634in}{2.666110in}}%
\pgfpathlineto{\pgfqpoint{3.035476in}{2.665744in}}%
\pgfpathlineto{\pgfqpoint{3.032856in}{2.659850in}}%
\pgfpathlineto{\pgfqpoint{3.030010in}{2.653589in}}%
\pgfpathlineto{\pgfqpoint{3.029215in}{2.651878in}}%
\pgfpathlineto{\pgfqpoint{3.027040in}{2.647329in}}%
\pgfpathlineto{\pgfqpoint{3.023985in}{2.641068in}}%
\pgfpathlineto{\pgfqpoint{3.022955in}{2.638996in}}%
\pgfpathlineto{\pgfqpoint{3.020805in}{2.634808in}}%
\pgfpathlineto{\pgfqpoint{3.017537in}{2.628548in}}%
\pgfpathlineto{\pgfqpoint{3.016694in}{2.626959in}}%
\pgfpathlineto{\pgfqpoint{3.014129in}{2.622287in}}%
\pgfpathlineto{\pgfqpoint{3.010641in}{2.616027in}}%
\pgfpathlineto{\pgfqpoint{3.010434in}{2.615660in}}%
\pgfpathlineto{\pgfqpoint{3.006982in}{2.609766in}}%
\pgfpathlineto{\pgfqpoint{3.004174in}{2.605031in}}%
\pgfpathlineto{\pgfqpoint{3.003232in}{2.603506in}}%
\pgfpathlineto{\pgfqpoint{2.999327in}{2.597245in}}%
\pgfpathlineto{\pgfqpoint{2.997913in}{2.594999in}}%
\pgfpathlineto{\pgfqpoint{2.995274in}{2.590985in}}%
\pgfpathlineto{\pgfqpoint{2.991653in}{2.585520in}}%
\pgfpathlineto{\pgfqpoint{2.991100in}{2.584725in}}%
\pgfpathlineto{\pgfqpoint{2.986727in}{2.578464in}}%
\pgfpathlineto{\pgfqpoint{2.985392in}{2.576564in}}%
\pgfpathlineto{\pgfqpoint{2.982172in}{2.572204in}}%
\pgfpathlineto{\pgfqpoint{2.979132in}{2.568101in}}%
\pgfpathlineto{\pgfqpoint{2.977443in}{2.565943in}}%
\pgfpathlineto{\pgfqpoint{2.972871in}{2.560113in}}%
\pgfpathlineto{\pgfqpoint{2.972514in}{2.559683in}}%
\pgfpathlineto{\pgfqpoint{2.967308in}{2.553422in}}%
\pgfpathlineto{\pgfqpoint{2.966611in}{2.552584in}}%
\pgfpathlineto{\pgfqpoint{2.961813in}{2.547162in}}%
\pgfpathlineto{\pgfqpoint{2.960351in}{2.545505in}}%
\pgfpathlineto{\pgfqpoint{2.956001in}{2.540902in}}%
\pgfpathlineto{\pgfqpoint{2.954090in}{2.538870in}}%
\pgfpathlineto{\pgfqpoint{2.949806in}{2.534641in}}%
\pgfpathlineto{\pgfqpoint{2.947830in}{2.532677in}}%
\pgfpathlineto{\pgfqpoint{2.943142in}{2.528381in}}%
\pgfpathlineto{\pgfqpoint{2.941569in}{2.526926in}}%
\pgfpathlineto{\pgfqpoint{2.935892in}{2.522120in}}%
\pgfpathlineto{\pgfqpoint{2.935309in}{2.521622in}}%
\pgfpathlineto{\pgfqpoint{2.929048in}{2.516762in}}%
\pgfpathlineto{\pgfqpoint{2.927753in}{2.515860in}}%
\pgfpathlineto{\pgfqpoint{2.922788in}{2.512355in}}%
\pgfpathlineto{\pgfqpoint{2.918389in}{2.509599in}}%
\pgfpathlineto{\pgfqpoint{2.916528in}{2.508416in}}%
\pgfpathlineto{\pgfqpoint{2.910267in}{2.504946in}}%
\pgfpathlineto{\pgfqpoint{2.906868in}{2.503339in}}%
\pgfpathlineto{\pgfqpoint{2.904007in}{2.501961in}}%
\pgfpathlineto{\pgfqpoint{2.897746in}{2.499472in}}%
\pgfpathlineto{\pgfqpoint{2.891486in}{2.497500in}}%
\pgfpathlineto{\pgfqpoint{2.889638in}{2.497078in}}%
\pgfpathlineto{\pgfqpoint{2.885225in}{2.496051in}}%
\pgfpathlineto{\pgfqpoint{2.878965in}{2.495153in}}%
\pgfpathlineto{\pgfqpoint{2.872705in}{2.494833in}}%
\pgfpathlineto{\pgfqpoint{2.866444in}{2.495119in}}%
\pgfpathlineto{\pgfqpoint{2.860184in}{2.496051in}}%
\pgfpathclose%
\pgfusepath{fill}%
\end{pgfscope}%
\begin{pgfscope}%
\pgfpathrectangle{\pgfqpoint{0.500000in}{0.500000in}}{\pgfqpoint{3.750000in}{3.750000in}}%
\pgfusepath{clip}%
\pgfsetbuttcap%
\pgfsetroundjoin%
\definecolor{currentfill}{rgb}{0.183883,0.546498,0.743837}%
\pgfsetfillcolor{currentfill}%
\pgfsetlinewidth{0.000000pt}%
\definecolor{currentstroke}{rgb}{0.000000,0.000000,0.000000}%
\pgfsetstrokecolor{currentstroke}%
\pgfsetdash{}{0pt}%
\pgfpathmoveto{\pgfqpoint{2.065109in}{2.659300in}}%
\pgfpathlineto{\pgfqpoint{2.071369in}{2.658322in}}%
\pgfpathlineto{\pgfqpoint{2.077629in}{2.657731in}}%
\pgfpathlineto{\pgfqpoint{2.083890in}{2.657619in}}%
\pgfpathlineto{\pgfqpoint{2.090150in}{2.658118in}}%
\pgfpathlineto{\pgfqpoint{2.096411in}{2.659408in}}%
\pgfpathlineto{\pgfqpoint{2.097694in}{2.659850in}}%
\pgfpathlineto{\pgfqpoint{2.102671in}{2.662171in}}%
\pgfpathlineto{\pgfqpoint{2.107908in}{2.666110in}}%
\pgfpathlineto{\pgfqpoint{2.108932in}{2.667312in}}%
\pgfpathlineto{\pgfqpoint{2.112059in}{2.672371in}}%
\pgfpathlineto{\pgfqpoint{2.113916in}{2.678631in}}%
\pgfpathlineto{\pgfqpoint{2.114302in}{2.684891in}}%
\pgfpathlineto{\pgfqpoint{2.113629in}{2.691152in}}%
\pgfpathlineto{\pgfqpoint{2.112162in}{2.697412in}}%
\pgfpathlineto{\pgfqpoint{2.110082in}{2.703673in}}%
\pgfpathlineto{\pgfqpoint{2.108932in}{2.706368in}}%
\pgfpathlineto{\pgfqpoint{2.107400in}{2.709933in}}%
\pgfpathlineto{\pgfqpoint{2.104225in}{2.716194in}}%
\pgfpathlineto{\pgfqpoint{2.102671in}{2.718875in}}%
\pgfpathlineto{\pgfqpoint{2.100584in}{2.722454in}}%
\pgfpathlineto{\pgfqpoint{2.096571in}{2.728715in}}%
\pgfpathlineto{\pgfqpoint{2.096411in}{2.728939in}}%
\pgfpathlineto{\pgfqpoint{2.092065in}{2.734975in}}%
\pgfpathlineto{\pgfqpoint{2.090150in}{2.737452in}}%
\pgfpathlineto{\pgfqpoint{2.087204in}{2.741235in}}%
\pgfpathlineto{\pgfqpoint{2.083890in}{2.745218in}}%
\pgfpathlineto{\pgfqpoint{2.081978in}{2.747496in}}%
\pgfpathlineto{\pgfqpoint{2.077629in}{2.752365in}}%
\pgfpathlineto{\pgfqpoint{2.076376in}{2.753756in}}%
\pgfpathlineto{\pgfqpoint{2.071369in}{2.758997in}}%
\pgfpathlineto{\pgfqpoint{2.070386in}{2.760017in}}%
\pgfpathlineto{\pgfqpoint{2.065109in}{2.765196in}}%
\pgfpathlineto{\pgfqpoint{2.063995in}{2.766277in}}%
\pgfpathlineto{\pgfqpoint{2.058848in}{2.771025in}}%
\pgfpathlineto{\pgfqpoint{2.057190in}{2.772538in}}%
\pgfpathlineto{\pgfqpoint{2.052588in}{2.776537in}}%
\pgfpathlineto{\pgfqpoint{2.049957in}{2.778798in}}%
\pgfpathlineto{\pgfqpoint{2.046327in}{2.781775in}}%
\pgfpathlineto{\pgfqpoint{2.042278in}{2.785058in}}%
\pgfpathlineto{\pgfqpoint{2.040067in}{2.786774in}}%
\pgfpathlineto{\pgfqpoint{2.034137in}{2.791319in}}%
\pgfpathlineto{\pgfqpoint{2.033806in}{2.791562in}}%
\pgfpathlineto{\pgfqpoint{2.027546in}{2.796029in}}%
\pgfpathlineto{\pgfqpoint{2.025337in}{2.797579in}}%
\pgfpathlineto{\pgfqpoint{2.021285in}{2.800306in}}%
\pgfpathlineto{\pgfqpoint{2.015957in}{2.803840in}}%
\pgfpathlineto{\pgfqpoint{2.015025in}{2.804434in}}%
\pgfpathlineto{\pgfqpoint{2.008765in}{2.808277in}}%
\pgfpathlineto{\pgfqpoint{2.005721in}{2.810100in}}%
\pgfpathlineto{\pgfqpoint{2.002504in}{2.811953in}}%
\pgfpathlineto{\pgfqpoint{1.996244in}{2.815451in}}%
\pgfpathlineto{\pgfqpoint{1.994538in}{2.816361in}}%
\pgfpathlineto{\pgfqpoint{1.989983in}{2.818695in}}%
\pgfpathlineto{\pgfqpoint{1.983723in}{2.821799in}}%
\pgfpathlineto{\pgfqpoint{1.981957in}{2.822621in}}%
\pgfpathlineto{\pgfqpoint{1.977462in}{2.824632in}}%
\pgfpathlineto{\pgfqpoint{1.971202in}{2.827288in}}%
\pgfpathlineto{\pgfqpoint{1.967181in}{2.828881in}}%
\pgfpathlineto{\pgfqpoint{1.964942in}{2.829734in}}%
\pgfpathlineto{\pgfqpoint{1.958681in}{2.831882in}}%
\pgfpathlineto{\pgfqpoint{1.952421in}{2.833843in}}%
\pgfpathlineto{\pgfqpoint{1.947672in}{2.835142in}}%
\pgfpathlineto{\pgfqpoint{1.946160in}{2.835538in}}%
\pgfpathlineto{\pgfqpoint{1.939900in}{2.836849in}}%
\pgfpathlineto{\pgfqpoint{1.933639in}{2.837847in}}%
\pgfpathlineto{\pgfqpoint{1.927379in}{2.838465in}}%
\pgfpathlineto{\pgfqpoint{1.921119in}{2.838612in}}%
\pgfpathlineto{\pgfqpoint{1.914858in}{2.838168in}}%
\pgfpathlineto{\pgfqpoint{1.908598in}{2.836964in}}%
\pgfpathlineto{\pgfqpoint{1.903337in}{2.835142in}}%
\pgfpathlineto{\pgfqpoint{1.902337in}{2.834674in}}%
\pgfpathlineto{\pgfqpoint{1.896077in}{2.830186in}}%
\pgfpathlineto{\pgfqpoint{1.894799in}{2.828881in}}%
\pgfpathlineto{\pgfqpoint{1.891004in}{2.822621in}}%
\pgfpathlineto{\pgfqpoint{1.889816in}{2.818393in}}%
\pgfpathlineto{\pgfqpoint{1.889365in}{2.816361in}}%
\pgfpathlineto{\pgfqpoint{1.889149in}{2.810100in}}%
\pgfpathlineto{\pgfqpoint{1.889816in}{2.804553in}}%
\pgfpathlineto{\pgfqpoint{1.889903in}{2.803840in}}%
\pgfpathlineto{\pgfqpoint{1.891523in}{2.797579in}}%
\pgfpathlineto{\pgfqpoint{1.893756in}{2.791319in}}%
\pgfpathlineto{\pgfqpoint{1.896077in}{2.786034in}}%
\pgfpathlineto{\pgfqpoint{1.896508in}{2.785058in}}%
\pgfpathlineto{\pgfqpoint{1.899854in}{2.778798in}}%
\pgfpathlineto{\pgfqpoint{1.902337in}{2.774640in}}%
\pgfpathlineto{\pgfqpoint{1.903602in}{2.772538in}}%
\pgfpathlineto{\pgfqpoint{1.907791in}{2.766277in}}%
\pgfpathlineto{\pgfqpoint{1.908598in}{2.765182in}}%
\pgfpathlineto{\pgfqpoint{1.912431in}{2.760017in}}%
\pgfpathlineto{\pgfqpoint{1.914858in}{2.756970in}}%
\pgfpathlineto{\pgfqpoint{1.917440in}{2.753756in}}%
\pgfpathlineto{\pgfqpoint{1.921119in}{2.749470in}}%
\pgfpathlineto{\pgfqpoint{1.922828in}{2.747496in}}%
\pgfpathlineto{\pgfqpoint{1.927379in}{2.742552in}}%
\pgfpathlineto{\pgfqpoint{1.928603in}{2.741235in}}%
\pgfpathlineto{\pgfqpoint{1.933639in}{2.736117in}}%
\pgfpathlineto{\pgfqpoint{1.934775in}{2.734975in}}%
\pgfpathlineto{\pgfqpoint{1.939900in}{2.730087in}}%
\pgfpathlineto{\pgfqpoint{1.941354in}{2.728715in}}%
\pgfpathlineto{\pgfqpoint{1.946160in}{2.724400in}}%
\pgfpathlineto{\pgfqpoint{1.948352in}{2.722454in}}%
\pgfpathlineto{\pgfqpoint{1.952421in}{2.719007in}}%
\pgfpathlineto{\pgfqpoint{1.955780in}{2.716194in}}%
\pgfpathlineto{\pgfqpoint{1.958681in}{2.713869in}}%
\pgfpathlineto{\pgfqpoint{1.963651in}{2.709933in}}%
\pgfpathlineto{\pgfqpoint{1.964942in}{2.708953in}}%
\pgfpathlineto{\pgfqpoint{1.971202in}{2.704286in}}%
\pgfpathlineto{\pgfqpoint{1.972046in}{2.703673in}}%
\pgfpathlineto{\pgfqpoint{1.977462in}{2.699893in}}%
\pgfpathlineto{\pgfqpoint{1.981067in}{2.697412in}}%
\pgfpathlineto{\pgfqpoint{1.983723in}{2.695654in}}%
\pgfpathlineto{\pgfqpoint{1.989983in}{2.691589in}}%
\pgfpathlineto{\pgfqpoint{1.990683in}{2.691152in}}%
\pgfpathlineto{\pgfqpoint{1.996244in}{2.687810in}}%
\pgfpathlineto{\pgfqpoint{2.001176in}{2.684891in}}%
\pgfpathlineto{\pgfqpoint{2.002504in}{2.684133in}}%
\pgfpathlineto{\pgfqpoint{2.008765in}{2.680722in}}%
\pgfpathlineto{\pgfqpoint{2.012716in}{2.678631in}}%
\pgfpathlineto{\pgfqpoint{2.015025in}{2.677452in}}%
\pgfpathlineto{\pgfqpoint{2.021285in}{2.674419in}}%
\pgfpathlineto{\pgfqpoint{2.025689in}{2.672371in}}%
\pgfpathlineto{\pgfqpoint{2.027546in}{2.671537in}}%
\pgfpathlineto{\pgfqpoint{2.033806in}{2.668933in}}%
\pgfpathlineto{\pgfqpoint{2.040067in}{2.666453in}}%
\pgfpathlineto{\pgfqpoint{2.041037in}{2.666110in}}%
\pgfpathlineto{\pgfqpoint{2.046327in}{2.664308in}}%
\pgfpathlineto{\pgfqpoint{2.052588in}{2.662380in}}%
\pgfpathlineto{\pgfqpoint{2.058848in}{2.660674in}}%
\pgfpathlineto{\pgfqpoint{2.062545in}{2.659850in}}%
\pgfpathclose%
\pgfusepath{fill}%
\end{pgfscope}%
\begin{pgfscope}%
\pgfpathrectangle{\pgfqpoint{0.500000in}{0.500000in}}{\pgfqpoint{3.750000in}{3.750000in}}%
\pgfusepath{clip}%
\pgfsetbuttcap%
\pgfsetroundjoin%
\definecolor{currentfill}{rgb}{0.018408,0.412826,0.648212}%
\pgfsetfillcolor{currentfill}%
\pgfsetlinewidth{0.000000pt}%
\definecolor{currentstroke}{rgb}{0.000000,0.000000,0.000000}%
\pgfsetstrokecolor{currentstroke}%
\pgfsetdash{}{0pt}%
\pgfpathmoveto{\pgfqpoint{2.872705in}{1.870112in}}%
\pgfpathlineto{\pgfqpoint{2.878965in}{1.868974in}}%
\pgfpathlineto{\pgfqpoint{2.885225in}{1.868363in}}%
\pgfpathlineto{\pgfqpoint{2.891486in}{1.868243in}}%
\pgfpathlineto{\pgfqpoint{2.897746in}{1.868583in}}%
\pgfpathlineto{\pgfqpoint{2.904007in}{1.869354in}}%
\pgfpathlineto{\pgfqpoint{2.910267in}{1.870531in}}%
\pgfpathlineto{\pgfqpoint{2.912261in}{1.871035in}}%
\pgfpathlineto{\pgfqpoint{2.916528in}{1.872126in}}%
\pgfpathlineto{\pgfqpoint{2.922788in}{1.874110in}}%
\pgfpathlineto{\pgfqpoint{2.929048in}{1.876452in}}%
\pgfpathlineto{\pgfqpoint{2.930997in}{1.877295in}}%
\pgfpathlineto{\pgfqpoint{2.935309in}{1.879190in}}%
\pgfpathlineto{\pgfqpoint{2.941569in}{1.882289in}}%
\pgfpathlineto{\pgfqpoint{2.943869in}{1.883556in}}%
\pgfpathlineto{\pgfqpoint{2.947830in}{1.885775in}}%
\pgfpathlineto{\pgfqpoint{2.954090in}{1.889617in}}%
\pgfpathlineto{\pgfqpoint{2.954388in}{1.889816in}}%
\pgfpathlineto{\pgfqpoint{2.960351in}{1.893886in}}%
\pgfpathlineto{\pgfqpoint{2.963331in}{1.896077in}}%
\pgfpathlineto{\pgfqpoint{2.966611in}{1.898541in}}%
\pgfpathlineto{\pgfqpoint{2.971331in}{1.902337in}}%
\pgfpathlineto{\pgfqpoint{2.972871in}{1.903606in}}%
\pgfpathlineto{\pgfqpoint{2.978560in}{1.908598in}}%
\pgfpathlineto{\pgfqpoint{2.979132in}{1.909112in}}%
\pgfpathlineto{\pgfqpoint{2.985152in}{1.914858in}}%
\pgfpathlineto{\pgfqpoint{2.985392in}{1.915094in}}%
\pgfpathlineto{\pgfqpoint{2.991208in}{1.921119in}}%
\pgfpathlineto{\pgfqpoint{2.991653in}{1.921594in}}%
\pgfpathlineto{\pgfqpoint{2.996808in}{1.927379in}}%
\pgfpathlineto{\pgfqpoint{2.997913in}{1.928661in}}%
\pgfpathlineto{\pgfqpoint{3.002015in}{1.933639in}}%
\pgfpathlineto{\pgfqpoint{3.004174in}{1.936353in}}%
\pgfpathlineto{\pgfqpoint{3.006879in}{1.939900in}}%
\pgfpathlineto{\pgfqpoint{3.010434in}{1.944737in}}%
\pgfpathlineto{\pgfqpoint{3.011440in}{1.946160in}}%
\pgfpathlineto{\pgfqpoint{3.015687in}{1.952421in}}%
\pgfpathlineto{\pgfqpoint{3.016694in}{1.953972in}}%
\pgfpathlineto{\pgfqpoint{3.019645in}{1.958681in}}%
\pgfpathlineto{\pgfqpoint{3.022955in}{1.964207in}}%
\pgfpathlineto{\pgfqpoint{3.023381in}{1.964942in}}%
\pgfpathlineto{\pgfqpoint{3.026826in}{1.971202in}}%
\pgfpathlineto{\pgfqpoint{3.029215in}{1.975775in}}%
\pgfpathlineto{\pgfqpoint{3.030071in}{1.977462in}}%
\pgfpathlineto{\pgfqpoint{3.033062in}{1.983723in}}%
\pgfpathlineto{\pgfqpoint{3.035476in}{1.989090in}}%
\pgfpathlineto{\pgfqpoint{3.035867in}{1.989983in}}%
\pgfpathlineto{\pgfqpoint{3.038417in}{1.996244in}}%
\pgfpathlineto{\pgfqpoint{3.040792in}{2.002504in}}%
\pgfpathlineto{\pgfqpoint{3.041736in}{2.005212in}}%
\pgfpathlineto{\pgfqpoint{3.042945in}{2.008765in}}%
\pgfpathlineto{\pgfqpoint{3.044886in}{2.015025in}}%
\pgfpathlineto{\pgfqpoint{3.046643in}{2.021285in}}%
\pgfpathlineto{\pgfqpoint{3.047997in}{2.026700in}}%
\pgfpathlineto{\pgfqpoint{3.048204in}{2.027546in}}%
\pgfpathlineto{\pgfqpoint{3.049529in}{2.033806in}}%
\pgfpathlineto{\pgfqpoint{3.050658in}{2.040067in}}%
\pgfpathlineto{\pgfqpoint{3.051586in}{2.046327in}}%
\pgfpathlineto{\pgfqpoint{3.052304in}{2.052588in}}%
\pgfpathlineto{\pgfqpoint{3.052806in}{2.058848in}}%
\pgfpathlineto{\pgfqpoint{3.053084in}{2.065109in}}%
\pgfpathlineto{\pgfqpoint{3.053129in}{2.071369in}}%
\pgfpathlineto{\pgfqpoint{3.052931in}{2.077629in}}%
\pgfpathlineto{\pgfqpoint{3.052481in}{2.083890in}}%
\pgfpathlineto{\pgfqpoint{3.051767in}{2.090150in}}%
\pgfpathlineto{\pgfqpoint{3.050778in}{2.096411in}}%
\pgfpathlineto{\pgfqpoint{3.049503in}{2.102671in}}%
\pgfpathlineto{\pgfqpoint{3.047997in}{2.108668in}}%
\pgfpathlineto{\pgfqpoint{3.047926in}{2.108932in}}%
\pgfpathlineto{\pgfqpoint{3.045943in}{2.115192in}}%
\pgfpathlineto{\pgfqpoint{3.043616in}{2.121452in}}%
\pgfpathlineto{\pgfqpoint{3.041736in}{2.125868in}}%
\pgfpathlineto{\pgfqpoint{3.040883in}{2.127713in}}%
\pgfpathlineto{\pgfqpoint{3.037639in}{2.133973in}}%
\pgfpathlineto{\pgfqpoint{3.035476in}{2.137716in}}%
\pgfpathlineto{\pgfqpoint{3.033878in}{2.140234in}}%
\pgfpathlineto{\pgfqpoint{3.029506in}{2.146494in}}%
\pgfpathlineto{\pgfqpoint{3.029215in}{2.146879in}}%
\pgfpathlineto{\pgfqpoint{3.024293in}{2.152755in}}%
\pgfpathlineto{\pgfqpoint{3.022955in}{2.154239in}}%
\pgfpathlineto{\pgfqpoint{3.018123in}{2.159015in}}%
\pgfpathlineto{\pgfqpoint{3.016694in}{2.160343in}}%
\pgfpathlineto{\pgfqpoint{3.010670in}{2.165275in}}%
\pgfpathlineto{\pgfqpoint{3.010434in}{2.165459in}}%
\pgfpathlineto{\pgfqpoint{3.004174in}{2.169727in}}%
\pgfpathlineto{\pgfqpoint{3.001063in}{2.171536in}}%
\pgfpathlineto{\pgfqpoint{2.997913in}{2.173305in}}%
\pgfpathlineto{\pgfqpoint{2.991653in}{2.176260in}}%
\pgfpathlineto{\pgfqpoint{2.987664in}{2.177796in}}%
\pgfpathlineto{\pgfqpoint{2.985392in}{2.178653in}}%
\pgfpathlineto{\pgfqpoint{2.979132in}{2.180517in}}%
\pgfpathlineto{\pgfqpoint{2.972871in}{2.181894in}}%
\pgfpathlineto{\pgfqpoint{2.966611in}{2.182799in}}%
\pgfpathlineto{\pgfqpoint{2.960351in}{2.183247in}}%
\pgfpathlineto{\pgfqpoint{2.954090in}{2.183244in}}%
\pgfpathlineto{\pgfqpoint{2.947830in}{2.182796in}}%
\pgfpathlineto{\pgfqpoint{2.941569in}{2.181901in}}%
\pgfpathlineto{\pgfqpoint{2.935309in}{2.180556in}}%
\pgfpathlineto{\pgfqpoint{2.929048in}{2.178750in}}%
\pgfpathlineto{\pgfqpoint{2.926411in}{2.177796in}}%
\pgfpathlineto{\pgfqpoint{2.922788in}{2.176476in}}%
\pgfpathlineto{\pgfqpoint{2.916528in}{2.173724in}}%
\pgfpathlineto{\pgfqpoint{2.912300in}{2.171536in}}%
\pgfpathlineto{\pgfqpoint{2.910267in}{2.170472in}}%
\pgfpathlineto{\pgfqpoint{2.904007in}{2.166709in}}%
\pgfpathlineto{\pgfqpoint{2.901899in}{2.165275in}}%
\pgfpathlineto{\pgfqpoint{2.897746in}{2.162410in}}%
\pgfpathlineto{\pgfqpoint{2.893351in}{2.159015in}}%
\pgfpathlineto{\pgfqpoint{2.891486in}{2.157550in}}%
\pgfpathlineto{\pgfqpoint{2.885964in}{2.152755in}}%
\pgfpathlineto{\pgfqpoint{2.885225in}{2.152100in}}%
\pgfpathlineto{\pgfqpoint{2.879438in}{2.146494in}}%
\pgfpathlineto{\pgfqpoint{2.878965in}{2.146025in}}%
\pgfpathlineto{\pgfqpoint{2.873571in}{2.140234in}}%
\pgfpathlineto{\pgfqpoint{2.872705in}{2.139279in}}%
\pgfpathlineto{\pgfqpoint{2.868224in}{2.133973in}}%
\pgfpathlineto{\pgfqpoint{2.866444in}{2.131806in}}%
\pgfpathlineto{\pgfqpoint{2.863298in}{2.127713in}}%
\pgfpathlineto{\pgfqpoint{2.860184in}{2.123534in}}%
\pgfpathlineto{\pgfqpoint{2.858722in}{2.121452in}}%
\pgfpathlineto{\pgfqpoint{2.854466in}{2.115192in}}%
\pgfpathlineto{\pgfqpoint{2.853923in}{2.114369in}}%
\pgfpathlineto{\pgfqpoint{2.850525in}{2.108932in}}%
\pgfpathlineto{\pgfqpoint{2.847663in}{2.104168in}}%
\pgfpathlineto{\pgfqpoint{2.846806in}{2.102671in}}%
\pgfpathlineto{\pgfqpoint{2.843348in}{2.096411in}}%
\pgfpathlineto{\pgfqpoint{2.841402in}{2.092737in}}%
\pgfpathlineto{\pgfqpoint{2.840091in}{2.090150in}}%
\pgfpathlineto{\pgfqpoint{2.837047in}{2.083890in}}%
\pgfpathlineto{\pgfqpoint{2.835142in}{2.079780in}}%
\pgfpathlineto{\pgfqpoint{2.834182in}{2.077629in}}%
\pgfpathlineto{\pgfqpoint{2.831518in}{2.071369in}}%
\pgfpathlineto{\pgfqpoint{2.829000in}{2.065109in}}%
\pgfpathlineto{\pgfqpoint{2.828881in}{2.064799in}}%
\pgfpathlineto{\pgfqpoint{2.826686in}{2.058848in}}%
\pgfpathlineto{\pgfqpoint{2.824519in}{2.052588in}}%
\pgfpathlineto{\pgfqpoint{2.822621in}{2.046723in}}%
\pgfpathlineto{\pgfqpoint{2.822497in}{2.046327in}}%
\pgfpathlineto{\pgfqpoint{2.820664in}{2.040067in}}%
\pgfpathlineto{\pgfqpoint{2.818975in}{2.033806in}}%
\pgfpathlineto{\pgfqpoint{2.817431in}{2.027546in}}%
\pgfpathlineto{\pgfqpoint{2.816361in}{2.022748in}}%
\pgfpathlineto{\pgfqpoint{2.816043in}{2.021285in}}%
\pgfpathlineto{\pgfqpoint{2.814829in}{2.015025in}}%
\pgfpathlineto{\pgfqpoint{2.813769in}{2.008765in}}%
\pgfpathlineto{\pgfqpoint{2.812865in}{2.002504in}}%
\pgfpathlineto{\pgfqpoint{2.812123in}{1.996244in}}%
\pgfpathlineto{\pgfqpoint{2.811549in}{1.989983in}}%
\pgfpathlineto{\pgfqpoint{2.811150in}{1.983723in}}%
\pgfpathlineto{\pgfqpoint{2.810933in}{1.977462in}}%
\pgfpathlineto{\pgfqpoint{2.810909in}{1.971202in}}%
\pgfpathlineto{\pgfqpoint{2.811086in}{1.964942in}}%
\pgfpathlineto{\pgfqpoint{2.811478in}{1.958681in}}%
\pgfpathlineto{\pgfqpoint{2.812098in}{1.952421in}}%
\pgfpathlineto{\pgfqpoint{2.812959in}{1.946160in}}%
\pgfpathlineto{\pgfqpoint{2.814081in}{1.939900in}}%
\pgfpathlineto{\pgfqpoint{2.815480in}{1.933639in}}%
\pgfpathlineto{\pgfqpoint{2.816361in}{1.930364in}}%
\pgfpathlineto{\pgfqpoint{2.817219in}{1.927379in}}%
\pgfpathlineto{\pgfqpoint{2.819344in}{1.921119in}}%
\pgfpathlineto{\pgfqpoint{2.821845in}{1.914858in}}%
\pgfpathlineto{\pgfqpoint{2.822621in}{1.913158in}}%
\pgfpathlineto{\pgfqpoint{2.824889in}{1.908598in}}%
\pgfpathlineto{\pgfqpoint{2.828470in}{1.902337in}}%
\pgfpathlineto{\pgfqpoint{2.828881in}{1.901694in}}%
\pgfpathlineto{\pgfqpoint{2.832877in}{1.896077in}}%
\pgfpathlineto{\pgfqpoint{2.835142in}{1.893245in}}%
\pgfpathlineto{\pgfqpoint{2.838247in}{1.889816in}}%
\pgfpathlineto{\pgfqpoint{2.841402in}{1.886679in}}%
\pgfpathlineto{\pgfqpoint{2.845043in}{1.883556in}}%
\pgfpathlineto{\pgfqpoint{2.847663in}{1.881510in}}%
\pgfpathlineto{\pgfqpoint{2.853923in}{1.877412in}}%
\pgfpathlineto{\pgfqpoint{2.854139in}{1.877295in}}%
\pgfpathlineto{\pgfqpoint{2.860184in}{1.874277in}}%
\pgfpathlineto{\pgfqpoint{2.866444in}{1.871853in}}%
\pgfpathlineto{\pgfqpoint{2.869279in}{1.871035in}}%
\pgfpathclose%
\pgfpathmoveto{\pgfqpoint{2.911314in}{1.971202in}}%
\pgfpathlineto{\pgfqpoint{2.910267in}{1.971528in}}%
\pgfpathlineto{\pgfqpoint{2.904007in}{1.975622in}}%
\pgfpathlineto{\pgfqpoint{2.902243in}{1.977462in}}%
\pgfpathlineto{\pgfqpoint{2.897898in}{1.983723in}}%
\pgfpathlineto{\pgfqpoint{2.897746in}{1.984046in}}%
\pgfpathlineto{\pgfqpoint{2.895658in}{1.989983in}}%
\pgfpathlineto{\pgfqpoint{2.894416in}{1.996244in}}%
\pgfpathlineto{\pgfqpoint{2.893994in}{2.002504in}}%
\pgfpathlineto{\pgfqpoint{2.894274in}{2.008765in}}%
\pgfpathlineto{\pgfqpoint{2.895165in}{2.015025in}}%
\pgfpathlineto{\pgfqpoint{2.896594in}{2.021285in}}%
\pgfpathlineto{\pgfqpoint{2.897746in}{2.025014in}}%
\pgfpathlineto{\pgfqpoint{2.898635in}{2.027546in}}%
\pgfpathlineto{\pgfqpoint{2.901357in}{2.033806in}}%
\pgfpathlineto{\pgfqpoint{2.904007in}{2.039045in}}%
\pgfpathlineto{\pgfqpoint{2.904603in}{2.040067in}}%
\pgfpathlineto{\pgfqpoint{2.908785in}{2.046327in}}%
\pgfpathlineto{\pgfqpoint{2.910267in}{2.048322in}}%
\pgfpathlineto{\pgfqpoint{2.914001in}{2.052588in}}%
\pgfpathlineto{\pgfqpoint{2.916528in}{2.055224in}}%
\pgfpathlineto{\pgfqpoint{2.920764in}{2.058848in}}%
\pgfpathlineto{\pgfqpoint{2.922788in}{2.060445in}}%
\pgfpathlineto{\pgfqpoint{2.929048in}{2.064258in}}%
\pgfpathlineto{\pgfqpoint{2.931094in}{2.065109in}}%
\pgfpathlineto{\pgfqpoint{2.935309in}{2.066734in}}%
\pgfpathlineto{\pgfqpoint{2.941569in}{2.067946in}}%
\pgfpathlineto{\pgfqpoint{2.947830in}{2.067821in}}%
\pgfpathlineto{\pgfqpoint{2.954090in}{2.066139in}}%
\pgfpathlineto{\pgfqpoint{2.955995in}{2.065109in}}%
\pgfpathlineto{\pgfqpoint{2.960351in}{2.062160in}}%
\pgfpathlineto{\pgfqpoint{2.963462in}{2.058848in}}%
\pgfpathlineto{\pgfqpoint{2.966611in}{2.054289in}}%
\pgfpathlineto{\pgfqpoint{2.967470in}{2.052588in}}%
\pgfpathlineto{\pgfqpoint{2.969634in}{2.046327in}}%
\pgfpathlineto{\pgfqpoint{2.970818in}{2.040067in}}%
\pgfpathlineto{\pgfqpoint{2.971151in}{2.033806in}}%
\pgfpathlineto{\pgfqpoint{2.970743in}{2.027546in}}%
\pgfpathlineto{\pgfqpoint{2.969689in}{2.021285in}}%
\pgfpathlineto{\pgfqpoint{2.968067in}{2.015025in}}%
\pgfpathlineto{\pgfqpoint{2.966611in}{2.010788in}}%
\pgfpathlineto{\pgfqpoint{2.965823in}{2.008765in}}%
\pgfpathlineto{\pgfqpoint{2.962821in}{2.002504in}}%
\pgfpathlineto{\pgfqpoint{2.960351in}{1.998042in}}%
\pgfpathlineto{\pgfqpoint{2.959209in}{1.996244in}}%
\pgfpathlineto{\pgfqpoint{2.954714in}{1.989983in}}%
\pgfpathlineto{\pgfqpoint{2.954090in}{1.989205in}}%
\pgfpathlineto{\pgfqpoint{2.948979in}{1.983723in}}%
\pgfpathlineto{\pgfqpoint{2.947830in}{1.982593in}}%
\pgfpathlineto{\pgfqpoint{2.941578in}{1.977462in}}%
\pgfpathlineto{\pgfqpoint{2.941569in}{1.977456in}}%
\pgfpathlineto{\pgfqpoint{2.935309in}{1.973797in}}%
\pgfpathlineto{\pgfqpoint{2.929188in}{1.971202in}}%
\pgfpathlineto{\pgfqpoint{2.929048in}{1.971147in}}%
\pgfpathlineto{\pgfqpoint{2.922788in}{1.969891in}}%
\pgfpathlineto{\pgfqpoint{2.916528in}{1.969916in}}%
\pgfpathclose%
\pgfusepath{fill}%
\end{pgfscope}%
\begin{pgfscope}%
\pgfpathrectangle{\pgfqpoint{0.500000in}{0.500000in}}{\pgfqpoint{3.750000in}{3.750000in}}%
\pgfusepath{clip}%
\pgfsetbuttcap%
\pgfsetroundjoin%
\definecolor{currentfill}{rgb}{0.018408,0.412826,0.648212}%
\pgfsetfillcolor{currentfill}%
\pgfsetlinewidth{0.000000pt}%
\definecolor{currentstroke}{rgb}{0.000000,0.000000,0.000000}%
\pgfsetstrokecolor{currentstroke}%
\pgfsetdash{}{0pt}%
\pgfpathmoveto{\pgfqpoint{2.860184in}{2.496051in}}%
\pgfpathlineto{\pgfqpoint{2.866444in}{2.495119in}}%
\pgfpathlineto{\pgfqpoint{2.872705in}{2.494833in}}%
\pgfpathlineto{\pgfqpoint{2.878965in}{2.495153in}}%
\pgfpathlineto{\pgfqpoint{2.885225in}{2.496051in}}%
\pgfpathlineto{\pgfqpoint{2.889638in}{2.497078in}}%
\pgfpathlineto{\pgfqpoint{2.891486in}{2.497500in}}%
\pgfpathlineto{\pgfqpoint{2.897746in}{2.499472in}}%
\pgfpathlineto{\pgfqpoint{2.904007in}{2.501961in}}%
\pgfpathlineto{\pgfqpoint{2.906868in}{2.503339in}}%
\pgfpathlineto{\pgfqpoint{2.910267in}{2.504946in}}%
\pgfpathlineto{\pgfqpoint{2.916528in}{2.508416in}}%
\pgfpathlineto{\pgfqpoint{2.918389in}{2.509599in}}%
\pgfpathlineto{\pgfqpoint{2.922788in}{2.512355in}}%
\pgfpathlineto{\pgfqpoint{2.927753in}{2.515860in}}%
\pgfpathlineto{\pgfqpoint{2.929048in}{2.516762in}}%
\pgfpathlineto{\pgfqpoint{2.935309in}{2.521622in}}%
\pgfpathlineto{\pgfqpoint{2.935892in}{2.522120in}}%
\pgfpathlineto{\pgfqpoint{2.941569in}{2.526926in}}%
\pgfpathlineto{\pgfqpoint{2.943142in}{2.528381in}}%
\pgfpathlineto{\pgfqpoint{2.947830in}{2.532677in}}%
\pgfpathlineto{\pgfqpoint{2.949806in}{2.534641in}}%
\pgfpathlineto{\pgfqpoint{2.954090in}{2.538870in}}%
\pgfpathlineto{\pgfqpoint{2.956001in}{2.540902in}}%
\pgfpathlineto{\pgfqpoint{2.960351in}{2.545505in}}%
\pgfpathlineto{\pgfqpoint{2.961813in}{2.547162in}}%
\pgfpathlineto{\pgfqpoint{2.966611in}{2.552584in}}%
\pgfpathlineto{\pgfqpoint{2.967308in}{2.553422in}}%
\pgfpathlineto{\pgfqpoint{2.972514in}{2.559683in}}%
\pgfpathlineto{\pgfqpoint{2.972871in}{2.560113in}}%
\pgfpathlineto{\pgfqpoint{2.977443in}{2.565943in}}%
\pgfpathlineto{\pgfqpoint{2.979132in}{2.568101in}}%
\pgfpathlineto{\pgfqpoint{2.982172in}{2.572204in}}%
\pgfpathlineto{\pgfqpoint{2.985392in}{2.576564in}}%
\pgfpathlineto{\pgfqpoint{2.986727in}{2.578464in}}%
\pgfpathlineto{\pgfqpoint{2.991100in}{2.584725in}}%
\pgfpathlineto{\pgfqpoint{2.991653in}{2.585520in}}%
\pgfpathlineto{\pgfqpoint{2.995274in}{2.590985in}}%
\pgfpathlineto{\pgfqpoint{2.997913in}{2.594999in}}%
\pgfpathlineto{\pgfqpoint{2.999327in}{2.597245in}}%
\pgfpathlineto{\pgfqpoint{3.003232in}{2.603506in}}%
\pgfpathlineto{\pgfqpoint{3.004174in}{2.605031in}}%
\pgfpathlineto{\pgfqpoint{3.006982in}{2.609766in}}%
\pgfpathlineto{\pgfqpoint{3.010434in}{2.615660in}}%
\pgfpathlineto{\pgfqpoint{3.010641in}{2.616027in}}%
\pgfpathlineto{\pgfqpoint{3.014129in}{2.622287in}}%
\pgfpathlineto{\pgfqpoint{3.016694in}{2.626959in}}%
\pgfpathlineto{\pgfqpoint{3.017537in}{2.628548in}}%
\pgfpathlineto{\pgfqpoint{3.020805in}{2.634808in}}%
\pgfpathlineto{\pgfqpoint{3.022955in}{2.638996in}}%
\pgfpathlineto{\pgfqpoint{3.023985in}{2.641068in}}%
\pgfpathlineto{\pgfqpoint{3.027040in}{2.647329in}}%
\pgfpathlineto{\pgfqpoint{3.029215in}{2.651878in}}%
\pgfpathlineto{\pgfqpoint{3.030010in}{2.653589in}}%
\pgfpathlineto{\pgfqpoint{3.032856in}{2.659850in}}%
\pgfpathlineto{\pgfqpoint{3.035476in}{2.665744in}}%
\pgfpathlineto{\pgfqpoint{3.035634in}{2.666110in}}%
\pgfpathlineto{\pgfqpoint{3.038276in}{2.672371in}}%
\pgfpathlineto{\pgfqpoint{3.040851in}{2.678631in}}%
\pgfpathlineto{\pgfqpoint{3.041736in}{2.680845in}}%
\pgfpathlineto{\pgfqpoint{3.043316in}{2.684891in}}%
\pgfpathlineto{\pgfqpoint{3.045687in}{2.691152in}}%
\pgfpathlineto{\pgfqpoint{3.047988in}{2.697412in}}%
\pgfpathlineto{\pgfqpoint{3.047997in}{2.697436in}}%
\pgfpathlineto{\pgfqpoint{3.050161in}{2.703673in}}%
\pgfpathlineto{\pgfqpoint{3.052260in}{2.709933in}}%
\pgfpathlineto{\pgfqpoint{3.054257in}{2.716114in}}%
\pgfpathlineto{\pgfqpoint{3.054282in}{2.716194in}}%
\pgfpathlineto{\pgfqpoint{3.056182in}{2.722454in}}%
\pgfpathlineto{\pgfqpoint{3.058003in}{2.728715in}}%
\pgfpathlineto{\pgfqpoint{3.059744in}{2.734975in}}%
\pgfpathlineto{\pgfqpoint{3.060518in}{2.737902in}}%
\pgfpathlineto{\pgfqpoint{3.061383in}{2.741235in}}%
\pgfpathlineto{\pgfqpoint{3.062921in}{2.747496in}}%
\pgfpathlineto{\pgfqpoint{3.064374in}{2.753756in}}%
\pgfpathlineto{\pgfqpoint{3.065740in}{2.760017in}}%
\pgfpathlineto{\pgfqpoint{3.066778in}{2.765110in}}%
\pgfpathlineto{\pgfqpoint{3.067012in}{2.766277in}}%
\pgfpathlineto{\pgfqpoint{3.068169in}{2.772538in}}%
\pgfpathlineto{\pgfqpoint{3.069234in}{2.778798in}}%
\pgfpathlineto{\pgfqpoint{3.070203in}{2.785058in}}%
\pgfpathlineto{\pgfqpoint{3.071073in}{2.791319in}}%
\pgfpathlineto{\pgfqpoint{3.071843in}{2.797579in}}%
\pgfpathlineto{\pgfqpoint{3.072508in}{2.803840in}}%
\pgfpathlineto{\pgfqpoint{3.073038in}{2.809810in}}%
\pgfpathlineto{\pgfqpoint{3.073064in}{2.810100in}}%
\pgfpathlineto{\pgfqpoint{3.073497in}{2.816361in}}%
\pgfpathlineto{\pgfqpoint{3.073816in}{2.822621in}}%
\pgfpathlineto{\pgfqpoint{3.074016in}{2.828881in}}%
\pgfpathlineto{\pgfqpoint{3.074094in}{2.835142in}}%
\pgfpathlineto{\pgfqpoint{3.074044in}{2.841402in}}%
\pgfpathlineto{\pgfqpoint{3.073860in}{2.847663in}}%
\pgfpathlineto{\pgfqpoint{3.073537in}{2.853923in}}%
\pgfpathlineto{\pgfqpoint{3.073067in}{2.860184in}}%
\pgfpathlineto{\pgfqpoint{3.073038in}{2.860475in}}%
\pgfpathlineto{\pgfqpoint{3.072422in}{2.866444in}}%
\pgfpathlineto{\pgfqpoint{3.071607in}{2.872705in}}%
\pgfpathlineto{\pgfqpoint{3.070614in}{2.878965in}}%
\pgfpathlineto{\pgfqpoint{3.069432in}{2.885225in}}%
\pgfpathlineto{\pgfqpoint{3.068051in}{2.891486in}}%
\pgfpathlineto{\pgfqpoint{3.066778in}{2.896507in}}%
\pgfpathlineto{\pgfqpoint{3.066441in}{2.897746in}}%
\pgfpathlineto{\pgfqpoint{3.064529in}{2.904007in}}%
\pgfpathlineto{\pgfqpoint{3.062359in}{2.910267in}}%
\pgfpathlineto{\pgfqpoint{3.060518in}{2.915009in}}%
\pgfpathlineto{\pgfqpoint{3.059876in}{2.916528in}}%
\pgfpathlineto{\pgfqpoint{3.056958in}{2.922788in}}%
\pgfpathlineto{\pgfqpoint{3.054257in}{2.927990in}}%
\pgfpathlineto{\pgfqpoint{3.053649in}{2.929048in}}%
\pgfpathlineto{\pgfqpoint{3.049724in}{2.935309in}}%
\pgfpathlineto{\pgfqpoint{3.047997in}{2.937825in}}%
\pgfpathlineto{\pgfqpoint{3.045099in}{2.941569in}}%
\pgfpathlineto{\pgfqpoint{3.041736in}{2.945558in}}%
\pgfpathlineto{\pgfqpoint{3.039533in}{2.947830in}}%
\pgfpathlineto{\pgfqpoint{3.035476in}{2.951705in}}%
\pgfpathlineto{\pgfqpoint{3.032529in}{2.954090in}}%
\pgfpathlineto{\pgfqpoint{3.029215in}{2.956594in}}%
\pgfpathlineto{\pgfqpoint{3.023139in}{2.960351in}}%
\pgfpathlineto{\pgfqpoint{3.022955in}{2.960457in}}%
\pgfpathlineto{\pgfqpoint{3.016694in}{2.963390in}}%
\pgfpathlineto{\pgfqpoint{3.010434in}{2.965566in}}%
\pgfpathlineto{\pgfqpoint{3.006059in}{2.966611in}}%
\pgfpathlineto{\pgfqpoint{3.004174in}{2.967040in}}%
\pgfpathlineto{\pgfqpoint{2.997913in}{2.967866in}}%
\pgfpathlineto{\pgfqpoint{2.991653in}{2.968112in}}%
\pgfpathlineto{\pgfqpoint{2.985392in}{2.967816in}}%
\pgfpathlineto{\pgfqpoint{2.979132in}{2.967009in}}%
\pgfpathlineto{\pgfqpoint{2.977223in}{2.966611in}}%
\pgfpathlineto{\pgfqpoint{2.972871in}{2.965705in}}%
\pgfpathlineto{\pgfqpoint{2.966611in}{2.963931in}}%
\pgfpathlineto{\pgfqpoint{2.960351in}{2.961716in}}%
\pgfpathlineto{\pgfqpoint{2.957133in}{2.960351in}}%
\pgfpathlineto{\pgfqpoint{2.954090in}{2.959057in}}%
\pgfpathlineto{\pgfqpoint{2.947830in}{2.955966in}}%
\pgfpathlineto{\pgfqpoint{2.944481in}{2.954090in}}%
\pgfpathlineto{\pgfqpoint{2.941569in}{2.952453in}}%
\pgfpathlineto{\pgfqpoint{2.935309in}{2.948524in}}%
\pgfpathlineto{\pgfqpoint{2.934307in}{2.947830in}}%
\pgfpathlineto{\pgfqpoint{2.929048in}{2.944163in}}%
\pgfpathlineto{\pgfqpoint{2.925636in}{2.941569in}}%
\pgfpathlineto{\pgfqpoint{2.922788in}{2.939388in}}%
\pgfpathlineto{\pgfqpoint{2.917866in}{2.935309in}}%
\pgfpathlineto{\pgfqpoint{2.916528in}{2.934190in}}%
\pgfpathlineto{\pgfqpoint{2.910806in}{2.929048in}}%
\pgfpathlineto{\pgfqpoint{2.910267in}{2.928559in}}%
\pgfpathlineto{\pgfqpoint{2.904318in}{2.922788in}}%
\pgfpathlineto{\pgfqpoint{2.904007in}{2.922483in}}%
\pgfpathlineto{\pgfqpoint{2.898296in}{2.916528in}}%
\pgfpathlineto{\pgfqpoint{2.897746in}{2.915947in}}%
\pgfpathlineto{\pgfqpoint{2.892662in}{2.910267in}}%
\pgfpathlineto{\pgfqpoint{2.891486in}{2.908935in}}%
\pgfpathlineto{\pgfqpoint{2.887355in}{2.904007in}}%
\pgfpathlineto{\pgfqpoint{2.885225in}{2.901429in}}%
\pgfpathlineto{\pgfqpoint{2.882326in}{2.897746in}}%
\pgfpathlineto{\pgfqpoint{2.878965in}{2.893408in}}%
\pgfpathlineto{\pgfqpoint{2.877540in}{2.891486in}}%
\pgfpathlineto{\pgfqpoint{2.872979in}{2.885225in}}%
\pgfpathlineto{\pgfqpoint{2.872705in}{2.884842in}}%
\pgfpathlineto{\pgfqpoint{2.868670in}{2.878965in}}%
\pgfpathlineto{\pgfqpoint{2.866444in}{2.875662in}}%
\pgfpathlineto{\pgfqpoint{2.864525in}{2.872705in}}%
\pgfpathlineto{\pgfqpoint{2.860545in}{2.866444in}}%
\pgfpathlineto{\pgfqpoint{2.860184in}{2.865864in}}%
\pgfpathlineto{\pgfqpoint{2.856769in}{2.860184in}}%
\pgfpathlineto{\pgfqpoint{2.853923in}{2.855345in}}%
\pgfpathlineto{\pgfqpoint{2.853113in}{2.853923in}}%
\pgfpathlineto{\pgfqpoint{2.849629in}{2.847663in}}%
\pgfpathlineto{\pgfqpoint{2.847663in}{2.844045in}}%
\pgfpathlineto{\pgfqpoint{2.846268in}{2.841402in}}%
\pgfpathlineto{\pgfqpoint{2.843045in}{2.835142in}}%
\pgfpathlineto{\pgfqpoint{2.841402in}{2.831867in}}%
\pgfpathlineto{\pgfqpoint{2.839944in}{2.828881in}}%
\pgfpathlineto{\pgfqpoint{2.836968in}{2.822621in}}%
\pgfpathlineto{\pgfqpoint{2.835142in}{2.818674in}}%
\pgfpathlineto{\pgfqpoint{2.834097in}{2.816361in}}%
\pgfpathlineto{\pgfqpoint{2.831353in}{2.810100in}}%
\pgfpathlineto{\pgfqpoint{2.828881in}{2.804295in}}%
\pgfpathlineto{\pgfqpoint{2.828692in}{2.803840in}}%
\pgfpathlineto{\pgfqpoint{2.826167in}{2.797579in}}%
\pgfpathlineto{\pgfqpoint{2.823720in}{2.791319in}}%
\pgfpathlineto{\pgfqpoint{2.822621in}{2.788408in}}%
\pgfpathlineto{\pgfqpoint{2.821380in}{2.785058in}}%
\pgfpathlineto{\pgfqpoint{2.819141in}{2.778798in}}%
\pgfpathlineto{\pgfqpoint{2.816980in}{2.772538in}}%
\pgfpathlineto{\pgfqpoint{2.816361in}{2.770672in}}%
\pgfpathlineto{\pgfqpoint{2.814927in}{2.766277in}}%
\pgfpathlineto{\pgfqpoint{2.812964in}{2.760017in}}%
\pgfpathlineto{\pgfqpoint{2.811079in}{2.753756in}}%
\pgfpathlineto{\pgfqpoint{2.810100in}{2.750358in}}%
\pgfpathlineto{\pgfqpoint{2.809287in}{2.747496in}}%
\pgfpathlineto{\pgfqpoint{2.807594in}{2.741235in}}%
\pgfpathlineto{\pgfqpoint{2.805978in}{2.734975in}}%
\pgfpathlineto{\pgfqpoint{2.804441in}{2.728715in}}%
\pgfpathlineto{\pgfqpoint{2.803840in}{2.726122in}}%
\pgfpathlineto{\pgfqpoint{2.802999in}{2.722454in}}%
\pgfpathlineto{\pgfqpoint{2.801650in}{2.716194in}}%
\pgfpathlineto{\pgfqpoint{2.800382in}{2.709933in}}%
\pgfpathlineto{\pgfqpoint{2.799195in}{2.703673in}}%
\pgfpathlineto{\pgfqpoint{2.798090in}{2.697412in}}%
\pgfpathlineto{\pgfqpoint{2.797579in}{2.694258in}}%
\pgfpathlineto{\pgfqpoint{2.797081in}{2.691152in}}%
\pgfpathlineto{\pgfqpoint{2.796168in}{2.684891in}}%
\pgfpathlineto{\pgfqpoint{2.795343in}{2.678631in}}%
\pgfpathlineto{\pgfqpoint{2.794607in}{2.672371in}}%
\pgfpathlineto{\pgfqpoint{2.793964in}{2.666110in}}%
\pgfpathlineto{\pgfqpoint{2.793416in}{2.659850in}}%
\pgfpathlineto{\pgfqpoint{2.792965in}{2.653589in}}%
\pgfpathlineto{\pgfqpoint{2.792616in}{2.647329in}}%
\pgfpathlineto{\pgfqpoint{2.792373in}{2.641068in}}%
\pgfpathlineto{\pgfqpoint{2.792239in}{2.634808in}}%
\pgfpathlineto{\pgfqpoint{2.792220in}{2.628548in}}%
\pgfpathlineto{\pgfqpoint{2.792321in}{2.622287in}}%
\pgfpathlineto{\pgfqpoint{2.792549in}{2.616027in}}%
\pgfpathlineto{\pgfqpoint{2.792909in}{2.609766in}}%
\pgfpathlineto{\pgfqpoint{2.793409in}{2.603506in}}%
\pgfpathlineto{\pgfqpoint{2.794056in}{2.597245in}}%
\pgfpathlineto{\pgfqpoint{2.794861in}{2.590985in}}%
\pgfpathlineto{\pgfqpoint{2.795831in}{2.584725in}}%
\pgfpathlineto{\pgfqpoint{2.796978in}{2.578464in}}%
\pgfpathlineto{\pgfqpoint{2.797579in}{2.575599in}}%
\pgfpathlineto{\pgfqpoint{2.798344in}{2.572204in}}%
\pgfpathlineto{\pgfqpoint{2.799950in}{2.565943in}}%
\pgfpathlineto{\pgfqpoint{2.801783in}{2.559683in}}%
\pgfpathlineto{\pgfqpoint{2.803840in}{2.553482in}}%
\pgfpathlineto{\pgfqpoint{2.803861in}{2.553422in}}%
\pgfpathlineto{\pgfqpoint{2.806335in}{2.547162in}}%
\pgfpathlineto{\pgfqpoint{2.809116in}{2.540902in}}%
\pgfpathlineto{\pgfqpoint{2.810100in}{2.538870in}}%
\pgfpathlineto{\pgfqpoint{2.812376in}{2.534641in}}%
\pgfpathlineto{\pgfqpoint{2.816098in}{2.528381in}}%
\pgfpathlineto{\pgfqpoint{2.816361in}{2.527970in}}%
\pgfpathlineto{\pgfqpoint{2.820586in}{2.522120in}}%
\pgfpathlineto{\pgfqpoint{2.822621in}{2.519524in}}%
\pgfpathlineto{\pgfqpoint{2.825936in}{2.515860in}}%
\pgfpathlineto{\pgfqpoint{2.828881in}{2.512823in}}%
\pgfpathlineto{\pgfqpoint{2.832582in}{2.509599in}}%
\pgfpathlineto{\pgfqpoint{2.835142in}{2.507496in}}%
\pgfpathlineto{\pgfqpoint{2.841332in}{2.503339in}}%
\pgfpathlineto{\pgfqpoint{2.841402in}{2.503294in}}%
\pgfpathlineto{\pgfqpoint{2.847663in}{2.500067in}}%
\pgfpathlineto{\pgfqpoint{2.853923in}{2.497678in}}%
\pgfpathlineto{\pgfqpoint{2.856188in}{2.497078in}}%
\pgfpathclose%
\pgfpathmoveto{\pgfqpoint{2.884563in}{2.616027in}}%
\pgfpathlineto{\pgfqpoint{2.878965in}{2.619816in}}%
\pgfpathlineto{\pgfqpoint{2.876336in}{2.622287in}}%
\pgfpathlineto{\pgfqpoint{2.872705in}{2.626186in}}%
\pgfpathlineto{\pgfqpoint{2.870984in}{2.628548in}}%
\pgfpathlineto{\pgfqpoint{2.867058in}{2.634808in}}%
\pgfpathlineto{\pgfqpoint{2.866444in}{2.635951in}}%
\pgfpathlineto{\pgfqpoint{2.864160in}{2.641068in}}%
\pgfpathlineto{\pgfqpoint{2.861841in}{2.647329in}}%
\pgfpathlineto{\pgfqpoint{2.860184in}{2.652848in}}%
\pgfpathlineto{\pgfqpoint{2.859992in}{2.653589in}}%
\pgfpathlineto{\pgfqpoint{2.858664in}{2.659850in}}%
\pgfpathlineto{\pgfqpoint{2.857654in}{2.666110in}}%
\pgfpathlineto{\pgfqpoint{2.856934in}{2.672371in}}%
\pgfpathlineto{\pgfqpoint{2.856480in}{2.678631in}}%
\pgfpathlineto{\pgfqpoint{2.856272in}{2.684891in}}%
\pgfpathlineto{\pgfqpoint{2.856290in}{2.691152in}}%
\pgfpathlineto{\pgfqpoint{2.856520in}{2.697412in}}%
\pgfpathlineto{\pgfqpoint{2.856946in}{2.703673in}}%
\pgfpathlineto{\pgfqpoint{2.857556in}{2.709933in}}%
\pgfpathlineto{\pgfqpoint{2.858339in}{2.716194in}}%
\pgfpathlineto{\pgfqpoint{2.859285in}{2.722454in}}%
\pgfpathlineto{\pgfqpoint{2.860184in}{2.727551in}}%
\pgfpathlineto{\pgfqpoint{2.860400in}{2.728715in}}%
\pgfpathlineto{\pgfqpoint{2.861735in}{2.734975in}}%
\pgfpathlineto{\pgfqpoint{2.863215in}{2.741235in}}%
\pgfpathlineto{\pgfqpoint{2.864832in}{2.747496in}}%
\pgfpathlineto{\pgfqpoint{2.866444in}{2.753260in}}%
\pgfpathlineto{\pgfqpoint{2.866591in}{2.753756in}}%
\pgfpathlineto{\pgfqpoint{2.868599in}{2.760017in}}%
\pgfpathlineto{\pgfqpoint{2.870730in}{2.766277in}}%
\pgfpathlineto{\pgfqpoint{2.872705in}{2.771762in}}%
\pgfpathlineto{\pgfqpoint{2.873000in}{2.772538in}}%
\pgfpathlineto{\pgfqpoint{2.875537in}{2.778798in}}%
\pgfpathlineto{\pgfqpoint{2.878183in}{2.785058in}}%
\pgfpathlineto{\pgfqpoint{2.878965in}{2.786818in}}%
\pgfpathlineto{\pgfqpoint{2.881090in}{2.791319in}}%
\pgfpathlineto{\pgfqpoint{2.884158in}{2.797579in}}%
\pgfpathlineto{\pgfqpoint{2.885225in}{2.799669in}}%
\pgfpathlineto{\pgfqpoint{2.887499in}{2.803840in}}%
\pgfpathlineto{\pgfqpoint{2.891019in}{2.810100in}}%
\pgfpathlineto{\pgfqpoint{2.891486in}{2.810899in}}%
\pgfpathlineto{\pgfqpoint{2.894908in}{2.816361in}}%
\pgfpathlineto{\pgfqpoint{2.897746in}{2.820775in}}%
\pgfpathlineto{\pgfqpoint{2.899029in}{2.822621in}}%
\pgfpathlineto{\pgfqpoint{2.903495in}{2.828881in}}%
\pgfpathlineto{\pgfqpoint{2.904007in}{2.829578in}}%
\pgfpathlineto{\pgfqpoint{2.908450in}{2.835142in}}%
\pgfpathlineto{\pgfqpoint{2.910267in}{2.837369in}}%
\pgfpathlineto{\pgfqpoint{2.913880in}{2.841402in}}%
\pgfpathlineto{\pgfqpoint{2.916528in}{2.844298in}}%
\pgfpathlineto{\pgfqpoint{2.919943in}{2.847663in}}%
\pgfpathlineto{\pgfqpoint{2.922788in}{2.850413in}}%
\pgfpathlineto{\pgfqpoint{2.926875in}{2.853923in}}%
\pgfpathlineto{\pgfqpoint{2.929048in}{2.855757in}}%
\pgfpathlineto{\pgfqpoint{2.935053in}{2.860184in}}%
\pgfpathlineto{\pgfqpoint{2.935309in}{2.860369in}}%
\pgfpathlineto{\pgfqpoint{2.941569in}{2.864184in}}%
\pgfpathlineto{\pgfqpoint{2.946099in}{2.866444in}}%
\pgfpathlineto{\pgfqpoint{2.947830in}{2.867295in}}%
\pgfpathlineto{\pgfqpoint{2.954090in}{2.869602in}}%
\pgfpathlineto{\pgfqpoint{2.960351in}{2.871131in}}%
\pgfpathlineto{\pgfqpoint{2.966611in}{2.871796in}}%
\pgfpathlineto{\pgfqpoint{2.972871in}{2.871493in}}%
\pgfpathlineto{\pgfqpoint{2.979132in}{2.870095in}}%
\pgfpathlineto{\pgfqpoint{2.985392in}{2.867450in}}%
\pgfpathlineto{\pgfqpoint{2.986995in}{2.866444in}}%
\pgfpathlineto{\pgfqpoint{2.991653in}{2.863153in}}%
\pgfpathlineto{\pgfqpoint{2.994749in}{2.860184in}}%
\pgfpathlineto{\pgfqpoint{2.997913in}{2.856701in}}%
\pgfpathlineto{\pgfqpoint{2.999915in}{2.853923in}}%
\pgfpathlineto{\pgfqpoint{3.003772in}{2.847663in}}%
\pgfpathlineto{\pgfqpoint{3.004174in}{2.846902in}}%
\pgfpathlineto{\pgfqpoint{3.006601in}{2.841402in}}%
\pgfpathlineto{\pgfqpoint{3.008878in}{2.835142in}}%
\pgfpathlineto{\pgfqpoint{3.010434in}{2.829843in}}%
\pgfpathlineto{\pgfqpoint{3.010679in}{2.828881in}}%
\pgfpathlineto{\pgfqpoint{3.011966in}{2.822621in}}%
\pgfpathlineto{\pgfqpoint{3.012930in}{2.816361in}}%
\pgfpathlineto{\pgfqpoint{3.013599in}{2.810100in}}%
\pgfpathlineto{\pgfqpoint{3.013995in}{2.803840in}}%
\pgfpathlineto{\pgfqpoint{3.014141in}{2.797579in}}%
\pgfpathlineto{\pgfqpoint{3.014054in}{2.791319in}}%
\pgfpathlineto{\pgfqpoint{3.013752in}{2.785058in}}%
\pgfpathlineto{\pgfqpoint{3.013248in}{2.778798in}}%
\pgfpathlineto{\pgfqpoint{3.012558in}{2.772538in}}%
\pgfpathlineto{\pgfqpoint{3.011692in}{2.766277in}}%
\pgfpathlineto{\pgfqpoint{3.010661in}{2.760017in}}%
\pgfpathlineto{\pgfqpoint{3.010434in}{2.758839in}}%
\pgfpathlineto{\pgfqpoint{3.009402in}{2.753756in}}%
\pgfpathlineto{\pgfqpoint{3.007975in}{2.747496in}}%
\pgfpathlineto{\pgfqpoint{3.006404in}{2.741235in}}%
\pgfpathlineto{\pgfqpoint{3.004697in}{2.734975in}}%
\pgfpathlineto{\pgfqpoint{3.004174in}{2.733209in}}%
\pgfpathlineto{\pgfqpoint{3.002765in}{2.728715in}}%
\pgfpathlineto{\pgfqpoint{3.000669in}{2.722454in}}%
\pgfpathlineto{\pgfqpoint{2.998456in}{2.716194in}}%
\pgfpathlineto{\pgfqpoint{2.997913in}{2.714748in}}%
\pgfpathlineto{\pgfqpoint{2.995995in}{2.709933in}}%
\pgfpathlineto{\pgfqpoint{2.993388in}{2.703673in}}%
\pgfpathlineto{\pgfqpoint{2.991653in}{2.699680in}}%
\pgfpathlineto{\pgfqpoint{2.990603in}{2.697412in}}%
\pgfpathlineto{\pgfqpoint{2.987582in}{2.691152in}}%
\pgfpathlineto{\pgfqpoint{2.985392in}{2.686758in}}%
\pgfpathlineto{\pgfqpoint{2.984396in}{2.684891in}}%
\pgfpathlineto{\pgfqpoint{2.980942in}{2.678631in}}%
\pgfpathlineto{\pgfqpoint{2.979132in}{2.675440in}}%
\pgfpathlineto{\pgfqpoint{2.977259in}{2.672371in}}%
\pgfpathlineto{\pgfqpoint{2.973349in}{2.666110in}}%
\pgfpathlineto{\pgfqpoint{2.972871in}{2.665366in}}%
\pgfpathlineto{\pgfqpoint{2.969041in}{2.659850in}}%
\pgfpathlineto{\pgfqpoint{2.966611in}{2.656411in}}%
\pgfpathlineto{\pgfqpoint{2.964437in}{2.653589in}}%
\pgfpathlineto{\pgfqpoint{2.960351in}{2.648368in}}%
\pgfpathlineto{\pgfqpoint{2.959455in}{2.647329in}}%
\pgfpathlineto{\pgfqpoint{2.954090in}{2.641186in}}%
\pgfpathlineto{\pgfqpoint{2.953975in}{2.641068in}}%
\pgfpathlineto{\pgfqpoint{2.947830in}{2.634823in}}%
\pgfpathlineto{\pgfqpoint{2.947813in}{2.634808in}}%
\pgfpathlineto{\pgfqpoint{2.941569in}{2.629241in}}%
\pgfpathlineto{\pgfqpoint{2.940676in}{2.628548in}}%
\pgfpathlineto{\pgfqpoint{2.935309in}{2.624412in}}%
\pgfpathlineto{\pgfqpoint{2.932073in}{2.622287in}}%
\pgfpathlineto{\pgfqpoint{2.929048in}{2.620312in}}%
\pgfpathlineto{\pgfqpoint{2.922788in}{2.616960in}}%
\pgfpathlineto{\pgfqpoint{2.920507in}{2.616027in}}%
\pgfpathlineto{\pgfqpoint{2.916528in}{2.614403in}}%
\pgfpathlineto{\pgfqpoint{2.910267in}{2.612666in}}%
\pgfpathlineto{\pgfqpoint{2.904007in}{2.611801in}}%
\pgfpathlineto{\pgfqpoint{2.897746in}{2.611912in}}%
\pgfpathlineto{\pgfqpoint{2.891486in}{2.613132in}}%
\pgfpathlineto{\pgfqpoint{2.885225in}{2.615626in}}%
\pgfpathclose%
\pgfusepath{fill}%
\end{pgfscope}%
\begin{pgfscope}%
\pgfpathrectangle{\pgfqpoint{0.500000in}{0.500000in}}{\pgfqpoint{3.750000in}{3.750000in}}%
\pgfusepath{clip}%
\pgfsetbuttcap%
\pgfsetroundjoin%
\definecolor{currentfill}{rgb}{0.012272,0.294902,0.462468}%
\pgfsetfillcolor{currentfill}%
\pgfsetlinewidth{0.000000pt}%
\definecolor{currentstroke}{rgb}{0.000000,0.000000,0.000000}%
\pgfsetstrokecolor{currentstroke}%
\pgfsetdash{}{0pt}%
\pgfpathmoveto{\pgfqpoint{2.916528in}{1.969916in}}%
\pgfpathlineto{\pgfqpoint{2.922788in}{1.969891in}}%
\pgfpathlineto{\pgfqpoint{2.929048in}{1.971147in}}%
\pgfpathlineto{\pgfqpoint{2.929188in}{1.971202in}}%
\pgfpathlineto{\pgfqpoint{2.935309in}{1.973797in}}%
\pgfpathlineto{\pgfqpoint{2.941569in}{1.977456in}}%
\pgfpathlineto{\pgfqpoint{2.941578in}{1.977462in}}%
\pgfpathlineto{\pgfqpoint{2.947830in}{1.982593in}}%
\pgfpathlineto{\pgfqpoint{2.948979in}{1.983723in}}%
\pgfpathlineto{\pgfqpoint{2.954090in}{1.989205in}}%
\pgfpathlineto{\pgfqpoint{2.954714in}{1.989983in}}%
\pgfpathlineto{\pgfqpoint{2.959209in}{1.996244in}}%
\pgfpathlineto{\pgfqpoint{2.960351in}{1.998042in}}%
\pgfpathlineto{\pgfqpoint{2.962821in}{2.002504in}}%
\pgfpathlineto{\pgfqpoint{2.965823in}{2.008765in}}%
\pgfpathlineto{\pgfqpoint{2.966611in}{2.010788in}}%
\pgfpathlineto{\pgfqpoint{2.968067in}{2.015025in}}%
\pgfpathlineto{\pgfqpoint{2.969689in}{2.021285in}}%
\pgfpathlineto{\pgfqpoint{2.970743in}{2.027546in}}%
\pgfpathlineto{\pgfqpoint{2.971151in}{2.033806in}}%
\pgfpathlineto{\pgfqpoint{2.970818in}{2.040067in}}%
\pgfpathlineto{\pgfqpoint{2.969634in}{2.046327in}}%
\pgfpathlineto{\pgfqpoint{2.967470in}{2.052588in}}%
\pgfpathlineto{\pgfqpoint{2.966611in}{2.054289in}}%
\pgfpathlineto{\pgfqpoint{2.963462in}{2.058848in}}%
\pgfpathlineto{\pgfqpoint{2.960351in}{2.062160in}}%
\pgfpathlineto{\pgfqpoint{2.955995in}{2.065109in}}%
\pgfpathlineto{\pgfqpoint{2.954090in}{2.066139in}}%
\pgfpathlineto{\pgfqpoint{2.947830in}{2.067821in}}%
\pgfpathlineto{\pgfqpoint{2.941569in}{2.067946in}}%
\pgfpathlineto{\pgfqpoint{2.935309in}{2.066734in}}%
\pgfpathlineto{\pgfqpoint{2.931094in}{2.065109in}}%
\pgfpathlineto{\pgfqpoint{2.929048in}{2.064258in}}%
\pgfpathlineto{\pgfqpoint{2.922788in}{2.060445in}}%
\pgfpathlineto{\pgfqpoint{2.920764in}{2.058848in}}%
\pgfpathlineto{\pgfqpoint{2.916528in}{2.055224in}}%
\pgfpathlineto{\pgfqpoint{2.914001in}{2.052588in}}%
\pgfpathlineto{\pgfqpoint{2.910267in}{2.048322in}}%
\pgfpathlineto{\pgfqpoint{2.908785in}{2.046327in}}%
\pgfpathlineto{\pgfqpoint{2.904603in}{2.040067in}}%
\pgfpathlineto{\pgfqpoint{2.904007in}{2.039045in}}%
\pgfpathlineto{\pgfqpoint{2.901357in}{2.033806in}}%
\pgfpathlineto{\pgfqpoint{2.898635in}{2.027546in}}%
\pgfpathlineto{\pgfqpoint{2.897746in}{2.025014in}}%
\pgfpathlineto{\pgfqpoint{2.896594in}{2.021285in}}%
\pgfpathlineto{\pgfqpoint{2.895165in}{2.015025in}}%
\pgfpathlineto{\pgfqpoint{2.894274in}{2.008765in}}%
\pgfpathlineto{\pgfqpoint{2.893994in}{2.002504in}}%
\pgfpathlineto{\pgfqpoint{2.894416in}{1.996244in}}%
\pgfpathlineto{\pgfqpoint{2.895658in}{1.989983in}}%
\pgfpathlineto{\pgfqpoint{2.897746in}{1.984046in}}%
\pgfpathlineto{\pgfqpoint{2.897898in}{1.983723in}}%
\pgfpathlineto{\pgfqpoint{2.902243in}{1.977462in}}%
\pgfpathlineto{\pgfqpoint{2.904007in}{1.975622in}}%
\pgfpathlineto{\pgfqpoint{2.910267in}{1.971528in}}%
\pgfpathlineto{\pgfqpoint{2.911314in}{1.971202in}}%
\pgfpathclose%
\pgfusepath{fill}%
\end{pgfscope}%
\begin{pgfscope}%
\pgfpathrectangle{\pgfqpoint{0.500000in}{0.500000in}}{\pgfqpoint{3.750000in}{3.750000in}}%
\pgfusepath{clip}%
\pgfsetbuttcap%
\pgfsetroundjoin%
\definecolor{currentfill}{rgb}{0.012272,0.294902,0.462468}%
\pgfsetfillcolor{currentfill}%
\pgfsetlinewidth{0.000000pt}%
\definecolor{currentstroke}{rgb}{0.000000,0.000000,0.000000}%
\pgfsetstrokecolor{currentstroke}%
\pgfsetdash{}{0pt}%
\pgfpathmoveto{\pgfqpoint{2.885225in}{2.615626in}}%
\pgfpathlineto{\pgfqpoint{2.891486in}{2.613132in}}%
\pgfpathlineto{\pgfqpoint{2.897746in}{2.611912in}}%
\pgfpathlineto{\pgfqpoint{2.904007in}{2.611801in}}%
\pgfpathlineto{\pgfqpoint{2.910267in}{2.612666in}}%
\pgfpathlineto{\pgfqpoint{2.916528in}{2.614403in}}%
\pgfpathlineto{\pgfqpoint{2.920507in}{2.616027in}}%
\pgfpathlineto{\pgfqpoint{2.922788in}{2.616960in}}%
\pgfpathlineto{\pgfqpoint{2.929048in}{2.620312in}}%
\pgfpathlineto{\pgfqpoint{2.932073in}{2.622287in}}%
\pgfpathlineto{\pgfqpoint{2.935309in}{2.624412in}}%
\pgfpathlineto{\pgfqpoint{2.940676in}{2.628548in}}%
\pgfpathlineto{\pgfqpoint{2.941569in}{2.629241in}}%
\pgfpathlineto{\pgfqpoint{2.947813in}{2.634808in}}%
\pgfpathlineto{\pgfqpoint{2.947830in}{2.634823in}}%
\pgfpathlineto{\pgfqpoint{2.953975in}{2.641068in}}%
\pgfpathlineto{\pgfqpoint{2.954090in}{2.641186in}}%
\pgfpathlineto{\pgfqpoint{2.959455in}{2.647329in}}%
\pgfpathlineto{\pgfqpoint{2.960351in}{2.648368in}}%
\pgfpathlineto{\pgfqpoint{2.964437in}{2.653589in}}%
\pgfpathlineto{\pgfqpoint{2.966611in}{2.656411in}}%
\pgfpathlineto{\pgfqpoint{2.969041in}{2.659850in}}%
\pgfpathlineto{\pgfqpoint{2.972871in}{2.665366in}}%
\pgfpathlineto{\pgfqpoint{2.973349in}{2.666110in}}%
\pgfpathlineto{\pgfqpoint{2.977259in}{2.672371in}}%
\pgfpathlineto{\pgfqpoint{2.979132in}{2.675440in}}%
\pgfpathlineto{\pgfqpoint{2.980942in}{2.678631in}}%
\pgfpathlineto{\pgfqpoint{2.984396in}{2.684891in}}%
\pgfpathlineto{\pgfqpoint{2.985392in}{2.686758in}}%
\pgfpathlineto{\pgfqpoint{2.987582in}{2.691152in}}%
\pgfpathlineto{\pgfqpoint{2.990603in}{2.697412in}}%
\pgfpathlineto{\pgfqpoint{2.991653in}{2.699680in}}%
\pgfpathlineto{\pgfqpoint{2.993388in}{2.703673in}}%
\pgfpathlineto{\pgfqpoint{2.995995in}{2.709933in}}%
\pgfpathlineto{\pgfqpoint{2.997913in}{2.714748in}}%
\pgfpathlineto{\pgfqpoint{2.998456in}{2.716194in}}%
\pgfpathlineto{\pgfqpoint{3.000669in}{2.722454in}}%
\pgfpathlineto{\pgfqpoint{3.002765in}{2.728715in}}%
\pgfpathlineto{\pgfqpoint{3.004174in}{2.733209in}}%
\pgfpathlineto{\pgfqpoint{3.004697in}{2.734975in}}%
\pgfpathlineto{\pgfqpoint{3.006404in}{2.741235in}}%
\pgfpathlineto{\pgfqpoint{3.007975in}{2.747496in}}%
\pgfpathlineto{\pgfqpoint{3.009402in}{2.753756in}}%
\pgfpathlineto{\pgfqpoint{3.010434in}{2.758839in}}%
\pgfpathlineto{\pgfqpoint{3.010661in}{2.760017in}}%
\pgfpathlineto{\pgfqpoint{3.011692in}{2.766277in}}%
\pgfpathlineto{\pgfqpoint{3.012558in}{2.772538in}}%
\pgfpathlineto{\pgfqpoint{3.013248in}{2.778798in}}%
\pgfpathlineto{\pgfqpoint{3.013752in}{2.785058in}}%
\pgfpathlineto{\pgfqpoint{3.014054in}{2.791319in}}%
\pgfpathlineto{\pgfqpoint{3.014141in}{2.797579in}}%
\pgfpathlineto{\pgfqpoint{3.013995in}{2.803840in}}%
\pgfpathlineto{\pgfqpoint{3.013599in}{2.810100in}}%
\pgfpathlineto{\pgfqpoint{3.012930in}{2.816361in}}%
\pgfpathlineto{\pgfqpoint{3.011966in}{2.822621in}}%
\pgfpathlineto{\pgfqpoint{3.010679in}{2.828881in}}%
\pgfpathlineto{\pgfqpoint{3.010434in}{2.829843in}}%
\pgfpathlineto{\pgfqpoint{3.008878in}{2.835142in}}%
\pgfpathlineto{\pgfqpoint{3.006601in}{2.841402in}}%
\pgfpathlineto{\pgfqpoint{3.004174in}{2.846902in}}%
\pgfpathlineto{\pgfqpoint{3.003772in}{2.847663in}}%
\pgfpathlineto{\pgfqpoint{2.999915in}{2.853923in}}%
\pgfpathlineto{\pgfqpoint{2.997913in}{2.856701in}}%
\pgfpathlineto{\pgfqpoint{2.994749in}{2.860184in}}%
\pgfpathlineto{\pgfqpoint{2.991653in}{2.863153in}}%
\pgfpathlineto{\pgfqpoint{2.986995in}{2.866444in}}%
\pgfpathlineto{\pgfqpoint{2.985392in}{2.867450in}}%
\pgfpathlineto{\pgfqpoint{2.979132in}{2.870095in}}%
\pgfpathlineto{\pgfqpoint{2.972871in}{2.871493in}}%
\pgfpathlineto{\pgfqpoint{2.966611in}{2.871796in}}%
\pgfpathlineto{\pgfqpoint{2.960351in}{2.871131in}}%
\pgfpathlineto{\pgfqpoint{2.954090in}{2.869602in}}%
\pgfpathlineto{\pgfqpoint{2.947830in}{2.867295in}}%
\pgfpathlineto{\pgfqpoint{2.946099in}{2.866444in}}%
\pgfpathlineto{\pgfqpoint{2.941569in}{2.864184in}}%
\pgfpathlineto{\pgfqpoint{2.935309in}{2.860369in}}%
\pgfpathlineto{\pgfqpoint{2.935053in}{2.860184in}}%
\pgfpathlineto{\pgfqpoint{2.929048in}{2.855757in}}%
\pgfpathlineto{\pgfqpoint{2.926875in}{2.853923in}}%
\pgfpathlineto{\pgfqpoint{2.922788in}{2.850413in}}%
\pgfpathlineto{\pgfqpoint{2.919943in}{2.847663in}}%
\pgfpathlineto{\pgfqpoint{2.916528in}{2.844298in}}%
\pgfpathlineto{\pgfqpoint{2.913880in}{2.841402in}}%
\pgfpathlineto{\pgfqpoint{2.910267in}{2.837369in}}%
\pgfpathlineto{\pgfqpoint{2.908450in}{2.835142in}}%
\pgfpathlineto{\pgfqpoint{2.904007in}{2.829578in}}%
\pgfpathlineto{\pgfqpoint{2.903495in}{2.828881in}}%
\pgfpathlineto{\pgfqpoint{2.899029in}{2.822621in}}%
\pgfpathlineto{\pgfqpoint{2.897746in}{2.820775in}}%
\pgfpathlineto{\pgfqpoint{2.894908in}{2.816361in}}%
\pgfpathlineto{\pgfqpoint{2.891486in}{2.810899in}}%
\pgfpathlineto{\pgfqpoint{2.891019in}{2.810100in}}%
\pgfpathlineto{\pgfqpoint{2.887499in}{2.803840in}}%
\pgfpathlineto{\pgfqpoint{2.885225in}{2.799669in}}%
\pgfpathlineto{\pgfqpoint{2.884158in}{2.797579in}}%
\pgfpathlineto{\pgfqpoint{2.881090in}{2.791319in}}%
\pgfpathlineto{\pgfqpoint{2.878965in}{2.786818in}}%
\pgfpathlineto{\pgfqpoint{2.878183in}{2.785058in}}%
\pgfpathlineto{\pgfqpoint{2.875537in}{2.778798in}}%
\pgfpathlineto{\pgfqpoint{2.873000in}{2.772538in}}%
\pgfpathlineto{\pgfqpoint{2.872705in}{2.771762in}}%
\pgfpathlineto{\pgfqpoint{2.870730in}{2.766277in}}%
\pgfpathlineto{\pgfqpoint{2.868599in}{2.760017in}}%
\pgfpathlineto{\pgfqpoint{2.866591in}{2.753756in}}%
\pgfpathlineto{\pgfqpoint{2.866444in}{2.753260in}}%
\pgfpathlineto{\pgfqpoint{2.864832in}{2.747496in}}%
\pgfpathlineto{\pgfqpoint{2.863215in}{2.741235in}}%
\pgfpathlineto{\pgfqpoint{2.861735in}{2.734975in}}%
\pgfpathlineto{\pgfqpoint{2.860400in}{2.728715in}}%
\pgfpathlineto{\pgfqpoint{2.860184in}{2.727551in}}%
\pgfpathlineto{\pgfqpoint{2.859285in}{2.722454in}}%
\pgfpathlineto{\pgfqpoint{2.858339in}{2.716194in}}%
\pgfpathlineto{\pgfqpoint{2.857556in}{2.709933in}}%
\pgfpathlineto{\pgfqpoint{2.856946in}{2.703673in}}%
\pgfpathlineto{\pgfqpoint{2.856520in}{2.697412in}}%
\pgfpathlineto{\pgfqpoint{2.856290in}{2.691152in}}%
\pgfpathlineto{\pgfqpoint{2.856272in}{2.684891in}}%
\pgfpathlineto{\pgfqpoint{2.856480in}{2.678631in}}%
\pgfpathlineto{\pgfqpoint{2.856934in}{2.672371in}}%
\pgfpathlineto{\pgfqpoint{2.857654in}{2.666110in}}%
\pgfpathlineto{\pgfqpoint{2.858664in}{2.659850in}}%
\pgfpathlineto{\pgfqpoint{2.859992in}{2.653589in}}%
\pgfpathlineto{\pgfqpoint{2.860184in}{2.652848in}}%
\pgfpathlineto{\pgfqpoint{2.861841in}{2.647329in}}%
\pgfpathlineto{\pgfqpoint{2.864160in}{2.641068in}}%
\pgfpathlineto{\pgfqpoint{2.866444in}{2.635951in}}%
\pgfpathlineto{\pgfqpoint{2.867058in}{2.634808in}}%
\pgfpathlineto{\pgfqpoint{2.870984in}{2.628548in}}%
\pgfpathlineto{\pgfqpoint{2.872705in}{2.626186in}}%
\pgfpathlineto{\pgfqpoint{2.876336in}{2.622287in}}%
\pgfpathlineto{\pgfqpoint{2.878965in}{2.619816in}}%
\pgfpathlineto{\pgfqpoint{2.884563in}{2.616027in}}%
\pgfpathclose%
\pgfusepath{fill}%
\end{pgfscope}%
\begin{pgfscope}%
\pgfsetrectcap%
\pgfsetmiterjoin%
\pgfsetlinewidth{1.003750pt}%
\definecolor{currentstroke}{rgb}{0.000000,0.000000,0.000000}%
\pgfsetstrokecolor{currentstroke}%
\pgfsetdash{}{0pt}%
\pgfpathmoveto{\pgfqpoint{0.500000in}{0.500000in}}%
\pgfpathlineto{\pgfqpoint{0.500000in}{4.250000in}}%
\pgfusepath{stroke}%
\end{pgfscope}%
\begin{pgfscope}%
\pgfsetrectcap%
\pgfsetmiterjoin%
\pgfsetlinewidth{1.003750pt}%
\definecolor{currentstroke}{rgb}{0.000000,0.000000,0.000000}%
\pgfsetstrokecolor{currentstroke}%
\pgfsetdash{}{0pt}%
\pgfpathmoveto{\pgfqpoint{0.500000in}{0.500000in}}%
\pgfpathlineto{\pgfqpoint{4.250000in}{0.500000in}}%
\pgfusepath{stroke}%
\end{pgfscope}%
\begin{pgfscope}%
\definecolor{textcolor}{rgb}{0.000000,0.000000,0.000000}%
\pgfsetstrokecolor{textcolor}%
\pgfsetfillcolor{textcolor}%
\pgftext[x=2.375000in,y=0.430556in,,top]{\color{textcolor}\sffamily\fontsize{20.000000}{24.000000}\selectfont \(\displaystyle \mu\)}%
\end{pgfscope}%
\begin{pgfscope}%
\definecolor{textcolor}{rgb}{0.000000,0.000000,0.000000}%
\pgfsetstrokecolor{textcolor}%
\pgfsetfillcolor{textcolor}%
\pgftext[x=0.430556in,y=2.375000in,,bottom,rotate=90.000000]{\color{textcolor}\sffamily\fontsize{20.000000}{24.000000}\selectfont \(\displaystyle \nu\)}%
\end{pgfscope}%
\end{pgfpicture}%
\makeatother%
\endgroup%
}
    \caption{\label{fig:Wasserstein transportation} Wasserstein transportation}
\end{figure}
\end{minipage}

Wasserstein distance can overcome these difficulty. Wasserstein distance between distribution $\mu$ and $\nu$ can be regarded as the minimal effort to transport mass from distribution $\mu$ to $\nu$, which is often described as earth move distance. Further the mass moves, more the effort we need to give. The transport plan between 2 different one-dimensional random variable distribution is described by $\gamma(x, y)$, which is a two-dimensional random variable distribution. $d(x,y)^{p}$ in the formula~\eqref{eq:w-dist-def} is the Borel measure on Polish spaces \cite{villani_2009}. There is a minimum function in the functional space of $\gamma(x, y)$, whose corresponding $W_{p}(\mu,\nu)$ is the Wasserstein distance between $\mu$ and $\nu$. 

The calculation of Wasserstein distance, however, is not always easy and may suffer from the deficiency of analytical expression. In practice, we set $p=1$, then the hittime in our waveform analysis results is discrete value (integers in a DAQ window), so the loss, Wasserstein distance (abbreviated as W-dist) can be calculated numerically. The corresponding charge of PE is each hittime can be regarded as weight in the distribution of hittime. Suppose $A$ and $B$ represent two sets of PE's hittime and charge. First the hittime $t$ are rounded to the nearest integer in the DAQ window. Then we subtract the cumulative distribution function (CDF) of A, $F_{A}$, and the CDF of B, $F_{B}$. Finally we sum the absolute value of the subtraction, which is formula \eqref{eq:numerical}. The whole process is equivalent to calculated the L1 distance of cumulative distribution function of distribution $A$ \& $B$. It is proved that the distance we define here is equivalent to Wasserstein distance with minor loss of accuracy (caused by rounding). 

\begin{equation}
    W_{d} = \sum_t|F_{A}(t) - F_{B}(t)|
    \label{eq:numerical}
\end{equation}