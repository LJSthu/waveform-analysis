Research of events in neutrino and dark matter detectors based on PMTs rely on PMT waveform analysis, especially high energy events. Commonly, PhotoMultiplier Tubes (PMT) waveform analysis method records total charge and first hittime of photoelectrons (PE) which results in lost of full information. A series of new method of waveform analysis were developed and tested. We use Wasserstein distance as the figure of merit to measure the difference between simulation truth and reconstruction results. For the DAQ windowsize of 1029ns in Jinping Neutrino Experiment 1ton prototype. The Wasserstein distance for the best method --- CNN --- is $0.246^{+0.175}_{-0.198}\mathrm{ns}$. 