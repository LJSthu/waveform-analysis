\section{Introduction}
\label{sec:introduction}

Waveform analysis of photomultiplier tubes~(PMT) is ubiquitous in fluid-based neutrino and dark matter experiments.  Accurate hit time and number of photoelectrons~(PE) provide detailed measurement of a detector's optical response, therefore enhance event reconstruction, particle identification, definition of fiducial volume, and consequently promote the physics prospect.

In Super-Kamiokande~\cite{noauthor_super-kamiokande_2003}, Borexino~(early phase)~\cite{lagomarsino_gateless_1999} and SNO~\cite{dunger_event_2019} the time/charge to digital converters~(TDC/QDC) recorded the threshold crossing times and integrated charges of PMT waveforms.  Experimentalists deployed fast analog to digital converters to record full PMT waveforms in KamLAND~\cite{the_kamland_collaboration_production_2010}, Borexino~(upgraded)[citation], Daya Bay~\cite{huang_flash_2018}, XMASS~\cite{abe_xmass_2013}, XENON1T~\cite{aprile_xenon1t_2019}, the future JUNO~\cite{an_neutrino_2016}, etc.  This technology advancement opened the flexibility of offline waveform analysis after data acquisition.  Nevertheless, limited by data volume and computational efficiency, early adopters emulated TDC/QDC in software with thresholding and integration algorithms.  Only recently people explored methods to extract charge and hit time of each PE~\cite{zhang_comparison_2019}.

This work focus on precise time and charge resolution. We simulate waveforms by toy Monte Carlo (toy MC) to assess waveform analysis.  In section~\ref{sec:toyMC}, we discuss the principles of PE measurement in PMT-based detectors.  In section~\ref{sec:algorithm}, we introduce waveform analysis algorithms and characterize their time performance.  Finally, in section~\ref{sec:discussion}, we discuss the impact on event reconstruction by a comparison of time and charge resolutions.
