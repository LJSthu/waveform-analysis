\section{Introduction} % (fold)
\label{sec:introduction}

Photomultiplier tube~(PMT) are sensitive to a single photon and widely utilized in fluid-based neutrino and dark matter experiments to measure Cherenkov or scintillation lights initiated by charged particles.

Traditionally, waveform within a period of time is integrated when the voltage exceeds a certain threshold to obtain charges, such as Borexino \cite{lagomarsino_gateless_1999}, SNO \cite{dunger_event_2019} and XMASS \cite{abe_xmass_2013}. The first hittime of each PMT is recorded. Other experiment explored some simple programs including Fourier deconvolution to extract charge and hittime for each photoelectron, such as Daya Bay \cite{huang_flash_2018}, KamLAND \cite{the_kamland_collaboration_production_2010}, JUNO \cite{zhang_comparison_2019} and XENON1T \cite{aprile_xenon1t_2019}. The efforts exerted on waveform analysis in these experiments are mostly limited by the amount of stored data and computational efficiency. 

Extracting the timing and number of hit photons will provide detailed information of the light transmit process in the detector and more explicit model parameters in the time profile. These precise analysis results will finally enhance the performance of event reconstruction, particle identification, and definition of fiducial volume. Furthermore, the precise description of the events in detectors will promote the physical prospect of neutrino experiments such as Jinping \cite{beacom_physics_2017}, JUNO \cite{an_neutrino_2016} and SuperK \cite{noauthor_super-kamiokande_2003}. Timing resolution is determined by PMT and waveform analysis. Therefore PMT waveform analysis is necessary when we pursue complete usage of PMT output and the improvement of timing resolution. 

This work focus on waveform analysis and precise timing resolution. A toy Monte Carlo (toy MC) has been developed to generate simulated waveform data. Several analysis methods have been tested to demonstrate the timing resolution improvement. In Section \ref{sec:toyMC}, we discuss the basic physical process in liquid detectors including PMT response and the configuration of the toy MC. In Section \ref{sec:algorithm}, we present a survey of waveform analysis methods as well as their performance. In Section \ref{sec:discussion}, we compare these algorithms and discuss their extensibility. 

% section Introduction (end)
