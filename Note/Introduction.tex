\section{Introduction}
\label{sec:introduction}

Photomultiplier tube~(PMT) is sensitive to a single photon and widely utilized in fluid-based neutrino and dark matter experiments to measure Cherenkov or scintillation lights initiated by charged particles.

Traditionally, in some experiments, such as Borexino~\cite{lagomarsino_gateless_1999}, SNO~\cite{dunger_event_2019}, and XMASS~\cite{abe_xmass_2013}, the electronic system only records the first hit time of photoelectron~(PE) and integrates the PMT waveform within a given time window to obtain charges. Other experiments, such as Daya Bay~\cite{huang_flash_2018}, KamLAND~\cite{the_kamland_collaboration_production_2010}, JUNO~\cite{zhang_comparison_2019}, and XENON1T~\cite{aprile_xenon1t_2019}, explored some simple programs, including Fourier deconvolution to extract charge and hit time for each PE. The efforts exerted on waveform analysis in these experiments are limited mainly by stored data and computational efficiency. 

Accurate hit time and number of PEs provide detailed information on the detector's light transmit process and explicit parameters in the time profile model. These precise analysis results finally enhance event reconstruction, particle identification, and definition of fiducial volume. Furthermore, precise description of the events in detectors promotes the physical prospect of neutrino experiments such as Jinping~\cite{beacom_physics_2017}, JUNO~\cite{an_neutrino_2016}, and SuperK~\cite{noauthor_super-kamiokande_2003}. Since waveform analysis influences time and charge resolution, it is necessary when pursuing unlimited PMT output usage. 

This work focus on precise time and charge resolution. We developed a toy Monte Carlo (toy MC) to simulate waveform data and tested several analysis methods to demonstrate their improvement. In section~\ref{sec:toyMC}, we discuss the basic physical process in liquid detectors, including PMT response and the configuration of the toy MC. In section~\ref{sec:algorithm}, we present a survey of waveform analysis methods as well as their performance. Finally, in section~\ref{sec:discussion}, we compare these algorithms and discuss their extensibility in the context of neutrino and dark matter experiments. 
