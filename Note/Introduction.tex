\section{Introduction} % (fold)
\label{sec:Introduction}
PMT is involved in many neutrino and dark matter experiments. Traditionally, waveform within a period of time is integrated when the voltage exceed a certain threshold to obtain charge, such as Borexino\cite{lagomarsino_gateless_1999}, SNO\cite{dunger_event_2019} and XMASS\cite{abe_xmass_2013}. Meanwhile the first hittime of each PMT is also recorded. Other experiment employed some simple programs including Fourier deconvolution to extract charge and hittime for each photoelectron, such as Daya Bay\cite{huang_flash_2018}, KamLAND\cite{the_kamland_collaboration_production_2010}, JUNO\cite{zhang_comparison_2019} and XENON1T\cite{aprile_xenon1t_2019}. The efforts exerted on waveform analysis in these experiments is limited by amount of stored data and computational efficiency. 

Extracting the information of time and number of photons hitting will provide more detailed information of light transmit model in detector and will finally enhance the performance of event reconstruction, even particle identification. Therefore PMT waveform analysis is necessary when we pursue completely usage of PMT output. 

\paragraph{Notations}
\begin{itemize}
    \item $t$ - Time of waveform
    \item $t_{H}$ - Hittime, time of PE hitting first dynode
    \item $v_{w}(t)$ - Pedestal reduced origin waveform voltage
    \item $v_{spe}(t)$ - Single-photoelectron (SPE) response
    \item $v_{th}$ - Voltage threshold
    \item $q_{r}(t_{H})$ - Reconstructed charge
    \item $n_{r}(t_{H})$ - Reconstructed number of PE
    \item $v_{r}(t)$ - Reconstructed waveform
    \item $Q$ - Total Charge in a DAQ window
    \item $N_{pe}$ - Total PE number in a DAQ window
    \item $N_{pos}$ - Total number of hittime in a DAQ window
    \item $W_{d}$ - Wasserstein distance
\end{itemize}
The $t$ and $t_{H}$ we defined here is discrete value in 1 DAQ window (from 0 to 1029ns, step size is 1ns in Jinping 1ton prototype DAQ system). 
% section Introduction (end)