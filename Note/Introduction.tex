\section{Introduction} % (fold)
\label{sec:introduction}

PMT is a device which is highly sensitive to even single photon which is involved in many neutrino and dark matter experiments based on liquid detectors. In neutrino experiments, liquid scintillator emits light after excited by candidate particles and charged particles emit Cherenkov light in liquid. 

Traditionally, waveform within a period of time is integrated when the voltage exceed a certain threshold to obtain charge, such as Borexino\cite{lagomarsino_gateless_1999}, SNO\cite{dunger_event_2019} and XMASS\cite{abe_xmass_2013}. Meanwhile the first hittime of each PMT is also recorded. Other experiment employed some simple programs including Fourier deconvolution to extract charge and hittime for each photoelectron, such as Daya Bay\cite{huang_flash_2018}, KamLAND\cite{the_kamland_collaboration_production_2010}, JUNO\cite{zhang_comparison_2019} and XENON1T\cite{aprile_xenon1t_2019}. The efforts exerted on waveform analysis in these experiments is limited by amount of stored data and computational efficiency. 

Extracting the time information and number of photons hitting will provide more detailed information of light transmit process in detector and more explicit model parameters in time profile and will finally enhance the performance of event reconstruction, even particle identification. Timing resolution is determined by PMT and waveform analysis. Therefore PMT waveform analysis is necessary when we pursue completely usage of PMT output. 

This work focus on waveform analysis and precise time resolution. A toy Monte Carlo (toy MC) has been developed to generate simulated waveform data. Several analysis methods has been tested to demonstrate the time resolution improvement. In Section \ref{sec:toyMC}, we discuss the basic physical process in liquid detector including PMT response and the configuration of the toy MC. In Section \ref{sec:algorithm}, we present a survey of waveform analysis methods as well as their performance. In Section \ref{sec:discussion}, we compare these algorithms and discuss their extensibility. 

% section Introduction (end)