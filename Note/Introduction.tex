\section{Introduction}
\label{sec:introduction}

Photomultiplier tube~(PMT) is sensitive to a single photon and widely utilized in fluid-based neutrino and dark matter experiments to measure Cherenkov or scintillation lights initiated by charged particles.

Traditionally, in some experiments, such as Borexino~\cite{lagomarsino_gateless_1999}, SNO~\cite{dunger_event_2019}, and XMASS~\cite{abe_xmass_2013}, the electronic system integrates the PMT waveform within a given time window when the voltage exceeds a certain threshold to obtain charges and only records the first hit time. Other experiments, such as Daya Bay~\cite{huang_flash_2018}, KamLAND~\cite{the_kamland_collaboration_production_2010}, JUNO~\cite{zhang_comparison_2019}, and XENON1T~\cite{aprile_xenon1t_2019}, explored some simple programs, including Fourier deconvolution to extract charge and hit time for each photoelectron. The efforts exerted on waveform analysis in these experiments are limited mainly by stored data and computational efficiency. 

Extracting the hit time and the number of photoelectrons will provide detailed information on the detector's light transmit process and more explicit model parameters in the time profile. These precise analysis results will finally enhance the performance of event reconstruction, particle identification, and definition of fiducial volume. Furthermore, the accurate description of the events in detectors may provide an opportunity to promote the physical prospect of neutrino experiments such as Jinping~\cite{beacom_physics_2017}, JUNO~\cite{an_neutrino_2016}, and SuperK~\cite{noauthor_super-kamiokande_2003}. Since PMT and waveform analysis determine time resolution, waveform analysis is necessary when pursuing unlimited PMT output usage and improving timing resolution. 

This work focus on precise timing and charge resolution. We developed a toy Monte Carlo (toy MC) to generate simulated waveform data and tested it with several analysis methods to demonstrate the improvement of timing and charge resolution. Section~\ref{sec:toyMC} will discuss the basic physical process in liquid detectors, including PMT response and the configuration of the toy MC. Section~\ref{sec:algorithm} will present a survey of waveform analysis methods as well as their performance. Finally, Section~\ref{sec:discussion} will compare these algorithms and discuss their extensibility in the context of neutrino and dark matter experiments. 
