\section{Waveform of PMT} % (fold)
PMT is a device which is highly sensitive to even single photon. Therefore, PMT is widely used in neutrino experiments based on liquid and dark matter experiment. In neutrino experiments, liquid scintillator emits light after excited by candidate particles and charged particles emit Cherenkov light in liquid. Timing resolution is crucial in neutrino event reconstruction. 

Timing resolution is determined by PMT and waveform analysis. Typical PMT response includes 3 individual processes: PE conversion happened on photocathode. Electron collection by the first dynode. And amplification of electrons between dynodes. So 1 photon incoming has a certain probability to be observed via PMT voltage (See figure~\ref{fig:spe}). But if photons hit the PMT continually, the PE response will pile-up (see figure~\ref{fig:pile}) and the waveform analysis will be difficult. Pile-up will significantly worsen timing resolution. 

\begin{figure}[H]
    \begin{subfigure}{0.5\textwidth}
        \centering
        \scalebox{0.4}{version https://git-lfs.github.com/spec/v1
oid sha256:afb86d573dc3f30901897a7c61af20f9ad79e1649cd8a2f885bab885b4d96208
size 40038
}
        \caption{\label{fig:spe} SPE response of PMT}
    \end{subfigure}
    \begin{subfigure}{0.5\textwidth}
        \centering
        \scalebox{0.4}{%% Creator: Matplotlib, PGF backend
%%
%% To include the figure in your LaTeX document, write
%%   \input{<filename>.pgf}
%%
%% Make sure the required packages are loaded in your preamble
%%   \usepackage{pgf}
%%
%% and, on pdftex
%%   \usepackage[utf8]{inputenc}\DeclareUnicodeCharacter{2212}{-}
%%
%% or, on luatex and xetex
%%   \usepackage{unicode-math}
%%
%% Figures using additional raster images can only be included by \input if
%% they are in the same directory as the main LaTeX file. For loading figures
%% from other directories you can use the `import` package
%%   \usepackage{import}
%%
%% and then include the figures with
%%   \import{<path to file>}{<filename>.pgf}
%%
%% Matplotlib used the following preamble
%%   \usepackage[detect-all,locale=DE]{siunitx}
%%
\begingroup%
\makeatletter%
\begin{pgfpicture}%
\pgfpathrectangle{\pgfpointorigin}{\pgfqpoint{8.000000in}{6.000000in}}%
\pgfusepath{use as bounding box, clip}%
\begin{pgfscope}%
\pgfsetbuttcap%
\pgfsetmiterjoin%
\definecolor{currentfill}{rgb}{1.000000,1.000000,1.000000}%
\pgfsetfillcolor{currentfill}%
\pgfsetlinewidth{0.000000pt}%
\definecolor{currentstroke}{rgb}{1.000000,1.000000,1.000000}%
\pgfsetstrokecolor{currentstroke}%
\pgfsetdash{}{0pt}%
\pgfpathmoveto{\pgfqpoint{0.000000in}{0.000000in}}%
\pgfpathlineto{\pgfqpoint{8.000000in}{0.000000in}}%
\pgfpathlineto{\pgfqpoint{8.000000in}{6.000000in}}%
\pgfpathlineto{\pgfqpoint{0.000000in}{6.000000in}}%
\pgfpathclose%
\pgfusepath{fill}%
\end{pgfscope}%
\begin{pgfscope}%
\pgfsetbuttcap%
\pgfsetmiterjoin%
\definecolor{currentfill}{rgb}{1.000000,1.000000,1.000000}%
\pgfsetfillcolor{currentfill}%
\pgfsetlinewidth{0.000000pt}%
\definecolor{currentstroke}{rgb}{0.000000,0.000000,0.000000}%
\pgfsetstrokecolor{currentstroke}%
\pgfsetstrokeopacity{0.000000}%
\pgfsetdash{}{0pt}%
\pgfpathmoveto{\pgfqpoint{1.200000in}{0.900000in}}%
\pgfpathlineto{\pgfqpoint{6.800000in}{0.900000in}}%
\pgfpathlineto{\pgfqpoint{6.800000in}{5.700000in}}%
\pgfpathlineto{\pgfqpoint{1.200000in}{5.700000in}}%
\pgfpathclose%
\pgfusepath{fill}%
\end{pgfscope}%
\begin{pgfscope}%
\pgfpathrectangle{\pgfqpoint{1.200000in}{0.900000in}}{\pgfqpoint{5.600000in}{4.800000in}}%
\pgfusepath{clip}%
\pgfsetrectcap%
\pgfsetroundjoin%
\pgfsetlinewidth{2.007500pt}%
\definecolor{currentstroke}{rgb}{0.000000,0.000000,1.000000}%
\pgfsetstrokecolor{currentstroke}%
\pgfsetdash{}{0pt}%
\pgfpathmoveto{\pgfqpoint{1.200000in}{1.317391in}}%
\pgfpathlineto{\pgfqpoint{1.210884in}{1.317391in}}%
\pgfpathlineto{\pgfqpoint{1.216327in}{1.233913in}}%
\pgfpathlineto{\pgfqpoint{1.221769in}{1.400870in}}%
\pgfpathlineto{\pgfqpoint{1.232653in}{1.400870in}}%
\pgfpathlineto{\pgfqpoint{1.238095in}{1.317391in}}%
\pgfpathlineto{\pgfqpoint{1.243537in}{1.317391in}}%
\pgfpathlineto{\pgfqpoint{1.248980in}{1.150435in}}%
\pgfpathlineto{\pgfqpoint{1.254422in}{1.150435in}}%
\pgfpathlineto{\pgfqpoint{1.265306in}{1.317391in}}%
\pgfpathlineto{\pgfqpoint{1.287075in}{1.317391in}}%
\pgfpathlineto{\pgfqpoint{1.297959in}{1.484348in}}%
\pgfpathlineto{\pgfqpoint{1.303401in}{1.233913in}}%
\pgfpathlineto{\pgfqpoint{1.308844in}{1.400870in}}%
\pgfpathlineto{\pgfqpoint{1.314286in}{1.233913in}}%
\pgfpathlineto{\pgfqpoint{1.319728in}{1.317391in}}%
\pgfpathlineto{\pgfqpoint{1.325170in}{1.317391in}}%
\pgfpathlineto{\pgfqpoint{1.330612in}{1.233913in}}%
\pgfpathlineto{\pgfqpoint{1.336054in}{1.233913in}}%
\pgfpathlineto{\pgfqpoint{1.341497in}{1.317391in}}%
\pgfpathlineto{\pgfqpoint{1.352381in}{1.317391in}}%
\pgfpathlineto{\pgfqpoint{1.357823in}{1.150435in}}%
\pgfpathlineto{\pgfqpoint{1.363265in}{1.150435in}}%
\pgfpathlineto{\pgfqpoint{1.368707in}{1.400870in}}%
\pgfpathlineto{\pgfqpoint{1.379592in}{1.400870in}}%
\pgfpathlineto{\pgfqpoint{1.385034in}{1.233913in}}%
\pgfpathlineto{\pgfqpoint{1.390476in}{1.317391in}}%
\pgfpathlineto{\pgfqpoint{1.395918in}{1.317391in}}%
\pgfpathlineto{\pgfqpoint{1.401361in}{1.400870in}}%
\pgfpathlineto{\pgfqpoint{1.406803in}{1.317391in}}%
\pgfpathlineto{\pgfqpoint{1.412245in}{1.150435in}}%
\pgfpathlineto{\pgfqpoint{1.417687in}{1.400870in}}%
\pgfpathlineto{\pgfqpoint{1.423129in}{1.233913in}}%
\pgfpathlineto{\pgfqpoint{1.428571in}{1.400870in}}%
\pgfpathlineto{\pgfqpoint{1.450340in}{1.400870in}}%
\pgfpathlineto{\pgfqpoint{1.455782in}{1.317391in}}%
\pgfpathlineto{\pgfqpoint{1.461224in}{1.150435in}}%
\pgfpathlineto{\pgfqpoint{1.466667in}{1.484348in}}%
\pgfpathlineto{\pgfqpoint{1.472109in}{1.317391in}}%
\pgfpathlineto{\pgfqpoint{1.499320in}{1.317391in}}%
\pgfpathlineto{\pgfqpoint{1.504762in}{1.400870in}}%
\pgfpathlineto{\pgfqpoint{1.515646in}{1.233913in}}%
\pgfpathlineto{\pgfqpoint{1.521088in}{1.233913in}}%
\pgfpathlineto{\pgfqpoint{1.526531in}{1.400870in}}%
\pgfpathlineto{\pgfqpoint{1.531973in}{1.400870in}}%
\pgfpathlineto{\pgfqpoint{1.537415in}{1.233913in}}%
\pgfpathlineto{\pgfqpoint{1.542857in}{1.233913in}}%
\pgfpathlineto{\pgfqpoint{1.553741in}{1.400870in}}%
\pgfpathlineto{\pgfqpoint{1.564626in}{1.233913in}}%
\pgfpathlineto{\pgfqpoint{1.570068in}{1.233913in}}%
\pgfpathlineto{\pgfqpoint{1.575510in}{1.317391in}}%
\pgfpathlineto{\pgfqpoint{1.580952in}{1.317391in}}%
\pgfpathlineto{\pgfqpoint{1.586395in}{1.233913in}}%
\pgfpathlineto{\pgfqpoint{1.591837in}{1.233913in}}%
\pgfpathlineto{\pgfqpoint{1.597279in}{1.317391in}}%
\pgfpathlineto{\pgfqpoint{1.602721in}{1.317391in}}%
\pgfpathlineto{\pgfqpoint{1.608163in}{1.233913in}}%
\pgfpathlineto{\pgfqpoint{1.613605in}{1.317391in}}%
\pgfpathlineto{\pgfqpoint{1.619048in}{1.317391in}}%
\pgfpathlineto{\pgfqpoint{1.624490in}{1.400870in}}%
\pgfpathlineto{\pgfqpoint{1.629932in}{1.317391in}}%
\pgfpathlineto{\pgfqpoint{1.635374in}{1.317391in}}%
\pgfpathlineto{\pgfqpoint{1.640816in}{1.400870in}}%
\pgfpathlineto{\pgfqpoint{1.646259in}{1.233913in}}%
\pgfpathlineto{\pgfqpoint{1.657143in}{1.233913in}}%
\pgfpathlineto{\pgfqpoint{1.662585in}{1.317391in}}%
\pgfpathlineto{\pgfqpoint{1.668027in}{1.233913in}}%
\pgfpathlineto{\pgfqpoint{1.673469in}{1.233913in}}%
\pgfpathlineto{\pgfqpoint{1.678912in}{1.317391in}}%
\pgfpathlineto{\pgfqpoint{1.684354in}{1.317391in}}%
\pgfpathlineto{\pgfqpoint{1.689796in}{1.400870in}}%
\pgfpathlineto{\pgfqpoint{1.695238in}{1.317391in}}%
\pgfpathlineto{\pgfqpoint{1.700680in}{1.400870in}}%
\pgfpathlineto{\pgfqpoint{1.711565in}{1.400870in}}%
\pgfpathlineto{\pgfqpoint{1.717007in}{1.317391in}}%
\pgfpathlineto{\pgfqpoint{1.722449in}{1.400870in}}%
\pgfpathlineto{\pgfqpoint{1.727891in}{1.233913in}}%
\pgfpathlineto{\pgfqpoint{1.733333in}{1.233913in}}%
\pgfpathlineto{\pgfqpoint{1.738776in}{1.400870in}}%
\pgfpathlineto{\pgfqpoint{1.744218in}{1.317391in}}%
\pgfpathlineto{\pgfqpoint{1.749660in}{1.400870in}}%
\pgfpathlineto{\pgfqpoint{1.755102in}{1.317391in}}%
\pgfpathlineto{\pgfqpoint{1.760544in}{1.317391in}}%
\pgfpathlineto{\pgfqpoint{1.765986in}{1.400870in}}%
\pgfpathlineto{\pgfqpoint{1.771429in}{1.400870in}}%
\pgfpathlineto{\pgfqpoint{1.776871in}{1.233913in}}%
\pgfpathlineto{\pgfqpoint{1.782313in}{1.400870in}}%
\pgfpathlineto{\pgfqpoint{1.787755in}{1.400870in}}%
\pgfpathlineto{\pgfqpoint{1.793197in}{1.317391in}}%
\pgfpathlineto{\pgfqpoint{1.798639in}{1.484348in}}%
\pgfpathlineto{\pgfqpoint{1.804082in}{1.484348in}}%
\pgfpathlineto{\pgfqpoint{1.809524in}{1.317391in}}%
\pgfpathlineto{\pgfqpoint{1.814966in}{1.233913in}}%
\pgfpathlineto{\pgfqpoint{1.820408in}{1.317391in}}%
\pgfpathlineto{\pgfqpoint{1.825850in}{1.484348in}}%
\pgfpathlineto{\pgfqpoint{1.831293in}{1.317391in}}%
\pgfpathlineto{\pgfqpoint{1.836735in}{1.400870in}}%
\pgfpathlineto{\pgfqpoint{1.847619in}{1.233913in}}%
\pgfpathlineto{\pgfqpoint{1.858503in}{1.400870in}}%
\pgfpathlineto{\pgfqpoint{1.863946in}{1.150435in}}%
\pgfpathlineto{\pgfqpoint{1.869388in}{1.317391in}}%
\pgfpathlineto{\pgfqpoint{1.874830in}{1.233913in}}%
\pgfpathlineto{\pgfqpoint{1.891156in}{1.233913in}}%
\pgfpathlineto{\pgfqpoint{1.896599in}{1.317391in}}%
\pgfpathlineto{\pgfqpoint{1.902041in}{1.233913in}}%
\pgfpathlineto{\pgfqpoint{1.907483in}{1.484348in}}%
\pgfpathlineto{\pgfqpoint{1.912925in}{1.317391in}}%
\pgfpathlineto{\pgfqpoint{1.918367in}{1.317391in}}%
\pgfpathlineto{\pgfqpoint{1.923810in}{1.484348in}}%
\pgfpathlineto{\pgfqpoint{1.929252in}{1.317391in}}%
\pgfpathlineto{\pgfqpoint{1.934694in}{1.317391in}}%
\pgfpathlineto{\pgfqpoint{1.945578in}{1.150435in}}%
\pgfpathlineto{\pgfqpoint{1.951020in}{1.400870in}}%
\pgfpathlineto{\pgfqpoint{1.961905in}{1.233913in}}%
\pgfpathlineto{\pgfqpoint{1.967347in}{1.400870in}}%
\pgfpathlineto{\pgfqpoint{1.978231in}{1.233913in}}%
\pgfpathlineto{\pgfqpoint{1.989116in}{1.233913in}}%
\pgfpathlineto{\pgfqpoint{1.994558in}{1.150435in}}%
\pgfpathlineto{\pgfqpoint{2.000000in}{1.233913in}}%
\pgfpathlineto{\pgfqpoint{2.005442in}{1.150435in}}%
\pgfpathlineto{\pgfqpoint{2.010884in}{1.317391in}}%
\pgfpathlineto{\pgfqpoint{2.027211in}{1.317391in}}%
\pgfpathlineto{\pgfqpoint{2.032653in}{1.233913in}}%
\pgfpathlineto{\pgfqpoint{2.038095in}{1.233913in}}%
\pgfpathlineto{\pgfqpoint{2.043537in}{1.400870in}}%
\pgfpathlineto{\pgfqpoint{2.048980in}{1.317391in}}%
\pgfpathlineto{\pgfqpoint{2.054422in}{1.400870in}}%
\pgfpathlineto{\pgfqpoint{2.059864in}{1.233913in}}%
\pgfpathlineto{\pgfqpoint{2.065306in}{1.233913in}}%
\pgfpathlineto{\pgfqpoint{2.070748in}{1.317391in}}%
\pgfpathlineto{\pgfqpoint{2.076190in}{1.317391in}}%
\pgfpathlineto{\pgfqpoint{2.081633in}{1.400870in}}%
\pgfpathlineto{\pgfqpoint{2.087075in}{1.233913in}}%
\pgfpathlineto{\pgfqpoint{2.092517in}{1.484348in}}%
\pgfpathlineto{\pgfqpoint{2.103401in}{1.317391in}}%
\pgfpathlineto{\pgfqpoint{2.108844in}{1.400870in}}%
\pgfpathlineto{\pgfqpoint{2.119728in}{1.233913in}}%
\pgfpathlineto{\pgfqpoint{2.125170in}{1.400870in}}%
\pgfpathlineto{\pgfqpoint{2.130612in}{1.400870in}}%
\pgfpathlineto{\pgfqpoint{2.141497in}{1.233913in}}%
\pgfpathlineto{\pgfqpoint{2.146939in}{1.317391in}}%
\pgfpathlineto{\pgfqpoint{2.152381in}{1.233913in}}%
\pgfpathlineto{\pgfqpoint{2.157823in}{1.317391in}}%
\pgfpathlineto{\pgfqpoint{2.163265in}{1.317391in}}%
\pgfpathlineto{\pgfqpoint{2.168707in}{1.233913in}}%
\pgfpathlineto{\pgfqpoint{2.174150in}{1.317391in}}%
\pgfpathlineto{\pgfqpoint{2.179592in}{1.233913in}}%
\pgfpathlineto{\pgfqpoint{2.185034in}{1.400870in}}%
\pgfpathlineto{\pgfqpoint{2.190476in}{1.400870in}}%
\pgfpathlineto{\pgfqpoint{2.195918in}{1.317391in}}%
\pgfpathlineto{\pgfqpoint{2.206803in}{1.317391in}}%
\pgfpathlineto{\pgfqpoint{2.217687in}{1.484348in}}%
\pgfpathlineto{\pgfqpoint{2.223129in}{1.317391in}}%
\pgfpathlineto{\pgfqpoint{2.228571in}{1.233913in}}%
\pgfpathlineto{\pgfqpoint{2.234014in}{1.400870in}}%
\pgfpathlineto{\pgfqpoint{2.239456in}{1.233913in}}%
\pgfpathlineto{\pgfqpoint{2.244898in}{1.317391in}}%
\pgfpathlineto{\pgfqpoint{2.250340in}{1.317391in}}%
\pgfpathlineto{\pgfqpoint{2.255782in}{1.233913in}}%
\pgfpathlineto{\pgfqpoint{2.261224in}{1.317391in}}%
\pgfpathlineto{\pgfqpoint{2.266667in}{1.233913in}}%
\pgfpathlineto{\pgfqpoint{2.272109in}{1.233913in}}%
\pgfpathlineto{\pgfqpoint{2.282993in}{1.400870in}}%
\pgfpathlineto{\pgfqpoint{2.288435in}{1.317391in}}%
\pgfpathlineto{\pgfqpoint{2.293878in}{1.317391in}}%
\pgfpathlineto{\pgfqpoint{2.299320in}{1.400870in}}%
\pgfpathlineto{\pgfqpoint{2.304762in}{1.233913in}}%
\pgfpathlineto{\pgfqpoint{2.310204in}{1.233913in}}%
\pgfpathlineto{\pgfqpoint{2.315646in}{1.317391in}}%
\pgfpathlineto{\pgfqpoint{2.321088in}{1.317391in}}%
\pgfpathlineto{\pgfqpoint{2.326531in}{1.400870in}}%
\pgfpathlineto{\pgfqpoint{2.337415in}{1.233913in}}%
\pgfpathlineto{\pgfqpoint{2.342857in}{1.400870in}}%
\pgfpathlineto{\pgfqpoint{2.348299in}{1.150435in}}%
\pgfpathlineto{\pgfqpoint{2.353741in}{1.317391in}}%
\pgfpathlineto{\pgfqpoint{2.359184in}{1.400870in}}%
\pgfpathlineto{\pgfqpoint{2.364626in}{1.317391in}}%
\pgfpathlineto{\pgfqpoint{2.370068in}{1.484348in}}%
\pgfpathlineto{\pgfqpoint{2.375510in}{1.317391in}}%
\pgfpathlineto{\pgfqpoint{2.380952in}{1.233913in}}%
\pgfpathlineto{\pgfqpoint{2.386395in}{1.317391in}}%
\pgfpathlineto{\pgfqpoint{2.391837in}{1.317391in}}%
\pgfpathlineto{\pgfqpoint{2.397279in}{1.400870in}}%
\pgfpathlineto{\pgfqpoint{2.402721in}{1.317391in}}%
\pgfpathlineto{\pgfqpoint{2.408163in}{1.317391in}}%
\pgfpathlineto{\pgfqpoint{2.413605in}{1.400870in}}%
\pgfpathlineto{\pgfqpoint{2.419048in}{1.400870in}}%
\pgfpathlineto{\pgfqpoint{2.429932in}{1.233913in}}%
\pgfpathlineto{\pgfqpoint{2.435374in}{1.317391in}}%
\pgfpathlineto{\pgfqpoint{2.440816in}{1.317391in}}%
\pgfpathlineto{\pgfqpoint{2.446259in}{1.400870in}}%
\pgfpathlineto{\pgfqpoint{2.451701in}{1.317391in}}%
\pgfpathlineto{\pgfqpoint{2.457143in}{1.066957in}}%
\pgfpathlineto{\pgfqpoint{2.462585in}{1.317391in}}%
\pgfpathlineto{\pgfqpoint{2.468027in}{1.400870in}}%
\pgfpathlineto{\pgfqpoint{2.473469in}{1.317391in}}%
\pgfpathlineto{\pgfqpoint{2.478912in}{1.400870in}}%
\pgfpathlineto{\pgfqpoint{2.484354in}{1.400870in}}%
\pgfpathlineto{\pgfqpoint{2.489796in}{1.233913in}}%
\pgfpathlineto{\pgfqpoint{2.495238in}{1.150435in}}%
\pgfpathlineto{\pgfqpoint{2.511565in}{1.400870in}}%
\pgfpathlineto{\pgfqpoint{2.522449in}{1.400870in}}%
\pgfpathlineto{\pgfqpoint{2.527891in}{1.233913in}}%
\pgfpathlineto{\pgfqpoint{2.533333in}{1.484348in}}%
\pgfpathlineto{\pgfqpoint{2.538776in}{1.484348in}}%
\pgfpathlineto{\pgfqpoint{2.549660in}{1.317391in}}%
\pgfpathlineto{\pgfqpoint{2.555102in}{1.400870in}}%
\pgfpathlineto{\pgfqpoint{2.560544in}{1.233913in}}%
\pgfpathlineto{\pgfqpoint{2.565986in}{1.317391in}}%
\pgfpathlineto{\pgfqpoint{2.571429in}{1.233913in}}%
\pgfpathlineto{\pgfqpoint{2.576871in}{1.233913in}}%
\pgfpathlineto{\pgfqpoint{2.582313in}{1.150435in}}%
\pgfpathlineto{\pgfqpoint{2.593197in}{1.317391in}}%
\pgfpathlineto{\pgfqpoint{2.604082in}{1.317391in}}%
\pgfpathlineto{\pgfqpoint{2.609524in}{1.400870in}}%
\pgfpathlineto{\pgfqpoint{2.614966in}{1.400870in}}%
\pgfpathlineto{\pgfqpoint{2.620408in}{1.317391in}}%
\pgfpathlineto{\pgfqpoint{2.625850in}{1.150435in}}%
\pgfpathlineto{\pgfqpoint{2.631293in}{1.066957in}}%
\pgfpathlineto{\pgfqpoint{2.636735in}{1.317391in}}%
\pgfpathlineto{\pgfqpoint{2.647619in}{1.317391in}}%
\pgfpathlineto{\pgfqpoint{2.653061in}{1.400870in}}%
\pgfpathlineto{\pgfqpoint{2.658503in}{1.317391in}}%
\pgfpathlineto{\pgfqpoint{2.669388in}{1.484348in}}%
\pgfpathlineto{\pgfqpoint{2.674830in}{1.400870in}}%
\pgfpathlineto{\pgfqpoint{2.680272in}{1.233913in}}%
\pgfpathlineto{\pgfqpoint{2.685714in}{1.317391in}}%
\pgfpathlineto{\pgfqpoint{2.691156in}{1.233913in}}%
\pgfpathlineto{\pgfqpoint{2.696599in}{1.400870in}}%
\pgfpathlineto{\pgfqpoint{2.702041in}{1.484348in}}%
\pgfpathlineto{\pgfqpoint{2.707483in}{1.317391in}}%
\pgfpathlineto{\pgfqpoint{2.712925in}{1.400870in}}%
\pgfpathlineto{\pgfqpoint{2.718367in}{1.233913in}}%
\pgfpathlineto{\pgfqpoint{2.729252in}{1.400870in}}%
\pgfpathlineto{\pgfqpoint{2.734694in}{1.317391in}}%
\pgfpathlineto{\pgfqpoint{2.740136in}{1.317391in}}%
\pgfpathlineto{\pgfqpoint{2.745578in}{1.400870in}}%
\pgfpathlineto{\pgfqpoint{2.751020in}{1.317391in}}%
\pgfpathlineto{\pgfqpoint{2.756463in}{1.317391in}}%
\pgfpathlineto{\pgfqpoint{2.761905in}{1.233913in}}%
\pgfpathlineto{\pgfqpoint{2.767347in}{1.233913in}}%
\pgfpathlineto{\pgfqpoint{2.772789in}{1.317391in}}%
\pgfpathlineto{\pgfqpoint{2.778231in}{1.233913in}}%
\pgfpathlineto{\pgfqpoint{2.783673in}{1.233913in}}%
\pgfpathlineto{\pgfqpoint{2.794558in}{1.400870in}}%
\pgfpathlineto{\pgfqpoint{2.805442in}{1.400870in}}%
\pgfpathlineto{\pgfqpoint{2.810884in}{1.150435in}}%
\pgfpathlineto{\pgfqpoint{2.821769in}{1.484348in}}%
\pgfpathlineto{\pgfqpoint{2.827211in}{2.068696in}}%
\pgfpathlineto{\pgfqpoint{2.832653in}{2.820000in}}%
\pgfpathlineto{\pgfqpoint{2.838095in}{3.821739in}}%
\pgfpathlineto{\pgfqpoint{2.843537in}{4.573043in}}%
\pgfpathlineto{\pgfqpoint{2.848980in}{5.073913in}}%
\pgfpathlineto{\pgfqpoint{2.854422in}{5.240870in}}%
\pgfpathlineto{\pgfqpoint{2.859864in}{5.157391in}}%
\pgfpathlineto{\pgfqpoint{2.876190in}{3.988696in}}%
\pgfpathlineto{\pgfqpoint{2.881633in}{3.654783in}}%
\pgfpathlineto{\pgfqpoint{2.887075in}{3.487826in}}%
\pgfpathlineto{\pgfqpoint{2.892517in}{3.070435in}}%
\pgfpathlineto{\pgfqpoint{2.897959in}{2.820000in}}%
\pgfpathlineto{\pgfqpoint{2.903401in}{2.986957in}}%
\pgfpathlineto{\pgfqpoint{2.914286in}{3.153913in}}%
\pgfpathlineto{\pgfqpoint{2.919728in}{3.153913in}}%
\pgfpathlineto{\pgfqpoint{2.925170in}{3.070435in}}%
\pgfpathlineto{\pgfqpoint{2.930612in}{2.903478in}}%
\pgfpathlineto{\pgfqpoint{2.936054in}{2.820000in}}%
\pgfpathlineto{\pgfqpoint{2.941497in}{2.569565in}}%
\pgfpathlineto{\pgfqpoint{2.946939in}{2.486087in}}%
\pgfpathlineto{\pgfqpoint{2.952381in}{2.235652in}}%
\pgfpathlineto{\pgfqpoint{2.963265in}{2.235652in}}%
\pgfpathlineto{\pgfqpoint{2.968707in}{2.736522in}}%
\pgfpathlineto{\pgfqpoint{2.974150in}{2.736522in}}%
\pgfpathlineto{\pgfqpoint{2.979592in}{2.903478in}}%
\pgfpathlineto{\pgfqpoint{2.985034in}{2.986957in}}%
\pgfpathlineto{\pgfqpoint{2.990476in}{2.820000in}}%
\pgfpathlineto{\pgfqpoint{2.995918in}{2.736522in}}%
\pgfpathlineto{\pgfqpoint{3.001361in}{2.569565in}}%
\pgfpathlineto{\pgfqpoint{3.006803in}{2.569565in}}%
\pgfpathlineto{\pgfqpoint{3.012245in}{2.319130in}}%
\pgfpathlineto{\pgfqpoint{3.017687in}{2.319130in}}%
\pgfpathlineto{\pgfqpoint{3.023129in}{1.985217in}}%
\pgfpathlineto{\pgfqpoint{3.028571in}{2.235652in}}%
\pgfpathlineto{\pgfqpoint{3.034014in}{2.319130in}}%
\pgfpathlineto{\pgfqpoint{3.039456in}{2.486087in}}%
\pgfpathlineto{\pgfqpoint{3.044898in}{2.569565in}}%
\pgfpathlineto{\pgfqpoint{3.050340in}{2.820000in}}%
\pgfpathlineto{\pgfqpoint{3.055782in}{2.736522in}}%
\pgfpathlineto{\pgfqpoint{3.066667in}{2.736522in}}%
\pgfpathlineto{\pgfqpoint{3.077551in}{2.402609in}}%
\pgfpathlineto{\pgfqpoint{3.082993in}{2.152174in}}%
\pgfpathlineto{\pgfqpoint{3.088435in}{2.235652in}}%
\pgfpathlineto{\pgfqpoint{3.093878in}{2.569565in}}%
\pgfpathlineto{\pgfqpoint{3.104762in}{2.903478in}}%
\pgfpathlineto{\pgfqpoint{3.110204in}{2.903478in}}%
\pgfpathlineto{\pgfqpoint{3.115646in}{2.986957in}}%
\pgfpathlineto{\pgfqpoint{3.121088in}{2.820000in}}%
\pgfpathlineto{\pgfqpoint{3.126531in}{2.736522in}}%
\pgfpathlineto{\pgfqpoint{3.131973in}{2.486087in}}%
\pgfpathlineto{\pgfqpoint{3.137415in}{2.569565in}}%
\pgfpathlineto{\pgfqpoint{3.142857in}{2.319130in}}%
\pgfpathlineto{\pgfqpoint{3.148299in}{2.152174in}}%
\pgfpathlineto{\pgfqpoint{3.159184in}{1.985217in}}%
\pgfpathlineto{\pgfqpoint{3.164626in}{1.818261in}}%
\pgfpathlineto{\pgfqpoint{3.170068in}{1.734783in}}%
\pgfpathlineto{\pgfqpoint{3.175510in}{2.152174in}}%
\pgfpathlineto{\pgfqpoint{3.180952in}{2.402609in}}%
\pgfpathlineto{\pgfqpoint{3.186395in}{2.319130in}}%
\pgfpathlineto{\pgfqpoint{3.191837in}{2.569565in}}%
\pgfpathlineto{\pgfqpoint{3.197279in}{2.653043in}}%
\pgfpathlineto{\pgfqpoint{3.202721in}{2.569565in}}%
\pgfpathlineto{\pgfqpoint{3.208163in}{2.235652in}}%
\pgfpathlineto{\pgfqpoint{3.213605in}{2.235652in}}%
\pgfpathlineto{\pgfqpoint{3.219048in}{2.068696in}}%
\pgfpathlineto{\pgfqpoint{3.224490in}{2.068696in}}%
\pgfpathlineto{\pgfqpoint{3.229932in}{1.901739in}}%
\pgfpathlineto{\pgfqpoint{3.235374in}{1.901739in}}%
\pgfpathlineto{\pgfqpoint{3.246259in}{1.567826in}}%
\pgfpathlineto{\pgfqpoint{3.251701in}{1.567826in}}%
\pgfpathlineto{\pgfqpoint{3.257143in}{1.734783in}}%
\pgfpathlineto{\pgfqpoint{3.262585in}{1.651304in}}%
\pgfpathlineto{\pgfqpoint{3.268027in}{1.484348in}}%
\pgfpathlineto{\pgfqpoint{3.273469in}{1.567826in}}%
\pgfpathlineto{\pgfqpoint{3.278912in}{1.400870in}}%
\pgfpathlineto{\pgfqpoint{3.284354in}{1.484348in}}%
\pgfpathlineto{\pgfqpoint{3.289796in}{1.400870in}}%
\pgfpathlineto{\pgfqpoint{3.295238in}{1.484348in}}%
\pgfpathlineto{\pgfqpoint{3.300680in}{1.317391in}}%
\pgfpathlineto{\pgfqpoint{3.311565in}{1.317391in}}%
\pgfpathlineto{\pgfqpoint{3.317007in}{1.150435in}}%
\pgfpathlineto{\pgfqpoint{3.322449in}{1.484348in}}%
\pgfpathlineto{\pgfqpoint{3.327891in}{1.317391in}}%
\pgfpathlineto{\pgfqpoint{3.333333in}{1.400870in}}%
\pgfpathlineto{\pgfqpoint{3.338776in}{1.317391in}}%
\pgfpathlineto{\pgfqpoint{3.344218in}{1.400870in}}%
\pgfpathlineto{\pgfqpoint{3.355102in}{1.400870in}}%
\pgfpathlineto{\pgfqpoint{3.360544in}{1.317391in}}%
\pgfpathlineto{\pgfqpoint{3.365986in}{1.484348in}}%
\pgfpathlineto{\pgfqpoint{3.371429in}{1.233913in}}%
\pgfpathlineto{\pgfqpoint{3.376871in}{1.317391in}}%
\pgfpathlineto{\pgfqpoint{3.382313in}{1.150435in}}%
\pgfpathlineto{\pgfqpoint{3.387755in}{1.484348in}}%
\pgfpathlineto{\pgfqpoint{3.393197in}{1.400870in}}%
\pgfpathlineto{\pgfqpoint{3.404082in}{1.400870in}}%
\pgfpathlineto{\pgfqpoint{3.409524in}{1.317391in}}%
\pgfpathlineto{\pgfqpoint{3.420408in}{1.317391in}}%
\pgfpathlineto{\pgfqpoint{3.425850in}{1.484348in}}%
\pgfpathlineto{\pgfqpoint{3.436735in}{1.484348in}}%
\pgfpathlineto{\pgfqpoint{3.442177in}{1.317391in}}%
\pgfpathlineto{\pgfqpoint{3.447619in}{1.400870in}}%
\pgfpathlineto{\pgfqpoint{3.458503in}{1.400870in}}%
\pgfpathlineto{\pgfqpoint{3.463946in}{1.317391in}}%
\pgfpathlineto{\pgfqpoint{3.469388in}{1.400870in}}%
\pgfpathlineto{\pgfqpoint{3.480272in}{1.400870in}}%
\pgfpathlineto{\pgfqpoint{3.485714in}{1.567826in}}%
\pgfpathlineto{\pgfqpoint{3.491156in}{1.317391in}}%
\pgfpathlineto{\pgfqpoint{3.496599in}{1.150435in}}%
\pgfpathlineto{\pgfqpoint{3.502041in}{1.400870in}}%
\pgfpathlineto{\pgfqpoint{3.507483in}{1.233913in}}%
\pgfpathlineto{\pgfqpoint{3.512925in}{1.233913in}}%
\pgfpathlineto{\pgfqpoint{3.518367in}{1.150435in}}%
\pgfpathlineto{\pgfqpoint{3.523810in}{1.150435in}}%
\pgfpathlineto{\pgfqpoint{3.529252in}{1.233913in}}%
\pgfpathlineto{\pgfqpoint{3.534694in}{1.400870in}}%
\pgfpathlineto{\pgfqpoint{3.540136in}{1.317391in}}%
\pgfpathlineto{\pgfqpoint{3.545578in}{1.484348in}}%
\pgfpathlineto{\pgfqpoint{3.551020in}{1.233913in}}%
\pgfpathlineto{\pgfqpoint{3.556463in}{1.233913in}}%
\pgfpathlineto{\pgfqpoint{3.567347in}{1.400870in}}%
\pgfpathlineto{\pgfqpoint{3.572789in}{1.400870in}}%
\pgfpathlineto{\pgfqpoint{3.578231in}{1.150435in}}%
\pgfpathlineto{\pgfqpoint{3.583673in}{1.233913in}}%
\pgfpathlineto{\pgfqpoint{3.589116in}{1.233913in}}%
\pgfpathlineto{\pgfqpoint{3.594558in}{1.400870in}}%
\pgfpathlineto{\pgfqpoint{3.600000in}{1.233913in}}%
\pgfpathlineto{\pgfqpoint{3.605442in}{1.233913in}}%
\pgfpathlineto{\pgfqpoint{3.610884in}{1.317391in}}%
\pgfpathlineto{\pgfqpoint{3.616327in}{1.484348in}}%
\pgfpathlineto{\pgfqpoint{3.621769in}{1.400870in}}%
\pgfpathlineto{\pgfqpoint{3.627211in}{1.400870in}}%
\pgfpathlineto{\pgfqpoint{3.638095in}{1.233913in}}%
\pgfpathlineto{\pgfqpoint{3.643537in}{1.317391in}}%
\pgfpathlineto{\pgfqpoint{3.654422in}{1.317391in}}%
\pgfpathlineto{\pgfqpoint{3.665306in}{1.484348in}}%
\pgfpathlineto{\pgfqpoint{3.670748in}{1.317391in}}%
\pgfpathlineto{\pgfqpoint{3.676190in}{1.233913in}}%
\pgfpathlineto{\pgfqpoint{3.681633in}{1.317391in}}%
\pgfpathlineto{\pgfqpoint{3.687075in}{1.233913in}}%
\pgfpathlineto{\pgfqpoint{3.692517in}{1.317391in}}%
\pgfpathlineto{\pgfqpoint{3.697959in}{1.484348in}}%
\pgfpathlineto{\pgfqpoint{3.703401in}{1.400870in}}%
\pgfpathlineto{\pgfqpoint{3.708844in}{1.400870in}}%
\pgfpathlineto{\pgfqpoint{3.714286in}{1.233913in}}%
\pgfpathlineto{\pgfqpoint{3.719728in}{1.484348in}}%
\pgfpathlineto{\pgfqpoint{3.730612in}{1.317391in}}%
\pgfpathlineto{\pgfqpoint{3.741497in}{1.317391in}}%
\pgfpathlineto{\pgfqpoint{3.746939in}{1.484348in}}%
\pgfpathlineto{\pgfqpoint{3.752381in}{1.317391in}}%
\pgfpathlineto{\pgfqpoint{3.757823in}{1.400870in}}%
\pgfpathlineto{\pgfqpoint{3.768707in}{1.400870in}}%
\pgfpathlineto{\pgfqpoint{3.774150in}{1.233913in}}%
\pgfpathlineto{\pgfqpoint{3.779592in}{1.317391in}}%
\pgfpathlineto{\pgfqpoint{3.790476in}{1.317391in}}%
\pgfpathlineto{\pgfqpoint{3.795918in}{1.400870in}}%
\pgfpathlineto{\pgfqpoint{3.801361in}{1.233913in}}%
\pgfpathlineto{\pgfqpoint{3.806803in}{1.400870in}}%
\pgfpathlineto{\pgfqpoint{3.812245in}{1.317391in}}%
\pgfpathlineto{\pgfqpoint{3.823129in}{1.317391in}}%
\pgfpathlineto{\pgfqpoint{3.828571in}{1.233913in}}%
\pgfpathlineto{\pgfqpoint{3.834014in}{1.400870in}}%
\pgfpathlineto{\pgfqpoint{3.839456in}{1.233913in}}%
\pgfpathlineto{\pgfqpoint{3.844898in}{1.150435in}}%
\pgfpathlineto{\pgfqpoint{3.850340in}{1.317391in}}%
\pgfpathlineto{\pgfqpoint{3.861224in}{1.317391in}}%
\pgfpathlineto{\pgfqpoint{3.866667in}{1.150435in}}%
\pgfpathlineto{\pgfqpoint{3.872109in}{1.400870in}}%
\pgfpathlineto{\pgfqpoint{3.877551in}{1.317391in}}%
\pgfpathlineto{\pgfqpoint{3.882993in}{1.150435in}}%
\pgfpathlineto{\pgfqpoint{3.888435in}{1.317391in}}%
\pgfpathlineto{\pgfqpoint{3.893878in}{1.400870in}}%
\pgfpathlineto{\pgfqpoint{3.904762in}{1.400870in}}%
\pgfpathlineto{\pgfqpoint{3.910204in}{1.317391in}}%
\pgfpathlineto{\pgfqpoint{3.915646in}{1.317391in}}%
\pgfpathlineto{\pgfqpoint{3.921088in}{1.400870in}}%
\pgfpathlineto{\pgfqpoint{3.931973in}{1.400870in}}%
\pgfpathlineto{\pgfqpoint{3.937415in}{1.233913in}}%
\pgfpathlineto{\pgfqpoint{3.942857in}{1.484348in}}%
\pgfpathlineto{\pgfqpoint{3.948299in}{1.150435in}}%
\pgfpathlineto{\pgfqpoint{3.953741in}{1.317391in}}%
\pgfpathlineto{\pgfqpoint{3.959184in}{1.400870in}}%
\pgfpathlineto{\pgfqpoint{3.964626in}{1.150435in}}%
\pgfpathlineto{\pgfqpoint{3.970068in}{1.233913in}}%
\pgfpathlineto{\pgfqpoint{3.975510in}{1.233913in}}%
\pgfpathlineto{\pgfqpoint{3.980952in}{1.400870in}}%
\pgfpathlineto{\pgfqpoint{3.986395in}{1.317391in}}%
\pgfpathlineto{\pgfqpoint{3.991837in}{1.400870in}}%
\pgfpathlineto{\pgfqpoint{3.997279in}{1.400870in}}%
\pgfpathlineto{\pgfqpoint{4.008163in}{1.233913in}}%
\pgfpathlineto{\pgfqpoint{4.013605in}{1.400870in}}%
\pgfpathlineto{\pgfqpoint{4.019048in}{1.317391in}}%
\pgfpathlineto{\pgfqpoint{4.024490in}{1.317391in}}%
\pgfpathlineto{\pgfqpoint{4.029932in}{1.484348in}}%
\pgfpathlineto{\pgfqpoint{4.035374in}{1.233913in}}%
\pgfpathlineto{\pgfqpoint{4.046259in}{1.400870in}}%
\pgfpathlineto{\pgfqpoint{4.051701in}{1.233913in}}%
\pgfpathlineto{\pgfqpoint{4.057143in}{1.317391in}}%
\pgfpathlineto{\pgfqpoint{4.062585in}{1.233913in}}%
\pgfpathlineto{\pgfqpoint{4.068027in}{1.484348in}}%
\pgfpathlineto{\pgfqpoint{4.073469in}{1.317391in}}%
\pgfpathlineto{\pgfqpoint{4.078912in}{1.233913in}}%
\pgfpathlineto{\pgfqpoint{4.089796in}{1.233913in}}%
\pgfpathlineto{\pgfqpoint{4.095238in}{1.400870in}}%
\pgfpathlineto{\pgfqpoint{4.100680in}{1.400870in}}%
\pgfpathlineto{\pgfqpoint{4.106122in}{1.233913in}}%
\pgfpathlineto{\pgfqpoint{4.111565in}{1.400870in}}%
\pgfpathlineto{\pgfqpoint{4.117007in}{1.233913in}}%
\pgfpathlineto{\pgfqpoint{4.122449in}{1.233913in}}%
\pgfpathlineto{\pgfqpoint{4.127891in}{1.400870in}}%
\pgfpathlineto{\pgfqpoint{4.138776in}{1.233913in}}%
\pgfpathlineto{\pgfqpoint{4.144218in}{1.233913in}}%
\pgfpathlineto{\pgfqpoint{4.149660in}{1.317391in}}%
\pgfpathlineto{\pgfqpoint{4.160544in}{1.317391in}}%
\pgfpathlineto{\pgfqpoint{4.165986in}{1.400870in}}%
\pgfpathlineto{\pgfqpoint{4.171429in}{1.066957in}}%
\pgfpathlineto{\pgfqpoint{4.176871in}{1.400870in}}%
\pgfpathlineto{\pgfqpoint{4.182313in}{1.233913in}}%
\pgfpathlineto{\pgfqpoint{4.187755in}{1.484348in}}%
\pgfpathlineto{\pgfqpoint{4.193197in}{1.317391in}}%
\pgfpathlineto{\pgfqpoint{4.198639in}{1.317391in}}%
\pgfpathlineto{\pgfqpoint{4.204082in}{1.150435in}}%
\pgfpathlineto{\pgfqpoint{4.209524in}{1.400870in}}%
\pgfpathlineto{\pgfqpoint{4.214966in}{1.317391in}}%
\pgfpathlineto{\pgfqpoint{4.220408in}{1.400870in}}%
\pgfpathlineto{\pgfqpoint{4.231293in}{1.400870in}}%
\pgfpathlineto{\pgfqpoint{4.236735in}{1.317391in}}%
\pgfpathlineto{\pgfqpoint{4.242177in}{1.317391in}}%
\pgfpathlineto{\pgfqpoint{4.253061in}{1.484348in}}%
\pgfpathlineto{\pgfqpoint{4.258503in}{1.400870in}}%
\pgfpathlineto{\pgfqpoint{4.263946in}{1.400870in}}%
\pgfpathlineto{\pgfqpoint{4.269388in}{1.150435in}}%
\pgfpathlineto{\pgfqpoint{4.280272in}{1.317391in}}%
\pgfpathlineto{\pgfqpoint{4.285714in}{1.233913in}}%
\pgfpathlineto{\pgfqpoint{4.291156in}{1.233913in}}%
\pgfpathlineto{\pgfqpoint{4.296599in}{1.400870in}}%
\pgfpathlineto{\pgfqpoint{4.302041in}{1.233913in}}%
\pgfpathlineto{\pgfqpoint{4.307483in}{1.233913in}}%
\pgfpathlineto{\pgfqpoint{4.312925in}{1.400870in}}%
\pgfpathlineto{\pgfqpoint{4.318367in}{1.233913in}}%
\pgfpathlineto{\pgfqpoint{4.323810in}{1.317391in}}%
\pgfpathlineto{\pgfqpoint{4.329252in}{1.317391in}}%
\pgfpathlineto{\pgfqpoint{4.334694in}{1.233913in}}%
\pgfpathlineto{\pgfqpoint{4.340136in}{1.233913in}}%
\pgfpathlineto{\pgfqpoint{4.351020in}{1.400870in}}%
\pgfpathlineto{\pgfqpoint{4.356463in}{1.400870in}}%
\pgfpathlineto{\pgfqpoint{4.361905in}{1.150435in}}%
\pgfpathlineto{\pgfqpoint{4.372789in}{1.150435in}}%
\pgfpathlineto{\pgfqpoint{4.378231in}{1.317391in}}%
\pgfpathlineto{\pgfqpoint{4.383673in}{1.317391in}}%
\pgfpathlineto{\pgfqpoint{4.394558in}{1.484348in}}%
\pgfpathlineto{\pgfqpoint{4.400000in}{1.484348in}}%
\pgfpathlineto{\pgfqpoint{4.405442in}{1.317391in}}%
\pgfpathlineto{\pgfqpoint{4.410884in}{1.317391in}}%
\pgfpathlineto{\pgfqpoint{4.421769in}{1.150435in}}%
\pgfpathlineto{\pgfqpoint{4.427211in}{1.233913in}}%
\pgfpathlineto{\pgfqpoint{4.432653in}{1.484348in}}%
\pgfpathlineto{\pgfqpoint{4.438095in}{1.233913in}}%
\pgfpathlineto{\pgfqpoint{4.443537in}{1.233913in}}%
\pgfpathlineto{\pgfqpoint{4.448980in}{1.317391in}}%
\pgfpathlineto{\pgfqpoint{4.459864in}{1.317391in}}%
\pgfpathlineto{\pgfqpoint{4.465306in}{1.233913in}}%
\pgfpathlineto{\pgfqpoint{4.470748in}{1.317391in}}%
\pgfpathlineto{\pgfqpoint{4.476190in}{1.317391in}}%
\pgfpathlineto{\pgfqpoint{4.481633in}{1.233913in}}%
\pgfpathlineto{\pgfqpoint{4.487075in}{1.317391in}}%
\pgfpathlineto{\pgfqpoint{4.492517in}{1.233913in}}%
\pgfpathlineto{\pgfqpoint{4.497959in}{1.317391in}}%
\pgfpathlineto{\pgfqpoint{4.503401in}{1.317391in}}%
\pgfpathlineto{\pgfqpoint{4.508844in}{1.400870in}}%
\pgfpathlineto{\pgfqpoint{4.514286in}{1.317391in}}%
\pgfpathlineto{\pgfqpoint{4.519728in}{1.317391in}}%
\pgfpathlineto{\pgfqpoint{4.525170in}{1.150435in}}%
\pgfpathlineto{\pgfqpoint{4.530612in}{1.400870in}}%
\pgfpathlineto{\pgfqpoint{4.536054in}{1.400870in}}%
\pgfpathlineto{\pgfqpoint{4.541497in}{1.317391in}}%
\pgfpathlineto{\pgfqpoint{4.546939in}{1.317391in}}%
\pgfpathlineto{\pgfqpoint{4.552381in}{1.233913in}}%
\pgfpathlineto{\pgfqpoint{4.557823in}{1.233913in}}%
\pgfpathlineto{\pgfqpoint{4.563265in}{1.317391in}}%
\pgfpathlineto{\pgfqpoint{4.568707in}{1.484348in}}%
\pgfpathlineto{\pgfqpoint{4.574150in}{1.400870in}}%
\pgfpathlineto{\pgfqpoint{4.579592in}{1.400870in}}%
\pgfpathlineto{\pgfqpoint{4.585034in}{1.233913in}}%
\pgfpathlineto{\pgfqpoint{4.590476in}{1.317391in}}%
\pgfpathlineto{\pgfqpoint{4.595918in}{1.317391in}}%
\pgfpathlineto{\pgfqpoint{4.601361in}{1.150435in}}%
\pgfpathlineto{\pgfqpoint{4.606803in}{1.400870in}}%
\pgfpathlineto{\pgfqpoint{4.612245in}{1.400870in}}%
\pgfpathlineto{\pgfqpoint{4.617687in}{1.150435in}}%
\pgfpathlineto{\pgfqpoint{4.623129in}{1.317391in}}%
\pgfpathlineto{\pgfqpoint{4.628571in}{1.233913in}}%
\pgfpathlineto{\pgfqpoint{4.634014in}{1.066957in}}%
\pgfpathlineto{\pgfqpoint{4.639456in}{1.567826in}}%
\pgfpathlineto{\pgfqpoint{4.650340in}{1.233913in}}%
\pgfpathlineto{\pgfqpoint{4.655782in}{1.317391in}}%
\pgfpathlineto{\pgfqpoint{4.661224in}{1.317391in}}%
\pgfpathlineto{\pgfqpoint{4.666667in}{1.400870in}}%
\pgfpathlineto{\pgfqpoint{4.672109in}{1.400870in}}%
\pgfpathlineto{\pgfqpoint{4.677551in}{1.484348in}}%
\pgfpathlineto{\pgfqpoint{4.682993in}{1.317391in}}%
\pgfpathlineto{\pgfqpoint{4.688435in}{1.484348in}}%
\pgfpathlineto{\pgfqpoint{4.693878in}{1.400870in}}%
\pgfpathlineto{\pgfqpoint{4.699320in}{1.233913in}}%
\pgfpathlineto{\pgfqpoint{4.704762in}{1.317391in}}%
\pgfpathlineto{\pgfqpoint{4.715646in}{1.317391in}}%
\pgfpathlineto{\pgfqpoint{4.721088in}{1.233913in}}%
\pgfpathlineto{\pgfqpoint{4.726531in}{1.400870in}}%
\pgfpathlineto{\pgfqpoint{4.731973in}{1.317391in}}%
\pgfpathlineto{\pgfqpoint{4.737415in}{1.317391in}}%
\pgfpathlineto{\pgfqpoint{4.742857in}{1.233913in}}%
\pgfpathlineto{\pgfqpoint{4.748299in}{1.233913in}}%
\pgfpathlineto{\pgfqpoint{4.753741in}{1.400870in}}%
\pgfpathlineto{\pgfqpoint{4.759184in}{1.317391in}}%
\pgfpathlineto{\pgfqpoint{4.775510in}{1.317391in}}%
\pgfpathlineto{\pgfqpoint{4.780952in}{1.400870in}}%
\pgfpathlineto{\pgfqpoint{4.786395in}{1.233913in}}%
\pgfpathlineto{\pgfqpoint{4.791837in}{1.400870in}}%
\pgfpathlineto{\pgfqpoint{4.797279in}{1.317391in}}%
\pgfpathlineto{\pgfqpoint{4.808163in}{1.317391in}}%
\pgfpathlineto{\pgfqpoint{4.813605in}{1.484348in}}%
\pgfpathlineto{\pgfqpoint{4.819048in}{1.317391in}}%
\pgfpathlineto{\pgfqpoint{4.824490in}{1.400870in}}%
\pgfpathlineto{\pgfqpoint{4.835374in}{1.400870in}}%
\pgfpathlineto{\pgfqpoint{4.840816in}{1.317391in}}%
\pgfpathlineto{\pgfqpoint{4.851701in}{1.317391in}}%
\pgfpathlineto{\pgfqpoint{4.857143in}{1.233913in}}%
\pgfpathlineto{\pgfqpoint{4.862585in}{1.317391in}}%
\pgfpathlineto{\pgfqpoint{4.868027in}{1.233913in}}%
\pgfpathlineto{\pgfqpoint{4.873469in}{1.400870in}}%
\pgfpathlineto{\pgfqpoint{4.878912in}{1.400870in}}%
\pgfpathlineto{\pgfqpoint{4.884354in}{1.484348in}}%
\pgfpathlineto{\pgfqpoint{4.889796in}{1.233913in}}%
\pgfpathlineto{\pgfqpoint{4.895238in}{1.400870in}}%
\pgfpathlineto{\pgfqpoint{4.906122in}{1.233913in}}%
\pgfpathlineto{\pgfqpoint{4.911565in}{1.400870in}}%
\pgfpathlineto{\pgfqpoint{4.917007in}{1.233913in}}%
\pgfpathlineto{\pgfqpoint{4.922449in}{1.233913in}}%
\pgfpathlineto{\pgfqpoint{4.927891in}{1.150435in}}%
\pgfpathlineto{\pgfqpoint{4.933333in}{1.400870in}}%
\pgfpathlineto{\pgfqpoint{4.938776in}{1.317391in}}%
\pgfpathlineto{\pgfqpoint{4.949660in}{1.317391in}}%
\pgfpathlineto{\pgfqpoint{4.955102in}{1.567826in}}%
\pgfpathlineto{\pgfqpoint{4.960544in}{1.317391in}}%
\pgfpathlineto{\pgfqpoint{4.965986in}{1.233913in}}%
\pgfpathlineto{\pgfqpoint{4.971429in}{1.484348in}}%
\pgfpathlineto{\pgfqpoint{4.976871in}{1.317391in}}%
\pgfpathlineto{\pgfqpoint{4.982313in}{1.233913in}}%
\pgfpathlineto{\pgfqpoint{4.987755in}{1.233913in}}%
\pgfpathlineto{\pgfqpoint{4.993197in}{1.150435in}}%
\pgfpathlineto{\pgfqpoint{4.998639in}{1.317391in}}%
\pgfpathlineto{\pgfqpoint{5.004082in}{1.233913in}}%
\pgfpathlineto{\pgfqpoint{5.014966in}{1.400870in}}%
\pgfpathlineto{\pgfqpoint{5.020408in}{1.317391in}}%
\pgfpathlineto{\pgfqpoint{5.025850in}{1.317391in}}%
\pgfpathlineto{\pgfqpoint{5.031293in}{1.484348in}}%
\pgfpathlineto{\pgfqpoint{5.036735in}{1.233913in}}%
\pgfpathlineto{\pgfqpoint{5.042177in}{1.233913in}}%
\pgfpathlineto{\pgfqpoint{5.047619in}{1.400870in}}%
\pgfpathlineto{\pgfqpoint{5.053061in}{1.233913in}}%
\pgfpathlineto{\pgfqpoint{5.058503in}{1.400870in}}%
\pgfpathlineto{\pgfqpoint{5.063946in}{1.317391in}}%
\pgfpathlineto{\pgfqpoint{5.069388in}{1.150435in}}%
\pgfpathlineto{\pgfqpoint{5.074830in}{1.317391in}}%
\pgfpathlineto{\pgfqpoint{5.080272in}{1.233913in}}%
\pgfpathlineto{\pgfqpoint{5.085714in}{1.233913in}}%
\pgfpathlineto{\pgfqpoint{5.091156in}{1.317391in}}%
\pgfpathlineto{\pgfqpoint{5.096599in}{1.233913in}}%
\pgfpathlineto{\pgfqpoint{5.102041in}{1.233913in}}%
\pgfpathlineto{\pgfqpoint{5.107483in}{1.651304in}}%
\pgfpathlineto{\pgfqpoint{5.112925in}{1.400870in}}%
\pgfpathlineto{\pgfqpoint{5.118367in}{1.400870in}}%
\pgfpathlineto{\pgfqpoint{5.123810in}{1.317391in}}%
\pgfpathlineto{\pgfqpoint{5.129252in}{1.317391in}}%
\pgfpathlineto{\pgfqpoint{5.134694in}{1.400870in}}%
\pgfpathlineto{\pgfqpoint{5.140136in}{1.400870in}}%
\pgfpathlineto{\pgfqpoint{5.145578in}{1.233913in}}%
\pgfpathlineto{\pgfqpoint{5.151020in}{1.150435in}}%
\pgfpathlineto{\pgfqpoint{5.156463in}{1.400870in}}%
\pgfpathlineto{\pgfqpoint{5.161905in}{1.484348in}}%
\pgfpathlineto{\pgfqpoint{5.167347in}{1.233913in}}%
\pgfpathlineto{\pgfqpoint{5.189116in}{1.233913in}}%
\pgfpathlineto{\pgfqpoint{5.194558in}{1.317391in}}%
\pgfpathlineto{\pgfqpoint{5.205442in}{1.150435in}}%
\pgfpathlineto{\pgfqpoint{5.210884in}{1.317391in}}%
\pgfpathlineto{\pgfqpoint{5.216327in}{1.317391in}}%
\pgfpathlineto{\pgfqpoint{5.227211in}{1.150435in}}%
\pgfpathlineto{\pgfqpoint{5.232653in}{1.317391in}}%
\pgfpathlineto{\pgfqpoint{5.238095in}{1.400870in}}%
\pgfpathlineto{\pgfqpoint{5.243537in}{1.400870in}}%
\pgfpathlineto{\pgfqpoint{5.248980in}{1.317391in}}%
\pgfpathlineto{\pgfqpoint{5.254422in}{1.317391in}}%
\pgfpathlineto{\pgfqpoint{5.259864in}{1.233913in}}%
\pgfpathlineto{\pgfqpoint{5.265306in}{1.484348in}}%
\pgfpathlineto{\pgfqpoint{5.270748in}{1.317391in}}%
\pgfpathlineto{\pgfqpoint{5.276190in}{1.484348in}}%
\pgfpathlineto{\pgfqpoint{5.281633in}{1.233913in}}%
\pgfpathlineto{\pgfqpoint{5.287075in}{1.317391in}}%
\pgfpathlineto{\pgfqpoint{5.292517in}{1.150435in}}%
\pgfpathlineto{\pgfqpoint{5.297959in}{1.317391in}}%
\pgfpathlineto{\pgfqpoint{5.303401in}{1.233913in}}%
\pgfpathlineto{\pgfqpoint{5.308844in}{1.317391in}}%
\pgfpathlineto{\pgfqpoint{5.314286in}{1.317391in}}%
\pgfpathlineto{\pgfqpoint{5.319728in}{1.400870in}}%
\pgfpathlineto{\pgfqpoint{5.325170in}{1.233913in}}%
\pgfpathlineto{\pgfqpoint{5.336054in}{1.400870in}}%
\pgfpathlineto{\pgfqpoint{5.341497in}{1.317391in}}%
\pgfpathlineto{\pgfqpoint{5.346939in}{1.150435in}}%
\pgfpathlineto{\pgfqpoint{5.352381in}{1.484348in}}%
\pgfpathlineto{\pgfqpoint{5.357823in}{1.233913in}}%
\pgfpathlineto{\pgfqpoint{5.363265in}{1.317391in}}%
\pgfpathlineto{\pgfqpoint{5.368707in}{1.233913in}}%
\pgfpathlineto{\pgfqpoint{5.374150in}{1.233913in}}%
\pgfpathlineto{\pgfqpoint{5.379592in}{1.317391in}}%
\pgfpathlineto{\pgfqpoint{5.385034in}{1.150435in}}%
\pgfpathlineto{\pgfqpoint{5.390476in}{1.317391in}}%
\pgfpathlineto{\pgfqpoint{5.395918in}{1.317391in}}%
\pgfpathlineto{\pgfqpoint{5.401361in}{1.484348in}}%
\pgfpathlineto{\pgfqpoint{5.412245in}{1.317391in}}%
\pgfpathlineto{\pgfqpoint{5.417687in}{1.400870in}}%
\pgfpathlineto{\pgfqpoint{5.428571in}{1.400870in}}%
\pgfpathlineto{\pgfqpoint{5.439456in}{1.233913in}}%
\pgfpathlineto{\pgfqpoint{5.444898in}{1.317391in}}%
\pgfpathlineto{\pgfqpoint{5.450340in}{1.317391in}}%
\pgfpathlineto{\pgfqpoint{5.455782in}{1.400870in}}%
\pgfpathlineto{\pgfqpoint{5.466667in}{1.400870in}}%
\pgfpathlineto{\pgfqpoint{5.472109in}{1.150435in}}%
\pgfpathlineto{\pgfqpoint{5.477551in}{1.317391in}}%
\pgfpathlineto{\pgfqpoint{5.482993in}{1.400870in}}%
\pgfpathlineto{\pgfqpoint{5.488435in}{1.233913in}}%
\pgfpathlineto{\pgfqpoint{5.493878in}{1.317391in}}%
\pgfpathlineto{\pgfqpoint{5.499320in}{1.233913in}}%
\pgfpathlineto{\pgfqpoint{5.504762in}{1.400870in}}%
\pgfpathlineto{\pgfqpoint{5.510204in}{1.317391in}}%
\pgfpathlineto{\pgfqpoint{5.515646in}{1.400870in}}%
\pgfpathlineto{\pgfqpoint{5.521088in}{1.317391in}}%
\pgfpathlineto{\pgfqpoint{5.526531in}{1.317391in}}%
\pgfpathlineto{\pgfqpoint{5.531973in}{1.400870in}}%
\pgfpathlineto{\pgfqpoint{5.537415in}{1.317391in}}%
\pgfpathlineto{\pgfqpoint{5.548299in}{1.317391in}}%
\pgfpathlineto{\pgfqpoint{5.553741in}{1.233913in}}%
\pgfpathlineto{\pgfqpoint{5.559184in}{1.400870in}}%
\pgfpathlineto{\pgfqpoint{5.564626in}{1.317391in}}%
\pgfpathlineto{\pgfqpoint{5.570068in}{1.317391in}}%
\pgfpathlineto{\pgfqpoint{5.575510in}{1.567826in}}%
\pgfpathlineto{\pgfqpoint{5.580952in}{1.400870in}}%
\pgfpathlineto{\pgfqpoint{5.586395in}{1.317391in}}%
\pgfpathlineto{\pgfqpoint{5.591837in}{1.400870in}}%
\pgfpathlineto{\pgfqpoint{5.597279in}{1.233913in}}%
\pgfpathlineto{\pgfqpoint{5.602721in}{1.317391in}}%
\pgfpathlineto{\pgfqpoint{5.613605in}{1.150435in}}%
\pgfpathlineto{\pgfqpoint{5.619048in}{1.400870in}}%
\pgfpathlineto{\pgfqpoint{5.624490in}{1.400870in}}%
\pgfpathlineto{\pgfqpoint{5.629932in}{1.233913in}}%
\pgfpathlineto{\pgfqpoint{5.635374in}{1.317391in}}%
\pgfpathlineto{\pgfqpoint{5.640816in}{1.317391in}}%
\pgfpathlineto{\pgfqpoint{5.651701in}{1.150435in}}%
\pgfpathlineto{\pgfqpoint{5.657143in}{1.233913in}}%
\pgfpathlineto{\pgfqpoint{5.662585in}{1.567826in}}%
\pgfpathlineto{\pgfqpoint{5.668027in}{1.317391in}}%
\pgfpathlineto{\pgfqpoint{5.673469in}{1.317391in}}%
\pgfpathlineto{\pgfqpoint{5.678912in}{1.150435in}}%
\pgfpathlineto{\pgfqpoint{5.684354in}{1.400870in}}%
\pgfpathlineto{\pgfqpoint{5.689796in}{1.400870in}}%
\pgfpathlineto{\pgfqpoint{5.695238in}{1.233913in}}%
\pgfpathlineto{\pgfqpoint{5.700680in}{1.317391in}}%
\pgfpathlineto{\pgfqpoint{5.706122in}{1.317391in}}%
\pgfpathlineto{\pgfqpoint{5.711565in}{1.400870in}}%
\pgfpathlineto{\pgfqpoint{5.722449in}{1.400870in}}%
\pgfpathlineto{\pgfqpoint{5.727891in}{1.317391in}}%
\pgfpathlineto{\pgfqpoint{5.733333in}{1.484348in}}%
\pgfpathlineto{\pgfqpoint{5.738776in}{1.233913in}}%
\pgfpathlineto{\pgfqpoint{5.744218in}{1.400870in}}%
\pgfpathlineto{\pgfqpoint{5.755102in}{1.233913in}}%
\pgfpathlineto{\pgfqpoint{5.765986in}{1.233913in}}%
\pgfpathlineto{\pgfqpoint{5.776871in}{1.400870in}}%
\pgfpathlineto{\pgfqpoint{5.782313in}{1.317391in}}%
\pgfpathlineto{\pgfqpoint{5.787755in}{1.484348in}}%
\pgfpathlineto{\pgfqpoint{5.793197in}{1.317391in}}%
\pgfpathlineto{\pgfqpoint{5.798639in}{1.400870in}}%
\pgfpathlineto{\pgfqpoint{5.804082in}{1.317391in}}%
\pgfpathlineto{\pgfqpoint{5.809524in}{1.400870in}}%
\pgfpathlineto{\pgfqpoint{5.814966in}{1.317391in}}%
\pgfpathlineto{\pgfqpoint{5.820408in}{1.400870in}}%
\pgfpathlineto{\pgfqpoint{5.825850in}{1.400870in}}%
\pgfpathlineto{\pgfqpoint{5.831293in}{1.233913in}}%
\pgfpathlineto{\pgfqpoint{5.836735in}{1.233913in}}%
\pgfpathlineto{\pgfqpoint{5.842177in}{1.400870in}}%
\pgfpathlineto{\pgfqpoint{5.847619in}{1.400870in}}%
\pgfpathlineto{\pgfqpoint{5.858503in}{1.233913in}}%
\pgfpathlineto{\pgfqpoint{5.863946in}{1.317391in}}%
\pgfpathlineto{\pgfqpoint{5.869388in}{1.150435in}}%
\pgfpathlineto{\pgfqpoint{5.874830in}{1.400870in}}%
\pgfpathlineto{\pgfqpoint{5.880272in}{1.317391in}}%
\pgfpathlineto{\pgfqpoint{5.885714in}{1.400870in}}%
\pgfpathlineto{\pgfqpoint{5.891156in}{1.150435in}}%
\pgfpathlineto{\pgfqpoint{5.896599in}{1.317391in}}%
\pgfpathlineto{\pgfqpoint{5.902041in}{1.400870in}}%
\pgfpathlineto{\pgfqpoint{5.907483in}{1.317391in}}%
\pgfpathlineto{\pgfqpoint{5.918367in}{1.484348in}}%
\pgfpathlineto{\pgfqpoint{5.923810in}{1.400870in}}%
\pgfpathlineto{\pgfqpoint{5.929252in}{1.484348in}}%
\pgfpathlineto{\pgfqpoint{5.934694in}{1.233913in}}%
\pgfpathlineto{\pgfqpoint{5.940136in}{1.400870in}}%
\pgfpathlineto{\pgfqpoint{5.945578in}{1.317391in}}%
\pgfpathlineto{\pgfqpoint{5.951020in}{1.484348in}}%
\pgfpathlineto{\pgfqpoint{5.956463in}{1.317391in}}%
\pgfpathlineto{\pgfqpoint{5.972789in}{1.317391in}}%
\pgfpathlineto{\pgfqpoint{5.978231in}{1.233913in}}%
\pgfpathlineto{\pgfqpoint{5.983673in}{1.400870in}}%
\pgfpathlineto{\pgfqpoint{5.989116in}{1.484348in}}%
\pgfpathlineto{\pgfqpoint{5.994558in}{1.317391in}}%
\pgfpathlineto{\pgfqpoint{6.005442in}{1.317391in}}%
\pgfpathlineto{\pgfqpoint{6.010884in}{1.150435in}}%
\pgfpathlineto{\pgfqpoint{6.016327in}{1.233913in}}%
\pgfpathlineto{\pgfqpoint{6.021769in}{1.150435in}}%
\pgfpathlineto{\pgfqpoint{6.027211in}{1.317391in}}%
\pgfpathlineto{\pgfqpoint{6.032653in}{1.400870in}}%
\pgfpathlineto{\pgfqpoint{6.038095in}{1.317391in}}%
\pgfpathlineto{\pgfqpoint{6.043537in}{1.400870in}}%
\pgfpathlineto{\pgfqpoint{6.048980in}{1.400870in}}%
\pgfpathlineto{\pgfqpoint{6.059864in}{1.233913in}}%
\pgfpathlineto{\pgfqpoint{6.065306in}{1.233913in}}%
\pgfpathlineto{\pgfqpoint{6.070748in}{1.400870in}}%
\pgfpathlineto{\pgfqpoint{6.081633in}{1.233913in}}%
\pgfpathlineto{\pgfqpoint{6.087075in}{1.317391in}}%
\pgfpathlineto{\pgfqpoint{6.092517in}{1.233913in}}%
\pgfpathlineto{\pgfqpoint{6.103401in}{1.400870in}}%
\pgfpathlineto{\pgfqpoint{6.108844in}{1.317391in}}%
\pgfpathlineto{\pgfqpoint{6.114286in}{1.150435in}}%
\pgfpathlineto{\pgfqpoint{6.119728in}{1.317391in}}%
\pgfpathlineto{\pgfqpoint{6.125170in}{1.150435in}}%
\pgfpathlineto{\pgfqpoint{6.130612in}{1.484348in}}%
\pgfpathlineto{\pgfqpoint{6.136054in}{1.317391in}}%
\pgfpathlineto{\pgfqpoint{6.141497in}{1.317391in}}%
\pgfpathlineto{\pgfqpoint{6.146939in}{1.233913in}}%
\pgfpathlineto{\pgfqpoint{6.152381in}{1.400870in}}%
\pgfpathlineto{\pgfqpoint{6.157823in}{1.317391in}}%
\pgfpathlineto{\pgfqpoint{6.163265in}{1.484348in}}%
\pgfpathlineto{\pgfqpoint{6.179592in}{1.233913in}}%
\pgfpathlineto{\pgfqpoint{6.185034in}{1.317391in}}%
\pgfpathlineto{\pgfqpoint{6.190476in}{1.317391in}}%
\pgfpathlineto{\pgfqpoint{6.195918in}{1.484348in}}%
\pgfpathlineto{\pgfqpoint{6.201361in}{1.233913in}}%
\pgfpathlineto{\pgfqpoint{6.206803in}{1.400870in}}%
\pgfpathlineto{\pgfqpoint{6.212245in}{1.484348in}}%
\pgfpathlineto{\pgfqpoint{6.217687in}{1.400870in}}%
\pgfpathlineto{\pgfqpoint{6.223129in}{1.484348in}}%
\pgfpathlineto{\pgfqpoint{6.228571in}{1.400870in}}%
\pgfpathlineto{\pgfqpoint{6.234014in}{1.233913in}}%
\pgfpathlineto{\pgfqpoint{6.239456in}{1.317391in}}%
\pgfpathlineto{\pgfqpoint{6.244898in}{1.233913in}}%
\pgfpathlineto{\pgfqpoint{6.250340in}{1.400870in}}%
\pgfpathlineto{\pgfqpoint{6.255782in}{1.317391in}}%
\pgfpathlineto{\pgfqpoint{6.261224in}{1.317391in}}%
\pgfpathlineto{\pgfqpoint{6.266667in}{1.233913in}}%
\pgfpathlineto{\pgfqpoint{6.272109in}{1.317391in}}%
\pgfpathlineto{\pgfqpoint{6.277551in}{1.233913in}}%
\pgfpathlineto{\pgfqpoint{6.282993in}{1.317391in}}%
\pgfpathlineto{\pgfqpoint{6.288435in}{1.317391in}}%
\pgfpathlineto{\pgfqpoint{6.293878in}{1.400870in}}%
\pgfpathlineto{\pgfqpoint{6.299320in}{1.233913in}}%
\pgfpathlineto{\pgfqpoint{6.304762in}{1.484348in}}%
\pgfpathlineto{\pgfqpoint{6.310204in}{1.317391in}}%
\pgfpathlineto{\pgfqpoint{6.315646in}{1.484348in}}%
\pgfpathlineto{\pgfqpoint{6.321088in}{1.317391in}}%
\pgfpathlineto{\pgfqpoint{6.326531in}{1.233913in}}%
\pgfpathlineto{\pgfqpoint{6.331973in}{1.233913in}}%
\pgfpathlineto{\pgfqpoint{6.342857in}{1.400870in}}%
\pgfpathlineto{\pgfqpoint{6.348299in}{1.400870in}}%
\pgfpathlineto{\pgfqpoint{6.353741in}{1.233913in}}%
\pgfpathlineto{\pgfqpoint{6.359184in}{1.400870in}}%
\pgfpathlineto{\pgfqpoint{6.364626in}{1.233913in}}%
\pgfpathlineto{\pgfqpoint{6.370068in}{1.317391in}}%
\pgfpathlineto{\pgfqpoint{6.375510in}{1.233913in}}%
\pgfpathlineto{\pgfqpoint{6.380952in}{1.400870in}}%
\pgfpathlineto{\pgfqpoint{6.391837in}{1.233913in}}%
\pgfpathlineto{\pgfqpoint{6.397279in}{1.317391in}}%
\pgfpathlineto{\pgfqpoint{6.402721in}{1.233913in}}%
\pgfpathlineto{\pgfqpoint{6.413605in}{1.400870in}}%
\pgfpathlineto{\pgfqpoint{6.419048in}{1.317391in}}%
\pgfpathlineto{\pgfqpoint{6.424490in}{1.400870in}}%
\pgfpathlineto{\pgfqpoint{6.435374in}{1.233913in}}%
\pgfpathlineto{\pgfqpoint{6.440816in}{1.233913in}}%
\pgfpathlineto{\pgfqpoint{6.446259in}{1.317391in}}%
\pgfpathlineto{\pgfqpoint{6.451701in}{1.233913in}}%
\pgfpathlineto{\pgfqpoint{6.457143in}{1.317391in}}%
\pgfpathlineto{\pgfqpoint{6.462585in}{1.233913in}}%
\pgfpathlineto{\pgfqpoint{6.468027in}{1.233913in}}%
\pgfpathlineto{\pgfqpoint{6.473469in}{1.150435in}}%
\pgfpathlineto{\pgfqpoint{6.478912in}{1.150435in}}%
\pgfpathlineto{\pgfqpoint{6.484354in}{1.317391in}}%
\pgfpathlineto{\pgfqpoint{6.495238in}{1.484348in}}%
\pgfpathlineto{\pgfqpoint{6.506122in}{1.317391in}}%
\pgfpathlineto{\pgfqpoint{6.511565in}{1.400870in}}%
\pgfpathlineto{\pgfqpoint{6.517007in}{1.317391in}}%
\pgfpathlineto{\pgfqpoint{6.522449in}{1.317391in}}%
\pgfpathlineto{\pgfqpoint{6.527891in}{1.233913in}}%
\pgfpathlineto{\pgfqpoint{6.538776in}{1.400870in}}%
\pgfpathlineto{\pgfqpoint{6.549660in}{1.233913in}}%
\pgfpathlineto{\pgfqpoint{6.555102in}{1.233913in}}%
\pgfpathlineto{\pgfqpoint{6.560544in}{1.400870in}}%
\pgfpathlineto{\pgfqpoint{6.565986in}{1.233913in}}%
\pgfpathlineto{\pgfqpoint{6.571429in}{1.400870in}}%
\pgfpathlineto{\pgfqpoint{6.576871in}{1.233913in}}%
\pgfpathlineto{\pgfqpoint{6.582313in}{1.567826in}}%
\pgfpathlineto{\pgfqpoint{6.587755in}{1.233913in}}%
\pgfpathlineto{\pgfqpoint{6.593197in}{1.400870in}}%
\pgfpathlineto{\pgfqpoint{6.598639in}{1.317391in}}%
\pgfpathlineto{\pgfqpoint{6.604082in}{1.400870in}}%
\pgfpathlineto{\pgfqpoint{6.609524in}{1.233913in}}%
\pgfpathlineto{\pgfqpoint{6.620408in}{1.400870in}}%
\pgfpathlineto{\pgfqpoint{6.625850in}{1.317391in}}%
\pgfpathlineto{\pgfqpoint{6.642177in}{1.317391in}}%
\pgfpathlineto{\pgfqpoint{6.647619in}{1.400870in}}%
\pgfpathlineto{\pgfqpoint{6.653061in}{1.233913in}}%
\pgfpathlineto{\pgfqpoint{6.658503in}{1.317391in}}%
\pgfpathlineto{\pgfqpoint{6.663946in}{1.317391in}}%
\pgfpathlineto{\pgfqpoint{6.669388in}{1.233913in}}%
\pgfpathlineto{\pgfqpoint{6.674830in}{1.317391in}}%
\pgfpathlineto{\pgfqpoint{6.680272in}{1.150435in}}%
\pgfpathlineto{\pgfqpoint{6.691156in}{1.317391in}}%
\pgfpathlineto{\pgfqpoint{6.696599in}{1.233913in}}%
\pgfpathlineto{\pgfqpoint{6.702041in}{1.317391in}}%
\pgfpathlineto{\pgfqpoint{6.707483in}{1.317391in}}%
\pgfpathlineto{\pgfqpoint{6.718367in}{1.150435in}}%
\pgfpathlineto{\pgfqpoint{6.729252in}{1.484348in}}%
\pgfpathlineto{\pgfqpoint{6.745578in}{1.233913in}}%
\pgfpathlineto{\pgfqpoint{6.751020in}{1.317391in}}%
\pgfpathlineto{\pgfqpoint{6.756463in}{1.233913in}}%
\pgfpathlineto{\pgfqpoint{6.761905in}{1.317391in}}%
\pgfpathlineto{\pgfqpoint{6.767347in}{1.233913in}}%
\pgfpathlineto{\pgfqpoint{6.772789in}{1.317391in}}%
\pgfpathlineto{\pgfqpoint{6.778231in}{1.484348in}}%
\pgfpathlineto{\pgfqpoint{6.783673in}{1.150435in}}%
\pgfpathlineto{\pgfqpoint{6.789116in}{1.317391in}}%
\pgfpathlineto{\pgfqpoint{6.794558in}{1.400870in}}%
\pgfpathlineto{\pgfqpoint{6.794558in}{1.400870in}}%
\pgfusepath{stroke}%
\end{pgfscope}%
\begin{pgfscope}%
\pgfsetrectcap%
\pgfsetmiterjoin%
\pgfsetlinewidth{1.003750pt}%
\definecolor{currentstroke}{rgb}{0.000000,0.000000,0.000000}%
\pgfsetstrokecolor{currentstroke}%
\pgfsetdash{}{0pt}%
\pgfpathmoveto{\pgfqpoint{1.200000in}{0.900000in}}%
\pgfpathlineto{\pgfqpoint{1.200000in}{5.700000in}}%
\pgfusepath{stroke}%
\end{pgfscope}%
\begin{pgfscope}%
\pgfsetrectcap%
\pgfsetmiterjoin%
\pgfsetlinewidth{1.003750pt}%
\definecolor{currentstroke}{rgb}{0.000000,0.000000,0.000000}%
\pgfsetstrokecolor{currentstroke}%
\pgfsetdash{}{0pt}%
\pgfpathmoveto{\pgfqpoint{6.800000in}{0.900000in}}%
\pgfpathlineto{\pgfqpoint{6.800000in}{5.700000in}}%
\pgfusepath{stroke}%
\end{pgfscope}%
\begin{pgfscope}%
\pgfsetrectcap%
\pgfsetmiterjoin%
\pgfsetlinewidth{1.003750pt}%
\definecolor{currentstroke}{rgb}{0.000000,0.000000,0.000000}%
\pgfsetstrokecolor{currentstroke}%
\pgfsetdash{}{0pt}%
\pgfpathmoveto{\pgfqpoint{1.200000in}{0.900000in}}%
\pgfpathlineto{\pgfqpoint{6.800000in}{0.900000in}}%
\pgfusepath{stroke}%
\end{pgfscope}%
\begin{pgfscope}%
\pgfsetrectcap%
\pgfsetmiterjoin%
\pgfsetlinewidth{1.003750pt}%
\definecolor{currentstroke}{rgb}{0.000000,0.000000,0.000000}%
\pgfsetstrokecolor{currentstroke}%
\pgfsetdash{}{0pt}%
\pgfpathmoveto{\pgfqpoint{1.200000in}{5.700000in}}%
\pgfpathlineto{\pgfqpoint{6.800000in}{5.700000in}}%
\pgfusepath{stroke}%
\end{pgfscope}%
\begin{pgfscope}%
\pgfsetbuttcap%
\pgfsetroundjoin%
\definecolor{currentfill}{rgb}{0.000000,0.000000,0.000000}%
\pgfsetfillcolor{currentfill}%
\pgfsetlinewidth{0.501875pt}%
\definecolor{currentstroke}{rgb}{0.000000,0.000000,0.000000}%
\pgfsetstrokecolor{currentstroke}%
\pgfsetdash{}{0pt}%
\pgfsys@defobject{currentmarker}{\pgfqpoint{0.000000in}{0.000000in}}{\pgfqpoint{0.000000in}{0.055556in}}{%
\pgfpathmoveto{\pgfqpoint{0.000000in}{0.000000in}}%
\pgfpathlineto{\pgfqpoint{0.000000in}{0.055556in}}%
\pgfusepath{stroke,fill}%
}%
\begin{pgfscope}%
\pgfsys@transformshift{1.200000in}{0.900000in}%
\pgfsys@useobject{currentmarker}{}%
\end{pgfscope}%
\end{pgfscope}%
\begin{pgfscope}%
\pgfsetbuttcap%
\pgfsetroundjoin%
\definecolor{currentfill}{rgb}{0.000000,0.000000,0.000000}%
\pgfsetfillcolor{currentfill}%
\pgfsetlinewidth{0.501875pt}%
\definecolor{currentstroke}{rgb}{0.000000,0.000000,0.000000}%
\pgfsetstrokecolor{currentstroke}%
\pgfsetdash{}{0pt}%
\pgfsys@defobject{currentmarker}{\pgfqpoint{0.000000in}{-0.055556in}}{\pgfqpoint{0.000000in}{0.000000in}}{%
\pgfpathmoveto{\pgfqpoint{0.000000in}{0.000000in}}%
\pgfpathlineto{\pgfqpoint{0.000000in}{-0.055556in}}%
\pgfusepath{stroke,fill}%
}%
\begin{pgfscope}%
\pgfsys@transformshift{1.200000in}{5.700000in}%
\pgfsys@useobject{currentmarker}{}%
\end{pgfscope}%
\end{pgfscope}%
\begin{pgfscope}%
\definecolor{textcolor}{rgb}{0.000000,0.000000,0.000000}%
\pgfsetstrokecolor{textcolor}%
\pgfsetfillcolor{textcolor}%
\pgftext[x=1.200000in,y=0.844444in,,top]{\color{textcolor}\sffamily\fontsize{20.000000}{24.000000}\selectfont \(\displaystyle {0}\)}%
\end{pgfscope}%
\begin{pgfscope}%
\pgfsetbuttcap%
\pgfsetroundjoin%
\definecolor{currentfill}{rgb}{0.000000,0.000000,0.000000}%
\pgfsetfillcolor{currentfill}%
\pgfsetlinewidth{0.501875pt}%
\definecolor{currentstroke}{rgb}{0.000000,0.000000,0.000000}%
\pgfsetstrokecolor{currentstroke}%
\pgfsetdash{}{0pt}%
\pgfsys@defobject{currentmarker}{\pgfqpoint{0.000000in}{0.000000in}}{\pgfqpoint{0.000000in}{0.055556in}}{%
\pgfpathmoveto{\pgfqpoint{0.000000in}{0.000000in}}%
\pgfpathlineto{\pgfqpoint{0.000000in}{0.055556in}}%
\pgfusepath{stroke,fill}%
}%
\begin{pgfscope}%
\pgfsys@transformshift{2.288435in}{0.900000in}%
\pgfsys@useobject{currentmarker}{}%
\end{pgfscope}%
\end{pgfscope}%
\begin{pgfscope}%
\pgfsetbuttcap%
\pgfsetroundjoin%
\definecolor{currentfill}{rgb}{0.000000,0.000000,0.000000}%
\pgfsetfillcolor{currentfill}%
\pgfsetlinewidth{0.501875pt}%
\definecolor{currentstroke}{rgb}{0.000000,0.000000,0.000000}%
\pgfsetstrokecolor{currentstroke}%
\pgfsetdash{}{0pt}%
\pgfsys@defobject{currentmarker}{\pgfqpoint{0.000000in}{-0.055556in}}{\pgfqpoint{0.000000in}{0.000000in}}{%
\pgfpathmoveto{\pgfqpoint{0.000000in}{0.000000in}}%
\pgfpathlineto{\pgfqpoint{0.000000in}{-0.055556in}}%
\pgfusepath{stroke,fill}%
}%
\begin{pgfscope}%
\pgfsys@transformshift{2.288435in}{5.700000in}%
\pgfsys@useobject{currentmarker}{}%
\end{pgfscope}%
\end{pgfscope}%
\begin{pgfscope}%
\definecolor{textcolor}{rgb}{0.000000,0.000000,0.000000}%
\pgfsetstrokecolor{textcolor}%
\pgfsetfillcolor{textcolor}%
\pgftext[x=2.288435in,y=0.844444in,,top]{\color{textcolor}\sffamily\fontsize{20.000000}{24.000000}\selectfont \(\displaystyle {200}\)}%
\end{pgfscope}%
\begin{pgfscope}%
\pgfsetbuttcap%
\pgfsetroundjoin%
\definecolor{currentfill}{rgb}{0.000000,0.000000,0.000000}%
\pgfsetfillcolor{currentfill}%
\pgfsetlinewidth{0.501875pt}%
\definecolor{currentstroke}{rgb}{0.000000,0.000000,0.000000}%
\pgfsetstrokecolor{currentstroke}%
\pgfsetdash{}{0pt}%
\pgfsys@defobject{currentmarker}{\pgfqpoint{0.000000in}{0.000000in}}{\pgfqpoint{0.000000in}{0.055556in}}{%
\pgfpathmoveto{\pgfqpoint{0.000000in}{0.000000in}}%
\pgfpathlineto{\pgfqpoint{0.000000in}{0.055556in}}%
\pgfusepath{stroke,fill}%
}%
\begin{pgfscope}%
\pgfsys@transformshift{3.376871in}{0.900000in}%
\pgfsys@useobject{currentmarker}{}%
\end{pgfscope}%
\end{pgfscope}%
\begin{pgfscope}%
\pgfsetbuttcap%
\pgfsetroundjoin%
\definecolor{currentfill}{rgb}{0.000000,0.000000,0.000000}%
\pgfsetfillcolor{currentfill}%
\pgfsetlinewidth{0.501875pt}%
\definecolor{currentstroke}{rgb}{0.000000,0.000000,0.000000}%
\pgfsetstrokecolor{currentstroke}%
\pgfsetdash{}{0pt}%
\pgfsys@defobject{currentmarker}{\pgfqpoint{0.000000in}{-0.055556in}}{\pgfqpoint{0.000000in}{0.000000in}}{%
\pgfpathmoveto{\pgfqpoint{0.000000in}{0.000000in}}%
\pgfpathlineto{\pgfqpoint{0.000000in}{-0.055556in}}%
\pgfusepath{stroke,fill}%
}%
\begin{pgfscope}%
\pgfsys@transformshift{3.376871in}{5.700000in}%
\pgfsys@useobject{currentmarker}{}%
\end{pgfscope}%
\end{pgfscope}%
\begin{pgfscope}%
\definecolor{textcolor}{rgb}{0.000000,0.000000,0.000000}%
\pgfsetstrokecolor{textcolor}%
\pgfsetfillcolor{textcolor}%
\pgftext[x=3.376871in,y=0.844444in,,top]{\color{textcolor}\sffamily\fontsize{20.000000}{24.000000}\selectfont \(\displaystyle {400}\)}%
\end{pgfscope}%
\begin{pgfscope}%
\pgfsetbuttcap%
\pgfsetroundjoin%
\definecolor{currentfill}{rgb}{0.000000,0.000000,0.000000}%
\pgfsetfillcolor{currentfill}%
\pgfsetlinewidth{0.501875pt}%
\definecolor{currentstroke}{rgb}{0.000000,0.000000,0.000000}%
\pgfsetstrokecolor{currentstroke}%
\pgfsetdash{}{0pt}%
\pgfsys@defobject{currentmarker}{\pgfqpoint{0.000000in}{0.000000in}}{\pgfqpoint{0.000000in}{0.055556in}}{%
\pgfpathmoveto{\pgfqpoint{0.000000in}{0.000000in}}%
\pgfpathlineto{\pgfqpoint{0.000000in}{0.055556in}}%
\pgfusepath{stroke,fill}%
}%
\begin{pgfscope}%
\pgfsys@transformshift{4.465306in}{0.900000in}%
\pgfsys@useobject{currentmarker}{}%
\end{pgfscope}%
\end{pgfscope}%
\begin{pgfscope}%
\pgfsetbuttcap%
\pgfsetroundjoin%
\definecolor{currentfill}{rgb}{0.000000,0.000000,0.000000}%
\pgfsetfillcolor{currentfill}%
\pgfsetlinewidth{0.501875pt}%
\definecolor{currentstroke}{rgb}{0.000000,0.000000,0.000000}%
\pgfsetstrokecolor{currentstroke}%
\pgfsetdash{}{0pt}%
\pgfsys@defobject{currentmarker}{\pgfqpoint{0.000000in}{-0.055556in}}{\pgfqpoint{0.000000in}{0.000000in}}{%
\pgfpathmoveto{\pgfqpoint{0.000000in}{0.000000in}}%
\pgfpathlineto{\pgfqpoint{0.000000in}{-0.055556in}}%
\pgfusepath{stroke,fill}%
}%
\begin{pgfscope}%
\pgfsys@transformshift{4.465306in}{5.700000in}%
\pgfsys@useobject{currentmarker}{}%
\end{pgfscope}%
\end{pgfscope}%
\begin{pgfscope}%
\definecolor{textcolor}{rgb}{0.000000,0.000000,0.000000}%
\pgfsetstrokecolor{textcolor}%
\pgfsetfillcolor{textcolor}%
\pgftext[x=4.465306in,y=0.844444in,,top]{\color{textcolor}\sffamily\fontsize{20.000000}{24.000000}\selectfont \(\displaystyle {600}\)}%
\end{pgfscope}%
\begin{pgfscope}%
\pgfsetbuttcap%
\pgfsetroundjoin%
\definecolor{currentfill}{rgb}{0.000000,0.000000,0.000000}%
\pgfsetfillcolor{currentfill}%
\pgfsetlinewidth{0.501875pt}%
\definecolor{currentstroke}{rgb}{0.000000,0.000000,0.000000}%
\pgfsetstrokecolor{currentstroke}%
\pgfsetdash{}{0pt}%
\pgfsys@defobject{currentmarker}{\pgfqpoint{0.000000in}{0.000000in}}{\pgfqpoint{0.000000in}{0.055556in}}{%
\pgfpathmoveto{\pgfqpoint{0.000000in}{0.000000in}}%
\pgfpathlineto{\pgfqpoint{0.000000in}{0.055556in}}%
\pgfusepath{stroke,fill}%
}%
\begin{pgfscope}%
\pgfsys@transformshift{5.553741in}{0.900000in}%
\pgfsys@useobject{currentmarker}{}%
\end{pgfscope}%
\end{pgfscope}%
\begin{pgfscope}%
\pgfsetbuttcap%
\pgfsetroundjoin%
\definecolor{currentfill}{rgb}{0.000000,0.000000,0.000000}%
\pgfsetfillcolor{currentfill}%
\pgfsetlinewidth{0.501875pt}%
\definecolor{currentstroke}{rgb}{0.000000,0.000000,0.000000}%
\pgfsetstrokecolor{currentstroke}%
\pgfsetdash{}{0pt}%
\pgfsys@defobject{currentmarker}{\pgfqpoint{0.000000in}{-0.055556in}}{\pgfqpoint{0.000000in}{0.000000in}}{%
\pgfpathmoveto{\pgfqpoint{0.000000in}{0.000000in}}%
\pgfpathlineto{\pgfqpoint{0.000000in}{-0.055556in}}%
\pgfusepath{stroke,fill}%
}%
\begin{pgfscope}%
\pgfsys@transformshift{5.553741in}{5.700000in}%
\pgfsys@useobject{currentmarker}{}%
\end{pgfscope}%
\end{pgfscope}%
\begin{pgfscope}%
\definecolor{textcolor}{rgb}{0.000000,0.000000,0.000000}%
\pgfsetstrokecolor{textcolor}%
\pgfsetfillcolor{textcolor}%
\pgftext[x=5.553741in,y=0.844444in,,top]{\color{textcolor}\sffamily\fontsize{20.000000}{24.000000}\selectfont \(\displaystyle {800}\)}%
\end{pgfscope}%
\begin{pgfscope}%
\pgfsetbuttcap%
\pgfsetroundjoin%
\definecolor{currentfill}{rgb}{0.000000,0.000000,0.000000}%
\pgfsetfillcolor{currentfill}%
\pgfsetlinewidth{0.501875pt}%
\definecolor{currentstroke}{rgb}{0.000000,0.000000,0.000000}%
\pgfsetstrokecolor{currentstroke}%
\pgfsetdash{}{0pt}%
\pgfsys@defobject{currentmarker}{\pgfqpoint{0.000000in}{0.000000in}}{\pgfqpoint{0.000000in}{0.055556in}}{%
\pgfpathmoveto{\pgfqpoint{0.000000in}{0.000000in}}%
\pgfpathlineto{\pgfqpoint{0.000000in}{0.055556in}}%
\pgfusepath{stroke,fill}%
}%
\begin{pgfscope}%
\pgfsys@transformshift{6.642177in}{0.900000in}%
\pgfsys@useobject{currentmarker}{}%
\end{pgfscope}%
\end{pgfscope}%
\begin{pgfscope}%
\pgfsetbuttcap%
\pgfsetroundjoin%
\definecolor{currentfill}{rgb}{0.000000,0.000000,0.000000}%
\pgfsetfillcolor{currentfill}%
\pgfsetlinewidth{0.501875pt}%
\definecolor{currentstroke}{rgb}{0.000000,0.000000,0.000000}%
\pgfsetstrokecolor{currentstroke}%
\pgfsetdash{}{0pt}%
\pgfsys@defobject{currentmarker}{\pgfqpoint{0.000000in}{-0.055556in}}{\pgfqpoint{0.000000in}{0.000000in}}{%
\pgfpathmoveto{\pgfqpoint{0.000000in}{0.000000in}}%
\pgfpathlineto{\pgfqpoint{0.000000in}{-0.055556in}}%
\pgfusepath{stroke,fill}%
}%
\begin{pgfscope}%
\pgfsys@transformshift{6.642177in}{5.700000in}%
\pgfsys@useobject{currentmarker}{}%
\end{pgfscope}%
\end{pgfscope}%
\begin{pgfscope}%
\definecolor{textcolor}{rgb}{0.000000,0.000000,0.000000}%
\pgfsetstrokecolor{textcolor}%
\pgfsetfillcolor{textcolor}%
\pgftext[x=6.642177in,y=0.844444in,,top]{\color{textcolor}\sffamily\fontsize{20.000000}{24.000000}\selectfont \(\displaystyle {1000}\)}%
\end{pgfscope}%
\begin{pgfscope}%
\definecolor{textcolor}{rgb}{0.000000,0.000000,0.000000}%
\pgfsetstrokecolor{textcolor}%
\pgfsetfillcolor{textcolor}%
\pgftext[x=4.000000in,y=0.518932in,,top]{\color{textcolor}\sffamily\fontsize{20.000000}{24.000000}\selectfont \(\displaystyle \mathrm{t}/\si{ns}\)}%
\end{pgfscope}%
\begin{pgfscope}%
\pgfsetbuttcap%
\pgfsetroundjoin%
\definecolor{currentfill}{rgb}{0.000000,0.000000,0.000000}%
\pgfsetfillcolor{currentfill}%
\pgfsetlinewidth{0.501875pt}%
\definecolor{currentstroke}{rgb}{0.000000,0.000000,0.000000}%
\pgfsetstrokecolor{currentstroke}%
\pgfsetdash{}{0pt}%
\pgfsys@defobject{currentmarker}{\pgfqpoint{0.000000in}{0.000000in}}{\pgfqpoint{0.055556in}{0.000000in}}{%
\pgfpathmoveto{\pgfqpoint{0.000000in}{0.000000in}}%
\pgfpathlineto{\pgfqpoint{0.055556in}{0.000000in}}%
\pgfusepath{stroke,fill}%
}%
\begin{pgfscope}%
\pgfsys@transformshift{1.200000in}{1.317391in}%
\pgfsys@useobject{currentmarker}{}%
\end{pgfscope}%
\end{pgfscope}%
\begin{pgfscope}%
\pgfsetbuttcap%
\pgfsetroundjoin%
\definecolor{currentfill}{rgb}{0.000000,0.000000,0.000000}%
\pgfsetfillcolor{currentfill}%
\pgfsetlinewidth{0.501875pt}%
\definecolor{currentstroke}{rgb}{0.000000,0.000000,0.000000}%
\pgfsetstrokecolor{currentstroke}%
\pgfsetdash{}{0pt}%
\pgfsys@defobject{currentmarker}{\pgfqpoint{-0.055556in}{0.000000in}}{\pgfqpoint{-0.000000in}{0.000000in}}{%
\pgfpathmoveto{\pgfqpoint{-0.000000in}{0.000000in}}%
\pgfpathlineto{\pgfqpoint{-0.055556in}{0.000000in}}%
\pgfusepath{stroke,fill}%
}%
\begin{pgfscope}%
\pgfsys@transformshift{6.800000in}{1.317391in}%
\pgfsys@useobject{currentmarker}{}%
\end{pgfscope}%
\end{pgfscope}%
\begin{pgfscope}%
\definecolor{textcolor}{rgb}{0.000000,0.000000,0.000000}%
\pgfsetstrokecolor{textcolor}%
\pgfsetfillcolor{textcolor}%
\pgftext[x=1.144444in,y=1.317391in,right,]{\color{textcolor}\sffamily\fontsize{20.000000}{24.000000}\selectfont \(\displaystyle {0}\)}%
\end{pgfscope}%
\begin{pgfscope}%
\pgfsetbuttcap%
\pgfsetroundjoin%
\definecolor{currentfill}{rgb}{0.000000,0.000000,0.000000}%
\pgfsetfillcolor{currentfill}%
\pgfsetlinewidth{0.501875pt}%
\definecolor{currentstroke}{rgb}{0.000000,0.000000,0.000000}%
\pgfsetstrokecolor{currentstroke}%
\pgfsetdash{}{0pt}%
\pgfsys@defobject{currentmarker}{\pgfqpoint{0.000000in}{0.000000in}}{\pgfqpoint{0.055556in}{0.000000in}}{%
\pgfpathmoveto{\pgfqpoint{0.000000in}{0.000000in}}%
\pgfpathlineto{\pgfqpoint{0.055556in}{0.000000in}}%
\pgfusepath{stroke,fill}%
}%
\begin{pgfscope}%
\pgfsys@transformshift{1.200000in}{2.152174in}%
\pgfsys@useobject{currentmarker}{}%
\end{pgfscope}%
\end{pgfscope}%
\begin{pgfscope}%
\pgfsetbuttcap%
\pgfsetroundjoin%
\definecolor{currentfill}{rgb}{0.000000,0.000000,0.000000}%
\pgfsetfillcolor{currentfill}%
\pgfsetlinewidth{0.501875pt}%
\definecolor{currentstroke}{rgb}{0.000000,0.000000,0.000000}%
\pgfsetstrokecolor{currentstroke}%
\pgfsetdash{}{0pt}%
\pgfsys@defobject{currentmarker}{\pgfqpoint{-0.055556in}{0.000000in}}{\pgfqpoint{-0.000000in}{0.000000in}}{%
\pgfpathmoveto{\pgfqpoint{-0.000000in}{0.000000in}}%
\pgfpathlineto{\pgfqpoint{-0.055556in}{0.000000in}}%
\pgfusepath{stroke,fill}%
}%
\begin{pgfscope}%
\pgfsys@transformshift{6.800000in}{2.152174in}%
\pgfsys@useobject{currentmarker}{}%
\end{pgfscope}%
\end{pgfscope}%
\begin{pgfscope}%
\definecolor{textcolor}{rgb}{0.000000,0.000000,0.000000}%
\pgfsetstrokecolor{textcolor}%
\pgfsetfillcolor{textcolor}%
\pgftext[x=1.144444in,y=2.152174in,right,]{\color{textcolor}\sffamily\fontsize{20.000000}{24.000000}\selectfont \(\displaystyle {10}\)}%
\end{pgfscope}%
\begin{pgfscope}%
\pgfsetbuttcap%
\pgfsetroundjoin%
\definecolor{currentfill}{rgb}{0.000000,0.000000,0.000000}%
\pgfsetfillcolor{currentfill}%
\pgfsetlinewidth{0.501875pt}%
\definecolor{currentstroke}{rgb}{0.000000,0.000000,0.000000}%
\pgfsetstrokecolor{currentstroke}%
\pgfsetdash{}{0pt}%
\pgfsys@defobject{currentmarker}{\pgfqpoint{0.000000in}{0.000000in}}{\pgfqpoint{0.055556in}{0.000000in}}{%
\pgfpathmoveto{\pgfqpoint{0.000000in}{0.000000in}}%
\pgfpathlineto{\pgfqpoint{0.055556in}{0.000000in}}%
\pgfusepath{stroke,fill}%
}%
\begin{pgfscope}%
\pgfsys@transformshift{1.200000in}{2.986957in}%
\pgfsys@useobject{currentmarker}{}%
\end{pgfscope}%
\end{pgfscope}%
\begin{pgfscope}%
\pgfsetbuttcap%
\pgfsetroundjoin%
\definecolor{currentfill}{rgb}{0.000000,0.000000,0.000000}%
\pgfsetfillcolor{currentfill}%
\pgfsetlinewidth{0.501875pt}%
\definecolor{currentstroke}{rgb}{0.000000,0.000000,0.000000}%
\pgfsetstrokecolor{currentstroke}%
\pgfsetdash{}{0pt}%
\pgfsys@defobject{currentmarker}{\pgfqpoint{-0.055556in}{0.000000in}}{\pgfqpoint{-0.000000in}{0.000000in}}{%
\pgfpathmoveto{\pgfqpoint{-0.000000in}{0.000000in}}%
\pgfpathlineto{\pgfqpoint{-0.055556in}{0.000000in}}%
\pgfusepath{stroke,fill}%
}%
\begin{pgfscope}%
\pgfsys@transformshift{6.800000in}{2.986957in}%
\pgfsys@useobject{currentmarker}{}%
\end{pgfscope}%
\end{pgfscope}%
\begin{pgfscope}%
\definecolor{textcolor}{rgb}{0.000000,0.000000,0.000000}%
\pgfsetstrokecolor{textcolor}%
\pgfsetfillcolor{textcolor}%
\pgftext[x=1.144444in,y=2.986957in,right,]{\color{textcolor}\sffamily\fontsize{20.000000}{24.000000}\selectfont \(\displaystyle {20}\)}%
\end{pgfscope}%
\begin{pgfscope}%
\pgfsetbuttcap%
\pgfsetroundjoin%
\definecolor{currentfill}{rgb}{0.000000,0.000000,0.000000}%
\pgfsetfillcolor{currentfill}%
\pgfsetlinewidth{0.501875pt}%
\definecolor{currentstroke}{rgb}{0.000000,0.000000,0.000000}%
\pgfsetstrokecolor{currentstroke}%
\pgfsetdash{}{0pt}%
\pgfsys@defobject{currentmarker}{\pgfqpoint{0.000000in}{0.000000in}}{\pgfqpoint{0.055556in}{0.000000in}}{%
\pgfpathmoveto{\pgfqpoint{0.000000in}{0.000000in}}%
\pgfpathlineto{\pgfqpoint{0.055556in}{0.000000in}}%
\pgfusepath{stroke,fill}%
}%
\begin{pgfscope}%
\pgfsys@transformshift{1.200000in}{3.821739in}%
\pgfsys@useobject{currentmarker}{}%
\end{pgfscope}%
\end{pgfscope}%
\begin{pgfscope}%
\pgfsetbuttcap%
\pgfsetroundjoin%
\definecolor{currentfill}{rgb}{0.000000,0.000000,0.000000}%
\pgfsetfillcolor{currentfill}%
\pgfsetlinewidth{0.501875pt}%
\definecolor{currentstroke}{rgb}{0.000000,0.000000,0.000000}%
\pgfsetstrokecolor{currentstroke}%
\pgfsetdash{}{0pt}%
\pgfsys@defobject{currentmarker}{\pgfqpoint{-0.055556in}{0.000000in}}{\pgfqpoint{-0.000000in}{0.000000in}}{%
\pgfpathmoveto{\pgfqpoint{-0.000000in}{0.000000in}}%
\pgfpathlineto{\pgfqpoint{-0.055556in}{0.000000in}}%
\pgfusepath{stroke,fill}%
}%
\begin{pgfscope}%
\pgfsys@transformshift{6.800000in}{3.821739in}%
\pgfsys@useobject{currentmarker}{}%
\end{pgfscope}%
\end{pgfscope}%
\begin{pgfscope}%
\definecolor{textcolor}{rgb}{0.000000,0.000000,0.000000}%
\pgfsetstrokecolor{textcolor}%
\pgfsetfillcolor{textcolor}%
\pgftext[x=1.144444in,y=3.821739in,right,]{\color{textcolor}\sffamily\fontsize{20.000000}{24.000000}\selectfont \(\displaystyle {30}\)}%
\end{pgfscope}%
\begin{pgfscope}%
\pgfsetbuttcap%
\pgfsetroundjoin%
\definecolor{currentfill}{rgb}{0.000000,0.000000,0.000000}%
\pgfsetfillcolor{currentfill}%
\pgfsetlinewidth{0.501875pt}%
\definecolor{currentstroke}{rgb}{0.000000,0.000000,0.000000}%
\pgfsetstrokecolor{currentstroke}%
\pgfsetdash{}{0pt}%
\pgfsys@defobject{currentmarker}{\pgfqpoint{0.000000in}{0.000000in}}{\pgfqpoint{0.055556in}{0.000000in}}{%
\pgfpathmoveto{\pgfqpoint{0.000000in}{0.000000in}}%
\pgfpathlineto{\pgfqpoint{0.055556in}{0.000000in}}%
\pgfusepath{stroke,fill}%
}%
\begin{pgfscope}%
\pgfsys@transformshift{1.200000in}{4.656522in}%
\pgfsys@useobject{currentmarker}{}%
\end{pgfscope}%
\end{pgfscope}%
\begin{pgfscope}%
\pgfsetbuttcap%
\pgfsetroundjoin%
\definecolor{currentfill}{rgb}{0.000000,0.000000,0.000000}%
\pgfsetfillcolor{currentfill}%
\pgfsetlinewidth{0.501875pt}%
\definecolor{currentstroke}{rgb}{0.000000,0.000000,0.000000}%
\pgfsetstrokecolor{currentstroke}%
\pgfsetdash{}{0pt}%
\pgfsys@defobject{currentmarker}{\pgfqpoint{-0.055556in}{0.000000in}}{\pgfqpoint{-0.000000in}{0.000000in}}{%
\pgfpathmoveto{\pgfqpoint{-0.000000in}{0.000000in}}%
\pgfpathlineto{\pgfqpoint{-0.055556in}{0.000000in}}%
\pgfusepath{stroke,fill}%
}%
\begin{pgfscope}%
\pgfsys@transformshift{6.800000in}{4.656522in}%
\pgfsys@useobject{currentmarker}{}%
\end{pgfscope}%
\end{pgfscope}%
\begin{pgfscope}%
\definecolor{textcolor}{rgb}{0.000000,0.000000,0.000000}%
\pgfsetstrokecolor{textcolor}%
\pgfsetfillcolor{textcolor}%
\pgftext[x=1.144444in,y=4.656522in,right,]{\color{textcolor}\sffamily\fontsize{20.000000}{24.000000}\selectfont \(\displaystyle {40}\)}%
\end{pgfscope}%
\begin{pgfscope}%
\pgfsetbuttcap%
\pgfsetroundjoin%
\definecolor{currentfill}{rgb}{0.000000,0.000000,0.000000}%
\pgfsetfillcolor{currentfill}%
\pgfsetlinewidth{0.501875pt}%
\definecolor{currentstroke}{rgb}{0.000000,0.000000,0.000000}%
\pgfsetstrokecolor{currentstroke}%
\pgfsetdash{}{0pt}%
\pgfsys@defobject{currentmarker}{\pgfqpoint{0.000000in}{0.000000in}}{\pgfqpoint{0.055556in}{0.000000in}}{%
\pgfpathmoveto{\pgfqpoint{0.000000in}{0.000000in}}%
\pgfpathlineto{\pgfqpoint{0.055556in}{0.000000in}}%
\pgfusepath{stroke,fill}%
}%
\begin{pgfscope}%
\pgfsys@transformshift{1.200000in}{5.491304in}%
\pgfsys@useobject{currentmarker}{}%
\end{pgfscope}%
\end{pgfscope}%
\begin{pgfscope}%
\pgfsetbuttcap%
\pgfsetroundjoin%
\definecolor{currentfill}{rgb}{0.000000,0.000000,0.000000}%
\pgfsetfillcolor{currentfill}%
\pgfsetlinewidth{0.501875pt}%
\definecolor{currentstroke}{rgb}{0.000000,0.000000,0.000000}%
\pgfsetstrokecolor{currentstroke}%
\pgfsetdash{}{0pt}%
\pgfsys@defobject{currentmarker}{\pgfqpoint{-0.055556in}{0.000000in}}{\pgfqpoint{-0.000000in}{0.000000in}}{%
\pgfpathmoveto{\pgfqpoint{-0.000000in}{0.000000in}}%
\pgfpathlineto{\pgfqpoint{-0.055556in}{0.000000in}}%
\pgfusepath{stroke,fill}%
}%
\begin{pgfscope}%
\pgfsys@transformshift{6.800000in}{5.491304in}%
\pgfsys@useobject{currentmarker}{}%
\end{pgfscope}%
\end{pgfscope}%
\begin{pgfscope}%
\definecolor{textcolor}{rgb}{0.000000,0.000000,0.000000}%
\pgfsetstrokecolor{textcolor}%
\pgfsetfillcolor{textcolor}%
\pgftext[x=1.144444in,y=5.491304in,right,]{\color{textcolor}\sffamily\fontsize{20.000000}{24.000000}\selectfont \(\displaystyle {50}\)}%
\end{pgfscope}%
\begin{pgfscope}%
\definecolor{textcolor}{rgb}{0.000000,0.000000,0.000000}%
\pgfsetstrokecolor{textcolor}%
\pgfsetfillcolor{textcolor}%
\pgftext[x=0.810785in,y=3.300000in,,bottom,rotate=90.000000]{\color{textcolor}\sffamily\fontsize{20.000000}{24.000000}\selectfont \(\displaystyle \mathrm{Voltage}/\si{mV}\)}%
\end{pgfscope}%
\begin{pgfscope}%
\pgfsetbuttcap%
\pgfsetmiterjoin%
\definecolor{currentfill}{rgb}{1.000000,1.000000,1.000000}%
\pgfsetfillcolor{currentfill}%
\pgfsetlinewidth{1.003750pt}%
\definecolor{currentstroke}{rgb}{0.000000,0.000000,0.000000}%
\pgfsetstrokecolor{currentstroke}%
\pgfsetdash{}{0pt}%
\pgfpathmoveto{\pgfqpoint{4.066020in}{4.959484in}}%
\pgfpathlineto{\pgfqpoint{6.633333in}{4.959484in}}%
\pgfpathlineto{\pgfqpoint{6.633333in}{5.533333in}}%
\pgfpathlineto{\pgfqpoint{4.066020in}{5.533333in}}%
\pgfpathclose%
\pgfusepath{stroke,fill}%
\end{pgfscope}%
\begin{pgfscope}%
\pgfsetrectcap%
\pgfsetroundjoin%
\pgfsetlinewidth{2.007500pt}%
\definecolor{currentstroke}{rgb}{0.000000,0.000000,1.000000}%
\pgfsetstrokecolor{currentstroke}%
\pgfsetdash{}{0pt}%
\pgfpathmoveto{\pgfqpoint{4.299353in}{5.276697in}}%
\pgfpathlineto{\pgfqpoint{4.766020in}{5.276697in}}%
\pgfusepath{stroke}%
\end{pgfscope}%
\begin{pgfscope}%
\definecolor{textcolor}{rgb}{0.000000,0.000000,0.000000}%
\pgfsetstrokecolor{textcolor}%
\pgfsetfillcolor{textcolor}%
\pgftext[x=5.132687in,y=5.160031in,left,base]{\color{textcolor}\sffamily\fontsize{24.000000}{28.800000}\selectfont Waveform}%
\end{pgfscope}%
\end{pgfpicture}%
\makeatother%
\endgroup%
}
        \caption{\label{fig:pile} Pile-up in waveform}
    \end{subfigure}
\end{figure}

Naively, when handling PMT waveform we record the first $t_{H}$ according to $v_{th}$ and $Q$. One waveform is converted to a pair of numbers. More detailed information of the waveform (see figure~\ref{fig:tradi}) was lost. The new goal in this work is to extract information of all hits in 1 DAQ window including $t_{H}$ \& $q_{r}(t_{H})$ or $n_{r}(t_{H})$ (see figure~\ref{fig:new}). 

\begin{figure}[H]
    \begin{subfigure}{0.5\textwidth}
        \centering
        \scalebox{0.37}{%% Creator: Matplotlib, PGF backend
%%
%% To include the figure in your LaTeX document, write
%%   \input{<filename>.pgf}
%%
%% Make sure the required packages are loaded in your preamble
%%   \usepackage{pgf}
%%
%% and, on pdftex
%%   \usepackage[utf8]{inputenc}\DeclareUnicodeCharacter{2212}{-}
%%
%% or, on luatex and xetex
%%   \usepackage{unicode-math}
%%
%% Figures using additional raster images can only be included by \input if
%% they are in the same directory as the main LaTeX file. For loading figures
%% from other directories you can use the `import` package
%%   \usepackage{import}
%%
%% and then include the figures with
%%   \import{<path to file>}{<filename>.pgf}
%%
%% Matplotlib used the following preamble
%%
\begingroup%
\makeatletter%
\begin{pgfpicture}%
\pgfpathrectangle{\pgfpointorigin}{\pgfqpoint{8.000000in}{6.000000in}}%
\pgfusepath{use as bounding box, clip}%
\begin{pgfscope}%
\pgfsetbuttcap%
\pgfsetmiterjoin%
\definecolor{currentfill}{rgb}{1.000000,1.000000,1.000000}%
\pgfsetfillcolor{currentfill}%
\pgfsetlinewidth{0.000000pt}%
\definecolor{currentstroke}{rgb}{1.000000,1.000000,1.000000}%
\pgfsetstrokecolor{currentstroke}%
\pgfsetdash{}{0pt}%
\pgfpathmoveto{\pgfqpoint{0.000000in}{0.000000in}}%
\pgfpathlineto{\pgfqpoint{8.000000in}{0.000000in}}%
\pgfpathlineto{\pgfqpoint{8.000000in}{6.000000in}}%
\pgfpathlineto{\pgfqpoint{0.000000in}{6.000000in}}%
\pgfpathclose%
\pgfusepath{fill}%
\end{pgfscope}%
\begin{pgfscope}%
\pgfsetbuttcap%
\pgfsetmiterjoin%
\definecolor{currentfill}{rgb}{1.000000,1.000000,1.000000}%
\pgfsetfillcolor{currentfill}%
\pgfsetlinewidth{0.000000pt}%
\definecolor{currentstroke}{rgb}{0.000000,0.000000,0.000000}%
\pgfsetstrokecolor{currentstroke}%
\pgfsetstrokeopacity{0.000000}%
\pgfsetdash{}{0pt}%
\pgfpathmoveto{\pgfqpoint{1.200000in}{0.900000in}}%
\pgfpathlineto{\pgfqpoint{6.800000in}{0.900000in}}%
\pgfpathlineto{\pgfqpoint{6.800000in}{5.700000in}}%
\pgfpathlineto{\pgfqpoint{1.200000in}{5.700000in}}%
\pgfpathclose%
\pgfusepath{fill}%
\end{pgfscope}%
\begin{pgfscope}%
\pgfpathrectangle{\pgfqpoint{1.200000in}{0.900000in}}{\pgfqpoint{5.600000in}{4.800000in}}%
\pgfusepath{clip}%
\pgfsetbuttcap%
\pgfsetroundjoin%
\pgfsetlinewidth{2.007500pt}%
\definecolor{currentstroke}{rgb}{0.000000,0.500000,0.000000}%
\pgfsetstrokecolor{currentstroke}%
\pgfsetdash{}{0pt}%
\pgfpathmoveto{\pgfqpoint{1.190000in}{1.450881in}}%
\pgfpathlineto{\pgfqpoint{6.810000in}{1.450881in}}%
\pgfusepath{stroke}%
\end{pgfscope}%
\begin{pgfscope}%
\pgfpathrectangle{\pgfqpoint{1.200000in}{0.900000in}}{\pgfqpoint{5.600000in}{4.800000in}}%
\pgfusepath{clip}%
\pgfsetrectcap%
\pgfsetroundjoin%
\pgfsetlinewidth{2.007500pt}%
\definecolor{currentstroke}{rgb}{0.000000,0.000000,1.000000}%
\pgfsetstrokecolor{currentstroke}%
\pgfsetdash{}{0pt}%
\pgfpathmoveto{\pgfqpoint{1.190000in}{1.259759in}}%
\pgfpathlineto{\pgfqpoint{1.200000in}{1.372183in}}%
\pgfpathlineto{\pgfqpoint{1.214000in}{1.293486in}}%
\pgfpathlineto{\pgfqpoint{1.228000in}{1.293486in}}%
\pgfpathlineto{\pgfqpoint{1.242000in}{1.214789in}}%
\pgfpathlineto{\pgfqpoint{1.256000in}{1.214789in}}%
\pgfpathlineto{\pgfqpoint{1.270000in}{1.372183in}}%
\pgfpathlineto{\pgfqpoint{1.284000in}{1.293486in}}%
\pgfpathlineto{\pgfqpoint{1.298000in}{1.293486in}}%
\pgfpathlineto{\pgfqpoint{1.312000in}{1.372183in}}%
\pgfpathlineto{\pgfqpoint{1.326000in}{1.372183in}}%
\pgfpathlineto{\pgfqpoint{1.340000in}{1.214789in}}%
\pgfpathlineto{\pgfqpoint{1.354000in}{1.293486in}}%
\pgfpathlineto{\pgfqpoint{1.368000in}{1.529578in}}%
\pgfpathlineto{\pgfqpoint{1.382000in}{1.214789in}}%
\pgfpathlineto{\pgfqpoint{1.396000in}{1.293486in}}%
\pgfpathlineto{\pgfqpoint{1.410000in}{1.214789in}}%
\pgfpathlineto{\pgfqpoint{1.424000in}{1.372183in}}%
\pgfpathlineto{\pgfqpoint{1.438000in}{1.293486in}}%
\pgfpathlineto{\pgfqpoint{1.452000in}{1.529578in}}%
\pgfpathlineto{\pgfqpoint{1.466000in}{1.293486in}}%
\pgfpathlineto{\pgfqpoint{1.480000in}{1.372183in}}%
\pgfpathlineto{\pgfqpoint{1.494000in}{1.372183in}}%
\pgfpathlineto{\pgfqpoint{1.508000in}{1.293486in}}%
\pgfpathlineto{\pgfqpoint{1.536000in}{1.293486in}}%
\pgfpathlineto{\pgfqpoint{1.550000in}{1.214789in}}%
\pgfpathlineto{\pgfqpoint{1.578000in}{1.214789in}}%
\pgfpathlineto{\pgfqpoint{1.592000in}{1.293486in}}%
\pgfpathlineto{\pgfqpoint{1.606000in}{1.214789in}}%
\pgfpathlineto{\pgfqpoint{1.620000in}{1.293486in}}%
\pgfpathlineto{\pgfqpoint{1.648000in}{1.293486in}}%
\pgfpathlineto{\pgfqpoint{1.662000in}{1.057394in}}%
\pgfpathlineto{\pgfqpoint{1.676000in}{1.372183in}}%
\pgfpathlineto{\pgfqpoint{1.690000in}{1.372183in}}%
\pgfpathlineto{\pgfqpoint{1.704000in}{1.293486in}}%
\pgfpathlineto{\pgfqpoint{1.718000in}{1.293486in}}%
\pgfpathlineto{\pgfqpoint{1.732000in}{1.214789in}}%
\pgfpathlineto{\pgfqpoint{1.746000in}{1.450881in}}%
\pgfpathlineto{\pgfqpoint{1.760000in}{1.214789in}}%
\pgfpathlineto{\pgfqpoint{1.774000in}{1.293486in}}%
\pgfpathlineto{\pgfqpoint{1.788000in}{1.214789in}}%
\pgfpathlineto{\pgfqpoint{1.802000in}{1.214789in}}%
\pgfpathlineto{\pgfqpoint{1.816000in}{1.293486in}}%
\pgfpathlineto{\pgfqpoint{1.830000in}{1.214789in}}%
\pgfpathlineto{\pgfqpoint{1.858000in}{1.214789in}}%
\pgfpathlineto{\pgfqpoint{1.886000in}{1.372183in}}%
\pgfpathlineto{\pgfqpoint{1.900000in}{1.293486in}}%
\pgfpathlineto{\pgfqpoint{1.914000in}{1.372183in}}%
\pgfpathlineto{\pgfqpoint{1.928000in}{1.214789in}}%
\pgfpathlineto{\pgfqpoint{1.942000in}{1.372183in}}%
\pgfpathlineto{\pgfqpoint{1.956000in}{1.372183in}}%
\pgfpathlineto{\pgfqpoint{1.970000in}{1.450881in}}%
\pgfpathlineto{\pgfqpoint{1.984000in}{1.214789in}}%
\pgfpathlineto{\pgfqpoint{1.998000in}{1.293486in}}%
\pgfpathlineto{\pgfqpoint{2.012000in}{1.293486in}}%
\pgfpathlineto{\pgfqpoint{2.026000in}{1.214789in}}%
\pgfpathlineto{\pgfqpoint{2.040000in}{1.293486in}}%
\pgfpathlineto{\pgfqpoint{2.054000in}{1.214789in}}%
\pgfpathlineto{\pgfqpoint{2.068000in}{1.372183in}}%
\pgfpathlineto{\pgfqpoint{2.082000in}{1.214789in}}%
\pgfpathlineto{\pgfqpoint{2.096000in}{1.293486in}}%
\pgfpathlineto{\pgfqpoint{2.110000in}{1.214789in}}%
\pgfpathlineto{\pgfqpoint{2.124000in}{1.372183in}}%
\pgfpathlineto{\pgfqpoint{2.152000in}{1.214789in}}%
\pgfpathlineto{\pgfqpoint{2.166000in}{1.372183in}}%
\pgfpathlineto{\pgfqpoint{2.180000in}{1.214789in}}%
\pgfpathlineto{\pgfqpoint{2.194000in}{1.372183in}}%
\pgfpathlineto{\pgfqpoint{2.208000in}{1.293486in}}%
\pgfpathlineto{\pgfqpoint{2.222000in}{1.293486in}}%
\pgfpathlineto{\pgfqpoint{2.236000in}{1.372183in}}%
\pgfpathlineto{\pgfqpoint{2.250000in}{1.765670in}}%
\pgfpathlineto{\pgfqpoint{2.264000in}{2.080458in}}%
\pgfpathlineto{\pgfqpoint{2.292000in}{2.552642in}}%
\pgfpathlineto{\pgfqpoint{2.306000in}{2.631339in}}%
\pgfpathlineto{\pgfqpoint{2.320000in}{2.552642in}}%
\pgfpathlineto{\pgfqpoint{2.334000in}{2.552642in}}%
\pgfpathlineto{\pgfqpoint{2.348000in}{2.316550in}}%
\pgfpathlineto{\pgfqpoint{2.362000in}{2.237853in}}%
\pgfpathlineto{\pgfqpoint{2.376000in}{2.080458in}}%
\pgfpathlineto{\pgfqpoint{2.390000in}{2.237853in}}%
\pgfpathlineto{\pgfqpoint{2.404000in}{2.552642in}}%
\pgfpathlineto{\pgfqpoint{2.418000in}{2.946128in}}%
\pgfpathlineto{\pgfqpoint{2.446000in}{3.103523in}}%
\pgfpathlineto{\pgfqpoint{2.460000in}{3.260917in}}%
\pgfpathlineto{\pgfqpoint{2.474000in}{3.024825in}}%
\pgfpathlineto{\pgfqpoint{2.502000in}{2.710036in}}%
\pgfpathlineto{\pgfqpoint{2.516000in}{2.631339in}}%
\pgfpathlineto{\pgfqpoint{2.544000in}{2.316550in}}%
\pgfpathlineto{\pgfqpoint{2.558000in}{1.923064in}}%
\pgfpathlineto{\pgfqpoint{2.600000in}{1.923064in}}%
\pgfpathlineto{\pgfqpoint{2.614000in}{1.765670in}}%
\pgfpathlineto{\pgfqpoint{2.628000in}{2.080458in}}%
\pgfpathlineto{\pgfqpoint{2.642000in}{2.316550in}}%
\pgfpathlineto{\pgfqpoint{2.670000in}{2.631339in}}%
\pgfpathlineto{\pgfqpoint{2.684000in}{2.710036in}}%
\pgfpathlineto{\pgfqpoint{2.698000in}{2.552642in}}%
\pgfpathlineto{\pgfqpoint{2.726000in}{2.395247in}}%
\pgfpathlineto{\pgfqpoint{2.740000in}{2.080458in}}%
\pgfpathlineto{\pgfqpoint{2.754000in}{2.080458in}}%
\pgfpathlineto{\pgfqpoint{2.768000in}{2.001761in}}%
\pgfpathlineto{\pgfqpoint{2.782000in}{1.765670in}}%
\pgfpathlineto{\pgfqpoint{2.810000in}{1.765670in}}%
\pgfpathlineto{\pgfqpoint{2.824000in}{1.608275in}}%
\pgfpathlineto{\pgfqpoint{2.852000in}{1.450881in}}%
\pgfpathlineto{\pgfqpoint{2.894000in}{1.450881in}}%
\pgfpathlineto{\pgfqpoint{2.908000in}{1.372183in}}%
\pgfpathlineto{\pgfqpoint{2.922000in}{1.372183in}}%
\pgfpathlineto{\pgfqpoint{2.936000in}{1.450881in}}%
\pgfpathlineto{\pgfqpoint{2.950000in}{1.293486in}}%
\pgfpathlineto{\pgfqpoint{2.964000in}{1.372183in}}%
\pgfpathlineto{\pgfqpoint{2.978000in}{1.293486in}}%
\pgfpathlineto{\pgfqpoint{2.992000in}{1.450881in}}%
\pgfpathlineto{\pgfqpoint{3.006000in}{1.450881in}}%
\pgfpathlineto{\pgfqpoint{3.020000in}{1.686972in}}%
\pgfpathlineto{\pgfqpoint{3.034000in}{2.237853in}}%
\pgfpathlineto{\pgfqpoint{3.062000in}{3.024825in}}%
\pgfpathlineto{\pgfqpoint{3.076000in}{3.024825in}}%
\pgfpathlineto{\pgfqpoint{3.090000in}{3.103523in}}%
\pgfpathlineto{\pgfqpoint{3.118000in}{2.946128in}}%
\pgfpathlineto{\pgfqpoint{3.146000in}{2.631339in}}%
\pgfpathlineto{\pgfqpoint{3.160000in}{2.552642in}}%
\pgfpathlineto{\pgfqpoint{3.174000in}{2.237853in}}%
\pgfpathlineto{\pgfqpoint{3.188000in}{2.159156in}}%
\pgfpathlineto{\pgfqpoint{3.202000in}{1.923064in}}%
\pgfpathlineto{\pgfqpoint{3.216000in}{1.844367in}}%
\pgfpathlineto{\pgfqpoint{3.230000in}{1.844367in}}%
\pgfpathlineto{\pgfqpoint{3.244000in}{1.608275in}}%
\pgfpathlineto{\pgfqpoint{3.258000in}{1.608275in}}%
\pgfpathlineto{\pgfqpoint{3.272000in}{1.686972in}}%
\pgfpathlineto{\pgfqpoint{3.286000in}{1.844367in}}%
\pgfpathlineto{\pgfqpoint{3.300000in}{2.159156in}}%
\pgfpathlineto{\pgfqpoint{3.314000in}{2.159156in}}%
\pgfpathlineto{\pgfqpoint{3.328000in}{2.237853in}}%
\pgfpathlineto{\pgfqpoint{3.356000in}{2.237853in}}%
\pgfpathlineto{\pgfqpoint{3.370000in}{2.080458in}}%
\pgfpathlineto{\pgfqpoint{3.398000in}{2.080458in}}%
\pgfpathlineto{\pgfqpoint{3.440000in}{1.608275in}}%
\pgfpathlineto{\pgfqpoint{3.454000in}{1.529578in}}%
\pgfpathlineto{\pgfqpoint{3.482000in}{1.529578in}}%
\pgfpathlineto{\pgfqpoint{3.496000in}{1.608275in}}%
\pgfpathlineto{\pgfqpoint{3.524000in}{1.450881in}}%
\pgfpathlineto{\pgfqpoint{3.538000in}{1.293486in}}%
\pgfpathlineto{\pgfqpoint{3.552000in}{1.450881in}}%
\pgfpathlineto{\pgfqpoint{3.566000in}{1.293486in}}%
\pgfpathlineto{\pgfqpoint{3.580000in}{1.450881in}}%
\pgfpathlineto{\pgfqpoint{3.594000in}{1.293486in}}%
\pgfpathlineto{\pgfqpoint{3.608000in}{1.529578in}}%
\pgfpathlineto{\pgfqpoint{3.622000in}{1.372183in}}%
\pgfpathlineto{\pgfqpoint{3.636000in}{1.372183in}}%
\pgfpathlineto{\pgfqpoint{3.650000in}{1.293486in}}%
\pgfpathlineto{\pgfqpoint{3.664000in}{1.293486in}}%
\pgfpathlineto{\pgfqpoint{3.678000in}{1.214789in}}%
\pgfpathlineto{\pgfqpoint{3.692000in}{1.372183in}}%
\pgfpathlineto{\pgfqpoint{3.706000in}{1.293486in}}%
\pgfpathlineto{\pgfqpoint{3.720000in}{1.372183in}}%
\pgfpathlineto{\pgfqpoint{3.748000in}{1.372183in}}%
\pgfpathlineto{\pgfqpoint{3.762000in}{1.214789in}}%
\pgfpathlineto{\pgfqpoint{3.776000in}{1.214789in}}%
\pgfpathlineto{\pgfqpoint{3.790000in}{1.372183in}}%
\pgfpathlineto{\pgfqpoint{3.804000in}{1.214789in}}%
\pgfpathlineto{\pgfqpoint{3.818000in}{1.214789in}}%
\pgfpathlineto{\pgfqpoint{3.832000in}{1.450881in}}%
\pgfpathlineto{\pgfqpoint{3.846000in}{1.214789in}}%
\pgfpathlineto{\pgfqpoint{3.860000in}{1.214789in}}%
\pgfpathlineto{\pgfqpoint{3.874000in}{1.293486in}}%
\pgfpathlineto{\pgfqpoint{3.888000in}{1.450881in}}%
\pgfpathlineto{\pgfqpoint{3.902000in}{1.136092in}}%
\pgfpathlineto{\pgfqpoint{3.916000in}{1.372183in}}%
\pgfpathlineto{\pgfqpoint{3.930000in}{1.372183in}}%
\pgfpathlineto{\pgfqpoint{3.944000in}{1.214789in}}%
\pgfpathlineto{\pgfqpoint{3.958000in}{1.372183in}}%
\pgfpathlineto{\pgfqpoint{3.972000in}{1.293486in}}%
\pgfpathlineto{\pgfqpoint{3.986000in}{1.450881in}}%
\pgfpathlineto{\pgfqpoint{4.000000in}{1.214789in}}%
\pgfpathlineto{\pgfqpoint{4.014000in}{1.214789in}}%
\pgfpathlineto{\pgfqpoint{4.028000in}{1.293486in}}%
\pgfpathlineto{\pgfqpoint{4.042000in}{1.293486in}}%
\pgfpathlineto{\pgfqpoint{4.056000in}{1.214789in}}%
\pgfpathlineto{\pgfqpoint{4.084000in}{1.214789in}}%
\pgfpathlineto{\pgfqpoint{4.098000in}{1.450881in}}%
\pgfpathlineto{\pgfqpoint{4.112000in}{1.214789in}}%
\pgfpathlineto{\pgfqpoint{4.126000in}{1.214789in}}%
\pgfpathlineto{\pgfqpoint{4.140000in}{1.293486in}}%
\pgfpathlineto{\pgfqpoint{4.154000in}{1.136092in}}%
\pgfpathlineto{\pgfqpoint{4.168000in}{1.293486in}}%
\pgfpathlineto{\pgfqpoint{4.182000in}{1.293486in}}%
\pgfpathlineto{\pgfqpoint{4.196000in}{1.372183in}}%
\pgfpathlineto{\pgfqpoint{4.210000in}{1.214789in}}%
\pgfpathlineto{\pgfqpoint{4.252000in}{1.450881in}}%
\pgfpathlineto{\pgfqpoint{4.266000in}{1.214789in}}%
\pgfpathlineto{\pgfqpoint{4.280000in}{1.136092in}}%
\pgfpathlineto{\pgfqpoint{4.294000in}{1.214789in}}%
\pgfpathlineto{\pgfqpoint{4.308000in}{1.372183in}}%
\pgfpathlineto{\pgfqpoint{4.322000in}{1.214789in}}%
\pgfpathlineto{\pgfqpoint{4.336000in}{1.450881in}}%
\pgfpathlineto{\pgfqpoint{4.350000in}{1.293486in}}%
\pgfpathlineto{\pgfqpoint{4.364000in}{1.293486in}}%
\pgfpathlineto{\pgfqpoint{4.378000in}{1.214789in}}%
\pgfpathlineto{\pgfqpoint{4.392000in}{1.372183in}}%
\pgfpathlineto{\pgfqpoint{4.406000in}{1.214789in}}%
\pgfpathlineto{\pgfqpoint{4.420000in}{1.293486in}}%
\pgfpathlineto{\pgfqpoint{4.448000in}{1.293486in}}%
\pgfpathlineto{\pgfqpoint{4.462000in}{1.214789in}}%
\pgfpathlineto{\pgfqpoint{4.476000in}{1.372183in}}%
\pgfpathlineto{\pgfqpoint{4.490000in}{1.372183in}}%
\pgfpathlineto{\pgfqpoint{4.504000in}{1.214789in}}%
\pgfpathlineto{\pgfqpoint{4.518000in}{1.293486in}}%
\pgfpathlineto{\pgfqpoint{4.532000in}{1.214789in}}%
\pgfpathlineto{\pgfqpoint{4.546000in}{1.450881in}}%
\pgfpathlineto{\pgfqpoint{4.588000in}{1.214789in}}%
\pgfpathlineto{\pgfqpoint{4.602000in}{1.372183in}}%
\pgfpathlineto{\pgfqpoint{4.616000in}{1.450881in}}%
\pgfpathlineto{\pgfqpoint{4.630000in}{1.372183in}}%
\pgfpathlineto{\pgfqpoint{4.644000in}{1.214789in}}%
\pgfpathlineto{\pgfqpoint{4.658000in}{1.372183in}}%
\pgfpathlineto{\pgfqpoint{4.672000in}{1.372183in}}%
\pgfpathlineto{\pgfqpoint{4.686000in}{1.214789in}}%
\pgfpathlineto{\pgfqpoint{4.700000in}{1.372183in}}%
\pgfpathlineto{\pgfqpoint{4.714000in}{1.293486in}}%
\pgfpathlineto{\pgfqpoint{4.728000in}{1.372183in}}%
\pgfpathlineto{\pgfqpoint{4.742000in}{1.293486in}}%
\pgfpathlineto{\pgfqpoint{4.756000in}{1.057394in}}%
\pgfpathlineto{\pgfqpoint{4.770000in}{1.293486in}}%
\pgfpathlineto{\pgfqpoint{4.784000in}{1.372183in}}%
\pgfpathlineto{\pgfqpoint{4.798000in}{1.293486in}}%
\pgfpathlineto{\pgfqpoint{4.812000in}{1.372183in}}%
\pgfpathlineto{\pgfqpoint{4.826000in}{1.372183in}}%
\pgfpathlineto{\pgfqpoint{4.868000in}{1.136092in}}%
\pgfpathlineto{\pgfqpoint{4.882000in}{1.293486in}}%
\pgfpathlineto{\pgfqpoint{4.896000in}{1.372183in}}%
\pgfpathlineto{\pgfqpoint{4.910000in}{1.214789in}}%
\pgfpathlineto{\pgfqpoint{4.924000in}{1.214789in}}%
\pgfpathlineto{\pgfqpoint{4.938000in}{1.450881in}}%
\pgfpathlineto{\pgfqpoint{4.952000in}{1.450881in}}%
\pgfpathlineto{\pgfqpoint{4.966000in}{1.214789in}}%
\pgfpathlineto{\pgfqpoint{4.980000in}{1.293486in}}%
\pgfpathlineto{\pgfqpoint{4.994000in}{1.293486in}}%
\pgfpathlineto{\pgfqpoint{5.022000in}{1.450881in}}%
\pgfpathlineto{\pgfqpoint{5.036000in}{1.293486in}}%
\pgfpathlineto{\pgfqpoint{5.064000in}{1.450881in}}%
\pgfpathlineto{\pgfqpoint{5.078000in}{1.214789in}}%
\pgfpathlineto{\pgfqpoint{5.092000in}{1.214789in}}%
\pgfpathlineto{\pgfqpoint{5.106000in}{1.136092in}}%
\pgfpathlineto{\pgfqpoint{5.120000in}{1.214789in}}%
\pgfpathlineto{\pgfqpoint{5.134000in}{1.214789in}}%
\pgfpathlineto{\pgfqpoint{5.148000in}{1.372183in}}%
\pgfpathlineto{\pgfqpoint{5.162000in}{1.214789in}}%
\pgfpathlineto{\pgfqpoint{5.176000in}{1.372183in}}%
\pgfpathlineto{\pgfqpoint{5.190000in}{1.214789in}}%
\pgfpathlineto{\pgfqpoint{5.204000in}{1.372183in}}%
\pgfpathlineto{\pgfqpoint{5.218000in}{1.293486in}}%
\pgfpathlineto{\pgfqpoint{5.232000in}{1.450881in}}%
\pgfpathlineto{\pgfqpoint{5.246000in}{1.372183in}}%
\pgfpathlineto{\pgfqpoint{5.260000in}{1.214789in}}%
\pgfpathlineto{\pgfqpoint{5.274000in}{1.214789in}}%
\pgfpathlineto{\pgfqpoint{5.288000in}{1.293486in}}%
\pgfpathlineto{\pgfqpoint{5.302000in}{1.136092in}}%
\pgfpathlineto{\pgfqpoint{5.316000in}{1.372183in}}%
\pgfpathlineto{\pgfqpoint{5.344000in}{1.214789in}}%
\pgfpathlineto{\pgfqpoint{5.372000in}{1.372183in}}%
\pgfpathlineto{\pgfqpoint{5.386000in}{1.372183in}}%
\pgfpathlineto{\pgfqpoint{5.400000in}{1.214789in}}%
\pgfpathlineto{\pgfqpoint{5.414000in}{1.214789in}}%
\pgfpathlineto{\pgfqpoint{5.428000in}{1.293486in}}%
\pgfpathlineto{\pgfqpoint{5.442000in}{1.293486in}}%
\pgfpathlineto{\pgfqpoint{5.456000in}{1.372183in}}%
\pgfpathlineto{\pgfqpoint{5.470000in}{1.214789in}}%
\pgfpathlineto{\pgfqpoint{5.484000in}{1.450881in}}%
\pgfpathlineto{\pgfqpoint{5.498000in}{1.214789in}}%
\pgfpathlineto{\pgfqpoint{5.512000in}{1.214789in}}%
\pgfpathlineto{\pgfqpoint{5.526000in}{1.293486in}}%
\pgfpathlineto{\pgfqpoint{5.554000in}{1.293486in}}%
\pgfpathlineto{\pgfqpoint{5.568000in}{1.214789in}}%
\pgfpathlineto{\pgfqpoint{5.582000in}{1.450881in}}%
\pgfpathlineto{\pgfqpoint{5.596000in}{1.293486in}}%
\pgfpathlineto{\pgfqpoint{5.610000in}{1.372183in}}%
\pgfpathlineto{\pgfqpoint{5.624000in}{1.293486in}}%
\pgfpathlineto{\pgfqpoint{5.638000in}{1.372183in}}%
\pgfpathlineto{\pgfqpoint{5.652000in}{1.293486in}}%
\pgfpathlineto{\pgfqpoint{5.666000in}{1.372183in}}%
\pgfpathlineto{\pgfqpoint{5.680000in}{1.214789in}}%
\pgfpathlineto{\pgfqpoint{5.694000in}{1.372183in}}%
\pgfpathlineto{\pgfqpoint{5.708000in}{1.293486in}}%
\pgfpathlineto{\pgfqpoint{5.722000in}{1.372183in}}%
\pgfpathlineto{\pgfqpoint{5.736000in}{1.136092in}}%
\pgfpathlineto{\pgfqpoint{5.750000in}{1.372183in}}%
\pgfpathlineto{\pgfqpoint{5.764000in}{1.136092in}}%
\pgfpathlineto{\pgfqpoint{5.778000in}{1.372183in}}%
\pgfpathlineto{\pgfqpoint{5.806000in}{1.214789in}}%
\pgfpathlineto{\pgfqpoint{5.820000in}{1.293486in}}%
\pgfpathlineto{\pgfqpoint{5.834000in}{1.136092in}}%
\pgfpathlineto{\pgfqpoint{5.848000in}{1.293486in}}%
\pgfpathlineto{\pgfqpoint{5.904000in}{1.293486in}}%
\pgfpathlineto{\pgfqpoint{5.918000in}{1.372183in}}%
\pgfpathlineto{\pgfqpoint{5.932000in}{1.293486in}}%
\pgfpathlineto{\pgfqpoint{5.974000in}{1.293486in}}%
\pgfpathlineto{\pgfqpoint{5.988000in}{1.529578in}}%
\pgfpathlineto{\pgfqpoint{6.002000in}{1.293486in}}%
\pgfpathlineto{\pgfqpoint{6.016000in}{1.214789in}}%
\pgfpathlineto{\pgfqpoint{6.030000in}{1.293486in}}%
\pgfpathlineto{\pgfqpoint{6.058000in}{1.293486in}}%
\pgfpathlineto{\pgfqpoint{6.072000in}{1.372183in}}%
\pgfpathlineto{\pgfqpoint{6.086000in}{1.293486in}}%
\pgfpathlineto{\pgfqpoint{6.128000in}{1.293486in}}%
\pgfpathlineto{\pgfqpoint{6.142000in}{1.214789in}}%
\pgfpathlineto{\pgfqpoint{6.170000in}{1.372183in}}%
\pgfpathlineto{\pgfqpoint{6.184000in}{1.214789in}}%
\pgfpathlineto{\pgfqpoint{6.198000in}{1.372183in}}%
\pgfpathlineto{\pgfqpoint{6.212000in}{1.214789in}}%
\pgfpathlineto{\pgfqpoint{6.240000in}{1.372183in}}%
\pgfpathlineto{\pgfqpoint{6.254000in}{1.372183in}}%
\pgfpathlineto{\pgfqpoint{6.268000in}{1.057394in}}%
\pgfpathlineto{\pgfqpoint{6.282000in}{1.372183in}}%
\pgfpathlineto{\pgfqpoint{6.296000in}{1.293486in}}%
\pgfpathlineto{\pgfqpoint{6.310000in}{1.372183in}}%
\pgfpathlineto{\pgfqpoint{6.338000in}{1.214789in}}%
\pgfpathlineto{\pgfqpoint{6.352000in}{1.214789in}}%
\pgfpathlineto{\pgfqpoint{6.366000in}{1.372183in}}%
\pgfpathlineto{\pgfqpoint{6.380000in}{1.293486in}}%
\pgfpathlineto{\pgfqpoint{6.422000in}{1.293486in}}%
\pgfpathlineto{\pgfqpoint{6.436000in}{1.372183in}}%
\pgfpathlineto{\pgfqpoint{6.450000in}{1.214789in}}%
\pgfpathlineto{\pgfqpoint{6.464000in}{1.293486in}}%
\pgfpathlineto{\pgfqpoint{6.478000in}{1.293486in}}%
\pgfpathlineto{\pgfqpoint{6.492000in}{1.214789in}}%
\pgfpathlineto{\pgfqpoint{6.506000in}{1.450881in}}%
\pgfpathlineto{\pgfqpoint{6.520000in}{1.450881in}}%
\pgfpathlineto{\pgfqpoint{6.534000in}{1.293486in}}%
\pgfpathlineto{\pgfqpoint{6.548000in}{1.293486in}}%
\pgfpathlineto{\pgfqpoint{6.562000in}{1.214789in}}%
\pgfpathlineto{\pgfqpoint{6.576000in}{1.293486in}}%
\pgfpathlineto{\pgfqpoint{6.590000in}{1.214789in}}%
\pgfpathlineto{\pgfqpoint{6.604000in}{1.372183in}}%
\pgfpathlineto{\pgfqpoint{6.632000in}{1.372183in}}%
\pgfpathlineto{\pgfqpoint{6.646000in}{1.293486in}}%
\pgfpathlineto{\pgfqpoint{6.688000in}{1.293486in}}%
\pgfpathlineto{\pgfqpoint{6.702000in}{1.372183in}}%
\pgfpathlineto{\pgfqpoint{6.716000in}{1.214789in}}%
\pgfpathlineto{\pgfqpoint{6.730000in}{1.293486in}}%
\pgfpathlineto{\pgfqpoint{6.744000in}{1.293486in}}%
\pgfpathlineto{\pgfqpoint{6.758000in}{1.214789in}}%
\pgfpathlineto{\pgfqpoint{6.786000in}{1.372183in}}%
\pgfpathlineto{\pgfqpoint{6.800000in}{1.372183in}}%
\pgfpathlineto{\pgfqpoint{6.810000in}{1.315971in}}%
\pgfpathlineto{\pgfqpoint{6.810000in}{1.315971in}}%
\pgfusepath{stroke}%
\end{pgfscope}%
\begin{pgfscope}%
\pgfsetrectcap%
\pgfsetmiterjoin%
\pgfsetlinewidth{1.003750pt}%
\definecolor{currentstroke}{rgb}{0.000000,0.000000,0.000000}%
\pgfsetstrokecolor{currentstroke}%
\pgfsetdash{}{0pt}%
\pgfpathmoveto{\pgfqpoint{1.200000in}{0.900000in}}%
\pgfpathlineto{\pgfqpoint{1.200000in}{5.700000in}}%
\pgfusepath{stroke}%
\end{pgfscope}%
\begin{pgfscope}%
\pgfsetrectcap%
\pgfsetmiterjoin%
\pgfsetlinewidth{1.003750pt}%
\definecolor{currentstroke}{rgb}{0.000000,0.000000,0.000000}%
\pgfsetstrokecolor{currentstroke}%
\pgfsetdash{}{0pt}%
\pgfpathmoveto{\pgfqpoint{6.800000in}{0.900000in}}%
\pgfpathlineto{\pgfqpoint{6.800000in}{5.700000in}}%
\pgfusepath{stroke}%
\end{pgfscope}%
\begin{pgfscope}%
\pgfsetrectcap%
\pgfsetmiterjoin%
\pgfsetlinewidth{1.003750pt}%
\definecolor{currentstroke}{rgb}{0.000000,0.000000,0.000000}%
\pgfsetstrokecolor{currentstroke}%
\pgfsetdash{}{0pt}%
\pgfpathmoveto{\pgfqpoint{1.200000in}{0.900000in}}%
\pgfpathlineto{\pgfqpoint{6.800000in}{0.900000in}}%
\pgfusepath{stroke}%
\end{pgfscope}%
\begin{pgfscope}%
\pgfsetrectcap%
\pgfsetmiterjoin%
\pgfsetlinewidth{1.003750pt}%
\definecolor{currentstroke}{rgb}{0.000000,0.000000,0.000000}%
\pgfsetstrokecolor{currentstroke}%
\pgfsetdash{}{0pt}%
\pgfpathmoveto{\pgfqpoint{1.200000in}{5.700000in}}%
\pgfpathlineto{\pgfqpoint{6.800000in}{5.700000in}}%
\pgfusepath{stroke}%
\end{pgfscope}%
\begin{pgfscope}%
\pgfsetbuttcap%
\pgfsetroundjoin%
\definecolor{currentfill}{rgb}{0.000000,0.000000,0.000000}%
\pgfsetfillcolor{currentfill}%
\pgfsetlinewidth{0.501875pt}%
\definecolor{currentstroke}{rgb}{0.000000,0.000000,0.000000}%
\pgfsetstrokecolor{currentstroke}%
\pgfsetdash{}{0pt}%
\pgfsys@defobject{currentmarker}{\pgfqpoint{0.000000in}{0.000000in}}{\pgfqpoint{0.000000in}{0.055556in}}{%
\pgfpathmoveto{\pgfqpoint{0.000000in}{0.000000in}}%
\pgfpathlineto{\pgfqpoint{0.000000in}{0.055556in}}%
\pgfusepath{stroke,fill}%
}%
\begin{pgfscope}%
\pgfsys@transformshift{1.200000in}{0.900000in}%
\pgfsys@useobject{currentmarker}{}%
\end{pgfscope}%
\end{pgfscope}%
\begin{pgfscope}%
\pgfsetbuttcap%
\pgfsetroundjoin%
\definecolor{currentfill}{rgb}{0.000000,0.000000,0.000000}%
\pgfsetfillcolor{currentfill}%
\pgfsetlinewidth{0.501875pt}%
\definecolor{currentstroke}{rgb}{0.000000,0.000000,0.000000}%
\pgfsetstrokecolor{currentstroke}%
\pgfsetdash{}{0pt}%
\pgfsys@defobject{currentmarker}{\pgfqpoint{0.000000in}{-0.055556in}}{\pgfqpoint{0.000000in}{0.000000in}}{%
\pgfpathmoveto{\pgfqpoint{0.000000in}{0.000000in}}%
\pgfpathlineto{\pgfqpoint{0.000000in}{-0.055556in}}%
\pgfusepath{stroke,fill}%
}%
\begin{pgfscope}%
\pgfsys@transformshift{1.200000in}{5.700000in}%
\pgfsys@useobject{currentmarker}{}%
\end{pgfscope}%
\end{pgfscope}%
\begin{pgfscope}%
\definecolor{textcolor}{rgb}{0.000000,0.000000,0.000000}%
\pgfsetstrokecolor{textcolor}%
\pgfsetfillcolor{textcolor}%
\pgftext[x=1.200000in,y=0.844444in,,top]{\color{textcolor}\sffamily\fontsize{20.000000}{24.000000}\selectfont \(\displaystyle {300}\)}%
\end{pgfscope}%
\begin{pgfscope}%
\pgfsetbuttcap%
\pgfsetroundjoin%
\definecolor{currentfill}{rgb}{0.000000,0.000000,0.000000}%
\pgfsetfillcolor{currentfill}%
\pgfsetlinewidth{0.501875pt}%
\definecolor{currentstroke}{rgb}{0.000000,0.000000,0.000000}%
\pgfsetstrokecolor{currentstroke}%
\pgfsetdash{}{0pt}%
\pgfsys@defobject{currentmarker}{\pgfqpoint{0.000000in}{0.000000in}}{\pgfqpoint{0.000000in}{0.055556in}}{%
\pgfpathmoveto{\pgfqpoint{0.000000in}{0.000000in}}%
\pgfpathlineto{\pgfqpoint{0.000000in}{0.055556in}}%
\pgfusepath{stroke,fill}%
}%
\begin{pgfscope}%
\pgfsys@transformshift{1.900000in}{0.900000in}%
\pgfsys@useobject{currentmarker}{}%
\end{pgfscope}%
\end{pgfscope}%
\begin{pgfscope}%
\pgfsetbuttcap%
\pgfsetroundjoin%
\definecolor{currentfill}{rgb}{0.000000,0.000000,0.000000}%
\pgfsetfillcolor{currentfill}%
\pgfsetlinewidth{0.501875pt}%
\definecolor{currentstroke}{rgb}{0.000000,0.000000,0.000000}%
\pgfsetstrokecolor{currentstroke}%
\pgfsetdash{}{0pt}%
\pgfsys@defobject{currentmarker}{\pgfqpoint{0.000000in}{-0.055556in}}{\pgfqpoint{0.000000in}{0.000000in}}{%
\pgfpathmoveto{\pgfqpoint{0.000000in}{0.000000in}}%
\pgfpathlineto{\pgfqpoint{0.000000in}{-0.055556in}}%
\pgfusepath{stroke,fill}%
}%
\begin{pgfscope}%
\pgfsys@transformshift{1.900000in}{5.700000in}%
\pgfsys@useobject{currentmarker}{}%
\end{pgfscope}%
\end{pgfscope}%
\begin{pgfscope}%
\definecolor{textcolor}{rgb}{0.000000,0.000000,0.000000}%
\pgfsetstrokecolor{textcolor}%
\pgfsetfillcolor{textcolor}%
\pgftext[x=1.900000in,y=0.844444in,,top]{\color{textcolor}\sffamily\fontsize{20.000000}{24.000000}\selectfont \(\displaystyle {350}\)}%
\end{pgfscope}%
\begin{pgfscope}%
\pgfsetbuttcap%
\pgfsetroundjoin%
\definecolor{currentfill}{rgb}{0.000000,0.000000,0.000000}%
\pgfsetfillcolor{currentfill}%
\pgfsetlinewidth{0.501875pt}%
\definecolor{currentstroke}{rgb}{0.000000,0.000000,0.000000}%
\pgfsetstrokecolor{currentstroke}%
\pgfsetdash{}{0pt}%
\pgfsys@defobject{currentmarker}{\pgfqpoint{0.000000in}{0.000000in}}{\pgfqpoint{0.000000in}{0.055556in}}{%
\pgfpathmoveto{\pgfqpoint{0.000000in}{0.000000in}}%
\pgfpathlineto{\pgfqpoint{0.000000in}{0.055556in}}%
\pgfusepath{stroke,fill}%
}%
\begin{pgfscope}%
\pgfsys@transformshift{2.600000in}{0.900000in}%
\pgfsys@useobject{currentmarker}{}%
\end{pgfscope}%
\end{pgfscope}%
\begin{pgfscope}%
\pgfsetbuttcap%
\pgfsetroundjoin%
\definecolor{currentfill}{rgb}{0.000000,0.000000,0.000000}%
\pgfsetfillcolor{currentfill}%
\pgfsetlinewidth{0.501875pt}%
\definecolor{currentstroke}{rgb}{0.000000,0.000000,0.000000}%
\pgfsetstrokecolor{currentstroke}%
\pgfsetdash{}{0pt}%
\pgfsys@defobject{currentmarker}{\pgfqpoint{0.000000in}{-0.055556in}}{\pgfqpoint{0.000000in}{0.000000in}}{%
\pgfpathmoveto{\pgfqpoint{0.000000in}{0.000000in}}%
\pgfpathlineto{\pgfqpoint{0.000000in}{-0.055556in}}%
\pgfusepath{stroke,fill}%
}%
\begin{pgfscope}%
\pgfsys@transformshift{2.600000in}{5.700000in}%
\pgfsys@useobject{currentmarker}{}%
\end{pgfscope}%
\end{pgfscope}%
\begin{pgfscope}%
\definecolor{textcolor}{rgb}{0.000000,0.000000,0.000000}%
\pgfsetstrokecolor{textcolor}%
\pgfsetfillcolor{textcolor}%
\pgftext[x=2.600000in,y=0.844444in,,top]{\color{textcolor}\sffamily\fontsize{20.000000}{24.000000}\selectfont \(\displaystyle {400}\)}%
\end{pgfscope}%
\begin{pgfscope}%
\pgfsetbuttcap%
\pgfsetroundjoin%
\definecolor{currentfill}{rgb}{0.000000,0.000000,0.000000}%
\pgfsetfillcolor{currentfill}%
\pgfsetlinewidth{0.501875pt}%
\definecolor{currentstroke}{rgb}{0.000000,0.000000,0.000000}%
\pgfsetstrokecolor{currentstroke}%
\pgfsetdash{}{0pt}%
\pgfsys@defobject{currentmarker}{\pgfqpoint{0.000000in}{0.000000in}}{\pgfqpoint{0.000000in}{0.055556in}}{%
\pgfpathmoveto{\pgfqpoint{0.000000in}{0.000000in}}%
\pgfpathlineto{\pgfqpoint{0.000000in}{0.055556in}}%
\pgfusepath{stroke,fill}%
}%
\begin{pgfscope}%
\pgfsys@transformshift{3.300000in}{0.900000in}%
\pgfsys@useobject{currentmarker}{}%
\end{pgfscope}%
\end{pgfscope}%
\begin{pgfscope}%
\pgfsetbuttcap%
\pgfsetroundjoin%
\definecolor{currentfill}{rgb}{0.000000,0.000000,0.000000}%
\pgfsetfillcolor{currentfill}%
\pgfsetlinewidth{0.501875pt}%
\definecolor{currentstroke}{rgb}{0.000000,0.000000,0.000000}%
\pgfsetstrokecolor{currentstroke}%
\pgfsetdash{}{0pt}%
\pgfsys@defobject{currentmarker}{\pgfqpoint{0.000000in}{-0.055556in}}{\pgfqpoint{0.000000in}{0.000000in}}{%
\pgfpathmoveto{\pgfqpoint{0.000000in}{0.000000in}}%
\pgfpathlineto{\pgfqpoint{0.000000in}{-0.055556in}}%
\pgfusepath{stroke,fill}%
}%
\begin{pgfscope}%
\pgfsys@transformshift{3.300000in}{5.700000in}%
\pgfsys@useobject{currentmarker}{}%
\end{pgfscope}%
\end{pgfscope}%
\begin{pgfscope}%
\definecolor{textcolor}{rgb}{0.000000,0.000000,0.000000}%
\pgfsetstrokecolor{textcolor}%
\pgfsetfillcolor{textcolor}%
\pgftext[x=3.300000in,y=0.844444in,,top]{\color{textcolor}\sffamily\fontsize{20.000000}{24.000000}\selectfont \(\displaystyle {450}\)}%
\end{pgfscope}%
\begin{pgfscope}%
\pgfsetbuttcap%
\pgfsetroundjoin%
\definecolor{currentfill}{rgb}{0.000000,0.000000,0.000000}%
\pgfsetfillcolor{currentfill}%
\pgfsetlinewidth{0.501875pt}%
\definecolor{currentstroke}{rgb}{0.000000,0.000000,0.000000}%
\pgfsetstrokecolor{currentstroke}%
\pgfsetdash{}{0pt}%
\pgfsys@defobject{currentmarker}{\pgfqpoint{0.000000in}{0.000000in}}{\pgfqpoint{0.000000in}{0.055556in}}{%
\pgfpathmoveto{\pgfqpoint{0.000000in}{0.000000in}}%
\pgfpathlineto{\pgfqpoint{0.000000in}{0.055556in}}%
\pgfusepath{stroke,fill}%
}%
\begin{pgfscope}%
\pgfsys@transformshift{4.000000in}{0.900000in}%
\pgfsys@useobject{currentmarker}{}%
\end{pgfscope}%
\end{pgfscope}%
\begin{pgfscope}%
\pgfsetbuttcap%
\pgfsetroundjoin%
\definecolor{currentfill}{rgb}{0.000000,0.000000,0.000000}%
\pgfsetfillcolor{currentfill}%
\pgfsetlinewidth{0.501875pt}%
\definecolor{currentstroke}{rgb}{0.000000,0.000000,0.000000}%
\pgfsetstrokecolor{currentstroke}%
\pgfsetdash{}{0pt}%
\pgfsys@defobject{currentmarker}{\pgfqpoint{0.000000in}{-0.055556in}}{\pgfqpoint{0.000000in}{0.000000in}}{%
\pgfpathmoveto{\pgfqpoint{0.000000in}{0.000000in}}%
\pgfpathlineto{\pgfqpoint{0.000000in}{-0.055556in}}%
\pgfusepath{stroke,fill}%
}%
\begin{pgfscope}%
\pgfsys@transformshift{4.000000in}{5.700000in}%
\pgfsys@useobject{currentmarker}{}%
\end{pgfscope}%
\end{pgfscope}%
\begin{pgfscope}%
\definecolor{textcolor}{rgb}{0.000000,0.000000,0.000000}%
\pgfsetstrokecolor{textcolor}%
\pgfsetfillcolor{textcolor}%
\pgftext[x=4.000000in,y=0.844444in,,top]{\color{textcolor}\sffamily\fontsize{20.000000}{24.000000}\selectfont \(\displaystyle {500}\)}%
\end{pgfscope}%
\begin{pgfscope}%
\pgfsetbuttcap%
\pgfsetroundjoin%
\definecolor{currentfill}{rgb}{0.000000,0.000000,0.000000}%
\pgfsetfillcolor{currentfill}%
\pgfsetlinewidth{0.501875pt}%
\definecolor{currentstroke}{rgb}{0.000000,0.000000,0.000000}%
\pgfsetstrokecolor{currentstroke}%
\pgfsetdash{}{0pt}%
\pgfsys@defobject{currentmarker}{\pgfqpoint{0.000000in}{0.000000in}}{\pgfqpoint{0.000000in}{0.055556in}}{%
\pgfpathmoveto{\pgfqpoint{0.000000in}{0.000000in}}%
\pgfpathlineto{\pgfqpoint{0.000000in}{0.055556in}}%
\pgfusepath{stroke,fill}%
}%
\begin{pgfscope}%
\pgfsys@transformshift{4.700000in}{0.900000in}%
\pgfsys@useobject{currentmarker}{}%
\end{pgfscope}%
\end{pgfscope}%
\begin{pgfscope}%
\pgfsetbuttcap%
\pgfsetroundjoin%
\definecolor{currentfill}{rgb}{0.000000,0.000000,0.000000}%
\pgfsetfillcolor{currentfill}%
\pgfsetlinewidth{0.501875pt}%
\definecolor{currentstroke}{rgb}{0.000000,0.000000,0.000000}%
\pgfsetstrokecolor{currentstroke}%
\pgfsetdash{}{0pt}%
\pgfsys@defobject{currentmarker}{\pgfqpoint{0.000000in}{-0.055556in}}{\pgfqpoint{0.000000in}{0.000000in}}{%
\pgfpathmoveto{\pgfqpoint{0.000000in}{0.000000in}}%
\pgfpathlineto{\pgfqpoint{0.000000in}{-0.055556in}}%
\pgfusepath{stroke,fill}%
}%
\begin{pgfscope}%
\pgfsys@transformshift{4.700000in}{5.700000in}%
\pgfsys@useobject{currentmarker}{}%
\end{pgfscope}%
\end{pgfscope}%
\begin{pgfscope}%
\definecolor{textcolor}{rgb}{0.000000,0.000000,0.000000}%
\pgfsetstrokecolor{textcolor}%
\pgfsetfillcolor{textcolor}%
\pgftext[x=4.700000in,y=0.844444in,,top]{\color{textcolor}\sffamily\fontsize{20.000000}{24.000000}\selectfont \(\displaystyle {550}\)}%
\end{pgfscope}%
\begin{pgfscope}%
\pgfsetbuttcap%
\pgfsetroundjoin%
\definecolor{currentfill}{rgb}{0.000000,0.000000,0.000000}%
\pgfsetfillcolor{currentfill}%
\pgfsetlinewidth{0.501875pt}%
\definecolor{currentstroke}{rgb}{0.000000,0.000000,0.000000}%
\pgfsetstrokecolor{currentstroke}%
\pgfsetdash{}{0pt}%
\pgfsys@defobject{currentmarker}{\pgfqpoint{0.000000in}{0.000000in}}{\pgfqpoint{0.000000in}{0.055556in}}{%
\pgfpathmoveto{\pgfqpoint{0.000000in}{0.000000in}}%
\pgfpathlineto{\pgfqpoint{0.000000in}{0.055556in}}%
\pgfusepath{stroke,fill}%
}%
\begin{pgfscope}%
\pgfsys@transformshift{5.400000in}{0.900000in}%
\pgfsys@useobject{currentmarker}{}%
\end{pgfscope}%
\end{pgfscope}%
\begin{pgfscope}%
\pgfsetbuttcap%
\pgfsetroundjoin%
\definecolor{currentfill}{rgb}{0.000000,0.000000,0.000000}%
\pgfsetfillcolor{currentfill}%
\pgfsetlinewidth{0.501875pt}%
\definecolor{currentstroke}{rgb}{0.000000,0.000000,0.000000}%
\pgfsetstrokecolor{currentstroke}%
\pgfsetdash{}{0pt}%
\pgfsys@defobject{currentmarker}{\pgfqpoint{0.000000in}{-0.055556in}}{\pgfqpoint{0.000000in}{0.000000in}}{%
\pgfpathmoveto{\pgfqpoint{0.000000in}{0.000000in}}%
\pgfpathlineto{\pgfqpoint{0.000000in}{-0.055556in}}%
\pgfusepath{stroke,fill}%
}%
\begin{pgfscope}%
\pgfsys@transformshift{5.400000in}{5.700000in}%
\pgfsys@useobject{currentmarker}{}%
\end{pgfscope}%
\end{pgfscope}%
\begin{pgfscope}%
\definecolor{textcolor}{rgb}{0.000000,0.000000,0.000000}%
\pgfsetstrokecolor{textcolor}%
\pgfsetfillcolor{textcolor}%
\pgftext[x=5.400000in,y=0.844444in,,top]{\color{textcolor}\sffamily\fontsize{20.000000}{24.000000}\selectfont \(\displaystyle {600}\)}%
\end{pgfscope}%
\begin{pgfscope}%
\pgfsetbuttcap%
\pgfsetroundjoin%
\definecolor{currentfill}{rgb}{0.000000,0.000000,0.000000}%
\pgfsetfillcolor{currentfill}%
\pgfsetlinewidth{0.501875pt}%
\definecolor{currentstroke}{rgb}{0.000000,0.000000,0.000000}%
\pgfsetstrokecolor{currentstroke}%
\pgfsetdash{}{0pt}%
\pgfsys@defobject{currentmarker}{\pgfqpoint{0.000000in}{0.000000in}}{\pgfqpoint{0.000000in}{0.055556in}}{%
\pgfpathmoveto{\pgfqpoint{0.000000in}{0.000000in}}%
\pgfpathlineto{\pgfqpoint{0.000000in}{0.055556in}}%
\pgfusepath{stroke,fill}%
}%
\begin{pgfscope}%
\pgfsys@transformshift{6.100000in}{0.900000in}%
\pgfsys@useobject{currentmarker}{}%
\end{pgfscope}%
\end{pgfscope}%
\begin{pgfscope}%
\pgfsetbuttcap%
\pgfsetroundjoin%
\definecolor{currentfill}{rgb}{0.000000,0.000000,0.000000}%
\pgfsetfillcolor{currentfill}%
\pgfsetlinewidth{0.501875pt}%
\definecolor{currentstroke}{rgb}{0.000000,0.000000,0.000000}%
\pgfsetstrokecolor{currentstroke}%
\pgfsetdash{}{0pt}%
\pgfsys@defobject{currentmarker}{\pgfqpoint{0.000000in}{-0.055556in}}{\pgfqpoint{0.000000in}{0.000000in}}{%
\pgfpathmoveto{\pgfqpoint{0.000000in}{0.000000in}}%
\pgfpathlineto{\pgfqpoint{0.000000in}{-0.055556in}}%
\pgfusepath{stroke,fill}%
}%
\begin{pgfscope}%
\pgfsys@transformshift{6.100000in}{5.700000in}%
\pgfsys@useobject{currentmarker}{}%
\end{pgfscope}%
\end{pgfscope}%
\begin{pgfscope}%
\definecolor{textcolor}{rgb}{0.000000,0.000000,0.000000}%
\pgfsetstrokecolor{textcolor}%
\pgfsetfillcolor{textcolor}%
\pgftext[x=6.100000in,y=0.844444in,,top]{\color{textcolor}\sffamily\fontsize{20.000000}{24.000000}\selectfont \(\displaystyle {650}\)}%
\end{pgfscope}%
\begin{pgfscope}%
\pgfsetbuttcap%
\pgfsetroundjoin%
\definecolor{currentfill}{rgb}{0.000000,0.000000,0.000000}%
\pgfsetfillcolor{currentfill}%
\pgfsetlinewidth{0.501875pt}%
\definecolor{currentstroke}{rgb}{0.000000,0.000000,0.000000}%
\pgfsetstrokecolor{currentstroke}%
\pgfsetdash{}{0pt}%
\pgfsys@defobject{currentmarker}{\pgfqpoint{0.000000in}{0.000000in}}{\pgfqpoint{0.000000in}{0.055556in}}{%
\pgfpathmoveto{\pgfqpoint{0.000000in}{0.000000in}}%
\pgfpathlineto{\pgfqpoint{0.000000in}{0.055556in}}%
\pgfusepath{stroke,fill}%
}%
\begin{pgfscope}%
\pgfsys@transformshift{6.800000in}{0.900000in}%
\pgfsys@useobject{currentmarker}{}%
\end{pgfscope}%
\end{pgfscope}%
\begin{pgfscope}%
\pgfsetbuttcap%
\pgfsetroundjoin%
\definecolor{currentfill}{rgb}{0.000000,0.000000,0.000000}%
\pgfsetfillcolor{currentfill}%
\pgfsetlinewidth{0.501875pt}%
\definecolor{currentstroke}{rgb}{0.000000,0.000000,0.000000}%
\pgfsetstrokecolor{currentstroke}%
\pgfsetdash{}{0pt}%
\pgfsys@defobject{currentmarker}{\pgfqpoint{0.000000in}{-0.055556in}}{\pgfqpoint{0.000000in}{0.000000in}}{%
\pgfpathmoveto{\pgfqpoint{0.000000in}{0.000000in}}%
\pgfpathlineto{\pgfqpoint{0.000000in}{-0.055556in}}%
\pgfusepath{stroke,fill}%
}%
\begin{pgfscope}%
\pgfsys@transformshift{6.800000in}{5.700000in}%
\pgfsys@useobject{currentmarker}{}%
\end{pgfscope}%
\end{pgfscope}%
\begin{pgfscope}%
\definecolor{textcolor}{rgb}{0.000000,0.000000,0.000000}%
\pgfsetstrokecolor{textcolor}%
\pgfsetfillcolor{textcolor}%
\pgftext[x=6.800000in,y=0.844444in,,top]{\color{textcolor}\sffamily\fontsize{20.000000}{24.000000}\selectfont \(\displaystyle {700}\)}%
\end{pgfscope}%
\begin{pgfscope}%
\definecolor{textcolor}{rgb}{0.000000,0.000000,0.000000}%
\pgfsetstrokecolor{textcolor}%
\pgfsetfillcolor{textcolor}%
\pgftext[x=4.000000in,y=0.518932in,,top]{\color{textcolor}\sffamily\fontsize{20.000000}{24.000000}\selectfont \(\displaystyle t/\mathrm{ns}\)}%
\end{pgfscope}%
\begin{pgfscope}%
\pgfsetbuttcap%
\pgfsetroundjoin%
\definecolor{currentfill}{rgb}{0.000000,0.000000,0.000000}%
\pgfsetfillcolor{currentfill}%
\pgfsetlinewidth{0.501875pt}%
\definecolor{currentstroke}{rgb}{0.000000,0.000000,0.000000}%
\pgfsetstrokecolor{currentstroke}%
\pgfsetdash{}{0pt}%
\pgfsys@defobject{currentmarker}{\pgfqpoint{0.000000in}{0.000000in}}{\pgfqpoint{0.055556in}{0.000000in}}{%
\pgfpathmoveto{\pgfqpoint{0.000000in}{0.000000in}}%
\pgfpathlineto{\pgfqpoint{0.055556in}{0.000000in}}%
\pgfusepath{stroke,fill}%
}%
\begin{pgfscope}%
\pgfsys@transformshift{1.200000in}{1.293486in}%
\pgfsys@useobject{currentmarker}{}%
\end{pgfscope}%
\end{pgfscope}%
\begin{pgfscope}%
\definecolor{textcolor}{rgb}{0.000000,0.000000,0.000000}%
\pgfsetstrokecolor{textcolor}%
\pgfsetfillcolor{textcolor}%
\pgftext[x=1.144444in,y=1.293486in,right,]{\color{textcolor}\sffamily\fontsize{20.000000}{24.000000}\selectfont \(\displaystyle {0}\)}%
\end{pgfscope}%
\begin{pgfscope}%
\pgfsetbuttcap%
\pgfsetroundjoin%
\definecolor{currentfill}{rgb}{0.000000,0.000000,0.000000}%
\pgfsetfillcolor{currentfill}%
\pgfsetlinewidth{0.501875pt}%
\definecolor{currentstroke}{rgb}{0.000000,0.000000,0.000000}%
\pgfsetstrokecolor{currentstroke}%
\pgfsetdash{}{0pt}%
\pgfsys@defobject{currentmarker}{\pgfqpoint{0.000000in}{0.000000in}}{\pgfqpoint{0.055556in}{0.000000in}}{%
\pgfpathmoveto{\pgfqpoint{0.000000in}{0.000000in}}%
\pgfpathlineto{\pgfqpoint{0.055556in}{0.000000in}}%
\pgfusepath{stroke,fill}%
}%
\begin{pgfscope}%
\pgfsys@transformshift{1.200000in}{2.080458in}%
\pgfsys@useobject{currentmarker}{}%
\end{pgfscope}%
\end{pgfscope}%
\begin{pgfscope}%
\definecolor{textcolor}{rgb}{0.000000,0.000000,0.000000}%
\pgfsetstrokecolor{textcolor}%
\pgfsetfillcolor{textcolor}%
\pgftext[x=1.144444in,y=2.080458in,right,]{\color{textcolor}\sffamily\fontsize{20.000000}{24.000000}\selectfont \(\displaystyle {10}\)}%
\end{pgfscope}%
\begin{pgfscope}%
\pgfsetbuttcap%
\pgfsetroundjoin%
\definecolor{currentfill}{rgb}{0.000000,0.000000,0.000000}%
\pgfsetfillcolor{currentfill}%
\pgfsetlinewidth{0.501875pt}%
\definecolor{currentstroke}{rgb}{0.000000,0.000000,0.000000}%
\pgfsetstrokecolor{currentstroke}%
\pgfsetdash{}{0pt}%
\pgfsys@defobject{currentmarker}{\pgfqpoint{0.000000in}{0.000000in}}{\pgfqpoint{0.055556in}{0.000000in}}{%
\pgfpathmoveto{\pgfqpoint{0.000000in}{0.000000in}}%
\pgfpathlineto{\pgfqpoint{0.055556in}{0.000000in}}%
\pgfusepath{stroke,fill}%
}%
\begin{pgfscope}%
\pgfsys@transformshift{1.200000in}{2.867431in}%
\pgfsys@useobject{currentmarker}{}%
\end{pgfscope}%
\end{pgfscope}%
\begin{pgfscope}%
\definecolor{textcolor}{rgb}{0.000000,0.000000,0.000000}%
\pgfsetstrokecolor{textcolor}%
\pgfsetfillcolor{textcolor}%
\pgftext[x=1.144444in,y=2.867431in,right,]{\color{textcolor}\sffamily\fontsize{20.000000}{24.000000}\selectfont \(\displaystyle {20}\)}%
\end{pgfscope}%
\begin{pgfscope}%
\pgfsetbuttcap%
\pgfsetroundjoin%
\definecolor{currentfill}{rgb}{0.000000,0.000000,0.000000}%
\pgfsetfillcolor{currentfill}%
\pgfsetlinewidth{0.501875pt}%
\definecolor{currentstroke}{rgb}{0.000000,0.000000,0.000000}%
\pgfsetstrokecolor{currentstroke}%
\pgfsetdash{}{0pt}%
\pgfsys@defobject{currentmarker}{\pgfqpoint{0.000000in}{0.000000in}}{\pgfqpoint{0.055556in}{0.000000in}}{%
\pgfpathmoveto{\pgfqpoint{0.000000in}{0.000000in}}%
\pgfpathlineto{\pgfqpoint{0.055556in}{0.000000in}}%
\pgfusepath{stroke,fill}%
}%
\begin{pgfscope}%
\pgfsys@transformshift{1.200000in}{3.654403in}%
\pgfsys@useobject{currentmarker}{}%
\end{pgfscope}%
\end{pgfscope}%
\begin{pgfscope}%
\definecolor{textcolor}{rgb}{0.000000,0.000000,0.000000}%
\pgfsetstrokecolor{textcolor}%
\pgfsetfillcolor{textcolor}%
\pgftext[x=1.144444in,y=3.654403in,right,]{\color{textcolor}\sffamily\fontsize{20.000000}{24.000000}\selectfont \(\displaystyle {30}\)}%
\end{pgfscope}%
\begin{pgfscope}%
\pgfsetbuttcap%
\pgfsetroundjoin%
\definecolor{currentfill}{rgb}{0.000000,0.000000,0.000000}%
\pgfsetfillcolor{currentfill}%
\pgfsetlinewidth{0.501875pt}%
\definecolor{currentstroke}{rgb}{0.000000,0.000000,0.000000}%
\pgfsetstrokecolor{currentstroke}%
\pgfsetdash{}{0pt}%
\pgfsys@defobject{currentmarker}{\pgfqpoint{0.000000in}{0.000000in}}{\pgfqpoint{0.055556in}{0.000000in}}{%
\pgfpathmoveto{\pgfqpoint{0.000000in}{0.000000in}}%
\pgfpathlineto{\pgfqpoint{0.055556in}{0.000000in}}%
\pgfusepath{stroke,fill}%
}%
\begin{pgfscope}%
\pgfsys@transformshift{1.200000in}{4.441375in}%
\pgfsys@useobject{currentmarker}{}%
\end{pgfscope}%
\end{pgfscope}%
\begin{pgfscope}%
\definecolor{textcolor}{rgb}{0.000000,0.000000,0.000000}%
\pgfsetstrokecolor{textcolor}%
\pgfsetfillcolor{textcolor}%
\pgftext[x=1.144444in,y=4.441375in,right,]{\color{textcolor}\sffamily\fontsize{20.000000}{24.000000}\selectfont \(\displaystyle {40}\)}%
\end{pgfscope}%
\begin{pgfscope}%
\pgfsetbuttcap%
\pgfsetroundjoin%
\definecolor{currentfill}{rgb}{0.000000,0.000000,0.000000}%
\pgfsetfillcolor{currentfill}%
\pgfsetlinewidth{0.501875pt}%
\definecolor{currentstroke}{rgb}{0.000000,0.000000,0.000000}%
\pgfsetstrokecolor{currentstroke}%
\pgfsetdash{}{0pt}%
\pgfsys@defobject{currentmarker}{\pgfqpoint{0.000000in}{0.000000in}}{\pgfqpoint{0.055556in}{0.000000in}}{%
\pgfpathmoveto{\pgfqpoint{0.000000in}{0.000000in}}%
\pgfpathlineto{\pgfqpoint{0.055556in}{0.000000in}}%
\pgfusepath{stroke,fill}%
}%
\begin{pgfscope}%
\pgfsys@transformshift{1.200000in}{5.228348in}%
\pgfsys@useobject{currentmarker}{}%
\end{pgfscope}%
\end{pgfscope}%
\begin{pgfscope}%
\definecolor{textcolor}{rgb}{0.000000,0.000000,0.000000}%
\pgfsetstrokecolor{textcolor}%
\pgfsetfillcolor{textcolor}%
\pgftext[x=1.144444in,y=5.228348in,right,]{\color{textcolor}\sffamily\fontsize{20.000000}{24.000000}\selectfont \(\displaystyle {50}\)}%
\end{pgfscope}%
\begin{pgfscope}%
\definecolor{textcolor}{rgb}{0.000000,0.000000,0.000000}%
\pgfsetstrokecolor{textcolor}%
\pgfsetfillcolor{textcolor}%
\pgftext[x=0.810785in,y=3.300000in,,bottom,rotate=90.000000]{\color{textcolor}\sffamily\fontsize{20.000000}{24.000000}\selectfont \(\displaystyle Voltage/\mathrm{mV}\)}%
\end{pgfscope}%
\begin{pgfscope}%
\pgfpathrectangle{\pgfqpoint{1.200000in}{0.900000in}}{\pgfqpoint{5.600000in}{4.800000in}}%
\pgfusepath{clip}%
\pgfsetbuttcap%
\pgfsetroundjoin%
\pgfsetlinewidth{2.007500pt}%
\definecolor{currentstroke}{rgb}{1.000000,0.000000,0.000000}%
\pgfsetstrokecolor{currentstroke}%
\pgfsetdash{}{0pt}%
\pgfpathmoveto{\pgfqpoint{2.198626in}{1.293486in}}%
\pgfpathlineto{\pgfqpoint{2.198626in}{4.808206in}}%
\pgfusepath{stroke}%
\end{pgfscope}%
\begin{pgfscope}%
\pgfsetrectcap%
\pgfsetmiterjoin%
\pgfsetlinewidth{1.003750pt}%
\definecolor{currentstroke}{rgb}{0.000000,0.000000,0.000000}%
\pgfsetstrokecolor{currentstroke}%
\pgfsetdash{}{0pt}%
\pgfpathmoveto{\pgfqpoint{1.200000in}{0.900000in}}%
\pgfpathlineto{\pgfqpoint{1.200000in}{5.700000in}}%
\pgfusepath{stroke}%
\end{pgfscope}%
\begin{pgfscope}%
\pgfsetrectcap%
\pgfsetmiterjoin%
\pgfsetlinewidth{1.003750pt}%
\definecolor{currentstroke}{rgb}{0.000000,0.000000,0.000000}%
\pgfsetstrokecolor{currentstroke}%
\pgfsetdash{}{0pt}%
\pgfpathmoveto{\pgfqpoint{6.800000in}{0.900000in}}%
\pgfpathlineto{\pgfqpoint{6.800000in}{5.700000in}}%
\pgfusepath{stroke}%
\end{pgfscope}%
\begin{pgfscope}%
\pgfsetrectcap%
\pgfsetmiterjoin%
\pgfsetlinewidth{1.003750pt}%
\definecolor{currentstroke}{rgb}{0.000000,0.000000,0.000000}%
\pgfsetstrokecolor{currentstroke}%
\pgfsetdash{}{0pt}%
\pgfpathmoveto{\pgfqpoint{1.200000in}{0.900000in}}%
\pgfpathlineto{\pgfqpoint{6.800000in}{0.900000in}}%
\pgfusepath{stroke}%
\end{pgfscope}%
\begin{pgfscope}%
\pgfsetrectcap%
\pgfsetmiterjoin%
\pgfsetlinewidth{1.003750pt}%
\definecolor{currentstroke}{rgb}{0.000000,0.000000,0.000000}%
\pgfsetstrokecolor{currentstroke}%
\pgfsetdash{}{0pt}%
\pgfpathmoveto{\pgfqpoint{1.200000in}{5.700000in}}%
\pgfpathlineto{\pgfqpoint{6.800000in}{5.700000in}}%
\pgfusepath{stroke}%
\end{pgfscope}%
\begin{pgfscope}%
\pgfsetbuttcap%
\pgfsetroundjoin%
\definecolor{currentfill}{rgb}{0.000000,0.000000,0.000000}%
\pgfsetfillcolor{currentfill}%
\pgfsetlinewidth{0.501875pt}%
\definecolor{currentstroke}{rgb}{0.000000,0.000000,0.000000}%
\pgfsetstrokecolor{currentstroke}%
\pgfsetdash{}{0pt}%
\pgfsys@defobject{currentmarker}{\pgfqpoint{-0.055556in}{0.000000in}}{\pgfqpoint{-0.000000in}{0.000000in}}{%
\pgfpathmoveto{\pgfqpoint{-0.000000in}{0.000000in}}%
\pgfpathlineto{\pgfqpoint{-0.055556in}{0.000000in}}%
\pgfusepath{stroke,fill}%
}%
\begin{pgfscope}%
\pgfsys@transformshift{6.800000in}{1.293486in}%
\pgfsys@useobject{currentmarker}{}%
\end{pgfscope}%
\end{pgfscope}%
\begin{pgfscope}%
\definecolor{textcolor}{rgb}{0.000000,0.000000,0.000000}%
\pgfsetstrokecolor{textcolor}%
\pgfsetfillcolor{textcolor}%
\pgftext[x=6.855556in,y=1.293486in,left,]{\color{textcolor}\sffamily\fontsize{20.000000}{24.000000}\selectfont \(\displaystyle {0}\)}%
\end{pgfscope}%
\begin{pgfscope}%
\pgfsetbuttcap%
\pgfsetroundjoin%
\definecolor{currentfill}{rgb}{0.000000,0.000000,0.000000}%
\pgfsetfillcolor{currentfill}%
\pgfsetlinewidth{0.501875pt}%
\definecolor{currentstroke}{rgb}{0.000000,0.000000,0.000000}%
\pgfsetstrokecolor{currentstroke}%
\pgfsetdash{}{0pt}%
\pgfsys@defobject{currentmarker}{\pgfqpoint{-0.055556in}{0.000000in}}{\pgfqpoint{-0.000000in}{0.000000in}}{%
\pgfpathmoveto{\pgfqpoint{-0.000000in}{0.000000in}}%
\pgfpathlineto{\pgfqpoint{-0.055556in}{0.000000in}}%
\pgfusepath{stroke,fill}%
}%
\begin{pgfscope}%
\pgfsys@transformshift{6.800000in}{1.992933in}%
\pgfsys@useobject{currentmarker}{}%
\end{pgfscope}%
\end{pgfscope}%
\begin{pgfscope}%
\definecolor{textcolor}{rgb}{0.000000,0.000000,0.000000}%
\pgfsetstrokecolor{textcolor}%
\pgfsetfillcolor{textcolor}%
\pgftext[x=6.855556in,y=1.992933in,left,]{\color{textcolor}\sffamily\fontsize{20.000000}{24.000000}\selectfont \(\displaystyle {200}\)}%
\end{pgfscope}%
\begin{pgfscope}%
\pgfsetbuttcap%
\pgfsetroundjoin%
\definecolor{currentfill}{rgb}{0.000000,0.000000,0.000000}%
\pgfsetfillcolor{currentfill}%
\pgfsetlinewidth{0.501875pt}%
\definecolor{currentstroke}{rgb}{0.000000,0.000000,0.000000}%
\pgfsetstrokecolor{currentstroke}%
\pgfsetdash{}{0pt}%
\pgfsys@defobject{currentmarker}{\pgfqpoint{-0.055556in}{0.000000in}}{\pgfqpoint{-0.000000in}{0.000000in}}{%
\pgfpathmoveto{\pgfqpoint{-0.000000in}{0.000000in}}%
\pgfpathlineto{\pgfqpoint{-0.055556in}{0.000000in}}%
\pgfusepath{stroke,fill}%
}%
\begin{pgfscope}%
\pgfsys@transformshift{6.800000in}{2.692379in}%
\pgfsys@useobject{currentmarker}{}%
\end{pgfscope}%
\end{pgfscope}%
\begin{pgfscope}%
\definecolor{textcolor}{rgb}{0.000000,0.000000,0.000000}%
\pgfsetstrokecolor{textcolor}%
\pgfsetfillcolor{textcolor}%
\pgftext[x=6.855556in,y=2.692379in,left,]{\color{textcolor}\sffamily\fontsize{20.000000}{24.000000}\selectfont \(\displaystyle {400}\)}%
\end{pgfscope}%
\begin{pgfscope}%
\pgfsetbuttcap%
\pgfsetroundjoin%
\definecolor{currentfill}{rgb}{0.000000,0.000000,0.000000}%
\pgfsetfillcolor{currentfill}%
\pgfsetlinewidth{0.501875pt}%
\definecolor{currentstroke}{rgb}{0.000000,0.000000,0.000000}%
\pgfsetstrokecolor{currentstroke}%
\pgfsetdash{}{0pt}%
\pgfsys@defobject{currentmarker}{\pgfqpoint{-0.055556in}{0.000000in}}{\pgfqpoint{-0.000000in}{0.000000in}}{%
\pgfpathmoveto{\pgfqpoint{-0.000000in}{0.000000in}}%
\pgfpathlineto{\pgfqpoint{-0.055556in}{0.000000in}}%
\pgfusepath{stroke,fill}%
}%
\begin{pgfscope}%
\pgfsys@transformshift{6.800000in}{3.391826in}%
\pgfsys@useobject{currentmarker}{}%
\end{pgfscope}%
\end{pgfscope}%
\begin{pgfscope}%
\definecolor{textcolor}{rgb}{0.000000,0.000000,0.000000}%
\pgfsetstrokecolor{textcolor}%
\pgfsetfillcolor{textcolor}%
\pgftext[x=6.855556in,y=3.391826in,left,]{\color{textcolor}\sffamily\fontsize{20.000000}{24.000000}\selectfont \(\displaystyle {600}\)}%
\end{pgfscope}%
\begin{pgfscope}%
\pgfsetbuttcap%
\pgfsetroundjoin%
\definecolor{currentfill}{rgb}{0.000000,0.000000,0.000000}%
\pgfsetfillcolor{currentfill}%
\pgfsetlinewidth{0.501875pt}%
\definecolor{currentstroke}{rgb}{0.000000,0.000000,0.000000}%
\pgfsetstrokecolor{currentstroke}%
\pgfsetdash{}{0pt}%
\pgfsys@defobject{currentmarker}{\pgfqpoint{-0.055556in}{0.000000in}}{\pgfqpoint{-0.000000in}{0.000000in}}{%
\pgfpathmoveto{\pgfqpoint{-0.000000in}{0.000000in}}%
\pgfpathlineto{\pgfqpoint{-0.055556in}{0.000000in}}%
\pgfusepath{stroke,fill}%
}%
\begin{pgfscope}%
\pgfsys@transformshift{6.800000in}{4.091273in}%
\pgfsys@useobject{currentmarker}{}%
\end{pgfscope}%
\end{pgfscope}%
\begin{pgfscope}%
\definecolor{textcolor}{rgb}{0.000000,0.000000,0.000000}%
\pgfsetstrokecolor{textcolor}%
\pgfsetfillcolor{textcolor}%
\pgftext[x=6.855556in,y=4.091273in,left,]{\color{textcolor}\sffamily\fontsize{20.000000}{24.000000}\selectfont \(\displaystyle {800}\)}%
\end{pgfscope}%
\begin{pgfscope}%
\pgfsetbuttcap%
\pgfsetroundjoin%
\definecolor{currentfill}{rgb}{0.000000,0.000000,0.000000}%
\pgfsetfillcolor{currentfill}%
\pgfsetlinewidth{0.501875pt}%
\definecolor{currentstroke}{rgb}{0.000000,0.000000,0.000000}%
\pgfsetstrokecolor{currentstroke}%
\pgfsetdash{}{0pt}%
\pgfsys@defobject{currentmarker}{\pgfqpoint{-0.055556in}{0.000000in}}{\pgfqpoint{-0.000000in}{0.000000in}}{%
\pgfpathmoveto{\pgfqpoint{-0.000000in}{0.000000in}}%
\pgfpathlineto{\pgfqpoint{-0.055556in}{0.000000in}}%
\pgfusepath{stroke,fill}%
}%
\begin{pgfscope}%
\pgfsys@transformshift{6.800000in}{4.790719in}%
\pgfsys@useobject{currentmarker}{}%
\end{pgfscope}%
\end{pgfscope}%
\begin{pgfscope}%
\definecolor{textcolor}{rgb}{0.000000,0.000000,0.000000}%
\pgfsetstrokecolor{textcolor}%
\pgfsetfillcolor{textcolor}%
\pgftext[x=6.855556in,y=4.790719in,left,]{\color{textcolor}\sffamily\fontsize{20.000000}{24.000000}\selectfont \(\displaystyle {1000}\)}%
\end{pgfscope}%
\begin{pgfscope}%
\pgfsetbuttcap%
\pgfsetroundjoin%
\definecolor{currentfill}{rgb}{0.000000,0.000000,0.000000}%
\pgfsetfillcolor{currentfill}%
\pgfsetlinewidth{0.501875pt}%
\definecolor{currentstroke}{rgb}{0.000000,0.000000,0.000000}%
\pgfsetstrokecolor{currentstroke}%
\pgfsetdash{}{0pt}%
\pgfsys@defobject{currentmarker}{\pgfqpoint{-0.055556in}{0.000000in}}{\pgfqpoint{-0.000000in}{0.000000in}}{%
\pgfpathmoveto{\pgfqpoint{-0.000000in}{0.000000in}}%
\pgfpathlineto{\pgfqpoint{-0.055556in}{0.000000in}}%
\pgfusepath{stroke,fill}%
}%
\begin{pgfscope}%
\pgfsys@transformshift{6.800000in}{5.490166in}%
\pgfsys@useobject{currentmarker}{}%
\end{pgfscope}%
\end{pgfscope}%
\begin{pgfscope}%
\definecolor{textcolor}{rgb}{0.000000,0.000000,0.000000}%
\pgfsetstrokecolor{textcolor}%
\pgfsetfillcolor{textcolor}%
\pgftext[x=6.855556in,y=5.490166in,left,]{\color{textcolor}\sffamily\fontsize{20.000000}{24.000000}\selectfont \(\displaystyle {1200}\)}%
\end{pgfscope}%
\begin{pgfscope}%
\definecolor{textcolor}{rgb}{0.000000,0.000000,0.000000}%
\pgfsetstrokecolor{textcolor}%
\pgfsetfillcolor{textcolor}%
\pgftext[x=7.453430in,y=3.300000in,,top,rotate=90.000000]{\color{textcolor}\sffamily\fontsize{20.000000}{24.000000}\selectfont \(\displaystyle Charge/\mathrm{mV}\cdot\mathrm{ns}\)}%
\end{pgfscope}%
\begin{pgfscope}%
\pgfsetroundcap%
\pgfsetroundjoin%
\definecolor{currentfill}{rgb}{0.000000,0.000000,0.000000}%
\pgfsetfillcolor{currentfill}%
\pgfsetlinewidth{1.003750pt}%
\definecolor{currentstroke}{rgb}{0.000000,0.000000,0.000000}%
\pgfsetstrokecolor{currentstroke}%
\pgfsetdash{}{0pt}%
\pgfpathmoveto{\pgfqpoint{3.677660in}{3.296316in}}%
\pgfpathquadraticcurveto{\pgfqpoint{3.103262in}{3.428205in}}{\pgfqpoint{2.528863in}{3.560094in}}%
\pgfpathlineto{\pgfqpoint{2.525910in}{3.547234in}}%
\pgfpathquadraticcurveto{\pgfqpoint{2.446254in}{3.572649in}}{\pgfqpoint{2.366597in}{3.598064in}}%
\pgfpathquadraticcurveto{\pgfqpoint{2.449362in}{3.586186in}}{\pgfqpoint{2.532127in}{3.574307in}}%
\pgfpathlineto{\pgfqpoint{2.529174in}{3.561447in}}%
\pgfpathquadraticcurveto{\pgfqpoint{3.103572in}{3.429558in}}{\pgfqpoint{3.677971in}{3.297670in}}%
\pgfpathlineto{\pgfqpoint{3.677660in}{3.296316in}}%
\pgfpathclose%
\pgfusepath{stroke,fill}%
\end{pgfscope}%
\begin{pgfscope}%
\definecolor{textcolor}{rgb}{0.000000,0.000000,0.000000}%
\pgfsetstrokecolor{textcolor}%
\pgfsetfillcolor{textcolor}%
\pgftext[x=3.878626in,y=3.050846in,left,base]{\color{textcolor}\sffamily\fontsize{20.000000}{24.000000}\selectfont \(\displaystyle 371.3\mathrm{ns}, 1.005e+03\mathrm{mV}\cdot\mathrm{ns}\)}%
\end{pgfscope}%
\begin{pgfscope}%
\pgfsetbuttcap%
\pgfsetmiterjoin%
\definecolor{currentfill}{rgb}{1.000000,1.000000,1.000000}%
\pgfsetfillcolor{currentfill}%
\pgfsetlinewidth{1.003750pt}%
\definecolor{currentstroke}{rgb}{0.000000,0.000000,0.000000}%
\pgfsetstrokecolor{currentstroke}%
\pgfsetdash{}{0pt}%
\pgfpathmoveto{\pgfqpoint{4.066020in}{4.011785in}}%
\pgfpathlineto{\pgfqpoint{6.633333in}{4.011785in}}%
\pgfpathlineto{\pgfqpoint{6.633333in}{5.533333in}}%
\pgfpathlineto{\pgfqpoint{4.066020in}{5.533333in}}%
\pgfpathclose%
\pgfusepath{stroke,fill}%
\end{pgfscope}%
\begin{pgfscope}%
\pgfsetrectcap%
\pgfsetroundjoin%
\pgfsetlinewidth{2.007500pt}%
\definecolor{currentstroke}{rgb}{0.000000,0.000000,1.000000}%
\pgfsetstrokecolor{currentstroke}%
\pgfsetdash{}{0pt}%
\pgfpathmoveto{\pgfqpoint{4.299353in}{5.276697in}}%
\pgfpathlineto{\pgfqpoint{4.766020in}{5.276697in}}%
\pgfusepath{stroke}%
\end{pgfscope}%
\begin{pgfscope}%
\definecolor{textcolor}{rgb}{0.000000,0.000000,0.000000}%
\pgfsetstrokecolor{textcolor}%
\pgfsetfillcolor{textcolor}%
\pgftext[x=5.132687in,y=5.160031in,left,base]{\color{textcolor}\sffamily\fontsize{24.000000}{28.800000}\selectfont Waveform}%
\end{pgfscope}%
\begin{pgfscope}%
\pgfsetbuttcap%
\pgfsetroundjoin%
\pgfsetlinewidth{2.007500pt}%
\definecolor{currentstroke}{rgb}{0.000000,0.500000,0.000000}%
\pgfsetstrokecolor{currentstroke}%
\pgfsetdash{}{0pt}%
\pgfpathmoveto{\pgfqpoint{4.299353in}{4.802848in}}%
\pgfpathlineto{\pgfqpoint{4.532687in}{4.802848in}}%
\pgfpathlineto{\pgfqpoint{4.766020in}{4.802848in}}%
\pgfusepath{stroke}%
\end{pgfscope}%
\begin{pgfscope}%
\definecolor{textcolor}{rgb}{0.000000,0.000000,0.000000}%
\pgfsetstrokecolor{textcolor}%
\pgfsetfillcolor{textcolor}%
\pgftext[x=5.132687in,y=4.686181in,left,base]{\color{textcolor}\sffamily\fontsize{24.000000}{28.800000}\selectfont Threshold}%
\end{pgfscope}%
\begin{pgfscope}%
\pgfsetbuttcap%
\pgfsetroundjoin%
\pgfsetlinewidth{2.007500pt}%
\definecolor{currentstroke}{rgb}{1.000000,0.000000,0.000000}%
\pgfsetstrokecolor{currentstroke}%
\pgfsetdash{}{0pt}%
\pgfpathmoveto{\pgfqpoint{4.299353in}{4.328998in}}%
\pgfpathlineto{\pgfqpoint{4.532687in}{4.328998in}}%
\pgfpathlineto{\pgfqpoint{4.766020in}{4.328998in}}%
\pgfusepath{stroke}%
\end{pgfscope}%
\begin{pgfscope}%
\definecolor{textcolor}{rgb}{0.000000,0.000000,0.000000}%
\pgfsetstrokecolor{textcolor}%
\pgfsetfillcolor{textcolor}%
\pgftext[x=5.132687in,y=4.212332in,left,base]{\color{textcolor}\sffamily\fontsize{24.000000}{28.800000}\selectfont Record}%
\end{pgfscope}%
\end{pgfpicture}%
\makeatother%
\endgroup%
}
        \caption{\label{fig:tradi} Traditional Recorded Waveform}
    \end{subfigure}
    \begin{subfigure}{0.5\textwidth}
        \centering
        \scalebox{0.37}{%% Creator: Matplotlib, PGF backend
%%
%% To include the figure in your LaTeX document, write
%%   \input{<filename>.pgf}
%%
%% Make sure the required packages are loaded in your preamble
%%   \usepackage{pgf}
%%
%% and, on pdftex
%%   \usepackage[utf8]{inputenc}\DeclareUnicodeCharacter{2212}{-}
%%
%% or, on luatex and xetex
%%   \usepackage{unicode-math}
%%
%% Figures using additional raster images can only be included by \input if
%% they are in the same directory as the main LaTeX file. For loading figures
%% from other directories you can use the `import` package
%%   \usepackage{import}
%%
%% and then include the figures with
%%   \import{<path to file>}{<filename>.pgf}
%%
%% Matplotlib used the following preamble
%%   \usepackage[detect-all,locale=DE]{siunitx}
%%
\begingroup%
\makeatletter%
\begin{pgfpicture}%
\pgfpathrectangle{\pgfpointorigin}{\pgfqpoint{8.000000in}{6.000000in}}%
\pgfusepath{use as bounding box, clip}%
\begin{pgfscope}%
\pgfsetbuttcap%
\pgfsetmiterjoin%
\definecolor{currentfill}{rgb}{1.000000,1.000000,1.000000}%
\pgfsetfillcolor{currentfill}%
\pgfsetlinewidth{0.000000pt}%
\definecolor{currentstroke}{rgb}{1.000000,1.000000,1.000000}%
\pgfsetstrokecolor{currentstroke}%
\pgfsetdash{}{0pt}%
\pgfpathmoveto{\pgfqpoint{0.000000in}{0.000000in}}%
\pgfpathlineto{\pgfqpoint{8.000000in}{0.000000in}}%
\pgfpathlineto{\pgfqpoint{8.000000in}{6.000000in}}%
\pgfpathlineto{\pgfqpoint{0.000000in}{6.000000in}}%
\pgfpathclose%
\pgfusepath{fill}%
\end{pgfscope}%
\begin{pgfscope}%
\pgfsetbuttcap%
\pgfsetmiterjoin%
\definecolor{currentfill}{rgb}{1.000000,1.000000,1.000000}%
\pgfsetfillcolor{currentfill}%
\pgfsetlinewidth{0.000000pt}%
\definecolor{currentstroke}{rgb}{0.000000,0.000000,0.000000}%
\pgfsetstrokecolor{currentstroke}%
\pgfsetstrokeopacity{0.000000}%
\pgfsetdash{}{0pt}%
\pgfpathmoveto{\pgfqpoint{1.200000in}{0.900000in}}%
\pgfpathlineto{\pgfqpoint{6.800000in}{0.900000in}}%
\pgfpathlineto{\pgfqpoint{6.800000in}{5.700000in}}%
\pgfpathlineto{\pgfqpoint{1.200000in}{5.700000in}}%
\pgfpathclose%
\pgfusepath{fill}%
\end{pgfscope}%
\begin{pgfscope}%
\pgfpathrectangle{\pgfqpoint{1.200000in}{0.900000in}}{\pgfqpoint{5.600000in}{4.800000in}}%
\pgfusepath{clip}%
\pgfsetbuttcap%
\pgfsetroundjoin%
\pgfsetlinewidth{2.007500pt}%
\definecolor{currentstroke}{rgb}{0.000000,0.500000,0.000000}%
\pgfsetstrokecolor{currentstroke}%
\pgfsetdash{}{0pt}%
\pgfpathmoveto{\pgfqpoint{1.190000in}{1.287406in}}%
\pgfpathlineto{\pgfqpoint{6.810000in}{1.287406in}}%
\pgfusepath{stroke}%
\end{pgfscope}%
\begin{pgfscope}%
\pgfpathrectangle{\pgfqpoint{1.200000in}{0.900000in}}{\pgfqpoint{5.600000in}{4.800000in}}%
\pgfusepath{clip}%
\pgfsetrectcap%
\pgfsetroundjoin%
\pgfsetlinewidth{2.007500pt}%
\definecolor{currentstroke}{rgb}{0.000000,0.000000,1.000000}%
\pgfsetstrokecolor{currentstroke}%
\pgfsetdash{}{0pt}%
\pgfpathmoveto{\pgfqpoint{1.190000in}{1.216250in}}%
\pgfpathlineto{\pgfqpoint{1.200000in}{1.176718in}}%
\pgfpathlineto{\pgfqpoint{1.214000in}{1.176718in}}%
\pgfpathlineto{\pgfqpoint{1.228000in}{1.232062in}}%
\pgfpathlineto{\pgfqpoint{1.242000in}{1.121375in}}%
\pgfpathlineto{\pgfqpoint{1.256000in}{1.121375in}}%
\pgfpathlineto{\pgfqpoint{1.270000in}{1.176718in}}%
\pgfpathlineto{\pgfqpoint{1.284000in}{1.176718in}}%
\pgfpathlineto{\pgfqpoint{1.298000in}{1.232062in}}%
\pgfpathlineto{\pgfqpoint{1.326000in}{1.121375in}}%
\pgfpathlineto{\pgfqpoint{1.340000in}{1.232062in}}%
\pgfpathlineto{\pgfqpoint{1.354000in}{1.066031in}}%
\pgfpathlineto{\pgfqpoint{1.368000in}{1.176718in}}%
\pgfpathlineto{\pgfqpoint{1.382000in}{1.232062in}}%
\pgfpathlineto{\pgfqpoint{1.396000in}{1.176718in}}%
\pgfpathlineto{\pgfqpoint{1.410000in}{1.287406in}}%
\pgfpathlineto{\pgfqpoint{1.424000in}{1.176718in}}%
\pgfpathlineto{\pgfqpoint{1.438000in}{1.121375in}}%
\pgfpathlineto{\pgfqpoint{1.452000in}{1.176718in}}%
\pgfpathlineto{\pgfqpoint{1.466000in}{1.176718in}}%
\pgfpathlineto{\pgfqpoint{1.480000in}{1.232062in}}%
\pgfpathlineto{\pgfqpoint{1.494000in}{1.176718in}}%
\pgfpathlineto{\pgfqpoint{1.508000in}{1.176718in}}%
\pgfpathlineto{\pgfqpoint{1.522000in}{1.232062in}}%
\pgfpathlineto{\pgfqpoint{1.536000in}{1.232062in}}%
\pgfpathlineto{\pgfqpoint{1.564000in}{1.121375in}}%
\pgfpathlineto{\pgfqpoint{1.578000in}{1.176718in}}%
\pgfpathlineto{\pgfqpoint{1.592000in}{1.176718in}}%
\pgfpathlineto{\pgfqpoint{1.606000in}{1.232062in}}%
\pgfpathlineto{\pgfqpoint{1.620000in}{1.176718in}}%
\pgfpathlineto{\pgfqpoint{1.634000in}{1.010687in}}%
\pgfpathlineto{\pgfqpoint{1.648000in}{1.176718in}}%
\pgfpathlineto{\pgfqpoint{1.662000in}{1.232062in}}%
\pgfpathlineto{\pgfqpoint{1.676000in}{1.176718in}}%
\pgfpathlineto{\pgfqpoint{1.690000in}{1.232062in}}%
\pgfpathlineto{\pgfqpoint{1.704000in}{1.232062in}}%
\pgfpathlineto{\pgfqpoint{1.718000in}{1.121375in}}%
\pgfpathlineto{\pgfqpoint{1.732000in}{1.066031in}}%
\pgfpathlineto{\pgfqpoint{1.774000in}{1.232062in}}%
\pgfpathlineto{\pgfqpoint{1.802000in}{1.232062in}}%
\pgfpathlineto{\pgfqpoint{1.816000in}{1.121375in}}%
\pgfpathlineto{\pgfqpoint{1.830000in}{1.287406in}}%
\pgfpathlineto{\pgfqpoint{1.844000in}{1.287406in}}%
\pgfpathlineto{\pgfqpoint{1.872000in}{1.176718in}}%
\pgfpathlineto{\pgfqpoint{1.886000in}{1.232062in}}%
\pgfpathlineto{\pgfqpoint{1.900000in}{1.121375in}}%
\pgfpathlineto{\pgfqpoint{1.914000in}{1.176718in}}%
\pgfpathlineto{\pgfqpoint{1.928000in}{1.121375in}}%
\pgfpathlineto{\pgfqpoint{1.942000in}{1.121375in}}%
\pgfpathlineto{\pgfqpoint{1.956000in}{1.066031in}}%
\pgfpathlineto{\pgfqpoint{1.984000in}{1.176718in}}%
\pgfpathlineto{\pgfqpoint{2.012000in}{1.176718in}}%
\pgfpathlineto{\pgfqpoint{2.026000in}{1.232062in}}%
\pgfpathlineto{\pgfqpoint{2.040000in}{1.232062in}}%
\pgfpathlineto{\pgfqpoint{2.054000in}{1.176718in}}%
\pgfpathlineto{\pgfqpoint{2.068000in}{1.066031in}}%
\pgfpathlineto{\pgfqpoint{2.082000in}{1.010687in}}%
\pgfpathlineto{\pgfqpoint{2.096000in}{1.176718in}}%
\pgfpathlineto{\pgfqpoint{2.124000in}{1.176718in}}%
\pgfpathlineto{\pgfqpoint{2.138000in}{1.232062in}}%
\pgfpathlineto{\pgfqpoint{2.152000in}{1.176718in}}%
\pgfpathlineto{\pgfqpoint{2.180000in}{1.287406in}}%
\pgfpathlineto{\pgfqpoint{2.194000in}{1.232062in}}%
\pgfpathlineto{\pgfqpoint{2.208000in}{1.121375in}}%
\pgfpathlineto{\pgfqpoint{2.222000in}{1.176718in}}%
\pgfpathlineto{\pgfqpoint{2.236000in}{1.121375in}}%
\pgfpathlineto{\pgfqpoint{2.250000in}{1.232062in}}%
\pgfpathlineto{\pgfqpoint{2.264000in}{1.287406in}}%
\pgfpathlineto{\pgfqpoint{2.278000in}{1.176718in}}%
\pgfpathlineto{\pgfqpoint{2.292000in}{1.232062in}}%
\pgfpathlineto{\pgfqpoint{2.306000in}{1.121375in}}%
\pgfpathlineto{\pgfqpoint{2.334000in}{1.232062in}}%
\pgfpathlineto{\pgfqpoint{2.348000in}{1.176718in}}%
\pgfpathlineto{\pgfqpoint{2.362000in}{1.176718in}}%
\pgfpathlineto{\pgfqpoint{2.376000in}{1.232062in}}%
\pgfpathlineto{\pgfqpoint{2.390000in}{1.176718in}}%
\pgfpathlineto{\pgfqpoint{2.404000in}{1.176718in}}%
\pgfpathlineto{\pgfqpoint{2.418000in}{1.121375in}}%
\pgfpathlineto{\pgfqpoint{2.432000in}{1.121375in}}%
\pgfpathlineto{\pgfqpoint{2.446000in}{1.176718in}}%
\pgfpathlineto{\pgfqpoint{2.460000in}{1.121375in}}%
\pgfpathlineto{\pgfqpoint{2.474000in}{1.121375in}}%
\pgfpathlineto{\pgfqpoint{2.502000in}{1.232062in}}%
\pgfpathlineto{\pgfqpoint{2.530000in}{1.232062in}}%
\pgfpathlineto{\pgfqpoint{2.544000in}{1.066031in}}%
\pgfpathlineto{\pgfqpoint{2.572000in}{1.287406in}}%
\pgfpathlineto{\pgfqpoint{2.586000in}{1.674812in}}%
\pgfpathlineto{\pgfqpoint{2.600000in}{2.172905in}}%
\pgfpathlineto{\pgfqpoint{2.614000in}{2.837029in}}%
\pgfpathlineto{\pgfqpoint{2.628000in}{3.335122in}}%
\pgfpathlineto{\pgfqpoint{2.642000in}{3.667184in}}%
\pgfpathlineto{\pgfqpoint{2.656000in}{3.777871in}}%
\pgfpathlineto{\pgfqpoint{2.670000in}{3.722528in}}%
\pgfpathlineto{\pgfqpoint{2.684000in}{3.445809in}}%
\pgfpathlineto{\pgfqpoint{2.698000in}{3.224435in}}%
\pgfpathlineto{\pgfqpoint{2.712000in}{2.947716in}}%
\pgfpathlineto{\pgfqpoint{2.726000in}{2.726341in}}%
\pgfpathlineto{\pgfqpoint{2.740000in}{2.615654in}}%
\pgfpathlineto{\pgfqpoint{2.754000in}{2.338936in}}%
\pgfpathlineto{\pgfqpoint{2.768000in}{2.172905in}}%
\pgfpathlineto{\pgfqpoint{2.782000in}{2.283592in}}%
\pgfpathlineto{\pgfqpoint{2.810000in}{2.394279in}}%
\pgfpathlineto{\pgfqpoint{2.824000in}{2.394279in}}%
\pgfpathlineto{\pgfqpoint{2.838000in}{2.338936in}}%
\pgfpathlineto{\pgfqpoint{2.852000in}{2.228248in}}%
\pgfpathlineto{\pgfqpoint{2.866000in}{2.172905in}}%
\pgfpathlineto{\pgfqpoint{2.880000in}{2.006874in}}%
\pgfpathlineto{\pgfqpoint{2.894000in}{1.951530in}}%
\pgfpathlineto{\pgfqpoint{2.908000in}{1.785499in}}%
\pgfpathlineto{\pgfqpoint{2.936000in}{1.785499in}}%
\pgfpathlineto{\pgfqpoint{2.950000in}{2.117561in}}%
\pgfpathlineto{\pgfqpoint{2.964000in}{2.117561in}}%
\pgfpathlineto{\pgfqpoint{2.978000in}{2.228248in}}%
\pgfpathlineto{\pgfqpoint{2.992000in}{2.283592in}}%
\pgfpathlineto{\pgfqpoint{3.006000in}{2.172905in}}%
\pgfpathlineto{\pgfqpoint{3.020000in}{2.117561in}}%
\pgfpathlineto{\pgfqpoint{3.034000in}{2.006874in}}%
\pgfpathlineto{\pgfqpoint{3.048000in}{2.006874in}}%
\pgfpathlineto{\pgfqpoint{3.062000in}{1.840843in}}%
\pgfpathlineto{\pgfqpoint{3.076000in}{1.840843in}}%
\pgfpathlineto{\pgfqpoint{3.090000in}{1.619468in}}%
\pgfpathlineto{\pgfqpoint{3.104000in}{1.785499in}}%
\pgfpathlineto{\pgfqpoint{3.118000in}{1.840843in}}%
\pgfpathlineto{\pgfqpoint{3.132000in}{1.951530in}}%
\pgfpathlineto{\pgfqpoint{3.146000in}{2.006874in}}%
\pgfpathlineto{\pgfqpoint{3.160000in}{2.172905in}}%
\pgfpathlineto{\pgfqpoint{3.174000in}{2.117561in}}%
\pgfpathlineto{\pgfqpoint{3.202000in}{2.117561in}}%
\pgfpathlineto{\pgfqpoint{3.230000in}{1.896186in}}%
\pgfpathlineto{\pgfqpoint{3.244000in}{1.730155in}}%
\pgfpathlineto{\pgfqpoint{3.258000in}{1.785499in}}%
\pgfpathlineto{\pgfqpoint{3.272000in}{2.006874in}}%
\pgfpathlineto{\pgfqpoint{3.300000in}{2.228248in}}%
\pgfpathlineto{\pgfqpoint{3.314000in}{2.228248in}}%
\pgfpathlineto{\pgfqpoint{3.328000in}{2.283592in}}%
\pgfpathlineto{\pgfqpoint{3.342000in}{2.172905in}}%
\pgfpathlineto{\pgfqpoint{3.356000in}{2.117561in}}%
\pgfpathlineto{\pgfqpoint{3.370000in}{1.951530in}}%
\pgfpathlineto{\pgfqpoint{3.384000in}{2.006874in}}%
\pgfpathlineto{\pgfqpoint{3.398000in}{1.840843in}}%
\pgfpathlineto{\pgfqpoint{3.412000in}{1.730155in}}%
\pgfpathlineto{\pgfqpoint{3.440000in}{1.619468in}}%
\pgfpathlineto{\pgfqpoint{3.454000in}{1.508780in}}%
\pgfpathlineto{\pgfqpoint{3.468000in}{1.453437in}}%
\pgfpathlineto{\pgfqpoint{3.482000in}{1.730155in}}%
\pgfpathlineto{\pgfqpoint{3.496000in}{1.896186in}}%
\pgfpathlineto{\pgfqpoint{3.510000in}{1.840843in}}%
\pgfpathlineto{\pgfqpoint{3.524000in}{2.006874in}}%
\pgfpathlineto{\pgfqpoint{3.538000in}{2.062217in}}%
\pgfpathlineto{\pgfqpoint{3.552000in}{2.006874in}}%
\pgfpathlineto{\pgfqpoint{3.566000in}{1.785499in}}%
\pgfpathlineto{\pgfqpoint{3.580000in}{1.785499in}}%
\pgfpathlineto{\pgfqpoint{3.594000in}{1.674812in}}%
\pgfpathlineto{\pgfqpoint{3.608000in}{1.674812in}}%
\pgfpathlineto{\pgfqpoint{3.622000in}{1.564124in}}%
\pgfpathlineto{\pgfqpoint{3.636000in}{1.564124in}}%
\pgfpathlineto{\pgfqpoint{3.664000in}{1.342749in}}%
\pgfpathlineto{\pgfqpoint{3.678000in}{1.342749in}}%
\pgfpathlineto{\pgfqpoint{3.692000in}{1.453437in}}%
\pgfpathlineto{\pgfqpoint{3.706000in}{1.398093in}}%
\pgfpathlineto{\pgfqpoint{3.720000in}{1.287406in}}%
\pgfpathlineto{\pgfqpoint{3.734000in}{1.342749in}}%
\pgfpathlineto{\pgfqpoint{3.748000in}{1.232062in}}%
\pgfpathlineto{\pgfqpoint{3.762000in}{1.287406in}}%
\pgfpathlineto{\pgfqpoint{3.776000in}{1.232062in}}%
\pgfpathlineto{\pgfqpoint{3.790000in}{1.287406in}}%
\pgfpathlineto{\pgfqpoint{3.804000in}{1.176718in}}%
\pgfpathlineto{\pgfqpoint{3.832000in}{1.176718in}}%
\pgfpathlineto{\pgfqpoint{3.846000in}{1.066031in}}%
\pgfpathlineto{\pgfqpoint{3.860000in}{1.287406in}}%
\pgfpathlineto{\pgfqpoint{3.874000in}{1.176718in}}%
\pgfpathlineto{\pgfqpoint{3.888000in}{1.232062in}}%
\pgfpathlineto{\pgfqpoint{3.902000in}{1.176718in}}%
\pgfpathlineto{\pgfqpoint{3.916000in}{1.232062in}}%
\pgfpathlineto{\pgfqpoint{3.944000in}{1.232062in}}%
\pgfpathlineto{\pgfqpoint{3.958000in}{1.176718in}}%
\pgfpathlineto{\pgfqpoint{3.972000in}{1.287406in}}%
\pgfpathlineto{\pgfqpoint{3.986000in}{1.121375in}}%
\pgfpathlineto{\pgfqpoint{4.000000in}{1.176718in}}%
\pgfpathlineto{\pgfqpoint{4.014000in}{1.066031in}}%
\pgfpathlineto{\pgfqpoint{4.028000in}{1.287406in}}%
\pgfpathlineto{\pgfqpoint{4.042000in}{1.232062in}}%
\pgfpathlineto{\pgfqpoint{4.070000in}{1.232062in}}%
\pgfpathlineto{\pgfqpoint{4.084000in}{1.176718in}}%
\pgfpathlineto{\pgfqpoint{4.112000in}{1.176718in}}%
\pgfpathlineto{\pgfqpoint{4.126000in}{1.287406in}}%
\pgfpathlineto{\pgfqpoint{4.154000in}{1.287406in}}%
\pgfpathlineto{\pgfqpoint{4.168000in}{1.176718in}}%
\pgfpathlineto{\pgfqpoint{4.182000in}{1.232062in}}%
\pgfpathlineto{\pgfqpoint{4.210000in}{1.232062in}}%
\pgfpathlineto{\pgfqpoint{4.224000in}{1.176718in}}%
\pgfpathlineto{\pgfqpoint{4.238000in}{1.232062in}}%
\pgfpathlineto{\pgfqpoint{4.266000in}{1.232062in}}%
\pgfpathlineto{\pgfqpoint{4.280000in}{1.342749in}}%
\pgfpathlineto{\pgfqpoint{4.294000in}{1.176718in}}%
\pgfpathlineto{\pgfqpoint{4.308000in}{1.066031in}}%
\pgfpathlineto{\pgfqpoint{4.322000in}{1.232062in}}%
\pgfpathlineto{\pgfqpoint{4.336000in}{1.121375in}}%
\pgfpathlineto{\pgfqpoint{4.350000in}{1.121375in}}%
\pgfpathlineto{\pgfqpoint{4.364000in}{1.066031in}}%
\pgfpathlineto{\pgfqpoint{4.378000in}{1.066031in}}%
\pgfpathlineto{\pgfqpoint{4.392000in}{1.121375in}}%
\pgfpathlineto{\pgfqpoint{4.406000in}{1.232062in}}%
\pgfpathlineto{\pgfqpoint{4.420000in}{1.176718in}}%
\pgfpathlineto{\pgfqpoint{4.434000in}{1.287406in}}%
\pgfpathlineto{\pgfqpoint{4.448000in}{1.121375in}}%
\pgfpathlineto{\pgfqpoint{4.462000in}{1.121375in}}%
\pgfpathlineto{\pgfqpoint{4.490000in}{1.232062in}}%
\pgfpathlineto{\pgfqpoint{4.504000in}{1.232062in}}%
\pgfpathlineto{\pgfqpoint{4.518000in}{1.066031in}}%
\pgfpathlineto{\pgfqpoint{4.532000in}{1.121375in}}%
\pgfpathlineto{\pgfqpoint{4.546000in}{1.121375in}}%
\pgfpathlineto{\pgfqpoint{4.560000in}{1.232062in}}%
\pgfpathlineto{\pgfqpoint{4.574000in}{1.121375in}}%
\pgfpathlineto{\pgfqpoint{4.588000in}{1.121375in}}%
\pgfpathlineto{\pgfqpoint{4.602000in}{1.176718in}}%
\pgfpathlineto{\pgfqpoint{4.616000in}{1.287406in}}%
\pgfpathlineto{\pgfqpoint{4.630000in}{1.232062in}}%
\pgfpathlineto{\pgfqpoint{4.644000in}{1.232062in}}%
\pgfpathlineto{\pgfqpoint{4.672000in}{1.121375in}}%
\pgfpathlineto{\pgfqpoint{4.686000in}{1.176718in}}%
\pgfpathlineto{\pgfqpoint{4.714000in}{1.176718in}}%
\pgfpathlineto{\pgfqpoint{4.742000in}{1.287406in}}%
\pgfpathlineto{\pgfqpoint{4.756000in}{1.176718in}}%
\pgfpathlineto{\pgfqpoint{4.770000in}{1.121375in}}%
\pgfpathlineto{\pgfqpoint{4.784000in}{1.176718in}}%
\pgfpathlineto{\pgfqpoint{4.798000in}{1.121375in}}%
\pgfpathlineto{\pgfqpoint{4.812000in}{1.176718in}}%
\pgfpathlineto{\pgfqpoint{4.826000in}{1.287406in}}%
\pgfpathlineto{\pgfqpoint{4.840000in}{1.232062in}}%
\pgfpathlineto{\pgfqpoint{4.854000in}{1.232062in}}%
\pgfpathlineto{\pgfqpoint{4.868000in}{1.121375in}}%
\pgfpathlineto{\pgfqpoint{4.882000in}{1.287406in}}%
\pgfpathlineto{\pgfqpoint{4.910000in}{1.176718in}}%
\pgfpathlineto{\pgfqpoint{4.938000in}{1.176718in}}%
\pgfpathlineto{\pgfqpoint{4.952000in}{1.287406in}}%
\pgfpathlineto{\pgfqpoint{4.966000in}{1.176718in}}%
\pgfpathlineto{\pgfqpoint{4.980000in}{1.232062in}}%
\pgfpathlineto{\pgfqpoint{5.008000in}{1.232062in}}%
\pgfpathlineto{\pgfqpoint{5.022000in}{1.121375in}}%
\pgfpathlineto{\pgfqpoint{5.036000in}{1.176718in}}%
\pgfpathlineto{\pgfqpoint{5.064000in}{1.176718in}}%
\pgfpathlineto{\pgfqpoint{5.078000in}{1.232062in}}%
\pgfpathlineto{\pgfqpoint{5.092000in}{1.121375in}}%
\pgfpathlineto{\pgfqpoint{5.106000in}{1.232062in}}%
\pgfpathlineto{\pgfqpoint{5.120000in}{1.176718in}}%
\pgfpathlineto{\pgfqpoint{5.148000in}{1.176718in}}%
\pgfpathlineto{\pgfqpoint{5.162000in}{1.121375in}}%
\pgfpathlineto{\pgfqpoint{5.176000in}{1.232062in}}%
\pgfpathlineto{\pgfqpoint{5.190000in}{1.121375in}}%
\pgfpathlineto{\pgfqpoint{5.204000in}{1.066031in}}%
\pgfpathlineto{\pgfqpoint{5.218000in}{1.176718in}}%
\pgfpathlineto{\pgfqpoint{5.246000in}{1.176718in}}%
\pgfpathlineto{\pgfqpoint{5.260000in}{1.066031in}}%
\pgfpathlineto{\pgfqpoint{5.274000in}{1.232062in}}%
\pgfpathlineto{\pgfqpoint{5.288000in}{1.176718in}}%
\pgfpathlineto{\pgfqpoint{5.302000in}{1.066031in}}%
\pgfpathlineto{\pgfqpoint{5.316000in}{1.176718in}}%
\pgfpathlineto{\pgfqpoint{5.330000in}{1.232062in}}%
\pgfpathlineto{\pgfqpoint{5.358000in}{1.232062in}}%
\pgfpathlineto{\pgfqpoint{5.372000in}{1.176718in}}%
\pgfpathlineto{\pgfqpoint{5.386000in}{1.176718in}}%
\pgfpathlineto{\pgfqpoint{5.400000in}{1.232062in}}%
\pgfpathlineto{\pgfqpoint{5.428000in}{1.232062in}}%
\pgfpathlineto{\pgfqpoint{5.442000in}{1.121375in}}%
\pgfpathlineto{\pgfqpoint{5.456000in}{1.287406in}}%
\pgfpathlineto{\pgfqpoint{5.470000in}{1.066031in}}%
\pgfpathlineto{\pgfqpoint{5.484000in}{1.176718in}}%
\pgfpathlineto{\pgfqpoint{5.498000in}{1.232062in}}%
\pgfpathlineto{\pgfqpoint{5.512000in}{1.066031in}}%
\pgfpathlineto{\pgfqpoint{5.526000in}{1.121375in}}%
\pgfpathlineto{\pgfqpoint{5.540000in}{1.121375in}}%
\pgfpathlineto{\pgfqpoint{5.554000in}{1.232062in}}%
\pgfpathlineto{\pgfqpoint{5.568000in}{1.176718in}}%
\pgfpathlineto{\pgfqpoint{5.582000in}{1.232062in}}%
\pgfpathlineto{\pgfqpoint{5.596000in}{1.232062in}}%
\pgfpathlineto{\pgfqpoint{5.624000in}{1.121375in}}%
\pgfpathlineto{\pgfqpoint{5.638000in}{1.232062in}}%
\pgfpathlineto{\pgfqpoint{5.652000in}{1.176718in}}%
\pgfpathlineto{\pgfqpoint{5.666000in}{1.176718in}}%
\pgfpathlineto{\pgfqpoint{5.680000in}{1.287406in}}%
\pgfpathlineto{\pgfqpoint{5.694000in}{1.121375in}}%
\pgfpathlineto{\pgfqpoint{5.722000in}{1.232062in}}%
\pgfpathlineto{\pgfqpoint{5.736000in}{1.121375in}}%
\pgfpathlineto{\pgfqpoint{5.750000in}{1.176718in}}%
\pgfpathlineto{\pgfqpoint{5.764000in}{1.121375in}}%
\pgfpathlineto{\pgfqpoint{5.778000in}{1.287406in}}%
\pgfpathlineto{\pgfqpoint{5.792000in}{1.176718in}}%
\pgfpathlineto{\pgfqpoint{5.806000in}{1.121375in}}%
\pgfpathlineto{\pgfqpoint{5.834000in}{1.121375in}}%
\pgfpathlineto{\pgfqpoint{5.848000in}{1.232062in}}%
\pgfpathlineto{\pgfqpoint{5.862000in}{1.232062in}}%
\pgfpathlineto{\pgfqpoint{5.876000in}{1.121375in}}%
\pgfpathlineto{\pgfqpoint{5.890000in}{1.232062in}}%
\pgfpathlineto{\pgfqpoint{5.904000in}{1.121375in}}%
\pgfpathlineto{\pgfqpoint{5.918000in}{1.121375in}}%
\pgfpathlineto{\pgfqpoint{5.932000in}{1.232062in}}%
\pgfpathlineto{\pgfqpoint{5.960000in}{1.121375in}}%
\pgfpathlineto{\pgfqpoint{5.974000in}{1.121375in}}%
\pgfpathlineto{\pgfqpoint{5.988000in}{1.176718in}}%
\pgfpathlineto{\pgfqpoint{6.016000in}{1.176718in}}%
\pgfpathlineto{\pgfqpoint{6.030000in}{1.232062in}}%
\pgfpathlineto{\pgfqpoint{6.044000in}{1.010687in}}%
\pgfpathlineto{\pgfqpoint{6.058000in}{1.232062in}}%
\pgfpathlineto{\pgfqpoint{6.072000in}{1.121375in}}%
\pgfpathlineto{\pgfqpoint{6.086000in}{1.287406in}}%
\pgfpathlineto{\pgfqpoint{6.100000in}{1.176718in}}%
\pgfpathlineto{\pgfqpoint{6.114000in}{1.176718in}}%
\pgfpathlineto{\pgfqpoint{6.128000in}{1.066031in}}%
\pgfpathlineto{\pgfqpoint{6.142000in}{1.232062in}}%
\pgfpathlineto{\pgfqpoint{6.156000in}{1.176718in}}%
\pgfpathlineto{\pgfqpoint{6.170000in}{1.232062in}}%
\pgfpathlineto{\pgfqpoint{6.198000in}{1.232062in}}%
\pgfpathlineto{\pgfqpoint{6.212000in}{1.176718in}}%
\pgfpathlineto{\pgfqpoint{6.226000in}{1.176718in}}%
\pgfpathlineto{\pgfqpoint{6.254000in}{1.287406in}}%
\pgfpathlineto{\pgfqpoint{6.268000in}{1.232062in}}%
\pgfpathlineto{\pgfqpoint{6.282000in}{1.232062in}}%
\pgfpathlineto{\pgfqpoint{6.296000in}{1.066031in}}%
\pgfpathlineto{\pgfqpoint{6.324000in}{1.176718in}}%
\pgfpathlineto{\pgfqpoint{6.338000in}{1.121375in}}%
\pgfpathlineto{\pgfqpoint{6.352000in}{1.121375in}}%
\pgfpathlineto{\pgfqpoint{6.366000in}{1.232062in}}%
\pgfpathlineto{\pgfqpoint{6.380000in}{1.121375in}}%
\pgfpathlineto{\pgfqpoint{6.394000in}{1.121375in}}%
\pgfpathlineto{\pgfqpoint{6.408000in}{1.232062in}}%
\pgfpathlineto{\pgfqpoint{6.422000in}{1.121375in}}%
\pgfpathlineto{\pgfqpoint{6.436000in}{1.176718in}}%
\pgfpathlineto{\pgfqpoint{6.450000in}{1.176718in}}%
\pgfpathlineto{\pgfqpoint{6.464000in}{1.121375in}}%
\pgfpathlineto{\pgfqpoint{6.478000in}{1.121375in}}%
\pgfpathlineto{\pgfqpoint{6.506000in}{1.232062in}}%
\pgfpathlineto{\pgfqpoint{6.520000in}{1.232062in}}%
\pgfpathlineto{\pgfqpoint{6.534000in}{1.066031in}}%
\pgfpathlineto{\pgfqpoint{6.562000in}{1.066031in}}%
\pgfpathlineto{\pgfqpoint{6.576000in}{1.176718in}}%
\pgfpathlineto{\pgfqpoint{6.590000in}{1.176718in}}%
\pgfpathlineto{\pgfqpoint{6.618000in}{1.287406in}}%
\pgfpathlineto{\pgfqpoint{6.632000in}{1.287406in}}%
\pgfpathlineto{\pgfqpoint{6.646000in}{1.176718in}}%
\pgfpathlineto{\pgfqpoint{6.660000in}{1.176718in}}%
\pgfpathlineto{\pgfqpoint{6.688000in}{1.066031in}}%
\pgfpathlineto{\pgfqpoint{6.702000in}{1.121375in}}%
\pgfpathlineto{\pgfqpoint{6.716000in}{1.287406in}}%
\pgfpathlineto{\pgfqpoint{6.730000in}{1.121375in}}%
\pgfpathlineto{\pgfqpoint{6.744000in}{1.121375in}}%
\pgfpathlineto{\pgfqpoint{6.758000in}{1.176718in}}%
\pgfpathlineto{\pgfqpoint{6.786000in}{1.176718in}}%
\pgfpathlineto{\pgfqpoint{6.800000in}{1.121375in}}%
\pgfpathlineto{\pgfqpoint{6.810000in}{1.160906in}}%
\pgfpathlineto{\pgfqpoint{6.810000in}{1.160906in}}%
\pgfusepath{stroke}%
\end{pgfscope}%
\begin{pgfscope}%
\pgfsetrectcap%
\pgfsetmiterjoin%
\pgfsetlinewidth{1.003750pt}%
\definecolor{currentstroke}{rgb}{0.000000,0.000000,0.000000}%
\pgfsetstrokecolor{currentstroke}%
\pgfsetdash{}{0pt}%
\pgfpathmoveto{\pgfqpoint{1.200000in}{0.900000in}}%
\pgfpathlineto{\pgfqpoint{1.200000in}{5.700000in}}%
\pgfusepath{stroke}%
\end{pgfscope}%
\begin{pgfscope}%
\pgfsetrectcap%
\pgfsetmiterjoin%
\pgfsetlinewidth{1.003750pt}%
\definecolor{currentstroke}{rgb}{0.000000,0.000000,0.000000}%
\pgfsetstrokecolor{currentstroke}%
\pgfsetdash{}{0pt}%
\pgfpathmoveto{\pgfqpoint{6.800000in}{0.900000in}}%
\pgfpathlineto{\pgfqpoint{6.800000in}{5.700000in}}%
\pgfusepath{stroke}%
\end{pgfscope}%
\begin{pgfscope}%
\pgfsetrectcap%
\pgfsetmiterjoin%
\pgfsetlinewidth{1.003750pt}%
\definecolor{currentstroke}{rgb}{0.000000,0.000000,0.000000}%
\pgfsetstrokecolor{currentstroke}%
\pgfsetdash{}{0pt}%
\pgfpathmoveto{\pgfqpoint{1.200000in}{0.900000in}}%
\pgfpathlineto{\pgfqpoint{6.800000in}{0.900000in}}%
\pgfusepath{stroke}%
\end{pgfscope}%
\begin{pgfscope}%
\pgfsetrectcap%
\pgfsetmiterjoin%
\pgfsetlinewidth{1.003750pt}%
\definecolor{currentstroke}{rgb}{0.000000,0.000000,0.000000}%
\pgfsetstrokecolor{currentstroke}%
\pgfsetdash{}{0pt}%
\pgfpathmoveto{\pgfqpoint{1.200000in}{5.700000in}}%
\pgfpathlineto{\pgfqpoint{6.800000in}{5.700000in}}%
\pgfusepath{stroke}%
\end{pgfscope}%
\begin{pgfscope}%
\pgfsetbuttcap%
\pgfsetroundjoin%
\definecolor{currentfill}{rgb}{0.000000,0.000000,0.000000}%
\pgfsetfillcolor{currentfill}%
\pgfsetlinewidth{0.501875pt}%
\definecolor{currentstroke}{rgb}{0.000000,0.000000,0.000000}%
\pgfsetstrokecolor{currentstroke}%
\pgfsetdash{}{0pt}%
\pgfsys@defobject{currentmarker}{\pgfqpoint{0.000000in}{0.000000in}}{\pgfqpoint{0.000000in}{0.055556in}}{%
\pgfpathmoveto{\pgfqpoint{0.000000in}{0.000000in}}%
\pgfpathlineto{\pgfqpoint{0.000000in}{0.055556in}}%
\pgfusepath{stroke,fill}%
}%
\begin{pgfscope}%
\pgfsys@transformshift{1.200000in}{0.900000in}%
\pgfsys@useobject{currentmarker}{}%
\end{pgfscope}%
\end{pgfscope}%
\begin{pgfscope}%
\pgfsetbuttcap%
\pgfsetroundjoin%
\definecolor{currentfill}{rgb}{0.000000,0.000000,0.000000}%
\pgfsetfillcolor{currentfill}%
\pgfsetlinewidth{0.501875pt}%
\definecolor{currentstroke}{rgb}{0.000000,0.000000,0.000000}%
\pgfsetstrokecolor{currentstroke}%
\pgfsetdash{}{0pt}%
\pgfsys@defobject{currentmarker}{\pgfqpoint{0.000000in}{-0.055556in}}{\pgfqpoint{0.000000in}{0.000000in}}{%
\pgfpathmoveto{\pgfqpoint{0.000000in}{0.000000in}}%
\pgfpathlineto{\pgfqpoint{0.000000in}{-0.055556in}}%
\pgfusepath{stroke,fill}%
}%
\begin{pgfscope}%
\pgfsys@transformshift{1.200000in}{5.700000in}%
\pgfsys@useobject{currentmarker}{}%
\end{pgfscope}%
\end{pgfscope}%
\begin{pgfscope}%
\definecolor{textcolor}{rgb}{0.000000,0.000000,0.000000}%
\pgfsetstrokecolor{textcolor}%
\pgfsetfillcolor{textcolor}%
\pgftext[x=1.200000in,y=0.844444in,,top]{\color{textcolor}\sffamily\fontsize{20.000000}{24.000000}\selectfont \(\displaystyle {200}\)}%
\end{pgfscope}%
\begin{pgfscope}%
\pgfsetbuttcap%
\pgfsetroundjoin%
\definecolor{currentfill}{rgb}{0.000000,0.000000,0.000000}%
\pgfsetfillcolor{currentfill}%
\pgfsetlinewidth{0.501875pt}%
\definecolor{currentstroke}{rgb}{0.000000,0.000000,0.000000}%
\pgfsetstrokecolor{currentstroke}%
\pgfsetdash{}{0pt}%
\pgfsys@defobject{currentmarker}{\pgfqpoint{0.000000in}{0.000000in}}{\pgfqpoint{0.000000in}{0.055556in}}{%
\pgfpathmoveto{\pgfqpoint{0.000000in}{0.000000in}}%
\pgfpathlineto{\pgfqpoint{0.000000in}{0.055556in}}%
\pgfusepath{stroke,fill}%
}%
\begin{pgfscope}%
\pgfsys@transformshift{1.900000in}{0.900000in}%
\pgfsys@useobject{currentmarker}{}%
\end{pgfscope}%
\end{pgfscope}%
\begin{pgfscope}%
\pgfsetbuttcap%
\pgfsetroundjoin%
\definecolor{currentfill}{rgb}{0.000000,0.000000,0.000000}%
\pgfsetfillcolor{currentfill}%
\pgfsetlinewidth{0.501875pt}%
\definecolor{currentstroke}{rgb}{0.000000,0.000000,0.000000}%
\pgfsetstrokecolor{currentstroke}%
\pgfsetdash{}{0pt}%
\pgfsys@defobject{currentmarker}{\pgfqpoint{0.000000in}{-0.055556in}}{\pgfqpoint{0.000000in}{0.000000in}}{%
\pgfpathmoveto{\pgfqpoint{0.000000in}{0.000000in}}%
\pgfpathlineto{\pgfqpoint{0.000000in}{-0.055556in}}%
\pgfusepath{stroke,fill}%
}%
\begin{pgfscope}%
\pgfsys@transformshift{1.900000in}{5.700000in}%
\pgfsys@useobject{currentmarker}{}%
\end{pgfscope}%
\end{pgfscope}%
\begin{pgfscope}%
\definecolor{textcolor}{rgb}{0.000000,0.000000,0.000000}%
\pgfsetstrokecolor{textcolor}%
\pgfsetfillcolor{textcolor}%
\pgftext[x=1.900000in,y=0.844444in,,top]{\color{textcolor}\sffamily\fontsize{20.000000}{24.000000}\selectfont \(\displaystyle {250}\)}%
\end{pgfscope}%
\begin{pgfscope}%
\pgfsetbuttcap%
\pgfsetroundjoin%
\definecolor{currentfill}{rgb}{0.000000,0.000000,0.000000}%
\pgfsetfillcolor{currentfill}%
\pgfsetlinewidth{0.501875pt}%
\definecolor{currentstroke}{rgb}{0.000000,0.000000,0.000000}%
\pgfsetstrokecolor{currentstroke}%
\pgfsetdash{}{0pt}%
\pgfsys@defobject{currentmarker}{\pgfqpoint{0.000000in}{0.000000in}}{\pgfqpoint{0.000000in}{0.055556in}}{%
\pgfpathmoveto{\pgfqpoint{0.000000in}{0.000000in}}%
\pgfpathlineto{\pgfqpoint{0.000000in}{0.055556in}}%
\pgfusepath{stroke,fill}%
}%
\begin{pgfscope}%
\pgfsys@transformshift{2.600000in}{0.900000in}%
\pgfsys@useobject{currentmarker}{}%
\end{pgfscope}%
\end{pgfscope}%
\begin{pgfscope}%
\pgfsetbuttcap%
\pgfsetroundjoin%
\definecolor{currentfill}{rgb}{0.000000,0.000000,0.000000}%
\pgfsetfillcolor{currentfill}%
\pgfsetlinewidth{0.501875pt}%
\definecolor{currentstroke}{rgb}{0.000000,0.000000,0.000000}%
\pgfsetstrokecolor{currentstroke}%
\pgfsetdash{}{0pt}%
\pgfsys@defobject{currentmarker}{\pgfqpoint{0.000000in}{-0.055556in}}{\pgfqpoint{0.000000in}{0.000000in}}{%
\pgfpathmoveto{\pgfqpoint{0.000000in}{0.000000in}}%
\pgfpathlineto{\pgfqpoint{0.000000in}{-0.055556in}}%
\pgfusepath{stroke,fill}%
}%
\begin{pgfscope}%
\pgfsys@transformshift{2.600000in}{5.700000in}%
\pgfsys@useobject{currentmarker}{}%
\end{pgfscope}%
\end{pgfscope}%
\begin{pgfscope}%
\definecolor{textcolor}{rgb}{0.000000,0.000000,0.000000}%
\pgfsetstrokecolor{textcolor}%
\pgfsetfillcolor{textcolor}%
\pgftext[x=2.600000in,y=0.844444in,,top]{\color{textcolor}\sffamily\fontsize{20.000000}{24.000000}\selectfont \(\displaystyle {300}\)}%
\end{pgfscope}%
\begin{pgfscope}%
\pgfsetbuttcap%
\pgfsetroundjoin%
\definecolor{currentfill}{rgb}{0.000000,0.000000,0.000000}%
\pgfsetfillcolor{currentfill}%
\pgfsetlinewidth{0.501875pt}%
\definecolor{currentstroke}{rgb}{0.000000,0.000000,0.000000}%
\pgfsetstrokecolor{currentstroke}%
\pgfsetdash{}{0pt}%
\pgfsys@defobject{currentmarker}{\pgfqpoint{0.000000in}{0.000000in}}{\pgfqpoint{0.000000in}{0.055556in}}{%
\pgfpathmoveto{\pgfqpoint{0.000000in}{0.000000in}}%
\pgfpathlineto{\pgfqpoint{0.000000in}{0.055556in}}%
\pgfusepath{stroke,fill}%
}%
\begin{pgfscope}%
\pgfsys@transformshift{3.300000in}{0.900000in}%
\pgfsys@useobject{currentmarker}{}%
\end{pgfscope}%
\end{pgfscope}%
\begin{pgfscope}%
\pgfsetbuttcap%
\pgfsetroundjoin%
\definecolor{currentfill}{rgb}{0.000000,0.000000,0.000000}%
\pgfsetfillcolor{currentfill}%
\pgfsetlinewidth{0.501875pt}%
\definecolor{currentstroke}{rgb}{0.000000,0.000000,0.000000}%
\pgfsetstrokecolor{currentstroke}%
\pgfsetdash{}{0pt}%
\pgfsys@defobject{currentmarker}{\pgfqpoint{0.000000in}{-0.055556in}}{\pgfqpoint{0.000000in}{0.000000in}}{%
\pgfpathmoveto{\pgfqpoint{0.000000in}{0.000000in}}%
\pgfpathlineto{\pgfqpoint{0.000000in}{-0.055556in}}%
\pgfusepath{stroke,fill}%
}%
\begin{pgfscope}%
\pgfsys@transformshift{3.300000in}{5.700000in}%
\pgfsys@useobject{currentmarker}{}%
\end{pgfscope}%
\end{pgfscope}%
\begin{pgfscope}%
\definecolor{textcolor}{rgb}{0.000000,0.000000,0.000000}%
\pgfsetstrokecolor{textcolor}%
\pgfsetfillcolor{textcolor}%
\pgftext[x=3.300000in,y=0.844444in,,top]{\color{textcolor}\sffamily\fontsize{20.000000}{24.000000}\selectfont \(\displaystyle {350}\)}%
\end{pgfscope}%
\begin{pgfscope}%
\pgfsetbuttcap%
\pgfsetroundjoin%
\definecolor{currentfill}{rgb}{0.000000,0.000000,0.000000}%
\pgfsetfillcolor{currentfill}%
\pgfsetlinewidth{0.501875pt}%
\definecolor{currentstroke}{rgb}{0.000000,0.000000,0.000000}%
\pgfsetstrokecolor{currentstroke}%
\pgfsetdash{}{0pt}%
\pgfsys@defobject{currentmarker}{\pgfqpoint{0.000000in}{0.000000in}}{\pgfqpoint{0.000000in}{0.055556in}}{%
\pgfpathmoveto{\pgfqpoint{0.000000in}{0.000000in}}%
\pgfpathlineto{\pgfqpoint{0.000000in}{0.055556in}}%
\pgfusepath{stroke,fill}%
}%
\begin{pgfscope}%
\pgfsys@transformshift{4.000000in}{0.900000in}%
\pgfsys@useobject{currentmarker}{}%
\end{pgfscope}%
\end{pgfscope}%
\begin{pgfscope}%
\pgfsetbuttcap%
\pgfsetroundjoin%
\definecolor{currentfill}{rgb}{0.000000,0.000000,0.000000}%
\pgfsetfillcolor{currentfill}%
\pgfsetlinewidth{0.501875pt}%
\definecolor{currentstroke}{rgb}{0.000000,0.000000,0.000000}%
\pgfsetstrokecolor{currentstroke}%
\pgfsetdash{}{0pt}%
\pgfsys@defobject{currentmarker}{\pgfqpoint{0.000000in}{-0.055556in}}{\pgfqpoint{0.000000in}{0.000000in}}{%
\pgfpathmoveto{\pgfqpoint{0.000000in}{0.000000in}}%
\pgfpathlineto{\pgfqpoint{0.000000in}{-0.055556in}}%
\pgfusepath{stroke,fill}%
}%
\begin{pgfscope}%
\pgfsys@transformshift{4.000000in}{5.700000in}%
\pgfsys@useobject{currentmarker}{}%
\end{pgfscope}%
\end{pgfscope}%
\begin{pgfscope}%
\definecolor{textcolor}{rgb}{0.000000,0.000000,0.000000}%
\pgfsetstrokecolor{textcolor}%
\pgfsetfillcolor{textcolor}%
\pgftext[x=4.000000in,y=0.844444in,,top]{\color{textcolor}\sffamily\fontsize{20.000000}{24.000000}\selectfont \(\displaystyle {400}\)}%
\end{pgfscope}%
\begin{pgfscope}%
\pgfsetbuttcap%
\pgfsetroundjoin%
\definecolor{currentfill}{rgb}{0.000000,0.000000,0.000000}%
\pgfsetfillcolor{currentfill}%
\pgfsetlinewidth{0.501875pt}%
\definecolor{currentstroke}{rgb}{0.000000,0.000000,0.000000}%
\pgfsetstrokecolor{currentstroke}%
\pgfsetdash{}{0pt}%
\pgfsys@defobject{currentmarker}{\pgfqpoint{0.000000in}{0.000000in}}{\pgfqpoint{0.000000in}{0.055556in}}{%
\pgfpathmoveto{\pgfqpoint{0.000000in}{0.000000in}}%
\pgfpathlineto{\pgfqpoint{0.000000in}{0.055556in}}%
\pgfusepath{stroke,fill}%
}%
\begin{pgfscope}%
\pgfsys@transformshift{4.700000in}{0.900000in}%
\pgfsys@useobject{currentmarker}{}%
\end{pgfscope}%
\end{pgfscope}%
\begin{pgfscope}%
\pgfsetbuttcap%
\pgfsetroundjoin%
\definecolor{currentfill}{rgb}{0.000000,0.000000,0.000000}%
\pgfsetfillcolor{currentfill}%
\pgfsetlinewidth{0.501875pt}%
\definecolor{currentstroke}{rgb}{0.000000,0.000000,0.000000}%
\pgfsetstrokecolor{currentstroke}%
\pgfsetdash{}{0pt}%
\pgfsys@defobject{currentmarker}{\pgfqpoint{0.000000in}{-0.055556in}}{\pgfqpoint{0.000000in}{0.000000in}}{%
\pgfpathmoveto{\pgfqpoint{0.000000in}{0.000000in}}%
\pgfpathlineto{\pgfqpoint{0.000000in}{-0.055556in}}%
\pgfusepath{stroke,fill}%
}%
\begin{pgfscope}%
\pgfsys@transformshift{4.700000in}{5.700000in}%
\pgfsys@useobject{currentmarker}{}%
\end{pgfscope}%
\end{pgfscope}%
\begin{pgfscope}%
\definecolor{textcolor}{rgb}{0.000000,0.000000,0.000000}%
\pgfsetstrokecolor{textcolor}%
\pgfsetfillcolor{textcolor}%
\pgftext[x=4.700000in,y=0.844444in,,top]{\color{textcolor}\sffamily\fontsize{20.000000}{24.000000}\selectfont \(\displaystyle {450}\)}%
\end{pgfscope}%
\begin{pgfscope}%
\pgfsetbuttcap%
\pgfsetroundjoin%
\definecolor{currentfill}{rgb}{0.000000,0.000000,0.000000}%
\pgfsetfillcolor{currentfill}%
\pgfsetlinewidth{0.501875pt}%
\definecolor{currentstroke}{rgb}{0.000000,0.000000,0.000000}%
\pgfsetstrokecolor{currentstroke}%
\pgfsetdash{}{0pt}%
\pgfsys@defobject{currentmarker}{\pgfqpoint{0.000000in}{0.000000in}}{\pgfqpoint{0.000000in}{0.055556in}}{%
\pgfpathmoveto{\pgfqpoint{0.000000in}{0.000000in}}%
\pgfpathlineto{\pgfqpoint{0.000000in}{0.055556in}}%
\pgfusepath{stroke,fill}%
}%
\begin{pgfscope}%
\pgfsys@transformshift{5.400000in}{0.900000in}%
\pgfsys@useobject{currentmarker}{}%
\end{pgfscope}%
\end{pgfscope}%
\begin{pgfscope}%
\pgfsetbuttcap%
\pgfsetroundjoin%
\definecolor{currentfill}{rgb}{0.000000,0.000000,0.000000}%
\pgfsetfillcolor{currentfill}%
\pgfsetlinewidth{0.501875pt}%
\definecolor{currentstroke}{rgb}{0.000000,0.000000,0.000000}%
\pgfsetstrokecolor{currentstroke}%
\pgfsetdash{}{0pt}%
\pgfsys@defobject{currentmarker}{\pgfqpoint{0.000000in}{-0.055556in}}{\pgfqpoint{0.000000in}{0.000000in}}{%
\pgfpathmoveto{\pgfqpoint{0.000000in}{0.000000in}}%
\pgfpathlineto{\pgfqpoint{0.000000in}{-0.055556in}}%
\pgfusepath{stroke,fill}%
}%
\begin{pgfscope}%
\pgfsys@transformshift{5.400000in}{5.700000in}%
\pgfsys@useobject{currentmarker}{}%
\end{pgfscope}%
\end{pgfscope}%
\begin{pgfscope}%
\definecolor{textcolor}{rgb}{0.000000,0.000000,0.000000}%
\pgfsetstrokecolor{textcolor}%
\pgfsetfillcolor{textcolor}%
\pgftext[x=5.400000in,y=0.844444in,,top]{\color{textcolor}\sffamily\fontsize{20.000000}{24.000000}\selectfont \(\displaystyle {500}\)}%
\end{pgfscope}%
\begin{pgfscope}%
\pgfsetbuttcap%
\pgfsetroundjoin%
\definecolor{currentfill}{rgb}{0.000000,0.000000,0.000000}%
\pgfsetfillcolor{currentfill}%
\pgfsetlinewidth{0.501875pt}%
\definecolor{currentstroke}{rgb}{0.000000,0.000000,0.000000}%
\pgfsetstrokecolor{currentstroke}%
\pgfsetdash{}{0pt}%
\pgfsys@defobject{currentmarker}{\pgfqpoint{0.000000in}{0.000000in}}{\pgfqpoint{0.000000in}{0.055556in}}{%
\pgfpathmoveto{\pgfqpoint{0.000000in}{0.000000in}}%
\pgfpathlineto{\pgfqpoint{0.000000in}{0.055556in}}%
\pgfusepath{stroke,fill}%
}%
\begin{pgfscope}%
\pgfsys@transformshift{6.100000in}{0.900000in}%
\pgfsys@useobject{currentmarker}{}%
\end{pgfscope}%
\end{pgfscope}%
\begin{pgfscope}%
\pgfsetbuttcap%
\pgfsetroundjoin%
\definecolor{currentfill}{rgb}{0.000000,0.000000,0.000000}%
\pgfsetfillcolor{currentfill}%
\pgfsetlinewidth{0.501875pt}%
\definecolor{currentstroke}{rgb}{0.000000,0.000000,0.000000}%
\pgfsetstrokecolor{currentstroke}%
\pgfsetdash{}{0pt}%
\pgfsys@defobject{currentmarker}{\pgfqpoint{0.000000in}{-0.055556in}}{\pgfqpoint{0.000000in}{0.000000in}}{%
\pgfpathmoveto{\pgfqpoint{0.000000in}{0.000000in}}%
\pgfpathlineto{\pgfqpoint{0.000000in}{-0.055556in}}%
\pgfusepath{stroke,fill}%
}%
\begin{pgfscope}%
\pgfsys@transformshift{6.100000in}{5.700000in}%
\pgfsys@useobject{currentmarker}{}%
\end{pgfscope}%
\end{pgfscope}%
\begin{pgfscope}%
\definecolor{textcolor}{rgb}{0.000000,0.000000,0.000000}%
\pgfsetstrokecolor{textcolor}%
\pgfsetfillcolor{textcolor}%
\pgftext[x=6.100000in,y=0.844444in,,top]{\color{textcolor}\sffamily\fontsize{20.000000}{24.000000}\selectfont \(\displaystyle {550}\)}%
\end{pgfscope}%
\begin{pgfscope}%
\pgfsetbuttcap%
\pgfsetroundjoin%
\definecolor{currentfill}{rgb}{0.000000,0.000000,0.000000}%
\pgfsetfillcolor{currentfill}%
\pgfsetlinewidth{0.501875pt}%
\definecolor{currentstroke}{rgb}{0.000000,0.000000,0.000000}%
\pgfsetstrokecolor{currentstroke}%
\pgfsetdash{}{0pt}%
\pgfsys@defobject{currentmarker}{\pgfqpoint{0.000000in}{0.000000in}}{\pgfqpoint{0.000000in}{0.055556in}}{%
\pgfpathmoveto{\pgfqpoint{0.000000in}{0.000000in}}%
\pgfpathlineto{\pgfqpoint{0.000000in}{0.055556in}}%
\pgfusepath{stroke,fill}%
}%
\begin{pgfscope}%
\pgfsys@transformshift{6.800000in}{0.900000in}%
\pgfsys@useobject{currentmarker}{}%
\end{pgfscope}%
\end{pgfscope}%
\begin{pgfscope}%
\pgfsetbuttcap%
\pgfsetroundjoin%
\definecolor{currentfill}{rgb}{0.000000,0.000000,0.000000}%
\pgfsetfillcolor{currentfill}%
\pgfsetlinewidth{0.501875pt}%
\definecolor{currentstroke}{rgb}{0.000000,0.000000,0.000000}%
\pgfsetstrokecolor{currentstroke}%
\pgfsetdash{}{0pt}%
\pgfsys@defobject{currentmarker}{\pgfqpoint{0.000000in}{-0.055556in}}{\pgfqpoint{0.000000in}{0.000000in}}{%
\pgfpathmoveto{\pgfqpoint{0.000000in}{0.000000in}}%
\pgfpathlineto{\pgfqpoint{0.000000in}{-0.055556in}}%
\pgfusepath{stroke,fill}%
}%
\begin{pgfscope}%
\pgfsys@transformshift{6.800000in}{5.700000in}%
\pgfsys@useobject{currentmarker}{}%
\end{pgfscope}%
\end{pgfscope}%
\begin{pgfscope}%
\definecolor{textcolor}{rgb}{0.000000,0.000000,0.000000}%
\pgfsetstrokecolor{textcolor}%
\pgfsetfillcolor{textcolor}%
\pgftext[x=6.800000in,y=0.844444in,,top]{\color{textcolor}\sffamily\fontsize{20.000000}{24.000000}\selectfont \(\displaystyle {600}\)}%
\end{pgfscope}%
\begin{pgfscope}%
\definecolor{textcolor}{rgb}{0.000000,0.000000,0.000000}%
\pgfsetstrokecolor{textcolor}%
\pgfsetfillcolor{textcolor}%
\pgftext[x=4.000000in,y=0.518932in,,top]{\color{textcolor}\sffamily\fontsize{20.000000}{24.000000}\selectfont \(\displaystyle \mathrm{t}/\si{ns}\)}%
\end{pgfscope}%
\begin{pgfscope}%
\pgfsetbuttcap%
\pgfsetroundjoin%
\definecolor{currentfill}{rgb}{0.000000,0.000000,0.000000}%
\pgfsetfillcolor{currentfill}%
\pgfsetlinewidth{0.501875pt}%
\definecolor{currentstroke}{rgb}{0.000000,0.000000,0.000000}%
\pgfsetstrokecolor{currentstroke}%
\pgfsetdash{}{0pt}%
\pgfsys@defobject{currentmarker}{\pgfqpoint{0.000000in}{0.000000in}}{\pgfqpoint{0.055556in}{0.000000in}}{%
\pgfpathmoveto{\pgfqpoint{0.000000in}{0.000000in}}%
\pgfpathlineto{\pgfqpoint{0.055556in}{0.000000in}}%
\pgfusepath{stroke,fill}%
}%
\begin{pgfscope}%
\pgfsys@transformshift{1.200000in}{1.176718in}%
\pgfsys@useobject{currentmarker}{}%
\end{pgfscope}%
\end{pgfscope}%
\begin{pgfscope}%
\definecolor{textcolor}{rgb}{0.000000,0.000000,0.000000}%
\pgfsetstrokecolor{textcolor}%
\pgfsetfillcolor{textcolor}%
\pgftext[x=1.144444in,y=1.176718in,right,]{\color{textcolor}\sffamily\fontsize{20.000000}{24.000000}\selectfont \(\displaystyle {0}\)}%
\end{pgfscope}%
\begin{pgfscope}%
\pgfsetbuttcap%
\pgfsetroundjoin%
\definecolor{currentfill}{rgb}{0.000000,0.000000,0.000000}%
\pgfsetfillcolor{currentfill}%
\pgfsetlinewidth{0.501875pt}%
\definecolor{currentstroke}{rgb}{0.000000,0.000000,0.000000}%
\pgfsetstrokecolor{currentstroke}%
\pgfsetdash{}{0pt}%
\pgfsys@defobject{currentmarker}{\pgfqpoint{0.000000in}{0.000000in}}{\pgfqpoint{0.055556in}{0.000000in}}{%
\pgfpathmoveto{\pgfqpoint{0.000000in}{0.000000in}}%
\pgfpathlineto{\pgfqpoint{0.055556in}{0.000000in}}%
\pgfusepath{stroke,fill}%
}%
\begin{pgfscope}%
\pgfsys@transformshift{1.200000in}{2.283592in}%
\pgfsys@useobject{currentmarker}{}%
\end{pgfscope}%
\end{pgfscope}%
\begin{pgfscope}%
\definecolor{textcolor}{rgb}{0.000000,0.000000,0.000000}%
\pgfsetstrokecolor{textcolor}%
\pgfsetfillcolor{textcolor}%
\pgftext[x=1.144444in,y=2.283592in,right,]{\color{textcolor}\sffamily\fontsize{20.000000}{24.000000}\selectfont \(\displaystyle {20}\)}%
\end{pgfscope}%
\begin{pgfscope}%
\pgfsetbuttcap%
\pgfsetroundjoin%
\definecolor{currentfill}{rgb}{0.000000,0.000000,0.000000}%
\pgfsetfillcolor{currentfill}%
\pgfsetlinewidth{0.501875pt}%
\definecolor{currentstroke}{rgb}{0.000000,0.000000,0.000000}%
\pgfsetstrokecolor{currentstroke}%
\pgfsetdash{}{0pt}%
\pgfsys@defobject{currentmarker}{\pgfqpoint{0.000000in}{0.000000in}}{\pgfqpoint{0.055556in}{0.000000in}}{%
\pgfpathmoveto{\pgfqpoint{0.000000in}{0.000000in}}%
\pgfpathlineto{\pgfqpoint{0.055556in}{0.000000in}}%
\pgfusepath{stroke,fill}%
}%
\begin{pgfscope}%
\pgfsys@transformshift{1.200000in}{3.390466in}%
\pgfsys@useobject{currentmarker}{}%
\end{pgfscope}%
\end{pgfscope}%
\begin{pgfscope}%
\definecolor{textcolor}{rgb}{0.000000,0.000000,0.000000}%
\pgfsetstrokecolor{textcolor}%
\pgfsetfillcolor{textcolor}%
\pgftext[x=1.144444in,y=3.390466in,right,]{\color{textcolor}\sffamily\fontsize{20.000000}{24.000000}\selectfont \(\displaystyle {40}\)}%
\end{pgfscope}%
\begin{pgfscope}%
\pgfsetbuttcap%
\pgfsetroundjoin%
\definecolor{currentfill}{rgb}{0.000000,0.000000,0.000000}%
\pgfsetfillcolor{currentfill}%
\pgfsetlinewidth{0.501875pt}%
\definecolor{currentstroke}{rgb}{0.000000,0.000000,0.000000}%
\pgfsetstrokecolor{currentstroke}%
\pgfsetdash{}{0pt}%
\pgfsys@defobject{currentmarker}{\pgfqpoint{0.000000in}{0.000000in}}{\pgfqpoint{0.055556in}{0.000000in}}{%
\pgfpathmoveto{\pgfqpoint{0.000000in}{0.000000in}}%
\pgfpathlineto{\pgfqpoint{0.055556in}{0.000000in}}%
\pgfusepath{stroke,fill}%
}%
\begin{pgfscope}%
\pgfsys@transformshift{1.200000in}{4.497339in}%
\pgfsys@useobject{currentmarker}{}%
\end{pgfscope}%
\end{pgfscope}%
\begin{pgfscope}%
\definecolor{textcolor}{rgb}{0.000000,0.000000,0.000000}%
\pgfsetstrokecolor{textcolor}%
\pgfsetfillcolor{textcolor}%
\pgftext[x=1.144444in,y=4.497339in,right,]{\color{textcolor}\sffamily\fontsize{20.000000}{24.000000}\selectfont \(\displaystyle {60}\)}%
\end{pgfscope}%
\begin{pgfscope}%
\pgfsetbuttcap%
\pgfsetroundjoin%
\definecolor{currentfill}{rgb}{0.000000,0.000000,0.000000}%
\pgfsetfillcolor{currentfill}%
\pgfsetlinewidth{0.501875pt}%
\definecolor{currentstroke}{rgb}{0.000000,0.000000,0.000000}%
\pgfsetstrokecolor{currentstroke}%
\pgfsetdash{}{0pt}%
\pgfsys@defobject{currentmarker}{\pgfqpoint{0.000000in}{0.000000in}}{\pgfqpoint{0.055556in}{0.000000in}}{%
\pgfpathmoveto{\pgfqpoint{0.000000in}{0.000000in}}%
\pgfpathlineto{\pgfqpoint{0.055556in}{0.000000in}}%
\pgfusepath{stroke,fill}%
}%
\begin{pgfscope}%
\pgfsys@transformshift{1.200000in}{5.604213in}%
\pgfsys@useobject{currentmarker}{}%
\end{pgfscope}%
\end{pgfscope}%
\begin{pgfscope}%
\definecolor{textcolor}{rgb}{0.000000,0.000000,0.000000}%
\pgfsetstrokecolor{textcolor}%
\pgfsetfillcolor{textcolor}%
\pgftext[x=1.144444in,y=5.604213in,right,]{\color{textcolor}\sffamily\fontsize{20.000000}{24.000000}\selectfont \(\displaystyle {80}\)}%
\end{pgfscope}%
\begin{pgfscope}%
\definecolor{textcolor}{rgb}{0.000000,0.000000,0.000000}%
\pgfsetstrokecolor{textcolor}%
\pgfsetfillcolor{textcolor}%
\pgftext[x=0.810785in,y=3.300000in,,bottom,rotate=90.000000]{\color{textcolor}\sffamily\fontsize{20.000000}{24.000000}\selectfont \(\displaystyle \mathrm{Voltage}/\si{mV}\)}%
\end{pgfscope}%
\begin{pgfscope}%
\pgfpathrectangle{\pgfqpoint{1.200000in}{0.900000in}}{\pgfqpoint{5.600000in}{4.800000in}}%
\pgfusepath{clip}%
\pgfsetbuttcap%
\pgfsetroundjoin%
\pgfsetlinewidth{2.007500pt}%
\definecolor{currentstroke}{rgb}{1.000000,0.000000,0.000000}%
\pgfsetstrokecolor{currentstroke}%
\pgfsetdash{}{0pt}%
\pgfpathmoveto{\pgfqpoint{2.537047in}{1.176718in}}%
\pgfpathlineto{\pgfqpoint{2.537047in}{4.943475in}}%
\pgfusepath{stroke}%
\end{pgfscope}%
\begin{pgfscope}%
\pgfpathrectangle{\pgfqpoint{1.200000in}{0.900000in}}{\pgfqpoint{5.600000in}{4.800000in}}%
\pgfusepath{clip}%
\pgfsetbuttcap%
\pgfsetroundjoin%
\pgfsetlinewidth{2.007500pt}%
\definecolor{currentstroke}{rgb}{1.000000,0.000000,0.000000}%
\pgfsetstrokecolor{currentstroke}%
\pgfsetdash{}{0pt}%
\pgfpathmoveto{\pgfqpoint{2.540315in}{1.176718in}}%
\pgfpathlineto{\pgfqpoint{2.540315in}{5.169623in}}%
\pgfusepath{stroke}%
\end{pgfscope}%
\begin{pgfscope}%
\pgfpathrectangle{\pgfqpoint{1.200000in}{0.900000in}}{\pgfqpoint{5.600000in}{4.800000in}}%
\pgfusepath{clip}%
\pgfsetbuttcap%
\pgfsetroundjoin%
\pgfsetlinewidth{2.007500pt}%
\definecolor{currentstroke}{rgb}{1.000000,0.000000,0.000000}%
\pgfsetstrokecolor{currentstroke}%
\pgfsetdash{}{0pt}%
\pgfpathmoveto{\pgfqpoint{2.552956in}{1.176718in}}%
\pgfpathlineto{\pgfqpoint{2.552956in}{5.334693in}}%
\pgfusepath{stroke}%
\end{pgfscope}%
\begin{pgfscope}%
\pgfpathrectangle{\pgfqpoint{1.200000in}{0.900000in}}{\pgfqpoint{5.600000in}{4.800000in}}%
\pgfusepath{clip}%
\pgfsetbuttcap%
\pgfsetroundjoin%
\pgfsetlinewidth{2.007500pt}%
\definecolor{currentstroke}{rgb}{1.000000,0.000000,0.000000}%
\pgfsetstrokecolor{currentstroke}%
\pgfsetdash{}{0pt}%
\pgfpathmoveto{\pgfqpoint{2.726232in}{1.176718in}}%
\pgfpathlineto{\pgfqpoint{2.726232in}{4.611768in}}%
\pgfusepath{stroke}%
\end{pgfscope}%
\begin{pgfscope}%
\pgfpathrectangle{\pgfqpoint{1.200000in}{0.900000in}}{\pgfqpoint{5.600000in}{4.800000in}}%
\pgfusepath{clip}%
\pgfsetbuttcap%
\pgfsetroundjoin%
\pgfsetlinewidth{2.007500pt}%
\definecolor{currentstroke}{rgb}{1.000000,0.000000,0.000000}%
\pgfsetstrokecolor{currentstroke}%
\pgfsetdash{}{0pt}%
\pgfpathmoveto{\pgfqpoint{2.886646in}{1.176718in}}%
\pgfpathlineto{\pgfqpoint{2.886646in}{5.054054in}}%
\pgfusepath{stroke}%
\end{pgfscope}%
\begin{pgfscope}%
\pgfpathrectangle{\pgfqpoint{1.200000in}{0.900000in}}{\pgfqpoint{5.600000in}{4.800000in}}%
\pgfusepath{clip}%
\pgfsetbuttcap%
\pgfsetroundjoin%
\pgfsetlinewidth{2.007500pt}%
\definecolor{currentstroke}{rgb}{1.000000,0.000000,0.000000}%
\pgfsetstrokecolor{currentstroke}%
\pgfsetdash{}{0pt}%
\pgfpathmoveto{\pgfqpoint{3.063224in}{1.176718in}}%
\pgfpathlineto{\pgfqpoint{3.063224in}{5.222180in}}%
\pgfusepath{stroke}%
\end{pgfscope}%
\begin{pgfscope}%
\pgfpathrectangle{\pgfqpoint{1.200000in}{0.900000in}}{\pgfqpoint{5.600000in}{4.800000in}}%
\pgfusepath{clip}%
\pgfsetbuttcap%
\pgfsetroundjoin%
\pgfsetlinewidth{2.007500pt}%
\definecolor{currentstroke}{rgb}{1.000000,0.000000,0.000000}%
\pgfsetstrokecolor{currentstroke}%
\pgfsetdash{}{0pt}%
\pgfpathmoveto{\pgfqpoint{3.214882in}{1.176718in}}%
\pgfpathlineto{\pgfqpoint{3.214882in}{5.068511in}}%
\pgfusepath{stroke}%
\end{pgfscope}%
\begin{pgfscope}%
\pgfpathrectangle{\pgfqpoint{1.200000in}{0.900000in}}{\pgfqpoint{5.600000in}{4.800000in}}%
\pgfusepath{clip}%
\pgfsetbuttcap%
\pgfsetroundjoin%
\pgfsetlinewidth{2.007500pt}%
\definecolor{currentstroke}{rgb}{1.000000,0.000000,0.000000}%
\pgfsetstrokecolor{currentstroke}%
\pgfsetdash{}{0pt}%
\pgfpathmoveto{\pgfqpoint{3.427660in}{1.176718in}}%
\pgfpathlineto{\pgfqpoint{3.427660in}{4.386756in}}%
\pgfusepath{stroke}%
\end{pgfscope}%
\begin{pgfscope}%
\pgfsetrectcap%
\pgfsetmiterjoin%
\pgfsetlinewidth{1.003750pt}%
\definecolor{currentstroke}{rgb}{0.000000,0.000000,0.000000}%
\pgfsetstrokecolor{currentstroke}%
\pgfsetdash{}{0pt}%
\pgfpathmoveto{\pgfqpoint{1.200000in}{0.900000in}}%
\pgfpathlineto{\pgfqpoint{1.200000in}{5.700000in}}%
\pgfusepath{stroke}%
\end{pgfscope}%
\begin{pgfscope}%
\pgfsetrectcap%
\pgfsetmiterjoin%
\pgfsetlinewidth{1.003750pt}%
\definecolor{currentstroke}{rgb}{0.000000,0.000000,0.000000}%
\pgfsetstrokecolor{currentstroke}%
\pgfsetdash{}{0pt}%
\pgfpathmoveto{\pgfqpoint{6.800000in}{0.900000in}}%
\pgfpathlineto{\pgfqpoint{6.800000in}{5.700000in}}%
\pgfusepath{stroke}%
\end{pgfscope}%
\begin{pgfscope}%
\pgfsetrectcap%
\pgfsetmiterjoin%
\pgfsetlinewidth{1.003750pt}%
\definecolor{currentstroke}{rgb}{0.000000,0.000000,0.000000}%
\pgfsetstrokecolor{currentstroke}%
\pgfsetdash{}{0pt}%
\pgfpathmoveto{\pgfqpoint{1.200000in}{0.900000in}}%
\pgfpathlineto{\pgfqpoint{6.800000in}{0.900000in}}%
\pgfusepath{stroke}%
\end{pgfscope}%
\begin{pgfscope}%
\pgfsetrectcap%
\pgfsetmiterjoin%
\pgfsetlinewidth{1.003750pt}%
\definecolor{currentstroke}{rgb}{0.000000,0.000000,0.000000}%
\pgfsetstrokecolor{currentstroke}%
\pgfsetdash{}{0pt}%
\pgfpathmoveto{\pgfqpoint{1.200000in}{5.700000in}}%
\pgfpathlineto{\pgfqpoint{6.800000in}{5.700000in}}%
\pgfusepath{stroke}%
\end{pgfscope}%
\begin{pgfscope}%
\pgfsetbuttcap%
\pgfsetroundjoin%
\definecolor{currentfill}{rgb}{0.000000,0.000000,0.000000}%
\pgfsetfillcolor{currentfill}%
\pgfsetlinewidth{0.501875pt}%
\definecolor{currentstroke}{rgb}{0.000000,0.000000,0.000000}%
\pgfsetstrokecolor{currentstroke}%
\pgfsetdash{}{0pt}%
\pgfsys@defobject{currentmarker}{\pgfqpoint{-0.055556in}{0.000000in}}{\pgfqpoint{-0.000000in}{0.000000in}}{%
\pgfpathmoveto{\pgfqpoint{-0.000000in}{0.000000in}}%
\pgfpathlineto{\pgfqpoint{-0.055556in}{0.000000in}}%
\pgfusepath{stroke,fill}%
}%
\begin{pgfscope}%
\pgfsys@transformshift{6.800000in}{1.176718in}%
\pgfsys@useobject{currentmarker}{}%
\end{pgfscope}%
\end{pgfscope}%
\begin{pgfscope}%
\definecolor{textcolor}{rgb}{0.000000,0.000000,0.000000}%
\pgfsetstrokecolor{textcolor}%
\pgfsetfillcolor{textcolor}%
\pgftext[x=6.855556in,y=1.176718in,left,]{\color{textcolor}\sffamily\fontsize{20.000000}{24.000000}\selectfont \(\displaystyle {0}\)}%
\end{pgfscope}%
\begin{pgfscope}%
\pgfsetbuttcap%
\pgfsetroundjoin%
\definecolor{currentfill}{rgb}{0.000000,0.000000,0.000000}%
\pgfsetfillcolor{currentfill}%
\pgfsetlinewidth{0.501875pt}%
\definecolor{currentstroke}{rgb}{0.000000,0.000000,0.000000}%
\pgfsetstrokecolor{currentstroke}%
\pgfsetdash{}{0pt}%
\pgfsys@defobject{currentmarker}{\pgfqpoint{-0.055556in}{0.000000in}}{\pgfqpoint{-0.000000in}{0.000000in}}{%
\pgfpathmoveto{\pgfqpoint{-0.000000in}{0.000000in}}%
\pgfpathlineto{\pgfqpoint{-0.055556in}{0.000000in}}%
\pgfusepath{stroke,fill}%
}%
\begin{pgfscope}%
\pgfsys@transformshift{6.800000in}{2.307539in}%
\pgfsys@useobject{currentmarker}{}%
\end{pgfscope}%
\end{pgfscope}%
\begin{pgfscope}%
\definecolor{textcolor}{rgb}{0.000000,0.000000,0.000000}%
\pgfsetstrokecolor{textcolor}%
\pgfsetfillcolor{textcolor}%
\pgftext[x=6.855556in,y=2.307539in,left,]{\color{textcolor}\sffamily\fontsize{20.000000}{24.000000}\selectfont \(\displaystyle {50}\)}%
\end{pgfscope}%
\begin{pgfscope}%
\pgfsetbuttcap%
\pgfsetroundjoin%
\definecolor{currentfill}{rgb}{0.000000,0.000000,0.000000}%
\pgfsetfillcolor{currentfill}%
\pgfsetlinewidth{0.501875pt}%
\definecolor{currentstroke}{rgb}{0.000000,0.000000,0.000000}%
\pgfsetstrokecolor{currentstroke}%
\pgfsetdash{}{0pt}%
\pgfsys@defobject{currentmarker}{\pgfqpoint{-0.055556in}{0.000000in}}{\pgfqpoint{-0.000000in}{0.000000in}}{%
\pgfpathmoveto{\pgfqpoint{-0.000000in}{0.000000in}}%
\pgfpathlineto{\pgfqpoint{-0.055556in}{0.000000in}}%
\pgfusepath{stroke,fill}%
}%
\begin{pgfscope}%
\pgfsys@transformshift{6.800000in}{3.438359in}%
\pgfsys@useobject{currentmarker}{}%
\end{pgfscope}%
\end{pgfscope}%
\begin{pgfscope}%
\definecolor{textcolor}{rgb}{0.000000,0.000000,0.000000}%
\pgfsetstrokecolor{textcolor}%
\pgfsetfillcolor{textcolor}%
\pgftext[x=6.855556in,y=3.438359in,left,]{\color{textcolor}\sffamily\fontsize{20.000000}{24.000000}\selectfont \(\displaystyle {100}\)}%
\end{pgfscope}%
\begin{pgfscope}%
\pgfsetbuttcap%
\pgfsetroundjoin%
\definecolor{currentfill}{rgb}{0.000000,0.000000,0.000000}%
\pgfsetfillcolor{currentfill}%
\pgfsetlinewidth{0.501875pt}%
\definecolor{currentstroke}{rgb}{0.000000,0.000000,0.000000}%
\pgfsetstrokecolor{currentstroke}%
\pgfsetdash{}{0pt}%
\pgfsys@defobject{currentmarker}{\pgfqpoint{-0.055556in}{0.000000in}}{\pgfqpoint{-0.000000in}{0.000000in}}{%
\pgfpathmoveto{\pgfqpoint{-0.000000in}{0.000000in}}%
\pgfpathlineto{\pgfqpoint{-0.055556in}{0.000000in}}%
\pgfusepath{stroke,fill}%
}%
\begin{pgfscope}%
\pgfsys@transformshift{6.800000in}{4.569180in}%
\pgfsys@useobject{currentmarker}{}%
\end{pgfscope}%
\end{pgfscope}%
\begin{pgfscope}%
\definecolor{textcolor}{rgb}{0.000000,0.000000,0.000000}%
\pgfsetstrokecolor{textcolor}%
\pgfsetfillcolor{textcolor}%
\pgftext[x=6.855556in,y=4.569180in,left,]{\color{textcolor}\sffamily\fontsize{20.000000}{24.000000}\selectfont \(\displaystyle {150}\)}%
\end{pgfscope}%
\begin{pgfscope}%
\pgfsetbuttcap%
\pgfsetroundjoin%
\definecolor{currentfill}{rgb}{0.000000,0.000000,0.000000}%
\pgfsetfillcolor{currentfill}%
\pgfsetlinewidth{0.501875pt}%
\definecolor{currentstroke}{rgb}{0.000000,0.000000,0.000000}%
\pgfsetstrokecolor{currentstroke}%
\pgfsetdash{}{0pt}%
\pgfsys@defobject{currentmarker}{\pgfqpoint{-0.055556in}{0.000000in}}{\pgfqpoint{-0.000000in}{0.000000in}}{%
\pgfpathmoveto{\pgfqpoint{-0.000000in}{0.000000in}}%
\pgfpathlineto{\pgfqpoint{-0.055556in}{0.000000in}}%
\pgfusepath{stroke,fill}%
}%
\begin{pgfscope}%
\pgfsys@transformshift{6.800000in}{5.700000in}%
\pgfsys@useobject{currentmarker}{}%
\end{pgfscope}%
\end{pgfscope}%
\begin{pgfscope}%
\definecolor{textcolor}{rgb}{0.000000,0.000000,0.000000}%
\pgfsetstrokecolor{textcolor}%
\pgfsetfillcolor{textcolor}%
\pgftext[x=6.855556in,y=5.700000in,left,]{\color{textcolor}\sffamily\fontsize{20.000000}{24.000000}\selectfont \(\displaystyle {200}\)}%
\end{pgfscope}%
\begin{pgfscope}%
\definecolor{textcolor}{rgb}{0.000000,0.000000,0.000000}%
\pgfsetstrokecolor{textcolor}%
\pgfsetfillcolor{textcolor}%
\pgftext[x=7.321322in,y=3.300000in,,top,rotate=90.000000]{\color{textcolor}\sffamily\fontsize{20.000000}{24.000000}\selectfont \(\displaystyle \mathrm{Charge}/\si{mV\cdot ns}\)}%
\end{pgfscope}%
\begin{pgfscope}%
\pgfsetbuttcap%
\pgfsetmiterjoin%
\definecolor{currentfill}{rgb}{1.000000,1.000000,1.000000}%
\pgfsetfillcolor{currentfill}%
\pgfsetlinewidth{1.003750pt}%
\definecolor{currentstroke}{rgb}{0.000000,0.000000,0.000000}%
\pgfsetstrokecolor{currentstroke}%
\pgfsetdash{}{0pt}%
\pgfpathmoveto{\pgfqpoint{4.066020in}{4.011785in}}%
\pgfpathlineto{\pgfqpoint{6.633333in}{4.011785in}}%
\pgfpathlineto{\pgfqpoint{6.633333in}{5.533333in}}%
\pgfpathlineto{\pgfqpoint{4.066020in}{5.533333in}}%
\pgfpathclose%
\pgfusepath{stroke,fill}%
\end{pgfscope}%
\begin{pgfscope}%
\pgfsetrectcap%
\pgfsetroundjoin%
\pgfsetlinewidth{2.007500pt}%
\definecolor{currentstroke}{rgb}{0.000000,0.000000,1.000000}%
\pgfsetstrokecolor{currentstroke}%
\pgfsetdash{}{0pt}%
\pgfpathmoveto{\pgfqpoint{4.299353in}{5.276697in}}%
\pgfpathlineto{\pgfqpoint{4.766020in}{5.276697in}}%
\pgfusepath{stroke}%
\end{pgfscope}%
\begin{pgfscope}%
\definecolor{textcolor}{rgb}{0.000000,0.000000,0.000000}%
\pgfsetstrokecolor{textcolor}%
\pgfsetfillcolor{textcolor}%
\pgftext[x=5.132687in,y=5.160031in,left,base]{\color{textcolor}\sffamily\fontsize{24.000000}{28.800000}\selectfont Waveform}%
\end{pgfscope}%
\begin{pgfscope}%
\pgfsetbuttcap%
\pgfsetroundjoin%
\pgfsetlinewidth{2.007500pt}%
\definecolor{currentstroke}{rgb}{0.000000,0.500000,0.000000}%
\pgfsetstrokecolor{currentstroke}%
\pgfsetdash{}{0pt}%
\pgfpathmoveto{\pgfqpoint{4.299353in}{4.802848in}}%
\pgfpathlineto{\pgfqpoint{4.532687in}{4.802848in}}%
\pgfpathlineto{\pgfqpoint{4.766020in}{4.802848in}}%
\pgfusepath{stroke}%
\end{pgfscope}%
\begin{pgfscope}%
\definecolor{textcolor}{rgb}{0.000000,0.000000,0.000000}%
\pgfsetstrokecolor{textcolor}%
\pgfsetfillcolor{textcolor}%
\pgftext[x=5.132687in,y=4.686181in,left,base]{\color{textcolor}\sffamily\fontsize{24.000000}{28.800000}\selectfont Threshold}%
\end{pgfscope}%
\begin{pgfscope}%
\pgfsetbuttcap%
\pgfsetroundjoin%
\pgfsetlinewidth{2.007500pt}%
\definecolor{currentstroke}{rgb}{1.000000,0.000000,0.000000}%
\pgfsetstrokecolor{currentstroke}%
\pgfsetdash{}{0pt}%
\pgfpathmoveto{\pgfqpoint{4.299353in}{4.328998in}}%
\pgfpathlineto{\pgfqpoint{4.532687in}{4.328998in}}%
\pgfpathlineto{\pgfqpoint{4.766020in}{4.328998in}}%
\pgfusepath{stroke}%
\end{pgfscope}%
\begin{pgfscope}%
\definecolor{textcolor}{rgb}{0.000000,0.000000,0.000000}%
\pgfsetstrokecolor{textcolor}%
\pgfsetfillcolor{textcolor}%
\pgftext[x=5.132687in,y=4.212332in,left,base]{\color{textcolor}\sffamily\fontsize{24.000000}{28.800000}\selectfont Charge}%
\end{pgfscope}%
\end{pgfpicture}%
\makeatother%
\endgroup%
}
        \caption{\label{fig:new} New Goal Recorded Waveform}
    \end{subfigure}
\end{figure}

Charge is the integration of waveform component which induced by SPE. SPE induced charge can be a wide distribution, rather than a single value. Reconstructing $n_{r}(t_{H})$ is more difficult than reconstructing $q_{r}(t_{H})$. 

\begin{figure}[H]
    \centering
    \includegraphics[width=0.6\linewidth]{figures/chargehist.png}
    \caption{\label{fig:charge} Distribution of Charge}
\end{figure}

% section Waveform of PMT (end)